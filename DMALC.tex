
\documentclass{article}

\usepackage{amsmath}
\usepackage{ebproof}
\usepackage{fullpage}
\usepackage[utf8]{inputenc}
\usepackage{newunicodechar}
\usepackage{stix}

\newunicodechar{Γ}{\Gamma}
\newunicodechar{Π}{\Pi}
\newunicodechar{Δ}{\Delta}
\newunicodechar{Λ}{\Lambda}
\newunicodechar{Θ}{\Theta}

\newunicodechar{⊢}{\vdash}

\newunicodechar{⊕}{\oplus}
\newunicodechar{¬}{\neg}

\newunicodechar{⅋}{\upand}
\newunicodechar{↛}{\nrightarrow}
\newunicodechar{↚}{\nleftarrow}

\newunicodechar{⊥}{\bot}
\newunicodechar{⊤}{\top}

\setlength{\parindent}{0em}

\author{James Martin, Ian D.L.N. Mclean}
\title{Dual Multiplicative-Additive Lambek Sequent Calculus}

\begin{document}

\maketitle

\begin{abstract}

\end{abstract}

\section{Operational Rules}
Uncertain about proper dualization that preserves the proper ordering for the dual.
\begin{center}
	
	\[
	\begin{prooftree}
	\infer0{ ⊥ ⊢ }
	\end{prooftree}
	\quad
	\begin{prooftree}
	\hypo{C ⊢ Δ_{L}, Δ_{R}}
	\infer1{C ⊢ Δ_{L}, ⊥, Δ_{R}}
	\end{prooftree}
	\]
	
	\[
	\begin{prooftree}
	\infer0{ 0 ⊢ Δ}
	\end{prooftree}
	\quad
	\begin{prooftree}
	\infer0{C ⊢ Δ_{L}, ⊤, Δ_{R}}
	\end{prooftree}
	\]
	
	\[
	\begin{prooftree}
	\hypo{B ⊢ Δ_{L}}
	\hypo{A ⊢ Δ_{R}}
	\infer2{B ⅋ A ⊢ Δ_{L}, Δ_{R}}
	\end{prooftree}
	\quad
	\begin{prooftree}
	\hypo{C ⊢ Δ_{L}, B, A, Δ_{L}}
	\infer1{C ⊢ Δ_{L}, B ⅋ A, Δ_{L}}
	\end{prooftree}
	\]
	
	\[
	\begin{prooftree}
	\hypo{B ⊢ Δ, A}
	\infer1{A ↚ B ⊢ Δ}
	\end{prooftree}
	\quad
	\begin{prooftree}
	\hypo{A ⊢ Δ}
	\hypo{C ⊢ Δ_{L}, B, Δ_{R}}
	\infer2{C ⊢ Δ_{L}, A ↚ B, Δ, Δ_{R}}
	\end{prooftree}
	\]
	
	\[
	\quad
	\begin{prooftree}
	\hypo{B ⊢ A, Δ}
	\infer1{A ↛ B ⊢ Δ}
	\end{prooftree}
	\quad
	\begin{prooftree}
	\hypo{A ⊢ Δ}
	\hypo{C ⊢ Δ_{L}, B, Δ_{R}}
	\infer2{C ⊢ Δ_{L}, Δ, B ↛ A, Δ_{R}}
	\end{prooftree}
	\]
	
	\[
	\begin{prooftree}
	\hypo{B ⊢ Δ}
	\hypo{A ⊢ Δ}
	\infer2{B ⊕ A ⊢ Δ}
	\end{prooftree}
	\quad
	\begin{prooftree}
	\hypo{C ⊢ Δ_{L}, A, Δ_{R}}
	\infer1{C ⊢ Δ_{L}, B ⊕ A, Δ_{R}}
	\end{prooftree}
	\quad
	\begin{prooftree}
	\hypo{C ⊢ Δ_{L}, B, Δ_{R}}
	\infer1{C ⊢ Δ_{L}, B ⊕ A, Δ_{R}}
	\end{prooftree}
	\]

	\[
	\begin{prooftree}
	\hypo{A ⊢ Δ}
	\infer1{A \& B ⊢ Δ}
	\end{prooftree}
	\quad
	\begin{prooftree}
	\hypo{B ⊢ Δ}
	\infer1{A \& B ⊢ Δ}
	\end{prooftree}
	\quad
	\begin{prooftree}
	\hypo{ C ⊢ Δ_{L}, B, Δ_{R}}
	\hypo{ C ⊢ Δ_{L}, A, Δ_{R}}
	\infer2{ C ⊢ Δ_{L}, B \& A, Δ_{R}}
	\end{prooftree}
	\]
	
	\[
	\begin{prooftree}
	\hypo{ ⊢ A, Δ}
	\infer1{ ¬ A ⊢ Δ}
	\end{prooftree}
	\quad
	\begin{prooftree}
	\hypo{ ⊢ Δ, A}
	\infer1{ ¬ A⊢ Δ}
	\end{prooftree}
	\quad
	\begin{prooftree}
	\hypo{ A ⊢ Δ}
	\infer1{ ⊢ ¬ A, Δ}
	\end{prooftree}
	\quad
	\begin{prooftree}
	\hypo{ A ⊢ Δ}
	\infer1{ ⊢ Δ, ¬ A}
	\end{prooftree}
	\]

\end{center}

\section{Structural Rules}
Uses a variation of cut right.

\begin{center}
	\[
	\begin{prooftree}
	\infer0[Id]{A ⊢ A}
	\end{prooftree}
	\]
	
	\[
	\begin{prooftree}
	\hypo{C ⊢ Δ_{L}, A, Δ_{R}}
	\hypo{A ⊢ Δ}
	\infer2[CutR]{C ⊢ Δ_{L}, Δ, Δ_{R} }
	\end{prooftree}
	\]
\end{center}

\end{document}
