\documentclass{article}
\usepackage{amsmath, amssymb, amsthm}

\newtheorem{definition}{Definition}
\newtheorem{theorem}{Theorem}
\newtheorem{proposition}{Proposition}
\newtheorem{lemma}{Lemma}
\newtheorem{corollary}{Corollary}
\newtheorem{axiom}{Axiom}
\newtheorem{remark}{Remark}

\begin{document}
	
	\title{Quantization of Usable Entropy and the Landauer Principle in Particle-Mediated Heat Dynamics}
	\author{}
	\date{}
	\maketitle
	
	\begin{abstract}
		This thesis explores the interplay between the quantization of usable entropy, Landauer's principle, and heat dynamics mediated by particles at the microscopic level. We examine how the discrete nature of energy and information in quantum systems, particularly in the context of electron and photon interactions within a chemical system, relates to the fundamental thermodynamic limits imposed by Landauer's principle on information erasure. The role of particle energy, described by both photon energy and the matter wave equation for massive particles, in the dissipation of heat associated with irreversible information processing is analyzed. We further discuss the connection of usable entropy to the holographic principle and its implications for systems with spatial and temporal extent.
	\end{abstract}
	
	\section{Introduction}
	The concepts of entropy and information are deeply intertwined in physics. Usable entropy, often associated with the information content of a system, plays a crucial role in thermodynamics and the limits of computation. At the quantum level, energy and information are quantized, leading to discrete states and transitions. Landauer's principle provides a fundamental link between information and thermodynamics, stating that the erasure of one bit of information requires a minimum energy dissipation into the environment. This thesis aims to investigate how these principles manifest in the context of heat dynamics mediated by particles, such as electrons and photons, within a chemical system where energy exchange and information changes occur through quantized processes. We will also explore the relationship between usable entropy and the holographic bound, considering the spatial and temporal dimensions of the system.
	
	\section{Theoretical Framework}
	
	\subsection{Quantization of Entropy}
	In statistical mechanics, the entropy $S$ of a system is given by Boltzmann's equation:
	\begin{equation}
		S = k_B \ln \Omega
	\end{equation}
	where $k_B$ is the Boltzmann constant and $\Omega$ is the number of microstates accessible to the system at a given energy. In quantum systems, these microstates are quantized, characterized by discrete energy levels and quantum numbers, including spin. Usable entropy can be considered the entropy associated with the information that can be encoded in these quantized states. For a system with $N$ independent bits of information (e.g., $N$ electrons each with two spin states), the maximum usable entropy is $N k_B \ln 2$.
	
	\subsection{Landauer's Principle}
	Landauer's principle sets a lower bound on the energy required to erase one bit of information:
	\begin{theorem}[Landauer's Principle]
		The minimum energy $E_{cost}$ required to erase one bit of information at a temperature $T$ is given by:
		\begin{equation}
			E_{cost} \ge k_B T \ln 2
		\end{equation}
		More generally, for an irreversible process that reduces the entropy of a system by $\Delta S$, the energy dissipated as heat into the environment must satisfy:
		\begin{equation}
			E_{dissipated} \ge T |\Delta S|
		\end{equation}
		This principle arises from the second law of thermodynamics, which requires that the total entropy of the universe (system + environment) cannot decrease.
	\end{theorem}
		\subsection{Particle-Mediated Heat Dynamics}
		At the microscopic level, energy exchange and heat transfer in a chemical system are mediated by quantized particles:
		\begin{itemize}
			\item \textbf{Photons:} Electromagnetic radiation with quantized energy $E = h f$, where $h$ is Planck's constant and $f$ is the frequency. Photons are involved in radiative heat transfer and electronic transitions between energy levels in atoms and molecules.
			\item \textbf{Electrons:} Fundamental particles with mass and charge, whose behavior is governed by quantum mechanics, including the wave-particle duality described by the de Broglie relation: $\lambda = h/p$, where $\lambda$ is the wavelength and $p$ is the momentum. The kinetic energy of an electron is $E = p^2 / (2m_e)$ (non-relativistic). Electrons mediate electrical and thermal conductivity and are involved in chemical bonding and ionization.
		\end{itemize}
		
		\subsection{Holographic Principle and Entropy Bounds}
		The holographic principle posits that the maximum amount of information that can be contained within a volume of space is bounded by the area of its boundary. This principle suggests that the entropy of a system is not proportional to its volume but rather to its surface area, when considered at a fundamental level involving quantum gravity. For a system with a characteristic spatial size $L$, the boundary area scales as $L^2$. In spacetime, particularly within a causal diamond defined by spatial and temporal extents, the holographic bound on entropy relates to the area of the boundary of this region, which depends on both its spatial and temporal dimensions.
		
		\section{Integration and Analysis}
		
		\subsection{Quantized Entropy and Information Erasure}
		The erasure of information in a physical system must correspond to a reduction in the number of accessible microstates, thus a decrease in entropy. In a system where information is encoded in the state of electrons (e.g., presence/absence, spin), the erasure of one bit can be associated with forcing an electron to a specific state or removing it from the system in an irreversible manner. This process involves a quantized change in the system's state and its associated usable entropy $\Delta S$. If one bit of information is erased, $|\Delta S|$ is at least $k_B \ln 2$.
		
		\subsection{Energy Cost and Particle Energy}
		The minimum energy that must be dissipated as heat ($E_{cost}$) to the environment during such an irreversible information erasure must be carried away by particles.
		
		\begin{proposition}
			The minimum energy of a particle mediating heat dissipation due to information erasure is bounded by the Landauer limit.
		\end{proposition}
		\begin{proof}
			Let the temperature of the environment be $T$, and the entropy change associated with the erasure of information be $\Delta S$. The heat $Q$ that must be transferred to the environment is $Q \ge T |\Delta S|$. This heat is carried by particles with quantized energy.
			
			Case 1: Heat carried by photons. The energy of a photon is $E = h f$. Therefore, $h f \ge T |\Delta S|$. For one bit erasure, $h f \ge k_B T \ln 2$.
			
			Case 2: Heat carried by a massive particle (e.g., electron kinetic energy). The kinetic energy is $E = p^2 / (2m)$. Therefore, $p^2 / (2m) \ge T |\Delta S|$. Using $p = h / \lambda$, we get $h^2 / (2 m \lambda^2) \ge T |\Delta S|$.
			
			In both cases, the energy of the particle mediating the heat transfer is constrained by the Landauer limit.
		\end{proof}
		
		\subsection{Usable Entropy and Holographic Bound Scaling}
		The usable entropy of a system is also constrained by the holographic principle, particularly when considering its spatial and temporal extent. The formula:
		\begin{equation}
			S_{usable} \approx S_{max} \times \left( \frac{L_{usable}}{L_{max}} \right)^2 \times \left( \frac{T_{usable}}{T_{max}} \right)^2
		\end{equation}
		suggests that the usable entropy scales with the square of the ratio of the characteristic spatial and temporal sizes in the working range compared to those for the maximum theoretical entropy. This scaling is consistent with the idea that entropy is bounded by the area of the system's boundary in spacetime, where the area can be related to both spatial and temporal dimensions, especially in the context of causal diamonds. The maximum theoretical entropy $S_{max}$ can be thought of as being related to the holographic bound for a system of size $L_{max}$ and time $T_{max}$.
		
		\subsection{Example: Electron Emission and Information Deletion}
		Consider an atom that can exist in a state with or without an electron (representing a bit). Ionizing the atom by emitting the electron can be seen as deleting this bit of information if the process is forced to a specific final state. The energy required for ionization (ionization energy $IE$) must be supplied to the system. If this erasure happens at temperature $T$, the entropy of the system changes by $\Delta S$. The Landauer principle requires that at least $T |\Delta S|$ energy must be dissipated as heat, mediated by quantized particles.
		
		\section{Conclusion}
		The quantization of usable entropy, the Landauer principle, and particle-mediated heat dynamics are intrinsically linked. The discrete nature of quantum states and energy levels dictates that changes in information and entropy occur in quantized amounts. Landauer's principle provides a fundamental thermodynamic constraint on the energy cost of irreversible information processing, requiring heat dissipation carried by quantized particles like photons and electrons. Furthermore, the usable entropy of a system with spatial and temporal extent is related to the holographic bound, scaling with the square of these dimensions. This integrated framework underscores the profound connections between information, thermodynamics, and the quantum nature of matter and energy.
		
	\end{document}