\documentclass{article}
\usepackage{amsmath, amssymb, amsthm}
\usepackage{physics}

\newtheorem{definition}{Definition}
\newtheorem{theorem}{Theorem}
\newtheorem{proposition}{Proposition}
\newtheorem{lemma}{Lemma}
\newtheorem{corollary}{Corollary}
\newtheorem{axiom}{Axiom}
\newtheorem{remark}{Remark}

\begin{document}
	
	\title{Quantization of Usable Entropy and the Landauer Principle in Particle-Mediated Heat Dynamics: A Case Study of a Graphene-Silicene Heterostructure}
	\author{}
	\date{}
	\maketitle
	
	\begin{abstract}
		This thesis investigates the interplay between the quantization of usable entropy, Landauer's principle, and particle-mediated heat dynamics, using a 3-layer graphene-silicene-graphene heterostructure as a case study. We analyze the system's maximum theoretical entropy based on electron spin states and estimate its usable entropy by considering its physical size in relation to holographic bounds. We then explore the implications of Landauer's principle for information erasure within this system, focusing on the role of quantized particles, such as photons and electrons, in the dissipation of heat.
	\end{abstract}
	
	\section{Introduction}
	The profound connection between entropy and information forms a cornerstone of modern physics. While thermodynamic entropy quantifies the disorder of a system, informational entropy measures the uncertainty or the number of bits required to describe its state. In quantum systems, both energy and information are inherently quantized, leading to a discrete spectrum of accessible states. Landauer's principle bridges these concepts by establishing a fundamental thermodynamic cost for the irreversible erasure of information. This thesis delves into these principles within the context of particle-mediated heat dynamics in a specific chemical system: a 3-layer graphene-silicene-graphene heterostructure. By examining this case study, we aim to elucidate how the quantization of usable entropy and the Landauer principle are manifested at the microscopic level through the interactions and exchanges of particles that govern the system's thermal behavior.
	
	\section{Theoretical Framework}
	
	\subsection{Entropy and Information}
	The entropy $S$ of a thermodynamic system in statistical mechanics is defined by Boltzmann's equation:
	\begin{equation}
		S = k_B \ln \Omega
	\end{equation}
	where $k_B$ is the Boltzmann constant and $\Omega$ is the number of accessible microstates. Shannon entropy $H$ in information theory quantifies the uncertainty associated with a random variable and is given by $H = - \sum_i p_i \log_2 p_i$. For a system with $\Omega$ equally probable states, $H = \log_2 \Omega$, which is related to thermodynamic entropy by $S = k_B \ln 2 \times H$. Usable entropy refers to the entropy that is practically accessible or relevant for encoding and processing information within a system.
	
	\subsection{Quantization in Quantum Systems}
	In quantum mechanics, physical quantities such as energy, angular momentum, and spin are quantized, meaning they can only take on discrete values. For instance, electrons in atoms occupy discrete energy levels, and their spin can only be up or down. This quantization leads to a finite and countable number of microstates for a given system with specific constraints. Information stored in such systems, such as the state of an electron's spin, is also quantized as bits.
	
	\subsection{Landauer's Principle: Derivation and Implications}
	Landauer's principle can be derived from the second law of thermodynamics by considering a system that stores one bit of information in a double-well potential. Erasing the bit involves forcing the system into a single known state, reducing its entropy. To comply with the second law, this entropy reduction must be compensated by an increase in the entropy of the environment through heat dissipation.
	
	\begin{theorem}[Landauer's Principle (Restated)]
		The minimum energy $E_{cost}$ required to erase one bit of information at a temperature $T$ is:
		\begin{equation}
			E_{cost} = k_B T \ln 2
		\end{equation}
	\end{theorem}
	
	\subsection{Particle-Mediated Heat Dynamics}
	Heat transfer at the microscopic level in chemical systems occurs primarily through the exchange of quantized particles:
	\begin{itemize}
		\item \textbf{Photons:} Energy is exchanged via electromagnetic radiation in the form of photons with energy $E = h f$. Thermal emission and absorption involve these quantized packets of energy.
		\item \textbf{Electrons:} Changes in the system's energy and information state often involve the absorption or emission of electrons, with energy levels and transitions being quantized. The kinetic energy of electrons is related to their momentum through the de Broglie relation $\lambda = h/p$.
		\end{itemize}
		
		\subsection{Holographic Principle and Entropy Bounds}
		The holographic principle suggests that the maximum entropy that can be contained within a region of space is proportional to the area of its boundary. The Bekenstein-Hawking entropy of a black hole, $S_{BH} = \frac{k_B A}{4 l_P^2}$, where $A$ is the area of the event horizon and $l_P$ is the Planck length, exemplifies this principle. A generalized holographic entropy bound for any system in a region of space with characteristic size $L$ and temporal extent $T$ can be considered, where entropy scales with the boundary area related to these dimensions.
		
		\section{Case Study: 3-Layer Graphene-Silicene-Graphene Heterostructure}
		
		\subsection{System Description}
		Our system consists of 48 carbon atoms and 24 silicon atoms in a 3-layer graphene-silicene-graphene structure, totaling 72 atoms and 624 electrons.
		
		\subsection{Maximum Theoretical Entropy}
		Assuming all 624 electron spins are independent, the maximum number of spin states is $\Omega_{max} = 2^{624}$, leading to a maximum theoretical entropy $S_{max} = 624 k_B \ln 2$.
		
		\subsection{Physical Size Estimation and Holographic Size}
		The estimated physical size has a characteristic length $L_{usable} \sim 1-2$ nanometers. The size $L_{max} \approx 10.7$ nanometers was derived from a holographic-like bound for $S_{max}$.
		
		\section{Analysis and Results}
		
		\subsection{Usable Entropy Estimation}
		\begin{proposition}
			The usable entropy of the graphene-silicene heterostructure is approximately $5.4$ bits, based on the scaling of entropy with the square of the characteristic size as suggested by holographic principles.
		\end{proposition}
		\begin{proof}
			Assuming entropy scales with area ($S \propto L^2$), we have:
			$$ \frac{S_{usable}}{S_{max}} \approx \left( \frac{L_{usable}}{L_{max}} \right)^2 $$
			Using $L_{usable} \approx 1$ nm (as a lower bound for simplicity) and $L_{max} \approx 10.7$ nm,
			$$ \frac{S_{usable}}{624 k_B \ln 2} \approx \left( \frac{1}{10.7} \right)^2 \approx 0.0087 $$
			$$ S_{usable} \approx 624 \times 0.0087 \times k_B \ln 2 \approx 5.43 \times k_B \ln 2 $$
			Since $k_B \ln 2$ is the entropy of one bit, the usable entropy is approximately $5.4$ bits.
		\end{proof}
		
		\subsection{Landauer Limit for Bit Erasure}
		Consider erasing one bit of information within this system at a temperature $T$ (e.g., room temperature, $T = 300$ K). The minimum energy cost is:
		$$ E_{cost} = k_B T \ln 2 \approx (1.3806 \times 10^{-23} \text{ J/K}) \times (300 \text{ K}) \times \ln 2 \approx 2.87 \times 10^{-21} \text{ J} $$
		
		\subsection{Particle-Mediated Heat Dissipation of Landauer Energy}
		The energy $E_{cost}$ must be dissipated as heat, mediated by particles.
		
		Case 1: Photons. The minimum energy of a photon carrying this heat is $h f_{min} \ge E_{cost}$.
		$$ f_{min} \ge \frac{E_{cost}}{h} \approx \frac{2.87 \times 10^{-21} \text{ J}}{6.626 \times 10^{-34} \text{ J s}} \approx 4.33 \times 10^{12} \text{ Hz} $$
		This corresponds to the far-infrared region of the electromagnetic spectrum.
		
		Case 2: Electron Kinetic Energy. If an electron carries away this energy as kinetic energy that thermalizes, then $p^2 / (2m_e) \ge E_{cost}$.
		$$ p \ge \sqrt{2 m_e E_{cost}} \approx \sqrt{2 \times (9.109 \times 10^{-31} \text{ kg}) \times (2.87 \times 10^{-21} \text{ J})} \approx 7.23 \times 10^{-26} \text{ kg m/s} $$
		The de Broglie wavelength of such an electron would be $\lambda = h / p \approx 9.16 \times 10^{-9}$ meters, or $9.16$ nanometers.
		
		\section{Discussion}
		The significant reduction from the theoretical maximum of 624 bits to an estimated usable entropy of around 5 bits highlights the constraints imposed by the system's physical size and potentially the strong interactions between the electrons that limit their independent accessibility. The Landauer principle dictates a minimum energy cost for any irreversible information processing within this system, and this energy must be transferred to the environment through quantized particles. The frequencies of photons or the kinetic energies of electrons involved in such heat dissipation are determined by the temperature and the amount of entropy change.
		
		\section{Conclusion}
		This analysis of a 3-layer graphene-silicene-graphene heterostructure illustrates the fundamental principles governing the quantization of usable entropy and the Landauer principle in the context of particle-mediated heat dynamics. The usable information capacity of the system appears to be far below the theoretical maximum, likely due to size constraints and internal interactions. The Landauer limit sets a minimum energy scale for irreversible information processing, and this energy must be exchanged with the environment via quantized particles, providing a tangible link between information, thermodynamics, and quantum mechanics at the nanoscale.
		
	\end{document}