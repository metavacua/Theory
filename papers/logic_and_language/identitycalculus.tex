}

\maketitle

\begin{abstract}
A calculus of Horn clauses, equality, equivalence, identity, explicit substitution of expressions, reductions, and reflexivity.
\end{abstract}

\part{Preliminaries}
\begin{center}
	\begin{flushleft}
		The primary method is intended to be solely substitution of equals for equals.
	\end{flushleft}

	
\end{center}

\newpage
\part{Identity Calculi}
\begin{center}
	
	\section{Non-Structural Identity Calculus}
		\subsection{Structural Rules}
		\begin{center}
			\[
			\begin{prooftree}
			\infer0[Id]{A ⊢ A}
			\end{prooftree}
			\]
			
			\[
			\begin{prooftree}
			\hypo{Γ ⊢ A}
			\hypo{A ⊢ Δ}
			\infer2[Cut]{Γ ⊢ Δ}
			\end{prooftree}
			\]
		\end{center}
		
		\subsection{Operational Rules}
		\begin{center}
		
			\subsubsection{Multiplicatives}
			\begin{center}
				\[
				\begin{prooftree}
				\hypo{Γ ⊢ P}
				\infer1[Substitution of Expressions]{Γ ⊢ P[x/E]}
				\end{prooftree}
				\quad
				\begin{prooftree}
				\hypo{Γ ⊢ P = Q}
				\infer1[Equality of Expressions]{Γ ⊢ [x/P] = [x/Q]}
				\end{prooftree}
				\]
				
				\[
				\begin{prooftree}
				\hypo{Γ ⊢ P = Q}
				\hypo{Γ ⊢ Q = R}
				\infer2[Transitivity of Equality]{Γ ⊢ P = R}
				\end{prooftree}
				\quad
				\begin{prooftree}
				\hypo{Γ ⊢ P}
				\hypo{Γ ⊢ P \Leftrightarrow Q}
				\infer2[Equianimity]{Γ ⊢ Q}
				\end{prooftree}
				\]
				
				\[
				\begin{prooftree}
				\hypo{A, B ⊢}
				\hypo{⊢ A, B}
				\infer2{A \Leftrightarrow B ⊢ }
				\end{prooftree}
				\quad
				\begin{prooftree}
				\hypo{A ⊢ B}
				\hypo{B ⊢ A}
				\infer2{⊢ A \Leftrightarrow B}
				\end{prooftree}
				\]
			\end{center}
		\end{center}
		
		\subsection{Theorems}
		\begin{center}
			\begin{flushleft}
			\end{flushleft}
		\end{center}
\end{center}

\end{document}}*}
