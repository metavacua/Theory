\section{Introduction}

Building upon the foundational distinction between local and non-local relations, this thesis now focuses on the existential and universal quantifiers applied to locality and non-locality across physics, computation, and logic.  While our previous work established definitions and theorems pointing towards the refutation of universal locality, this iteration sharpens our focus on understanding the specific nature of existential non-locality and contrasting it with the hypothetical construct of universal locality.  We aim to clarify what it means for a process to be existentially non-local versus universally non-local in each domain, and to further explore the implications for our understanding of physical reality, computation, and logical inference.  The central question guiding this refined investigation is: Can we have existential locality and existential non-locality without either universal locality or universal non-locality?  And if universal locality is untenable, what does this imply about the nature of processes in these domains?

\section{Refined Definitions: Existential vs. Universal Locality and Non-Locality}

We refine our definitions to explicitly distinguish between existential and universal forms of locality and non-locality in physical, computational, and logical domains.

\subsection{Physical Domain}

\begin{definition}[Existential Physical Locality]
	\textbf{Existential physical locality} asserts that there exist at least some physical relations that are local, meaning influences are mediated at or below the speed of light, determined by properties within spacetime regions and their intersection.  Classical physics provides numerous examples of existentially local relations.
\end{definition}

\begin{definition}[Existential Physical Non-Locality]
	\textbf{Existential physical non-locality} asserts that there exists at least one physical relation that is non-local, violating the conditions of physical locality by exhibiting instantaneous or space-like separated correlations not explainable by local interactions. Quantum entanglement exemplifies existential physical non-locality.
\end{definition}

\begin{definition}[Universal Physical Locality]
	\textbf{Universal physical locality} (Hypothetical) posits that \textbf{all} physical relations in the universe are local. This would imply that no fundamental physical process exhibits non-local correlations, and the universe operates entirely according to local interactions. This is the null hypothesis (H0) we aim to refute.
\end{definition}

\begin{definition}[Universal Physical Non-Locality]
	\textbf{Universal physical non-locality} (Hypothetical) would assert that \textbf{all} fundamental physical relations are non-local. This extreme view would imply that locality is never fundamentally applicable in physics, and all correlations, even seemingly local ones, ultimately arise from non-local underpinnings.  This is a less explored and more speculative concept.
\end{definition}

\subsection{Computational Domain}

\begin{definition}[Existential Computational Locality]
	\textbf{Existential computational locality} describes computational processes that exhibit local causation (SG-LC) for at least some computational relations.  Traditional algorithms and classical computation largely operate within the bounds of existential computational locality.
\end{definition}

\begin{definition}[Existential Computational Non-Locality]
	\textbf{Existential computational non-locality} describes computational processes where at least one computational relation violates local causation. Quantum computation, particularly algorithms leveraging entanglement, can be considered existentially computationally non-local.
\end{definition}

\begin{definition}[Universal Computational Locality]
	\textbf{Universal computational locality} (Hypothetical) would mean that \textbf{all} computational relations in any possible computational system must adhere to local causation. This would restrict computation to step-by-step, locally determined processes, excluding any form of instantaneous or non-local computational influence.
\end{definition}

\begin{definition}[Universal Computational Non-Locality]
	\textbf{Universal computational non-locality} (Hypothetical) would imply that \textbf{all} computational relations are fundamentally non-local.  This could envision a form of computation where every step inherently involves instantaneous dependencies across the entire computational system, potentially exceeding the bounds of Turing-machine-like models.
\end{definition}

\subsection{Logical Domain}

\begin{definition}[Existential Logical Locality]
	\textbf{Existential logical locality} describes logical relations that are local for at least some logical contexts. Classical logic, with its context-independent truth values for many relations, embodies existential logical locality.
\end{definition}

\begin{definition}[Existential Logical Non-Locality]
	\textbf{Existential logical non-locality} describes logical relations whose truth values are context-dependent in a way that implies a dependence on broader, non-local logical structures for at least some logical relations. Paraconsistent logics, which handle contradictions that might arise from non-local or inconsistent information sources, can be seen as exploring existential logical non-locality.
\end{definition}

\begin{definition}[Universal Logical Locality]
	\textbf{Universal logical locality} (Hypothetical) would assert that \textbf{all} logical relations must be logically local. This would mean that context-dependence in logic is always reducible to local factors, and there are no truly non-local dependencies in logical truth.
\end{definition}

\begin{definition}[Universal Logical Non-Locality]
	\textbf{Universal logical non-locality} (Hypothetical) would imply that \textbf{all} logical relations are fundamentally logically non-local. In such a logical system, truth values would always be globally context-dependent, potentially making classical logical inference inapplicable.
\end{definition}

\section{Refined Hypotheses}

Our hypotheses are now refined to focus on the existential and universal nature of locality and non-locality.

\begin{hypothesis}[H0: Universal Locality (Null Hypothesis)]
	All fundamental physical, computational, and logical relations are universally local.  This implies universal physical locality, universal computational locality, and universal logical locality as defined above.
\end{hypothesis}

\begin{hypothesis}[H1: Existential Non-Locality (Alternative Hypothesis)]
	There exists existential non-locality in at least one of the physical, computational, or logical domains.  This means at least one of existential physical non-locality, existential computational non-locality, or existential logical non-locality is true.  This hypothesis does not preclude the existence of existential locality alongside existential non-locality.
\end{hypothesis}

\section{Theorems (Unchanged)}

Theorems 1 and 2 remain unchanged as they logically connect universal physical locality to universal computational and logical locality, and existential physical non-locality to existential computational or logical non-locality.  These theorems provide the theoretical backbone for testing our hypotheses.

\subsection{Theorem 1: Universal Physical Locality Implies Universal Computational and Logical Locality}
\subsection{Theorem 2: Existential Physical Non-Locality Implies Existential Computational or Logical Non-Locality}
\textit{(Proofs remain as in the previous version.)}

\section{Experiments (Unchanged)}

Experiments E0 and E1 are designed to test for deviations from universal locality and to confirm existential non-locality in the physical domain. Their descriptions and interpretations remain valid for testing our refined hypotheses.

\subsection{Experiment E0: Attempt to Refute Universal Locality (Null Hypothesis Test) - Classical System Locality Test}
\subsection{Experiment E1: Test for Existential Non-Locality (Alternative Hypothesis Test) - Quantum Entanglement Bell Test}
\textit{(Experiment descriptions and interpretations remain as in the previous version.)}

\section{Discussion: Existential Locality, Existential Non-Locality, and the Refutation of Universal Locality}

The refined definitions and hypotheses allow us to more precisely discuss the implications of locality and non-locality.

\subsection{Existential Locality: The Classical Approximation}

Existential locality is readily observed in classical physics, computation, and logic.  Classical mechanics, electromagnetism (in its original formulation), and general relativity are fundamentally local theories.  Classical computation, based on Turing machines and similar models, operates through step-by-step local causation.  Classical logic, such as propositional and first-order logic, often assumes context-independent truth values for many relations. These frameworks demonstrate the effectiveness and applicability of \textbf{existential locality} in describing a vast range of phenomena.  For instance:

\begin{itemize}
	\item \textbf{Physical:}  The propagation of sound waves, the motion of macroscopic objects under gravity, and electromagnetic waves in classical electrodynamics exemplify existentially local physical relations.
	\item \textbf{Computational:}  Sorting algorithms, search algorithms, and rule-based expert systems are examples of existentially computationally local processes.
	\item \textbf{Logical:}  The logical relation of implication in propositional logic, where $P \implies Q$ is determined solely by the truth values of $P$ and $Q$, is an example of existential logical locality.
\end{itemize}
However, the existence of these local descriptions does not necessitate \textbf{universal locality}.

\subsection{Existential Non-Locality: Quantum Reality and Beyond}

Experiment E1, the Bell test, is designed to demonstrate \textbf{existential physical non-locality} through the violation of Bell inequalities.  Quantum entanglement provides a clear example of a physical relation that is non-local, challenging the classical assumption of universal locality.  Furthermore, we can identify potential instances of existential non-locality in computation and logic:

\begin{itemize}
	\item \textbf{Physical:} Quantum entanglement, as experimentally verified, is the prime example of existential physical non-locality.
	\item \textbf{Computational:} Quantum algorithms that exploit entanglement, such as quantum teleportation or certain quantum search algorithms, demonstrate \textbf{existential computational non-locality}. The computational relations in these algorithms are not reducible to sequences of local causal steps in the classical sense.
	\item \textbf{Logical:} Paraconsistent logics, designed to handle contradictions without logical explosion, can be seen as exploring \textbf{existential logical non-locality}.  In situations where information sources are non-locally correlated or inconsistent (as might arise from quantum measurements or distributed systems), logical relations may need to be context-dependent in a non-local way to manage these inconsistencies. For example, consider a logical system reasoning about measurements on entangled particles; the logical relations describing correlations might need to reflect the non-local nature of entanglement.
\end{itemize}

\subsection{Hypothetical Universal Locality and Non-Locality}

\textbf{Universal locality}, if true, would simplify our understanding of the universe, implying a clockwork-like, deterministic, and locally interacting reality across all domains. However, experimental evidence, particularly from Bell tests, challenges this view in the physical domain.  If universal physical locality were true, by Theorem 1, we would also expect universal computational and logical locality, severely restricting the scope of computation and logic beyond classical paradigms.

\textbf{Universal non-locality}, while more speculative, presents a radical alternative.  A universe of universal physical non-locality would be profoundly interconnected, where every event could instantaneously influence every other event, challenging our notions of causality and spacetime.  In computation, universal computational non-locality might imply computational processes that are fundamentally holistic and non-sequential.  In logic, universal logical non-locality could lead to a completely context-dependent logic where truth is always relative to a global logical state. While intriguing, there is currently no empirical evidence suggesting universal non-locality in any domain, and it poses significant conceptual challenges.

\subsection{Refutation of Universal Locality and the Necessity of Existential Non-Locality}

The weight of experimental evidence from quantum mechanics, particularly Bell inequality violations, strongly suggests that \textbf{universal physical locality (H0) is false}. Experiment E1 is designed to provide further empirical support for this refutation.  Given Theorem 2, the existence of physical non-locality implies the necessity of considering non-locality in computation or logic as well.  Therefore, while \textbf{existential locality} provides a useful and often accurate approximation, especially in classical domains, it is not universally applicable.  The universe, at its fundamental level, appears to exhibit \textbf{existential non-locality}, demanding that our physical, computational, and logical frameworks must be expanded to accommodate and understand non-local relations.

\section{Conclusion}

This refined thesis has explored the concepts of existential and universal locality and non-locality across physical, computational, and logical domains. By distinguishing between existential and universal forms, we have clarified the scope and implications of locality and non-locality. We have argued that while existential locality is evident and useful in classical approximations, \textbf{universal locality (H0) is likely false}, particularly in light of quantum mechanics and the expected outcomes of Bell test experiments like E1.  The confirmation of existential physical non-locality necessitates the acceptance and exploration of \textbf{existential non-locality} in computational and logical domains as well.  Moving forward, the development of non-classical physical theories, non-local computational paradigms (like quantum computation), and context-sensitive logical systems (like paraconsistent logics) is crucial for a more complete and accurate understanding of reality, computation, and inference in a universe that transcends the limitations of universal locality and embraces the richness of existential non-locality.