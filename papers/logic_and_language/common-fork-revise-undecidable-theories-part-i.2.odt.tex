∃∀ \{⊥,⊤,¬,∨,∧,→,↛,↔,⊕,↓,↑\} \{p, q\} ⊢⊬⊨

\protect\hypertarget{anchor}{}{}Theories with Standard Formalization

The theories discussed in this paper will be referred to as
\emph{theories with standard formalization}. They can be briefly
characterized as theories which are formalized within the first-order
predicate logic (with identity, with quantifiers, without variable
predicates).

\protect\hypertarget{anchor-1}{}{}Syntactic Definitions

\protect\hypertarget{anchor-2}{}{}Definition: Variables and Constants of
a Theory

The symbols which occur in expressions of a given theory T are divided
into \emph{variables} and \emph{constants}.

The set of variables is assumed to be denumerable and hence infinite;
the set of constants is either finite or denumerable. All the variables
are treated as ranging over the same set of elements.

\protect\hypertarget{anchor-3}{}{}Definition: Logical Constants of a
Theory

The constants are divided into \emph{logical} and \emph{non-logical}
ones.

The logical constants are the \emph{sentential connectives}

the \emph{negation sign} ¬,

the \emph{implication sign} →,

the \emph{equivalence sign} ↔,

the \emph{disjunction sign} ∨,

the \emph{conjunction sign }∧;

the quantifiers

the \emph{universal quantifier} ∀ and \emph{existential quantification}
∃;

and finally, the i\emph{dentity symbol} \textbf{=}.

\protect\hypertarget{anchor-4}{}{}

\protect\hypertarget{anchor-5}{}{}Definition: non-Logical Constants of a
Theory AKA Signature of a Theory

The non-logical constants are the \emph{predicates} (or \emph{relation
symbols}), the \emph{operation symbols}, and the \emph{individual
constants}.

With every predicate and every operation symbol a positive integer is
correlated which is called the \emph{rank} of the symbol. Thus, we may
have in T \emph{unary} predicates and operation symbols (I.E. symbols of
rank 1), \emph{binary} predicates and operation symbols (symbols of rank
2), etc. The identity symbol, though regarded as a logical constant, is
included in the set of binary predicates.

In practice, in addition to variables and constants, the so-called
technical symbols like parentheses and commas, are also used in
constructing expressions; however, theoretically these technical symbols
can be dispensed with.

\protect\hypertarget{anchor-6}{}{}Definition: Terms

Among expressions (I.E. finite concatenations of symbols) we distinguish
\emph{terms} and \emph{formulas}.

The simplest, so-called atomic, terms are the variables and the
individual constants; a compound term is obtained by combining \emph{n}
simpler terms by means of an operation symbol of rank \emph{n}.

\protect\hypertarget{anchor-7}{}{}Definition: Formulas

Similarly, an atomic formula is obtained by combining \emph{n} arbitrary
terms by means of a predicate of rank \emph{n}; compound formulas are
built from simpler ones by means of sentential connectives and
quantifier expressions (I.E. quantifiers followed by variables like ∀x
or ∃y).

\protect\hypertarget{anchor-8}{}{}Definition: Sentences

An occurrence of a variable in a formula may be either \emph{free} or
\emph{bound}; a formula in which no variable occurs free is called a
\emph{sentence}.

\protect\hypertarget{anchor-9}{}{}Semantic Definitions

\protect\hypertarget{anchor-10}{}{}Proof-Theoretical Semantics

\protect\hypertarget{anchor-11}{}{}Definition: Axiomatic Proof

Two further notions, those of \emph{logical derivability} and
\emph{logical validity}, are involved in the metamathematical discussion
of any theory T.

First, we single out certain sentences of T which are referred to as
\emph{logical axioms}.

Secondly, we describe certain (finitary) operations, the so-called
\emph{operations of inference}, which when performed on sentences yield
new sentences.

Usually the set of logical axioms is infinite while the set of
operations of inference is finite.

\protect\hypertarget{anchor-12}{}{}Definition: Modus Ponens; the method
of affirmation.

The most important operation of inference is that of \emph{detachment}
(\emph{modus ponens}), which when applied to two sentences Φ and Φ→Ψ
yields the sentence Ψ.

In fact, it proves to be possible, by selecting a suitable set of
logical axioms, to use the operations of detachment as the only
operation of inference in formulating adequate definitions of
derivability and logical validity.

\protect\hypertarget{anchor-13}{}{}Definition: Derivable Sentences

A sentence is now said to be \emph{logically derivable} or simply
\emph{derivable} from a set A of sentences if it can be obtained from
sentences of A and from logical axioms by performing operations of
inference an arbitrary number of times.

\protect\hypertarget{anchor-14}{}{}Definition: Logically Valid or
Provable Sentences

A sentence is called \emph{logically valid} (or \emph{logically
provable}) if it is derivable from the set of logical axioms-\/-or what
amounts to the same thing, from the empty set of sentences.

\protect\hypertarget{anchor-15}{}{}Model-Theoretical Semantics

Another method of defining logically derivable and logically valid is
available which essentially involves the use of some semantical notions
and the notion of satisfaction.

\protect\hypertarget{anchor-16}{}{}Definition: Possible Realizations of
a Theory and the Universe of 𝕽

We assume that all the non-logical constants of T have been arranged in
a (finite or infinite) sequence \textless{}\textbf{C\_0}, \ldots,
\textbf{C\_n}, \ldots\textgreater, without repeating terms.

We consider systems 𝕽 formed by a non-empty set U and by a sequence
\textless C\_0, \ldots, C\_n, \ldots\textgreater{} of certain
mathematical entities, with the same number of terms as the sequence of
non-logical constants.

The mathematical nature of each C\_n depends on the logical character of
the corresponding constant \textbf{C\_n}. Thus,

if \textbf{C\_n} is a unary predicate, then C\_n is a subset of U; more
generally, if \textbf{C\_n} is an \emph{m}-ary predicate, then C\_n is
an \emph{m}-ary relation the field of which is a subset of U.

If \textbf{C\_n} is an \emph{m}-ary operation symbol, C\_n is an
\emph{m}-ary operation (function of \emph{m} arguments) defined over
arbitrary ordered \emph{m}-tuples \textless{}\emph{x\_1}, \ldots,
\emph{x\_m}\textgreater{} of elements of U and assuming elements of U as
values.

If \textbf{C\_n} is an individual constant, C\_n is simply an element of
U.

Such a system (sequence) 𝕽 = \textless U, C\_0, \ldots, C\_n,
\ldots\textgreater{} is called a \emph{possible realization} or simply a
\emph{realization} of T; the set U is called the \emph{universe} of 𝕽.

\protect\hypertarget{anchor-17}{}{}Definition: Satisfaction

We assume it to be clear under what conditions a sentence Φ of T is said
to \emph{be satisfied} or to \emph{hold} in a given realization 𝕽.

Roughly speaking, this means that Φ turns out to be true if

(i) all the variables occurring in T are assumed to range over the set
U;

(ii) the logical constants are interpreted in the usual way;

(iii) each of the non-logical constants \textbf{C\_n} is understood to
denote the corresponding term C\_n in 𝕽.

Assume, e.g., that the term \textbf{C\_n} in the sequence of constants
is a unary predicate and that consequently C\_n is a subset of U. Then
the sentence ∀x \textbf{C\_n} x holds in 𝕽 if and only if every element
of U is an element of C\_n and hence C\_n coincides with U.

\protect\hypertarget{anchor-18}{}{}Definition: Logical Consequence and
Logical Truth

A sentence Φ is said to be a \emph{logical consequence} of a set A of
sentences if it is satisfied in every realization 𝕽 in which all
sentences of A are satisfied; it is called \emph{logical true} if it is
satisfied in every possible realization.

As opposed to the notions of logical consequence and logical truth, the
related notions of logical derivability and logical validity, when
defined in terms of axioms and operations of inference, seem to have a
rather accidental and arbitrary character.

\protect\hypertarget{anchor-19}{}{}Equivalence of Proof-Theoretical and
Model-Theoretical Semantics for Theories in Standard Formalization

Hence, it might seem natural to redefine logical derivability and
logical validity simply by stipulating that a sentence is derivable from
A if it is a logical consequence of A, and by identifying logically
valid sentences with logically true sentences. From the results in Gödel
{[}5{]} it follows that under the systems of logical axioms and
operations of inference known from the literature the two methods of
defining derivability and logical validity are entirely equivalent (when
applied to theories with standard formalization).

An important property of the notion of derivability is stated in the
following well-known theorem, which is often applied in
meta-mathematical discussion:

\protect\hypertarget{anchor-20}{}{}Deduction Theorem I

Let A be a set of sentences of a theory T and let Φ\_1, ..., Φ\_n, Ψ be
any sentences of T.

For Ψ to be derivable from the set A supplemented by the sentences Φ\_1,
..., Φ\_n it is necessary and sufficient that the sentence

(Φ\_1∧...∧Φ\_n) → Ψ

be derivable from the set A alone.

\protect\hypertarget{anchor-21}{}{}Deduction Theorem II

Let A be a set of sentences of a theory T, and let Ψ be a sentence of T.

For Ψ to be derivable from A it is necessary and sufficient that A be
empty and Ψ be logically valid or else that A contain some sentences
Φ\_1, ..., Φ\_n such that the sentence

(Φ\_1∧...∧Φ\_n) → Ψ

is logically valid.

Thus the notion of derivability has a simple characterization in terms
of logically valid sentences.

\protect\hypertarget{anchor-22}{}{}Definition: Valid Sentences

To complete the description of a theory T we have to define what we mean
by a \emph{valid} sentence in general (as opposed to a logically valid
sentence). No uniform method for defining this notion is available.

\protect\hypertarget{anchor-23}{}{}Definition: Valid Sentences by
Non-Logical Axioms

Often we single out a (finite or infinite) set of sentences called
\emph{non-logical axioms}, and define a sentence to be valid if and only
if it is derivable from this set-\/-or, what amounts to the same, from
the set of all axioms, both logical and non-logical.

\protect\hypertarget{anchor-24}{}{}

\protect\hypertarget{anchor-25}{}{}Definition: Axiomatic Theories

Theories in which the notion of validity has been introduced in this way
are referred to as \emph{axiomatically built} or, simply
\emph{axiomatic} theories; when referring to such theories, we often use
the term ``\emph{provable}'' instead of ``\emph{valid}''.

We do not restrict ourselves to the discussion of axiomatic theories.
Sometimes we agree to consider as valid those and only those sentences
which are satisfied in a given realization or in all realizations of a
given class; sometimes we define validity for a theory in terms of
validity for some other theories for which this notion has been
previously defined.

We assume, however, that for each of the theories discussed the notion
of validity has been defined in one way or another.

\protect\hypertarget{anchor-26}{}{}Definition: General Condition for
Validity of Sentences

We assume that under this definition every sentence which is logically
derivable from a set of valid sentences is itself valid, and that
consequently every logically valid sentence is valid; this is the only
condition imposed upon the definition of validity.

\protect\hypertarget{anchor-27}{}{}Definition: Model of a Theory

A possible realization in which all valid sentences of a theory T are
satisfied is called a \emph{model} of T.

In all the theories with standard formalization the same symbols are
assumed to be used as variables and logical constants; apart from
differences in non-logical constants, the same expressions are regarded
as formulas, sentences, logical axioms, and logically valid sentences.

However, the notions of validity in these theories may of course exhibit
essential differences.

\protect\hypertarget{anchor-28}{}{}Theoretical Definitions

\protect\hypertarget{anchor-29}{}{}Definition: Uniqueness of a Theory in
Standard Formalization

A theory is uniquely determined by the set of all its valid sentences;
two theories are regarded as identical if their sets of valid sentences
coincide.

\protect\hypertarget{anchor-30}{}{}Definition: Uniqueness of an
Axiomatic Theory in Standard Formalization

An axiomatic theory is uniquely determined by its non-logical constants
and non-logical axioms.

\protect\hypertarget{anchor-31}{}{}Definition: Subtheory or Supertheory;
Subtension or Extension

A theory T\_1 is called a \emph{subtheory} of a theory T\_2 if every
sentence which is valid in T\_1 is also valid in T\_2; under the same
conditions T\_2 is referred to as an \emph{extension} of T\_1.

\protect\hypertarget{anchor-32}{}{}Definition: Inessential Extensions of
Theories in Standard Formalization

An extension T\_2 of T\_1 is called \emph{inessential} if every constant
of T\_2 which does not occur in T\_1 is an individual constant and if
every valid sentence of T\_2 is derivable in T\_2 from a set of valid
sentences of T\_1.

If T\_1 is axiomatic, then an inessential extension of T\_1 is obtained
by adding some new individual constants, but without adding any new
non-logical axioms.

By saying that a sentence Φ is derivable \emph{in a theory} T from a set
A we stress the fact that, in deriving Φ, we may use both sentences of A
and logical axioms of T. It is easily seen that, whenever Φ is derivable
from A in some theory T, it is also derivable from A in every theory
T\textquotesingle{} which contains all the non-logical constants
occurring in Φ and in sentences of A.

\protect\hypertarget{anchor-33}{}{}Definition: Finite Extensions of
Theories in Standard Formalization

An extension T\_2 of T\_1 is referred to as a \emph{finite} extension if
there is a finite set A of valid sentences of T\_2 such that every valid
sentence of T\_2 is derivable from a set of sentences which are valid in
T\_1 or belong to A. Clearly every inessential extension is a finite
extension.

\protect\hypertarget{anchor-34}{}{}Definition: Union of Theories in
Standard Formalization

Among the extensions common to two given theories T\_1 and T\_2 there is
always a smallest one, which is a subtheory of any other common
extension; this smallest common extension is referred to as the
\emph{union} of the given theories.

The union T of T\_1 and T\_2 is fully characterized by the following two
conditions:

(i) the set of all non-logical constants of T is the (set theoretical)
union of the set of all non-logical constants of T\_1 and T\_2;

(ii) a sentence is valid in T if and only if it is derivable in T from a
set of sentences which are valid in T\_1 or T\_2.

If the theories T\_1 and T\_2 are axiomatic, we can construct T by
postulating, in addition to (i), the analogous condition for the set of
non-logical axioms.

Notice that condition (i) unambiguously determines the notions of a
sentence of T and of a logical axiom of T, and hence also the notion of
derivability in T.

\protect\hypertarget{anchor-35}{}{}Definition: Consistent Theories

A theory T is called \emph{consistent} if not every sentence of T is
valid in T; or, in an equivalent formulation, if there exists no
sentence such that both Φ and ¬Φ are valid in T.

\protect\hypertarget{anchor-36}{}{}Definition: Complete Theories

A theory T is called \emph{complete} if there is no consistent extension
of T which is different from T, but which has the same constants as T;
equivalently, if, for every sentence Φ of T, either Φ or ¬Φ is valid in
T.

The proof of the equivalence of the two definitions of completeness is
based upon Deduction Theorem I.

\protect\hypertarget{anchor-37}{}{}Definition: Compatible Theories

Two theories T\_1 and T\_2 are said to be \emph{compatible} if they have
a common consistent extension; this is equivalent to saying that the
union of T\_1 and T\_2 is consistent.