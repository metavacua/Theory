}

∃∀¬∧∨→↛↔⊕

\hypertarget{theorem-1-of-undecidable-theories-part-i}{%
\subsection{Theorem 1 of Undecidable Theories Part
I}\label{theorem-1-of-undecidable-theories-part-i}}

For all theories T in standard formalization.

If T is complete then (T is decidable ⨁ T is essentially undecidable and
T is decidable ⇔ T is axiomatizable).

\hypertarget{theorem-1-complete}{%
\subsubsection{\texorpdfstring{Theorem 1 COMPLETE
}{Theorem 1 COMPLETE }}\label{theorem-1-complete}}

For any complete theory T the following conditions are equivalent: T is
undecidable. T is essentially undecidable. T is non-axiomatizable.

\hypertarget{theorem-1-axiom}{%
\subsubsection{Theorem 1 AXIOM ?}\label{theorem-1-axiom}}

For any non-axiomatizable theory T the following three conditions are
equivalent: T is undecidable. T is essentially undecidable. T is
complete?

\hypertarget{theorem-1-consist}{%
\subsubsection{Theorem 1 CONSIST ?}\label{theorem-1-consist}}

For every non-consistent theory the following are equivalent:

T is finitely axiomatizable, T is axiomatizable, T is complete.

fAX(T), AX(T), and Complete(T).

\hypertarget{theorem-1-decide-x}{%
\subsubsection{Theorem 1 DECIDE X}\label{theorem-1-decide-x}}

\st{For any decidable theory T the following three conditions are
equivalent: T is axiomatizable. T is finitely axiomatizable. T is
consistent.}\\
\strut \\
Probably deals with SU(T) and not complete(T).

\hypertarget{theorem-2-of-undecidable-theories-part-i}{%
\subsection{Theorem 2 of Undecidable Theories Part
I}\label{theorem-2-of-undecidable-theories-part-i}}

For all theories T in standard formalization.

T is essentially undecidable if and only if T is consistent and:\\
for all T\_e at least one of the following properties is false

T\_e is consistent.

T\_e is complete.

T\_e is an extension of T

T\_e has the same constants as T.

T\_e is axiomatizable.

For all theories T in standard formalization.

T is essentially undecidable if and only if T is consistent and:

for all T\_e. (T\_e is paraconsistent, T\_e is complete, T\_e is an
extension of T, T\_e has the same constants as T, T\_e is recursively
enumerably axiomatizable).

For all theories T in standard formalization.

T is essentially undecidable if and only if T is consistent and: for all
T\_e. (T\_e is consistent, T\_e is paracomplete, T\_e is an extension of
T, T\_e has the same constants as T, T\_e is recursively enumerably
axiomatizable).

For all theories T in standard formalization.

T is essentially undecidable if and only if T is consistent and: for all
T\_e. (T\_e is consistent, T\_e is complete, T\_e is a non-conservative
extension of T, T\_e has the same constants as T, T\_e is recursively
enumerably axiomatizable).

For all theories T in standard formalization.

T is essentially undecidable if and only if T is consistent and: for all
T\_e. (T\_e is consistent, T\_e is complete, T\_e is an extension of T,
T\_e has the different constants than T, T\_e is recursively enumerably
axiomatizable).

For all theories T in standard formalization.

T is essentially undecidable if and only if T is consistent and: for all
T\_e. (T\_e is consistent, T\_e is complete, T\_e is a extension of T,
T\_e has the same constants as T, T\_e is non-axiomatizable).

\hypertarget{lemma}{%
\subsubsection{\texorpdfstring{Lemma }{Lemma }}\label{lemma}}

∀T' in standard formalization.

Consistent(T') and Decide(T') if and only if
∃T\textquotesingle'{[}Consistent(T'\textquotesingle)∧
Complete(T'\textquotesingle)∧Decide(T'\textquotesingle)∧Axiomatize(T\textquotesingle')∧Extend(T',
T\textquotesingle\textquotesingle)∧sameConstants(T',
T\textquotesingle\textquotesingle){]}.

Equivalently,

∀T' in standard formalization.

NAND(Consistent(T'), Decide(T')) if and only if
∀T\textquotesingle'{[}¬Consistent(T'\textquotesingle)∨¬Complete(T'\textquotesingle)∨¬Decide(T'\textquotesingle)∨¬Axiomatize(T\textquotesingle')∨¬Extend(T',
T\textquotesingle\textquotesingle){]}.

\hypertarget{theorem-2-for-hereditary-undecidability}{%
\subsection{Theorem 2 for Hereditary
Undecidability}\label{theorem-2-for-hereditary-undecidability}}

For all theories T in standard formalization.

T is hereditarily undecidable if and only if T is consistent and:\\
for all T\_s at least one of the following properties is false

T\_s is consistent.

T\_s is complete.

T\_s is a subtheory of T

T\_s has the same constants as T.

T\_s is axiomatizable.

\hypertarget{theorem-2-exclusive-essential-undecidability-of-t}{%
\subsection{Theorem 2: Exclusive Essential Undecidability of
T}\label{theorem-2-exclusive-essential-undecidability-of-t}}

EU(T)↛HU(T) iff Consistent(T) ∧ ∀T\_e ∃T\_s{[}(Extend(T,
T\_e)→C(T\_e))∧(Subtend(T, T\_s)↛C(T\_s)){]}

\hypertarget{theorem-2-exclusive-hereditary-undecidability-of-t}{%
\subsection{Theorem 2: Exclusive Hereditary Undecidability of
T}\label{theorem-2-exclusive-hereditary-undecidability-of-t}}

HU(T)↛EU(T) iff Consistent(T) ∧ ∀T\_s ∃T\_e{[}(Subtend(T,
T\_s)→C(T\_s))∧(Extend(T, T\_e)↛C(T\_e)){]}

\hypertarget{theorem-2-total-undecidability-of-t}{%
\subsection{Theorem 2: Total Undecidability of
T}\label{theorem-2-total-undecidability-of-t}}

C(x) = ¬Consistent(x)∨¬Complete(x)∨¬Axiomatize(x)

EU(T) and HU(T) iff Consistent(T) ∧ ∀Y. ( (Extend(T, Y) or Subtend(T, Y)
) → C(Y) )

\hypertarget{theorem-2-denial-of-the-essential-or-hereditary-undecidability-of-t}{%
\subsection{Theorem 2: Denial of the Essential or Hereditary
Undecidability of
T}\label{theorem-2-denial-of-the-essential-or-hereditary-undecidability-of-t}}

Neither EU(T) nor HU(T) iff Consistent(T)→{[}∃T\_e. (Extend(T,
T\_e)↛C(T\_e))∧∃T\_s.(Subtend(T, T\_s)↛C(T\_s)){]}

\hypertarget{generalization-of-theorem-2-of-ut}{%
\subsection{Generalization of Theorem 2 of
UT}\label{generalization-of-theorem-2-of-ut}}

C(x) = ¬Consistent(x)∨¬Complete(x)∨¬Axiomatize(x)∨¬sameConstants(x)

EU(T) and HU(T) iff Consistent(T) ∧ ∀Y. ( (Extend(T, Y) or Subtend(T, Y)
) → C(Y) )

EU(T) and not HU(T) iff Consistent(T) ∧ ∀T\_e ∃T\_s{[}Extend(T,
T\_e)→C(T\_e){]}∧{[}Subtend(T, T\_s)∧¬C(T\_s){]}

not EU(T) and HU(T) iff Consistent(T) ∧ ∀T\_s ∃T\_e{[}Subtend(T,
T\_s)→C(T\_s){]}∧{[}Extend(T, T\_e)∧¬C(T\_e){]}

Neither EU(T) nor HU(T) iff Consistent(T)→{[}∃T\_e. (Extend(T,
T\_e)↛C(T\_e))∧∃T\_s.(Subtend(T, T\_s)↛C(T\_s)){]}

\hypertarget{theorem-3-of-ut-for-hu}{%
\subsection{Theorem 3 of UT for HU}\label{theorem-3-of-ut-for-hu}}

Let T\_1 and T\_2 be two theories such that T\_1 is a consistent
subtheory of T\_2. If HU(T\_2) then HU(T\_1).

\hypertarget{theorem-3-of-ut-for-tu}{%
\subsection{Theorem 3 of UT for TU}\label{theorem-3-of-ut-for-tu}}

If any theory T is isomorphic to or an extension of or a subtheory of a
totally undecidable theory then that theory is totally undecidable.

\hypertarget{definition}{%
\subsection{Definition}\label{definition}}

A subtheory T\_1 of T\_2 is called inessential if every constant of T\_2
which does not occur in T\_1 is an individual constant and if every
valid sentence of T\_2 is derivable in T\_2 from a set of valid
sentences of T\_1.

\hypertarget{theorem-4-of-ut-for-hu}{%
\subsection{Theorem 4 of UT for HU}\label{theorem-4-of-ut-for-hu}}

Let T\_1 and T\_2 be two theories such that T\_1 is an inessential
subtheory of T\_2.

T\_1 is undecidable or hereditarily undecidable if and only if T\_2 is
respectively undecidable or hereditarily undecidable.

\hypertarget{generalization-of-theorem-4-of-ut}{%
\subsection{Generalization of Theorem 4 of
UT}\label{generalization-of-theorem-4-of-ut}}

Let T\_1 and T\_2 be two theories such that T\_2 is an inessential
extension of T\_1 or T\_1 is an inessential subtheory of T\_2.

T\_1 is undecidable or totally undecidable if and only if T\_2 is
respectively undecidable or totally undecidable.

\hypertarget{theorem-5-of-ut-for-finite-subtheories}{%
\subsection{Theorem 5 of UT for Finite
Subtheories}\label{theorem-5-of-ut-for-finite-subtheories}}

Let T\_1 and T\_2 be two theories with the same constants such that T\_1
is a finite subtheory of T\_2.

If T\_1 is undecidable then T\_2 is also undecidable.

\hypertarget{generalization-of-theorem-6-of-ut}{%
\subsection{Generalization of Theorem 6 of
UT}\label{generalization-of-theorem-6-of-ut}}

Let T\_1 and T\_2 be two compatible theories such that every constant of
T\_2 is also a constant of T\_1. If {[}EU(T\_2) or TU(T\_2){]} and
fAxiomatize(T\_2), then {[}HU(T\_1) or TU(T\_1){]}.

\hypertarget{theorem}{%
\subsection{Theorem}\label{theorem}}

If a theory is essentially undecidable and it has a hereditarily
undecidable extension then both the theory and its supertheory is
totally undecidable.

For all T in STDF.

If HU(T) and there exists T\textquotesingle{} {[}EU(Subtend(T,
T\textquotesingle){]} then TU(T) and TU(T\textquotesingle).

\hypertarget{theorem-1}{%
\subsection{Theorem}\label{theorem-1}}

If a theory is essentially undecidable and a hereditarily undecidable
theory can be interpreted in an extension of the essentially undecidable
theory then the theory, the hereditarily undecidable theory, and the
extension of the essentially undecidable theory are all totally
undecidable.

\hypertarget{theorem-2}{%
\subsection{Theorem}\label{theorem-2}}

If a theory is hereditarily undecidable and it has an essentially
undecidable subtheory then both the theory and its subtheory are totally
undecidable.

For all T in STDF.

If EU(T) and there exists T\textquotesingle{}
{[}HU(Extend(T,T\textquotesingle){]} then TU(T) and
TU(T\textquotesingle).

\hypertarget{theorem-3}{%
\subsection{Theorem}\label{theorem-3}}

If a theory is hereditarily undecidable and an essentially undecidable
theory can be interpreted in a subtheory of the essentially undecidable
theory then the theory, the essentially undecidable theory, and the
subtheory of the hereditarily undecidable theory are all totally
undecidable.

\hypertarget{theorem-4}{%
\subsection{Theorem}\label{theorem-4}}

For all theories T in standard formalization.

T is totally undecidable if and only if there does not exist a
consistent decidable extension of HU(T) and there does not exist a
consistent decidable subtheory of EU(T).

\hypertarget{definition-1}{%
\subsection{Definition}\label{definition-1}}

A theory is said to be decidedly pinched if it only has HU subtensions
and EU extensions.

\hypertarget{definition-2}{%
\subsection{Definition}\label{definition-2}}

A theory is said to be essentially decidable if and only if the theory
is decidable and every extension of the theory is decidable.

\hypertarget{definition-3}{%
\subsection{Definition}\label{definition-3}}

A theory is said to be hereditarily decidable if and only if the theory
is decidable and every subtension of the theory is decidable.

\hypertarget{definition-4}{%
\subsection{Definition}\label{definition-4}}

A theory is said to be totally decidable if and only if the theory is
decidable and every extension and subtension of the theory is decidable.

\hypertarget{conjecture}{%
\subsection{Conjecture}\label{conjecture}}

There exists some morphism that preserves the non-triviality of a theory
in transformations from theory to theory.

\hypertarget{conjecture-1}{%
\subsection{Conjecture}\label{conjecture-1}}

There exists some morphism that preserves non-triviality and expressive
strength of a theory from theory to theory.

\hypertarget{lemma-of-theorem-2-of-ut}{%
\subsection{Lemma of Theorem 2 of UT}\label{lemma-of-theorem-2-of-ut}}

For all T' in standard formalization.

Consistent(T') and Decide(T') if and only if T' has a Consistent(T'),
Complete(T'), Decide(T'), Axiomatize(T'), Extend(T').

\hypertarget{theorem-5}{%
\subsection{Theorem}\label{theorem-5}}

A theory is essentially decidable if and only if Axiomatize(T) and for
all T\_e, Extend(T, T\_e) and Axiomatize(T\_e).

\hypertarget{theorem-6}{%
\subsection{Theorem}\label{theorem-6}}

A theory is totally decidable if and only if Axiomatize(T) and for all
Y, {[}Extend(T, Y) or Subtend(T, Y){]} and Axiomatize(Y).

\hypertarget{theorem-7}{%
\subsection{Theorem}\label{theorem-7}}

A theory is totally axiomatizable if and only if Axiomatize(T) and for
all Y, {[}Extend(T, Y) or Subtend(T, Y){]} and Axiomatize(Y).

\hypertarget{summary-of-combinations}{%
\subsection{Summary of combinations}\label{summary-of-combinations}}

{[}Complete(x)↔Axiomatize(x){]}→Decide(x).

¬Decide(x)→{[}Complete(x)⊕Axiomatize(x){]}.

SU(T)⊕Complete(T)

Any axiomatizable theory in SU will not be complete.

Theories which are axiomatizable can be in Decide, SU, or EU.

Theories which are non-axiomatizable can be in SU or EU or TU.

Theories which are complete can be in Decide or EU.

Theories which are not complete can be in Decide, SU, or EU.\\
\strut \\
Theories which are finitely axiomatizable can be in Decide, SU, or EU.

Theories which are not finitely axiomatizable can be in Decide, SU, or
EU.\\
\strut \\
Theories which are consistent can be in Decide, SU, or EU.\\
Theories which are not consistent can be in Decide or SU(?).\\
\strut \\
Theories which are inconsistent strictly in Decide.

Theories which are not axiomatizable and complete are strictly in EU.

Theories which are axiomatizable and not complete can be in Decide, SU,
or EU.

Theories which are not axiomatizable and not complete can be in SU or
EU.

Theories which are axiomatizable and complete strictly in Decide.

Theories which are finitely axiomatizable and complete are strictly in
Decide.

Theories which are finitely axiomatizable and not complete can be in
Decide, SU, or EU.

Theories which are not finitely axiomatizable and complete can be in
Decide, or EU.

Theories which are not finitely axiomatizable and not complete can be in
Decide, SU, or EU.

\hypertarget{jointly-exhaustive-and-mutually-exclusive-theories-theorem}{%
\subsection{Jointly Exhaustive and Mutually Exclusive Theories
Theorem}\label{jointly-exhaustive-and-mutually-exclusive-theories-theorem}}

{[}Axiomatize(T)↛fAxiomatize(T){]}⊕fAxiomatize(T)⊕¬Axiomatize(T) if and
only if Inconsistent(T)⊕Paraconsistent(T)⊕Consistent(T) if and only if
Decide(T)⊕SU(T)⊕EU(T) if and only if Incomplete⊕Complete⊕Overcomplete.

\hypertarget{inductive-su-theorem}{%
\subsection{Inductive SU Theorem}\label{inductive-su-theorem}}

Every SU(T) has a SU(T) extension that has a Consistent, Complete,
Decidable, and Axiomatizable extension.

\hypertarget{recursive-su-theorem}{%
\subsection{Recursive SU Theorem}\label{recursive-su-theorem}}

Every SU(T) has a SU(T) and a Decide(T) extension.

\hypertarget{theorem-8}{%
\subsection{Theorem}\label{theorem-8}}

The inductive and recursive definitions of SU(T) are equivalent.

\hypertarget{axiom}{%
\subsection{AXIOM}\label{axiom}}

\hypertarget{theorem-9}{%
\subsubsection{Theorem}\label{theorem-9}}

\st{A theory in standard formalization is axiomatizable if the set of
non-logical axioms of the theory can be put in one to one correspondence
with a countably infinite or countably finite set}; if the set is
countably infinite then the non-logical axioms form a set of all finite
strings of binary digits.

\hypertarget{theorem-10}{%
\subsubsection{Theorem}\label{theorem-10}}

A theory in standard formalization is finitely axiomatizable if the set
of non-logical axioms of the theory can be put into one to one
correspondence with a finite string of binary digits.

\hypertarget{theorem-11}{%
\subsubsection{Theorem}\label{theorem-11}}

\st{A theory in standard formalization is non-axiomatizable if the set
of non-logical axioms can neither be put into one to one correspondence
with a finite string of binary digits nor with a set of all finite
strings of binary digits.}

Equivalently, a theory in standard formalization is non-axiomatizable if
axiomatizing the theory would entail an arbitrary contradiction.

\hypertarget{conjecture-2}{%
\subsubsection{Conjecture}\label{conjecture-2}}

There exists theories in standard formalization that are relatively
non-axiomatizable in paraconsistent semantics.

\hypertarget{definition-5}{%
\paragraph{Definition}\label{definition-5}}

A paraconsistent theory is relatively (in classical logic)
non-axiomatizable if the contradiction that would be entailed by
axiomatization of the theory is tolerable or removable in the
paraconsistent theory or the paraconsistent metatheory.

\hypertarget{conjecture-3}{%
\subsubsection{Conjecture}\label{conjecture-3}}

There exists theories in standard formalization that are absolutely
non-axiomatizable in paraconsistent semantics.

\hypertarget{theorem-12}{%
\subsubsection{\texorpdfstring{\st{Theorem}}{Theorem}}\label{theorem-12}}

\st{A theory is finitely axiomatizable if and only if the theory is
axiomatizable and the complement of the theory is axiomatizable.}

\st{fAX(T) iff AX(T) ↔ AX(¬T)}

\hypertarget{theorem-13}{%
\subsubsection{\texorpdfstring{\st{Theorem}}{Theorem}}\label{theorem-13}}

\st{A theory is non-finitely axiomatizable if and only if the theory is
axiomatizable and the complement of the theory is non-axiomatizable.}

\st{iAX(T) iff AX(T) and ¬AX(¬T)}

\hypertarget{theorem-14}{%
\subsubsection{\texorpdfstring{\st{Theorem}}{Theorem}}\label{theorem-14}}

\st{A theory is non-axiomatizable if and only if the axiomatizability of
the theory implies that the complement of the theory is axiomatizable
and non-axiomatizable.}

\st{nAX(T) iff AX(T)→{[}AX(¬T) and ¬AX(¬T){]}}

\hypertarget{theorem-15}{%
\subsection{\texorpdfstring{\st{Theorem}}{Theorem}}\label{theorem-15}}

A paraconsistent theory is complete if and only if the complement of
that theory is paraconsistent and complete;

This is equivalent to saying that the theory and its complement are
axiomatizable which is equivalent to saying that the theory and its
complement are finitely axiomatizable.

A paraconsistent theory and its paraconsistent complement are complete
if and only if they are finitely axiomatizable.

\st{Paraconsistent(T) and Complete(T) iff Paraconsistent(¬T) and
Complete(¬T) iff AX(T) and AX(¬T) iff fAX(T).}

\end{document}}*}
