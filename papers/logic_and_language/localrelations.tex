	
	\begin{abstract}
		This thesis delves into the nuanced concepts of existential and universal locality and non-locality within physical, computational, and logical systems. We refine our definitions to distinguish between existential and universal forms of locality and non-locality. We hypothesize about the nature of these relations, focusing on existential non-locality (H1) as a counterpoint to the null hypothesis of universal locality (H0). We explore the implications of these concepts in each domain, providing illustrative examples and refining our understanding of what constitutes existentially and universally local and non-local processes.  The thesis expands on theorems linking physical locality to computational and logical locality, and further elaborates on experimental designs to test these hypotheses. We argue that while existential locality is evident in classical approximations, the universe fundamentally exhibits existential non-locality, necessitating a shift beyond universally local frameworks.
	\end{abstract}
	
	\section{Introduction}
	
	Building upon the foundational distinction between local and non-local relations, this thesis now focuses on the existential and universal quantifiers applied to locality and non-locality across physics, computation, and logic.  While our previous work established definitions and theorems pointing towards the refutation of universal locality, this iteration sharpens our focus on understanding the specific nature of existential non-locality and contrasting it with the hypothetical construct of universal locality.  We aim to clarify what it means for a process to be existentially non-local versus universally non-local in each domain, and to further explore the implications for our understanding of physical reality, computation, and logical inference.  The central question guiding this refined investigation is: Can we have existential locality and existential non-locality without either universal locality or universal non-locality?  And if universal locality is untenable, what does this imply about the nature of processes in these domains?

	\section{Refined Definitions: Existential vs. Universal Locality and Non-Locality}

	We refine our definitions to explicitly distinguish between existential and universal forms of locality and non-locality in physical, computational, and logical domains.

	\subsection{Physical Domain}

	\begin{definition}[Existential Physical Locality]
		\textbf{Existential physical locality} asserts that there exist at least some physical relations that are local, meaning influences are mediated at or below the speed of light, determined by properties within spacetime regions and their intersection.  Classical physics provides numerous examples of existentially local relations.
	\end{definition}

	\begin{definition}[Existential Physical Non-Locality]
		\textbf{Existential physical non-locality} asserts that there exists at least one physical relation that is non-local, violating the conditions of physical locality by exhibiting instantaneous or space-like separated correlations not explainable by local interactions. Quantum entanglement exemplifies existential physical non-locality.
	\end{definition}

	\begin{definition}[Universal Physical Locality]
		\textbf{Universal physical locality} (Hypothetical) posits that \textbf{all} physical relations in the universe are local. This would imply that no fundamental physical process exhibits non-local correlations, and the universe operates entirely according to local interactions. This is the null hypothesis (H0) we aim to refute.
	\end{definition}

	\begin{definition}[Universal Physical Non-Locality]
		\textbf{Universal physical non-locality} (Hypothetical) would assert that \textbf{all} fundamental physical relations are non-local. This extreme view would imply that locality is never fundamentally applicable in physics, and all correlations, even seemingly local ones, ultimately arise from non-local underpinnings.  This is a less explored and more speculative concept.
	\end{definition}
	
	\subsection{Computational Domain}
	
	\begin{definition}[Existential Computational Locality]
		\textbf{Existential computational locality} describes computational processes that exhibit local causation (SG-LC) for at least some computational relations.  Traditional algorithms and classical computation largely operate within the bounds of existential computational locality.
	\end{definition}
	
	\begin{definition}[Existential Computational Non-Locality]
		\textbf{Existential computational non-locality} describes computational processes where at least one computational relation violates local causation. Quantum computation, particularly algorithms leveraging entanglement, can be considered existentially computationally non-local.
	\end{definition}
	
	\begin{definition}[Universal Computational Locality]
		\textbf{Universal computational locality} (Hypothetical) would mean that \textbf{all} computational relations in any possible computational system must adhere to local causation. This would restrict computation to step-by-step, locally determined processes, excluding any form of instantaneous or non-local computational influence.
	\end{definition}
	
	\begin{definition}[Universal Computational Non-Locality]
		\textbf{Universal computational non-locality} (Hypothetical) would imply that \textbf{all} computational relations are fundamentally non-local.  This could envision a form of computation where every step inherently involves instantaneous dependencies across the entire computational system, potentially exceeding the bounds of Turing-machine-like models.
	\end{definition}
	
	\subsection{Logical Domain}

	\begin{definition}[Existential Logical Locality]
		\textbf{Existential logical locality} describes logical relations that are local for at least some logical contexts. Classical logic, with its context-independent truth values for many relations, embodies existential logical locality.
	\end{definition}
	
	\begin{definition}[Existential Logical Non-Locality]
		\textbf{Existential logical non-locality} describes logical relations whose truth values are context-dependent in a way that implies a dependence on broader, non-local logical structures for at least some logical relations. Paraconsistent logics, which handle contradictions that might arise from non-local or inconsistent information sources, can be seen as exploring existential logical non-locality.
	\end{definition}

	\begin{definition}[Universal Logical Locality]
		\textbf{Universal logical locality} (Hypothetical) would assert that \textbf{all} logical relations must be logically local. This would mean that context-dependence in logic is always reducible to local factors, and there are no truly non-local dependencies in logical truth.
	\end{definition}

	\begin{definition}[Universal Logical Non-Locality]
		\textbf{Universal logical non-locality} (Hypothetical) would imply that \textbf{all} logical relations are fundamentally logically non-local. In such a logical system, truth values would always be globally context-dependent, potentially making classical logical inference inapplicable.
	\end{definition}

	\section{Refined Hypotheses}

	Our hypotheses are now refined to focus on the existential and universal nature of locality and non-locality.
	
	\begin{hypothesis}[H0: Universal Locality (Null Hypothesis)]
		All fundamental physical, computational, and logical relations are universally local.  This implies universal physical locality, universal computational locality, and universal logical locality as defined above.
	\end{hypothesis}
	
	\begin{hypothesis}[H1: Existential Non-Locality (Alternative Hypothesis)]
		There exists existential non-locality in at least one of the physical, computational, or logical domains.  This means at least one of existential physical non-locality, existential computational non-locality, or existential logical non-locality is true.  This hypothesis does not preclude the existence of existential locality alongside existential non-locality.
	\end{hypothesis}
	
\section{Theorems}

\begin{theorem}[Theorem 1: Universal Physical Locality Implies Universal Computational and Logical Locality]
	If all fundamental physical relations are universally local, then all computational relations are universally computationally local (SG-LC holds universally), and all logical relations are universally logically local.
\end{theorem}
\begin{proof}
	\textbf{Proof of Theorem 1:} Assume universal physical locality. Consider any computational process. Computation is physically implemented. If all physical relations are local, then the physical processes underlying computation must also be governed by local interactions. By Definition 1 (Physical Locality), physical locality restricts influences to propagate at or below the speed of light via local interactions. If computation is physically realized and physical interactions are universally local, then computational state changes must be determined by local physical interactions, satisfying Computational Locality (SG-LC) as defined in Definition 2 (Computational Locality). Therefore, universal physical locality implies universal computational locality.
	
	Now consider logical relations. Logical systems are also physically instantiated when used by embodied reasoners or implemented in physical devices. If all physical relations are universally local, then the physical substrates implementing logical operations and relations must operate locally. By Definition 3 (Logical Locality), logical locality requires that the truth value of logical relations be determined by local properties. If the underlying physical reality is universally local, then any logical relation instantiated in this reality must also be logically local. Any apparent non-locality in logical inference would have to be reducible to a sequence of local physical and computational steps, thus adhering to Logical Locality as defined in Definition 3. Therefore, universal physical locality implies universal logical locality.
	
	Combining these implications, universal physical locality implies both universal computational locality and universal logical locality.
\end{proof}

\begin{theorem}[Theorem 2: Existential Physical Non-Locality Implies Existential Computational or Logical Non-Locality]
	If there exists at least one physical relation that is non-local, then there exists either at least one computational relation that is computationally non-local, or at least one logical relation that is logically non-local, or both.
\end{theorem}
\begin{proof}
	\textbf{Proof of Theorem 2:} Assume existential physical non-locality. This means there exists at least one physical relation between systems $A$ and $B$ that violates Physical Locality (Definition 1). Consider the implications for computation and logic, which are physically realized.
	
	\textbf{Case 1: Computational Non-Locality.} If physical non-locality exists, it can potentially be harnessed or reflected in computational processes. If a computational system were designed to exploit or simulate a non-local physical phenomenon, then the computational relations within such a system could inherit or mirror this non-locality. For instance, if quantum entanglement (a physically non-local phenomenon) is used to perform computation (as in quantum computing), then the computational relations within a quantum computer that are based on entanglement would exhibit computational non-locality, violating SG-LC (Definition 2). Thus, existential physical non-locality can lead to existential computational non-locality.
	
	\textbf{Case 2: Logical Non-Locality.}  If physical non-locality exists, it may necessitate the development of logical systems capable of reasoning about and describing non-local phenomena. Classical logic, presupposing locality, may be insufficient to capture the features of a non-local reality. To adequately model and reason about physically non-local systems, we might require logical systems that themselves are logically non-local, meaning that the truth values of logical relations become context-dependent in ways that reflect the underlying physical non-locality (Definition 3). For example, in reasoning about entangled quantum systems, logical relations describing correlations may need to be non-local to accurately represent the physical situation. Thus, existential physical non-locality can necessitate existential logical non-locality.
	
	Combining both cases, if there exists at least one physical non-local relation, then to either computationally simulate or logically describe this non-locality, we must admit either existential computational non-locality or existential logical non-locality, or both. Therefore, existential physical non-locality implies existential computational or logical non-locality.
\end{proof}
	
\section{Experiments}

\subsection{Experiment E0: Attempt to Refute Universal Locality (Null Hypothesis Test) - Classical System Locality Test}

\textbf{Objective:} To experimentally investigate whether a classical physical system, designed to embody local interactions, adheres to universal locality.  While refuting universal locality is the overarching goal, this experiment aims to test the limits of locality in a system explicitly constructed to be local, seeking potential unexpected non-local behaviors that would contradict H0 even in a seemingly classical setting.  Failure to find non-locality in this setup would not confirm H0 universally, but would strengthen the case for locality in classical approximations and highlight the domain of applicability for classical physics.

\textbf{System:} Construct a system of coupled oscillators, physically separated to ensure no direct immediate interaction faster than allowed by local signal propagation.  These oscillators could be mechanical pendulums, or electronic oscillators, coupled via a controlled medium (e.g., sound waves in a medium, or electromagnetic waves in a waveguide) designed to limit the speed of interaction.

\textbf{Procedure:}
\begin{enumerate}
	\item \textbf{Setup:} Establish two sets of oscillators (Set A and Set B) spatially separated. Couple them indirectly through a medium that enforces a speed limit on interactions (e.g., controlled acoustic or electromagnetic coupling with designed delay).
	\item \textbf{Initialization:} Initialize oscillators in Set A and Set B to specific initial states (e.g., specific amplitudes and phases of oscillation).
	\item \textbf{Perturbation:} Perturb one oscillator in Set A.
	\item \textbf{Measurement:} Precisely measure the response in oscillators in Set B over time. Measure the time delay between the perturbation in Set A and any detectable response in Set B.  Also, measure the correlation of states between Set A and Set B over time.
	\item \textbf{Vary Parameters:} Vary the distance between Set A and Set B, and the properties of the coupling medium to test different regimes of interaction.
\end{enumerate}

\textbf{Expected Outcome if Universal Locality Holds (H0 is True):} If universal locality holds, and our system is indeed locally causal, then:
\begin{itemize}
	\item \textbf{Time Delay:} A measurable and non-zero time delay in the response of Set B to perturbations in Set A should be consistently observed, corresponding to the propagation time of the interaction through the coupling medium, respecting the designed speed limit. The delay should increase with distance.
	\item \textbf{Correlation Limits:}  Correlations between the states of oscillators in Set A and Set B should be mediated by the local coupling and should not exhibit instantaneous correlations exceeding what is allowed by the local interaction mechanism and its speed limit. Any correlations should demonstrably arise after the time delay corresponding to local propagation.
\end{itemize}

\textbf{Refutation of Universal Locality (H0 is False) would be indicated by:}
\begin{itemize}
	\item \textbf{Instantaneous Correlation:} Observation of correlations between Set A and Set B that appear to be instantaneous, or faster than what is allowed by the designed local coupling mechanism, especially if this occurs regardless of spatial separation and coupling delay.
	\item \textbf{Unexplained Correlation Patterns:}  Correlation patterns that cannot be explained by the designed local interaction mechanism, suggesting influences beyond local causation.
\end{itemize}

\textbf{Interpretation of E0:} Failure to observe non-local correlations in this carefully designed, nominally classical system would not prove universal locality, but it would reinforce the robustness of local descriptions for systems engineered to be classical. Conversely, any observation of unexplained or seemingly instantaneous correlations would offer surprising evidence against universal locality even within a system intended to be local, significantly refuting H0 and strengthening the motivation for H1.  This experiment is designed to be sensitive to deviations from strict locality even in a classical context, pushing the boundaries of the null hypothesis.

\subsection{Experiment E1: Test for Existential Non-Locality (Alternative Hypothesis Test) - Quantum Entanglement Bell Test}

\textbf{Objective:} To experimentally test for existential non-locality by performing a Bell test using entangled photons. This experiment is designed to directly probe for non-local correlations as predicted by quantum mechanics and to refute local realism, thus providing evidence for existential non-locality and supporting H1.

\textbf{System:} Utilize a source of entangled photon pairs, a polarization analyzer for each photon, and detectors to measure photon polarizations in coincidence. This is a standard Bell test experimental setup based on Aspect-type experiments.

\textbf{Procedure:}
\begin{enumerate}
	\item \textbf{Entangled Photon Generation:} Generate pairs of entangled photons, for example, via spontaneous parametric down-conversion (SPDC). Ensure the photons are polarization-entangled.
	\item \textbf{Spatial Separation:} Separate the entangled photon pairs and direct them to spatially separated polarization analyzers and detectors. Ensure sufficient spatial separation such that any local communication between measurement stations within the measurement time is excluded (space-like separation).
	\item \textbf{Polarization Measurement Settings:} Randomly and independently choose polarization measurement settings for each photon analyzer (e.g., angles $\alpha$ and $\beta$).
	\item \textbf{Coincidence Detection:} Measure the polarization of each photon and record coincidence counts for different combinations of measurement settings $(\alpha, \beta)$.
	\item \textbf{Correlation Calculation:} Calculate the Bell inequality parameter (e.g., CHSH parameter $S$) from the coincidence counts for different measurement settings.
\end{enumerate}

\textbf{Expected Outcome if Local Realism and Universal Locality Hold (Refuting H1 and Supporting H0):} If local realism and universal locality were to hold, then the Bell inequality parameter $S$ would be constrained to satisfy $|S| \leq 2$. This is the bound derived by Bell under the assumptions of local realism.

\textbf{Evidence for Existential Non-Locality (Supporting H1 and Refuting H0):}  Quantum mechanics predicts, and experiments consistently show, that for certain measurement settings, the Bell inequality is violated, i.e., $|S| > 2$.  Specifically, quantum mechanics predicts a maximal violation up to $S = 2\sqrt{2}$ for optimal settings. Observation of a statistically significant violation of the Bell inequality (e.g., $S > 2 + \epsilon$, where $\epsilon$ is beyond experimental uncertainty) would constitute strong evidence against local realism and, by extension, against universal locality.

\textbf{Interpretation of E1:}  A statistically significant violation of the Bell inequality in Experiment E1 would empirically demonstrate that the correlations between entangled photons are non-local and cannot be explained by any local realistic theory. Given the assumptions of our theorem framework, and if we maintain logical determinism and reasoner independence as reasonable assumptions for interpreting quantum mechanics, then the violation of Bell inequalities directly implies the existence of physical non-locality. This experimental result would therefore provide strong empirical support for Hypothesis H1 (Existential Non-Locality) and serve as a direct refutation of Hypothesis H0 (Universal Locality), at least in the physical domain, and by Theorem 2, imply non-locality in computational or logical domains as well. This experiment directly tests for and is expected to confirm existential physical non-locality, thereby refuting the null hypothesis of universal locality.
	
	\section{Discussion: Existential Locality, Existential Non-Locality, and the Refutation of Universal Locality}
	
	The refined definitions and hypotheses allow us to more precisely discuss the implications of locality and non-locality.
	
	\subsection{Existential Locality: The Classical Approximation}
	
	Existential locality is readily observed in classical physics, computation, and logic.  Classical mechanics, electromagnetism (in its original formulation), and general relativity are fundamentally local theories.  Classical computation, based on Turing machines and similar models, operates through step-by-step local causation.  Classical logic, such as propositional and first-order logic, often assumes context-independent truth values for many relations. These frameworks demonstrate the effectiveness and applicability of \textbf{existential locality} in describing a vast range of phenomena.  For instance:
	
	\begin{itemize}
		\item \textbf{Physical:}  The propagation of sound waves, the motion of macroscopic objects under gravity, and electromagnetic waves in classical electrodynamics exemplify existentially local physical relations.
		\item \textbf{Computational:}  Sorting algorithms, search algorithms, and rule-based expert systems are examples of existentially computationally local processes.
		\item \textbf{Logical:}  The logical relation of implication in propositional logic, where $P \implies Q$ is determined solely by the truth values of $P$ and $Q$, is an example of existential logical locality.
	\end{itemize}
	However, the existence of these local descriptions does not necessitate \textbf{universal locality}.
	
	\subsection{Existential Non-Locality: Quantum Reality and Beyond}
	
	Experiment E1, the Bell test, is designed to demonstrate \textbf{existential physical non-locality} through the violation of Bell inequalities.  Quantum entanglement provides a clear example of a physical relation that is non-local, challenging the classical assumption of universal locality.  Furthermore, we can identify potential instances of existential non-locality in computation and logic:
	
	\begin{itemize}
		\item \textbf{Physical:} Quantum entanglement, as experimentally verified, is the prime example of existential physical non-locality.
		\item \textbf{Computational:} Quantum algorithms that exploit entanglement, such as quantum teleportation or certain quantum search algorithms, demonstrate \textbf{existential computational non-locality}. The computational relations in these algorithms are not reducible to sequences of local causal steps in the classical sense.
		\item \textbf{Logical:} Paraconsistent logics, designed to handle contradictions without logical explosion, can be seen as exploring \textbf{existential logical non-locality}.  In situations where information sources are non-locally correlated or inconsistent (as might arise from quantum measurements or distributed systems), logical relations may need to be context-dependent in a non-local way to manage these inconsistencies. For example, consider a logical system reasoning about measurements on entangled particles; the logical relations describing correlations might need to reflect the non-local nature of entanglement.
	\end{itemize}
	
	\subsection{Hypothetical Universal Locality and Non-Locality}
	
	\textbf{Universal locality}, if true, would simplify our understanding of the universe, implying a clockwork-like, deterministic, and locally interacting reality across all domains. However, experimental evidence, particularly from Bell tests, challenges this view in the physical domain.  If universal physical locality were true, by Theorem 1, we would also expect universal computational and logical locality, severely restricting the scope of computation and logic beyond classical paradigms.
	
	\textbf{Universal non-locality}, while more speculative, presents a radical alternative.  A universe of universal physical non-locality would be profoundly interconnected, where every event could instantaneously influence every other event, challenging our notions of causality and spacetime.  In computation, universal computational non-locality might imply computational processes that are fundamentally holistic and non-sequential.  In logic, universal logical non-locality could lead to a completely context-dependent logic where truth is always relative to a global logical state. While intriguing, there is currently no empirical evidence suggesting universal non-locality in any domain, and it poses significant conceptual challenges.
	
	\subsection{Refutation of Universal Locality and the Necessity of Existential Non-Locality}
	
	The weight of experimental evidence from quantum mechanics, particularly Bell inequality violations, strongly suggests that \textbf{universal physical locality (H0) is false}. Experiment E1 is designed to provide further empirical support for this refutation.  Given Theorem 2, the existence of physical non-locality implies the necessity of considering non-locality in computation or logic as well.  Therefore, while \textbf{existential locality} provides a useful and often accurate approximation, especially in classical domains, it is not universally applicable.  The universe, at its fundamental level, appears to exhibit \textbf{existential non-locality}, demanding that our physical, computational, and logical frameworks must be expanded to accommodate and understand non-local relations.
	
	\section{Conclusion}
	
	This refined thesis has explored the concepts of existential and universal locality and non-locality across physical, computational, and logical domains. By distinguishing between existential and universal forms, we have clarified the scope and implications of locality and non-locality. We have argued that while existential locality is evident and useful in classical approximations, \textbf{universal locality (H0) is likely false}, particularly in light of quantum mechanics and the expected outcomes of Bell test experiments like E1.  The confirmation of existential physical non-locality necessitates the acceptance and exploration of \textbf{existential non-locality} in computational and logical domains as well.  Moving forward, the development of non-classical physical theories, non-local computational paradigms (like quantum computation), and context-sensitive logical systems (like paraconsistent logics) is crucial for a more complete and accurate understanding of reality, computation, and inference in a universe that transcends the limitations of universal locality and embraces the richness of existential non-locality.
	
