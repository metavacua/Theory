\section{Introduction}

The distinction between local and non-local relations is fundamental across physics, computation, and logic. Classical frameworks often presuppose locality, asserting that influences and dependencies are mediated locally in space and time, or through discrete computational steps, and logical inferences are context-independent. However, quantum mechanics and advanced computational paradigms challenge this assumption, suggesting the existence of non-local correlations and dependencies. This thesis rigorously examines the concept of locality and non-locality across these domains, formulating testable hypotheses and designing experiments to probe the validity of universal locality. We aim to constructively integrate physical, computational, and logical perspectives to provide a comprehensive understanding of local and non-local relations, ultimately arguing for the refutation of universal locality in favor of the recognition of existential non-locality.

\section{Definitions}

\begin{definition}[Physical Locality]
	A physical relation between two physical systems $A$ and $B$ is \textbf{local} if any influence or dependency between $A$ and $B$ is mediated by interactions propagating at or below the speed of light within spacetime, and is solely determined by properties and states within their respective spacetime regions and their intersection. A physical relation is \textbf{non-local} if it violates this condition, exhibiting instantaneous or space-like separated correlations that cannot be explained by local interactions.
\end{definition}

\begin{definition}[Computational Locality (Sieg-Gandy Local Causation - SG-LC)]
	A computational process exhibits \textbf{local causation} if state changes are governed by local interactions, where each computational step depends only on the immediately preceding state and local inputs, with any propagation of influence being bounded in each discrete step. A computational relation is \textbf{computationally local} if it adheres to local causation. A computational relation is \textbf{computationally non-local} if it violates local causation, implying that a computational step can be instantaneously influenced by distant or non-adjacent parts of the computational system in a manner not reducible to a sequence of local steps.
\end{definition}

\begin{definition}[Logical Locality]
	A logical relation $R$ between logical objects $A$ and $B$ is \textbf{local} if the truth value of $R(A, B)$ is determined solely by the intrinsic properties of $A$ and $B$ and their local logical context, without dependence on non-local logical contexts or relations. A logical relation is \textbf{logically non-local} if its truth value is context-dependent in a way that cannot be reduced to the local properties of $A$ and $B$ and their immediate logical environment, implying a dependence on broader, non-local logical structures or relations.
\end{definition}

\begin{definition}[Universal Relation]
	A relation (physical, computational, or logical) is \textbf{universal} within a domain if it holds for all possible instances or pairs of objects within that domain. For example, universal physical locality would mean every physical relation is local.
\end{definition}

\begin{definition}[Existential Relation]
	A relation (physical, computational, or logical) is \textbf{existential} within a domain if there exists at least one instance or pair of objects within that domain for which the relation holds. For example, existential physical non-locality would mean there exists at least one physical relation that is non-local.
\end{definition}

\section{Hypotheses}

\begin{hypothesis}[H0: Universal Locality (Null Hypothesis)]
	All fundamental physical, computational, and logical relations are universally local. That is, \textbf{every} physical relation is local, \textbf{every} computational relation is computationally local, and \textbf{every} logical relation is logically local.
\end{hypothesis}

\begin{hypothesis}[H1: Existential Non-Locality (Alternative Hypothesis)]
	There exists at least one fundamental relation that is non-local in at least one of the physical, computational, or logical domains. That is, there exists at least one physical relation that is non-local, or at least one computational relation that is computationally non-local, or at least one logical relation that is logically non-local.
\end{hypothesis}

\section{Theorems}

\begin{theorem}[Theorem 1: Universal Physical Locality Implies Universal Computational and Logical Locality]
	If all fundamental physical relations are universally local, then all computational relations are universally computationally local (SG-LC holds universally), and all logical relations are universally logically local.
\end{theorem}
\begin{proof}
	\textbf{Proof of Theorem 1:} Assume universal physical locality. Consider any computational process. Computation is physically implemented. If all physical relations are local, then the physical processes underlying computation must also be governed by local interactions. By Definition 1 (Physical Locality), physical locality restricts influences to propagate at or below the speed of light via local interactions. If computation is physically realized and physical interactions are universally local, then computational state changes must be determined by local physical interactions, satisfying Computational Locality (SG-LC) as defined in Definition 2 (Computational Locality). Therefore, universal physical locality implies universal computational locality.
	
	Now consider logical relations. Logical systems are also physically instantiated when used by embodied reasoners or implemented in physical devices. If all physical relations are universally local, then the physical substrates implementing logical operations and relations must operate locally. By Definition 3 (Logical Locality), logical locality requires that the truth value of logical relations be determined by local properties. If the underlying physical reality is universally local, then any logical relation instantiated in this reality must also be logically local. Any apparent non-locality in logical inference would have to be reducible to a sequence of local physical and computational steps, thus adhering to Logical Locality as defined in Definition 3. Therefore, universal physical locality implies universal logical locality.
	
	Combining these implications, universal physical locality implies both universal computational locality and universal logical locality.
\end{proof}

\begin{theorem}[Theorem 2: Existential Physical Non-Locality Implies Existential Computational or Logical Non-Locality]
	If there exists at least one physical relation that is non-local, then there exists either at least one computational relation that is computationally non-local, or at least one logical relation that is logically non-local, or both.
\end{theorem}
\begin{proof}
	\textbf{Proof of Theorem 2:} Assume existential physical non-locality. This means there exists at least one physical relation between systems $A$ and $B$ that violates Physical Locality (Definition 1). Consider the implications for computation and logic, which are physically realized.
	
	\textbf{Case 1: Computational Non-Locality.} If physical non-locality exists, it can potentially be harnessed or reflected in computational processes. If a computational system were designed to exploit or simulate a non-local physical phenomenon, then the computational relations within such a system could inherit or mirror this non-locality. For instance, if quantum entanglement (a physically non-local phenomenon) is used to perform computation (as in quantum computing), then the computational relations within a quantum computer that are based on entanglement would exhibit computational non-locality, violating SG-LC (Definition 2). Thus, existential physical non-locality can lead to existential computational non-locality.
	
	\textbf{Case 2: Logical Non-Locality.}  If physical non-locality exists, it may necessitate the development of logical systems capable of reasoning about and describing non-local phenomena. Classical logic, presupposing locality, may be insufficient to capture the features of a non-local reality. To adequately model and reason about physically non-local systems, we might require logical systems that themselves are logically non-local, meaning that the truth values of logical relations become context-dependent in ways that reflect the underlying physical non-locality (Definition 3). For example, in reasoning about entangled quantum systems, logical relations describing correlations may need to be non-local to accurately represent the physical situation. Thus, existential physical non-locality can necessitate existential logical non-locality.
	
	Combining both cases, if there exists at least one physical non-local relation, then to either computationally simulate or logically describe this non-locality, we must admit either existential computational non-locality or existential logical non-locality, or both. Therefore, existential physical non-locality implies existential computational or logical non-locality.
\end{proof}

\section{Experiments}

\subsection{Experiment E0: Attempt to Refute Universal Locality (Null Hypothesis Test) - Classical System Locality Test}

\textbf{Objective:} To experimentally investigate whether a classical physical system, designed to embody local interactions, adheres to universal locality.  While refuting universal locality is the overarching goal, this experiment aims to test the limits of locality in a system explicitly constructed to be local, seeking potential unexpected non-local behaviors that would contradict H0 even in a seemingly classical setting.  Failure to find non-locality in this setup would not confirm H0 universally, but would strengthen the case for locality in classical approximations and highlight the domain of applicability for classical physics.

\textbf{System:} Construct a system of coupled oscillators, physically separated to ensure no direct immediate interaction faster than allowed by local signal propagation.  These oscillators could be mechanical pendulums, or electronic oscillators, coupled via a controlled medium (e.g., sound waves in a medium, or electromagnetic waves in a waveguide) designed to limit the speed of interaction.

\textbf{Procedure:}
\begin{enumerate}
	\item \textbf{Setup:} Establish two sets of oscillators (Set A and Set B) spatially separated. Couple them indirectly through a medium that enforces a speed limit on interactions (e.g., controlled acoustic or electromagnetic coupling with designed delay).
	\item \textbf{Initialization:} Initialize oscillators in Set A and Set B to specific initial states (e.g., specific amplitudes and phases of oscillation).
	\item \textbf{Perturbation:} Perturb one oscillator in Set A.
	\item \textbf{Measurement:} Precisely measure the response in oscillators in Set B over time. Measure the time delay between the perturbation in Set A and any detectable response in Set B.  Also, measure the correlation of states between Set A and Set B over time.
	\item \textbf{Vary Parameters:} Vary the distance between Set A and Set B, and the properties of the coupling medium to test different regimes of interaction.
\end{enumerate}

\textbf{Expected Outcome if Universal Locality Holds (H0 is True):} If universal locality holds, and our system is indeed locally causal, then:
\begin{itemize}
	\item \textbf{Time Delay:} A measurable and non-zero time delay in the response of Set B to perturbations in Set A should be consistently observed, corresponding to the propagation time of the interaction through the coupling medium, respecting the designed speed limit. The delay should increase with distance.
	\item \textbf{Correlation Limits:}  Correlations between the states of oscillators in Set A and Set B should be mediated by the local coupling and should not exhibit instantaneous correlations exceeding what is allowed by the local interaction mechanism and its speed limit. Any correlations should demonstrably arise after the time delay corresponding to local propagation.
\end{itemize}

\textbf{Refutation of Universal Locality (H0 is False) would be indicated by:}
\begin{itemize}
	\item \textbf{Instantaneous Correlation:} Observation of correlations between Set A and Set B that appear to be instantaneous, or faster than what is allowed by the designed local coupling mechanism, especially if this occurs regardless of spatial separation and coupling delay.
	\item \textbf{Unexplained Correlation Patterns:}  Correlation patterns that cannot be explained by the designed local interaction mechanism, suggesting influences beyond local causation.
\end{itemize}

\textbf{Interpretation of E0:} Failure to observe non-local correlations in this carefully designed, nominally classical system would not prove universal locality, but it would reinforce the robustness of local descriptions for systems engineered to be classical. Conversely, any observation of unexplained or seemingly instantaneous correlations would offer surprising evidence against universal locality even within a system intended to be local, significantly refuting H0 and strengthening the motivation for H1.  This experiment is designed to be sensitive to deviations from strict locality even in a classical context, pushing the boundaries of the null hypothesis.

\subsection{Experiment E1: Test for Existential Non-Locality (Alternative Hypothesis Test) - Quantum Entanglement Bell Test}

\textbf{Objective:} To experimentally test for existential non-locality by performing a Bell test using entangled photons. This experiment is designed to directly probe for non-local correlations as predicted by quantum mechanics and to refute local realism, thus providing evidence for existential non-locality and supporting H1.

\textbf{System:} Utilize a source of entangled photon pairs, a polarization analyzer for each photon, and detectors to measure photon polarizations in coincidence. This is a standard Bell test experimental setup based on Aspect-type experiments.

\textbf{Procedure:}
\begin{enumerate}
	\item \textbf{Entangled Photon Generation:} Generate pairs of entangled photons, for example, via spontaneous parametric down-conversion (SPDC). Ensure the photons are polarization-entangled.
	\item \textbf{Spatial Separation:} Separate the entangled photon pairs and direct them to spatially separated polarization analyzers and detectors. Ensure sufficient spatial separation such that any local communication between measurement stations within the measurement time is excluded (space-like separation).
	\item \textbf{Polarization Measurement Settings:} Randomly and independently choose polarization measurement settings for each photon analyzer (e.g., angles $\alpha$ and $\beta$).
	\item \textbf{Coincidence Detection:} Measure the polarization of each photon and record coincidence counts for different combinations of measurement settings $(\alpha, \beta)$.
	\item \textbf{Correlation Calculation:} Calculate the Bell inequality parameter (e.g., CHSH parameter $S$) from the coincidence counts for different measurement settings.
\end{enumerate}

\textbf{Expected Outcome if Local Realism and Universal Locality Hold (Refuting H1 and Supporting H0):} If local realism and universal locality were to hold, then the Bell inequality parameter $S$ would be constrained to satisfy $|S| \leq 2$. This is the bound derived by Bell under the assumptions of local realism.

\textbf{Evidence for Existential Non-Locality (Supporting H1 and Refuting H0):}  Quantum mechanics predicts, and experiments consistently show, that for certain measurement settings, the Bell inequality is violated, i.e., $|S| > 2$.  Specifically, quantum mechanics predicts a maximal violation up to $S = 2\sqrt{2}$ for optimal settings. Observation of a statistically significant violation of the Bell inequality (e.g., $S > 2 + \epsilon$, where $\epsilon$ is beyond experimental uncertainty) would constitute strong evidence against local realism and, by extension, against universal locality.

\textbf{Interpretation of E1:}  A statistically significant violation of the Bell inequality in Experiment E1 would empirically demonstrate that the correlations between entangled photons are non-local and cannot be explained by any local realistic theory. Given the assumptions of our theorem framework, and if we maintain logical determinism and reasoner independence as reasonable assumptions for interpreting quantum mechanics, then the violation of Bell inequalities directly implies the existence of physical non-locality. This experimental result would therefore provide strong empirical support for Hypothesis H1 (Existential Non-Locality) and serve as a direct refutation of Hypothesis H0 (Universal Locality), at least in the physical domain, and by Theorem 2, imply non-locality in computational or logical domains as well. This experiment directly tests for and is expected to confirm existential physical non-locality, thereby refuting the null hypothesis of universal locality.

\section{Discussion}

The experiments proposed are designed to directly confront the hypothesis of universal locality. Experiment E0 attempts to find the limits of locality even in a system designed to be locally causal. While a null result in E0 would not prove universal locality, any positive result indicating non-local behavior would be highly significant against H0. Experiment E1, the Bell test, is designed to directly test for and is expected to confirm existential physical non-locality, based on established quantum mechanical predictions and experimental validations. A successful Bell test, violating Bell inequalities, would provide strong empirical evidence against universal locality (H0) and in favor of existential non-locality (H1).

Theorem 1 and Theorem 2 provide a rigorous logical framework linking physical locality to computational and logical locality. Theorem 1 shows that universal physical locality would necessitate universal locality in computation and logic, reinforcing the classical expectation of a universally local world. Theorem 2, conversely, demonstrates that even a single instance of physical non-locality implies that non-locality must exist in either the computational or logical domain, or both, opening the door for non-classical computation and logic.

The combined theoretical and experimental approach of this thesis argues against universal locality. While classical physics and computation provide powerful approximations in many domains, the evidence from quantum mechanics, particularly Bell test experiments, strongly suggests that nature is fundamentally non-local at some level. This non-locality, as argued, has implications not only for physics but also for the foundations of computation and logic.  The exploration of non-local relations in computation (e.g., quantum computation) and logic (e.g., non-reflexive, non-transitive logics) becomes not just a theoretical curiosity but a necessity for a comprehensive understanding of reality and for developing computational and logical systems that can effectively model and harness non-local phenomena.

\section{Conclusion}

This thesis rigorously examined the concept of local and non-local relations across physical, computational, and logical domains. Through definitions, theorems, and proposed experiments, we have argued against the hypothesis of universal locality (H0) and in favor of existential non-locality (H1). Theorem 1 established that universal physical locality would imply universal computational and logical locality, while Theorem 2 showed that existential physical non-locality necessitates non-locality in computation or logic. Experiment E0 probes the limits of locality in a classical system, while Experiment E1, a Bell test, is designed to empirically demonstrate existential physical non-locality.

The anticipated outcome of Experiment E1, a violation of Bell inequalities, would serve as empirical refutation of universal locality and strong support for existential non-locality. This conclusion necessitates a paradigm shift in our understanding of physical, computational, and logical systems, moving beyond the classical assumption of universal locality to embrace non-classical frameworks that can accommodate and leverage non-local relations. The exploration and development of non-classical computation and logic are essential for advancing our scientific and technological capabilities in a universe that demonstrably transcends the limitations of universal locality.