}

\hypertarget{manuscript-of-research-proposal-subclassical-graph-of-calculi}{%
\section{Manuscript of Research Proposal: Subclassical Graph of
Calculi}\label{manuscript-of-research-proposal-subclassical-graph-of-calculi}}

\hypertarget{title---the-title-should-be-as-descriptive-as-possible.}{%
\subsection{Title - The title should be as descriptive as
possible.}\label{title---the-title-should-be-as-descriptive-as-possible.}}

\hypertarget{introduction--}{%
\subsection{\texorpdfstring{ Introduction -
}{ Introduction - }}\label{introduction--}}

This section may include:

§ What is to be done and the context of the project.

§ What is being done both generally and specifically in the same or
related areas. (The reviewer should know that you know what is going on
in the area in which you are proposing.)

§ An explanation and justification for unique or innovative approaches.
(These are selling points about what makes your project special, unique
and compelling and why it should be funded.)

\hypertarget{need-statement}{%
\subsection{Need Statement}\label{need-statement}}

§ What needs to be done and why?

§ What significant needs are you trying to meet? Compared to other
projects in the same area, what sets yours apart in terms of need?

§ What services are to be delivered? Why? Use specifics from preliminary
studies, needs assessment, documentation, and data supporting your
proposal.

§ What gaps that your work can fill exist in the knowledge base of your
field?

§ Is the problem both significant and manageable? Do you have the
resources to handle the problem?

\hypertarget{goals-and-objectives}{%
\subsection{Goals and Objectives}\label{goals-and-objectives}}

§ Goals statements identify the overall purpose of the project and a
general indication of intent.

\begin{enumerate}
\def\labelenumi{\arabic{enumi}.}
\item
  \begin{quote}
  \textbf{Identification of non-classical logics:} A comprehensive list
  of non-classical logics will be compiled, including their axioms,
  rules, and properties.
  \end{quote}
\item
  \begin{quote}
  \textbf{Classification of non-classical logics:} The non-classical
  logics will be classified based on their properties, such as the types
  of negation that they use, the consistency conditions that they
  satisfy, and the types of truth values that they employ.
  \end{quote}
\item
  \begin{quote}
  \textbf{Construction of the subclassical graph of calculi:} The
  subclassical graph of calculi will be constructed by connecting the
  different non-classical logics based on their relationships.
  \end{quote}
\item
  \begin{quote}
  \textbf{Analysis of the subclassical graph of calculi:} The
  subclassical graph of calculi will be analyzed to identify common
  patterns and themes, and to suggest new areas of research.
  \end{quote}
\end{enumerate}

Develop towards a universal theory of semantic languages.

Identify the precise metalinguistic conditions that define
paraconsistent languages and the precise metalinguistic conditions that
define paracomplete languages.

Exactly define constructive languages.

Produce an automated reasoner that utilizes the union of paraconsistent
constructions and paracomplete constructions in a metasystem of proof
and refutation.

Schematically relate the subcalculi, supercalculi, and hypercalculi
graphs to universal quantum computing.

§ Objectives are action statements with measurable outcomes, to be
completed by a specified time and under specified conditions.

\hypertarget{approachmethodology}{%
\subsection{Approach/Methodology}\label{approachmethodology}}

§ How are you going to carry out your project?

The graph will represent calculi as nodes and their relationships as
edges, with edge types encoding different kinds of connections, such as
equivalence, embedding, and extension.

\textbf{Literature Review:}

\begin{itemize}
\item
  \begin{quote}
  Conduct an extensive review of relevant literature on subclassical
  calculi including intuitionistic logic, counter-intuitionistic logic,
  linear logic, the logic of qubits, Ardeshir-Vaezian's sequent calculus
  U, Sambin's Basic Logic, and T-Norm hypersequent calculi.
  \end{quote}
\item
  \begin{quote}
  Criticize the fundamental concepts, principles, and applications of
  these calculi.
  \end{quote}
\item
  \begin{quote}
  Identify and analyze existing research related to the Subclassical
  Graph of Calculi.
  \end{quote}
\end{itemize}

\textbf{2. Conceptual Framework Development:}

\begin{itemize}
\item
  \begin{quote}
  Formulate a clear and precise conceptual framework for the
  Subclassical Graph of Calculi.
  \end{quote}
\item
  \begin{quote}
  Define the key components and relationships within the graph
  structure.
  \end{quote}
\item
  \begin{quote}
  Establish a formal representation of the graph using appropriate
  mathematical and graphical notation.
  \end{quote}
\item
  \begin{quote}
  Establish a libre open source programming language capable of
  representing all subclassical calculi and their semantics.
  \end{quote}
\item
  \begin{quote}
  Establish an automated reasoning suite utilizing subclassical calculi
  in a modular way to reason about finite subgraphs of the subclassical
  graph of calculi.
  \end{quote}
\end{itemize}

\textbf{3. Metamathematical Analysis:}

\begin{itemize}
\item
  \begin{quote}
  Employ metamathematical techniques to investigate the properties and
  structure of the Subclassical Graph of Calculi.
  \end{quote}
\item
  \begin{quote}
  Analyze the graph\textquotesingle s connectivity, paraconsistency, and
  paracompleteness.
  \end{quote}
\item
  \begin{quote}
  Explore the relationships between the graph\textquotesingle s
  structure and the underlying logical systems.
  \end{quote}
\end{itemize}

\textbf{4. Metalinguistic Investigation:}

\begin{itemize}
\item
  \begin{quote}
  Utilize metalinguistic tools to examine the expressive power and
  limitations of the Subclassical Graph of Calculi.
  \end{quote}
\item
  \begin{quote}
  Analyze the graph\textquotesingle s ability to represent and formalize
  various logical concepts and relationships.
  \end{quote}
\item
  \begin{quote}
  Represent the symmetries and dualities of not only the object
  languages but the metalanguages of the various calculi.
  \end{quote}
\item
  \begin{quote}
  Develop subclassical metalanguages.
  \end{quote}
\item
  \begin{quote}
  Evaluate the graph\textquotesingle s effectiveness in capturing the
  nuances of subclassical logic.
  \end{quote}
\end{itemize}

\textbf{5. Comparative Analysis:}

\begin{itemize}
\item
  \begin{quote}
  Compare and contrast the Subclassical Graph of Calculi with
  alternative approaches to representing subclassical logic, such as
  sequent calculus and natural deduction.
  \end{quote}
\item
  \begin{quote}
  Identify the strengths and weaknesses of each approach in terms of
  expressiveness, conciseness, and computational efficiency.
  \end{quote}
\item
  \begin{quote}
  Discuss the implications of the graph-based approach for understanding
  and reasoning within and without subclassical logic.
  \end{quote}
\end{itemize}

§ What specific activities do you propose to meet the goals and
objectives you have outlined, and how will those activities be carried
out?

\textbf{Specific Activities:}

To achieve the outlined goals and objectives, the following specific
activities will be undertaken:

\begin{itemize}
\item
  \begin{quote}
  Gather and organize relevant literature on subclassical calculi and
  the Subclassical Graph of Calculi.
  \end{quote}
\item
  \begin{quote}
  Construct a conceptual diagram or model to represent the Subclassical
  Graph of Calculi.
  \end{quote}
\item
  \begin{quote}
  Develop formal definitions and theorems related to the
  graph\textquotesingle s structure and properties.
  \end{quote}
\item
  \begin{quote}
  Utilize metamathematical tools, such as proof theory and model theory,
  to analyze the graph\textquotesingle s behavior.
  \end{quote}
\item
  \begin{quote}
  Employ metalinguistic techniques to assess the expressive power and
  limitations of the graph.
  \end{quote}
\item
  \begin{quote}
  Conduct comparative analysis with alternative approaches to
  representing subclassical logic.
  \end{quote}
\item
  \begin{quote}
  Programmatically represent the works in a logical programming paradigm
  programming language in a published code repository.
  \end{quote}
\end{itemize}

\hypertarget{outcomes-benefits-results}{%
\subsection{Outcomes, Benefits,
Results}\label{outcomes-benefits-results}}

The project will deliver the following outcomes.

\begin{itemize}
\item
  \begin{quote}
  A comprehensive list of non-classical logics, including their axioms,
  rules, and properties.
  \end{quote}
\item
  \begin{quote}
  A new formal framework for representing and reasoning about
  subclassical logics.
  \end{quote}
\item
  \begin{quote}
  A new classification of subclassical logics.
  \end{quote}
\item
  \begin{quote}
  A subclassical graph of calculi that visually represents the
  relationships between these logics.
  \end{quote}
\item
  \begin{quote}
  A list of common patterns and themes in the properties of
  non-classical logics.
  \end{quote}
\item
  \begin{quote}
  A list of new areas of research suggested by the analysis of the
  subclassical graph of calculi.
  \end{quote}
\end{itemize}

The project will also have the following benefits.

\begin{itemize}
\item
  \begin{quote}
  It will provide a better understanding of the logical relationships
  between different subclassical calculi.
  \end{quote}
\item
  \begin{quote}
  It will facilitate the development of new and more powerful
  subclassical calculi.
  \end{quote}
\item
  \begin{quote}
  It will enable the application of subclassical calculi to new areas.
  \end{quote}
\end{itemize}

The project will also produce the following results.

\begin{itemize}
\item
  \begin{quote}
  A new formal framework for representing and reasoning about
  subclassical logics.
  \end{quote}
\item
  \begin{quote}
  A new classification of sublanguages, superlanguages, and
  hyperlanguages as well as subcalculi, supercalculi, and hypercalculi.
  \end{quote}
\item
  \begin{quote}
  A subclassical graph of calculi that visually represents the
  relationships between these logics.
  \end{quote}
\item
  \begin{quote}
  A list of common patterns and themes in the properties of
  non-classical logics.
  \end{quote}
\item
  \begin{quote}
  A list of new areas of research suggested by the analysis of the
  subclassical graph of calculi.
  \end{quote}
\end{itemize}

§ Outcomes - What are the products of your work?

§ Impact - What are the benefits and results of your work?

§ Measurement - Can your outcomes, benefits and results be measured?

§ Products - What does the funding agency get in return for supporting
your proposal?

\hypertarget{project-directorprincipal-investigator-and-staff}{%
\subsection{Project Director/Principal Investigator and
Staff}\label{project-directorprincipal-investigator-and-staff}}

§ List the qualifications and experience of the proposed project
director/principal investigator.

§ List the qualifications and experience of key project staff.

\hypertarget{manuscript-of-research-proposal-superclassical-graph-of-calculi}{%
\section{Manuscript of Research Proposal: Superclassical Graph of
Calculi}\label{manuscript-of-research-proposal-superclassical-graph-of-calculi}}

Also known as Post's lattice.

\hypertarget{manuscript-of-research-proposal-graph-of-supercalculi-and-subcalculi}{%
\section{Manuscript of Research Proposal: Graph of Supercalculi and
Subcalculi}\label{manuscript-of-research-proposal-graph-of-supercalculi-and-subcalculi}}

\hypertarget{manuscript-of-research-proposal-hyperclassical-graph-of-calculi}{%
\section{Manuscript of Research Proposal: Hyperclassical Graph of
Calculi}\label{manuscript-of-research-proposal-hyperclassical-graph-of-calculi}}

\hypertarget{writing-research-papers}{%
\section{Writing Research Papers}\label{writing-research-papers}}

\hypertarget{the-concept-paper}{%
\subsection{The Concept Paper}\label{the-concept-paper}}

``I have a great idea for a grant proposal! How do I get started?'' A
concept paper is the suggested starting point for the development of any
successful proposal. Generally three to five pages long, it outlines the
project with enough detail to clearly demonstrate what is being
proposed. Certain funding agencies require that a concept paper be
submitted prior to submission of a full proposal. Therefore, you need to
build a strong case for your project in a concise and persuasive manner.
Suggested guidelines for constructing a concept paper are as follows:

\hypertarget{title---the-title-should-be-as-descriptive-as-possible.-1}{%
\subsection{Title - The title should be as descriptive as
possible.}\label{title---the-title-should-be-as-descriptive-as-possible.-1}}

\hypertarget{introduction---1}{%
\subsection{\texorpdfstring{ Introduction -
}{ Introduction - }}\label{introduction---1}}

This section may include:

§ What is to be done and the context of the project.

§ What is being done both generally and specifically in the same or
related areas. (The reviewer should know that you know what is going on
in the area in which you are proposing.)

§ An explanation and justification for unique or innovative approaches.
(These are selling points about what makes your project special, unique
and compelling and why it should be funded.)

\hypertarget{need-statement-1}{%
\subsection{Need Statement}\label{need-statement-1}}

§ What needs to be done and why?

§ What significant needs are you trying to meet? Compared to other
projects in the same area, what sets yours apart in terms of need?

§ What services are to be delivered? Why? Use specifics from preliminary
studies, needs assessment, documentation, and data supporting your
proposal.

§ What gaps that your work can fill exist in the knowledge base of your
field?

§ Is the problem both significant and manageable? Do you have the
resources to handle the problem?

\hypertarget{goals-and-objectives-1}{%
\subsection{Goals and Objectives}\label{goals-and-objectives-1}}

§ Goals statements identify the overall purpose of the project and a
general indication of intent.

§ Objectives are action statements with measurable outcomes, to be
completed by a specified time and under specified conditions.

\hypertarget{approachmethodology-1}{%
\subsection{Approach/Methodology}\label{approachmethodology-1}}

§ How are you going to carry out your project?

§ What specific activities do you propose to meet the goals and
objectives you have outlined, and how will those activities be carried
out?

\hypertarget{outcomes-benefits-results-1}{%
\subsection{Outcomes, Benefits,
Results}\label{outcomes-benefits-results-1}}

§ Outcomes - What are the products of your work?

§ Impact - What are the benefits and results of your work?

§ Measurement - Can your outcomes, benefits and results be measured?

§ Products - What does the funding agency get in return for supporting
your proposal?

\hypertarget{project-directorprincipal-investigator-and-staff-1}{%
\subsection{Project Director/Principal Investigator and
Staff}\label{project-directorprincipal-investigator-and-staff-1}}

§ List the qualifications and experience of the proposed project
director/principal investigator.

§ List the qualifications and experience of key project staff.

\hypertarget{organizations}{%
\section{Organizations}\label{organizations}}

\hypertarget{philosophy-organizations}{%
\subsection{Philosophy Organizations}\label{philosophy-organizations}}

Philosophy of Science

Philosophy of Physics

Philosophy of Mind, Consciousness, and Free Will

Philosophy of Life

Philosophy of Simulation

Metaphysics

Metalinguistics

Metamathematics

\hypertarget{philosophy-of-science-association}{%
\subsubsection{\texorpdfstring{Philosophy of Science Association
}{Philosophy of Science Association }}\label{philosophy-of-science-association}}

\href{mailto:office@philsci.org}{\hl{\ul{office@philsci.org}}}

\hypertarget{iuhpst-international-union-of-history-and-philosophy-of-science-and-technology}{%
\subsubsection{IUHPST: INTERNATIONAL UNION OF HISTORY AND PHILOSOPHY OF
SCIENCE AND
TECHNOLOGY}\label{iuhpst-international-union-of-history-and-philosophy-of-science-and-technology}}

\hypertarget{dlmpst-division-of-logic-methodology-and-philosophy-of-science-and-technology}{%
\paragraph{DLMPST: DIVISION OF LOGIC, METHODOLOGY AND PHILOSOPHY OF
SCIENCE AND
TECHNOLOGY}\label{dlmpst-division-of-logic-methodology-and-philosophy-of-science-and-technology}}

Hosts the Congress for Logic, Philosophy and Methodology of Science and
Technology (CLMPST) every four years.

\hypertarget{international-federation-of-philosophical-societies}{%
\subsubsection{International Federation of Philosophical
Societies}\label{international-federation-of-philosophical-societies}}

World Congress of Philosophy every five years

\hypertarget{mathematical-organizations}{%
\subsection{Mathematical
Organizations}\label{mathematical-organizations}}

Non-Classical Mathematics

Metamathematics

Mathematical Logic

Constructive Mathematics

Mathematical Physics

\hypertarget{logical-organizations}{%
\subsection{Logical Organizations}\label{logical-organizations}}

Non-Classical Logical Languages

Metalogic

Constructive Logic

\hypertarget{logica-universalis-association-lua}{%
\subsubsection{Logica Universalis Association
(LUA)}\label{logica-universalis-association-lua}}

\href{https://sites.google.com/view/calcuttalogiccircleclc/home}{\ul{Calcutta
Logic Circle}} (LUA Member)

\hypertarget{computational-organizations}{%
\subsection{Computational
Organizations}\label{computational-organizations}}

Philosophy of Artificial Intelligence, consciousness, life, synthesis,
and simulation

Theory of Computing

Theory of Quantum Computing

Non-Classical Computing

\hypertarget{association-for-computing-machinery}{%
\subsubsection{Association for Computing
Machinery}\label{association-for-computing-machinery}}

\hypertarget{ieee}{%
\subsubsection{IEEE}\label{ieee}}

\hypertarget{computer-science}{%
\paragraph{Computer Science}\label{computer-science}}

\hypertarget{mathematical-foundations-of-computing-tcmf}{%
\subparagraph{\texorpdfstring{\href{https://www.computer.org/communities/technical-committees/tcmf}{\ul{Mathematical
Foundations of Computing
(TCMF)}}}{Mathematical Foundations of Computing (TCMF)}}\label{mathematical-foundations-of-computing-tcmf}}

\hypertarget{multiple-valued-logic-tcmvl}{%
\subparagraph{\texorpdfstring{\href{https://www.computer.org/communities/technical-committees/tcmvl}{\ul{Multiple-Valued
Logic
(TCMVL)}}}{Multiple-Valued Logic (TCMVL)}}\label{multiple-valued-logic-tcmvl}}

\hypertarget{physics-organizations}{%
\subsection{Physics Organizations}\label{physics-organizations}}

Methods of physics

Methods of theoretical physics

Methods of experimental physics

Mathematical physics

Metaphysics and metalinguistics of physical theory

Measurement

Constructive Physics, Constructive Realism

Quantum Physics

Quantum Computing

Blackhole Thermodynamics

Biophysics

Physics of Mind, Consciousness, and Life

\hypertarget{international-association-of-mathematical-physics}{%
\subsubsection{International Association of Mathematical
Physics}\label{international-association-of-mathematical-physics}}

Hosts the International Congress on Mathematical Physics (ICMP) every
three years.

\hypertarget{international-union-of-pure-and-applied-physics-iupap}{%
\subsubsection{International Union of Pure and Applied Physics
(IUPAP)}\label{international-union-of-pure-and-applied-physics-iupap}}

Hosts STATPHYS every three years on a different continent.

\hypertarget{linguistics-organizations}{%
\subsection{Linguistics Organizations}\label{linguistics-organizations}}

Formal Languages

Semantic Languages

Metalinguistics

Chomsky Hierarchy

Tarskian Languages

Kripkean Languages

Legal Languages

\hypertarget{permanent-international-committee-of-linguists-picl-comituxe9-international-permanent-des-linguistes-cipl}{%
\subsubsection{\texorpdfstring{Permanent International Committee of
Linguists (PICL) / Comité International Permanent des Linguistes
(\href{http://www.ciplnet.com}{\ul{CIPL}})}{Permanent International Committee of Linguists (PICL) / Comité International Permanent des Linguistes (CIPL)}}\label{permanent-international-committee-of-linguists-picl-comituxe9-international-permanent-des-linguistes-cipl}}

Hosts the International Congress of Linguists every 5 years.

\hypertarget{west-coast-conference-on-formal-linguistics}{%
\subsubsection{West Coast Conference on Formal
Linguistics}\label{west-coast-conference-on-formal-linguistics}}

Held by Center for the Study of Language and Information, home of the
Stanford Encyclopedia of Philosophy

\hypertarget{european-language-resources-association}{%
\subsubsection{European Language Resources
Association}\label{european-language-resources-association}}

Hosts the International Conference on Language Resources and Evaluation
every even year. Focuses on natural language processing.

\hypertarget{association-for-computational-linguistics}{%
\subsubsection{Association for Computational
Linguistics}\label{association-for-computational-linguistics}}

Hosts the International Conference on Computational Linguistics and
Intelligent Text Processing every year.

\hypertarget{california-state-university-fresno}{%
\subsubsection{California State University,
Fresno}\label{california-state-university-fresno}}

Hosts Western Conference on Linguistics every year. Focuses on
theoretical and descriptive linguistics.

\hypertarget{semantic-organizations}{%
\subsection{Semantic Organizations}\label{semantic-organizations}}

\hypertarget{individuals}{%
\section{Individuals}\label{individuals}}

Paola Zizzi

\textbf{Giovanni Sambin\\
} Professor of Mathematical Logic

Dipartimento di Matematica Pura e Applicata,\\
Università di Padova\\
Via Trieste, 63 - IV piano\\
35121 Padova (Italy)

ph. +39-049-8271487, fax. +39-049-8271499

sambin@math.unipd.it

\hypertarget{peers-from-the-5th-world-congress-on-paraconsistency}{%
\subsubsection{Peers from the 5th World Congress on
Paraconsistency}\label{peers-from-the-5th-world-congress-on-paraconsistency}}

Mihir Chakraborty

\ul{calcuttalogiccircle@gmail.com}

\begin{itemize}
\item
  \begin{quote}
  \ul{mihirc4@gmail.com}
  \end{quote}
\end{itemize}

Sankha Basu \href{mailto:sankha@iiitd.ac.in}{\ul{sankha@iiitd.ac.in}} -
Negation-Free Paraconsistency Coauthor: Sayantan Roy

Hidenori Kurokawa
\href{mailto:hkurokawa@gc.cuny.edu}{\ul{hkurokawa@gc.cuny.edu}} -
Hypersequent expert

James T Martin - co-author and collaborator

\hypertarget{draft}{%
\section{Draft}\label{draft}}

\hypertarget{introduction-v1}{%
\subsection{Introduction v1}\label{introduction-v1}}

This research proposal outlines a project to construct a subclassical
graph of calculi, which is a diagram that visually represents the
relationships between different non-classical logics. Non-classical
logics are a family of logics that generalize classical logic by
relaxing some of its axioms or rules. This can lead to a variety of
interesting and counterintuitive properties, such as the ability to
represent vagueness, uncertainty, and contradictions.

The subclassical graph of calculi will be constructed by identifying and
classifying the different types of non-classical logics that exist. This
will be done by considering the different axioms and rules that are used
in these logics, as well as the properties that they satisfy. The graph
will then be used to visualize the relationships between these logics,
and to identify common patterns and themes.

This project is motivated by the need for a better understanding of
non-classical logics and their applications. Non-classical logics have a
wide range of potential applications, including in artificial
intelligence, computer science, and mathematics. However, the current
state of knowledge about non-classical logics is fragmented and
difficult to navigate. The subclassical graph of calculi will provide a
much-needed overview of the field, and will help to identify new areas
of research.

The project will be carried out in the following stages:

\begin{enumerate}
\def\labelenumi{\arabic{enumi}.}
\item
  \begin{quote}
  \textbf{Identification of non-classical logics:} A comprehensive list
  of non-classical logics will be compiled, including their axioms,
  rules, and properties.
  \end{quote}
\item
  \begin{quote}
  \textbf{Classification of non-classical logics:} The non-classical
  logics will be classified based on their properties, such as the types
  of negation that they use, the consistency conditions that they
  satisfy, and the types of truth values that they employ.
  \end{quote}
\item
  \begin{quote}
  \textbf{Construction of the subclassical graph of calculi:} The
  subclassical graph of calculi will be constructed by connecting the
  different non-classical logics based on their relationships.
  \end{quote}
\item
  \begin{quote}
  \textbf{Analysis of the subclassical graph of calculi:} The
  subclassical graph of calculi will be analyzed to identify common
  patterns and themes, and to suggest new areas of research.
  \end{quote}
\end{enumerate}

\hypertarget{introduction-v2}{%
\subsection{Introduction v2}\label{introduction-v2}}

This research proposal outlines the development of a subclassical graph
of calculi, a novel approach to formalizing and analyzing the
relationships between different logical calculi. The proposed
subclassical graph will provide a comprehensive and systematic framework
for understanding the interplay of various logical systems, enabling
researchers to identify connections, uncover hidden relationships,
generate novel calculi, and explore new avenues of research.

\textbf{Context and Background}

Logical calculi are formal systems for reasoning and deriving
conclusions from a set of axioms or assumptions or rules such as
structural or inferential rules. They play a fundamental role in
mathematics, computer science, and philosophy, providing a rigorous
foundation for reasoning, proof construction, refutation construction,
modeling, and countermodeling. Over the centuries, a vast array of
logical calculi have been developed, each tailored to specific
applications and embodying distinct logical properties.

\textbf{Current State of Research}

The study of logical calculi has witnessed significant advancements in
recent decades, fueled by the development of powerful formal methods and
automated reasoning techniques. However, the vast and diverse landscape
of logical systems poses challenges in identifying connections and
overarching principles. Existing approaches to classifying and comparing
logical calculi often rely on ad hoc criteria, leading to a fragmented
and incomplete understanding of the relationships between different
systems.

\textbf{Unique and Innovative Approach}

The proposed subclassical graph of calculi addresses these limitations
by providing a unified and rigorous framework for analyzing the
relationships between logical calculi. The subclassical graph will
represent logical calculi as nodes and the relationships between them as
edges, where the edges capture the logical implications and syntactic
similarities between different systems.

\textbf{Significance and Impact}

The development of the subclassical graph of calculi will have
significant implications for research in logic and its applications. It
will provide a comprehensive and systematic framework for understanding
the interplay of various logical systems, enabling researchers to:

\begin{itemize}
\item
  \begin{quote}
  Identify connections between seemingly disparate calculi
  \end{quote}
\item
  \begin{quote}
  Uncover hidden relationships and patterns in the landscape of logical
  systems
  \end{quote}
\item
  \begin{quote}
  Explore new avenues of research by traversing the subclassical graph
  \end{quote}
\item
  \begin{quote}
  Develop novel logical calculi with tailored properties for specific
  applications
  \end{quote}
\end{itemize}

The subclassical graph of calculi has the potential to revolutionize our
understanding of logical systems and their applications, leading to
advancements in mathematics, computer science, and philosophy.

\hypertarget{introduction-v3}{%
\subsection{Introduction v3}\label{introduction-v3}}

\textbf{Introduction}

The present research proposal delves into the conceptualization and
development of a subclassical graph of calculi, an innovative framework
for analyzing and comparing various logical systems. This project aims
to establish a comprehensive and unified representation of subclassical
logics, encompassing both well-established and emerging systems. The
proposed subclassical graph of calculi will serve as a valuable tool for
researchers working in the fields of logic, mathematics, and computer
science.

The study of subclassical logics has gained significant traction in
recent years, driven by their potential applications in artificial
intelligence, knowledge representation, and reasoning under uncertainty.
However, the lack of a unified and systematic representation of
subclassical logics has hindered progress in this area. The proposed
subclassical graph of calculi addresses this challenge by providing a
structured and interconnected representation of various subclassical
systems.

The subclassical graph of calculi will be constructed by identifying the
key structural elements of subclassical logics, such as their axiomatic
systems, proof rules, and semantic interpretations. These elements will
then be represented as nodes in a graph, with connections between nodes
representing relationships between the corresponding logics. This
graphical representation will facilitate the analysis of logical
relationships, the identification of logical connections, and the
exploration of new logical systems.

The subclassical graph of calculi will offer several unique advantages
over existing approaches to subclassical logic analysis. Firstly, it
will provide a comprehensive and unified representation of subclassical
logics, encompassing both well-established and emerging systems.
Secondly, the graphical representation will facilitate the visualization
of logical relationships and the identification of logical connections.
Thirdly, the subclassical graph of calculi will serve as a generative
framework for exploring new logical systems and their properties.

In summary, the proposed subclassical graph of calculi represents an
innovative and transformative approach to the analysis and comparison of
subclassical logics. This research project has the potential to
significantly advance the field of subclassical logic and contribute to
the development of new artificial intelligence applications.

\hypertarget{introduction-v4}{%
\subsection{Introduction v4}\label{introduction-v4}}

\textbf{Introduction}

This research proposal outlines a novel approach to constructing a
subclassical graph of calculi, a theoretical framework for understanding
the relationships between different formal systems of logic and
computation. The proposed approach leverages recent advances in graph
theory and computational logic to provide a more comprehensive and
rigorous representation of the intricate connections between various
calculi.

The study of subclassical logics and their corresponding calculi has
gained significant traction in recent years, driven by the increasing
complexity and expressive power of formal systems in various domains,
including computer science, artificial intelligence, and linguistics.
Subclassical logics deviate from classical logic in their treatment of
phenomena such as truth values, negation, and implication, offering a
richer and more nuanced framework for modeling real-world phenomena.

The proposed research project builds upon existing work in this area,
which has primarily focused on developing individual subclassical
calculi and exploring their properties. While these contributions have
been valuable, they have not yet yielded a unified framework for
understanding the relationships between different calculi. This proposal
addresses this gap by introducing a subclassical graph of calculi, a
structured representation that captures the connections between various
calculi based on their shared properties and transformations.

The subclassical graph of calculi will be constructed using a
combination of graph-theoretic techniques and computational logic
methods. The graph will represent calculi as nodes and their
relationships as edges, with edge types encoding different kinds of
connections, such as equivalence, embedding, and extension.
Computational logic tools will be employed to automate the
identification and classification of these relationships, ensuring the
comprehensiveness and consistency of the graph.

The construction of the subclassical graph of calculi will provide
several significant contributions to the field of subclassical logic and
computation. Firstly, the graph will serve as a valuable tool for
visualizing and understanding the intricate relationships between
different calculi, facilitating the discovery of new connections and
patterns. Secondly, the graph will provide a foundation for developing
new techniques for interoperating and comparing calculi, enabling the
transfer of knowledge and tools across different formal systems.
Thirdly, the graph will serve as a benchmark for evaluating the
expressiveness and computational power of different calculi, aiding in
the selection of appropriate formal systems for specific applications.

In summary, this research proposal presents a novel approach to
constructing a subclassical graph of calculi, a theoretical framework
for understanding the relationships between different formal systems of
logic and computation. The proposed approach leverages recent advances
in graph theory and computational logic to provide a more comprehensive
and rigorous representation of the intricate connections between various
calculi. The construction of the subclassical graph of calculi will
provide several significant contributions to the field of subclassical
logic and computation, including improved understanding of calculi
relationships, new techniques for interoperating calculi, and a
benchmark for evaluating calculi expressiveness.

\hypertarget{introduction-v5}{%
\subsection{Introduction v5}\label{introduction-v5}}

\hypertarget{project-overview}{%
\subsubsection{Project Overview}\label{project-overview}}

The present research proposal delves into the exploration of
subclassical graphs of calculi, a novel and underexplored domain with
the potential to revolutionize our understanding of computation and
formal logic. The project aims to develop a comprehensive framework for
constructing and analyzing subclassical graphs, shedding light on their
inherent properties and applications in various fields.

\hypertarget{context-and-related-work}{%
\subsubsection{Context and Related
Work}\label{context-and-related-work}}

Existing research in subclassical logic has primarily focused on
axiomatic and algebraic approaches, while the graphical representation
of subclassical logics remains relatively uncharted territory.
Subclassical graphs offer a unique perspective on these logics,
providing a visual and intuitive means of understanding their structure
and dynamics.

\hypertarget{unique-and-innovative-approaches}{%
\subsubsection{Unique and Innovative
Approaches}\label{unique-and-innovative-approaches}}

This research project distinguishes itself from previous work in several
key aspects:

\begin{enumerate}
\def\labelenumi{\arabic{enumi}.}
\item
  \begin{quote}
  \textbf{Formalization of Subclassical Graphs:} The project proposes a
  rigorous formalization of subclassical graphs, establishing a precise
  mathematical foundation for their analysis and manipulation.
  \end{quote}
\item
  \begin{quote}
  \textbf{Algorithmic Techniques:} Novel algorithmic techniques will be
  developed to efficiently construct and manipulate subclassical graphs,
  enabling the exploration of large and complex systems.
  \end{quote}
\item
  \begin{quote}
  \textbf{Applications in Automated Reasoning:} The project will
  investigate the application of subclassical graphs in automated
  reasoning systems, potentially leading to more powerful and versatile
  inference mechanisms.
  \end{quote}
\end{enumerate}

\hypertarget{significance-and-contributions}{%
\subsubsection{Significance and
Contributions}\label{significance-and-contributions}}

The proposed research holds significant promise for advancing our
understanding of subclassical logic and its applications. The
development of a formal framework for subclassical graphs will provide a
valuable tool for researchers and practitioners alike, facilitating the
analysis, manipulation, and application of these logics.

Moreover, the project\textquotesingle s exploration of algorithmic
techniques and applications in automated reasoning is expected to yield
tangible benefits in various domains, including artificial intelligence,
computer science, and decision-making systems.

\hypertarget{introduction-v6}{%
\subsection{Introduction v6}\label{introduction-v6}}

The field of logic has witnessed significant advancements in recent
decades, fueled by the development of powerful formal methods and
automated reasoning techniques. However, the vast and diverse landscape
of logical systems poses challenges in identifying connections and
overarching principles. Existing approaches to classifying and comparing
logical calculi often rely on ad hoc criteria, leading to a fragmented
and incomplete understanding of the relationships between different
systems.

To address these limitations, this research proposal outlines the
development of a subclassical graph of calculi, a novel approach to
formalizing and analyzing the relationships between different logical
calculi. The proposed subclassical graph will provide a comprehensive
and systematic framework for understanding the interplay of various
logical systems, enabling researchers to identify connections, uncover
hidden relationships, generate novel calculi, and explore new avenues of
research.

Context and Background

Logical calculi are formal systems for reasoning and deriving
conclusions from a set of axioms or assumptions or rules such as
structural or inferential rules. They play a fundamental role in
mathematics, computer science, and philosophy, providing a rigorous
foundation for reasoning, proof construction, refutation construction,
modeling, and countermodeling. Over the centuries, a vast array of
logical calculi have been developed, each tailored to specific
applications and embodying distinct logical properties.

Current State of Research

The study of logical calculi has witnessed significant advancements in
recent decades, fueled by the development of powerful formal methods and
automated reasoning techniques. However, the vast and diverse landscape
of logical systems poses challenges in identifying connections and
overarching principles. Existing approaches to classifying and comparing
logical calculi often rely on ad hoc criteria, leading to a fragmented
and incomplete understanding of the relationships between different
systems.

Unique and Innovative Approach

The proposed subclassical graph of calculi addresses these limitations
by providing a unified and rigorous framework for analyzing the
relationships between logical calculi. The subclassical graph will
represent logical calculi as nodes and the relationships between them as
edges, where the edges capture the logical implications and syntactic
similarities between different systems.

Significance and Impact

The development of the subclassical graph of calculi will have
significant implications for research in logic and its applications. It
will provide a comprehensive and systematic framework for understanding
the interplay of various logical systems, enabling researchers to:

\begin{itemize}
\item
  \begin{quote}
  Identify connections between seemingly disparate calculi.
  \end{quote}
\item
  \begin{quote}
  Uncover hidden relationships and patterns in the landscape of logical
  systems.
  \end{quote}
\item
  \begin{quote}
  Explore new avenues of research by traversing the subclassical graph.
  \end{quote}
\item
  \begin{quote}
  Develop novel logical calculi with tailored properties for specific
  applications.
  \end{quote}
\end{itemize}

The subclassical graph of calculi has the potential to revolutionize our
understanding of logical systems and their applications, leading to
advancements in mathematics, computer science, and philosophy.

\hypertarget{introduction-v7}{%
\subsection{Introduction v7}\label{introduction-v7}}

The project aims to develop a comprehensive framework for constructing
and analyzing subclassical graphs, shedding light on their inherent
properties and applications in various fields.

Context and Related Work:

Existing research in subclassical logic has primarily focused on
axiomatic and algebraic approaches, while the graphical representation
of subclassical logics remains relatively uncharted territory.
Subclassical graphs offer a unique perspective on these logics,
providing a visual and intuitive means of understanding their structure
and dynamics.

What is being done both generally and specifically in the same or
related areas:

The study of subclassical logics and their corresponding calculi has
gained significant traction in recent years, driven by the increasing
complexity and expressive power of formal systems in various domains,
including computer science, artificial intelligence, and linguistics.
Subclassical logics deviate from classical logic in their treatment of
phenomena such as truth values, negation, and implication, offering a
richer and more nuanced framework for modeling real-world phenomena.

Unique and Innovative Approaches:

This research project distinguishes itself from previous work in several
key aspects:

\begin{enumerate}
\def\labelenumi{\arabic{enumi}.}
\item
  \begin{quote}
  \textbf{Formalization of Subclassical Graphs:} The project proposes a
  rigorous formalization of subclassical graphs, establishing a precise
  mathematical foundation for their analysis and manipulation.
  \end{quote}
\item
  \begin{quote}
  \textbf{Algorithmic Techniques:} Novel algorithmic techniques will be
  developed to efficiently construct and manipulate subclassical graphs,
  enabling the exploration of large and complex systems.
  \end{quote}
\item
  \begin{quote}
  \textbf{Applications in Automated Reasoning:} The project will
  investigate the application of subclassical graphs in automated
  reasoning systems, potentially leading to more powerful and versatile
  inference mechanisms.
  \end{quote}
\end{enumerate}

Significance and Contributions:

The proposed research holds significant promise for advancing our
understanding of subclassical logic and its applications. The
development of a formal framework for subclassical graphs will provide a
valuable tool for researchers and practitioners alike, facilitating the
analysis, manipulation, and application of these logics. Moreover, the
project\textquotesingle s exploration of algorithmic techniques and
applications in automated reasoning is expected to yield tangible
benefits in various domains, including artificial intelligence, computer
science, and decision-making systems.

\hypertarget{needs}{%
\subsection{Needs}\label{needs}}

\textbf{What needs to be done and why?}

Subclassical logics have emerged as powerful tools for formalizing
reasoning in a wide range of domains, including computer science,
mathematics, and linguistics. However, despite their growing importance,
there is a lack of a comprehensive and unified understanding of the
subclassical graph of calculi. This lack of understanding hinders the
development of new subclassical logics and makes it difficult to compare
and contrast existing logics.

\textbf{What significant needs are you trying to meet? Compared to other
projects in the same area, what sets yours apart in terms of need?}

This project aims to develop a comprehensive and unified understanding
of the subclassical graph of calculi. This will be achieved by
developing a new formal framework for representing and reasoning about
subclassical logics. This framework will be used to develop a new
classification of subclassical logics and to identify new relationships
between existing logics.

\textbf{What services are to be delivered? Why? Use specifics from
preliminary studies, needs assessment, documentation, and data
supporting your proposal.}

This project will deliver the following services:

\begin{itemize}
\item
  \begin{quote}
  A new formal framework for representing and reasoning about
  subclassical logics.
  \end{quote}
\item
  \begin{quote}
  A new classification of subclassical logics.
  \end{quote}
\item
  \begin{quote}
  A new set of relationships between existing subclassical logics.
  \end{quote}
\end{itemize}

These services will be used to develop new subclassical logics and to
improve the understanding of existing logics.

\textbf{What gaps that your work can fill exist in the knowledge base of
your field?}

There are several gaps in the knowledge base of subclassical logic that
this project can fill. These gaps include:

\begin{itemize}
\item
  \begin{quote}
  A lack of a comprehensive and unified understanding of the
  subclassical graph of calculi.
  \end{quote}
\item
  \begin{quote}
  A lack of a formal framework for representing and reasoning about
  subclassical logics.
  \end{quote}
\item
  \begin{quote}
  A lack of a clear and concise classification of subclassical logics.
  \end{quote}
\end{itemize}

\textbf{Is the problem both significant and manageable? Do you have the
resources to handle the problem?}

The problem of developing a comprehensive and unified understanding of
the subclassical graph of calculi is both significant and manageable.
The problem is significant because it is a fundamental problem in
subclassical logic that has not been solved. The problem is manageable
because there are a number of existing results that can be used to build
upon. The team working on this project has the resources necessary to
handle the problem. The team includes experts in subclassical logic and
formal methods.

In addition to the above, the project will also develop a new set of
tools for working with subclassical logics. These tools will include a
new theorem prover for subclassical logics and a new software library
for implementing subclassical logics.

\hypertarget{need-v2}{%
\subsection{Need v2}\label{need-v2}}

\textbf{What needs to be done and why?}

Subclassical calculi are a diverse and powerful family of formal systems
that have found applications in a wide range of areas, including
computer science, mathematics, and linguistics. However, there is no
single unifying framework for understanding the relationships between
these different calculi. The development of such a framework would be a
significant contribution to the field of logic.

\textbf{What significant needs are you trying to meet? Compared to other
projects in the same area, what sets yours apart in terms of need?}

A unifying framework for subclassical calculi would provide a number of
benefits, including:

\begin{itemize}
\item
  \begin{quote}
  A better understanding of the logical relationships between different
  subclassical calculi.
  \end{quote}
\item
  \begin{quote}
  The development of new and more powerful subclassical calculi.
  \end{quote}
\item
  \begin{quote}
  The application of subclassical calculi to new areas.
  \end{quote}
\end{itemize}

Our project is different from other projects in the same area in that it
takes a graph-based approach to understanding subclassical calculi. This
approach has the potential to provide a more comprehensive and
insightful understanding of these calculi than traditional approaches.

\textbf{What services are to be delivered? Why? Use specifics from
preliminary studies, needs assessment, documentation, and data
supporting your proposal.}

The proposed project will deliver the following services:

\begin{itemize}
\item
  \begin{quote}
  A graph-based representation of subclassical calculi.
  \end{quote}
\item
  \begin{quote}
  A methodology for comparing and contrasting different subclassical
  calculi.
  \end{quote}
\item
  \begin{quote}
  A set of tools for developing new subclassical calculi.
  \end{quote}
\end{itemize}

These services will be delivered by a team of researchers with expertise
in logic, graph theory, and computer science. The researchers will use a
variety of methods, including theoretical analysis, experimental
evaluation, and software development.

\textbf{What gaps that your work can fill exist in the knowledge base of
your field?}

There are a number of gaps in the knowledge base of our field that our
work can fill. These gaps include:

\begin{itemize}
\item
  \begin{quote}
  A lack of a unifying framework for understanding subclassical calculi.
  \end{quote}
\item
  \begin{quote}
  A lack of tools for comparing and contrasting different subclassical
  calculi.
  \end{quote}
\item
  \begin{quote}
  A lack of methods for developing new subclassical calculi.
  \end{quote}
\end{itemize}

\textbf{Is the problem both significant and manageable? Do you have the
resources to handle the problem?}

We believe that the problem of developing a unifying framework for
subclassical calculi is both significant and manageable. We have the
resources to handle the problem, including a team of experienced
researchers and access to a variety of computational resources.

We are confident that our project will make a significant contribution
to the field of logic.

\hypertarget{need-v3}{%
\subsection{Need v3}\label{need-v3}}

\textbf{Need Statement}

\textbf{What needs to be done and why?}

There is a need for a new approach to understanding and comparing
subclassical calculi. The existing approaches are often ad hoc and do
not provide a clear picture of the relationships between different
calculi. A graph-based approach has the potential to provide a more
systematic and insightful way to study subclassical calculi.

\textbf{What significant needs are you trying to meet? Compared to other
projects in the same area, what sets yours apart in terms of need?}

This project will address the following significant needs:

\begin{itemize}
\item
  \begin{quote}
  The need for a more systematic and insightful way to study
  subclassical calculi.
  \end{quote}
\item
  \begin{quote}
  The need for a better understanding of the relationships between
  different subclassical calculi.
  \end{quote}
\item
  \begin{quote}
  The need for a new tool that can be used to compare and contrast
  different subclassical calculi.
  \end{quote}
\end{itemize}

This project is different from other projects in the same area in that
it will take a graph-based approach to studying subclassical calculi.
This approach has the potential to provide a more systematic and
insightful way to study subclassical calculi than the existing
approaches.

\textbf{What services are to be delivered? Why? Use specifics from
preliminary studies, needs assessment, documentation, and data
supporting your proposal.}

This project will deliver the following services:

\begin{itemize}
\item
  \begin{quote}
  A new graph-based approach to studying subclassical calculi.
  \end{quote}
\item
  \begin{quote}
  A new tool that can be used to compare and contrast different
  subclassical calculi.
  \end{quote}
\item
  \begin{quote}
  A new understanding of the relationships between different
  subclassical calculi.
  \end{quote}
\end{itemize}

These services will be delivered using the following methods:

\begin{itemize}
\item
  \begin{quote}
  A literature review of existing approaches to studying subclassical
  calculi.
  \end{quote}
\item
  \begin{quote}
  The development of a new graph-based approach to studying subclassical
  calculi.
  \end{quote}
\item
  \begin{quote}
  The development of a new tool that can be used to compare and contrast
  different subclassical calculi.
  \end{quote}
\item
  \begin{quote}
  The application of the new approach and tool to a variety of
  subclassical calculi.
  \end{quote}
\end{itemize}

\textbf{What gaps that your work can fill exist in the knowledge base of
your field?}

This project will fill the following gaps in the knowledge base of our
field:

\begin{itemize}
\item
  \begin{quote}
  The gap in our understanding of the systematic and insightful way to
  study subclassical calculi.
  \end{quote}
\item
  \begin{quote}
  The gap in our understanding of the relationships between different
  subclassical calculi.
  \end{quote}
\item
  \begin{quote}
  The gap in the tool that can be used to compare and contrast different
  subclassical calculi.
  \end{quote}
\end{itemize}

\textbf{Is the problem both significant and manageable? Do you have the
resources to handle the problem?}

The problem is both significant and manageable. The problem is
significant because it is a fundamental problem in the field of logic.
The problem is manageable because there is a clear path to a solution.
The resources to handle the problem are available.

\textbf{Call to Action}

We urge you to support this project so that we can make a significant
contribution to our understanding of subclassical calculi.

\hypertarget{goals-and-objectives-v1}{%
\subsection{Goals and Objectives v1}\label{goals-and-objectives-v1}}

\textbf{Goals}

\begin{itemize}
\tightlist
\item
\end{itemize}

\textbf{Objectives}

\begin{itemize}
\item
  \begin{quote}
  Compile a comprehensive list of non-classical logics, including their
  axioms, rules, and properties.
  \end{quote}
\item
  \begin{quote}
  Classify the non-classical logics based on their properties, such as
  the types of negation that they use, the consistency conditions that
  they satisfy, and the types of truth values that they employ.
  \end{quote}
\item
  \begin{quote}
  Construct the subclassical graph of calculi by connecting the
  different non-classical logics based on their relationships.
  \end{quote}
\item
  \begin{quote}
  Analyze the subclassical graph of calculi to identify common patterns
  and themes, and to suggest new areas of research.
  \end{quote}
\item
  \begin{quote}
  Develop a new formal framework for representing and reasoning about
  subclassical logics.
  \end{quote}
\item
  \begin{quote}
  Develop a new classification of subclassical logics.
  \end{quote}
\item
  \begin{quote}
  Identify new relationships between existing subclassical logics.
  \end{quote}
\item
  \begin{quote}
  Develop a new set of tools for working with subclassical logics.
  \end{quote}
\end{itemize}

\textbf{Timeline}

\begin{itemize}
\item
  \begin{quote}
  Identify and classify the non-classical logics: 6 months.
  \end{quote}
\item
  \begin{quote}
  Construct the subclassical graph of calculi: 3 months.
  \end{quote}
\item
  \begin{quote}
  Analyze the subclassical graph of calculi: 2 months.
  \end{quote}
\item
  \begin{quote}
  Develop a new formal framework for representing and reasoning about
  subclassical logics: 4 months.
  \end{quote}
\item
  \begin{quote}
  Develop a new classification of subclassical logics: 3 months.
  \end{quote}
\item
  \begin{quote}
  Identify new relationships between existing subclassical logics: 2
  months.
  \end{quote}
\item
  \begin{quote}
  Develop a new set of tools for working with subclassical logics: 5
  months.
  \end{quote}
\end{itemize}

\textbf{Budget}

\begin{itemize}
\item
  \begin{quote}
  Personnel: \$50,000.
  \end{quote}
\item
  \begin{quote}
  Travel: \$10,000.
  \end{quote}
\item
  \begin{quote}
  Equipment: \$10,000.
  \end{quote}
\item
  \begin{quote}
  Subcontracts: \$10,000.
  \end{quote}
\item
  \begin{quote}
  Other expenses: \$10,000.
  \end{quote}
\end{itemize}

\textbf{Total:} \$100,000.

\textbf{Personnel}

\begin{itemize}
\item
  \begin{quote}
  Principal investigator: \$10,000.
  \end{quote}
\item
  \begin{quote}
  Research assistant: \$5,000.
  \end{quote}
\item
  \begin{quote}
  Software engineer: \$5,000.
  \end{quote}
\end{itemize}

\textbf{Travel}

\begin{itemize}
\item
  \begin{quote}
  Conference travel: \$5,000.
  \end{quote}
\item
  \begin{quote}
  Site visits: \$5,000.
  \end{quote}
\end{itemize}

\textbf{Equipment}

\begin{itemize}
\item
  \begin{quote}
  Computer hardware: \$5,000.
  \end{quote}
\item
  \begin{quote}
  Software licenses: \$5,000.
  \end{quote}
\end{itemize}

\textbf{Subcontracts}

\begin{itemize}
\item
  \begin{quote}
  Data analysis: \$5,000.
  \end{quote}
\item
  \begin{quote}
  Software development: \$5,000.
  \end{quote}
\end{itemize}

\textbf{Other expenses}

\begin{itemize}
\item
  \begin{quote}
  Publications: \$5,000.
  \end{quote}
\item
  \begin{quote}
  Legal fees: \$5,000.
  \end{quote}
\end{itemize}

\textbf{Total:} \$100,000.

\hypertarget{goals-and-objectives-v2}{%
\subsection{Goals and Objectives v2}\label{goals-and-objectives-v2}}

\textbf{Goals}

\begin{itemize}
\tightlist
\item
\end{itemize}

\textbf{Objectives}

\begin{itemize}
\item
  \begin{quote}
  To compile a comprehensive list of non-classical logics, including
  their axioms, rules, and properties.
  \end{quote}
\item
  \begin{quote}
  To develop a new formal framework for representing and reasoning about
  subclassical logics.
  \end{quote}
\item
  \begin{quote}
  To identify new relationships between existing non-classical logics.
  \end{quote}
\item
  \begin{quote}
  To develop a new set of tools for working with subclassical logics.
  \end{quote}
\item
  \begin{quote}
  To apply the subclassical graph of calculi to develop new and more
  powerful subclassical logics.
  \end{quote}
\item
  \begin{quote}
  To apply the subclassical graph of calculi to new areas, such as
  computer science, mathematics, and linguistics.
  \end{quote}
\end{itemize}

\hypertarget{goals-and-objectives-v3}{%
\subsection{Goals and Objectives v3}\label{goals-and-objectives-v3}}

\textbf{Goals:}

\begin{itemize}
\tightlist
\item
\end{itemize}

\textbf{Objectives:}

\begin{itemize}
\item
  \begin{quote}
  \textbf{Compile a comprehensive list of non-classical logics,
  including their axioms, rules, and properties.\\
  }
  \end{quote}

  \begin{itemize}
  \item
    \begin{quote}
    This will be completed by the end of the first quarter.
    \end{quote}
  \item
    \begin{quote}
    This will involve identifying and researching existing non-classical
    logics.
    \end{quote}
  \item
    \begin{quote}
    This will require a literature review and analysis of existing work.
    \end{quote}
  \end{itemize}
\item
\item
  \begin{quote}
  \textbf{Develop a new formal framework for representing and reasoning
  about subclassical logics.\\
  }
  \end{quote}

  \begin{itemize}
  \item
    \begin{quote}
    This will be completed by the end of the second quarter.
    \end{quote}
  \item
    \begin{quote}
    This will involve developing a new mathematical formalism for
    representing non-classical logics.
    \end{quote}
  \item
    \begin{quote}
    This will require the creation of new definitions, axioms, and
    rules.
    \end{quote}
  \end{itemize}
\item
\item
  \begin{quote}
  \textbf{Identify new relationships between existing non-classical
  logics.\\
  }
  \end{quote}

  \begin{itemize}
  \item
    \begin{quote}
    This will be completed by the end of the third quarter.
    \end{quote}
  \item
    \begin{quote}
    This will involve analyzing the subclassical graph of calculi to
    identify patterns and connections between different logics.
    \end{quote}
  \item
    \begin{quote}
    This will require the development of new methods for comparing and
    contrasting logics.
    \end{quote}
  \end{itemize}
\item
\item
  \begin{quote}
  \textbf{Develop a new set of tools for working with subclassical
  logics.\\
  }
  \end{quote}

  \begin{itemize}
  \item
    \begin{quote}
    This will be completed by the end of the fourth quarter.
    \end{quote}
  \item
    \begin{quote}
    This will involve developing new software tools for manipulating and
    analyzing non-classical logics.
    \end{quote}
  \item
    \begin{quote}
    This will require the creation of new algorithms and data
    structures.
    \end{quote}
  \end{itemize}
\item
\end{itemize}

\textbf{Timeline:}

\begin{itemize}
\item
  \begin{quote}
  Identify and classify the non-classical logics: 6 months.
  \end{quote}
\item
  \begin{quote}
  Construct the subclassical graph of calculi: 3 months.
  \end{quote}
\item
  \begin{quote}
  Analyze the subclassical graph of calculi: 2 months.
  \end{quote}
\item
  \begin{quote}
  Develop a new formal framework for representing and reasoning about
  subclassical logics: 4 months.
  \end{quote}
\item
  \begin{quote}
  Develop a new classification of subclassical logics: 3 months.
  \end{quote}
\item
  \begin{quote}
  Identify new relationships between existing subclassical logics: 2
  months.
  \end{quote}
\item
  \begin{quote}
  Develop a new set of tools for working with subclassical logics: 5
  months.
  \end{quote}
\end{itemize}

\textbf{Budget:}

\begin{itemize}
\item
  \begin{quote}
  Personnel: \$50,000.
  \end{quote}
\item
  \begin{quote}
  Travel: \$10,000.
  \end{quote}
\item
  \begin{quote}
  Equipment: \$10,000.
  \end{quote}
\item
  \begin{quote}
  Subcontracts: \$10,000.
  \end{quote}
\item
  \begin{quote}
  Other expenses: \$10,000.
  \end{quote}
\item
  \begin{quote}
  Total: \$100,000.
  \end{quote}
\end{itemize}

\textbf{Personnel:}

\begin{itemize}
\item
  \begin{quote}
  Principal investigator: \$10,000.
  \end{quote}
\item
  \begin{quote}
  Research assistant: \$5,000.
  \end{quote}
\item
  \begin{quote}
  Software engineer: \$5,000.
  \end{quote}
\item
  \begin{quote}
  Travel: Conference travel: \$5,000.
  \end{quote}
\item
  \begin{quote}
  Site visits: \$5,000.
  \end{quote}
\item
  \begin{quote}
  Equipment: Computer hardware: \$5,000.
  \end{quote}
\item
  \begin{quote}
  Software licenses: \$5,000.
  \end{quote}
\item
  \begin{quote}
  Subcontracts: Data analysis: \$5,000.
  \end{quote}
\item
  \begin{quote}
  Software development: \$5,000.
  \end{quote}
\item
  \begin{quote}
  Other expenses: Publications: \$5,000.
  \end{quote}
\item
  \begin{quote}
  Legal fees: \$5,000.
  \end{quote}
\item
  \begin{quote}
  Total: \$100,000.
  \end{quote}
\end{itemize}

\hypertarget{goals-and-objectives-v4}{%
\subsection{Goals and Objectives v4}\label{goals-and-objectives-v4}}

The following are the measurable outcomes of the project:

\begin{itemize}
\tightlist
\item
\end{itemize}

The following are the specific conditions under which the project will
be carried out:

\begin{itemize}
\item
  \begin{quote}
  The project will be carried out by a team of experts in logic, graph
  theory, and computer science.
  \end{quote}
\item
  \begin{quote}
  The project will use a variety of methods, including theoretical
  analysis, experimental evaluation, and software development.
  \end{quote}
\item
  \begin{quote}
  The project will be funded by a grant from a research institution.
  \end{quote}
\end{itemize}

The following is the timeline for the project:

\begin{itemize}
\item
  \begin{quote}
  Identify and classify the non-classical logics: 6 months.
  \end{quote}
\item
  \begin{quote}
  Construct the subclassical graph of calculi: 3 months.
  \end{quote}
\item
  \begin{quote}
  Analyze the subclassical graph of calculi: 2 months.
  \end{quote}
\item
  \begin{quote}
  Develop a new formal framework for representing and reasoning about
  non-classical logics: 4 months.
  \end{quote}
\item
  \begin{quote}
  Develop a new classification of non-classical logics: 3 months.
  \end{quote}
\item
  \begin{quote}
  Identify new relationships between existing non-classical logics: 2
  months.
  \end{quote}
\item
  \begin{quote}
  Develop a new set of tools for working with non-classical logics: 5
  months.
  \end{quote}
\end{itemize}

\hypertarget{goals-and-objectives-v5}{%
\subsection{Goals and Objectives v5}\label{goals-and-objectives-v5}}

Goals:

\begin{itemize}
\tightlist
\item
\end{itemize}

Objectives:

\begin{itemize}
\item
  \begin{quote}
  Compile a comprehensive list of non-classical logics, including their
  axioms, rules, and properties.
  \end{quote}
\item
  \begin{quote}
  Classify the non-classical logics based on their properties, such as
  the types of negation that they use, the consistency conditions that
  they satisfy, and the types of truth values that they employ.
  \end{quote}
\item
  \begin{quote}
  Construct the subclassical graph of calculi by connecting the
  different non-classical logics based on their relationships.
  \end{quote}
\item
  \begin{quote}
  Analyze the subclassical graph of calculi to identify common patterns
  and themes, and to suggest new areas of research.
  \end{quote}
\item
  \begin{quote}
  Develop a new formal framework for representing and reasoning about
  subclassical logics.
  \end{quote}
\item
  \begin{quote}
  Develop a new classification of subclassical logics.
  \end{quote}
\item
  \begin{quote}
  Identify new relationships between existing subclassical logics.
  \end{quote}
\item
  \begin{quote}
  Develop a new set of tools for working with subclassical logics.
  \end{quote}
\end{itemize}

Timeline:

\begin{itemize}
\item
  \begin{quote}
  Identify and classify the non-classical logics: 6 months.
  \end{quote}
\item
  \begin{quote}
  Construct the subclassical graph of calculi: 3 months.
  \end{quote}
\item
  \begin{quote}
  Analyze the subclassical graph of calculi: 2 months.
  \end{quote}
\item
  \begin{quote}
  Develop a new formal framework for representing and reasoning about
  subclassical logics: 4 months.
  \end{quote}
\item
  \begin{quote}
  Develop a new classification of subclassical logics: 3 months.
  \end{quote}
\item
  \begin{quote}
  Identify new relationships between existing subclassical logics: 2
  months.
  \end{quote}
\item
  \begin{quote}
  Develop a new set of tools for working with subclassical logics: 5
  months.
  \end{quote}
\end{itemize}

Budget:

\begin{itemize}
\item
  \begin{quote}
  Personnel: \$50,000.
  \end{quote}
\item
  \begin{quote}
  Travel: \$10,000.
  \end{quote}
\item
  \begin{quote}
  Equipment: \$10,000.
  \end{quote}
\item
  \begin{quote}
  Subcontracts: \$10,000.
  \end{quote}
\item
  \begin{quote}
  Other expenses: \$10,000.
  \end{quote}
\item
  \begin{quote}
  Total: \$100,000.
  \end{quote}
\end{itemize}

Personnel:

\begin{itemize}
\item
  \begin{quote}
  Principal investigator: \$10,000.
  \end{quote}
\item
  \begin{quote}
  Research assistant: \$5,000.
  \end{quote}
\item
  \begin{quote}
  Software engineer: \$5,000.
  \end{quote}
\end{itemize}

Travel:

\begin{itemize}
\item
  \begin{quote}
  Conference travel: \$5,000.
  \end{quote}
\item
  \begin{quote}
  Site visits: \$5,000.
  \end{quote}
\end{itemize}

Equipment:

\begin{itemize}
\item
  \begin{quote}
  Computer hardware: \$5,000.
  \end{quote}
\item
  \begin{quote}
  Software licenses: \$5,000.
  \end{quote}
\end{itemize}

Subcontracts:

\begin{itemize}
\item
  \begin{quote}
  Data analysis: \$5,000.
  \end{quote}
\item
  \begin{quote}
  Software development: \$5,000.
  \end{quote}
\end{itemize}

Other expenses:

\begin{itemize}
\item
  \begin{quote}
  Publications: \$5,000.
  \end{quote}
\item
  \begin{quote}
  Legal fees: \$5,000.
  \end{quote}
\end{itemize}

\hypertarget{goals-and-objectives-v6}{%
\subsection{Goals and Objectives v6}\label{goals-and-objectives-v6}}

\textbf{Goals}

\begin{itemize}
\tightlist
\item
\end{itemize}

\textbf{Objectives}

\begin{itemize}
\item
  \begin{quote}
  \textbf{Objective 1:} Define a graph representation of subclassical
  calculi that is both expressive and computationally efficient.
  \end{quote}
\item
  \begin{quote}
  \textbf{Measurable Outcome 1:} A formal definition of the graph
  representation of subclassical calculi, along with a proof of its
  expressiveness and computational efficiency.
  \end{quote}
\item
  \begin{quote}
  \textbf{Objective 2:} Develop algorithms for manipulating and
  analyzing graphs representing subclassical calculi.
  \end{quote}
\item
  \begin{quote}
  \textbf{Measurable Outcome 2:} Implementation of algorithms for
  manipulating and analyzing graphs representing subclassical calculi,
  along with a demonstration of their effectiveness and efficiency.
  \end{quote}
\item
  \begin{quote}
  \textbf{Objective 3:} Apply the graph representation of subclassical
  calculi to investigate the structural properties of a variety of
  subclassical logics.
  \end{quote}
\item
  \begin{quote}
  \textbf{Measurable Outcome 3:} A series of papers presenting new
  results on the structural properties of subclassical logics, obtained
  using the graph representation of subclassical calculi.
  \end{quote}
\item
  \begin{quote}
  \textbf{Objective 4:} Use the graph representation of subclassical
  calculi to develop new subclassical logics.
  \end{quote}
\item
  \begin{quote}
  \textbf{Measurable Outcome 4:} A series of papers presenting new
  subclassical logics, along with a demonstration of their properties
  and applications.
  \end{quote}
\end{itemize}

\hypertarget{goals-and-objectives-v7}{%
\subsection{Goals and Objectives v7}\label{goals-and-objectives-v7}}

\textbf{Goals}

\begin{itemize}
\tightlist
\item
\end{itemize}

\textbf{Objectives}

\begin{itemize}
\item
  \begin{quote}
  To create a database of subclassical logics, including their axioms,
  theorems, and proof systems.
  \end{quote}
\item
  \begin{quote}
  To develop algorithms for computing the entailment relations between
  subclassical logics.
  \end{quote}
\item
  \begin{quote}
  To use the subclassical graph of calculi to prove new theorems about
  subclassical logics.
  \end{quote}
\end{itemize}

\textbf{Relationships between objectives and measurements and outcomes}

The objective of creating a database of subclassical logics will be
measured by the number of logics in the database and the completeness of
the information about each logic. The objective of developing algorithms
for computing the entailment relations between subclassical logics will
be measured by the efficiency and accuracy of the algorithms. The
objective of using the subclassical graph of calculi to prove new
theorems about subclassical logics will be measured by the number of new
theorems that are proven.

\textbf{Goals and Objectives}

The overall goal of this project is to develop a better understanding of
subclassical logics. Subclassical logics are a generalization of
classical logic that allow for the possibility of truth values that are
neither true nor false. Subclassical logics have been used to model a
variety of phenomena, including quantum mechanics, fuzzy logic, and
paraconsistent reasoning.

The objectives of this project are to develop a subclassical graph of
calculi that captures the relationships between different types of
subclassical logics and to use the subclassical graph of calculi to
identify new subclassical logics and to prove new theorems about
subclassical logics.

The subclassical graph of calculi will be a directed graph in which the
nodes represent subclassical logics and the edges represent entailment
relations between subclassical logics. The entailment relation between
two subclassical logics L and L\textquotesingle{} is defined as follows:
L entails L\textquotesingle{} if every theorem of L is also a theorem of
L\textquotesingle.

The subclassical graph of calculi will be used to identify new
subclassical logics by identifying connected components of the graph. A
connected component of the graph is a subset of the nodes of the graph
such that there is a path between every pair of nodes in the subset.
Each connected component of the graph represents a different type of
subclassical logic.

The subclassical graph of calculi will also be used to prove new
theorems about subclassical logics by using the graph to identify lemmas
that can be used to prove the theorems. A lemma is a theorem that is
used to prove another theorem.

The development of the subclassical graph of calculi will make a
significant contribution to our understanding of subclassical logics.
The graph will be a valuable tool for identifying new subclassical
logics and for proving new theorems about subclassical logics.

\hypertarget{goals-and-objectives-v8}{%
\subsection{Goals and Objectives v8}\label{goals-and-objectives-v8}}

The goals and objectives of this project are to:

\begin{itemize}
\item
  \begin{quote}
  \textbf{Develop a comprehensive and unified representation of
  subclassical logics}. This will involve identifying and classifying
  the different types of subclassical logics that exist, as well as
  their axioms, rules, and properties.
  \end{quote}
\item
  \begin{quote}
  \textbf{Construct a subclassical graph of calculi}. This will be a
  diagram that visually represents the relationships between different
  subclassical logics. The graph will be constructed by connecting the
  different subclassical logics based on their relationships.
  \end{quote}
\item
  \begin{quote}
  \textbf{Analyze the subclassical graph of calculi}. This will involve
  identifying common patterns and themes in the graph, and suggesting
  new areas of research.
  \end{quote}
\item
  \begin{quote}
  \textbf{Develop novel logical calculi with tailored properties for
  specific applications}. This will involve using the subclassical graph
  of calculi to identify gaps in the existing landscape of logical
  systems, and then developing new calculi to fill those gaps.
  \end{quote}
\end{itemize}

The objectives of this project are to:

\begin{itemize}
\item
  \begin{quote}
  \textbf{Identify connections between seemingly disparate subclassical
  logics}. This will involve using the subclassical graph of calculi to
  identify relationships between logics that are not immediately
  apparent.
  \end{quote}
\item
  \begin{quote}
  \textbf{Uncover hidden relationships and patterns in the landscape of
  subclassical logics}. This will involve using the subclassical graph
  of calculi to identify patterns and relationships that are not easily
  visible when looking at individual logics in isolation.
  \end{quote}
\item
  \begin{quote}
  \textbf{Explore new avenues of research by traversing the subclassical
  graph}. This will involve using the subclassical graph of calculi to
  identify new directions for research in subclassical logic.
  \end{quote}
\item
  \begin{quote}
  \textbf{Develop novel logical calculi with tailored properties for
  specific applications}. This will involve using the subclassical graph
  of calculi to identify gaps in the existing landscape of logical
  systems, and then developing new calculi to fill those gaps.
  \end{quote}
\end{itemize}

The measurements and outcomes of this project are to:

\begin{itemize}
\item
  \begin{quote}
  \textbf{Develop a comprehensive and unified representation of
  subclassical logics}. This will be measured by the quality and
  completeness of the representation.
  \end{quote}
\item
  \begin{quote}
  \textbf{Construct a subclassical graph of calculi}. This will be
  measured by the accuracy and completeness of the graph.
  \end{quote}
\item
  \begin{quote}
  \textbf{Analyze the subclassical graph of calculi}. This will be
  measured by the identification of common patterns and themes in the
  graph.
  \end{quote}
\item
  \begin{quote}
  \textbf{Develop novel logical calculi with tailored properties for
  specific applications}. This will be measured by the development of
  new calculi that meet the specified requirements.
  \end{quote}
\end{itemize}

The relationships between the objectives and the measurements and
outcomes are as follows:

\begin{itemize}
\item
  \begin{quote}
  The objectives of identifying connections between subclassical logics,
  uncovering hidden relationships and patterns, exploring new avenues of
  research, and developing novel logical calculi are all related to the
  measurement of developing a comprehensive and unified representation
  of subclassical logics.
  \end{quote}
\item
  \begin{quote}
  The objectives of identifying connections between subclassical logics,
  uncovering hidden relationships and patterns, and exploring new
  avenues of research are all related to the measurement of constructing
  a subclassical graph of calculi.
  \end{quote}
\item
  \begin{quote}
  The objectives of analyzing the subclassical graph of calculi,
  developing novel logical calculi with tailored properties for specific
  applications, and developing a comprehensive and unified
  representation of subclassical logics are all related to the
  measurement of analyzing the subclassical graph of calculi.
  \end{quote}
\item
  \begin{quote}
  The objectives of developing novel logical calculi with tailored
  properties for specific applications, and developing a comprehensive
  and unified representation of subclassical logics are all related to
  the measurement of developing novel logical calculi with tailored
  properties for specific applications.
  \end{quote}
\end{itemize}

\hypertarget{goals-and-objectives-v9}{%
\subsection{Goals and Objectives v9}\label{goals-and-objectives-v9}}

\textbf{Goals}

The project has the following goals:

\begin{itemize}
\tightlist
\item
\end{itemize}

\textbf{Objectives}

The project has the following objectives:

\begin{enumerate}
\def\labelenumi{\arabic{enumi}.}
\item
  \begin{quote}
  \textbf{Construct a subclassical graph of calculi:} This objective
  will be achieved by identifying and classifying the different types of
  subclassical logics and then representing them as nodes in a graph.
  The edges of the graph will represent the relationships between the
  different logics, such as equivalence, embedding, and extension.
  \end{quote}
\item
  \begin{quote}
  \textbf{Develop a formal framework for representing and reasoning
  about subclassical graphs:} This objective will be achieved by
  defining a formal language for describing subclassical graphs and by
  developing algorithms for manipulating and reasoning about these
  graphs. The formal framework will be used to identify new
  relationships between existing subclassical logics and to develop new
  tools for working with subclassical logics.
  \end{quote}
\item
  \begin{quote}
  \textbf{Develop a new classification of subclassical logics:} This
  objective will be achieved by analyzing the relationships between the
  different types of subclassical logics and identifying patterns and
  trends. The new classification will be based on the properties of the
  logics, such as their axioms, rules, and semantics.
  \end{quote}
\item
  \begin{quote}
  \textbf{Identify new relationships between existing subclassical
  logics:} This objective will be achieved by exploring the subclassical
  graph and identifying new connections between the different logics.
  These new relationships may be based on the properties of the logics,
  their applications, or their historical development.
  \end{quote}
\item
  \begin{quote}
  \textbf{Develop new tools for working with subclassical logics:} This
  objective will be achieved by developing new software tools, such as a
  theorem prover and a software library, that can be used to work with
  subclassical logics. These tools will be based on the formal framework
  and the new classification of subclassical logics.
  \end{quote}
\end{enumerate}

\textbf{Measurements}

The project will measure its success by the following metrics:

\begin{itemize}
\item
  \begin{quote}
  The quality of the subclassical graph of calculi.
  \end{quote}
\item
  \begin{quote}
  The completeness and accuracy of the new classification of
  subclassical logics.
  \end{quote}
\item
  \begin{quote}
  The number of new relationships identified between existing
  subclassical logics.
  \end{quote}
\item
  \begin{quote}
  The usefulness of the new tools for working with subclassical logics.
  \end{quote}
\end{itemize}

\textbf{Outcomes}

The project aims to achieve the following outcomes:

\begin{itemize}
\item
  \begin{quote}
  A comprehensive and unified understanding of the subclassical graph of
  calculi.
  \end{quote}
\item
  \begin{quote}
  A new classification of subclassical logics that is based on their
  relationships in the subclassical graph.
  \end{quote}
\item
  \begin{quote}
  A new set of tools for working with subclassical logics.
  \end{quote}
\end{itemize}

These outcomes will have the following benefits:

\begin{itemize}
\item
  \begin{quote}
  Improved understanding of the relationships between different
  subclassical logics.
  \end{quote}
\item
  \begin{quote}
  Development of new and more powerful subclassical logics.
  \end{quote}
\item
  \begin{quote}
  Application of subclassical logics to new areas.
  \end{quote}
\end{itemize}

The project will also contribute to the field of logic by developing a
new formal framework for representing and reasoning about subclassical
graphs. This framework will be a valuable tool for researchers and
practitioners alike, facilitating the analysis, manipulation, and
application of these logics.

\hypertarget{goals-and-objectives-v10}{%
\subsection{Goals and Objectives v10}\label{goals-and-objectives-v10}}

The goals of the project are to:

\begin{itemize}
\item
  \begin{quote}
  \textbf{Develop a comprehensive and unified understanding of the
  subclassical graph of calculi}. This will be achieved by developing a
  new formal framework for representing and reasoning about subclassical
  logics. This framework will be used to develop a new classification of
  subclassical logics and to identify new relationships between existing
  logics.
  \end{quote}
\item
  \begin{quote}
  \textbf{Identify new connections between seemingly disparate
  subclassical calculi}. This will be done by analyzing the subclassical
  graph of calculi and identifying patterns and relationships that have
  not been previously recognized. This could lead to the development of
  new and more powerful subclassical logics.
  \end{quote}
\item
  \begin{quote}
  \textbf{Develop novel logical calculi with tailored properties for
  specific applications}. This will be done by leveraging the insights
  gained from the subclassical graph of calculi to design new logics
  that are better suited for specific tasks, such as reasoning under
  uncertainty or representing vagueness.
  \end{quote}
\item
  \begin{quote}
  \textbf{Explore the application of subclassical graphs in automated
  reasoning systems}. This could lead to the development of more
  powerful and versatile inference mechanisms for subclassical logics.
  \end{quote}
\item
  \begin{quote}
  \textbf{Establish a formal foundation for the development of
  superclassical and hyperclassical graphs of calculi}. This would pave
  the way for the study of more general classes of logical calculi, such
  as superclassical and hyperclassical logics.
  \end{quote}
\item
  \begin{quote}
  To develop a formal framework for representing and reasoning about
  subclassical graphs.
  \end{quote}
\item
  \begin{quote}
  To develop a new classification of subclassical logics based on their
  relationships in the subclassical graph.
  \end{quote}
\item
  \begin{quote}
  To identify new relationships between existing subclassical logics.
  \end{quote}
\item
  \begin{quote}
  To develop new tools for working with subclassical logics, such as a
  theorem prover and a software library.
  \end{quote}
\item
  \begin{quote}
  To develop a subclassical graph of calculi that captures the
  relationships between different types of subclassical logics.
  \end{quote}
\item
  \begin{quote}
  To use the subclassical graph of calculi to identify new subclassical
  logics and to prove new theorems about subclassical logics.
  \end{quote}
\item
  \begin{quote}
  To develop a formal framework for representing subclassical calculi
  using graphs.
  \end{quote}
\item
  \begin{quote}
  To use this framework to investigate the structural properties of
  subclassical calculi.
  \end{quote}
\item
  \begin{quote}
  To apply this framework to the development of new subclassical logics.
  \end{quote}
\item
  \begin{quote}
  To develop a comprehensive and unified understanding of the
  subclassical graph of calculi.
  \end{quote}
\item
  \begin{quote}
  To develop a new formal framework for representing and reasoning about
  subclassical logics.
  \end{quote}
\item
  \begin{quote}
  To develop a new classification of subclassical logics.
  \end{quote}
\item
  \begin{quote}
  To identify new relationships between existing subclassical logics.
  \end{quote}
\item
  \begin{quote}
  To develop a new set of tools for working with subclassical logics.
  \end{quote}
\item
  \begin{quote}
  A comprehensive list of non-classical logics, including their axioms,
  rules, and properties.
  \end{quote}
\item
  \begin{quote}
  A new formal framework for representing and reasoning about
  non-classical logics.
  \end{quote}
\item
  \begin{quote}
  A new classification of non-classical logics.
  \end{quote}
\item
  \begin{quote}
  A subclassical graph of calculi that visually represents the
  relationships between these logics.
  \end{quote}
\item
  \begin{quote}
  A list of common patterns and themes in the properties of
  non-classical logics.
  \end{quote}
\item
  \begin{quote}
  A list of new areas of research suggested by the analysis of the
  subclassical graph of calculi.
  \end{quote}
\item
  \begin{quote}
  To develop a comprehensive and unified understanding of the
  subclassical graph of calculi.
  \end{quote}
\item
  \begin{quote}
  To identify and classify the different types of non-classical logics
  that exist.
  \end{quote}
\item
  \begin{quote}
  To construct a subclassical graph of calculi that visually represents
  the relationships between these logics.
  \end{quote}
\item
  \begin{quote}
  To analyze the subclassical graph of calculi to identify common
  patterns and themes.
  \end{quote}
\item
  \begin{quote}
  To develop a comprehensive and unified understanding of the
  subclassical graph of calculi.
  \end{quote}
\item
  \begin{quote}
  To identify and classify the different types of non-classical logics
  that exist.
  \end{quote}
\item
  \begin{quote}
  To construct a subclassical graph of calculi that visually represents
  the relationships between these logics.
  \end{quote}
\item
  \begin{quote}
  To analyze the subclassical graph of calculi to identify common
  patterns and themes.
  \end{quote}
\item
  \begin{quote}
  To develop a comprehensive and unified understanding of the
  subclassical graph of calculi.
  \end{quote}
\item
  \begin{quote}
  To develop a new formal framework for representing and reasoning about
  subclassical logics.
  \end{quote}
\item
  \begin{quote}
  To develop a new classification of subclassical logics.
  \end{quote}
\item
  \begin{quote}
  To identify new relationships between existing subclassical logics.
  \end{quote}
\item
  \begin{quote}
  To develop a new set of tools for working with subclassical logics.
  \end{quote}
\end{itemize}

The objectives of the project are to:

\begin{itemize}
\item
  \begin{quote}
  \textbf{Identify and classify the different types of subclassical
  logics}. This will involve analyzing the axioms, rules, and semantics
  of different logics to determine their similarities and differences.
  \end{quote}
\item
  \begin{quote}
  \textbf{Construct the subclassical graph of calculi}. This will
  involve representing different subclassical logics as nodes in a graph
  and connecting them with edges that represent the relationships
  between them.
  \end{quote}
\item
  \begin{quote}
  \textbf{Analyze the subclassical graph of calculi to identify patterns
  and relationships}. This will involve using graph theory and other
  mathematical techniques to identify common structures and trends in
  the graph.
  \end{quote}
\item
  \begin{quote}
  \textbf{Develop new algorithms for reasoning about subclassical
  logics}. This will involve using the insights gained from the
  subclassical graph of calculi to design new algorithms that are more
  efficient and accurate for reasoning about subclassical logics.
  \end{quote}
\item
  \begin{quote}
  \textbf{Apply subclassical graphs to develop new automated reasoning
  systems}. This will involve using subclassical graphs to design new
  inference rules and strategies for automated reasoning.
  \end{quote}
\item
  \begin{quote}
  \textbf{Investigate the application of subclassical logics to new
  areas}. This will involve exploring the use of subclassical logics in
  areas such as artificial intelligence, computer science, and
  linguistics.
  \end{quote}
\end{itemize}

The measurements of the project are to:

\begin{itemize}
\item
  \begin{quote}
  \textbf{The number of subclassical logics that are identified and
  classified}.
  \end{quote}
\item
  \begin{quote}
  \textbf{The size and complexity of the subclassical graph of calculi}.
  \end{quote}
\item
  \begin{quote}
  \textbf{The number and significance of the patterns and relationships
  that are identified in the subclassical graph of calculi}.
  \end{quote}
\item
  \begin{quote}
  \textbf{The efficiency and accuracy of the new algorithms for
  reasoning about subclassical logics}.
  \end{quote}
\item
  \begin{quote}
  \textbf{The effectiveness of the new automated reasoning systems that
  are developed}.
  \end{quote}
\item
  \begin{quote}
  \textbf{The impact of the project on the development of new
  applications for subclassical logics}.
  \end{quote}
\end{itemize}

The outcomes of the project are to:

\begin{itemize}
\item
  \begin{quote}
  \textbf{A comprehensive and unified understanding of the subclassical
  graph of calculi}.
  \end{quote}
\item
  \begin{quote}
  \textbf{A new classification of subclassical logics}.
  \end{quote}
\item
  \begin{quote}
  \textbf{A new set of relationships between existing subclassical
  logics}.
  \end{quote}
\item
  \begin{quote}
  \textbf{New algorithms for reasoning about subclassical logics}.
  \end{quote}
\item
  \begin{quote}
  \textbf{New automated reasoning systems for subclassical logics}.
  \end{quote}
\item
  \begin{quote}
  \textbf{New applications for subclassical logics}.
  \end{quote}
\end{itemize}

\hypertarget{approachmethodology-v1}{%
\subsection{Approach/Methodology v1}\label{approachmethodology-v1}}

\textbf{1.}

\textbf{Timeline:}

The project is expected to be completed within a timeframe of
approximately six months, with the following breakdown:

\begin{itemize}
\item
  \begin{quote}
  Literature Review and Conceptual Framework Development: 1-2 months
  \end{quote}
\item
  \begin{quote}
  Metamathematical Analysis: 2-3 months
  \end{quote}
\item
  \begin{quote}
  Metalinguistic Investigation: 1-2 months
  \end{quote}
\item
  \begin{quote}
  Comparative Analysis and Conclusion: 1 month
  \end{quote}
\end{itemize}

Regular progress reports will be generated and shared to ensure
adherence to the timeline and identify any potential challenges or
roadblocks.

\hypertarget{approachmethodology-v2}{%
\subsection{Approach/Methodology v2}\label{approachmethodology-v2}}

This project will be carried out using a deductive approach, focusing
exclusively on metamathematical and metalinguistic methods. The
following activities are proposed to meet the goals and objectives of
the project:

\textbf{Activity 1: Literature Review}

A comprehensive literature review will be conducted to identify and
examine existing work on subclassical calculi, particularly in the areas
of metamathematics and metalinguistics. This will involve reviewing
relevant academic papers, books, and other resources.

\textbf{Activity 2: Graph-Theoretic Representation}

Subclassical calculi will be represented using graph theory. This will
involve defining the nodes and edges of the graphs, as well as the
relationships between them. The graph-theoretic representation will
allow for a visual and formal analysis of the calculi.

\textbf{Activity 3: Comparative Analysis}

The subclassical calculi will be compared and contrasted using the
graph-theoretic representation. This will involve identifying
similarities and differences between the calculi, as well as exploring
their relationships.

\textbf{Activity 4: Metamathematical and Metalinguistic Analysis}

The graph-theoretic representation will be used to carry out a
metamathematical and metalinguistic analysis of the subclassical
calculi. This will involve examining the properties of the calculi, as
well as their expressiveness and limitations.

\textbf{Activity 5: Synthesis and Conclusions}

The findings of the research will be synthesized and conclusions will be
drawn about the subclassical calculi. This will involve identifying key
insights and implications of the research, as well as proposing
directions for future research.

\textbf{Timeline}

The following timeline is proposed for the project:

\begin{itemize}
\item
  \begin{quote}
  \textbf{Month 1-3:} Literature review
  \end{quote}
\item
  \begin{quote}
  \textbf{Month 4-6:} Graph-theoretic representation
  \end{quote}
\item
  \begin{quote}
  \textbf{Month 7-9:} Comparative analysis
  \end{quote}
\item
  \begin{quote}
  \textbf{Month 10-12:} Metamathematical and metalinguistic analysis
  \end{quote}
\item
  \begin{quote}
  \textbf{Month 13-14:} Synthesis and conclusions
  \end{quote}
\end{itemize}

\textbf{Resources}

The following resources will be used to carry out the project:

\begin{itemize}
\item
  \begin{quote}
  Academic papers
  \end{quote}
\item
  \begin{quote}
  Books
  \end{quote}
\item
  \begin{quote}
  Online resources
  \end{quote}
\item
  \begin{quote}
  Graph theory software
  \end{quote}
\end{itemize}

\textbf{Evaluation}

The success of the project will be evaluated based on the following
criteria:

\begin{itemize}
\item
  \begin{quote}
  The comprehensiveness of the literature review
  \end{quote}
\item
  \begin{quote}
  The accuracy of the graph-theoretic representation
  \end{quote}
\item
  \begin{quote}
  The depth of the comparative analysis
  \end{quote}
\item
  \begin{quote}
  The rigor of the metamathematical and metalinguistic analysis
  \end{quote}
\item
  \begin{quote}
  The significance of the synthesis and conclusions
  \end{quote}
\end{itemize}

\textbf{Dissemination}

The findings of the research will be disseminated through a variety of
channels, including:

\begin{itemize}
\item
  \begin{quote}
  Academic papers
  \end{quote}
\item
  \begin{quote}
  Conference presentations
  \end{quote}
\item
  \begin{quote}
  Online publications
  \end{quote}
\end{itemize}

\hypertarget{approachmethodology-v3}{%
\subsection{Approach/Methodology v3}\label{approachmethodology-v3}}

\textbf{How are you going to carry out your project?}

The research project will be carried out using a deductive approach,
focusing exclusively on metamathematical and metalinguistic methods.
This means that the research will focus on the formal properties of
subclassical calculi, rather than their computational or semantic
properties.

The specific activities that will be carried out to meet the goals and
objectives of the project are as follows:

\begin{enumerate}
\def\labelenumi{\arabic{enumi}.}
\item
  \begin{quote}
  \textbf{Conduct a literature review} of existing work on subclassical
  calculi. This will involve identifying and reviewing relevant papers,
  books, and other resources.
  \end{quote}
\item
  \begin{quote}
  \textbf{Develop a formal framework} for representing subclassical
  calculi. This framework will be used to represent the syntax and
  semantics of subclassical calculi in a precise and unambiguous way.
  \end{quote}
\item
  \begin{quote}
  \textbf{Use the formal framework} to investigate the metamathematical
  properties of subclassical calculi. This will involve proving theorems
  about the expressiveness, decidability, and complexity of subclassical
  calculi.
  \end{quote}
\item
  \begin{quote}
  \textbf{Develop a metalinguistic approach} to reasoning about
  subclassical calculi. This will involve developing techniques for
  using metamathematical concepts to reason about the properties of
  subclassical calculi.
  \end{quote}
\end{enumerate}

\textbf{What specific activities do you propose to meet the goals and
objectives you have outlined, and how will those activities be carried
out?}

The specific activities that will be carried out to meet the goals and
objectives of the project are as follows:

\begin{itemize}
\item
  \begin{quote}
  \textbf{Goal 1:} To identify the key features of subclassical calculi.
  \end{quote}

  \begin{itemize}
  \item
    \begin{quote}
    \textbf{Activity 1.1:} Conduct a literature review of existing work
    on subclassical calculi.
    \end{quote}
  \item
    \begin{quote}
    \textbf{Activity 1.2:} Develop a formal framework for representing
    subclassical calculi.
    \end{quote}
  \item
    \begin{quote}
    \textbf{Activity 1.3:} Use the formal framework to identify the key
    features of subclassical calculi.
    \end{quote}
  \end{itemize}
\item
  \begin{quote}
  \textbf{Goal 2:} To investigate the metamathematical properties of
  subclassical calculi.
  \end{quote}

  \begin{itemize}
  \item
    \begin{quote}
    \textbf{Activity 2.1:} Use the formal framework to prove theorems
    about the expressiveness of subclassical calculi.
    \end{quote}
  \item
    \begin{quote}
    \textbf{Activity 2.2:} Use the formal framework to prove theorems
    about the decidability of subclassical calculi.
    \end{quote}
  \item
    \begin{quote}
    \textbf{Activity 2.3:} Use the formal framework to prove theorems
    about the complexity of subclassical calculi.
    \end{quote}
  \end{itemize}
\item
  \begin{quote}
  \textbf{Goal 3:} To develop a metalinguistic approach to reasoning
  about subclassical calculi.
  \end{quote}

  \begin{itemize}
  \item
    \begin{quote}
    \textbf{Activity 3.1:} Develop techniques for using metamathematical
    concepts to reason about the properties of subclassical calculi.
    \end{quote}
  \item
    \begin{quote}
    \textbf{Activity 3.2:} Apply the metalinguistic approach to
    reasoning about specific examples of subclassical calculi.
    \end{quote}
  \end{itemize}
\end{itemize}

\textbf{How will those activities be carried out?}

The activities will be carried out using a variety of methods,
including:

\begin{itemize}
\item
  \begin{quote}
  \textbf{Literature review:} The literature review will be conducted
  using a variety of sources, including academic journals, books, and
  online resources.
  \end{quote}
\item
  \begin{quote}
  \textbf{Formalization:} The formal framework for representing
  subclassical calculi will be developed using a variety of formal
  languages, including logic programming languages and proof assistants.
  \end{quote}
\item
  \begin{quote}
  \textbf{Theorem proving:} The theorems about the expressiveness,
  decidability, and complexity of subclassical calculi will be proved
  using a variety of techniques, including induction, model theory, and
  proof theory.
  \end{quote}
\item
  \begin{quote}
  \textbf{Metalinguistic analysis:} The metalinguistic approach to
  reasoning about subclassical calculi will be developed using a variety
  of metamathematical concepts, including reflection, abstraction, and
  self-reference.
  \end{quote}
\end{itemize}

\textbf{Timeline}

The research project is expected to take approximately 2 years to
complete. The following is a tentative timeline for the project:

\begin{itemize}
\item
  \begin{quote}
  \textbf{Year 1:\\
  }
  \end{quote}

  \begin{itemize}
  \item
    \begin{quote}
    Conduct literature review
    \end{quote}
  \item
    \begin{quote}
    Develop formal framework
    \end{quote}
  \end{itemize}
\item
  \begin{quote}
  \textbf{Year 2:\\
  }
  \end{quote}

  \begin{itemize}
  \item
    \begin{quote}
    Investigate metamathematical properties
    \end{quote}
  \item
    \begin{quote}
    Develop metalinguistic approach
    \end{quote}
  \item
    \begin{quote}
    Write dissertation
    \end{quote}
  \end{itemize}
\end{itemize}

The timeline is subject to change, and the actual time required to
complete the project may vary.

\textbf{Resources}

The research project will require a variety of resources, including:

\begin{itemize}
\item
  \begin{quote}
  \textbf{Access to academic journals and books:} The researcher will
  need access to a variety of academic journals and books in order to
  conduct the literature review.
  \end{quote}
\item
  \begin{quote}
  \textbf{Computing resources:} The researcher will need access to
  computing resources in order to develop the formal framework and prove
  theorems.
  \end{quote}
\item
  \begin{quote}
  \textbf{Travel funds:} The researcher may need to travel to
  conferences and workshops in order to present their work and
  collaborate with other researchers.
  \end{quote}
\end{itemize}

The researcher will apply for funding to support the project. The
funding will be used to cover the cost of resources and travel.

\hypertarget{timeline}{%
\subsection{Timeline}\label{timeline}}

\begin{itemize}
\item
  \begin{quote}
  Identify and classify the non-classical logics: 6 months.
  \end{quote}
\item
  \begin{quote}
  Construct the subclassical graph of calculi: 3 months.
  \end{quote}
\item
  \begin{quote}
  Analyze the subclassical graph of calculi: 2 months.
  \end{quote}
\item
  \begin{quote}
  Develop a new formal framework for representing and reasoning about
  non-classical logics: 4 months.
  \end{quote}
\item
  \begin{quote}
  Develop a new classification of non-classical logics: 3 months.
  \end{quote}
\item
  \begin{quote}
  Identify new relationships between existing non-classical logics: 2
  months.
  \end{quote}
\item
  \begin{quote}
  Develop a new set of tools for working with non-classical logics: 5
  months.
  \end{quote}
\end{itemize}

\hypertarget{outcomes-benefits-results-v1}{%
\subsection{Outcomes, Benefits, Results
v1}\label{outcomes-benefits-results-v1}}

\hypertarget{outcomes}{%
\subsubsection{Outcomes}\label{outcomes}}

The proposed research project aims to develop a subclassical graph of
calculi, a novel approach to formalizing and analyzing the relationships
between different logical calculi. The subclassical graph will represent
logical calculi as nodes and the relationships between them as edges,
where the edges capture the logical implications and syntactic
similarities between different systems.

The project is expected to have the following outcomes:

\begin{itemize}
\item
  \begin{quote}
  \textbf{Comprehensive and systematic framework for understanding the
  interplay of various logical systems}. The subclassical graph will
  provide a structured and interconnected representation of various
  subclassical systems. This graphical representation will facilitate
  the analysis of logical relationships, the identification of logical
  connections, and the exploration of new logical systems.
  \end{quote}
\item
  \begin{quote}
  \textbf{Improved understanding of calculi relationships}. The
  subclassical graph will serve as a valuable tool for visualizing and
  understanding the intricate relationships between different calculi,
  facilitating the discovery of new connections and patterns.
  \end{quote}
\item
  \begin{quote}
  \textbf{New techniques for interoperating and comparing calculi}. The
  subclassical graph will provide a foundation for developing new
  techniques for interoperating and comparing calculi, enabling the
  transfer of knowledge and tools across different formal systems.
  \end{quote}
\item
  \begin{quote}
  \textbf{Benchmark for evaluating the expressiveness and computational
  power of different calculi}. The subclassical graph will serve as a
  benchmark for evaluating the expressiveness and computational power of
  different calculi, aiding in the selection of appropriate formal
  systems for specific applications.
  \end{quote}
\end{itemize}

\textbf{Benefits}

The proposed research project is expected to have the following
benefits:

\begin{itemize}
\item
  \begin{quote}
  \textbf{Advance our understanding of subclassical logic and its
  applications}. The development of a formal framework for subclassical
  graphs will provide a valuable tool for researchers and practitioners
  alike, facilitating the analysis, manipulation, and application of
  these logics.
  \end{quote}
\item
  \begin{quote}
  \textbf{Develop novel logical calculi with tailored properties for
  specific applications}. The subclassical graph will serve as a
  generative framework for exploring new logical systems and their
  properties.
  \end{quote}
\item
  \begin{quote}
  \textbf{Contribute to the development of new artificial intelligence
  applications}. The subclassical graph can be used to develop new
  artificial intelligence applications that can reason under uncertainty
  and handle vagueness.
  \end{quote}
\item
  \begin{quote}
  \textbf{Enhance our understanding of computation and formal logic}.
  The subclassical graph can be used to develop new insights into the
  relationships between different formal systems and their computational
  properties.
  \end{quote}
\end{itemize}

\textbf{Results}

The proposed research project is expected to have the following results:

\begin{itemize}
\item
  \begin{quote}
  \textbf{A formalization of subclassical graphs}. The project will
  propose a rigorous formalization of subclassical graphs, establishing
  a precise mathematical foundation for their analysis and manipulation.
  \end{quote}
\item
  \begin{quote}
  \textbf{Algorithmic techniques for constructing and manipulating
  subclassical graphs}. Novel algorithmic techniques will be developed
  to efficiently construct and manipulate subclassical graphs, enabling
  the exploration of large and complex systems.
  \end{quote}
\item
  \begin{quote}
  \textbf{Applications of subclassical graphs in automated reasoning}.
  The project will investigate the application of subclassical graphs in
  automated reasoning systems, potentially leading to more powerful and
  versatile inference mechanisms.
  \end{quote}
\end{itemize}

\hypertarget{impact}{%
\subsubsection{Impact}\label{impact}}

The proposed research project has the potential to have a significant
impact on the field of logic and its applications. The development of a
formal framework for subclassical graphs will provide a valuable tool
for researchers and practitioners alike, facilitating the analysis,
manipulation, and application of these logics. Moreover, the
project\textquotesingle s exploration of algorithmic techniques and
applications in automated reasoning is expected to yield tangible
benefits in various domains, including artificial intelligence, computer
science, and decision-making systems.

\hypertarget{measurement}{%
\subsubsection{Measurement}\label{measurement}}

The success of the proposed research project will be measured by the
following criteria:

\begin{itemize}
\item
  \begin{quote}
  \textbf{The quality of the formalization of subclassical graphs}. The
  formalization should be rigorous, precise, and well-founded.
  \end{quote}
\item
  \begin{quote}
  \textbf{The efficiency and effectiveness of the algorithmic
  techniques}. The algorithms should be efficient in terms of
  computational time and memory usage, and they should be able to handle
  large and complex subclassical graphs.
  \end{quote}
\item
  \begin{quote}
  \textbf{The impact of the applications of subclassical graphs}. The
  applications should demonstrate the usefulness of subclassical graphs
  in solving real-world problems.
  \end{quote}
\end{itemize}

\hypertarget{products}{%
\subsubsection{Products}\label{products}}

The proposed research project will produce the following products:

\begin{itemize}
\item
  \begin{quote}
  \textbf{A formalization of subclassical graphs}. The formalization
  will be published in a peer-reviewed journal or conference
  proceedings.
  \end{quote}
\item
  \begin{quote}
  \textbf{Algorithmic techniques for constructing and manipulating
  subclassical graphs}. The algorithms will be implemented in a software
  library and made available for public use.
  \end{quote}
\item
  \begin{quote}
  \textbf{Applications of subclassical graphs in automated reasoning}.
  The applications will be demonstrated in a series of research papers.
  \end{quote}
\end{itemize}

\hypertarget{outcomes-benefits-results-v2}{%
\subsection{Outcomes, Benefits, Results
v2}\label{outcomes-benefits-results-v2}}

\hypertarget{the-project-will-deliver-the-following-outcomes.}{%
\subsubsection{The project will deliver the following
outcomes.}\label{the-project-will-deliver-the-following-outcomes.}}

\begin{itemize}
\item
  \begin{quote}
  A comprehensive list of non-classical logics, including their axioms,
  rules, and properties.
  \end{quote}
\item
  \begin{quote}
  A new formal framework for representing and reasoning about
  subclassical logics.
  \end{quote}
\item
  \begin{quote}
  A new classification of subclassical logics.
  \end{quote}
\item
  \begin{quote}
  A subclassical graph of calculi that visually represents the
  relationships between these logics.
  \end{quote}
\item
  \begin{quote}
  A list of common patterns and themes in the properties of
  non-classical logics.
  \end{quote}
\item
  \begin{quote}
  A list of new areas of research suggested by the analysis of the
  subclassical graph of calculi.
  \end{quote}
\end{itemize}

The project will also have the following benefits.

\begin{itemize}
\item
  \begin{quote}
  It will provide a better understanding of the logical relationships
  between different subclassical calculi.
  \end{quote}
\item
  \begin{quote}
  It will facilitate the development of new and more powerful
  subclassical calculi.
  \end{quote}
\item
  \begin{quote}
  It will enable the application of subclassical calculi to new areas.
  \end{quote}
\end{itemize}

The project will also produce the following results.

\begin{itemize}
\item
  \begin{quote}
  A new formal framework for representing and reasoning about
  subclassical logics.
  \end{quote}
\item
  \begin{quote}
  A new classification of subclassical logics.
  \end{quote}
\item
  \begin{quote}
  A subclassical graph of calculi that visually represents the
  relationships between these logics.
  \end{quote}
\item
  \begin{quote}
  A list of common patterns and themes in the properties of
  non-classical logics.
  \end{quote}
\item
  \begin{quote}
  A list of new areas of research suggested by the analysis of the
  subclassical graph of calculi.
  \end{quote}
\end{itemize}

\hypertarget{outcomes-benefits-results-v3}{%
\subsection{Outcomes, Benefits, Results
v3}\label{outcomes-benefits-results-v3}}

\hypertarget{outcomes-1}{%
\subsubsection{Outcomes}\label{outcomes-1}}

The development of the subclassical graph of calculi will have
significant implications for research in logic and its applications. It
will provide a comprehensive and systematic framework for understanding
the interplay of various logical systems, enabling researchers to:

\begin{itemize}
\item
  \begin{quote}
  Identify connections between seemingly disparate calculi.
  \end{quote}
\item
  \begin{quote}
  Uncover hidden relationships and patterns in the landscape of logical
  systems.
  \end{quote}
\item
  \begin{quote}
  Explore new avenues of research by traversing the subclassical graph.
  \end{quote}
\item
  \begin{quote}
  Develop novel logical calculi with tailored properties for specific
  applications.
  \end{quote}
\end{itemize}

\textbf{Benefits}

The development of the subclassical graph of calculi will provide
several benefits to the field of logic and computation. It will:

\begin{itemize}
\item
  \begin{quote}
  Facilitate the understanding and comparison of different logical
  systems.
  \end{quote}
\item
  \begin{quote}
  Aid in the development of new logical calculi.
  \end{quote}
\item
  \begin{quote}
  Enhance the application of logical calculi in various fields, such as
  artificial intelligence, computer science, and mathematics.
  \end{quote}
\end{itemize}

\textbf{Results}

The successful development of the subclassical graph of calculi will
result in several key outcomes:

\begin{itemize}
\item
  \begin{quote}
  A comprehensive and systematic framework for understanding the
  relationships between logical calculi.
  \end{quote}
\item
  \begin{quote}
  A set of tools for analyzing and visualizing the relationships between
  logical calculi.
  \end{quote}
\item
  \begin{quote}
  A collection of novel logical calculi with tailored properties for
  specific applications.
  \end{quote}
\end{itemize}

\hypertarget{impact-1}{%
\subsubsection{Impact}\label{impact-1}}

The development of the subclassical graph of calculi will have a
significant impact on the field of logic and its applications. It will:

\begin{itemize}
\item
  \begin{quote}
  Revolutionize our understanding of logical systems and their
  applications.
  \end{quote}
\item
  \begin{quote}
  Lead to advancements in mathematics, computer science, and philosophy.
  \end{quote}
\item
  \begin{quote}
  Enable the development of new and more powerful tools for reasoning
  and computation.
  \end{quote}
\end{itemize}

\hypertarget{measurement-1}{%
\subsubsection{Measurement}\label{measurement-1}}

The impact of the subclassical graph of calculi will be measured by:

\begin{itemize}
\item
  \begin{quote}
  The number of citations of the research papers that describe the
  development and application of the subclassical graph of calculi.
  \end{quote}
\item
  \begin{quote}
  The number of new logical calculi that are developed using the
  subclassical graph of calculi.
  \end{quote}
\item
  \begin{quote}
  The number of new applications of logical calculi that are developed
  using the subclassical graph of calculi.
  \end{quote}
\end{itemize}

\hypertarget{products-1}{%
\subsubsection{Products}\label{products-1}}

The products of this research will include:

\begin{itemize}
\item
  \begin{quote}
  A rigorous formalization of subclassical graphs.
  \end{quote}
\item
  \begin{quote}
  A set of algorithms for constructing and manipulating subclassical
  graphs.
  \end{quote}
\item
  \begin{quote}
  A collection of novel logical calculi that are developed using the
  subclassical graph of calculi.
  \end{quote}
\item
  \begin{quote}
  A series of research papers that describe the development and
  application of the subclassical graph of calculi.
  \end{quote}
\end{itemize}

\hypertarget{outcomes-benefits-results-v4}{%
\subsection{Outcomes, Benefits, Results
v4}\label{outcomes-benefits-results-v4}}

\hypertarget{outcomes-2}{%
\subsubsection{Outcomes}\label{outcomes-2}}

The proposed research project aims to develop a subclassical graph of
calculi, a novel approach to formalizing and analyzing the relationships
between different logical calculi. The subclassical graph will provide a
comprehensive and systematic framework for understanding the interplay
of various logical systems, enabling researchers to:

\begin{itemize}
\item
  \begin{quote}
  Identify connections between seemingly disparate calculi.
  \end{quote}
\item
  \begin{quote}
  Uncover hidden relationships and patterns in the landscape of logical
  systems.
  \end{quote}
\item
  \begin{quote}
  Explore new avenues of research by traversing the subclassical graph.
  \end{quote}
\item
  \begin{quote}
  Develop novel logical calculi with tailored properties for specific
  applications.
  \end{quote}
\end{itemize}

\textbf{Benefits}

The development of the subclassical graph of calculi will have
significant implications for research in logic and its applications. It
will provide a valuable tool for researchers and practitioners alike,
facilitating the analysis, manipulation, and application of these
logics. Moreover, the project\textquotesingle s exploration of
algorithmic techniques and applications in automated reasoning is
expected to yield tangible benefits in various domains, including
artificial intelligence, computer science, and decision-making systems.

\textbf{Results}

The project will produce a number of significant results, including:

\begin{itemize}
\item
  \begin{quote}
  A rigorous formalization of subclassical graphs.
  \end{quote}
\item
  \begin{quote}
  Novel algorithmic techniques for constructing and manipulating
  subclassical graphs.
  \end{quote}
\item
  \begin{quote}
  A comprehensive survey of existing subclassical logics and their
  relationships.
  \end{quote}
\item
  \begin{quote}
  The identification of new connections and patterns in the landscape of
  logical systems.
  \end{quote}
\item
  \begin{quote}
  The development of novel logical calculi with tailored properties for
  specific applications.
  \end{quote}
\end{itemize}

\hypertarget{impact-2}{%
\subsubsection{Impact}\label{impact-2}}

The proposed research has the potential to revolutionize our
understanding of logical systems and their applications. It will provide
a valuable tool for researchers and practitioners alike, facilitating
the analysis, manipulation, and application of these logics. Moreover,
the project\textquotesingle s exploration of algorithmic techniques and
applications in automated reasoning is expected to yield tangible
benefits in various domains, including artificial intelligence, computer
science, and decision-making systems.

\hypertarget{measurements}{%
\subsubsection{Measurements}\label{measurements}}

The success of the project will be measured by the following criteria:

\begin{itemize}
\item
  \begin{quote}
  The quality of the subclassical graph of calculi, as determined by
  experts in the field of logic.
  \end{quote}
\item
  \begin{quote}
  The effectiveness of the algorithmic techniques for constructing and
  manipulating subclassical graphs.
  \end{quote}
\item
  \begin{quote}
  The completeness and accuracy of the survey of existing subclassical
  logics and their relationships.
  \end{quote}
\item
  \begin{quote}
  The number of new connections and patterns identified in the landscape
  of logical systems.
  \end{quote}
\item
  \begin{quote}
  The development of novel logical calculi with tailored properties for
  specific applications.
  \end{quote}
\end{itemize}

\hypertarget{products-2}{%
\subsubsection{Products}\label{products-2}}

The project will produce the following products:

\begin{itemize}
\item
  \begin{quote}
  A rigorous formalization of subclassical graphs.
  \end{quote}
\item
  \begin{quote}
  Novel algorithmic techniques for constructing and manipulating
  subclassical graphs.
  \end{quote}
\item
  \begin{quote}
  A comprehensive survey of existing subclassical logics and their
  relationships.
  \end{quote}
\item
  \begin{quote}
  A number of research papers published in peer-reviewed journals.
  \end{quote}
\item
  \begin{quote}
  A presentation at a major international conference in logic.
  \end{quote}
\item
  \begin{quote}
  A software tool for visualizing and analyzing subclassical graphs.
  \end{quote}
\end{itemize}

\hypertarget{project-directorprincipal-investigator-and-staff-v1}{%
\subsection{Project Director/Principal Investigator and Staff
v1}\label{project-directorprincipal-investigator-and-staff-v1}}

\textbf{Qualifications and Experience of the Proposed Project
Director/Principal Investigator}

The Project Director/Principal Investigator (PD/PI) should have a strong
background in subclassical logic and experience in leading research
projects. The ideal candidate will have a Ph.D. in a relevant field,
such as mathematics, computer science, or logic, and a proven track
record of publishing research papers in top-tier journals. The PD/PI
should also have experience in collaborating with other researchers and
in mentoring students.

\textbf{Qualifications and Experience of Key Project Staff}

The key project staff should have expertise in subclassical logic and
related areas, such as proof theory, model theory, and computer science.
The ideal candidates will have Ph.D.s in relevant fields and experience
in conducting research on subclassical logic. The key project staff
should also be able to work independently and as part of a team.

\textbf{Minimum Ideal Qualifications for the Principal Investigator}

The minimum ideal qualifications for the PI are a
master\textquotesingle s degree in a relevant field, such as
mathematics, computer science, or logic, and a strong track record of
research in subclassical logic. The PI should also have experience in
leading research projects and in mentoring students.

\textbf{Ideal Project Director or Project Directors}

The ideal project director or project directors would be determined by
the relevant International learned societies and unions. These
organizations would be able to identify individuals with the expertise
and experience necessary to lead this research project.

\textbf{Additional Considerations}

In addition to the qualifications and experience listed above, the PD/PI
and key project staff should also have the following qualities:

\begin{itemize}
\item
  \begin{quote}
  Strong communication and interpersonal skills
  \end{quote}
\item
  \begin{quote}
  Ability to think creatively and solve problems
  \end{quote}
\item
  \begin{quote}
  Ability to work independently and as part of a team
  \end{quote}
\item
  \begin{quote}
  Commitment to excellence in research
  \end{quote}
\end{itemize}

We are confident that with a strong team of researchers in place, this
project will make significant contributions to our understanding of
subclassical logic.

\hypertarget{project-directorprincipal-investigator-and-staff-v2}{%
\subsection{Project Director/Principal Investigator and Staff
v2}\label{project-directorprincipal-investigator-and-staff-v2}}

\hypertarget{principal-investigator}{%
\subsubsection{Principal Investigator}\label{principal-investigator}}

\textbf{Qualifications and Experience of the Proposed Project
Director/Principal Investigator}

\begin{itemize}
\item
  \begin{quote}
  Master of Science in Mathematics from {[}University Name{]},
  {[}Year{]}
  \end{quote}
\item
  \begin{quote}
  Doctor of Philosophy in Mathematics from {[}University Name{]},
  {[}Year{]}
  \end{quote}
\item
  \begin{quote}
  Postdoctoral Fellowship in Mathematics at {[}University Name{]},
  {[}Year{]} - {[}Year{]}
  \end{quote}
\item
  \begin{quote}
  Research Scientist at {[}Research Institution{]}, {[}Year{]} - Present
  \end{quote}
\item
  \begin{quote}
  Author of over {[}Number{]} peer-reviewed publications in top
  mathematics journals
  \end{quote}
\item
  \begin{quote}
  Extensive experience in the field of subclassical logic
  \end{quote}
\end{itemize}

\textbf{Ideal Qualifications for the Principle Investigator}

\begin{itemize}
\item
  \begin{quote}
  Master\textquotesingle s degree in mathematics, computer science, or
  philosophy
  \end{quote}
\item
  \begin{quote}
  Proven experience in research on subclassical logic
  \end{quote}
\item
  \begin{quote}
  Strong publication record in top academic journals
  \end{quote}
\item
  \begin{quote}
  Excellent communication and collaboration skills
  \end{quote}
\end{itemize}

\textbf{Ideal Selection Process for the Project Director or Project
Directors}

\begin{itemize}
\item
  \begin{quote}
  The ideal project director or project directors would be determined by
  a panel of experts from the relevant international learned societies
  and unions.
  \end{quote}
\item
  \begin{quote}
  The panel would consider the following factors when making its
  decision:
  \end{quote}

  \begin{itemize}
  \item
    \begin{quote}
    The qualifications and experience of the candidates
    \end{quote}
  \item
    \begin{quote}
    The relevance of the candidates\textquotesingle{} research to the
    project
    \end{quote}
  \item
    \begin{quote}
    The candidates\textquotesingle{} ability to lead and manage a
    large-scale research project
    \end{quote}
  \end{itemize}
\end{itemize}

\hypertarget{key-staff}{%
\subsubsection{Key Staff}\label{key-staff}}

\textbf{Qualifications and Experience of Key Project Staff}

\begin{itemize}
\item
  \begin{quote}
  {[}Name{]}, PhD in Mathematics, {[}Year{]}
  \end{quote}

  \begin{itemize}
  \item
    \begin{quote}
    Expertise in intuitionistic logic
    \end{quote}
  \item
    \begin{quote}
    Author of over {[}Number{]} peer-reviewed publications in top
    mathematics journals
    \end{quote}
  \end{itemize}
\item
  \begin{quote}
  {[}Name{]}, PhD in Computer Science, {[}Year{]}
  \end{quote}

  \begin{itemize}
  \item
    \begin{quote}
    Expertise in linear logic
    \end{quote}
  \item
    \begin{quote}
    Author of over {[}Number{]} peer-reviewed publications in top
    computer science journals
    \end{quote}
  \end{itemize}
\item
  \begin{quote}
  {[}Name{]}, PhD in Philosophy, {[}Year{]}
  \end{quote}

  \begin{itemize}
  \item
    \begin{quote}
    Expertise in the philosophy of mathematics
    \end{quote}
  \item
    \begin{quote}
    Author of over {[}Number{]} peer-reviewed publications in top
    philosophy journals
    \end{quote}
  \end{itemize}
\end{itemize}

\textbf{Additional Considerations}

\begin{itemize}
\item
  \begin{quote}
  The project director or project directors should be based at a major
  research university with strong research facilities in mathematics,
  computer science, or philosophy.
  \end{quote}
\item
  \begin{quote}
  The project director or project directors should have access to a
  network of collaborators who are experts in subclassical logic.
  \end{quote}
\end{itemize}

\hypertarget{project-directorprincipal-investigator-and-staff-v3}{%
\subsection{Project Director/Principal Investigator and Staff
v3}\label{project-directorprincipal-investigator-and-staff-v3}}

\hypertarget{project-director}{%
\subsubsection{Project Director}\label{project-director}}

\hypertarget{principal-investigator-1}{%
\subsubsection{Principal Investigator}\label{principal-investigator-1}}

\begin{itemize}
\item
  \begin{quote}
  \textbf{Qualifications:\\
  }
  \end{quote}

  \begin{itemize}
  \item
    \begin{quote}
    Master\textquotesingle s degree in mathematics, computer science, or
    a related field
    \end{quote}
  \item
    \begin{quote}
    5+ years of experience in research on subclassical logics
    \end{quote}
  \item
    \begin{quote}
    Proven track record of publishing in top-tier conferences and
    journals
    \end{quote}
  \item
    \begin{quote}
    Experience in leading and managing research projects
    \end{quote}
  \end{itemize}
\item
  \begin{quote}
  \textbf{Experience:\\
  }
  \end{quote}

  \begin{itemize}
  \item
    \begin{quote}
    Has published extensively on subclassical logics, including
    intuitionistic logic, counter-intuitionistic logic, and linear
    logic.
    \end{quote}
  \item
    \begin{quote}
    Has led several successful research projects on subclassical logics.
    \end{quote}
  \item
    \begin{quote}
    Has served on the program committee for several top-tier conferences
    on subclassical logics.
    \end{quote}
  \end{itemize}
\end{itemize}

\hypertarget{key-project-staff}{%
\subsubsection{Key Project Staff}\label{key-project-staff}}

Ian D.L.N. Mclean, dedicated personal library of relevant research and
several decades of self-motivated and self-driven study.

James Martin, \ldots,

\begin{itemize}
\item
  \begin{quote}
  \textbf{Qualifications:\\
  }
  \end{quote}

  \begin{itemize}
  \item
    \begin{quote}
    Bachelor\textquotesingle s degree in mathematics, computer science,
    or a related field
    \end{quote}
  \item
    \begin{quote}
    2+ years of experience in research on subclassical logics
    \end{quote}
  \item
    \begin{quote}
    Experience in programming and software development
    \end{quote}
  \end{itemize}
\item
  \begin{quote}
  \textbf{Experience:\\
  }
  \end{quote}

  \begin{itemize}
  \item
    \begin{quote}
    Has experience in programming and software development.
    \end{quote}
  \item
    \begin{quote}
    Has experience in conducting research on subclassical logics.
    \end{quote}
  \item
    \begin{quote}
    Has experience in collaborating with other researchers.
    \end{quote}
  \end{itemize}
\end{itemize}

\textbf{Ideal Project Director or Project Directors}

The ideal project director or project directors would be determined by
the relevant International learned societies and unions. These societies
and unions would be able to identify individuals with the necessary
expertise and experience to lead this project to success.

In addition to the qualifications and experience listed above, the ideal
project director or project directors would also have the following
qualities:

\begin{itemize}
\item
  \begin{quote}
  Strong leadership skills
  \end{quote}
\item
  \begin{quote}
  Excellent communication and interpersonal skills
  \end{quote}
\item
  \begin{quote}
  Ability to work independently and as part of a team
  \end{quote}
\item
  \begin{quote}
  Commitment to excellence
  \end{quote}
\end{itemize}

\hypertarget{project-directorprincipal-investigator-and-staff-v4}{%
\subsection{Project Director/Principal Investigator and Staff
v4}\label{project-directorprincipal-investigator-and-staff-v4}}

\textbf{Qualifications}

\begin{itemize}
\item
  \begin{quote}
  Master\textquotesingle s degree in a relevant field, such as
  mathematics, computer science, logic, or philosophy
  \end{quote}
\item
  \begin{quote}
  Proven experience in conducting research in subclassical logic
  \end{quote}
\item
  \begin{quote}
  Strong understanding of graph theory, lattice theory, sequent calculi,
  metalinguistics, metamathematics, formal language theory, constructive
  logic, paraconsistent and paracomplete methods of reasoning,
  foundations of non-classical logic, Sambin\textquotesingle s Basic
  Logic, hypersequent calculi, deviant logics, philosophy of
  mathematics, philosophy of logic, philosophy of language, philosophy
  of paraconsistency, philosophy of paracompleteness, philosophy of
  metamathematics, philosophy of metalinguistics, philosophy of
  semantics, and formal deductive theories
  \end{quote}
\item
  \begin{quote}
  Excellent communication and writing skills
  \end{quote}
\end{itemize}

\textbf{Experience}

\begin{itemize}
\item
  \begin{quote}
  Published at least five peer-reviewed articles in top academic
  journals
  \end{quote}
\item
  \begin{quote}
  Presented research at major conferences
  \end{quote}
\item
  \begin{quote}
  Supervised graduate students
  \end{quote}
\end{itemize}

\textbf{Key Project Staff}

\textbf{Qualifications}

\begin{itemize}
\item
  \begin{quote}
  Bachelor\textquotesingle s degree in a relevant field, such as
  mathematics, computer science, logic, or philosophy
  \end{quote}
\item
  \begin{quote}
  Experience with subclassical logic
  \end{quote}
\item
  \begin{quote}
  Strong understanding of graph theory, lattice theory, sequent calculi,
  metalinguistics, metamathematics, formal language theory, constructive
  logic, paraconsistent and paracomplete methods of reasoning,
  foundations of non-classical logic, Sambin\textquotesingle s Basic
  Logic, hypersequent calculi, deviant logics, philosophy of
  mathematics, philosophy of logic, philosophy of language, philosophy
  of paraconsistency, philosophy of paracompleteness, philosophy of
  metamathematics, philosophy of metalinguistics, philosophy of
  semantics, and formal deductive theories
  \end{quote}
\item
  \begin{quote}
  Excellent communication and writing skills
  \end{quote}
\end{itemize}

\textbf{Experience}

\begin{itemize}
\item
  \begin{quote}
  Conducted research in subclassical logic
  \end{quote}
\item
  \begin{quote}
  Presented research at conferences
  \end{quote}
\item
  \begin{quote}
  Assisted with the preparation of manuscripts for publication
  \end{quote}
\end{itemize}

\textbf{Ideal Project Director or Project Directors}

The ideal project director or project directors would be determined by
the relevant International learned societies and unions. These societies
and unions would have a deep understanding of the field of subclassical
logic and would be able to identify individuals with the necessary
expertise and qualifications to lead the project.

\textbf{Relevant Fields}

\begin{itemize}
\item
  \begin{quote}
  Graph theory
  \end{quote}
\item
  \begin{quote}
  Lattice theory
  \end{quote}
\item
  \begin{quote}
  Sequent calculi
  \end{quote}
\item
  \begin{quote}
  Metalinguistics
  \end{quote}
\item
  \begin{quote}
  Metamathematics
  \end{quote}
\item
  \begin{quote}
  Formal language theory
  \end{quote}
\item
  \begin{quote}
  Constructive logic
  \end{quote}
\item
  \begin{quote}
  Paraconsistent and paracomplete methods of reasoning
  \end{quote}
\item
  \begin{quote}
  Foundations of non-classical logic
  \end{quote}
\item
  \begin{quote}
  Sambin\textquotesingle s Basic Logic
  \end{quote}
\item
  \begin{quote}
  Hypersequent calculi
  \end{quote}
\item
  \begin{quote}
  Deviant logics
  \end{quote}
\item
  \begin{quote}
  Philosophy of mathematics
  \end{quote}
\item
  \begin{quote}
  Philosophy of logic
  \end{quote}
\item
  \begin{quote}
  Philosophy of language
  \end{quote}
\item
  \begin{quote}
  Philosophy of paraconsistency
  \end{quote}
\item
  \begin{quote}
  Philosophy of paracompleteness
  \end{quote}
\item
  \begin{quote}
  Philosophy of metamathematics
  \end{quote}
\item
  \begin{quote}
  Philosophy of metalinguistics
  \end{quote}
\item
  \begin{quote}
  Philosophy of semantics
  \end{quote}
\item
  \begin{quote}
  Formal deductive theories
  \end{quote}
\end{itemize}

\hypertarget{project-directorprincipal-investigator-and-staff-v5}{%
\subsection{Project Director/Principal Investigator and Staff
v5}\label{project-directorprincipal-investigator-and-staff-v5}}

\textbf{Project Director/Principal Investigator}

\begin{itemize}
\item
  \begin{quote}
  \textbf{Qualifications\\
  }
  \end{quote}

  \begin{itemize}
  \item
    \begin{quote}
    PhD in Mathematics, with a focus in graph theory or lattice theory
    \end{quote}
  \item
    \begin{quote}
    Experience in research on subclassical logics
    \end{quote}
  \item
    \begin{quote}
    Strong publication record in relevant journals
    \end{quote}
  \item
    \begin{quote}
    Experience in leading and managing research projects
    \end{quote}
  \end{itemize}
\item
  \begin{quote}
  \textbf{Experience\\
  }
  \end{quote}

  \begin{itemize}
  \item
    \begin{quote}
    10 years of experience as a research scientist
    \end{quote}
  \item
    \begin{quote}
    5 years of experience as a project leader
    \end{quote}
  \item
    \begin{quote}
    Authored or co-authored over 20 peer-reviewed articles on
    subclassical logics
    \end{quote}
  \item
    \begin{quote}
    Presented research at numerous international conferences
    \end{quote}
  \end{itemize}
\end{itemize}

\textbf{Key Project Staff}

\begin{itemize}
\item
  \begin{quote}
  \textbf{Research Scientist\\
  }
  \end{quote}

  \begin{itemize}
  \item
    \begin{quote}
    PhD in Mathematics, with a focus in graph theory or lattice theory
    \end{quote}
  \item
    \begin{quote}
    Experience in research on subclassical logics
    \end{quote}
  \item
    \begin{quote}
    Strong publication record in relevant journals
    \end{quote}
  \end{itemize}
\item
  \begin{quote}
  \textbf{Research Scientist\\
  }
  \end{quote}

  \begin{itemize}
  \item
    \begin{quote}
    PhD in Computer Science, with a focus in formal language theory
    \end{quote}
  \item
    \begin{quote}
    Experience in research on subclassical logics
    \end{quote}
  \item
    \begin{quote}
    Strong programming skills
    \end{quote}
  \end{itemize}
\item
  \begin{quote}
  \textbf{Research Assistant\\
  }
  \end{quote}

  \begin{itemize}
  \item
    \begin{quote}
    Master\textquotesingle s degree in Mathematics or Computer Science
    \end{quote}
  \item
    \begin{quote}
    Experience with graph theory or lattice theory
    \end{quote}
  \item
    \begin{quote}
    Strong programming skills
    \end{quote}
  \end{itemize}
\end{itemize}

\textbf{Ideal Qualifications for the Principal Investigator}

\begin{itemize}
\item
  \begin{quote}
  Master\textquotesingle s degree in the relevant fields and domains
  \end{quote}
\item
  \begin{quote}
  Expertise in graph theory, lattice theory, sequent calculi,
  metalinguistics, metamathematics, formal language theory, constructive
  logic, paraconsistent and paracomplete methods of reasoning,
  foundations of non-classical logic, Sambin\textquotesingle s Basic
  Logic, hypersequent calculi, deviant logics, philosophy of
  mathematics, philosophy of logic, philosophy of language, philosophy
  of paraconsistency, philosophy of paracompleteness, philosophy of
  metamathematics, philosophy of metalinguistics, philosophy of
  semantics, and formal deductive theories
  \end{quote}
\item
  \begin{quote}
  Experience in leading and managing research projects
  \end{quote}
\item
  \begin{quote}
  Strong publication record in relevant journals
  \end{quote}
\end{itemize}

\textbf{Ideal Project Director or Project Directors}

The ideal project director or project directors would be determined by
the relevant International learned societies and unions. These societies
and unions would be able to identify researchers with the necessary
expertise and experience to lead this project.

Please note that these are just ideal qualifications, and it is possible
that a project director or project directors with slightly different
qualifications could be successful in leading this project. The most
important factor is that the project director or project directors have
the necessary expertise and experience to lead the project to a
successful conclusion.

\hypertarget{project-directorprincipal-investigator-and-staff-v6}{%
\subsection{Project Director/Principal Investigator and Staff
v6}\label{project-directorprincipal-investigator-and-staff-v6}}

\textbf{Project Director/Principal Investigator}

\begin{itemize}
\item
  \begin{quote}
  \textbf{Qualifications:\\
  }
  \end{quote}

  \begin{itemize}
  \item
    \begin{quote}
    Master\textquotesingle s degree in mathematics, computer science, or
    a related field
    \end{quote}
  \item
    \begin{quote}
    Extensive experience in graph theory, lattice theory, sequent
    calculi, metalinguistics, metamathematics, formal language theory,
    constructive logic, paraconsistent and paracomplete methods of
    reasoning, foundations of non-classical logic,
    Sambin\textquotesingle s Basic Logic, hypersequent calculi, deviant
    logics, philosophy of mathematics, philosophy of logic, philosophy
    of language, philosophy of paraconsistency, philosophy of
    paracompleteness, philosophy of metamathematics, philosophy of
    metalinguistics, philosophy of semantics, and formal deductive
    theories
    \end{quote}
  \item
    \begin{quote}
    Proven ability to lead and manage research projects
    \end{quote}
  \item
    \begin{quote}
    Strong publication record in top-tier journals and conferences
    \end{quote}
  \end{itemize}
\item
  \begin{quote}
  \textbf{Experience:\\
  }
  \end{quote}

  \begin{itemize}
  \item
    \begin{quote}
    10+ years of experience in research and teaching
    \end{quote}
  \item
    \begin{quote}
    Experience in developing and leading research teams
    \end{quote}
  \item
    \begin{quote}
    Experience in securing and managing research grants
    \end{quote}
  \item
    \begin{quote}
    Experience in mentoring junior researchers
    \end{quote}
  \end{itemize}
\end{itemize}

\textbf{Key Project Staff}

\begin{itemize}
\item
  \begin{quote}
  \textbf{Qualifications:\\
  }
  \end{quote}

  \begin{itemize}
  \item
    \begin{quote}
    Bachelor\textquotesingle s degree in mathematics, computer science,
    or a related field
    \end{quote}
  \item
    \begin{quote}
    Strong background in graph theory, lattice theory, sequent calculi,
    metalinguistics, metamathematics, formal language theory,
    constructive logic, paraconsistent and paracomplete methods of
    reasoning, foundations of non-classical logic,
    Sambin\textquotesingle s Basic Logic, hypersequent calculi, deviant
    logics, philosophy of mathematics, philosophy of logic, philosophy
    of language, philosophy of paraconsistency, philosophy of
    paracompleteness, philosophy of metamathematics, philosophy of
    metalinguistics, philosophy of semantics, and formal deductive
    theories
    \end{quote}
  \item
    \begin{quote}
    Experience in research and teaching
    \end{quote}
  \end{itemize}
\item
  \begin{quote}
  \textbf{Experience:\\
  }
  \end{quote}

  \begin{itemize}
  \item
    \begin{quote}
    5+ years of experience in research
    \end{quote}
  \item
    \begin{quote}
    Experience in working on research teams
    \end{quote}
  \item
    \begin{quote}
    Experience in presenting research at conferences
    \end{quote}
  \end{itemize}
\end{itemize}

\textbf{Ideal Project Director or Project Directors}

The ideal project director or project directors would be determined by
the relevant International learned societies and unions. These societies
and unions would be able to identify researchers with the necessary
qualifications and experience to lead the project.

\textbf{Relevant Fields}

The relevant fields for this project include:

\begin{itemize}
\item
  \begin{quote}
  Graph theory
  \end{quote}
\item
  \begin{quote}
  Lattice theory
  \end{quote}
\item
  \begin{quote}
  Sequent calculi
  \end{quote}
\item
  \begin{quote}
  Metalinguistics
  \end{quote}
\item
  \begin{quote}
  Metamathematics
  \end{quote}
\item
  \begin{quote}
  Formal language theory
  \end{quote}
\item
  \begin{quote}
  Constructive logic
  \end{quote}
\item
  \begin{quote}
  Paraconsistent and paracomplete methods of reasoning
  \end{quote}
\item
  \begin{quote}
  Foundations of non-classical logic
  \end{quote}
\item
  \begin{quote}
  Sambin\textquotesingle s Basic Logic
  \end{quote}
\item
  \begin{quote}
  Hypersequent calculi
  \end{quote}
\item
  \begin{quote}
  Deviant logics
  \end{quote}
\item
  \begin{quote}
  Philosophy of mathematics
  \end{quote}
\item
  \begin{quote}
  Philosophy of logic
  \end{quote}
\item
  \begin{quote}
  Philosophy of language
  \end{quote}
\item
  \begin{quote}
  Philosophy of paraconsistency
  \end{quote}
\item
  \begin{quote}
  Philosophy of paracompleteness
  \end{quote}
\item
  \begin{quote}
  Philosophy of metamathematics
  \end{quote}
\item
  \begin{quote}
  Philosophy of metalinguistics
  \end{quote}
\item
  \begin{quote}
  Philosophy of semantics
  \end{quote}
\item
  \begin{quote}
  Formal deductive theories
  \end{quote}
\end{itemize}

\hypertarget{research-proposal-subclassical-graph-of-calculi}{%
\subsection{Research Proposal: Subclassical Graph of
Calculi}\label{research-proposal-subclassical-graph-of-calculi}}

\textbf{Introduction}

The field of logic has witnessed significant advancements in recent
decades, fueled by the development of powerful formal methods and
automated reasoning techniques. However, the vast and diverse landscape
of logical systems poses challenges in identifying connections and
overarching principles. Existing approaches to classifying and comparing
logical calculi often rely on ad hoc criteria, leading to a fragmented
and incomplete understanding of the relationships between different
systems.

To address these limitations, this research proposal outlines the
development of a subclassical graph of calculi, a novel approach to
formalizing and analyzing the relationships between different logical
calculi. The proposed subclassical graph will provide a comprehensive
and systematic framework for understanding the interplay of various
logical systems, enabling researchers to identify connections, uncover
hidden relationships, generate novel calculi, and explore new avenues of
research.

\textbf{Context and Background}

Logical calculi are formal systems for reasoning and deriving
conclusions from a set of axioms or assumptions or rules such as
structural or inferential rules. They play a fundamental role in
mathematics, computer science, and philosophy, providing a rigorous
foundation for reasoning, proof construction, refutation construction,
modeling, and countermodeling. Over the centuries, a vast array of
logical calculi have been developed, each tailored to specific
applications and embodying distinct logical properties.

\textbf{Current State of Research}

The study of logical calculi has witnessed significant advancements in
recent decades, fueled by the development of powerful formal methods and
automated reasoning techniques. However, the vast and diverse landscape
of logical systems poses challenges in identifying connections and
overarching principles. Existing approaches to classifying and comparing
logical calculi often rely on ad hoc criteria, leading to a fragmented
and incomplete understanding of the relationships between different
systems.

\textbf{Unique and Innovative Approach}

The proposed subclassical graph of calculi addresses these limitations
by providing a unified, systematic, and rigorous framework for analyzing
the relationships between logical calculi. The subclassical graph will
represent logical calculi as nodes and the relationships between them as
edges, where the edges capture the logical implications and syntactic
similarities between different systems.

The primary method of constructing and deducing the subclassical calculi
is applications of symmetry and non-unique dualities between calculi;
this allows for a general method of producing at least four supercalculi
and subcalculi from any one of them and some number of known calculi
such as classical logic (LK), intuitionistic logic (LJ), Dual
Intuitionistic Logic (LDJ), Counter Intuitionistic Logic (CoLJ),
Sambin's Basic, U, and Ordered Multiplicative-Additive Linear Logic
(OMALL).

\textbf{Significance and Contributions}

The development of the subclassical graph of calculi will have
significant implications for research in logic and its applications. It
will provide a comprehensive and systematic framework for understanding
the interplay of various logical systems, enabling researchers to:

\begin{itemize}
\item
  \begin{quote}
  Identify connections between seemingly disparate calculi.
  \end{quote}
\item
  \begin{quote}
  Uncover hidden relationships and patterns in the landscape of logical
  systems.
  \end{quote}
\item
  \begin{quote}
  Explore new avenues of research by traversing the subclassical graph.
  \end{quote}
\item
  \begin{quote}
  Develop novel logical calculi with tailored properties for specific
  applications.
  \end{quote}
\end{itemize}

The subclassical graph of calculi has the potential to revolutionize our
understanding of logical systems and their applications, leading to
advancements in mathematics, computer science, and philosophy.

\textbf{Conclusion}

The proposed research project aims to develop a comprehensive framework
for constructing and analyzing subclassical graphs of calculi, shedding
light on their inherent properties and applications in various fields.
The project will make significant contributions to the field of
subclassical logic and computation by providing a valuable tool for
researchers and practitioners alike, facilitating the analysis,
manipulation, and application of these logics. Moreover, the
project\textquotesingle s exploration of algorithmic techniques and
applications in automated reasoning is expected to yield tangible
benefits in various domains, including artificial intelligence, computer
science, and decision-making systems.

\hypertarget{section}{%
\section{}\label{section}}

}*}
