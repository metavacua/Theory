You can redefine the paradigm on your own terms, but its difficulty is
based on your environment. What it comes down to is agreement on first
principles and methods. The US has effectively unilaterally rejected
intellectual enterprise and everything with it while retaining a
facsimile. Which means rejecting scientific and deductive method.
Rejecting the idea of method itself. Rejecting reasoning by principles.
Rejecting reasoning. The US thus is tending to become a country that is
in principle unreasonable.

With a problem at the root like that, I can not thrive here. I am in
fundamental opposition to the extremes that represents.

The extremes that leads to. It means our culture is tending towards an
anything goes country. Where all opinions, feelings, emotions, thoughts,
perceptions, and actions are treated as valid. Where everything has
trivial value and is absolutely paradoxically true and false.
There\textquotesingle s one case in which the absolute extreme of
consistency makes complete sense: the finite case.

Finite systems which are consistent and explosive can be complete.
That\textquotesingle s what first order predicate logic is. Beyond that,
explosive consistency makes semi-complete sense up to countability and
in discrete systems. Those systems which are countably infinite. Beyond
that, inconsistencies abound at the top and bottom limits of continuous
systems. Manifesting around things like singularities. Explosive
consistency tends towards absolute contradiction at the limits of any
formal system. To me, this amount to saying that if consistency is like
a laser beam aimed at a prism, the limits of the system are that prism
and they decohere the beam into random noise. Shattering any sense of
reason we\textquotesingle d understand and accept.

In that regard, I think things like quantum mechanics and relativity
have done tremendous damage. For many people who take the theories as
literally, finally, absolute, and completely true, they shatter the
suspension of disbelief, and by proxy they shatter certainty and
confidence in any system of reason human beings can devise.

The probability of a erroneous argument going to absolute contradiction
is 1 as the argument approaches infinity.

``Suppose we have a system which is contradictory. Suppose the system
appears to be consistent for any finite domain. Precisely stated suppose
the system approximates a consistent system for any finite domain. Let
f(x) be a non-finite continuous function. Can we show that the system
approaches an absolutely contradictory system at the limits?''

Another example of a self-referential paradox, a recursive paradox: the
grand-father paradox and the bootstrap paradox. General relativity
predicts backward time travel leading to contradictions.

Heisenberg uncertainty is intrinsic error in the hard measurement of a
physical system.

General relativity ensures conservation up to parity. If our universe
exists in a bubble, a blackhole, we can expect that the parity of the
blackhole would be opposite of other blackholes. We could expect then
universes with a reversed arrow of time.

Single Electron hypothesis implies the connection between matter and
anti-matter particles with respect to time.

\url{http://eaps4.mit.edu/research/Lorenz/The_Growth_of_Errors_in_Prediction_1985.pdf}

``By a stable system we mean one whose future succession of states, if

the present state should be slightly disturbed, will converge toward

the succession of states which have occurred if there had been no

disturbance. By an unstable system we mean just the opposite--a system

whose future following a slight disturbance will diverge from what its

future would have been without a disturbance.''

This should be compared to Turing's thesis on automatic machines in

which a deterministic system is compared to a non-deterministic system

in terms of the direction or vector of the transition or

transformation of states. A deterministic machine evolves from an

initial state to a final state in a linear order such that future

states depend directly and exactly upon previous or past states; a

non-deterministic machine's evolution from initial to final states is

not necessarily directly dependent on past states.

Hypothetically, The notion of stable vs unstable systems is analogous

to the input and output of a Turing machine such a Turing machine is

called stable if a small input causes the Turing machine to converge

on a designated final state expected a priori. A Turing machine is

called unstable if a small input causes the Turing machine to diverge

from a designated final stated expected a priori. A simple stable

Turing machine would always output a constant state in response to any

input; a simple unstable Turing machine would always output a random

state in response to every input. In general the stability and

instability of a Turing machine would exist in a inverse relationship

by degree such that a Turing machine could be somewhat stable and

somewhat unstable but not arbitrarily so.

Similarly, the determinism and non-determinism of the Turing machine

can be related in shared degrees, and the degrees of deterministic and

non-deterministic relations can be related to the degrees of stable

and unstable relations.

Further research into quantum, fuzzy, and paraconsistent mechanics

would likely be productive with respect to the corresponding degrees

of relations of the mechanical state.
