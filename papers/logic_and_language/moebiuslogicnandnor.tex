\documentclass{article}
\usepackage{amsmath}
\usepackage{amssymb}
\usepackage{amsfonts}
\usepackage{amsthm}
\usepackage{enumitem}

\title{Möbius NAND and NOR}
\author{}
\date{\today}

\begin{document}
	
	\maketitle
	
	Based on our defined logical operations in the M\"{o}bius logic on $\mathbb{CP}^1$:
	\begin{itemize}
		\item \textbf{Semantic Space:} $\mathbb{CP}^1 = \mathbb{C} \cup \{\infty\}$
		\item \textbf{Negation ($\neg$):} Inversion, $\neg z = 1/z$
		\item \textbf{Conjunction ($\wedge$):} Multiplication, $z_1 \wedge z_2 = z_1 \cdot z_2$
		\item \textbf{Disjunction ($\vee$):} Addition, $z_1 \vee z_2 = z_1 + z_2$
	\end{itemize}
	We can formalize the concepts of "M\"{o}bius NAND" and "M\"{o}bius NOR" as the negation of the conjunction and disjunction, respectively, analogous to their definitions in classical logic.
	
	\section{Möbius NAND ($N_{MNAND}$)}
	
	The M\"{o}bius NAND operation is defined as the negation of the M\"{o}bius conjunction:
	$$N_{MNAND}(z_1, z_2) = \neg(z_1 \wedge z_2) = \neg(z_1 \cdot z_2) = \frac{1}{z_1 \cdot z_2}$$
	This operation is well-defined on $\mathbb{CP}^1$ with the standard rules for arithmetic involving $\infty$ and $1/0 = \infty$.
	
	\section{Möbius NOR ($N_{MNOR}$)}
	
	The M\"{o}bius NOR operation is defined as the negation of the M\"{o}bius disjunction:
	$$N_{MNOR}(z_1, z_2) = \neg(z_1 \vee z_2) = \neg(z_1 + z_2) = \frac{1}{z_1 + z_2}$$
	This operation is well-defined on $\mathbb{CP}^1$ with the standard rules for arithmetic involving $\infty$ and $1/0 = \infty$.
	
	\section{Properties and Functional Completeness}
	
	Let's examine the properties of these operations and their potential to generate the basic operations of our M\"{o}bius logic ($1/z$, $z_1 \cdot z_2$, $z_1 + z_2$) through composition.
	
	\subsection{Möbius NAND ($N_{MNAND}(z_1, z_2) = \frac{1}{z_1 \cdot z_2}$)}
	
	\begin{itemize}
		\item \textbf{Negation ($\neg z = 1/z$):} Can be generated by fixing one input to 1:
		$$N_{MNAND}(z, 1) = \frac{1}{z \cdot 1} = \frac{1}{z}$$
		This requires the ability to generate the constant 1. We can generate 1:
		$$N_{MNAND}(z, 1/z) = \frac{1}{z \cdot (1/z)} = 1 \quad \text{(for } z \neq 0, \infty)$$
		\item \textbf{Conjunction ($z_1 \cdot z_2$):} Can be generated by negating the output of $N_{MNAND}$:
		$$\neg(N_{MNAND}(z_1, z_2)) = \neg\left(\frac{1}{z_1 \cdot z_2}\right) = \frac{1}{1/(z_1 \cdot z_2)} = z_1 \cdot z_2$$
		This requires the ability to generate negation, which we can do using $N_{MNAND}$ and the constant 1.
		\item \textbf{Disjunction ($z_1 + z_2$):} Generating disjunction using only $N_{MNAND}$ and compositions appears challenging due to the multiplicative and inversive nature of the operation. The classical De Morgan's law equivalence ($p \vee q \equiv \neg(\neg p \wedge \neg q)$) translates to $z_1 + z_2 \equiv \neg(\neg z_1 \wedge \neg z_2)$, which semantically is $z_1 + z_2 = \neg((1/z_1) \cdot (1/z_2)) = z_1 z_2$. This identity does not hold universally.
	\end{itemize}
	M\"{o}bius NAND can generate negation and conjunction (given the constant 1), but not generally disjunction.
	
	\subsection{Möbius NOR ($N_{MNOR}(z_1, z_2) = \frac{1}{z_1 + z_2}$)}
	
	\begin{itemize}
		\item \textbf{Negation ($\neg z = 1/z$):} Can be generated by fixing one input to 0:
		$$N_{MNOR}(z, 0) = \frac{1}{z + 0} = \frac{1}{z}$$
		This requires the ability to generate the constant 0. We can generate 0:
		$$N_{MNOR}(z, \infty) = \frac{1}{z + \infty} = \frac{1}{\infty} = 0 \quad \text{(for } z \neq \infty)$$
		\item \textbf{Disjunction ($z_1 + z_2$):} Can be generated by negating the output of $N_{MNOR}$:
		$$\neg(N_{MNOR}(z_1, z_2)) = \neg\left(\frac{1}{z_1 + z_2}\right) = \frac{1}{1/(z_1 + z_2)} = z_1 + z_2$$
		This requires the ability to generate negation, which we can do using $N_{MNOR}$ and the constant 0.
		\item \textbf{Conjunction ($z_1 \cdot z_2$):} Generating conjunction using only $N_{MNOR}$ and compositions appears challenging due to the additive and inversive nature of the operation. The classical De Morgan's law equivalence ($p \wedge q \equiv \neg(\neg p \vee \neg q)$) translates to $z_1 \cdot z_2 \equiv \neg(\neg z_1 \vee \neg z_2)$, which semantically is $z_1 \cdot z_2 = \neg((1/z_1) + (1/z_2)) = \frac{1}{1/z_1 + 1/z_2} = \frac{z_1 z_2}{z_1 + z_2}$. This identity does not hold universally.
	\end{itemize}
	M\"{o}bius NOR can generate negation and disjunction (given the constant 0), but not generally conjunction.
	
	\section{Conclusion on Functional Completeness}
	
	Based on this analysis, neither $N_{MNAND}(z_1, z_2) = \frac{1}{z_1 \cdot z_2}$ nor $N_{MNOR}(z_1, z_2) = \frac{1}{z_1 + z_2}$ appear to be \textbf{sole-sufficient} operations for generating \textit{all three} of our basic M\"{o}bius logic operations ($1/z$, $z_1 \cdot z_2$, $z_1 + z_2$) through composition alone. They can each generate negation and one of the binary operations, provided certain constants (0 or 1) are available.
	
	This suggests that the set $\{1/z, z_1 \cdot z_2, z_1 + z_2\}$ might be a minimal generating set for the operations of our M\"{o}bius logic, or that a single sole-sufficient operation would need to be a more complex function that inherently combines both multiplicative and additive structures in a fundamental way. The general form of the M\"{o}bius transformation $\frac{az+b}{cz+d}$ encapsulates this combination, suggesting that an operation based on this form might be a candidate for generating a broader class of functions on $\mathbb{CP}^1$.
	
	However, $N_{MNAND}$ and $N_{MNOR}$ are valid formalizations of the negated conjunction and disjunction within our defined M\"{o}bius logic and represent specific binary operations with distinct properties on $\mathbb{CP}^1$.
	
\end{document}
