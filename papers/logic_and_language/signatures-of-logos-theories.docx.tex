}

†A: Complex Conjugate/Transpose Negation (Orthogonal/Perpendicular
Shift).

A ∥ B: Additive Implication (Parallel Processes/Concurrent Choice). *

A ∦ B: Additive Non-Implication (Non-Parallel/Mutually Exclusive
Processes).

¬: Linear Negation (Classical).

⊸: Multiplicative Implication (Linear Implication).

⊸̸: Multiplicative Non-Implication (Type 1A).

▹A: Type 1A Negation (Paracomplete).

◃A: Type 1B Negation (Paraconsistent).

\&, ⊕, ⊗, ⅋: Additive and Multiplicative Conjunction/Disjunction.

∃∀ \{⊥,⊤,¬,∨,∧,→,↛,↔,⊕,↓,↑\} \{p, q\} ⊢⊬⊨ΓΔ

Signature of Theories with Logos Formalization

The theories discussed in this paper will be referred to as
\emph{theories with standard formalization}. They can be briefly
characterized as theories which are formalized within the first-order
predicate logic (with identity, without variable predicates).

\hypertarget{syntactic-definitions}{%
\subsection{Syntactic Definitions}\label{syntactic-definitions}}

\hypertarget{definition-variables-and-constants-of-a-theory}{%
\subsubsection{Definition: Variables and Constants of a
Theory}\label{definition-variables-and-constants-of-a-theory}}

The symbols which occur in expressions of a given theory T are divided
into \emph{variables} and \emph{constants}.

The set of variables is assumed to be denumerable and hence infinite;
the set of constants is either finite or denumerable. All the variables
are treated as ranging over the same set of elements.

\hypertarget{section}{%
\subsubsection{}\label{section}}

\hypertarget{definition-non-logical-constants-of-a-theory-aka-signature-of-a-theory}{%
\subsubsection{Definition: non-Logical Constants of a Theory AKA
Signature of a
Theory}\label{definition-non-logical-constants-of-a-theory-aka-signature-of-a-theory}}

The non-logical constants are the \emph{predicates} (or \emph{relation
symbols}), the \emph{operation symbols}, and the \emph{individual
constants}.

With every predicate and every operation symbol a positive integer is
correlated which is called the \emph{rank} of the symbol. Thus, we may
have in T \emph{unary} predicates and operation symbols (I.E. symbols of
rank 1), \emph{binary} predicates and operation symbols (symbols of rank
2), etc. The identity symbol, though regarded as a logical constant, is
included in the set of binary predicates.

\hypertarget{definition-terms}{%
\subsubsection{Definition: Terms}\label{definition-terms}}

Among expressions (I.E. finite concatenations of symbols) we distinguish
\emph{terms} and \emph{formulas}.

The simplest, so-called atomic, terms are the variables and the
individual constants; a compound term is obtained by combining \emph{n}
simpler terms by means of an operation symbol of rank \emph{n}.

\hypertarget{definition-formulas}{%
\subsubsection{Definition: Formulas}\label{definition-formulas}}

Similarly, an atomic formula is obtained by combining \emph{n} arbitrary
terms by means of a predicate of rank \emph{n}; compound formulas are
built from simpler ones by means of sentential connectives and
quantifier expressions (I.E. quantifiers followed by variables like ∀x
or ∃y).

\hypertarget{definition-sentences}{%
\subsubsection{Definition: Sentences}\label{definition-sentences}}

An occurrence of a variable in a formula may be either \emph{free} or
\emph{bound}; a formula in which no variable occurs free is called a
\emph{sentence}.

\hypertarget{semantic-definitions}{%
\subsection{Semantic Definitions}\label{semantic-definitions}}

\hypertarget{model-theoretical-semantics}{%
\paragraph{Model-Theoretical
Semantics}\label{model-theoretical-semantics}}

Another method of defining logically derivable and logically valid is
available which essentially involves the use of some semantical notions
and the notion of satisfaction.

\hypertarget{definition-possible-realizations-of-a-theory-and-the-universe-of-ux1d57d}{%
\subsubsection{Definition: Possible Realizations of a Theory and the
Universe of
𝕽}\label{definition-possible-realizations-of-a-theory-and-the-universe-of-ux1d57d}}

We assume that all the non-logical constants of T have been arranged in
a (finite or infinite) sequence \textless{}\textbf{C\_0}, \ldots,
\textbf{C\_n}, \ldots\textgreater, without repeating terms.

We consider systems 𝕽 formed by a non-empty set U and by a sequence
\textless C\_0, \ldots, C\_n, \ldots\textgreater{} of certain
mathematical entities, with the same number of terms as the sequence of
non-logical constants.

The mathematical nature of each C\_n depends on the logical character of
the corresponding constant \textbf{C\_n}. Thus,

if \textbf{C\_n} is a unary predicate, then C\_n is a subset of U; more
generally, if \textbf{C\_n} is an \emph{m}-ary predicate, then C\_n is
an \emph{m}-ary relation the field of which is a subset of U.

If \textbf{C\_n} is an \emph{m}-ary operation symbol, C\_n is an
\emph{m}-ary operation (function of \emph{m} arguments) defined over
arbitrary ordered \emph{m}-tuples \textless{}\emph{x\_1}, \ldots,
\emph{x\_m}\textgreater{} of elements of U and assuming elements of U as
values.

If \textbf{C\_n} is an individual constant, C\_n is simply an element of
U.

Such a system (sequence) 𝕽 = \textless U, C\_0, \ldots, C\_n,
\ldots\textgreater{} is called a \emph{possible realization} or simply a
\emph{realization} of T; the set U is called the \emph{universe} of 𝕽.

\hypertarget{definition-satisfaction}{%
\subsubsection{Definition: Satisfaction}\label{definition-satisfaction}}

We assume it to be clear under what conditions a sentence Φ of T is said
to \emph{be satisfied} or to \emph{hold} in a given realization 𝕽.

Roughly speaking, this means that Φ turns out to be true if

(i) all the variables occurring in T are assumed to range over the set
U;

(ii) the logical constants are interpreted in the usual way;

(iii) each of the non-logical constants \textbf{C\_n} is understood to
denote the corresponding term C\_n in 𝕽.

Assume, e.g., that the term \textbf{C\_n} in the sequence of constants
is a unary predicate and that consequently C\_n is a subset of U. Then
the sentence ∀x \textbf{C\_n} x holds in 𝕽 if and only if every element
of U is an element of C\_n and hence C\_n coincides with U.

\hypertarget{definition-valid-sentences-by-non-logical-axioms}{%
\subsubsection{Definition: Valid Sentences by Non-Logical
Axioms}\label{definition-valid-sentences-by-non-logical-axioms}}

Often we single out a (finite or infinite) set of sentences called
\emph{non-logical axioms}, and define a sentence to be valid if and only
if it is derivable from this set-\/-or, what amounts to the same, from
the set of all axioms, both logical and non-logical.

In all the theories with standard formalization the same symbols are
assumed to be used as variables and logical constants; apart from
differences in non-logical constants, the same expressions are regarded
as formulas, sentences, logical axioms, and logically valid sentences.

However, the notions of validity in these theories may of course exhibit
essential differences.

\hypertarget{theoretical-definitions}{%
\subsection{Theoretical Definitions}\label{theoretical-definitions}}

\hypertarget{definition-uniqueness-of-an-axiomatic-theory-in-standard-formalization}{%
\subsubsection{Definition: Uniqueness of an Axiomatic Theory in Standard
Formalization}\label{definition-uniqueness-of-an-axiomatic-theory-in-standard-formalization}}

An axiomatic theory is uniquely determined by its non-logical constants
and non-logical axioms.

\hypertarget{definition-inessential-extensions-of-theories-in-standard-formalization}{%
\subsubsection{Definition: Inessential Extensions of Theories in
Standard
Formalization}\label{definition-inessential-extensions-of-theories-in-standard-formalization}}

An extension T\_2 of T\_1 is called \emph{inessential} if every constant
of T\_2 which does not occur in T\_1 is an individual constant and if
every valid sentence of T\_2 is derivable in T\_2 from a set of valid
sentences of T\_1.

If T\_1 is axiomatic, then an inessential extension of T\_1 is obtained
by adding some new individual constants, but without adding any new
non-logical axioms.

By saying that a sentence Φ is derivable \emph{in a theory} T from a set
A we stress the fact that, in deriving Φ, we may use both sentences of A
and logical axioms of T. It is easily seen that, whenever Φ is derivable
from A in some theory T, it is also derivable from A in every theory
T\textquotesingle{} which contains all the non-logical constants
occurring in Φ and in sentences of A.

\hypertarget{definition-union-of-theories-in-standard-formalization}{%
\subsubsection{Definition: Union of Theories in Standard
Formalization}\label{definition-union-of-theories-in-standard-formalization}}

Among the extensions common to two given theories T\_1 and T\_2 there is
always a smallest one, which is a subtheory of any other common
extension; this smallest common extension is referred to as the
\emph{union} of the given theories.

The union T of T\_1 and T\_2 is fully characterized by the following two
conditions:

(i) the set of all non-logical constants of T is the (set theoretical)
union of the set of all non-logical constants of T\_1 and T\_2;

(ii) a sentence is valid in T if and only if it is derivable in T from a
set of sentences which are valid in T\_1 or T\_2.

If the theories T\_1 and T\_2 are axiomatic, we can construct T by
postulating, in addition to (i), the analogous condition for the set of
non-logical axioms.

\end{document}}*}
