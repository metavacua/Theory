





 % Added for hyperlinking references







	
	
	\section{The Semantic Space: The Complex Projective Line ($\mathbb{CP}^1$)}
	
	The Complex Projective Line, denoted $\mathbb{CP}^1$, serves as the semantic space for our proposed non-bivalent logic. It is defined as the set of equivalence classes of pairs of complex numbers $[z_0, z_1] \in \mathbb{C}^2 \setminus \{(0,0)\}$, where $[z_0, z_1] \sim [w_0, w_1]$ if $z_0 = \lambda w_0$ and $z_1 = \lambda w_1$ for some $\lambda \in \mathbb{C} \setminus \{0\}$.
	
	Equivalently, $\mathbb{CP}^1$ can be viewed as the extended complex plane $\mathbb{C} \cup \{\infty\}$, where the equivalence class $[0, 1]$ corresponds to the point at infinity, $\infty$. Geometrically, $\mathbb{CP}^1$ is homeomorphic to the 2-sphere, often called the Riemann sphere.
	
	The automorphisms of $\mathbb{CP}^1$ are the M\"{o}bius transformations, functions of the form $f(z) = \frac{az+b}{cz+d}$ with $ad-bc \neq 0$, where $a, b, c, d \in \mathbb{C}$. These transformations form a group under composition and are generated by translations ($z \mapsto z+b$), dilations/rotations ($z \mapsto az$), and inversion ($z \mapsto 1/z$).
	
	\section{Möbius Logic}
	
	Our proposed M\"{o}bius logic is a non-bivalent logic defined directly on the semantic space $\mathbb{CP}^1$. Its fundamental logical operations are inspired by the generators of the M\"{o}bius group.
	
	\subsection{Semantic Space}
	The semantic space is $\mathbb{CP}^1 = \mathbb{C} \cup \{\infty\}$.
	
	\subsection{Logical Operations}
	For $z, z_1, z_2 \in \mathbb{CP}^1$:
	\begin{itemize}
		\item \textbf{Negation ($\neg$):} Inversion, $\neg z = 1/z$.
		\item \textbf{Conjunction ($\wedge$):} Multiplication, $z_1 \wedge z_2 = z_1 \cdot z_2$.
		\item \textbf{Disjunction ($\vee$):} Addition, $z_1 \vee z_2 = z_1 + z_2$.
	\end{itemize}
	These operations are extended to include $\infty$ with standard rules from complex analysis: $z+\infty = \infty$ for $z \neq \infty$, $\infty+\infty=\infty$, $z \cdot \infty = \infty$ for $z \neq 0, \infty$, $0 \cdot \infty = 0$, $\infty \cdot \infty = \infty$, $1/0 = \infty$, $1/\infty = 0$.
	
	\subsection{Properties of Möbius Logic Operations}
	\begin{itemize}
		\item \textbf{Commutativity:} Both \(\wedge\) and \(\vee\) are commutative (\(z_1 \wedge z_2 = z_2 \wedge z_1\), \(z_1 \vee z_2 = z_2 \vee z_1\)).
		\item \textbf{Associativity:} Both \(\wedge\) and \(\vee\) are associative (\((z_1 \wedge z_2) \wedge z_3 = z_1 \wedge (z_2 \wedge z_3)\), \((z_1 \vee z_2) \vee z_3 = z_1 \vee (z_2 \vee z_3)\)).
		\item \textbf{Idempotency:}
		\begin{itemize}
			\item \(\wedge\) is not generally idempotent; the identity \(z \wedge z = z \cdot z = z\) holds only for \(z \in \{0, 1, \infty\}\).
			\item \(\vee\) is not generally idempotent; the identity \(z \vee z = z + z = z\) holds only for \(z \in \{0, \infty\}\).
		\end{itemize}
		\item \textbf{Involution:} \(\neg\) is an involution (\(\neg \neg z = 1/(1/z) = z\)).
		\item \textbf{Fixed Points of Negation:} The fixed points of negation (\(\neg z = z\)) are the solutions to \(1/z = z \implies z^2 = 1\), which are \(\{1, -1\}\).
		\item \textbf{De Morgan's Laws:} Standard De Morgan's Laws generally fail (e.g., \(\neg(z_1 \wedge z_2) = 1/(z_1 z_2)\) is not generally equal to \(\neg z_1 \vee \neg z_2 = 1/z_1 + 1/z_2\)).
	\end{itemize}
	
	\subsection{LNC and LEM Fixed Points in Möbius Logic}
	We analyze the fixed points of the identities corresponding to the Law of Non-Contradiction (LNC) and the Law of Excluded Middle (LEM) using our defined operations:
	\begin{itemize}
		\item \textbf{Law of Non-Contradiction (LNC):} The identity is \(\neg z \wedge z = z\), which is \((1/z) \cdot z = z\).
		For \(z \in \mathbb{C}, z \neq 0\), this simplifies to \(1 = z\), giving the solution \(z=1\).
		For \(z=0\), \(\neg 0 \wedge 0 = \infty \wedge 0 = 0\), which equals \(z=0\). So \(z=0\) is a fixed point.
		For \(z=\infty\), \(\neg \infty \wedge \infty = 0 \wedge \infty = 0\), which must equal \(z=\infty\). This is not a fixed point.
		The fixed points for the LNC identity are \(\{0, 1\}\).
		\item \textbf{Law of Excluded Middle (LEM):} The identity is \(\neg z \vee z = z\), which is \((1/z) + z = z\).
		For \(z \in \mathbb{C}, z \neq 0\), this simplifies to \(1/z = 0\), which has no solution in \(\mathbb{C}\).
		For \(z=0\), \(\neg 0 \vee 0 = \infty \vee 0 = \infty\), which must equal \(z=0\). This is not a fixed point.
		For \(z=\infty\), \(\neg \infty \vee \infty = 0 \vee \infty = \infty\), which equals \(z=\infty\). So \(z=\infty\) is a fixed point.
		The fixed points for the LEM identity are \(\{\infty\}\).
	\end{itemize}
	The LNC and LEM identities hold only for specific, discrete points in \(\mathbb{CP}^1\). This asymmetry in fixed points is a notable characteristic of this logic.
	
	\section{Multiplicative Linear Logic (MLL)}
	
	Multiplicative Linear Logic (MLL) is a fragment of Linear Logic characterized by its focus on resource-sensitive connectives and the absence of structural rules of weakening and contraction for its multiplicative connectives.
	
	\subsection{Syntax (Sequent Calculus)}
	MLL typically uses a two-sided sequent calculus with formulas built from propositional variables, linear negation (\(\neg\)), multiplicative conjunction (\(\otimes\)), and multiplicative disjunction (\(\wp\)).
	
	\subsection{Semantics (Categorical)}
	Standard semantics for MLL are given in \textbf{star-autonomous categories}. These are symmetric monoidal categories equipped with a strong duality (the star operation) that interprets linear negation.
	
	\subsection{Properties of MLL Connectives}
	\begin{itemize}
		\item \textbf{Commutativity:} \(\otimes\) and \(\wp\) are commutative.
		\item \textbf{Associativity:} \(\otimes\) and \(\wp\) are associative.
		\item \textbf{Idempotency:} \(\otimes\) and \(\wp\) are \textbf{not idempotent}. The identities \(A \otimes A \equiv A\) and \(A \wp A \equiv A\) are not theorems.
		\item \textbf{Involution:} \(\neg\) is an involution (\(\neg \neg A \equiv A\)).
		\item \textbf{Fixed Points of Negation:} In standard semantics, linear negation typically does not have fixed points (objects \(A\) where \(\neg A \cong A\)) for non-trivial objects.
		\item \textbf{De Morgan's Laws:} Standard De Morgan's Laws \textbf{hold}: \(\neg(A \otimes B) \equiv \neg A \wp \neg B\) and \(\neg(A \wp B) \equiv \neg A \otimes \neg B\).
	\end{itemize}
	
	\subsection{LNC and LEM Behavior in MLL}
	\begin{itemize}
		\item \textbf{Law of Non-Contradiction (LNC):} The formula is \(A \otimes \neg A\). In standard semantics, \(A \otimes \neg A\) is \textbf{not universally isomorphic to the unit object I} (which represents a tautology). MLL is paraconsistent.
		\item \textbf{Law of Excluded Middle (LEM):} The formula is \(A \wp \neg A\). In standard semantics, \(A \wp \neg A\) is \textbf{not universally isomorphic to the unit object I}. MLL is paracomplete.
	\end{itemize}
	Syntactically, the sequents \(A \otimes \neg A \vdash\) and \(\vdash A \wp \neg A\) are provable in MLL, but the identities \(A \otimes \neg A \equiv I\) and \(A \wp \neg A \equiv I\) are not universal.
	
	\section{Möbius Logic, Self-Reference, and the Liar Paradox}
	
	Drawing upon the work of Paola Zizzi on turning the Liar Paradox into a metatheorem of Basic Logic, we analyze the implications of our M\"{o}bius logic's properties for the formalization of self-reference.
	
	\subsection{Standard vs. Generalized Self-Reference}
	
	The classical Liar Paradox, typically formulated as ``This sentence is false,'' poses a significant challenge to standard logical systems. Zizzi argues that this paradox arises when self-reference is formalized using the **standard definition of self-reference** within **structural logics**.
	
	The standard definition of self-reference for a sentence \(S\) in a metalanguage \(L_m\) is given by:
	\[ S :\equiv F(\text{``}S\text{''}) \]
	where \(F\) is a function from the object-language \(L_0\) (containing names of sentences) to the metalanguage \(L_m\), and \(\text{``}S\text{''}\) is the name of \(S\) in \(L_0\). When \(F\) is the function \(\neg True\), this leads to the Liar sentence \(L :\equiv \neg True(\text{``}L\text{''})\). By Tarski's definition of truth (\(True(\text{``}A\text{''}) \equiv A\)), this equivalence becomes \(L \equiv \neg L\), the classical paradox. This paradox is unavoidable in structural logics because they possess properties (like the universal idempotence of connectives) that force this equivalence.
	
	Zizzi proposes a **generalized definition of self-reference** that is applicable in substructural logics:
	\[ S :\equiv F(f(\text{``}S\text{''})) \]
	Here, \(f\) is a function \(f: L_0 \to L_0\) defined as \(f(\text{``}S\text{''}) = \text{``}S\text{''} \bullet \text{``}S\text{''}\), where \(\bullet\) is a binary connective in the object-language used for duplicating the name of the sentence.
	
	Zizzi distinguishes two cases based on the property of the connective \(\bullet\):
	
	\begin{enumerate}
		\item \textbf{Case 1: \(\bullet\) is Idempotent.} If the connective \(\bullet\) is idempotent (\(A \bullet A = A\) for all formulas \(A\)), then the function \(f\) has a fixed point for all inputs: \(f(\text{``}S\text{''}) = \text{``}S\text{''})\) (Equation 17 in Zizzi's paper). In this case, the generalized definition \(S :\equiv F(f(\text{``}S\text{''}))\) reduces to the standard definition \(S :\equiv F(\text{``}S\text{''})\) (Equation 12). This is what happens in structural logics, and it allows the Liar paradox \(L \equiv \neg L\) to arise.
		
		\item \textbf{Case 2: \(\bullet\) is Non-Idempotent.} If the connective \(\bullet\) is non-idempotent (\(A \bullet A \neq A\) for some formula \(A\)), then the function \(f\) generally does not have a fixed point: \(f(\text{``}S\text{''}) = \text{``}\widehat{S}\text{''}) \neq \text{``}S\text{''}\) (Equation 18). In this case, the generalized definition becomes \(S :\equiv F(\text{``}\widehat{S}\text{''}))\) (Equation 19). For the Liar sentence, this leads to \(L \equiv \neg True(\text{``}\widehat{L}\text{''}))\), which by Tarski's definition becomes \(L \equiv \neg \widehat{L}\) (Equation 22). Since \(\widehat{L}\) is the name of a sentence different from \(L\), this is not a paradox. This scenario is characteristic of substructural logics with non-idempotent connectives.
	\end{enumerate}
	
	\subsection{Necessary and Sufficient Conditions for Standard Self-Reference}
	
	Zizzi further provides conditions under which the standard definition of self-reference (and thus the potential for paradox) is present:
	
	\begin{itemize}
		\item \textbf{Necessary Condition:} For the standard definition of self-reference to apply universally (for all sentences), the connective \(\bullet\) used for duplication in the object-language must be universally idempotent. The existence of a fixed point for the function \(f\) for all relevant inputs is a necessary condition for the standard definition to hold.
		
		\item \textbf{Sufficient Condition:} Even if the connective \(\bullet\) is non-idempotent in the object-language, the standard definition of self-reference can still be recovered if the corresponding physical link in the metalanguage allows for cloning of the relevant states (e.g., the ability to clone basis states in the quantum mechanical formalism, as shown in Equation 26 using the CNOT gate). This ability to clone in the metalanguage can, in certain cases, force a diagram to commute (Diagram 4 and Equation 28 in Zizzi's paper), effectively recovering the standard definition of self-reference for those specific states.
	\end{itemize}
	
	\subsection{Locally Tarskian Self-Reference}
	
	Based on our analysis of the M\"{o}bius logic on \(\mathbb{CP}^1\) and its non-universal idempotent connectives, we introduce the notion of \textbf{locally Tarskian self-reference}.
	
	\begin{itemize}
		\item \textbf{Definition:} Locally Tarskian self-reference is the phenomenon where the standard definition of self-reference, which can lead to Tarski-like paradoxes (such as the Liar paradox \(L \equiv \neg L\)), holds only for specific, localized regions or points within a semantic space, rather than universally for all possible semantic values.
		\item \textbf{Mechanism in Möbius Logic:} In our M\"{o}bius logic, the connectives (multiplication and addition) are not universally idempotent. Idempotency holds only for specific, discrete points in \(\mathbb{CP}^1\) (\{0, 1, \(\infty\)\} for multiplication, \{0, \(\infty\)\} for addition). These are points where the duplication function \(f\) *can* have a fixed point.
		\item \textbf{Implication:} For sentences whose semantic values fall within these localized fixed-point sets for idempotency, Zizzi's necessary condition for the standard definition of self-reference is met. This is where standard, potentially paradoxical self-reference can occur. For semantic values outside these localized regions, the non-idempotency of the connectives ensures that the generalized definition of self-reference applies, preventing the classical paradox.
	\end{itemize}
	Thus, in a logic with locally Tarskian self-reference, the problematic aspects of self-reference are confined to specific, structurally significant parts of the semantic space, rather than pervading the entire logic as they would in a structural logic.
	
	\subsection{Implications for Möbius Logic on \(\mathbb{CP}^1\)}
	
	Our M\"{o}bius logic on \(\mathbb{CP}^1\) uses complex multiplication (\(\wedge\)) and addition (\(\vee\)) as binary connectives, and inversion (\(\neg\)) as negation. We analyze the implications of Zizzi's framework using these operations as the potential connective \(\bullet\) for duplication in the self-reference construction.
	
	\begin{itemize}
		\item \textbf{Non-Idempotence:} The connectives \(\wedge\) (multiplication) and \(\vee\) (addition) in our M\"{o}bius logic are \textbf{not generally idempotent}. While they have fixed points for idempotency at specific points (\{0, 1, \(\infty\)\} for \(\wedge\), \{0, \(\infty\)\} for \(\vee\)), they do not satisfy the identity \(z \bullet z = z\) for all \(z \in \mathbb{CP}^1\). This aligns with Case 2 of Zizzi's generalized definition for most semantic values in \(\mathbb{CP}^1\).
		
		\item \textbf{Partial Idempotency:} The existence of specific fixed points for idempotency (\{0, 1, \(\infty\)\} for multiplication, \{0, \(\infty\)\} for addition) means that for sentences whose semantic values are precisely these points, the duplication operation (\(z \bullet z\)) *does* return the original value (\(z\)). For these specific semantic values, the function \(f\) *does* have a fixed point, and Zizzi's necessary condition for the standard definition of self-reference is met for these localized cases.
		
		\item \textbf{Necessary Condition in Möbius Logic:} The necessary condition for the standard definition of self-reference to apply is met \textbf{only for sentences whose semantic values are in the fixed point sets for idempotency}. For other semantic values in \(\mathbb{CP}^1\), the non-idempotence ensures that the function \(f\) lacks a fixed point (\(f(\text{``}S\text{''}) \neq \text{``}S\text{''})\)), and the generalized definition applies, leading to \(L \equiv \neg \widehat{L}\), which avoids the classical paradox.
		
		\item \textbf{Sufficient Condition and Cloning:} \(\mathbb{CP}^1\) is the space of pure states of a 2-level quantum system. The semantic values 0 and 1 correspond to clonable basis states (\(|0\rangle\) and \(|1\rangle\)) in the related quantum mechanical metalanguage. This suggests that Zizzi's sufficient condition for recovering the standard definition of self-reference \textbf{appears to be met for semantic values 0 and 1}. The ability to clone these corresponding quantum states in the metalanguage provides the "physical link" that could allow the standard definition of self-reference to apply for propositions with these specific semantic values. The status of \(\infty\) in relation to cloning of specific states would require further investigation.
	\end{itemize}
	
	\subsection{Conclusion on Self-Reference in Möbius Logic}
	
	The properties of our M\"{o}bius logic operations on \(\mathbb{CP}^1\) indicate that this semantic space accommodates both generalized and standard self-reference, with the latter being localized to specific points.
	
	\begin{itemize}
		\item For the vast majority of semantic values in \(\mathbb{CP}^1\), the non-idempotency of the connectives ensures that self-reference is formalized via Zizzi's generalized definition, preventing the classical Liar Paradox \(L \equiv \neg L\).
		\item However, specifically for self-referential statements whose semantic values are the fixed points for idempotency (\{0, 1, \(\infty\)\} for multiplication, \{0, \(\infty\)\} for addition), the necessary condition for the standard definition is met. Furthermore, the connection to the clonability of corresponding quantum states suggests the sufficient condition might also be met for points corresponding to clonable states (like 0 and 1).
	\end{itemize}
	This implies that the classical Liar Paradox, in its most problematic form, might not arise universally in M\"{o}bius logic but could potentially be localized to self-referential statements whose semantic values are precisely the fixed points for idempotency and correspond to clonable states. \(\mathbb{CP}^1\) thus provides a semantic framework where the behavior of self-reference is nuanced, depending on the specific semantic value involved, aligning the logic with substructural principles while acknowledging the unique properties of its complex semantic space and its connection to quantum information.
	
	\section{Specialized Semantic Spaces and Related Logics}
	
	Based on the properties of our M\"{o}bius logic on $\mathbb{CP}^1$, particularly the discrete fixed points for idempotency and the Law of Non-Contradiction, we can explore specialized semantic spaces that are subsets of $\mathbb{CP}^1$ and define logics on them. These specialized logics can help us understand the structure of logics related to the M\"{o}bius logic.
	
	Given that the M\"{o}bius Logic on $\mathbb{CP}^1$ is likely expressive enough to interpret arithmetic and thus potentially essentially undecidable, standard decidable logics like Classical Propositional Logic (LK) on a finite semantic space like $\\{0, 1\\}$ cannot be a conservative extension of it. Instead, such simpler logics on specialized semantic spaces represent reductions or specific models of fragments of the M\"{o}bius logic.
	
	Let's consider an **infinite, proper subset of $\mathbb{CP}^1$** as a specialized semantic space. The **extended real line ($\mathbb{R} \cup \{\infty\}$)** is a natural candidate as it includes key fixed points (0, 1, -1, $\infty$) and is a geometrically simpler, yet still infinite, subset of $\mathbb{CP}^1$.
	
	\subsection{Logic on the Extended Real Line ($\mathbb{R} \cup \{\infty\}$)}
	
	Let the specialized semantic space be $S'' = \mathbb{R} \cup \{\infty\} \subset \mathbb{CP}^1$. We can define a logic on this space using the same M\"{o}bius operations restricted to real numbers and infinity.
	
	\begin{itemize}
		\item \textbf{Semantic Space:} $\mathbb{R} \cup \{\infty\}$.
		\item \textbf{Negation ($\neg$):} Inversion, $\neg x = 1/x$. This is well-defined on $\mathbb{R} \cup \{\infty\}$, with $\neg 0 = \infty$ and $\neg \infty = 0$. Fixed points are $\{1, -1\}$, which are in $\mathbb{R}$.
		\item \textbf{Conjunction ($\wedge$):} Multiplication, $x_1 \wedge x_2 = x_1 \cdot x_2$. This is standard real multiplication extended to $\infty$. $x \cdot \infty = \infty$ for $x \neq 0$, $0 \cdot \infty = 0$, $\infty \cdot \infty = \infty$.
		\item \textbf{Disjunction ($\vee$):} Addition, $x_1 \vee x_2 = x_1 + x_2$. This is standard real addition extended to $\infty$. $x + \infty = \infty$ for $x \neq \infty$, $\infty + \infty = \infty$.
	\end{itemize}
	
	\subsection{Properties and Relationship to Möbius Logic}
	
	This logic on the extended real line shares many properties with the M\"{o}bius logic on $\mathbb{CP}^1$ but is restricted to a specific subset of the complex plane.
	
	\begin{itemize}
		\item \textbf{Commutativity and Associativity:} Both $\wedge$ and $\vee$ are commutative and associative on $\mathbb{R} \cup \{\infty\}$, as they are on $\mathbb{CP}^1$.
		\item \textbf{Idempotency:}
		\begin{itemize}
			\item \(\wedge\) is not generally idempotent; the identity \(x \wedge x = x \cdot x = x\) holds only for \(x \in \{0, 1, \infty\}\). These are the same fixed points as on $\mathbb{CP}^1$.
			\item \(\vee\) is not generally idempotent; the identity \(x \vee x = x + x = x\) holds only for \(x \in \{0, \infty\}\). These are the same fixed points as on $\mathbb{CP}^1$.
		\end{itemize}
		\item \textbf{Involution Negation:} $\neg x = 1/x$ is an involution on $\mathbb{R} \cup \{\infty\}$.
		\item \textbf{Fixed Points of Negation:} Fixed points are $\{1, -1\}$, which are in $\mathbb{R}$.
		\item \textbf{De Morgan's Laws:} Standard De Morgan's Laws generally fail on $\mathbb{R} \cup \{\infty\}$, just as they do on $\mathbb{CP}^1$.
		\item \textbf{LNC and LEM Fixed Points:}
		\begin{itemize}
			\item \textbf{Law of Non-Contradiction (LNC):} Identity \(\neg x \wedge x = x\). \((1/x) \cdot x = x\). Fixed points are $\{0, 1\}$. These are the same fixed points as on $\mathbb{CP}^1$.
			\item \textbf{Law of Excluded Middle (LEM):} Identity \(\neg x \vee x = x\). \((1/x) + x = x\). Fixed points are $\{\infty\}$. These are the same fixed points as on $\mathbb{CP}^1$.
		\end{itemize}
	\end{itemize}
	
	This logic on the extended real line is a **specialization** of the M\"{o}bius logic on $\mathbb{CP}^1$ in terms of its semantic space. It shares the core properties of non-idempotency, involutive negation, and the discrete fixed points for LNC and LEM. It is likely **not a conservative extension** in the standard sense, as restricting the semantic space might prevent some theorems of the full M\"{o}bius logic from holding universally (or introduce new ones). However, it represents a system with the same operational structure applied to a geometrically and algebraically simpler infinite subset.
	
	The relationship between the M\"{o}bius logic on $\mathbb{CP}^1$ and this logic on $\mathbb{R} \cup \{\infty\}$ could be explored through embeddings or restrictions, potentially revealing a different kind of categorical relationship than a simple initial/terminal one. This specialized infinite semantic space provides a valuable intermediate step between the full complexity of $\mathbb{CP}^1$ and finite semantic spaces.
	
	\section{Comparison and Relationship with MLL}
	
	Comparing M\"{o}bius logic and MLL reveals significant similarities and divergences:
	
	\subsection{Points of Correspondence}
	\begin{itemize}
		\item Both are \textbf{commutative} and \textbf{associative}.
		\item Both are \textbf{non-idempotent} (though M\"{o}bius logic has partial idempotency at specific points).
		\item Both have an \textbf{involutive negation}.
		\item Both are \textbf{paraconsistent} (LNC not universally valid) and \textbf{paracomplete} (LEM not universally valid).
	\end{itemize}
	
	\subsection{Points of Divergence}
	\begin{itemize}
		\item \textbf{Semantic Space:} M\"{o}bius logic has a set-based semantic space (\(\mathbb{CP}^1\)), while MLL has categorical semantics (star-autonomous categories).
		\item \textbf{Idempotency:} M\"{o}bius logic operations have specific fixed points for idempotency (\{0, 1, \(\infty\)\} for \(\wedge\), \{0, \(\infty\)\} for \(\vee\)), while MLL is radically non-idempotent.
		\item \textbf{Negation Fixed Points:} M\"{o}bius negation has fixed points (\{1, -1\}), while standard MLL negation typically does not have non-trivial fixed points in its standard semantics.
		\item \textbf{De Morgan's Laws:} De Morgan's Laws hold in MLL but generally fail in M\"{o}bius logic.
		\item \textbf{LNC/LEM Behavior:} In M\"{o}bius logic, the identities \(\neg z \wedge z = z\) and \(\neg z \vee z = z\) hold only for specific points. In MLL semantics, corresponding formulas \(A \otimes \neg A\) and \(A \wp \neg A\) are not universally tautologous objects, but the sequents $A \otimes \neg A \vdash$ and $\vdash A \wp \neg A$ are provable.
	\end{itemize}
	
	\subsection{Relationship}
	The divergences suggest that M\"{o}bius logic (as defined here) is not a simple conservative extension of standard MLL, nor is MLL a conservative extension of M\"{o}bius logic. The failure of De Morgan's laws in M\"{o}bius logic, which are theorems of MLL, prevents a direct conservative extension from MLL to M\"{o}bius logic under the natural mapping.
	
	The relationship is likely more complex, potentially involving:
	\begin{itemize}
		\item M\"{o}bius logic as a specific model or interpretation of MLL or a fragment thereof.
		\item Both logics being related as extensions of a common, more fundamental non-distributive substructural logic that is radically non-idempotent and lacks standard De Morgan duality.
		\item A non-standard translation or embedding that preserves certain structures but not others.
	\end{itemize}
	The semantic space \(\mathbb{CP}^1\) with its complex arithmetic and geometric structure provides a concrete setting for exploring non-idempotent, paraconsistent, and paracomplete logic, offering a unique perspective compared to the more abstract categorical semantics of MLL.
	
