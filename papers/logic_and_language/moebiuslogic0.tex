\documentclass{article}
\usepackage{amsmath}
\usepackage{amssymb}
\usepackage{amsfonts}
\usepackage{amsthm}
\usepackage{enumitem}

\title{Möbius Logic, Complex Projective Line, and Multiplicative Linear Logic}
\author{}
\date{\today}

\begin{document}
	
	\maketitle
	
	\section{The Semantic Space: The Complex Projective Line ($\mathbb{CP}^1$)}
	
	The Complex Projective Line, denoted $\mathbb{CP}^1$, serves as the semantic space for our Möbius logic. It is defined as the set of equivalence classes of pairs of complex numbers $[z_0, z_1] \in \mathbb{C}^2 \setminus \{(0,0)\}$, where $[z_0, z_1] \sim [w_0, w_1]$ if $z_0 = \lambda w_0$ and $z_1 = \lambda w_1$ for some $\lambda \in \mathbb{C} \setminus \{0\}$.
	
	Equivalently, $\mathbb{CP}^1$ can be viewed as the extended complex plane $\mathbb{C} \cup \{\infty\}$, where the equivalence class $[0, 1]$ corresponds to the point at infinity, $\infty$. Geometrically, $\mathbb{CP}^1$ is homeomorphic to the 2-sphere (the Riemann sphere).
	
	The automorphisms of $\mathbb{CP}^1$ are the M\"{o}bius transformations, functions of the form $f(z) = \frac{az+b}{cz+d}$ with $ad-bc \neq 0$, where $a, b, c, d \in \mathbb{C}$. These transformations are generated by translations ($z \mapsto z+b$), dilations/rotations ($z \mapsto az$), and inversion ($z \mapsto 1/z$).
	
	\section{Möbius Logic}
	
	Our proposed M\"{o}bius logic is a non-bivalent logic defined on the semantic space $\mathbb{CP}^1$. Its logical operations are directly inspired by the generators of the M\"{o}bius group.
	
	\subsection{Semantic Space}
	The semantic space is $\mathbb{CP}^1 = \mathbb{C} \cup \{\infty\}$.
	
	\subsection{Logical Operations}
	For $z, z_1, z_2 \in \mathbb{CP}^1$:
	\begin{itemize}
		\item \textbf{Negation ($\neg$):} Inversion, $\neg z = 1/z$.
		\item \textbf{Conjunction ($\wedge$):} Multiplication, $z_1 \wedge z_2 = z_1 \cdot z_2$.
		\item \textbf{Disjunction ($\vee$):} Addition, $z_1 \vee z_2 = z_1 + z_2$.
	\end{itemize}
	These operations are extended to $\infty$ with standard rules: $z+\infty = \infty$ for $z \neq \infty$, $\infty+\infty=\infty$, $z \cdot \infty = \infty$ for $z \neq 0, \infty$, $0 \cdot \infty = 0$, $\infty \cdot \infty = \infty$, $1/0 = \infty$, $1/\infty = 0$.
	
	\subsection{Properties of Möbius Logic Operations}
	\begin{itemize}
		\item \textbf{Commutativity:} Both $\wedge$ and $\vee$ are commutative ($z_1 \wedge z_2 = z_2 \wedge z_1$, $z_1 \vee z_2 = z_2 \vee z_1$).
		\item \textbf{Associativity:} Both $\wedge$ and $\vee$ are associative ($(z_1 \wedge z_2) \wedge z_3 = z_1 \wedge (z_2 \wedge z_3)$, $(z_1 \vee z_2) \vee z_3 = z_1 \vee (z_2 \vee z_3)$).
		\item \textbf{Idempotency:}
		\begin{itemize}
			\item $\wedge$ is not generally idempotent ($z \wedge z = z \cdot z = z$ only for $z=0, 1$).
			\item $\vee$ is not generally idempotent ($z \vee z = z + z = z$ only for $z=0$).
		\end{itemize}
		\item \textbf{Involution:} $\neg$ is an involution ($\neg \neg z = 1/(1/z) = z$).
		\item \textbf{Fixed Points of Negation:} $\neg z = z$ holds for $1/z = z \implies z^2 = 1$, so fixed points are $\{1, -1\}$.
		\item \textbf{De Morgan's Laws:} Standard De Morgan's Laws generally fail (e.g., $\neg(z_1 \wedge z_2) = 1/(z_1 z_2)$ is not generally equal to $\neg z_1 \vee \neg z_2 = 1/z_1 + 1/z_2$).
	\end{itemize}
	
	\subsection{LNC and LEM Fixed Points in Möbius Logic}
	\begin{itemize}
		\item \textbf{Law of Non-Contradiction (LNC):} Identity $\neg z \wedge z = z$.
		$(1/z) \cdot z = z$. Fixed points are $\{0, 1\}$.
		\item \textbf{Law of Excluded Middle (LEM):} Identity $\neg z \vee z = z$.
		$(1/z) + z = z$. Fixed points are $\{\infty\}$.
	\end{itemize}
	LNC and LEM identities hold only for specific, discrete points in $\mathbb{CP}^1$.
	
	\section{Multiplicative Linear Logic (MLL)}
	
	Multiplicative Linear Logic (MLL) is a fragment of Linear Logic characterized by its focus on resource-sensitive connectives and the absence of structural rules of weakening and contraction for its multiplicative connectives.
	
	\subsection{Syntax (Sequent Calculus)}
	MLL typically uses a two-sided sequent calculus with formulas built from propositional variables, linear negation ($\neg$), multiplicative conjunction ($\otimes$), and multiplicative disjunction ($\wp$).
	
	\subsection{Semantics (Categorical)}
	Standard semantics for MLL are given in \textbf{star-autonomous categories}. These are symmetric monoidal categories equipped with a strong duality (the star operation) that interprets linear negation.
	
	\subsection{Properties of MLL Connectives}
	\begin{itemize}
		\item \textbf{Commutativity:} $\otimes$ and $\wp$ are commutative.
		\item \textbf{Associativity:} $\otimes$ and $\wp$ are associative.
		\item \textbf{Idempotency:} $\otimes$ and $\wp$ are \textbf{not idempotent}. The identities $A \otimes A \equiv A$ and $A \wp A \equiv A$ are not theorems.
		\item \textbf{Involution:} $\neg$ is an involution ($\neg \neg A \equiv A$).
		\item \textbf{Fixed Points of Negation:} In standard semantics, linear negation typically does not have fixed points (objects $A$ where $\neg A \cong A$) for non-trivial objects.
		\item \textbf{De Morgan's Laws:} Standard De Morgan's Laws \textbf{hold}: $\neg(A \otimes B) \equiv \neg A \wp \neg B$ and $\neg(A \wp B) \equiv \neg A \otimes \neg B$.
	\end{itemize}
	
	\subsection{LNC and LEM Behavior in MLL}
	\begin{itemize}
		\item \textbf{Law of Non-Contradiction (LNC):} Formula $A \otimes \neg A$.
		In standard semantics, $A \otimes \neg A$ is \textbf{not universally isomorphic to the unit object I} (tautology). MLL is paraconsistent.
		\item \textbf{Law of Excluded Middle (LEM):} Formula $A \wp \neg A$.
		In standard semantics, $A \wp \neg A$ is \textbf{not universally isomorphic to the unit object I} (tautology). MLL is paracomplete.
	\end{itemize}
	Syntactically, the sequents $A \otimes \neg A \vdash$ and $\vdash A \wp \neg A$ are provable in MLL.
	
	\section{Comparison and Relationship}
	
	Comparing M\"{o}bius logic and MLL reveals significant similarities and divergences:
	
	\subsection{Points of Correspondence}
	\begin{itemize}
		\item Both are \textbf{commutative} and \textbf{associative}.
		\item Both are \textbf{non-idempotent} (though M\"{o}bius logic has partial idempotency at specific points).
		\item Both have an \textbf{involutive negation}.
		\item Both are \textbf{paraconsistent} (LNC not universal) and \textbf{paracomplete} (LEM not universal).
	\end{itemize}
	
	\subsection{Points of Divergence}
	\begin{itemize}
		\item \textbf{Semantic Space:} M\"{o}bius logic has a set-based semantic space ($\mathbb{CP}^1$), while MLL has categorical semantics (star-autonomous categories).
		\item \textbf{Idempotency:} M\"{o}bius logic operations have specific fixed points for idempotency ({0, 1} for $\wedge$, {0} for $\vee$), while MLL is radically non-idempotent.
		\item \textbf{Negation Fixed Points:} M\"{o}bius negation has fixed points ({1, -1}), while standard MLL negation does not.
		\item \textbf{De Morgan's Laws:} De Morgan's Laws hold in MLL but generally fail in M\"{o}bius logic.
		\item \textbf{LNC/LEM Behavior:} In M\"{o}bius logic, LNC/LEM identities hold for specific points. In MLL semantics, corresponding formulas are not universally tautologous objects.
	\end{itemize}
	
	\subsection{Relationship}
	The divergences suggest that M\"{o}bius logic (as defined here) is not a simple conservative extension of standard MLL, nor is MLL a conservative extension of M\"{o}bius logic. The failure of De Morgan's laws in M\"{o}bius logic, which are theorems of MLL, prevents a direct conservative extension from MLL to M\"{o}bius logic under the natural mapping.
	
	The relationship is likely more complex, potentially involving:
	\begin{itemize}
		\item M\"{o}bius logic as a specific model or interpretation of MLL or a fragment thereof.
		\item Both logics being conservative extensions of a common, more fundamental non-distributive substructural logic that is radically non-idempotent and lacks De Morgan duality.
		\item A non-standard translation or embedding that preserves certain structures but not others.
	\end{itemize}
	The semantic space $\mathbb{CP}^1$ with its complex arithmetic structure provides a concrete setting for exploring non-idempotent, paraconsistent, and paracomplete logic, offering a unique perspective compared to the more abstract categorical semantics of MLL.
	
\end{document}
