}

A theory T in standard formalization is a pair
(\textbf{Lang},\textbf{Axioms}), where:

\textbf{Lang} is a first-order language with the following components:

\begin{itemize}
\item
  \begin{quote}
  A set of variables V.
  \end{quote}
\item
  \begin{quote}
  A set of non-logical constants C.
  \end{quote}
\item
  \begin{quote}
  A set of logical constants, which includes the following symbols:
  \end{quote}

  \begin{itemize}
  \item
    \begin{quote}
    The connectives: ¬, ∨, ∧, →, ↔.
    \end{quote}
  \item
    \begin{quote}
    The quantifiers: ∀, ∃.
    \end{quote}
  \item
    \begin{quote}
    The identity symbol =.
    \end{quote}
  \end{itemize}
\item
  \begin{quote}
  A set of function symbols F, where each function symbol f has a
  natural number n as its rank, and for each n-tuple of terms t1, t2,
  ..., tn, the function symbol f applied to t1, t2, ..., tn is a term.
  \end{quote}
\item
  \begin{quote}
  A set of predicate symbols P, where each predicate symbol p has a
  natural number n as its rank, and for each n-tuple of terms t1, t2,
  ..., tn, the predicate symbol p applied to t1, t2, ..., tn is a
  formula.
  \end{quote}
\end{itemize}

A is a set of formulas in \textbf{L}. The formulas in \textbf{Axioms}
are called the axioms of T.

A theory T is said to be satisfiable if there exists a model M of T. A
model M of T is a structure M = (D, ⊦) such that:

\begin{itemize}
\item
  \begin{quote}
  D is a non-empty set called the domain of M.
  \end{quote}
\item
  \begin{quote}
  ⊦ is an interpretation function that assigns to each constant c in C
  an element d of D, to each function symbol f in F an n-ary function
  f\^{}M from D\^{}n to D, and to each predicate symbol p in P an n-ary
  relation p\^{}M over D.
  \end{quote}
\item
  \begin{quote}
  For each axiom α in \textbf{Axioms}, α is true in M, where "true in M"
  is defined inductively as follows:
  \end{quote}

  \begin{itemize}
  \item
    \begin{quote}
    Variables are interpreted as themselves.
    \end{quote}
  \item
    \begin{quote}
    ¬α is true in M if α is not true in M.
    \end{quote}
  \item
    \begin{quote}
    α ∨ β is true in M if α is true in M or β is true in M (or both).
    \end{quote}
  \item
    \begin{quote}
    α ∧ β is true in M if α is true in M and β is true in M.
    \end{quote}
  \item
    \begin{quote}
    α → β is true in M if α is not true in M or β is true in M.
    \end{quote}
  \item
    \begin{quote}
    ∀xα(x) is true in M if for all elements d of D, α(d) is true in M.
    \end{quote}
  \item
    \begin{quote}
    ∃xα(x) is true in M if there exists an element d of D such that α(d)
    is true in M.
    \end{quote}
  \end{itemize}
\end{itemize}

If there exists no model of T, then T is said to be unsatisfiable.

\end{document}}*}
