}

\hypertarget{manuscript-of-research-proposal-graph-of-supercalculi-conjugated-calculi-pairs-and-subcalculi}{%
\section{Manuscript of Research Proposal: Graph of Supercalculi,
conjugated calculi pairs, and
Subcalculi}\label{manuscript-of-research-proposal-graph-of-supercalculi-conjugated-calculi-pairs-and-subcalculi}}

\hypertarget{introduction--}{%
\subsection{\texorpdfstring{ \textbf{Introduction -}
}{ Introduction - }}\label{introduction--}}

This section may include:

§ What is to be done and the context of the project.

\begin{itemize}
\tightlist
\item
\end{itemize}

\begin{itemize}
\tightlist
\item
\end{itemize}

§ What is being done both generally and specifically in the same or
related areas. (The reviewer should know that you know what is going on
in the area in which you are proposing.)

§ An explanation and justification for unique or innovative approaches.
(These are selling points about what makes your project special, unique
and compelling and why it should be funded.)

\hypertarget{need-statement}{%
\subsection{\texorpdfstring{\textbf{Need
Statement}}{Need Statement}}\label{need-statement}}

§ What needs to be done and why?

§ What significant needs are you trying to meet? Compared to other
projects in the same area, what sets yours apart in terms of need?

§ What services are to be delivered? Why? Use specifics from preliminary
studies, needs assessment, documentation, and data supporting your
proposal.

§ What gaps that your work can fill exist in the knowledge base of your
field?

\textbf{Gaps in the knowledge base}

The current understanding of sequent calculi and their relationships is
fragmented and limited to specific systems or isolated aspects. There is
a lack of a comprehensive framework that encompasses the diverse
landscape of sequent calculi and their intricate connections. This
research aims to address this gap by constructing a comprehensive graph
of sequent calculi, providing a holistic view of the relationships and
properties within this domain.

\textbf{Significant needs}

The development of novel logical systems and the advancement of
automated reasoning techniques require a deeper understanding of the
structural and semantic foundations of sequent calculi. A comprehensive
graph of sequent calculi serves as a valuable tool for identifying
patterns, exploring uncharted territories, and deriving new theorems
within the realm of non-classical logic.

\textbf{Uniqueness of the approach}

This research distinguishes itself from existing work in its emphasis on
the graph-based representation of sequent calculi. This approach offers
a novel perspective on the relationships between different systems and
facilitates the identification of hidden connections and patterns.
Additionally, the focus on non-classical logics expands the scope of the
research, addressing a less explored area of sequent calculi.

\textbf{Compared to other projects in the same area, what sets yours
apart in terms of need?}

There have been several other projects that have proposed graph-based
representations of sequent calculi. However, these projects have
typically focused on specific aspects of sequent calculi, such as their
structural properties or their computational complexity. This research
proposal aims to develop a more comprehensive graph-based representation
that can capture the full range of features and relationships between
different sequent calculi.

In addition, this research proposal will develop a formal semantics for
the graph-based representation, ensuring that it is consistent with the
traditional semantics of sequent calculi. This will make the graph-based
representation a more powerful tool for understanding and analyzing
sequent calculi.

Finally, this research proposal will explore the expressiveness and
limitations of the graph-based representation, identifying the types of
logical systems that can be effectively represented and the limitations
of this approach. This will provide valuable information for researchers
who are considering using the graph-based representation for their own
work.

\textbf{Overall, this research addresses a significant gap in the
knowledge base of sequent calculi by providing a comprehensive and novel
representation that can be leveraged to address important needs in the
field of logic and automated reasoning.}

§ Is the problem both significant and manageable? Do you have the
resources to handle the problem?

\hypertarget{goals-and-objectives}{%
\subsection{\texorpdfstring{\textbf{Goals and
Objectives}}{Goals and Objectives}}\label{goals-and-objectives}}

\textbf{Goal:} Establish a foundational understanding of the graph of
sequent calculi for non-classical logics

\textbf{Objectives:}

1.1.1 Construct a comprehensive graph of sequent calculi for
non-classical logics, capturing the structural and semantic
relationships between different systems.

1.1.2 Analyze the graph to identify patterns, regularities, and
connections that characterize the landscape of non-classical logics.

1.1.3 Develop a framework for navigating and interpreting the graph,
enabling researchers to effectively explore and utilize this
representation of non-classical logics.

\textbf{Activities:}

1.1.1.1 Gather and analyze existing sequent calculi for non-classical
logics, encompassing a diverse range of systems and formalisms.

1.1.1.2 Identify the formal relationships between different sequent
calculi, considering structural similarities, semantic equivalences, and
embedding possibilities.

1.1.1.3 Represent the identified relationships as nodes and edges in a
graph, creating a comprehensive visual representation of the landscape
of non-classical logics.

1.1.2.1 Analyze the graph structure to identify patterns, regularities,
and connections that reveal underlying principles and relationships
within the realm of non-classical logics.

1.1.2.2 Investigate the implications of the identified patterns and
relationships for the expressiveness, computational complexity, and
decidability of non-classical logics.

1.1.2.3 Explore the potential of the graph to guide the development of
new non-classical logics with tailored properties and applications.

1.1.3.1 Develop a formal framework for interpreting and navigating the
graph, providing a structured approach to extracting information and
insights from this representation.

1.1.3.2 Define graph-based metrics and measures to characterize the
structural and semantic features of non-classical logics represented in
the graph.

1.1.3.3 Create a user-friendly interface or tool that facilitates the
exploration and analysis of the graph, enabling researchers to
effectively utilize this resource.

\textbf{Measurement:}

The success of this research will be evaluated based on the following
criteria:

\begin{itemize}
\item
  \begin{quote}
  Comprehensiveness of the constructed graph of sequent calculi for
  non-classical logics
  \end{quote}
\item
  \begin{quote}
  Identification of novel patterns, regularities, and connections within
  the graph
  \end{quote}
\item
  \begin{quote}
  Development of a comprehensive framework for interpreting and
  navigating the graph
  \end{quote}
\item
  \begin{quote}
  Creation of graph-based metrics and measures for characterizing
  non-classical logics
  \end{quote}
\item
  \begin{quote}
  Development of a user-friendly interface or tool for exploring and
  analyzing the graph
  \end{quote}
\end{itemize}

\textbf{Outcomes:}

\begin{itemize}
\item
  \begin{quote}
  A comprehensive graph of sequent calculi for non-classical logics,
  providing a visual representation of the relationships between
  different systems
  \end{quote}
\item
  \begin{quote}
  A deeper understanding of the structural and semantic landscape of
  non-classical logics, revealed through the analysis of the graph
  \end{quote}
\item
  \begin{quote}
  A framework for navigating and interpreting the graph, enabling
  researchers to effectively explore and utilize this representation
  \end{quote}
\item
  \begin{quote}
  Graph-based metrics and measures for characterizing non-classical
  logics
  \end{quote}
\item
  \begin{quote}
  A user-friendly interface or tool for exploring and analyzing the
  graph
  \end{quote}
\end{itemize}

\textbf{Products:}

\begin{itemize}
\item
  \begin{quote}
  A research paper presenting the constructed graph, its analysis, and
  the developed framework for interpretation and navigation
  \end{quote}
\item
  \begin{quote}
  Open-source software or tools for visualizing and analyzing the graph
  \end{quote}
\item
  \begin{quote}
  Presentations at conferences and workshops to disseminate the findings
  and promote the utilization of the graph
  \end{quote}
\end{itemize}

\textbf{Goal:} Establish a comprehensive framework for identifying and
classifying connections between sequent calculi

\textbf{Objectives:}

2.1.1 Develop a formal taxonomy of connections between sequent calculi,
encompassing various types such as extensions, subtensions, conjugate
duals, hyperextensions, hyposubtensions, supercalculi, subcalculi,
hypocalculi, hypercalculi, and hyperduals.

2.1.2 Design a systematic methodology for identifying and classifying
connections between sequent calculi based on their structural and
semantic features.

2.1.3 Implement a computational tool or algorithm that automates the
identification and classification of connections between sequent
calculi.

\textbf{Activities:}

2.1.1.1 Analyze the existing literature on connections between sequent
calculi to identify and categorize the different types of connections.

2.1.1.2 Define formal criteria and properties for each type of
connection, ensuring a clear and consistent classification scheme.

2.1.1.3 Construct a hierarchical taxonomy of connections, representing
the relationships and distinctions between different types.

2.1.2.1 Develop a systematic framework for analyzing the structural and
semantic features of sequent calculi to identify potential connections.

2.1.2.2 Define formal procedures for identifying and classifying
connections based on the identified structural and semantic features.

2.1.2.3 Validate the proposed methodology by applying it to a range of
examples, ensuring its effectiveness and generality.

2.1.3.1 Design and implement a computational tool or algorithm that
automates the identification and classification of connections based on
the developed methodology.

2.1.3.2 Test and evaluate the computational tool or algorithm using a
comprehensive set of examples, ensuring its accuracy and efficiency.

2.1.3.3 Develop a user-friendly interface or tool that facilitates the
utilization of the computational tool or algorithm for researchers and
practitioners.

\textbf{Measurement:}

The success of this research will be evaluated based on the following
criteria:

\begin{itemize}
\item
  \begin{quote}
  Comprehensiveness and clarity of the proposed taxonomy of connections
  between sequent calculi
  \end{quote}
\item
  \begin{quote}
  Effectiveness and generality of the systematic methodology for
  identifying and classifying connections
  \end{quote}
\item
  \begin{quote}
  Accuracy and efficiency of the computational tool or algorithm for
  automating the identification and classification of connections
  \end{quote}
\item
  \begin{quote}
  Usability and accessibility of the user-friendly interface or tool for
  researchers and practitioners
  \end{quote}
\end{itemize}

\textbf{Outcomes:}

\begin{itemize}
\item
  \begin{quote}
  A formal taxonomy of connections between sequent calculi, providing a
  structured and comprehensive classification of different types of
  connections
  \end{quote}
\item
  \begin{quote}
  A systematic methodology for identifying and classifying connections
  between sequent calculi, enabling researchers to effectively analyze
  and categorize relationships between different systems
  \end{quote}
\item
  \begin{quote}
  A computational tool or algorithm that automates the identification
  and classification of connections between sequent calculi,
  streamlining the process and making it more accessible
  \end{quote}
\item
  \begin{quote}
  A user-friendly interface or tool that facilitates the utilization of
  the computational tool or algorithm, providing researchers and
  practitioners with a convenient and accessible resource
  \end{quote}
\end{itemize}

\textbf{Products:}

\begin{itemize}
\item
  \begin{quote}
  A research paper presenting the proposed taxonomy, methodology, and
  computational tool or algorithm
  \end{quote}
\item
  \begin{quote}
  Open-source software or tools for identifying and classifying
  connections between sequent calculi
  \end{quote}
\item
  \begin{quote}
  Presentations at conferences and workshops to disseminate the findings
  and promote the utilization of the tools
  \end{quote}
\end{itemize}

\textbf{Goal:} Develop a systematic approach for designing novel logical
calculi with tailored properties

\textbf{Objectives:}

3.1 Establish a framework for identifying and classifying the desired
properties of novel logical calculi, considering factors such as
expressiveness, computational complexity, decidability, and
applications.

3.2 Develop a methodological framework for designing logical calculi
with specific properties, utilizing graph-based representations and
structural analysis techniques.

3.3 Implement a computational tool or algorithm that automates the
design and verification of logical calculi with tailored properties.

\textbf{Activities:}

3.1.1 Analyze existing logical calculi and their properties to identify
relationships between properties and structural features.

3.1.2 Define formal criteria and measures for evaluating the
expressiveness, computational complexity, decidability, and
applicability of logical calculi.

3.1.3 Develop a taxonomy of logical calculi based on their properties,
enabling researchers to systematically explore and categorize different
systems.

3.2.1 Design graph-based representations of logical calculi, capturing
their structural relationships and highlighting potential
property-determining features.

3.2.2 Develop structural analysis techniques for identifying and
characterizing properties of logical calculi based on their graph
representations.

3.2.3 Establish a systematic methodology for designing logical calculi
with specific properties, utilizing graph-based representations and
structural analysis techniques.

3.3.1 Design and implement a computational tool or algorithm that
automates the design and verification of logical calculi with tailored
properties.

3.1.3.2 Test and evaluate the computational tool or algorithm using a
comprehensive set of examples, ensuring its accuracy and effectiveness
in designing and verifying logical calculi with specific properties.

3.3.3 Develop a user-friendly interface or tool that facilitates the
utilization of the computational tool or algorithm for researchers and
practitioners.

\textbf{Measurement:}

The success of this research will be evaluated based on the following
criteria:

\begin{itemize}
\item
  \begin{quote}
  Comprehensiveness and clarity of the proposed framework for
  identifying and classifying properties of logical calculi
  \end{quote}
\item
  \begin{quote}
  Effectiveness and generality of the methodological framework for
  designing logical calculi with specific properties
  \end{quote}
\item
  \begin{quote}
  Accuracy and efficiency of the computational tool or algorithm for
  automating the design and verification of logical calculi with
  tailored properties
  \end{quote}
\item
  \begin{quote}
  Usability and accessibility of the user-friendly interface or tool for
  researchers and practitioners
  \end{quote}
\end{itemize}

\textbf{Outcomes:}

\begin{itemize}
\item
  \begin{quote}
  A formal framework for identifying and classifying the properties of
  logical calculi, providing a structured and comprehensive approach to
  understanding and characterizing different types of properties
  \end{quote}
\item
  \begin{quote}
  A methodological framework for designing logical calculi with specific
  properties, enabling researchers to systematically design and develop
  logical systems with tailored characteristics
  \end{quote}
\item
  \begin{quote}
  A computational tool or algorithm that automates the design and
  verification of logical calculi with tailored properties, streamlining
  the process and making it more accessible
  \end{quote}
\item
  \begin{quote}
  A user-friendly interface or tool that facilitates the utilization of
  the computational tool or algorithm, providing researchers and
  practitioners with a convenient and accessible resource for designing
  logical systems with specific properties
  \end{quote}
\end{itemize}

\textbf{Products:}

\begin{itemize}
\item
  \begin{quote}
  A research paper presenting the proposed framework, methodology, and
  computational tool or algorithm
  \end{quote}
\item
  \begin{quote}
  Open-source software or tools for designing and verifying logical
  calculi with tailored properties
  \end{quote}
\item
  \begin{quote}
  Presentations at conferences and workshops to disseminate the findings
  and promote the utilization of the tools
  \end{quote}
\end{itemize}

\textbf{Goal:} Explore and demonstrate the practical applications of the
graph of sequent calculi

\textbf{Objectives:}

4.1.1 Identify potential applications of the graph of sequent calculi in
various domains, including formal verification, automated reasoning,
proof search, and logical programming.

4.1.2 Develop concrete use cases and examples that showcase the
applicability of the graph in these domains, demonstrating its ability
to solve practical problems and enhance existing techniques.

4.1.3 Evaluate the effectiveness and efficiency of the graph-based
approach compared to traditional methods in addressing specific
application scenarios.

\textbf{Activities:}

4.1.1.1 Conduct a comprehensive survey of existing applications of
sequent calculi and related techniques to identify potential areas where
the graph representation can be applied.

4.1.1.2 Collaborate with experts from different domains, such as
software engineering, artificial intelligence, and computational
linguistics, to explore potential applications in their respective
fields.

4.1.1.3 Analyze the requirements and challenges of specific application
domains to identify how the graph can be effectively utilized to address
those challenges.

4.1.2.1 Develop concrete use cases for each identified application
domain, providing detailed descriptions of how the graph can be used to
solve specific problems or enhance existing techniques.

4.1.2.2 Implement prototypes or tools that demonstrate the practicality
of the graph-based approach in the selected use cases.

4.1.2.3 Evaluate the performance and effectiveness of the graph-based
approach compared to traditional methods in the chosen use cases.

4.1.3.1 Conduct rigorous testing and evaluation of the graph-based
approach on a variety of benchmark problems and real-world applications.

4.1.3.2 Compare the computational complexity and resource requirements
of the graph-based approach to traditional methods.

4.1.3.3 Analyze the qualitative aspects of the graph-based approach,
such as its ability to provide insights, identify patterns, and simplify
reasoning processes.

\textbf{Measurement:}

The success of this research will be evaluated based on the following
criteria:

\begin{itemize}
\item
  \begin{quote}
  Breadth and diversity of identified applications of the graph of
  sequent calculi
  \end{quote}
\item
  \begin{quote}
  Practicality and effectiveness of the graph-based approach in solving
  real-world problems
  \end{quote}
\item
  \begin{quote}
  Performance and efficiency of the graph-based approach compared to
  traditional methods
  \end{quote}
\item
  \begin{quote}
  Qualitative benefits of the graph-based approach, such as its ability
  to provide insights, identify patterns, and simplify reasoning
  processes
  \end{quote}
\end{itemize}

\textbf{Outcomes:}

\begin{itemize}
\item
  \begin{quote}
  A comprehensive catalog of potential applications of the graph of
  sequent calculi in various domains
  \end{quote}
\item
  \begin{quote}
  Concrete use cases and examples that demonstrate the applicability of
  the graph in solving practical problems and enhancing existing
  techniques
  \end{quote}
\item
  \begin{quote}
  Benchmark results and performance evaluations comparing the
  graph-based approach to traditional methods
  \end{quote}
\item
  \begin{quote}
  Case studies and real-world applications that showcase the
  effectiveness and practicality of the graph-based approach
  \end{quote}
\end{itemize}

\textbf{Products:}

\begin{itemize}
\item
  \begin{quote}
  A research paper presenting the identified applications, use cases,
  evaluations, and case studies
  \end{quote}
\item
  \begin{quote}
  Open-source software or tools that demonstrate the graph-based
  approach in specific application domains
  \end{quote}
\item
  \begin{quote}
  Presentations at conferences and workshops to disseminate the findings
  and promote the adoption of the graph-based approach
  \end{quote}
\end{itemize}

\textbf{Goal:} Establish a Formal Foundation for Graph-Based
Representations of Sequent Calculi

\textbf{Objectives:}

5.1.1 Develop a rigorous mathematical framework for representing sequent
calculi using graph theory concepts.

5.1.2 Define formal semantics for the graph-based representation of
sequent calculi, ensuring consistency with the traditional semantics of
sequent calculi.

5.1.3 Explore the expressiveness and limitations of the graph-based
representation, identifying the types of logical systems that can be
effectively represented and the limitations of this approach.

\textbf{Activities:}

5.1.1.1 Analyze the structural and semantic properties of sequent
calculi to identify the key aspects that can be captured by graph-based
representations.

5.1.1.2 Define graph-theoretic constructs and notations to represent the
logical rules, formulas, and derivations of sequent calculi.

5.1.1.3 Formalize the relationships between graph-theoretic constructs
and the corresponding components of sequent calculi.

5.1.2.1 Develop a formal interpretation of graph-based representations
of sequent calculi, translating graph structures into logical formulas
and derivations.

5.1.2.2 Verify the consistency of the graph-based semantics with the
traditional semantics of sequent calculi, ensuring that the two
approaches produce equivalent results.

5.1.2.3 Explore the implications of the graph-based semantics for the
logical properties of sequent calculi, such as consistency,
completeness, and decidability.

5.1.3.1 Investigate the range of logical systems that can be effectively
represented using the graph-based approach, considering the
expressiveness and limitations of graph theory.

5.1.3.2 Analyze the computational complexity of reasoning tasks, such as
proof search and verification, within the graph-based representation.

5.1.3.3 Identify potential applications of the graph-based approach in
various areas of formal logic and computer science, such as automated
reasoning, metamathematics, and program analysis.

\textbf{Measurement:}

The success of this research will be evaluated based on the following
criteria:

\begin{itemize}
\item
  \begin{quote}
  Clarity and formality of the proposed mathematical framework for
  graph-based representations of sequent calculi
  \end{quote}
\item
  \begin{quote}
  Consistency and equivalence of the graph-based semantics with the
  traditional semantics of sequent calculi
  \end{quote}
\item
  \begin{quote}
  Expressiveness and limitations of the graph-based representation in
  capturing various types of logical systems
  \end{quote}
\item
  \begin{quote}
  Impact of the graph-based approach on computational complexity and
  reasoning tasks in formal logic
  \end{quote}
\item
  \begin{quote}
  Applicability of the graph-based approach in diverse areas of formal
  logic and computer science
  \end{quote}
\end{itemize}

\textbf{Outcomes:}

\begin{itemize}
\item
  \begin{quote}
  A formal mathematical framework for representing sequent calculi using
  graph theory concepts
  \end{quote}
\item
  \begin{quote}
  A formal semantics for the graph-based representation of sequent
  calculi, ensuring consistency with traditional semantics
  \end{quote}
\item
  \begin{quote}
  An analysis of the expressiveness and limitations of the graph-based
  representation in capturing various types of logical systems
  \end{quote}
\item
  \begin{quote}
  Identification of potential applications of the graph-based approach
  in formal logic and computer science
  \end{quote}
\end{itemize}

\textbf{Products:}

\begin{itemize}
\item
  \begin{quote}
  A research paper presenting the proposed framework, semantics,
  analysis, and potential applications
  \end{quote}
\item
  \begin{quote}
  Open-source software or tools for implementing and analyzing
  graph-based representations of sequent calculi
  \end{quote}
\item
  \begin{quote}
  Presentations at conferences and workshops to disseminate the findings
  and promote the adoption of the graph-based approach
  \end{quote}
\end{itemize}

Deliverables:

A comprehensive list of non-classical logics, including their axioms,
rules, and properties.

A new formal framework for representing and reasoning about
non-classical logics.

A new classification of non-classical logics.

A graph of sequent calculi that visually represents the relationships
between these logics.

A list of common patterns and themes in the properties of non-classical
logics.

A list of new areas of research suggested by the analysis of the graph
of sequent calculi.

\hypertarget{section}{%
\subsubsection{}\label{section}}

\hypertarget{section-1}{%
\subsubsection{}\label{section-1}}

\hypertarget{section-2}{%
\subsubsection{}\label{section-2}}

\hypertarget{goal-1-investigating-the-structural-properties-of-sequent-calculi-using-graphs}{%
\subsubsection{Goal 1: Investigating the Structural Properties of
Sequent Calculi Using
Graphs}\label{goal-1-investigating-the-structural-properties-of-sequent-calculi-using-graphs}}

\textbf{Subgoal 1.1: Develop a formal framework for representing sequent
calculi using graphs}

\begin{itemize}
\item
  \begin{quote}
  \textbf{Activities:\\
  }
  \end{quote}

  \begin{itemize}
  \item
    \begin{quote}
    \textbf{Conduct a thorough review of existing formal frameworks for
    representing logical systems, including graph-based approaches and
    algebraic representations.}
    \end{quote}
  \item
    \begin{quote}
    \textbf{Identify the key properties and requirements for a formal
    framework specifically tailored to sequent calculi, considering
    their unique structural and semantic features.}
    \end{quote}
  \item
    \begin{quote}
    \textbf{Design a novel graph-based formal framework that effectively
    captures the structure and semantics of sequent calculi, including
    their inferential rules, proof structures, and logical
    relationships.}
    \end{quote}
  \item
    \begin{quote}
    \textbf{Evaluate the expressiveness and flexibility of the proposed
    framework through the representation of a variety of sequent
    calculi, encompassing both well-established systems such as
    classical and intuitionistic sequent calculus, and newly developed
    sequent calculi with unique features.}
    \end{quote}
  \end{itemize}
\item
  \begin{quote}
  \textbf{How the activities will be carried out:\\
  }
  \end{quote}

  \begin{itemize}
  \item
    \begin{quote}
    \textbf{The review of existing frameworks will involve examining
    academic literature, attending conferences, and consulting with
    experts in the field of sequent calculus and formal logic.}
    \end{quote}
  \item
    \begin{quote}
    \textbf{The identification of key properties and requirements will
    involve analyzing the characteristics of sequent calculi,
    considering their inferential mechanisms, proof structures, and
    logical properties, and the needs of researchers working with these
    systems.}
    \end{quote}
  \item
    \begin{quote}
    \textbf{The design of the novel framework will involve utilizing
    graph theory concepts, formal logic principles, and software
    engineering techniques, ensuring the framework can effectively
    represent the intricate relationships and interactions within
    sequent calculi.}
    \end{quote}
  \item
    \begin{quote}
    \textbf{The evaluation of the framework will involve representing a
    range of sequent calculi, including both well-established systems
    and newly developed ones, to assess its ability to capture their
    structural and semantic features.}
    \end{quote}
  \end{itemize}
\end{itemize}

\textbf{Subgoal 1.2: Analyze the structural properties of sequent
calculi using graph-based techniques}

\begin{itemize}
\item
  \begin{quote}
  \textbf{Activities:\\
  }
  \end{quote}

  \begin{itemize}
  \item
    \begin{quote}
    \textbf{Apply the developed graph-based framework to analyze the
    structural properties of a range of sequent calculi, including their
    connectivity, symmetries, decomposition patterns, and relationships
    between different proof structures.}
    \end{quote}
  \item
    \begin{quote}
    \textbf{Identify and characterize common structural motifs and
    patterns that emerge across different sequent calculi, providing
    insights into their underlying computational properties and logical
    relationships.}
    \end{quote}
  \item
    \begin{quote}
    \textbf{Develop formal theorems that establish connections between
    the structural properties of sequent calculi and their logical
    properties, such as soundness, completeness, and decidability.}
    \end{quote}
  \end{itemize}
\item
  \begin{quote}
  \textbf{How the activities will be carried out:\\
  }
  \end{quote}

  \begin{itemize}
  \item
    \begin{quote}
    \textbf{The analysis of structural properties will involve employing
    graph algorithms and techniques, such as graph isomorphism, subgraph
    detection, and graph decomposition, to examine the graph-based
    representations of sequent calculi.}
    \end{quote}
  \item
    \begin{quote}
    \textbf{The identification of common patterns will involve utilizing
    statistical analysis and pattern recognition methods to identify
    recurring structural motifs and patterns across different sequent
    calculi.}
    \end{quote}
  \item
    \begin{quote}
    \textbf{The development of formal theorems will involve rigorous
    mathematical proofs based on the formal framework, graph theory
    concepts, and logical reasoning principles, establishing connections
    between structural features and logical properties.}
    \end{quote}
  \end{itemize}
\end{itemize}

\textbf{1.3: Develop graph-based algorithms for analyzing the structural
properties of sequent calculi}

\begin{itemize}
\item
  \begin{quote}
  \textbf{Activities:\\
  }
  \end{quote}

  \begin{itemize}
  \item
    \begin{quote}
    \textbf{Design and implement efficient graph algorithms specifically
    tailored to the analysis of sequent calculi, leveraging graph theory
    concepts and formal logic principles.}
    \end{quote}
  \item
    \begin{quote}
    \textbf{Optimize the developed algorithms for performance and
    scalability, considering the potentially large size and complexity
    of sequent calculus representations.}
    \end{quote}
  \item
    \begin{quote}
    \textbf{Integrate the algorithms into a software framework or
    toolset for analyzing sequent calculi, providing researchers with a
    user-friendly environment for exploring structural properties.}
    \end{quote}
  \end{itemize}
\item
  \begin{quote}
  \textbf{How the activities will be carried out:\\
  }
  \end{quote}

  \begin{itemize}
  \item
    \begin{quote}
    \textbf{The design of algorithms will involve utilizing graph theory
    techniques such as graph traversal, pattern matching, and graph
    decomposition, tailoring them to the specific structure and
    semantics of sequent calculi.}
    \end{quote}
  \item
    \begin{quote}
    \textbf{The optimization of algorithms will involve employing
    techniques such as caching, data structures, and parallel
    processing, to improve their execution time and memory usage.}
    \end{quote}
  \item
    \begin{quote}
    \textbf{The integration of algorithms into a software framework will
    involve utilizing software engineering principles, modular design,
    and user interface considerations, to create a user-friendly and
    extensible toolset.}
    \end{quote}
  \end{itemize}
\end{itemize}

\textbf{Subgoal 1.4: Apply graph-based algorithms to investigate the
structural properties of sequent calculi}

\begin{itemize}
\item
  \begin{quote}
  \textbf{Activities:\\
  }
  \end{quote}

  \begin{itemize}
  \item
    \begin{quote}
    \textbf{Utilize the developed graph-based algorithms to analyze the
    structural properties of a range of sequent calculi, including their
    connectivity, symmetries, decomposition patterns, and relationships
    between different proof structures.}
    \end{quote}
  \item
    \begin{quote}
    \textbf{Investigate the impact of structural properties on the
    computational behavior of sequent calculi, such as proof search
    efficiency and the complexity of proof structures.}
    \end{quote}
  \item
    \begin{quote}
    \textbf{Explore the application of graph-based techniques to the
    analysis of sequent calculus extensions and variations, such as
    modal sequent calculi and substructural sequent calculi.}
    \end{quote}
  \end{itemize}
\item
  \begin{quote}
  \textbf{How the activities will be carried out:\\
  }
  \end{quote}

  \begin{itemize}
  \item
    \begin{quote}
    \textbf{The application of algorithms will involve executing the
    developed algorithms on the graph-based representations of sequent
    calculi, collecting and analyzing the resulting data.}
    \end{quote}
  \item
    \begin{quote}
    \textbf{The investigation of the impact of structural properties
    will involve correlating the identified structural features with the
    computational behavior of sequent calculi, using metrics such as
    proof length and search space size.}
    \end{quote}
  \item
    \begin{quote}
    \textbf{The exploration of extensions and variations will involve
    applying the graph-based techniques to the representations of these
    systems, analyzing their structural properties and identifying
    patterns.}
    \end{quote}
  \end{itemize}
\end{itemize}

\begin{itemize}
\tightlist
\item
\end{itemize}

\textbf{Goal: Construct a comprehensive graph of sequent calculi and
leverage it to identify uncharted non-classical languages and derive
novel theorems within the realm of non-classical logic.}

\textbf{Objectives:}

\textbf{1.5.1 Construct a comprehensive graph of sequent calculi.}

\textbf{1.5.2 Identify uncharted non-classical languages based on the
graph of sequent calculi.}

\textbf{1.5.3 Derive novel theorems within the realm of non-classical
logic using the graph of sequent calculi.}

\textbf{Activities:}

\textbf{1.5.1.1 Gather and analyze existing sequent calculi.}

\textbf{1.5.1.2 Identify the relationships between different sequent
calculi.}

\textbf{1.5.1.3 Represent the relationships between sequent calculi as a
graph.}

\textbf{1.5.2.1 Analyze the graph of sequent calculi to identify
potential new non-classical languages.}

\textbf{1.5.2.2 Formulate and investigate new non-classical languages
based on the identified potential.}

\textbf{1.5.3.1 Extract novel theorems from the graph of sequent
calculi.}

\textbf{1.5.3.2 Prove the validity of the extracted theorems.}

\textbf{Evaluation:}

\textbf{The success of this research will be evaluated based on the
following criteria:}

\begin{itemize}
\item
  \begin{quote}
  \textbf{The comprehensiveness of the graph of sequent calculi.\\
  }
  \end{quote}
\item
  \begin{quote}
  \textbf{The number of new non-classical languages identified.\\
  }
  \end{quote}
\item
  \begin{quote}
  \textbf{The novelty and significance of the derived theorems.\\
  }
  \end{quote}
\end{itemize}

\textbf{Timeline:}

\textbf{1.5.1.1 Gather and analyze existing sequent calculi: 6 months}

\textbf{1.5.1.2 Identify the relationships between different sequent
calculi: 3 months}

\textbf{1.5.1.3 Represent the relationships between sequent calculi as a
graph: 3 months}

\textbf{1.5.2.1 Analyze the graph of sequent calculi to identify
potential new non-classical languages: 6 months}

\textbf{1.5.2.2 Formulate and investigate new non-classical languages
based on the identified potential: 12 months}

\textbf{1.5.3.1 Extract novel theorems from the graph of sequent
calculi: 6 months}

\textbf{1.5.3.2 Prove the validity of the extracted theorems: 12 months}

\textbf{Resources:}

\textbf{The research will require the following resources:}

\begin{itemize}
\item
  \begin{quote}
  \textbf{Access to academic literature and databases\\
  }
  \end{quote}
\item
  \begin{quote}
  \textbf{Travel funds to attend conferences\\
  }
  \end{quote}
\item
  \begin{quote}
  \textbf{Collaborations with experts in the field of sequent calculi
  and non-classical logic\\
  }
  \end{quote}
\end{itemize}

\textbf{Expected Outcomes:}

\begin{itemize}
\item
  \begin{quote}
  \textbf{A comprehensive graph of sequent calculi\\
  }
  \end{quote}
\item
  \begin{quote}
  \textbf{Identification of uncharted non-classical languages\\
  }
  \end{quote}
\item
  \begin{quote}
  \textbf{Derivation of novel theorems within the realm of non-classical
  logic\\
  }
  \end{quote}
\end{itemize}

\textbf{Dissemination:}

\textbf{The findings of this research will be disseminated through
publications in peer-reviewed journals, presentations at conferences,
and open-source software development.}

\hypertarget{goal-leverage-the-graph-of-sequent-calculi-to-identify-uncharted-non-classical-languages-and-derive-novel-theorems-within-the-domain-of-non-classical-logic.}{%
\paragraph{Goal: Leverage the graph of sequent calculi to identify
uncharted non-classical languages and derive novel theorems within the
domain of non-classical
logic.}\label{goal-leverage-the-graph-of-sequent-calculi-to-identify-uncharted-non-classical-languages-and-derive-novel-theorems-within-the-domain-of-non-classical-logic.}}

\textbf{Objectives:}

\textbf{1.6.1 Identify uncharted non-classical languages based on the
graph of sequent calculi.}

\textbf{1.6.2 Derive novel theorems within the realm of non-classical
logic using the graph of sequent calculi.}

\textbf{Activities:}

\textbf{1.6.1.1 Analyze the graph of sequent calculi to identify
structural patterns and relationships that suggest the existence of
uncharted non-classical languages.}

\textbf{1.6.1.2 Formulate hypotheses regarding the characteristics and
properties of potential uncharted non-classical languages based on the
identified patterns and relationships.}

\textbf{1.6.1.3 Investigate the formulated hypotheses by constructing
formal representations and analyzing the logical properties of the
proposed uncharted non-classical languages.}

\textbf{1.6.2.1 Identify promising areas within the graph of sequent
calculi that exhibit potential for deriving novel theorems in
non-classical logic.}

\textbf{1.6.2.2 Extract potential theorems from the identified promising
areas by systematically examining the relationships and properties of
sequent calculi within those regions.}

\textbf{1.6.2.3 Formalize and prove the extracted potential theorems
using rigorous mathematical techniques and logical reasoning.}

\textbf{Evaluation:}

\textbf{The success of this research will be evaluated based on the
following criteria:}

\begin{itemize}
\item
  \begin{quote}
  \textbf{The number of uncharted non-classical languages identified.\\
  }
  \end{quote}
\item
  \begin{quote}
  \textbf{The novelty and significance of the identified uncharted
  non-classical languages.\\
  }
  \end{quote}
\item
  \begin{quote}
  \textbf{The number of novel theorems derived in non-classical logic.\\
  }
  \end{quote}
\item
  \begin{quote}
  \textbf{The novelty and significance of the derived novel theorems.\\
  }
  \end{quote}
\end{itemize}

\textbf{Timeline:}

\textbf{1.6.1.1 Analyze the graph of sequent calculi to identify
structural patterns and relationships: 3 months}

\textbf{1.6.1.2 Formulate hypotheses regarding the characteristics and
properties of potential uncharted non-classical languages: 3 months}

\textbf{1.6.1.3 Investigate the formulated hypotheses: 12 months}

\textbf{1.6.2.1 Identify promising areas within the graph of sequent
calculi for deriving novel theorems: 3 months}

\textbf{1.6.2.2 Extract potential theorems from the identified promising
areas: 3 months}

\textbf{1.6.2.3 Formalize and prove the extracted potential theorems: 12
months}

\textbf{Resources:}

\textbf{The research will require the following resources:}

\begin{itemize}
\item
  \begin{quote}
  \textbf{Access to academic literature and databases\\
  }
  \end{quote}
\item
  \begin{quote}
  \textbf{Collaborations with experts in the field of sequent calculi
  and non-classical logic\\
  }
  \end{quote}
\item
  \begin{quote}
  \textbf{Access to computational resources for graph analysis and
  theorem proving\\
  }
  \end{quote}
\end{itemize}

\textbf{Expected Outcomes:}

\begin{itemize}
\item
  \begin{quote}
  \textbf{Identification of uncharted non-classical languages with
  unique properties and applications\\
  }
  \end{quote}
\item
  \begin{quote}
  \textbf{Derivation of novel theorems that expand our understanding of
  non-classical logic and its capabilities\\
  }
  \end{quote}
\end{itemize}

\textbf{Dissemination:}

\textbf{The findings of this research will be disseminated through
publications in peer-reviewed journals, presentations at conferences,
and open-source software development.}

\hypertarget{goal-2-investigating-the-relationship-between-sequent-calculi-and-substructural-logics}{%
\subsubsection{Goal 2: Investigating the Relationship between Sequent
Calculi and Substructural
Logics}\label{goal-2-investigating-the-relationship-between-sequent-calculi-and-substructural-logics}}

\textbf{Subgoal 2.1: Develop a formal framework for representing
conjugated logic pairs using graphs}

\begin{itemize}
\item
  \begin{quote}
  \textbf{Activities:\\
  }
  \end{quote}

  \begin{enumerate}
  \def\labelenumi{\arabic{enumi}.}
  \item
    \begin{quote}
    \textbf{Conduct a comprehensive review of existing formal frameworks
    for representing logical systems, including graph-based approaches
    and algebraic representations, with a particular focus on conjugated
    logic pairs.\\
    }
    \end{quote}
  \item
    \begin{quote}
    \textbf{Identify the key properties and requirements for a formal
    framework specifically tailored to conjugated logic pairs,
    considering their unique structural and semantic features.\\
    }
    \end{quote}
  \item
    \begin{quote}
    \textbf{Design a novel graph-based formal framework that effectively
    captures the intricate structure and semantics of conjugated logic
    pairs, including their ability to handle both classical and
    non-classical logics.\\
    }
    \end{quote}
  \item
    \begin{quote}
    \textbf{Evaluate the expressiveness and flexibility of the proposed
    framework through the representation of a variety of conjugated
    logic pairs, encompassing both well-established systems and newly
    developed ones.\\
    }
    \end{quote}
  \end{enumerate}
\item
  \begin{quote}
  \textbf{How the activities will be carried out:\\
  }
  \end{quote}

  \begin{enumerate}
  \def\labelenumi{\arabic{enumi}.}
  \item
    \begin{quote}
    \textbf{The review of existing frameworks will involve examining
    academic literature, attending conferences, and consulting with
    experts in the field of conjugated logic pairs and formal logic.\\
    }
    \end{quote}
  \item
    \begin{quote}
    \textbf{The identification of key properties and requirements will
    involve analyzing the characteristics of conjugated logic pairs,
    considering their ability to express both classical and
    non-classical logics, and the needs of researchers working with
    these systems.\\
    }
    \end{quote}
  \item
    \begin{quote}
    \textbf{The design of the novel framework will involve utilizing
    graph theory concepts, formal logic principles, and software
    engineering techniques, ensuring the framework can effectively
    represent the complex relationships and interactions within
    conjugated logic pairs.\\
    }
    \end{quote}
  \item
    \begin{quote}
    \textbf{The evaluation of the framework will involve representing a
    range of conjugated logic pairs, including both well-established
    systems such as classical logic and intuitionistic logic paired with
    their non-classical counterparts, and newly developed conjugated
    logic pairs with unique features.\\
    }
    \end{quote}
  \end{enumerate}
\end{itemize}

\textbf{Subgoal 2.2: Establish a connection between the graph-based
representations of sequent calculi and conjugated logic pairs}

\begin{itemize}
\item
  \begin{quote}
  \textbf{Activities:\\
  }
  \end{quote}

  \begin{enumerate}
  \def\labelenumi{\arabic{enumi}.}
  \item
    \begin{quote}
    \textbf{Analyze the graphical representations of sequent calculi and
    conjugated logic pairs to identify structural similarities and
    correspondences.\\
    }
    \end{quote}
  \item
    \begin{quote}
    \textbf{Develop formal mappings between the graph-based
    representations of sequent calculi and conjugated logic pairs,
    preserving their structural and semantic properties.\\
    }
    \end{quote}
  \item
    \begin{quote}
    \textbf{Utilize the established mappings to translate proofs and
    theorems between sequent calculi and conjugated logic pairs.\\
    }
    \end{quote}
  \item
    \begin{quote}
    \textbf{Investigate the implications of these mappings on the
    relationship between the underlying logical systems.\\
    }
    \end{quote}
  \end{enumerate}
\item
  \begin{quote}
  \textbf{How the activities will be carried out:\\
  }
  \end{quote}

  \begin{enumerate}
  \def\labelenumi{\arabic{enumi}.}
  \item
    \begin{quote}
    \textbf{The analysis of graphical representations will involve
    employing graph comparison algorithms and techniques to identify
    common patterns and structural features.\\
    }
    \end{quote}
  \item
    \begin{quote}
    \textbf{The development of formal mappings will involve constructing
    bijections or homomorphisms between the graph-based representations,
    ensuring that logical relationships are maintained.\\
    }
    \end{quote}
  \item
    \begin{quote}
    \textbf{The translation of proofs and theorems will involve applying
    the established mappings to transform logical expressions and proof
    structures.\\
    }
    \end{quote}
  \item
    \begin{quote}
    \textbf{The investigation of implications will involve analyzing the
    consequences of the mappings on the metatheoretical properties of
    the logical systems.\\
    }
    \end{quote}
  \end{enumerate}
\end{itemize}

\begin{itemize}
\item
  \begin{quote}
  \textbf{\hfill\break
  }
  \end{quote}
\end{itemize}

\hypertarget{secondary-objective-harnessing-the-power-of-non-classical-logics}{%
\subsubsection{Secondary Objective: Harnessing the Power of
Non-Classical
Logics}\label{secondary-objective-harnessing-the-power-of-non-classical-logics}}

\hypertarget{goal-3-investigating-the-relationship-between-sequent-calculi-and-supercalculi}{%
\subsubsection{Goal 3: Investigating the Relationship between Sequent
Calculi and
Supercalculi}\label{goal-3-investigating-the-relationship-between-sequent-calculi-and-supercalculi}}

\textbf{Subgoal 3.1: Develop a formal framework for representing
supercalculi using graphs}

\begin{itemize}
\item
  \begin{quote}
  \textbf{Activities:\\
  }
  \end{quote}

  \begin{itemize}
  \item
    \begin{quote}
    Conduct a thorough review of existing formal frameworks for
    representing logical systems, including graph-based approaches and
    algebraic representations, with a particular focus on supercalculi.
    \end{quote}
  \item
    \begin{quote}
    Identify the key properties and requirements for a formal framework
    specifically tailored to supercalculi, considering their unique
    structural and semantic features.
    \end{quote}
  \item
    \begin{quote}
    Design a novel graph-based formal framework that effectively
    captures the intricate structure and semantics of supercalculi,
    including their ability to handle multiple modalities and
    non-standard connectives.
    \end{quote}
  \item
    \begin{quote}
    Evaluate the expressiveness and flexibility of the proposed
    framework through the representation of a variety of supercalculi,
    encompassing both well-established systems and newly developed ones.
    \end{quote}
  \end{itemize}
\item
  \begin{quote}
  \textbf{How the activities will be carried out:\\
  }
  \end{quote}

  \begin{itemize}
  \item
    \begin{quote}
    The review of existing frameworks will involve examining academic
    literature, attending conferences, and consulting with experts in
    the field of supercalculi and formal logic.
    \end{quote}
  \item
    \begin{quote}
    The identification of key properties and requirements will involve
    analyzing the characteristics of supercalculi, considering their
    ability to express multimodal and non-classical logics, and the
    needs of researchers working with these systems.
    \end{quote}
  \item
    \begin{quote}
    The design of the novel framework will involve utilizing graph
    theory concepts, formal logic principles, and software engineering
    techniques, ensuring the framework can effectively represent the
    complex relationships and interactions within supercalculi.
    \end{quote}
  \item
    \begin{quote}
    The evaluation of the framework will involve representing a range of
    supercalculi, including both well-established systems such as modal
    logics and hybrid logics, and newly developed supercalculi with
    unique features.
    \end{quote}
  \end{itemize}
\end{itemize}

\begin{itemize}
\tightlist
\item
\end{itemize}

\hypertarget{goal-4-investigating-the-relationship-between-sequent-calculi-and-substructural-logics}{%
\subsubsection{Goal 4: Investigating the Relationship between Sequent
Calculi and Substructural
Logics}\label{goal-4-investigating-the-relationship-between-sequent-calculi-and-substructural-logics}}

\textbf{Subgoal 4.1: Develop a formal framework for representing
substructural logics using graphs}

\begin{itemize}
\item
  \begin{quote}
  \textbf{Activities:\\
  }
  \end{quote}

  \begin{itemize}
  \item
    \begin{quote}
    Conduct a comprehensive review of existing formal frameworks for
    representing substructural logics, including graph-based approaches
    and algebraic representations.
    \end{quote}
  \item
    \begin{quote}
    Identify the key properties and requirements for a formal framework
    specifically tailored to substructural logics.
    \end{quote}
  \item
    \begin{quote}
    Design a novel graph-based formal framework that effectively
    captures the structure and semantics of substructural logics.
    \end{quote}
  \item
    \begin{quote}
    Evaluate the expressiveness and flexibility of the proposed
    framework through the representation of a variety of substructural
    logics, including linear logic, relevance logic, and intuitionistic
    logic.
    \end{quote}
  \end{itemize}
\item
\item
  \begin{quote}
  \textbf{How the activities will be carried out:\\
  }
  \end{quote}

  \begin{itemize}
  \item
    \begin{quote}
    The review of existing frameworks will involve examining academic
    literature, attending conferences, and consulting with experts in
    the field.
    \end{quote}
  \item
    \begin{quote}
    The identification of key properties and requirements will involve
    analyzing the characteristics of substructural logics and the needs
    of researchers working with these systems.
    \end{quote}
  \item
    \begin{quote}
    The design of the novel framework will involve utilizing graph
    theory concepts, formal logic principles, and software engineering
    techniques.
    \end{quote}
  \item
    \begin{quote}
    The evaluation of the framework will involve representing a range of
    substructural logics, including both well-established systems and
    newly developed ones.
    \end{quote}
  \end{itemize}
\item
\end{itemize}

\textbf{Subgoal 4.2: Establish a connection between the graph-based
representations of sequent calculi and substructural logics}

\begin{itemize}
\item
  \begin{quote}
  \textbf{Activities:\\
  }
  \end{quote}

  \begin{itemize}
  \item
    \begin{quote}
    Analyze the graphical representations of sequent calculi and
    substructural logics to identify structural similarities and
    correspondences.
    \end{quote}
  \item
    \begin{quote}
    Develop formal mappings between the graph-based representations of
    sequent calculi and substructural logics, preserving their
    structural and semantic properties.
    \end{quote}
  \item
    \begin{quote}
    Utilize the established mappings to translate proofs and theorems
    between sequent calculi and substructural logics.
    \end{quote}
  \item
    \begin{quote}
    Investigate the implications of these mappings on the relationship
    between the underlying logical systems.
    \end{quote}
  \end{itemize}
\item
\item
  \begin{quote}
  \textbf{How the activities will be carried out:\\
  }
  \end{quote}

  \begin{itemize}
  \item
    \begin{quote}
    The analysis of graphical representations will involve employing
    graph comparison algorithms and techniques to identify common
    patterns and structural features.
    \end{quote}
  \item
    \begin{quote}
    The development of formal mappings will involve constructing
    bijections or homomorphisms between the graph-based representations,
    ensuring that logical relationships are maintained.
    \end{quote}
  \item
    \begin{quote}
    The translation of proofs and theorems will involve applying the
    established mappings to transform logical expressions and proof
    structures.
    \end{quote}
  \item
    \begin{quote}
    The investigation of implications will involve analyzing the
    consequences of the mappings on the metatheoretical properties of
    the logical systems.
    \end{quote}
  \end{itemize}
\item
\end{itemize}

\hypertarget{foundational-objective-laying-a-rigorous-theoretical-framework}{%
\subsubsection{Foundational Objective: Laying a Rigorous Theoretical
Framework}\label{foundational-objective-laying-a-rigorous-theoretical-framework}}

\hypertarget{goal-5-establishing-a-formal-foundation-for-sequent-calculi}{%
\subsubsection{Goal 5: Establishing a Formal Foundation for Sequent
Calculi}\label{goal-5-establishing-a-formal-foundation-for-sequent-calculi}}

\textbf{5.1: Develop a formal framework for representing sequent calculi
using graphs}

\begin{itemize}
\item
  \begin{quote}
  \textbf{Activities:}
  \end{quote}

  \begin{itemize}
  \item
    \begin{quote}
    Conduct a thorough review of existing formal frameworks for
    representing logical systems, including graph-based approaches and
    algebraic representations.
    \end{quote}
  \item
    \begin{quote}
    Identify the key properties and requirements for a formal framework
    specifically tailored to sequent calculi.
    \end{quote}
  \item
    \begin{quote}
    Design a novel graph-based formal framework that effectively
    captures the structure and semantics of sequent calculi.
    \end{quote}
  \item
    \begin{quote}
    Evaluate the expressiveness and flexibility of the proposed
    framework through the representation of a variety of non-classical
    logics.
    \end{quote}
  \end{itemize}
\item
  \begin{quote}
  \textbf{How the activities will be carried out:}
  \end{quote}

  \begin{itemize}
  \item
    \begin{quote}
    The review of existing frameworks will involve examining academic
    literature, attending conferences, and consulting with experts in
    the field.
    \end{quote}
  \item
    \begin{quote}
    The identification of key properties and requirements will involve
    analyzing the characteristics of sequent calculi and the needs of
    researchers working with these systems.
    \end{quote}
  \item
    \begin{quote}
    The design of the novel framework will involve utilizing graph
    theory concepts, formal logic principles, and software engineering
    techniques.
    \end{quote}
  \item
    \begin{quote}
    The evaluation of the framework will involve representing a range of
    non-classical logics, including both well-established systems and
    newly developed ones.
    \end{quote}
  \end{itemize}
\end{itemize}

\textbf{5.2: Use this framework to investigate the structural properties
of sequent calculi}

\begin{itemize}
\item
  \begin{quote}
  \textbf{Activities:}
  \end{quote}

  \begin{itemize}
  \item
    \begin{quote}
    Apply the developed formal framework to analyze the structural
    properties of a range of sequent calculi, including their
    connectivity, symmetries, and decomposition patterns.
    \end{quote}
  \item
    \begin{quote}
    Identify and characterize common structural motifs and patterns that
    emerge across different sequent calculi.
    \end{quote}
  \item
    \begin{quote}
    Develop formal theorems that establish relationships between the
    structural properties of sequent calculi and their logical
    properties, such as soundness and completeness.
    \end{quote}
  \end{itemize}
\item
  \begin{quote}
  \textbf{How the activities will be carried out:}
  \end{quote}

  \begin{itemize}
  \item
    \begin{quote}
    The analysis of structural properties will involve applying graph
    algorithms and techniques to the graphical representation of sequent
    calculi.
    \end{quote}
  \item
    \begin{quote}
    The identification of common patterns will involve employing pattern
    recognition methods and statistical analysis.
    \end{quote}
  \item
    \begin{quote}
    The development of formal theorems will involve rigorous
    mathematical proofs based on the formal framework and logical
    principles.
    \end{quote}
  \end{itemize}
\end{itemize}

\textbf{5.3: Apply this framework to the development of new
non-classical languages}

\begin{itemize}
\item
  \begin{quote}
  \textbf{Activities:}
  \end{quote}

  \begin{itemize}
  \item
    \begin{quote}
    Utilize the formal framework to design and construct novel
    non-classical logics with tailored properties and applications.
    \end{quote}
  \item
    \begin{quote}
    Employ the framework to systematically explore the space of possible
    non-classical logics, identifying new and interesting logical
    systems.
    \end{quote}
  \item
    \begin{quote}
    Formally analyze and evaluate the properties of the newly developed
    non-classical logics, including their expressiveness, consistency,
    and decidability.
    \end{quote}
  \end{itemize}
\item
  \begin{quote}
  \textbf{How the activities will be carried out:}
  \end{quote}

  \begin{itemize}
  \item
    \begin{quote}
    The design of new logics will involve utilizing the
    framework\textquotesingle s constructions to represent the desired
    logical properties and relationships.
    \end{quote}
  \item
    \begin{quote}
    The exploration of the space of non-classical logics will involve
    employing systematic search algorithms and techniques guided by the
    framework\textquotesingle s constraints.
    \end{quote}
  \item
    \begin{quote}
    The formal analysis of newly developed logics will involve applying
    logical reasoning methods and tools to the
    framework\textquotesingle s representation of these logics.
    \end{quote}
  \end{itemize}
\end{itemize}

\textbf{Deliverables:}

\begin{itemize}
\item
  \begin{quote}
  A comprehensive compendium of non-classical logics meticulously
  detailing their axioms, rules, and distinctive properties.
  \end{quote}
\item
  \begin{quote}
  A novel formal framework for representing and analyzing non-classical
  languages empowering a deeper understanding of their intricate
  constructs.
  \end{quote}
\item
  \begin{quote}
  A refined classification of non-classical logics organizing these
  diverse systems based on their inherent relationships and shared
  characteristics.
  \end{quote}
\item
  \begin{quote}
  A comprehensive graph of sequent calculi visually depicting the
  intricate web of connections between diverse non-classical logics.
  \end{quote}
\item
  \begin{quote}
  A catalog of common patterns and thematic elements that characterize
  the properties of non-classical logics providing insights into their
  unifying principles.
  \end{quote}
\item
  \begin{quote}
  An exhaustive list of new areas of research suggested by the analysis
  of the graph of sequent calculi opening up new frontiers for
  exploration in the realm of non-classical logic.
  \end{quote}
\end{itemize}

§ Goals statements identify the overall purpose of the project and a
general indication of intent.

\begin{enumerate}
\def\labelenumi{\arabic{enumi}.}
\item
  \begin{quote}
  \textbf{Identification of non-classical logics:} A comprehensive list
  of non-classical logics will be compiled, including their axioms,
  rules, and properties.
  \end{quote}
\item
  \begin{quote}
  \textbf{Classification of non-classical logics:} The non-classical
  logics will be classified based on their properties, such as the types
  of negation that they use, the consistency conditions that they
  satisfy, and the types of truth values that they employ.
  \end{quote}
\item
  \begin{quote}
  \textbf{Construction of the graph of sequent calculi:} The graph of
  sequent calculi will be constructed by connecting the different
  non-classical logics based on their relationships.
  \end{quote}
\item
  \begin{quote}
  \textbf{Analysis of the graph of sequent calculi:} The graph of
  sequent calculi will be analyzed to identify common patterns and
  themes, and to suggest new areas of research.
  \end{quote}
\item
  \begin{quote}
  Develop the foundations and fundamentals for a universal theory of
  semantic languages.
  \end{quote}
\item
  \begin{quote}
  Identify the precise metalinguistic conditions that define
  paraconsistent languages
  \end{quote}
\item
  \begin{quote}
  Identify the precise metalinguistic conditions that define
  paracomplete languages.
  \end{quote}
\item
  \begin{quote}
  Identify the precise metalinguistic conditions that define
  constructive languages.
  \end{quote}
\item
  \begin{quote}
  Produce an automated reasoner that utilizes the union of
  paraconsistent constructions and paracomplete constructions in a
  metasystem of constructive proofs and constructive refutation
  refutations.
  \end{quote}
\item
  \begin{quote}
  Schematically relate the subcalculi, supercalculi, hypocalculi, and
  hypercalculi graphs to universal quantum computing.
  \end{quote}
\end{enumerate}

§ Objectives are action statements with measurable outcomes to be
completed by a specified time and under specified conditions.

\hypertarget{approachmethodology}{%
\subsection{\texorpdfstring{\textbf{Approach/Methodology}}{Approach/Methodology}}\label{approachmethodology}}

§ How are you going to carry out your project?

The graph will represent calculi as nodes and their relationships as
edges, with edge types encoding different kinds of connections, such as
equivalence, embedding, and extension.

\textbf{Literature Review:}

\begin{itemize}
\item
  \begin{quote}
  Conduct an extensive review of relevant literature on sequent calculi
  including intuitionistic logic, counter-intuitionistic logic, linear
  logic, the logic of qubits, Ardeshir-Vaezian's sequent calculus U,
  Sambin's Basic Logic, and T-Norm hypersequent calculi.
  \end{quote}
\item
  \begin{quote}
  Criticize the fundamental concepts, principles, and applications of
  these calculi.
  \end{quote}
\item
  \begin{quote}
  Identify and analyze existing research related to the graph of
  supercalculi, conjugated calculi pairs, and subcalculi.
  \end{quote}
\end{itemize}

\textbf{2. Conceptual Framework Development:}

\begin{itemize}
\item
  \begin{quote}
  Formulate a clear and precise conceptual framework for the graph of
  supercalculi, conjugated calculi pairs, and subcalculi.
  \end{quote}
\item
  \begin{quote}
  Define the key components and relationships within the graph
  structure.
  \end{quote}
\item
  \begin{quote}
  Establish a formal representation of the graph using appropriate
  mathematical and graphical notation.
  \end{quote}
\item
  \begin{quote}
  Establish a libre open source programming language capable of
  representing all supercalculi, conjugated calculi pairs, and
  subcalculi and their semantics.
  \end{quote}
\item
  \begin{quote}
  Establish an automated reasoning suite utilizing all sequent calculi
  in a modular way to reason about finite subgraphs of the graph of
  supercalculi, conjugated calculi pairs, and subcalculi.
  \end{quote}
\end{itemize}

\textbf{3. Metamathematical Analysis:}

\begin{itemize}
\item
  \begin{quote}
  Employ metamathematical techniques to investigate the properties and
  structure of the graph of supercalculi, conjugated calculi pairs, and
  subcalculi.
  \end{quote}
\item
  \begin{quote}
  Analyze the graph\textquotesingle s connectivity, paraconsistency, and
  paracompleteness.
  \end{quote}
\item
  \begin{quote}
  Explore the relationships between the graph\textquotesingle s
  structure and the underlying logical systems.
  \end{quote}
\end{itemize}

\textbf{4. Metalinguistic Investigation:}

\begin{itemize}
\item
  \begin{quote}
  Utilize metalinguistic tools to examine the expressive power and
  limitations of the graph of supercalculi, conjugated calculi pairs,
  and subcalculi.
  \end{quote}
\item
  \begin{quote}
  Analyze the graph\textquotesingle s ability to represent and formalize
  various logical concepts and relationships.
  \end{quote}
\item
  \begin{quote}
  Represent the symmetries and dualities of not only the object
  languages but the metalanguages of the various calculi.
  \end{quote}
\item
  \begin{quote}
  Develop metalanguages for all semantic languages.
  \end{quote}
\item
  \begin{quote}
  Evaluate the graph\textquotesingle s effectiveness in capturing the
  nuances of semantic languages.
  \end{quote}
\end{itemize}

\textbf{5. Comparative Analysis:}

\begin{itemize}
\item
  \begin{quote}
  Compare and contrast the graph of supercalculi, conjugated calculi
  pairs, and subcalculi with alternative approaches to representing
  semantic languages.
  \end{quote}
\item
  \begin{quote}
  Identify the strengths and weaknesses of each approach in terms of
  expressiveness, conciseness, and computational efficiency.
  \end{quote}
\item
  \begin{quote}
  Discuss the implications of the graph-based approach for understanding
  and reasoning within and without semantic language.
  \end{quote}
\end{itemize}

§ What specific activities do you propose to meet the goals and
objectives you have outlined, and how will those activities be carried
out?

\textbf{Specific Activities:}

To achieve the outlined goals and objectives, the following specific
activities will be undertaken:

\begin{itemize}
\item
  \begin{quote}
  Gather and organize relevant literature on semantic language and graph
  of supercalculi, conjugated calculi pairs, and subcalculi..
  \end{quote}
\item
  \begin{quote}
  Construct a conceptual diagram or model to represent the graph of
  supercalculi, conjugated calculi pairs, and subcalculi.
  \end{quote}
\item
  \begin{quote}
  Develop formal definitions and theorems related to the
  graph\textquotesingle s structure and properties.
  \end{quote}
\item
  \begin{quote}
  Utilize metamathematical tools, such as proof theory and model theory,
  to analyze the graph\textquotesingle s behavior.
  \end{quote}
\item
  \begin{quote}
  Employ metalinguistic techniques to assess the expressive power and
  limitations of the graph.
  \end{quote}
\item
  \begin{quote}
  Conduct comparative analysis with alternative approaches to
  representing non-classical languages.
  \end{quote}
\item
  \begin{quote}
  Programmatically represent the works in a logical programming paradigm
  programming language in a published code repository.
  \end{quote}
\end{itemize}

\hypertarget{outcomes-benefits-results}{%
\subsection{\texorpdfstring{\textbf{Outcomes, Benefits,
Results}}{Outcomes, Benefits, Results}}\label{outcomes-benefits-results}}

The project will deliver the following outcomes.

\begin{itemize}
\item
  \begin{quote}
  A comprehensive list of non-classical logics, including their axioms,
  rules, and properties.
  \end{quote}
\item
  \begin{quote}
  A new formal framework for representing and reasoning about semantic
  languages.
  \end{quote}
\item
  \begin{quote}
  A new classification of semantic languages.
  \end{quote}
\item
  \begin{quote}
  A graph of sequent calculi that visually represents the relationships
  between these logics.
  \end{quote}
\item
  \begin{quote}
  A list of common patterns and themes in the properties of
  non-classical logics.
  \end{quote}
\item
  \begin{quote}
  A list of new areas of research suggested by the analysis of the graph
  of supercalculi, conjugated calculi pairs, and subcalculi.
  \end{quote}
\end{itemize}

The project will also have the following benefits.

\begin{itemize}
\item
  \begin{quote}
  It will provide a better understanding of the logical relationships
  between different sequent calculi.
  \end{quote}
\item
  \begin{quote}
  It will facilitate the development of new and more powerful sequent
  calculi.
  \end{quote}
\item
  \begin{quote}
  It will enable the application of sequent calculi to new areas.
  \end{quote}
\end{itemize}

The project will also produce the following results.

\begin{itemize}
\item
  \begin{quote}
  A new formal framework for representing and reasoning about semantic
  languages.
  \end{quote}
\item
  \begin{quote}
  A new classification of sublanguages, superlanguages, and
  hyperlanguages as well as subcalculi, supercalculi, and hypercalculi.
  \end{quote}
\item
  \begin{quote}
  A graph of supercalculi, conjugated calculi pairs, and subcalculi that
  visually represents the relationships between these logics.
  \end{quote}
\item
  \begin{quote}
  A list of common patterns and themes in the properties of
  non-classical logics and non-classical languages.
  \end{quote}
\item
  \begin{quote}
  A list of new areas of research suggested by the analysis of the graph
  of supercalculi, conjugated calculi pairs, and subcalculi.
  \end{quote}
\end{itemize}

§ Outcomes - What are the products of your work?

§ Impact - What are the benefits and results of your work?

§ Measurement - Can your outcomes, benefits and results be measured?

§ Products - What does the funding agency get in return for supporting
your proposal?

\hypertarget{project-directorprincipal-investigator-and-staff}{%
\subsection{\texorpdfstring{\textbf{Project Director/Principal
Investigator and
Staff}}{Project Director/Principal Investigator and Staff}}\label{project-directorprincipal-investigator-and-staff}}

§ List the qualifications and experience of the proposed project
director/principal investigator.

§ List the qualifications and experience of key project staff.

\textbf{Primary Objective: Establishing a Unified Understanding of
Non-Classical Logics}

\begin{itemize}
\tightlist
\item
\end{itemize}

\textbf{Gaps that this work can fill:}

\begin{itemize}
\item
  \begin{quote}
  \textbf{There is a lack of efficient and scalable graph-based
  algorithms specifically designed for analyzing the structural
  properties of sequent calculi.}
  \end{quote}
\item
  \begin{quote}
  \textbf{The impact of structural properties on the computational
  behavior of sequent calculi remains incompletely understood.}
  \end{quote}
\item
  \begin{quote}
  \textbf{The applicability of graph-based techniques to the analysis of
  sequent calculus extensions and variations has not been fully
  explored.}
  \end{quote}
\end{itemize}

\textbf{Gaps that this work can fill:}

\begin{itemize}
\item
  \begin{quote}
  \textbf{The current understanding of the structural properties of
  sequent calculi is limited and lacks a comprehensive and systematic
  approach to analysis.}
  \end{quote}
\item
  \begin{quote}
  \textbf{There is a need for rigorous and expressive graph-based
  techniques to effectively capture the intricate structure and
  relationships within sequent calculi.}
  \end{quote}
\item
  \begin{quote}
  \textbf{The identification of common structural patterns and their
  relationship to logical properties remains incompletely understood.}
  \end{quote}
\item
  \begin{quote}
  \textbf{The development of formal theorems that connect structural
  features to computational and logical properties can provide a deeper
  understanding of the nature of sequent calculi.}
  \end{quote}
\end{itemize}

\textbf{Gaps that this work can fill:}

\begin{itemize}
\item
  \begin{quote}
  \textbf{The current understanding of the relationship between sequent
  calculi and conjugated logic pairs is limited and lacks a
  comprehensive formal framework for analysis.\\
  }
  \end{quote}
\item
  \begin{quote}
  \textbf{There is a need for a rigorous and expressive formal framework
  that can effectively capture the connections between the graphical
  representations of these systems.\\
  }
  \end{quote}
\item
  \begin{quote}
  \textbf{The precise relationship between the structural properties of
  sequent calculi and the logical features of conjugated logic pairs
  remains unclear.\\
  }
  \end{quote}
\item
  \begin{quote}
  \textbf{The implications of the relationship between sequent calculi
  and conjugated logic pairs on proof theory and automated reasoning
  have not been fully explored.}
  \end{quote}
\end{itemize}

\textbf{Gaps that this work can fill:}

\begin{itemize}
\item
  \begin{quote}
  \textbf{The current understanding of formal frameworks for
  representing supercalculi is limited and lacks a unified and
  expressive approach.}
  \end{quote}
\item
  \begin{quote}
  \textbf{There is a need for a rigorous and flexible formal framework
  that can effectively capture the intricacies of supercalculi,
  including their multimodal and non-classical nature.}
  \end{quote}
\item
  \begin{quote}
  \textbf{The structural and semantic relationships between supercalculi
  and their underlying logical systems remain incompletely understood.}
  \end{quote}
\item
  \begin{quote}
  \textbf{The development of a graph-based formal framework can
  facilitate the analysis and comparison of different supercalculi and
  their applications.}
  \end{quote}
\end{itemize}

\textbf{Gaps that this work can fill:}

\begin{itemize}
\item
  \begin{quote}
  The current understanding of the relationship between sequent calculi
  and substructural logics is limited and lacks a comprehensive formal
  framework for analysis.
  \end{quote}
\item
  \begin{quote}
  There is a need for a rigorous and expressive formal framework that
  can effectively capture the connections between the graphical
  representations of these systems.
  \end{quote}
\item
  \begin{quote}
  The precise relationship between the structural properties of sequent
  calculi and the logical features of substructural logics remains
  unclear.
  \end{quote}
\item
  \begin{quote}
  The implications of the relationship between sequent calculi and
  substructural logics on proof theory and automated reasoning have not
  been fully explored.
  \end{quote}
\end{itemize}

\textbf{Gaps that this work can fill:}

\begin{itemize}
\item
  \begin{quote}
  The current understanding of formal frameworks for representing
  sequent calculi is fragmented and lacks a unified approach.
  \end{quote}
\item
  \begin{quote}
  There is a need for a rigorous and expressive formal framework that
  can effectively capture the intricacies of sequent calculi.
  \end{quote}
\item
  \begin{quote}
  The structural properties of sequent calculi remain largely
  unexplored, and their relationships to logical properties are not
  fully understood.
  \end{quote}
\item
  \begin{quote}
  The development of new non-classical logics is often ad hoc and lacks
  a systematic approach based on formal frameworks.
  \end{quote}
\end{itemize}

\textbf{Context of the project:}

\begin{itemize}
\item
  \begin{quote}
  \textbf{Sequent calculi play a crucial role in formal logic, providing
  a powerful tool for representing and reasoning about logical systems.}
  \end{quote}
\item
  \begin{quote}
  \textbf{A comprehensive understanding of the structural properties of
  sequent calculi can provide insights into their computational
  efficiency, expressiveness, and limitations.}
  \end{quote}
\item
  \begin{quote}
  \textbf{The development of efficient and scalable graph-based
  algorithms can facilitate the systematic analysis of sequent calculi,
  leading to the discovery of new properties and optimization
  strategies.}
  \end{quote}
\item
  \begin{quote}
  \textbf{The exploration of the impact of structural properties and the
  application of graph-based techniques to extensions and variations can
  contribute to advancements in proof theory, automated reasoning, and
  the design of new logical systems.}
  \end{quote}
\end{itemize}

\textbf{Context of the project:}

\begin{itemize}
\item
  \begin{quote}
  \textbf{Sequent calculi have emerged as powerful tools for formalizing
  and reasoning about a wide range of logical systems, with applications
  in proof theory, automated reasoning, and artificial intelligence.}
  \end{quote}
\item
  \begin{quote}
  \textbf{Establishing a comprehensive understanding of the structural
  properties of sequent calculi can provide insights into their
  computational behavior, expressiveness, and limitations.}
  \end{quote}
\item
  \begin{quote}
  \textbf{The development of graph-based techniques for analyzing
  sequent calculi can facilitate the identification of common patterns,
  the discovery of new properties, and the development of optimization
  strategies for automated reasoning systems.}
  \end{quote}
\item
  \begin{quote}
  \textbf{The establishment of formal connections between structural
  features and logical properties can contribute to advancements in
  proof theory, automated reasoning, and the design of new logical
  systems.}
  \end{quote}
\end{itemize}

\textbf{Context of the project:}

\begin{itemize}
\item
  \begin{quote}
  \textbf{Sequent calculi and conjugated logic pairs are powerful tools
  for formalizing and reasoning about logical systems, with applications
  in diverse areas such as artificial intelligence, linguistics, and
  computer science.\\
  }
  \end{quote}
\item
  \begin{quote}
  \textbf{Establishing a clear connection between sequent calculi and
  conjugated logic pairs can provide a unified framework for
  understanding and analyzing these systems.\\
  }
  \end{quote}
\item
  \begin{quote}
  \textbf{The development of formal mappings between graphical
  representations can facilitate the translation of proofs and theorems,
  enhancing interoperability and cross-fertilization between the two
  domains.\\
  }
  \end{quote}
\item
  \begin{quote}
  \textbf{The investigation of the implications of these connections can
  lead to new insights into the nature of conjugated logic pairs and
  their applications in formal logic, artificial intelligence, and other
  areas.}
  \end{quote}
\end{itemize}

\textbf{Context of the project:}

\begin{itemize}
\item
  \begin{quote}
  \textbf{Supercalculi have emerged as powerful tools for formalizing
  and reasoning about multimodal and non-classical logics, with
  applications in diverse areas such as artificial intelligence,
  linguistics, and computer science.}
  \end{quote}
\item
  \begin{quote}
  \textbf{Establishing a comprehensive formal framework for representing
  supercalculi using graphs can provide a unified foundation for
  understanding, analyzing, and comparing these systems.}
  \end{quote}
\item
  \begin{quote}
  \textbf{The development of formal mappings between graphical
  representations of supercalculi can facilitate the translation of
  proofs and theorems, enhancing interoperability and
  cross-fertilization between different supercalculi.}
  \end{quote}
\item
  \begin{quote}
  \textbf{The investigation of the implications of these connections can
  lead to new insights into the nature of supercalculi and their
  applications in formal logic, artificial intelligence, and other
  areas.}
  \end{quote}
\end{itemize}

\textbf{Context of the project:}

\begin{itemize}
\item
  \begin{quote}
  Sequent calculi and substructural logics are powerful tools for
  formalizing and reasoning about logical systems with non-classical
  features.
  \end{quote}
\item
  \begin{quote}
  Establishing a clear connection between sequent calculi and
  substructural logics can provide a unified framework for understanding
  and analyzing these systems.
  \end{quote}
\item
  \begin{quote}
  The development of formal mappings between graphical representations
  can facilitate the translation of proofs and theorems, enhancing
  interoperability and cross-fertilization between the two domains.
  \end{quote}
\item
  \begin{quote}
  The investigation of the implications of these connections can lead to
  new insights into the nature of substructural logics and their
  applications in proof theory, automated reasoning, and other areas.
  \end{quote}
\end{itemize}

\textbf{Context of the project:}

\begin{itemize}
\item
  \begin{quote}
  The research on sequent calculi has gained significant traction in
  recent years, driven by their applications in automated reasoning,
  proof theory, and artificial intelligence.
  \end{quote}
\item
  \begin{quote}
  The development of a formal framework for representing sequent calculi
  using graphs can provide a solid foundation for further advancements
  in this area.
  \end{quote}
\item
  \begin{quote}
  The investigation of the structural properties of sequent calculi can
  lead to a deeper understanding of their logical behavior and potential
  applications.
  \end{quote}
\item
  \begin{quote}
  The design of new non-classical logics using the formal framework can
  expand the repertoire of logical systems available for modeling and
  reasoning about complex phenomena.
  \end{quote}
\end{itemize}

\end{document}}*}
