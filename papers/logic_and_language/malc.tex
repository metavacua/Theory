
\documentclass{article}

\usepackage{amsmath}
\usepackage{ebproof}
\usepackage{fullpage}
\usepackage[utf8]{inputenc}
\usepackage{newunicodechar}
\usepackage{stix}

\newunicodechar{Γ}{\Gamma}
\newunicodechar{Δ}{\Delta}

\newunicodechar{Θ}{\Theta}
\newunicodechar{Λ}{\Lambda}

\newunicodechar{Ξ}{\Xi}
\newunicodechar{Π}{\Pi}

\newunicodechar{Φ}{\Phi}
\newunicodechar{Ψ}{\Psi}

\newunicodechar{Ω}{\Omega}

\newunicodechar{⊢}{\vdash}

\newunicodechar{⊕}{\oplus}
\newunicodechar{¬}{\neg}

\newunicodechar{⊗}{\otimes}
\newunicodechar{→}{\rightarrow}
\newunicodechar{←}{\leftarrow}

\newunicodechar{⊥}{\bot}
\newunicodechar{⊤}{\top}

\setlength{\parindent}{0em}

\author{James Martin, Ian D.L.N. Mclean}
\title{Multiplicative-Additive Lambek Sequent Calculus}

\begin{document}

\maketitle

\begin{abstract}

\end{abstract}

\section{Operational Rules}
\begin{center}

	\[
	\begin{prooftree}
	\hypo{Γ_{L}, Γ_{R} ⊢ C}
	\infer1{Γ_{L}, 1, Γ_{R} ⊢ C}
	\end{prooftree}
	\quad
	\begin{prooftree}
	\infer0{ ⊢ 1}
	\end{prooftree}
	\]

	\[
	\begin{prooftree}
	\infer0{Γ_{L}, 0, Γ_{R} ⊢ C}
	\end{prooftree}
	\quad
	\begin{prooftree}
	\infer0{ Γ ⊢ ⊤}
	\end{prooftree}
	\]

	\[
	\begin{prooftree}
	\hypo{Γ_{L}, A, B, Γ_{R} ⊢ C}
	\infer1{Γ_{L}, A ⊗ B, Γ_{R} ⊢ C}
	\end{prooftree}
	\quad
	\begin{prooftree}
	\hypo{Γ_{L} ⊢ A}
	\hypo{Γ_{R} ⊢ B}
	\infer2{Γ_{L}, Γ_{R} ⊢ A ⊗ B}
	\end{prooftree}
	\]

	\[
	\begin{prooftree}
	\hypo{Γ ⊢ A}
	\hypo{Γ_{L}, B, Γ_{R} ⊢ C}
	\infer2{Γ_{L}, Γ, A → B, Γ_{R} ⊢ C}
	\end{prooftree}
	\quad
	\begin{prooftree}
	\hypo{A, Γ ⊢ B}
	\infer1{Γ ⊢ A → B}
	\end{prooftree}
	\]

	\[
	\begin{prooftree}
	\hypo{Γ ⊢ A}
	\hypo{Γ_{L}, B, Γ_{R} ⊢ C}
	\infer2{Γ_{L}, B ← A, Γ, Γ_{R} ⊢ C}
	\end{prooftree}
	\quad
	\begin{prooftree}
	\hypo{Γ, A ⊢ B}
	\infer1{Γ ⊢ B ← A}
	\end{prooftree}
	\]

	\[
	\begin{prooftree}
	\hypo{Γ_{L}, A, Γ_{R} ⊢ C}
	\hypo{Γ_{L}, B, Γ_{R} ⊢ C}
	\infer2{Γ_{L}, A ⊕ B, Γ_{R} ⊢ C}
	\end{prooftree}
	\quad
	\begin{prooftree}
	\hypo{Γ ⊢ A}
	\infer1{Γ ⊢ A ⊕ B}
	\end{prooftree}
	\quad
	\begin{prooftree}
	\hypo{Γ ⊢ B}
	\infer1{Γ ⊢ A ⊕ B}
	\end{prooftree}
	\]

	\[
	\begin{prooftree}
	\hypo{Γ_{L}, A, Γ_{R} ⊢ C}
	\infer1{Γ_{L}, A \& B, Γ_{R} ⊢ C}
	\end{prooftree}
	\quad
	\begin{prooftree}
	\hypo{Γ_{L}, B, Γ_{R} ⊢ C}
	\infer1{Γ_{L}, A \& B, Γ_{R} ⊢ C}
	\end{prooftree}
	\quad
	\begin{prooftree}
	\hypo{Γ ⊢ A}
	\hypo{Γ ⊢ B}
	\infer2{Γ ⊢ A \& B}
	\end{prooftree}
	\]

	\[
	\begin{prooftree}
	\hypo{ Γ ⊢ A}
	\infer1{ ¬_{C} A, Γ ⊢ }
	\end{prooftree}
	\quad
	\begin{prooftree}
	\hypo{ Γ ⊢ A}
	\infer1{ Γ, ¬_{C} A⊢ }
	\end{prooftree}
	\quad
	\begin{prooftree}
	\hypo{ A, Γ ⊢ }
	\infer1{ Γ ⊢ ¬_{L} A}
	\end{prooftree}
	\quad
	\begin{prooftree}
	\hypo{ Γ, A ⊢ }
	\infer1{ Γ ⊢ ¬_{R} A}
	\end{prooftree}
	\]

\end{center}

\section{Structural Rules}
Uses a variation of cut left.

\begin{center}
	\[
	\begin{prooftree}
	\infer0[Id]{A ⊢ A}
	\end{prooftree}
	\]

	\[
	\begin{prooftree}
	\hypo{Γ ⊢ A}
	\hypo{Δ, A, Π ⊢ C}
	\infer2[CutL]{Δ, Γ, Π ⊢ C}
	\end{prooftree}
	\]
\end{center}

\end{document}
