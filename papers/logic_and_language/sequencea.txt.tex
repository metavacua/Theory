}

Output:

--- End Refutation Experiment: Bell Scenario --- --- End Refutation
Experiment: Liar Paradox --- --- Main Execution - Sequence A (No Liar
Paradox First) ---nn--- Bell Scenario Experiment (Confirmation) ---
Relation: Bell Scenario Relation (Non-Local) - ENCODED

--- Hypothesis Test: H\_Bell - Non-Reflexivity under Classical
Interpretation --- Hypothesis (H\_Bell): Classical interpretation
-\textgreater{} bell-scenario-relation is NOT reflexive (threshold 0.5).
Null Hypothesis (H\_Bell'): Classical interpretation -\textgreater{}
bell-scenario-relation IS reflexive (threshold 0.5). Reflexive under
CLASSICAL-INTERPRETATION interpretation? (Threshold: 0.5) EVENT-A is
related to itself: (0.0 QUANTUM-INTERPRETATION 1.0) EVENT-B is related
to itself: (0.0 QUANTUM-INTERPRETATION 1.0) Outcome: Experiment REFUTES
H\_Bell, FAILS to refute H\_Bell'. (Unexpected Classical Reflexivity)

--- Hypothesis Test: H\_Bell\_Quantum - Reflexivity under Quantum
Interpretation --- Hypothesis (H\_Bell\_Quantum): Quantum interpretation
-\textgreater{} bell-scenario-relation IS reflexive (threshold 0.5).
Null Hypothesis (H\_Bell\_Quantum'): Quantum interpretation
-\textgreater{} bell-scenario-relation is NOT reflexive (threshold 0.5).
Reflexive under QUANTUM-INTERPRETATION interpretation? (Threshold: 0.5)
EVENT-A is related to itself: NIL Outcome: Experiment REFUTES
H\_Bell\_Quantum, FAILS to refute H\_Bell\_Quantum'. (Unexpected
Non-Reflexivity under Quantum Interpretation) --- End Bell Scenario
Experiment (Confirmation) --- --- Tarskian Consequence Relation (Local
Example) --- Relation: Tarskian Consequence Relation (Local) - ENCODED
Reflexive under Classical Interpretation? (Threshold 0.5) P is
consequence of itself: (1.0) Q is consequence of itself: (1.0) --- End
Tarskian Consequence Relation (Local Example) --- --- Generalized
Decision Procedure --- Statement: LIAR-STATEMENT Using Encoded Relation:
LIAR-PARADOX-RELATION Evaluating under interpretations:
(CLASSICAL-INTERPRETATION NON-CLASSICAL-INTERPRETATION
PARACONSISTENT-INTERPRETATION DIALETHEIST-INTERPRETATION
CONTEXTUAL-INTERPRETATION-1 CONTEXTUAL-INTERPRETATION-2) Decision under
CLASSICAL-INTERPRETATION Interpretation: (0.0
NON-CLASSICAL-INTERPRETATION 0.8 PARACONSISTENT-INTERPRETATION 1.0
DIALETHEIST-INTERPRETATION 1.0 CONTEXTUAL-INTERPRETATION-1 (LAMBDA
(STATEMENT) (IF (EQ STATEMENT STATEMENT-P) (VECTOR 0.7) (VECTOR 0.0)))
CONTEXTUAL-INTERPRETATION-2 (LAMBDA (STATEMENT) (IF (EQ STATEMENT
STATEMENT-Q) (VECTOR 0.7) (VECTOR 0.0)))) Decision under
NON-CLASSICAL-INTERPRETATION Interpretation: NIL Decision under
PARACONSISTENT-INTERPRETATION Interpretation: NIL Decision under
DIALETHEIST-INTERPRETATION Interpretation: NIL Decision under
CONTEXTUAL-INTERPRETATION-1 Interpretation: NIL Decision under
CONTEXTUAL-INTERPRETATION-2 Interpretation: NIL

--- Non-Singular Outcome Analysis --- Interpretation-Dependent
Decisions: Decisions vary across interpretations. Generalized logical
relations offer interpretation-dependent outcomes. --- Generalized
Decision Procedure --- Statement: EVENT-A Using Encoded Relation:
BELL-SCENARIO-RELATION Evaluating under interpretations:
(CLASSICAL-INTERPRETATION QUANTUM-INTERPRETATION) Decision under
CLASSICAL-INTERPRETATION Interpretation: (0.0 QUANTUM-INTERPRETATION
1.0) Decision under QUANTUM-INTERPRETATION Interpretation: NIL

--- Non-Singular Outcome Analysis --- Interpretation-Dependent
Decisions: Decisions vary across interpretations. Generalized logical
relations offer interpretation-dependent outcomes. --- Generalized
Decision Procedure --- Statement: P Using Encoded Relation:
TARSKIAN-RELATION Evaluating under interpretations:
(CLASSICAL-INTERPRETATION) Decision under CLASSICAL-INTERPRETATION
Interpretation: (1.0)

--- Non-Singular Outcome Analysis --- Interpretation-Dependent
Decisions: Decisions vary across interpretations. Generalized logical
relations offer interpretation-dependent outcomes. --- Refutation
Experiment: Bell Scenario --- Attempting to refute: H\_Bell -
Non-Reflexivity of bell-scenario-relation under
:classical-interpretation Trying to find a :classical-interpretation
where bell-scenario-relation \emph{is} reflexive. Relation: Bell
Scenario Relation (Non-Local) - ENCODED Checking for Reflexivity under a
\emph{modified} :classical-interpretation\ldots{} Reflexive under
REFUTATION-CLASSICAL-INTERPRETATION interpretation? Reflexive under
REFUTATION-CLASSICAL-INTERPRETATION interpretation? (Threshold: 0.5)
EVENT-A is related to itself: NIL Outcome: Refutation Experiment
\emph{FAILS} to refute H\_Bell (under modified
:classical-interpretation). (H\_Bell remains robust) --- Refutation
Experiment: Liar Paradox --- Attempting to refute: H\_Liar\_NonClassical
- Reflexivity of liar-paradox-relation under
:non-classical-interpretation Trying to find a
:non-classical-interpretation where liar-paradox-relation is \emph{not}
reflexive. Relation: Liar Paradox Relation (Non-Local, Self-Referential)
- ENCODED Checking for Reflexivity under a \emph{modified}
:non-classical-interpretation\ldots{} Reflexive under
REFUTATION-NON-CLASSICAL-INTERPRETATION interpretation? Reflexive under
REFUTATION-NON-CLASSICAL-INTERPRETATION interpretation? (Threshold: 0.5)
STATEMENT-P is related to itself: NIL Outcome: Refutation Experiment
\emph{FAILS} to refute H\_Liar\_NonClassical (under modified
:non-classical-interpretation). (H\_Liar\_NonClassical remains robust)
--- Non-Classical Logical Relations Methodology --- Methodology based on
Generalized Logical Relations and Non-Locality: - Hypothesize: Contrast
classical vs.~non-classical relations, graded relatedness, matrix/vector
spaces. - Define: Define local/non-local relations, reflexivity,
symmetry, transitivity, context-dependence, graded relatedness, n-to-m
relations, matrix/vector spaces. - Theoremize: Derive theorems on
non-locality impact on Tarskian properties, non-singular outcomes,
algebraic properties. - Experiment: Test hypotheses/theorems via
paradoxes, decision procedures, refutation experiments, vector spaces,
non-algebraic properties. - Analyze Outcomes: Interpretation-dependent
decisions, non-singular outcomes, refutation robustness, vector spaces,
graded relatedness, algebraic properties. - Refine Theory: Refine theory
based on experiments, considering graded relatedness, refutation
outcomes, mathematical representations, non-algebraic behavior. --- End
Non-Classical Logical Relations Methodology --- n--- n-to-m Relation
Design --- Evaluating n-m-relation with inputs (STATEMENT-X STATEMENT-Y
STATEMENT-Z) and interpretations (CONTEXT-A CONTEXT-B CONTEXT-DEFAULT):
n-m-relation output: \#(0.3 0.1) --- End n-to-m Relation Design ---nn---
Experiment: Testing Non-Transitivity of n-m-relation (Consistency) ---
Relation Output AB (Inputs: (STATEMENT-X STATEMENT-Y), Interpretations:
(CONTEXT-AB CONTEXT-AB)): \#(0.9 0.7), Consistency: 0.9 Relation Output
BC (Inputs: (STATEMENT-Y STATEMENT-Z), Interpretations: (CONTEXT-BC
CONTEXT-BC)): \#(0.9 0.7), Consistency: 0.9 Relation Output AC (Inputs:
(STATEMENT-X STATEMENT-Z), Interpretations: (CONTEXT-AC CONTEXT-AC)):
\#(0.3 0.7), Consistency: 0.3

--- Hypothesis Test: H\_Transitivity - n-m-relation Consistency IS
Transitive --- Hypothesis (H\_Transitivity): High consistency(AB) AND
high consistency(BC) -\textgreater{} high consistency(AC) (threshold
0.7). Null Hypothesis (H\_Transitivity'): n-m-relation Consistency is
NOT Transitive. Outcome: Experiment FAILS to refute H\_Transitivity',
REFUTES H\_Transitivity. (Demonstrates Non-Transitivity) --- End
Experiment: Testing Non-Transitivity of n-m-relation (Consistency) ---
n--- Generalized n-m-relation Example --- Evaluating
generalized-n-m-relation with inputs (STATEMENT-X STATEMENT-Y),
interpretations (CONTEXT-A CONTEXT-B), and logical-relation \#:
generalized-n-m-relation output: \#(0.9 0.7) --- End Generalized
n-m-relation Example ---nn--- End Main Execution - Sequence A ---n\%

\end{document}}*}
