}

\hypertarget{preliminaries}{%
\section{Preliminaries}\label{preliminaries}}

A set of logical connectives is said to be functionally complete when
combinations of the logical connectives in the set can express every
equivalent expression of the connectives in the set of \{⊥, ⊤, ¬, →, ↛,
⊕, ↔, ∧, ∨, ↓, ↑\}.

\{¬, →\} \{⊥, ⊤, ↛, ⊕, ↔, ∧, ∨\}; \{↓, ↑\}

`\{⊥, ⊤, ⊕, ↔\}, \{⊥, ↛, ⊕, ∧, ∨\}, \{⊤, ↔, ∧, ∨\}, \{⊥, ⊤, ∧, ∨\},
\{¬\}`

\{¬, →, ↛\}\{⊥, ⊤, ⊕, ↔, ∧, ∨\} \{↓, ↑\}

A set of logical connectives is said to be functionally incomplete when
combinations of the logical connectives can not express any functionally
complete subset of \{⊥, ⊤, ¬, →, ↛, ⊕, ↔, ∧, ∨, ↓, ↑\} or equivalently
can not express the set \{⊥, ⊤, ¬, →, ↛, ⊕, ↔, ∧, ∨, ↓, ↑\}.

The minimal sets of functionally complete logical connectives entail
that for defining the functionally incomplete sets the only relevant
logical connectives are in the set \{⊥, ⊤, ¬, →, ↛, ⊕, ↔, ∧, ∨\}.
Notably, any set that contains NOR or NAND is automatically functionally
complete.

\hypertarget{connective-properties}{%
\subsection{Connective Properties}\label{connective-properties}}

Non adjunctive paraconsistent logics These are logics in which the
conjunction

fails to obey the following law of adjunction: a, b entails a ∧ b

Non implicative paraconsistent logics These are logics in which the
implication

fails to obey the following law of implicativity: if entails a → b then
a entails b.

\hypertarget{maximal-functionally-incomplete-sets}{%
\subsection{Maximal Functionally Incomplete
Sets}\label{maximal-functionally-incomplete-sets}}

Affine: \{⊥, ⊤, ¬, ⊕, ↔\}; \{→\}, \{↛\}, \{∧\}, \{∨\}

False-preserving: \{⊥, ↛, ⊕, ∧, ∨\}; \{⊤\}, \{¬\}, \{→\}, \{↔\}

Truth-preserving: \{⊤, →, ↔, ∧, ∨\}; \{⊥\}, \{¬\}, \{↛\}, \{⊕\}

Monotonic: \{⊥, ⊤, ∧, ∨\}; \{⊕\}, \{↔\}, \{¬\}, \{→\}, \{↛\}

Self-dual: \{¬\} \{⊥, ⊤, ⊕, ↔\}; \{→\}, \{↛\}, \{∧\}, \{∨\}

\hypertarget{minimal-functionally-complete-sets}{%
\subsection{Minimal Functionally Complete
Sets}\label{minimal-functionally-complete-sets}}

\hypertarget{singles}{%
\subsubsection{Singles}\label{singles}}

\{↓\}

\{↑\}

\hypertarget{doubles}{%
\subsubsection{Doubles}\label{doubles}}

\{⊥, →\}

\{⊥, ←\}

\{⊤, ↛\}

\{⊤, ↚\}

\{¬, →\}

\{¬, ←\}

\{¬, ↛\}

\{¬, ↚\}

\{¬, ∨\}

\{¬, ∧\}

\{→, ↛\}

\{→, ↚\}

\{←, ↛\}

\{←, ↚\}

\{→, ⊕\}

\{←, ⊕\}

\{↛, ↔\}

\{↚, ↔\}

\hypertarget{triples}{%
\subsubsection{Triples}\label{triples}}

\{⊥, ↔, ∨\}

\{↔, ⊕, ∨\}

\{⊤, ⊕, ∨\}

\{⊥, ↔, ∧\}

\{↔, ⊕, ∧\}

\{⊤, ⊕, ∧\}

\hypertarget{functionally-incomplete-sets-of-logical-connectives}{%
\section{Functionally Incomplete Sets of Logical
Connectives}\label{functionally-incomplete-sets-of-logical-connectives}}

The following format of collections of sets are\\
The set; the functional completions of the set; the functional
incompletions of the set.

\hypertarget{singles-1}{%
\subsection{Singles}\label{singles-1}}

9 functionally incomplete singles.

\{⊥\}; \{→\}, \{↔, ∧\}, \{↔, ∨\}; ... \{⊤, ¬, ↔, ⊕\}, \{↛, ⊕, ∨, ∧\}

\{⊤\}; \{↛\}, \{⊕, ∧\}, \{⊕, ∨\}; ... \{⊥, ¬, ↔, ⊕\}, \{→, ↔, ∧, ∨\}

\{¬\}; \{→\}, \{↛\}, \{∧\}, \{∨\}; ... \{⊥, ⊤, ⊕, ↔\}

\{→\}; \{⊥\}, \{¬\}, \{↛\}, \{⊕\}; ... \{⊤,↔, ∧, ∨\}

\{↛\}; \{⊤\}, \{¬\}, \{→\}, \{↔\}; ... \{⊥, ⊕, ∧, ∨\}

\{⊕\}; \{→\}, \{⊤, ∧\}, \{⊤, ∨\}, \{↔, ∧\}, \{↔, ∨\}; ... \{⊥, ↛, ∨,
∧\}, \{⊥, ⊤, ¬, ↔\}

\{↔\}; \{↛\}, \{⊥, ∧\}, \{⊥, ∨\}, \{⊕, ∧\}, \{⊕, ∨\}; ... \{⊤, →, ∧,
∨\}, \{⊥, ⊤, ¬, ⊕\}

\{∧\}; \{¬\}, \{⊥, ↔\}, \{⊤, ⊕\}, \{⊕, ↔\}; \{⊥\}, \{⊤\}, \{↔\}, \{⊕\},
\{∨\}, \{→\}, \{↛\}, \{∨, →\}, \{∨, ↛\}, ..., \{⊥, ⊤, ∨\}, \{⊤, →, ↔,
∨\}, \{⊥, ↛, ⊕, ∨\}

\{∨\}; \{¬\}, \{⊥, ↔\}, \{⊤, ⊕\}, \{⊕, ↔\}; \ldots, \{⊥, ⊤, ∧\}, \{⊤, →,
∧, ↔\}, \{⊥, ↛, ⊕, ∧\}

\hypertarget{doubles-1}{%
\subsection{Doubles}\label{doubles-1}}

27 functionally incomplete doubles.

\hypertarget{affine}{%
\subsubsection{Affine}\label{affine}}

\{¬, ⊕\}; \{→\}, \{↛\}, \{∧\}, \{∨\}; \{⊥\}, \{⊤\}, \{↔\}, \{⊥, ⊤\},
\{⊥, ↔\}, \{⊤, ↔\}, \{⊥, ⊤, ↔\}

\{¬, ↔\}; \{→\}, \{↛\}, \{∧\}, \{∨\}; \{⊥\}, \{⊤\}, \{⊕\}, \{⊥, ⊤\},
\{⊥, ⊕\}, \{⊤, ⊕\}, \{⊥, ⊤, ⊕\}

\{⊥, ¬\}; \{→\}, \{↛\}, \{∧\}, \{∨\}; \{⊤\}, \{⊕\}, \{↔\}, \{⊤, ⊕\},
\{⊤, ↔\}, \{⊕, ↔\}, \{⊤, ⊕, ↔\}

\{⊥, ↔\}; \{→\}, \{↛\}, \{∧\}, \{∨\}; \{⊤\}, \{¬\}, \{⊕\}, \{⊤, ¬\},
\{⊤, ⊕\}, \{¬, ⊕\}, \{⊤, ¬, ⊕\}

\{⊤, ¬\}; \{→\}, \{↛\}, \{∧\}, \{∨\}; \{⊥\}, \{⊕\}, \{↔\}, \{⊥, ⊕\},
\{⊥, ↔\}, \{⊕, ↔\}, \{⊥, ⊕, ↔\}

\{⊤, ⊕\}; \{→\}, \{↛\}, \{∧\}, \{∨\}; \{⊥\}, \{¬\}, \{↔\}, \{⊥, ¬\},
\{⊥, ↔\}, \{¬, ↔\}, \{⊥, ¬, ↔\}

\{↔, ⊕\}; \{→\}, \{↛\}, \{∧\}, \{∨\}; \{⊥\}, \{⊤\}, \{¬\}, \{⊥, ⊤\},
\{⊥, ¬\}, \{⊤, ¬\}, \{⊥, ⊤, ¬\}

\hypertarget{affine-or-false-preserving}{%
\subsubsection{Affine or False
Preserving}\label{affine-or-false-preserving}}

\{⊥, ⊕\}; \{→\}, \{⊤, ∧\}, \{⊤, ∨\}, \{↔, ∧\}, \{↔, ∨\}; \ldots{} \{⊤,
¬, ↔\}, \{↛, ∨, ∧\}

\hypertarget{affine-or-truth-preserving}{%
\subsubsection{Affine or Truth
Preserving}\label{affine-or-truth-preserving}}

\{⊤, ↔\}; \{↛\}, \{⊥, ∧\}, \{⊥, ∨\}, \{⊕, ∧\}, \{⊕, ∨\}; \ldots{} \{⊥,
¬, ⊕\}, \{→, ∧, ∨\}

\hypertarget{affine-and-monotonic}{%
\subsubsection{Affine and Monotonic}\label{affine-and-monotonic}}

\{⊥, ⊤\}; \{→\}, \{↛\}, \{↔, ∧\}, \{↔, ∨\}, \{⊕, ∧\}, \{⊕, ∨\}; \{¬\},
\{↔\}, \{⊕\}, \{∧\}, \{∨\}, \{∧, ∨\}, \{¬, ⊕\}, \{¬, ↔\}, \{⊕, ↔\}, \{¬,
⊕, ↔\}

\hypertarget{truth-preserving}{%
\subsubsection{Truth Preserving}\label{truth-preserving}}

\hypertarget{generators}{%
\paragraph{Generators}\label{generators}}

\{→, ∧\}; \{⊥\}, \{¬\}, \{↛\}, \{⊕\}; \ldots{} \{⊤, ↔, ∨\}

\hypertarget{non-generators}{%
\paragraph{Non-generators}\label{non-generators}}

\{⊤, →\}; \{⊥\}, \{¬\}, \{↛\}, \{⊕\}; \ldots{} \{↔, ∧, ∨\}

\{→, ↔\}; \{⊥\}, \{¬\}, \{↛\}, \{⊕\}; \ldots{} \{⊤, ∧, ∨\}

\{→, ∨\}; \{⊥\}, \{¬\}, \{↛\}, \{⊕\}; \ldots{} \{⊤, ↔, ∧\}

\{↔, ∨\}; \{¬\}, \{↛\}, \{⊕\}, \{⊥\}; \ldots{} \{⊤, →, ∧\}

\{↔, ∧\}; \{¬\}, \{↛\}, \{⊕\}, \{⊥\}; \ldots{} \{⊤, →, ∨\}

\hypertarget{false-preserving-generator}{%
\subsubsection{False Preserving
Generator}\label{false-preserving-generator}}

\{↛, ∨\}; \{⊤\}, \{¬\}, \{→\}, \{↔\}

\hypertarget{false-preserving}{%
\subsubsection{False Preserving}\label{false-preserving}}

\{⊕, ∧\}; \{¬\}, \{→\}, \{↔\}, \{⊤\}; \ldots{} \{⊥, ↛, ∨\}

\{⊕, ∨\}; \{¬\}, \{→\}, \{↔\}, \{⊤\}; \ldots{} \{⊥, ↛, ∧\}

\{⊥, ↛\}; \{⊤\}, \{¬\}, \{→\}, \{↔\}; \ldots{} \{⊕, ∨, ∧\}

\{↛, ⊕\}; \{⊤\}, \{¬\}, \{→\}, \{↔\}; \ldots{} \{⊥, ∨, ∧\}

\{↛, ∧\}; \{⊤\}, \{¬\}, \{→\}, \{↔\}; \ldots{} \{⊥, ⊕, ∨\}

\hypertarget{monotonic-and-truth-preserving}{%
\subsubsection{Monotonic and Truth
Preserving}\label{monotonic-and-truth-preserving}}

\{⊤, ∨\}; \{¬\}, \{↛\}, \{⊕\}, \{⊥, ↔\}; \ldots{} \{→, ↔, ∧\}

\{⊤, ∧\}; \{¬\}, \{↛\}, \{⊕\}, \{⊥, ↔\}; \ldots{} \{→, ↔, ∨\}

\hypertarget{monotonic-and-false-preserving}{%
\subsubsection{Monotonic and False
Preserving}\label{monotonic-and-false-preserving}}

\{⊥, ∨\}; \{¬\}, \{→\}, \{↔\}, \{⊤, ⊕\}; \ldots, \{↛, ⊕, ∧\}

\{⊥, ∧\}; \{¬\}, \{→\}, \{↔\}, \{⊤, ⊕\}; \ldots, \{↛, ⊕, ∨\}

\hypertarget{section}{%
\subsubsection{}\label{section}}

\hypertarget{special}{%
\subsubsection{Special}\label{special}}

Truth or False Preserving

\{∧, ∨\}; \{¬\}, \{⊥, ↔\}, \{⊤, ⊕\}, \{↔, ⊕\}; \ldots{} \{⊥, ↛, ⊕\},
\{⊤, →, ↔\}

\hypertarget{triples-1}{%
\subsection{Triples}\label{triples-1}}

32 functionally incomplete triples.

\hypertarget{affine-connectives}{%
\subsubsection{Affine Connectives}\label{affine-connectives}}

\hypertarget{generators-1}{%
\paragraph{Generators}\label{generators-1}}

\{⊥, ¬, ⊕\}; \{→\}, \{↛\}, \{∧\}, \{∨\}; \{⊤\}, \{↔\}, \{⊤, ↔\}

\{⊥, ¬, ↔\}; \{→\}, \{↛\}, \{∧\}, \{∨\}; \{⊤\}, \{⊕\}, \{⊤, ⊕\}

\{⊤, ¬, ⊕\}; \{→\}, \{↛\}, \{∧\}, \{∨\}; \{⊥\}, \{↔\}, \{⊥, ↔\}

\{⊤, ¬, ↔\}; \{→\}, \{↛\}, \{∧\}, \{∨\}; \{⊥\}, \{⊕\}, \{⊥, ⊕\}

\{¬, ⊕, ↔\}; \{→\}, \{↛\}, \{∧\}, \{∨\}; \{⊥\}, \{⊤\}, \{⊥, ⊤\}

\hypertarget{non-generators-1}{%
\paragraph{Non-generators}\label{non-generators-1}}

\{⊥, ⊤, ¬\}; \{→\}, \{↛\}, \{∧\}, \{∨\}; \{⊕\}, \{↔\}, \{⊕, ↔\}

\{⊥, ⊤, ↔\}; \{→\}, \{↛\}, \{∧\}, \{∨\}; \{¬\}, \{⊕\}, \{¬, ⊕\}

\{⊥, ⊤, ⊕\}; \{→\}, \{↛\}, \{∧\}, \{∨\}; \{¬\}, \{↔\}, \{¬, ↔\}

\{⊥, ⊕, ↔\}; \{→\}, \{↛\}, \{∧\}, \{∨\}; \{⊤\}, \{¬\}, \{⊤, ¬\}

\{⊤, ⊕, ↔\}; \{→\}, \{↛\}, \{∧\}, \{∨\}; \{⊥\}, \{¬\}, \{⊥, ¬\}

\hypertarget{truth-preserving-connectives}{%
\subsubsection{Truth Preserving
Connectives}\label{truth-preserving-connectives}}

\hypertarget{generators-2}{%
\paragraph{Generators}\label{generators-2}}

\{⊤, →, ∧\}; \{⊥\}, \{¬\}, \{↛\}, \{⊕\}; \{↔\}, \{∨\}, \{↔, ∨\}

\{→, ↔, ∧\}; \{⊥\}, \{¬\}, \{↛\}, \{⊕\}; \{⊤\}, \{∨\}, \{⊤, ∨\}

\{→, ∧, ∨\}; \{⊥\}, \{¬\}, \{↛\}, \{⊕\}; \{⊤\}, \{↔\}, \{⊤, ↔\}

\hypertarget{non-generators-2}{%
\paragraph{Non-generators}\label{non-generators-2}}

\{⊤, →, ↔\}; \{⊥\}, \{¬\}, \{↛\}, \{⊕\}; \{∧\}, \{∨\}, \{∧, ∨\}

\{⊤, →, ∨\}; \{⊥\}, \{¬\}, \{↛\}, \{⊕\}; \{↔\}, \{∧\}, \{↔, ∧\}\\
\{⊤, ↔, ∨\}; \{⊥\}, \{¬\}, \{↛\}, \{⊕\}; \{→\}, \{∧\}, \{→, ∧\}\\
\{⊤, ↔, ∧\}; \{⊥\}, \{¬\}, \{↛\}, \{⊕\}; \{→\}, \{∨\}, \{→, ∨\}\\
\{→, ↔, ∨\}; \{⊥\}, \{¬\}, \{↛\}, \{⊕\}; \{⊤\}, \{∧\}, \{⊤, ∧\}\\
\{↔, ∧, ∨\}; \{⊥\}, \{¬\}, \{↛\}, \{⊕\}; \{⊤\}, \{→\}, \{⊤, →\}

\hypertarget{false-preserving-connectives}{%
\subsubsection{False Preserving
Connectives}\label{false-preserving-connectives}}

\{⊥, ↛, ⊕\}; \{⊤\}, \{¬\}, \{→\}, \{↔\}; \{∧\}, \{∨\}, \{∧, ∨\}

\{⊥, ⊕, ∨\}; \{⊤\}, \{¬\}, \{→\}, \{↔\}; \{↛\}, \{∧\}, \{↛, ∧\}

\{⊥, ⊕, ∧\}; \{⊤\}, \{¬\}, \{→\}, \{↔\}; \{↛\}, \{∨\}, \{↛, ∨\}

\{⊥, ↛, ∧\}; \{⊤\}, \{¬\}, \{→\}, \{↔\}; \{⊕\}, \{∨\}, \{⊕, ∨\}

\{⊥, ↛, ∨\}; \{⊤\}, \{¬\}, \{→\}, \{↔\}; \{∧\}, \{⊕\}, \{⊕, ∧\}

\{↛, ⊕, ∧\}; \{⊤\}, \{¬\}, \{→\}, \{↔\}; \{⊥\}, \{∨\}, \{⊥, ∨\}

\{↛, ⊕, ∨\}; \{⊤\}, \{¬\}, \{→\}, \{↔\}; \{⊥\}, \{∧\}, \{⊥, ∧\}\\
\{↛, ∧, ∨\}; \{⊤\}, \{¬\}, \{→\}, \{↔\}; \{⊥\}, \{⊕\}, \{⊥, ⊕\}

\{⊕, ∧, ∨\}; \{⊤\}, \{¬\}, \{→\}, \{↔\}; \{⊥\}, \{↛\}, \{⊥, ↛\}

\hypertarget{exclusively-monotonic-connectives}{%
\subsubsection{\texorpdfstring{Exclusively Monotonic Connectives
}{Exclusively Monotonic Connectives }}\label{exclusively-monotonic-connectives}}

\{⊥, ⊤, ∨\}; \{¬\}, \{→\}, \{↛\}, \{⊕\}, \{↔\}; \{∧\}

\{⊥, ⊤, ∧\}; \{¬\}, \{→\}, \{↛\}, \{⊕\}, \{↔\}; \{∨\}

\hypertarget{monotonic-intersections}{%
\subsubsection{Monotonic Intersections}\label{monotonic-intersections}}

Monotonic and Truth Preserving

\{⊥, ∧, ∨\}; \{¬\}, \{→\}, \{↔\}; \{⊤\}, \{↛\}, \{⊕\}, \{↛, ⊕\}

Monotonic and False Preserving\\
\{⊤, ∧, ∨\}; \{¬\}, \{↛\}, \{⊕\}; \{⊥\}, \{→\}, \{↔\}, \{→, ↔\}

\hypertarget{quadruples}{%
\subsection{Quadruples}\label{quadruples}}

16 functionally incomplete quadruples.

\hypertarget{affine-connectives-1}{%
\subsubsection{Affine Connectives}\label{affine-connectives-1}}

\{⊥, ⊤, ⊕, ↔\}; \{→\}, \{↛\}, \{∧\}, \{∨\}; \{¬\}

\{¬, ⊤, ⊕, ↔\}; \{→\}, \{↛\}, \{∧\}, \{∨\}; \{⊥\}

\{¬, ⊥, ⊕, ↔\}; \{→\}, \{↛\}, \{∧\}, \{∨\}; \{⊤\}

\{¬, ⊥, ⊤, ↔\}; \{→\}, \{↛\}, \{∧\}, \{∨\}; \{⊕\}

\{¬, ⊥, ⊤, ⊕\}; \{→\}, \{↛\}, \{∧\}, \{∨\}; \{↔\}

\hypertarget{truth-preserving-connectives-1}{%
\subsubsection{Truth Preserving
Connectives}\label{truth-preserving-connectives-1}}

\hypertarget{generators-3}{%
\paragraph{Generators}\label{generators-3}}

\{→, ⊤, ∧, ↔\}; \{⊥\}, \{¬\}, \{↛\}, \{⊕\}; \{∨\}

\{→, ∧, ∨, ↔\}; \{⊥\}, \{¬\}, \{↛\}, \{⊕\}; \{⊤\}

\{→, ⊤, ∧, ∨\}; \{⊥\}, \{¬\}, \{↛\}, \{⊕\}; \{↔\}

\hypertarget{non-generators-3}{%
\paragraph{Non-generators}\label{non-generators-3}}

\{⊤, ∧, ∨, ↔\}; \{⊥\}, \{¬\}, \{↛\}, \{⊕\}; \{→\}

\{→, ⊤, ∨, ↔\}; \{⊥\}, \{¬\}, \{↛\}, \{⊕\}; \{∧\}

\hypertarget{false-preserving-connectives-1}{%
\subsubsection{False Preserving
Connectives}\label{false-preserving-connectives-1}}

\hypertarget{generators-4}{%
\paragraph{Generators}\label{generators-4}}

\{⊥, ↛, ∨, ⊕\}; \{⊤\}, \{¬\}, \{→\}, \{↔\}; \{∧\}

\{⊥, ↛, ∧, ∨\}; \{⊤\}, \{¬\}, \{→\}, \{↔\}; \{⊕\}

\{↛, ∧, ∨, ⊕\}; \{⊤\}, \{¬\}, \{→\}, \{↔\}; \{⊥\}

\hypertarget{non-generators-4}{%
\paragraph{Non-generators}\label{non-generators-4}}

\{⊥, ↛, ∧, ⊕\}; \{⊤\}, \{¬\}, \{→\}, \{↔\}; \{∨\}

\{⊥, ∧, ∨, ⊕\}; \{⊤\}, \{¬\}, \{→\}, \{↔\}; \{↛\}

\hypertarget{monotonic-connectives}{%
\subsubsection{\texorpdfstring{Monotonic Connectives
}{Monotonic Connectives }}\label{monotonic-connectives}}

\{⊥, ⊤, ∧, ∨\}; \{¬\}, \{→\}, \{↛\}, \{⊕\}, \{↔\}

\hypertarget{quintuples}{%
\subsection{Quintuples}\label{quintuples}}

\hypertarget{affine-connectives-2}{%
\subsubsection{Affine Connectives}\label{affine-connectives-2}}

\{⊥, ⊤, ¬, ⊕, ↔\}; \{→\}, \{↛\}, \{∧\}, \{∨\}

\hypertarget{truth-preserving-connectives-2}{%
\subsubsection{Truth Preserving
Connectives}\label{truth-preserving-connectives-2}}

\{⊤, →, ↔, ∧, ∨\}; \{⊥\}, \{¬\}, \{↛\}, \{⊕\}

\hypertarget{false-preserving-connectives-2}{%
\subsubsection{False Preserving
Connectives}\label{false-preserving-connectives-2}}

\{⊥, ↛, ⊕, ∨, ∧\}; \{⊤\}, \{¬\}, \{→\}, \{↔\}

\hypertarget{sextuples-and-higher}{%
\subsection{Sextuples and higher}\label{sextuples-and-higher}}

All subsets of \{⊥, ⊤, ¬, →, ↛, ⊕, ↔, ∧, ∨\} of cardinality 6 or higher
are functionally complete.

Proof: \{⊥, ⊤, ⊕, ↔, ∧, ∨\} and \{¬, →, ↛\} are functionally complete.
Every 5 element subset of \{⊥, ⊤, ⊕, ↔, ∧, ∨\} is functionally complete;
there exist 4 element subsets which are functionally incomplete, but the
only way to get six element sets from those 4 element sets is to add two
elements from \{¬, →, ↛\} and any two elements of \{¬, →, ↛\} are
functionally complete together.

\hypertarget{special-cases-of-functional-incomplete-sets-of-operators}{%
\subsection{Special Cases of functional incomplete sets of
operators}\label{special-cases-of-functional-incomplete-sets-of-operators}}

\{⊥, ⊤, ⊕, ↔\} ⊬ \{¬\}

\{⊤, →, ↔, ∨\} ⊬ \{∨\}

\{⊥, ↛, ⊕, ∧\} ⊬ \{∧\}

\{¬, ⊕\} ⊢ \{⊥, ⊤, ¬, ⊕, ↔\}

\{¬, ↔\} ⊢ \{⊥, ⊤, ¬, ⊕, ↔\}

\{→, ∨\} ⊢ \{⊤, →, ∨\}

\{→, ∧\} ⊢ \{⊤, →, ↔, ∧\}

\{↛, ∨\} ⊢ \{⊥, ↛, ⊕, ∨\}

\{⊕, ↔\} ⊢ \{⊥, ⊤, ⊕, ↔\}

\{→, ↔\} ⊢ \{⊤, →, ↔\}

\{↛, ⊕\} ⊢ \{⊥, ↛, ⊕\}

\{↛\}⊢\{⊥, ↛\}

\{⊕\}⊢\{⊥, ⊕\}

\{→\}⊢\{⊤,→\}

\{↔\}⊢\{⊤, ↔\}

\hypertarget{functionally-complete-sets}{%
\section{Functionally Complete Sets}\label{functionally-complete-sets}}

There are 512 subsets of \{⊥, ⊤, ¬, →, ↛, ⊕, ↔, ∧, ∨\}. 425 subsets are
functionally complete.

There are 2048 subsets of \{⊥, ⊤, ¬, →, ↛, ⊕, ↔, ∧, ∨, ↓, ↑\}. 1961
subsets are functionally complete.

NUMERICAL ENCODING! Based on the MAXIMALLY INCOMPLETE SETS

Class 1: truth-preserving 10000\\
Class 2: false-preserving 01000\\
Class 3: affine 00100

Class 4: monotone 00010

Class 5: self-dual 00001

Any connective can be expressed uniquely as a binary numeral.\\
AND is truth preserving. 10010

\{⊥, ⊤, ¬, ⊕, ↔\} Affine set

\{⊥, ↛, ⊕, ∨, ∧\} False Preserving set

\{⊤, →, ↔, ∧, ∨\} Truth Preserving set

\{⊥, ⊤, ∧, ∨\} Monotonic set

\{¬\} Self-Dual set

LIST OF NUMERICAL ENCODINGS!

⊥: false preserving, affine. 01100

⊤: truth preserving, affine. 10100

¬: self-dual: 00001, affine: 00100. 00101

→ Truth Preserving 10000

↛ False Preserving 01000

⊕ False Preserving, Affine 01100

↔ Truth Preserving, Affine 10100

∧ False Preserving, Truth Preserving, Monotonic 11010

∨ False Preserving, Truth Preserving, Monotonic 11010

↓: 00000

↑: 00000

PROVING WHICH SETS ARE FUNCTIONALLY COMPLETE

A functionally complete set is defined by the property that when all its
members are summed (with bitwise OR), the sum is 00000.

\{↓\}: 00000

\{↑\}: 00000

\{⊥, →\} : 01100 \textbar\textbar{} 10100 = 00100

AND: AB

OR: 1 - (1 - A)(1 - B) = A + B - AB

NAND: 1 - AB

NOR: 1 - A - B + AB

IMPLIES: 1 - A + AB

NONIMPLIES: A(1-B)

\end{document}}*}
