\documentclass{report}
\usepackage{graphicx}
\usepackage{amsmath}
\usepackage[utf8]{inputenc}
\usepackage[T1]{fontenc}
\usepackage{lmodern}
\usepackage{amssymb}
\usepackage{amsthm}
\usepackage{latexsym}
\usepackage{centernot}
\usepackage{enumitem}
\usepackage{geometry}
\geometry{a4paper, margin=1in}
\usepackage{hyperref}
\usepackage{ragged2e}
\usepackage{microtype}
\usepackage{bussproofs}
\usepackage{setspace}
\usepackage{tensor}
\usepackage{cite}

% Define theorem/definition environments
\newtheorem{definition}{Definition}[section]
\newtheorem{proposition}{Proposition}[section]
\newtheorem{theorem}{Theorem}[section]
\newtheorem{remark}{Remark}[section]
\newtheorem{analogy}{Analogy}[section]
\newtheorem{axiom}{Axiom}[section]
\newtheorem{lemma}{Lemma}[section]
\newtheorem{corollary}{Corollary}[section]
\newtheorem{example}{Example}[section]
\newtheorem{hypothesis}{Hypothesis}[section]

% Define logical symbols
\newcommand{\negation}{\neg}
\newcommand{\conjunction}{\land}
\newcommand{\disjunction}{\lor}
\newcommand{\implication}{\rightarrow}
\newcommand{\entails}{\vdash}
\newcommand{\notentails}{\centernot{\vdash}}

% Define complexity classes
\newcommand{\SigmaZero}{\Sigma^0}
\newcommand{\PiZero}{\Pi^0}
\newcommand{\DeltaZero}{\Delta^0}

% Define meta-semantic concepts
\newcommand{\MetaProps}{\mathbf{M}}
\newcommand{\ContextParams}{\mathbf{C}}
\newcommand{\InterpFunc}{\mathbf{I}}
\newcommand{\RelPrinciples}{\mathbf{RP}}
\newcommand{\EvalSystem}{\mathbf{E}}
\newcommand{\ValueSpace}{\mathbf{V}}
\newcommand{\MetaLang}{\mathbf{ML}}

% Define meta-properties
\newcommand{\Saf}{\text{Saf}}
\newcommand{\Sec}{\text{Sec}}
\newcommand{\Comp}{\text{Comp}}
\newcommand{\Paracons}{\text{Paracons}}
\newcommand{\Paracomp}{\text{Paracomp}}

% Define meta-language types
\newcommand{\MLClass}{\text{Class}}
\newcommand{\MLParacons}{\text{Paracons but not Paracomp}}
\newcommand{\MLParacomp}{\text{Paracomplete but not Paracons}}
\newcommand{\MLBoth}{\text{Paracons and Paracomplete}}

% Define commands for L_0, M_0, L_infty, LT, LF for use in text and math
\newcommand{\LZero}{L_0} % Retain for general concept
\newcommand{\LZeroN}{L^0_n} % For specific diamond D_n
\newcommand{\LvdashN}{L^{\vdash}_n} % New notation for L^0_n
\newcommand{\MZero}{\mathcal{M}_0}
\newcommand{\LInf}{L_\infty} % Retain for general concept
\newcommand{\LInfN}{L^\infty_n} % For specific diamond D_n
\newcommand{\LGammaDeltaN}{L^{\Gamma\vdash\Delta}_n} % New notation for L^infty_n
\newcommand{\LT}{L_T} % Define LT with subscript
\newcommand{\LvdashDeltaN}{L^{\vdash\Delta}_n} % New notation for LT
\newcommand{\LF}{L_F} % Define LF with subscript
\newcommand{\LGammavdashN}{L^{\Gamma\vdash}_n} % New notation for LF


% Use texorpdfstring for titles and sectioning commands with math
\newcommand{\pdfLZero}{\texorpdfstring{\LZero}{L0}}
\newcommand{\pdfLZeroN}{\texorpdfstring{\LZeroN}{L0n}}
\newcommand{\pdfLvdashN}{\texorpdfstring{\LvdashN}{Lvdashn}}
\newcommand{\pdfMZero}{\texorpdfstring{\MZero}{M0}}
\newcommand{\pdfLInf}{\texorpdfstring{\LInf}{Linf}}
\newcommand{\pdfLInfN}{\texorpdfstring{\LInfN}{Linfn}}
\newcommand{\pdfLGammaDeltaN}{\texorpdfstring{\LGammaDeltaN}{LGammaDeltaN}}
\newcommand{\pdfLT}{\texorpdfstring{\LT}{LT}} % Define pdfLT with subscript
\newcommand{\pdfLvdashDeltaN}{\texorpdfstring{\LvdashDeltaN}{LvdashDeltaN}}
\newcommand{\pdfLF}{\texorpdfstring{\LF}{LF}} % Define pdfLF with subscript
\newcommand{\pdfLGammavdashN}{\texorpdfstring{\LGammavdashN}{LGammavdashN}}

\newcommand{\mathbbm}[1]{\text{\mathbb{#1}}}
\newcommand{\mathbfm}[1]{\mathbf{#1}}
\newcommand{\mathcalm}[1]{\mathcal{m}}
\newcommand{\rmm}[1]{\mathrm{#1}}

\title{My Thesis}
\author{Your Name}
\date{\today}

\begin{document}

\maketitle

\tableofcontents

\part{Physics}

\chapter{BHTD}
%input{papers/physics/bhtd.tex}

\chapter{BH Heisenberg Clock}
%input{papers/physics/bh-heisenberg-clock.tex}

\chapter{BS}
%input{papers/physics/bs.tex}

\chapter{Circuit Calc}
%input{papers/physics/circuitcalc.tex}

\chapter{Contradictory Universe}
%input{papers/physics/contradictoryuniverse.tex}

\chapter{Graphene Silicene Unit}
%input{papers/physics/graphenesiliceneunit.tex}

\chapter{Graphene Silicene Unit 0}
%input{papers/physics/graphenesiliceneunit0.tex}

\chapter{Measurement Theories}
%input{papers/physics/measurementtheories.tex}

\chapter{Measurement Theories 0}
%input{papers/physics/measurementtheories0.tex}

\chapter{Measurement Theories 1}
%input{papers/physics/measurementtheories1.tex}

\chapter{Planck Mass Black Hole}
%input{papers/physics/planckmassblackhole.tex}

\chapter{Planck Mass Black Hole conjectures}
%input{papers/physics/planckmassblackhole-conjectures.tex}

\chapter{Planck Mass Black Hole conjectures 0}
%input{papers/physics/planckmassblackhole-conjectures0.tex}

\chapter{Planck Mass Black Hole conjectures 1}
%input{papers/physics/planckmassblackhole-conjectures1.tex}

\chapter{Quantum Coherence Entropy}
%input{papers/physics/quantumcoherenceentropy.tex}

\part{Logic and Language}

\chapter{Algorithm Of Paradoxes}
%input{papers/logic_and_language/algorithmofparadoxes.tex}

\chapter{Algorithmic Decidability}
%input{papers/logic_and_language/algorithmicdecidability.docx.tex}

\chapter{Categorical Metalanguages}
%input{papers/logic_and_language/categoricalmetalanguages.tex}

\chapter{Common Fork Revise Undecidable Theories Part I.2}
%input{papers/logic_and_language/common-fork-revise-undecidable-theories-part-i.2.odt.tex}

\chapter{Constructive Procedures}
%input{papers/logic_and_language/constructive-procedures.docx.tex}

\chapter{DMALC}
%input{papers/logic_and_language/dmalc.tex}

\chapter{Existential Order Additive Sequent}
%input{papers/logic_and_language/existentialorderadditivesequent.tex}

\chapter{First Order Additive Sequent}
%input{papers/logic_and_language/firstorderadditivesequent.tex}

\chapter{First Order Structural Additive Sequent}
%input{papers/logic_and_language/firstorderstructuraladditivesequent.tex}

\chapter{Functionally Incomplete}
%input{papers/logic_and_language/functionally-incomplete.tex}

\chapter{Fundamentally in Error the Growth of Errors}
%input{papers/logic_and_language/fundamentally-in-error-the-growth-of-errors.docx.tex}

\chapter{Graph of Calculi Philosophy}
%input{papers/logic_and_language/graphofcalculiphilosophy.tex}

\chapter{Identity Calculus}
%input{papers/logic_and_language/identitycalculus.tex}

\chapter{Latin Theses}
%input{papers/logic_and_language/latintheses.tex}

\chapter{Local Relations}
%input{papers/logic_and_language/localrelations.tex}

\chapter{MALC}
%input{papers/logic_and_language/malc.tex}

\chapter{MAOLL}
%input{papers/logic_and_language/maoll.tex}

\chapter{MLC}
%input{papers/logic_and_language/mlc.tex}

\chapter{Moebius Logic}
%input{papers/logic_and_language/moebiuslogic.tex}

\chapter{OMALL}
%input{papers/logic_and_language/omall.tex}

\chapter{OMAffine}
%input{papers/logic_and_language/omaffine.tex}

\chapter{OMultiplicative}
%input{papers/logic_and_language/omultiplicative.tex}

\chapter{Order Additive Mix Visible Sequent}
%input{papers/logic_and_language/orderadditivemixvisiblesequent.tex}

\chapter{Order Additive NANDNOR Sequent}
%input{papers/logic_and_language/orderadditivenandnorsequent.tex}

\chapter{Order Basic Sequent}
%input{papers/logic_and_language/orderbasicsequent.tex}

\chapter{Ordinal Basic Sequent}
%input{papers/logic_and_language/ordinalbasicsequent.tex}

\chapter{Ordinary Monotonic Conjunct Sequent}
%input{papers/logic_and_language/ordinarymonotonicconjunctsequent.tex}

\chapter{Ordinary Monotonic Disjunct Sequent}
%input{papers/logic_and_language/ordinarymonotonicdisjunctsequent.tex}

\chapter{Ordinary Monotonic Sequent}
%input{papers/logic_and_language/ordinarymonotonicsequent.tex}

\chapter{Research Proposal Graph of Supercalculi, conjugat}
%input{papers/logic_and_language/research-proposal-graph-of-supercalculi,-conjugat.docx.tex}

\chapter{Research Proposal Subclassical Graph of Calculi}
%input{papers/logic_and_language/research-proposal-subclassical-graph-of-calculi.docx.tex}

\chapter{Sequence A}
%input{papers/logic_and_language/sequencea.txt.tex}

\chapter{Sequence B}
%input{papers/logic_and_language/sequenceb.txt.tex}

\chapter{Signatures of Logos Theories}
%input{papers/logic_and_language/signatures-of-logos-theories.docx.tex}

\chapter{Subclassical Systems}
%input{papers/logic_and_language/subclassicalsystems.tex}

\chapter{Theories in Standard Formalization}
%input{papers/logic_and_language/theories-in-standard-formalization.docx.tex}

\chapter{UD Theorem Summary}
%input{papers/logic_and_language/ud-theorem-summary.docx.tex}

\end{document}