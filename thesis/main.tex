\documentclass{report}
\usepackage{graphicx}
\usepackage{amsmath}
\usepackage[utf8]{inputenc}
\usepackage[T1]{fontenc}
\usepackage{lmodern}
\usepackage{amssymb}
\usepackage{amsthm}
\usepackage{latexsym}
\usepackage{centernot}
\usepackage{enumitem}
\usepackage{geometry}
\geometry{a4paper, margin=1in}
\usepackage{hyperref}
\usepackage{ragged2e}
\usepackage{microtype}
\usepackage{bussproofs}
\usepackage{setspace}
\usepackage{tensor}
\usepackage{cite}

% Define theorem/definition environments
\newtheorem{definition}{Definition}[section]
\newtheorem{proposition}{Proposition}[section]
\newtheorem{theorem}{Theorem}[section]
\newtheorem{remark}{Remark}[section]
\newtheorem{analogy}{Analogy}[section]
\newtheorem{axiom}{Axiom}[section]
\newtheorem{lemma}{Lemma}[section]
\newtheorem{corollary}{Corollary}[section]
\newtheorem{example}{Example}[section]
\newtheorem{hypothesis}{Hypothesis}[section]

% Define logical symbols
\newcommand{\negation}{\neg}
\newcommand{\conjunction}{\land}
\newcommand{\disjunction}{\lor}
\newcommand{\implication}{\rightarrow}
\newcommand{\entails}{\vdash}
\newcommand{\notentails}{\centernot{\vdash}}

% Define complexity classes
\newcommand{\SigmaZero}{\Sigma^0}
\newcommand{\PiZero}{\Pi^0}
\newcommand{\DeltaZero}{\Delta^0}

% Define meta-semantic concepts
\newcommand{\MetaProps}{\mathbf{M}}
\newcommand{\ContextParams}{\mathbf{C}}
\newcommand{\InterpFunc}{\mathbf{I}}
\newcommand{\RelPrinciples}{\mathbf{RP}}
\newcommand{\EvalSystem}{\mathbf{E}}
\newcommand{\ValueSpace}{\mathbf{V}}
\newcommand{\MetaLang}{\mathbf{ML}}

% Define meta-properties
\newcommand{\Saf}{\text{Saf}}
\newcommand{\Sec}{\text{Sec}}
\newcommand{\Comp}{\text{Comp}}
\newcommand{\Paracons}{\text{Paracons}}
\newcommand{\Paracomp}{\text{Paracomp}}

% Define meta-language types
\newcommand{\MLClass}{\text{Class}}
\newcommand{\MLParacons}{\text{Paracons but not Paracomp}}
\newcommand{\MLParacomp}{\text{Paracomplete but not Paracons}}
\newcommand{\MLBoth}{\text{Paracons and Paracomplete}}

% Define commands for L_0, M_0, L_infty, LT, LF for use in text and math
\newcommand{\LZero}{L_0} % Retain for general concept
\newcommand{\LZeroN}{L^0_n} % For specific diamond D_n
\newcommand{\LvdashN}{L^{\vdash}_n} % New notation for L^0_n
\newcommand{\MZero}{\mathcal{M}_0}
\newcommand{\LInf}{L_\infty} % Retain for general concept
\newcommand{\LInfN}{L^\infty_n} % For specific diamond D_n
\newcommand{\LGammaDeltaN}{L^{\Gamma\vdash\Delta}_n} % New notation for L^infty_n
\newcommand{\LT}{L_T} % Define LT with subscript
\newcommand{\LvdashDeltaN}{L^{\vdash\Delta}_n} % New notation for LT
\newcommand{\LF}{L_F} % Define LF with subscript
\newcommand{\LGammavdashN}{L^{\Gamma\vdash}_n} % New notation for LF


% Use texorpdfstring for titles and sectioning commands with math
\newcommand{\pdfLZero}{\texorpdfstring{\LZero}{L0}}
\newcommand{\pdfLZeroN}{\texorpdfstring{\LZeroN}{L0n}}
\newcommand{\pdfLvdashN}{\texorpdfstring{\LvdashN}{Lvdashn}}
\newcommand{\pdfMZero}{\texorpdfstring{\MZero}{M0}}
\newcommand{\pdfLInf}{\texorpdfstring{\LInf}{Linf}}
\newcommand{\pdfLInfN}{\texorpdfstring{\LInfN}{Linfn}}
\newcommand{\pdfLGammaDeltaN}{\texorpdfstring{\LGammaDeltaN}{LGammaDeltaN}}
\newcommand{\pdfLT}{\texorpdfstring{\LT}{LT}} % Define pdfLT with subscript
\newcommand{\pdfLvdashDeltaN}{\texorpdfstring{\LvdashDeltaN}{LvdashDeltaN}}
\newcommand{\pdfLF}{\texorpdfstring{\LF}{LF}} % Define pdfLF with subscript
\newcommand{\pdfLGammavdashN}{\texorpdfstring{\LGammavdashN}{LGammavdashN}}

\newcommand{\mathbbm}[1]{\text{\mathbb{#1}}}
\newcommand{\mathbfm}[1]{\mathbf{#1}}
\newcommand{\mathcalm}[1]{\mathcal{m}}
\newcommand{\rmm}[1]{\mathrm{#1}}

\title{My Thesis}
\author{Your Name}
\date{\today}

\begin{document}

\maketitle

\tableofcontents

\part{Physics}

\chapter{BHTD}
\begin{equation*}\n\displaystyle \begin{array}{lll}\\
 \displaystyle  \partial M = \frac{\kappa}{8 \pi} \partial A + \Omega \partial J + \Phi \partial Q \\\\
 \displaystyle  \partial M = 0 \text{ and } \partial A = 0, \partial J = - \frac{\Phi \partial Q}{\Omega} \\\\
 \displaystyle  \partial M = 0 \text{ and } \partial A = 0, \partial Q = - \frac{\Omega \partial J}{\Phi} \\\\
 \displaystyle  \partial M = 0, \partial J = -\frac{(\frac{\kappa}{8 \pi} \partial A + \Phi \partial Q)}{\Omega} \\\\
 \displaystyle  \partial M = 0, \partial Q = -\frac{(\frac{\kappa}{8 \pi} \partial A + \Omega \partial J)}{\Phi}  \\\\

\end{array}

\displaystyle \begin{array}{lll}\\
 \displaystyle  \partial M = \frac{\kappa}{8 \pi} \partial A + \Omega \partial J + \Phi \partial Q \\\\
 \displaystyle  0 = \frac{\kappa}{8 \pi} \partial A + \Omega \partial J + \Phi \partial Q - \partial M \\\\
 \displaystyle  \frac{\kappa}{8 \pi} \partial A = \partial M - \Omega \partial J - \Phi \partial Q \\\\
 \displaystyle  \Phi \partial Q = \partial M - \frac{\kappa}{8 \pi} \partial A - \Omega \partial J \\\\
 \displaystyle  \Omega \partial J = \partial M - \frac{\kappa}{8 \pi} \partial A - \Phi \partial Q \\\\

\end{array}

\displaystyle \begin{array}{lll}\\
 \displaystyle  \partial M = \frac{\kappa}{8 \pi} \partial A + \Omega \partial J + \Phi \partial Q \\\\
 \displaystyle  0 = \frac{\kappa}{8 \pi} \partial A + \Omega \partial J + \Phi \partial Q - \partial M \\\\
 \displaystyle  \frac{\kappa}{8 \pi} \partial A = \partial M - \Omega \partial J - \Phi \partial Q \\\\
 \displaystyle  \partial A = \frac{8 \pi(\partial M - \Omega \partial J - \Phi \partial Q)}{\kappa}\\\\

 \displaystyle  \Phi \partial Q = \partial M - \frac{\kappa}{8 \pi} \partial A - \Omega \partial J \\\\
 \displaystyle  \partial Q = \frac{\partial M - \frac{\kappa}{8 \pi} \partial A - \Omega \partial J}{\Phi} \\\\

 \displaystyle  \Omega \partial J = \partial M - \frac{\kappa}{8 \pi} \partial A - \Phi \partial Q \\\\
 \displaystyle  \partial J = \frac{\partial M - \frac{\kappa}{8 \pi} \partial A - \Phi \partial Q}{\Omega} \\\\

\end{array}

\displaystyle \begin{array}{lll}\\
 \displaystyle  \partial M = \frac{\kappa}{8 \pi} \partial A + \Omega \partial J + \Phi \partial Q \\\\
 \displaystyle  \partial A = \frac{8 \pi(\partial M - \Omega \partial J - \Phi \partial Q)}{\kappa}\\\\

 \displaystyle  \partial M - \partial A = \frac{\kappa}{8 \pi} \partial A + \Omega \partial J + \Phi \partial Q - \frac{8 \pi(\partial M - \Omega \partial J - \Phi \partial Q)}{\kappa}\\\\
 \displaystyle  \partial M - \partial A = \frac{\kappa}{8 \pi} \partial A + \Omega \partial J + \Phi \partial Q - (\frac{8 \pi \partial M}{\kappa} - \frac{8 \pi \Omega \partial J}{\kappa} - \frac{8 \pi \Phi \partial Q}{\kappa})\\\\
 \displaystyle  \partial M - \partial A = \frac{\kappa}{8 \pi} \partial A + \Omega \partial J + \Phi \partial Q - \frac{8 \pi \partial M}{\kappa} + \frac{8 \pi \Omega \partial J}{\kappa} + \frac{8 \pi \Phi \partial Q}{\kappa}\\\\
 \displaystyle  \partial M + \frac{8 \pi \partial M}{\kappa} - (\partial A + \frac{\kappa}{8 \pi} \partial A) = \Omega \partial J + \frac{8 \pi \Omega \partial J}{\kappa} + \Phi \partial Q + \frac{8 \pi \Phi \partial Q}{\kappa}\\\\

\end{array}

\displaystyle \begin{array}{lll}\\
 \displaystyle  0 = \frac{\kappa}{8 \pi} \partial A + \Omega \partial J + \Phi \partial Q - \partial M \\\\

 \displaystyle  \partial M = \frac{\kappa}{8 \pi} \partial A + \Omega \partial J + \Phi \partial Q \\\\


 \displaystyle  \partial A = \frac{8 \pi\partial M}{\kappa} - \frac{8 \pi \Omega \partial J}{\kappa} - \frac{8 \pi \Phi \partial Q}{\kappa}\\\\

 \displaystyle  \partial Q = \frac{\partial M}{\Phi} - \frac{\kappa}{8 \pi} \frac{ \partial A}{\Phi} - \frac{\Omega \partial J}{\Phi} \\\\

 \displaystyle  \partial J = \frac{\partial M}{\Omega} - \frac{\kappa}{8 \pi} \frac{ \partial A}{\Omega} - \frac{\Phi \partial Q}{\Omega} \\\\

\end{array}

\displaystyle \begin{array}{lll}\\
 \displaystyle \frac{\kappa}{8 \pi} \partial A + \Omega \partial J + \Phi \partial Q - \partial M = 0 \\\\

\begin{bmatrix}
\partial M \\
\partial A \\
\partial J \\
\partial Q \\
\end{bmatrix}
=
\begin{bmatrix}
1\\
\frac{8 \pi}{\kappa}\\
\frac{1}{\Omega}\\
\frac{1}{\Phi}\\
\end{bmatrix}

\begin{bmatrix}
0 & \frac{\kappa}{8 \pi} \partial A & \Omega \partial J & \Phi \partial Q\\
\partial M & 0 & - \Omega \partial J & - \Phi \partial Q\\
\partial M & - \frac{\kappa}{8 \pi} \partial A & 0 & - \Phi \partial Q\\
\partial M & - \frac{\kappa}{8 \pi} \partial A & - \Omega \partial J & 0\\
\end{bmatrix}


\end{array}\n\end{equation*}


\chapter{BH Heisenberg Clock}
Suppose we setup a situation such that a minimal black hole is formed and evaporated in a cycle and we use the time to evaporate as a clock in the system.

The change in energy of the black hole at each cycle would be related directly to the change of the time in the clock at each cycle. The Heisenberg uncertainty relation would be relevant.


\chapter{BS}
\section{Structural Rules}

\begin{center}
	\[
	\begin{prooftree}
	\infer0[Id]{A $\vdash$  A}
	\end{prooftree}
	\]
	
	\[
	\begin{prooftree}
	\hypo{$\Gamma$  $\vdash$  A}
	\hypo{A $\vdash$  $\Delta$ }
	\infer2[Cut]{$\Gamma$  $\vdash$  $\Delta$ }
	\end{prooftree}
	\]
\end{center}

\section{Unit Rules}
\begin{center}
		\[
	\begin{prooftree}
	\hypo{$\Gamma$ _{L}, $\Gamma$ _{R} $\vdash$  C}
	\infer1{$\Gamma$ _{L}, 1, $\Gamma$ _{R} $\vdash$  C}
	\end{prooftree}
	\quad
	\begin{prooftree}
	\hypo{C $\vdash$  $\Delta$ _{L}, $\Delta$ _{R}}
	\infer1{C $\vdash$  $\Delta$ _{L}, $\bot$ , $\Delta$ _{R}}
	\end{prooftree}
	\]
	
	\[
	\begin{prooftree}
	\infer0{ $\bot$  $\vdash$  }
	\end{prooftree}
	\quad
	\begin{prooftree}
	\infer0{ $\vdash$  1}
	\end{prooftree}
	\]
	
	\[
	\begin{prooftree}
	\infer0{ 0 $\vdash$  $\Delta$ }
	\end{prooftree}
	\quad
	\begin{prooftree}
	\infer0{ $\Gamma$  $\vdash$  $\top$ }
	\end{prooftree}
	\]
\end{center}

\section{Operational Rules}
\begin{center}
	\subsection{Negations}
	\begin{center}
		\[
		\begin{prooftree}
		\hypo{$\vdash$  A}
		\infer1{$\neg$  A $\vdash$  }
		\end{prooftree}
		\quad
		\begin{prooftree}
		\hypo{ A $\vdash$  }
		\infer1{ $\vdash$  $\neg$  A}
		\end{prooftree}
		\]
	\end{center}

	\subsection{Additives}
	\begin{center}
		\[
		\begin{prooftree}
		\hypo{A $\vdash$  $\Delta$ }
		\hypo{B $\vdash$  $\Delta$ }
		\infer2{A $\parr$  B $\vdash$  $\Delta$ }
		\end{prooftree}
		\quad
		\begin{prooftree}
		\hypo{$\Gamma$  $\vdash$  A, B }
		\infer1{$\Gamma$  $\vdash$  A $\parr$  B}
		\end{prooftree}
		\]
		
		\[
		\begin{prooftree}
		\hypo{$\Gamma$ , A, $\Phi$  $\vdash$  $\Psi$ }
		\infer1{$\Gamma$ , A \& B, $\Phi$  $\vdash$  $\Psi$ }
		\end{prooftree}
		\quad
		\begin{prooftree}
		\hypo{$\Gamma$ , B, $\Phi$  $\vdash$  $\Psi$ }
		\infer1{$\Gamma$ , A \& B, $\Phi$  $\vdash$  $\Psi$ }
		\end{prooftree}
		\quad
		\begin{prooftree}
		\hypo{ $\Gamma$  $\vdash$  $\Delta$ , B, $\Psi$ }
		\hypo{ $\Gamma$  $\vdash$  $\Delta$ , A, $\Psi$ }
		\infer2{ $\Gamma$  $\vdash$  $\Delta$ , B \& A, $\Psi$ }
		\end{prooftree}
		\]
		
		\[
		\begin{prooftree}
		\hypo{$\Gamma$ , A, $\Phi$  $\vdash$  C}
		\hypo{$\Gamma$ , B, $\Phi$  $\vdash$  C}
		\infer2{$\Gamma$ , A $\oplus$  B, $\Phi$  $\vdash$  C}
		\end{prooftree}
		\quad
		\begin{prooftree}
		\hypo{$\Gamma$  $\vdash$  A}
		\infer1{$\Gamma$  $\vdash$  A $\oplus$  B}
		\end{prooftree}
		\quad
		\begin{prooftree}
		\hypo{$\Gamma$  $\vdash$  B}
		\infer1{$\Gamma$  $\vdash$  A $\oplus$  B}
		\end{prooftree}
		\quad
		\begin{prooftree}
		\hypo{A $\vdash$  $\Delta$ }
		\infer1{A \& B $\vdash$  $\Delta$ }
		\end{prooftree}
		\quad
		\begin{prooftree}
		\hypo{B $\vdash$  $\Delta$ }
		\infer1{A \& B $\vdash$  $\Delta$ }
		\end{prooftree}
		\quad
		\begin{prooftree}
		\hypo{ C $\vdash$  $\Psi$ , B, $\Delta$ }
		\hypo{ C $\vdash$  $\Psi$ , A, $\Delta$ }
		\infer2{ C $\vdash$  $\Psi$ , B \& A, $\Delta$ }
		\end{prooftree}
		\]
		
		\[
		\begin{prooftree}
		\hypo{$\Gamma$ , A, $\Phi$  $\vdash$  C}
		\infer1{$\Gamma$ , A \& B, $\Phi$  $\vdash$  C}
		\end{prooftree}
		\quad
		\begin{prooftree}
		\hypo{$\Gamma$ , B, $\Phi$  $\vdash$  C}
		\infer1{$\Gamma$ , A \& B, $\Phi$  $\vdash$  C}
		\end{prooftree}
		\quad
		\begin{prooftree}
		\hypo{$\Gamma$  $\vdash$  A}
		\hypo{$\Gamma$  $\vdash$  B}
		\infer2{$\Gamma$  $\vdash$  A \& B}
		\end{prooftree}
		\quad
		\begin{prooftree}
		\hypo{B $\vdash$  $\Delta$ }
		\hypo{A $\vdash$  $\Delta$ }
		\infer2{B $\oplus$  A $\vdash$  $\Delta$ }
		\end{prooftree}
		\quad
		\begin{prooftree}
		\hypo{C $\vdash$  $\Psi$ , A, $\Delta$ }
		\infer1{C $\vdash$  $\Psi$ , B $\oplus$  A, $\Delta$ }
		\end{prooftree}
		\quad
		\begin{prooftree}
		\hypo{C $\vdash$  $\Psi$ , B, $\Delta$ }
		\infer1{C $\vdash$  $\Psi$ , B $\oplus$  A, $\Delta$ }
		\end{prooftree}
		\]
		
		
		\[
		\begin{prooftree}
		\hypo{$\Gamma$ , $\Phi$  $\vdash$  C, A}
		\hypo{$\Gamma$ , $\Phi$  $\vdash$  C, B}
		\infer2{$\Gamma$ , $\neg$  A $\oplus$  $\neg$  B, $\Phi$  $\vdash$  C}
		\end{prooftree}
		\quad
		\begin{prooftree}
		\hypo{A, $\Gamma$  $\vdash$  }
		\infer1{$\Gamma$  $\vdash$  $\neg$  A $\oplus$  $\neg$  B}
		\end{prooftree}
		\quad
		\begin{prooftree}
		\hypo{B, $\Gamma$  $\vdash$  }
		\infer1{$\Gamma$  $\vdash$  $\neg$  A $\oplus$  $\neg$  B}
		\end{prooftree}
		\quad
		\begin{prooftree}
		\hypo{$\vdash$  $\Delta$ , A}
		\infer1{$\neg$  B \& $\neg$  A $\vdash$  $\Delta$ }
		\end{prooftree}
		\quad
		\begin{prooftree}
		\hypo{$\vdash$  $\Delta$ , B}
		\infer1{$\neg$  B \& $\neg$  A $\vdash$  $\Delta$ }
		\end{prooftree}
		\quad
		\begin{prooftree}
		\hypo{ B, C $\vdash$  $\Psi$ , $\Delta$ }
		\hypo{ A, C $\vdash$  $\Psi$ , $\Delta$ }
		\infer2{ C $\vdash$  $\Psi$ , $\neg$  B \& $\neg$  A, $\Delta$ }
		\end{prooftree}
		\]
		
		\[
		\begin{prooftree}
		\hypo{$\Gamma$ , $\Phi$  $\vdash$  C, A}
		\infer1{$\Gamma$ , $\neg$  A \& $\neg$  B, $\Phi$  $\vdash$  C}
		\end{prooftree}
		\quad
		\begin{prooftree}
		\hypo{$\Gamma$ , $\Phi$  $\vdash$  C, B}
		\infer1{$\Gamma$ , $\neg$  A \& $\neg$  B, $\Phi$  $\vdash$  C}
		\end{prooftree}
		\quad
		\begin{prooftree}
		\hypo{A, $\Gamma$  $\vdash$  }
		\hypo{B, $\Gamma$  $\vdash$  }
		\infer2{$\Gamma$  $\vdash$  $\neg$  A \& $\neg$  B}
		\end{prooftree}
		\quad
		\begin{prooftree}
		\hypo{$\vdash$  $\Delta$ , B}
		\hypo{$\vdash$  $\Delta$ , A}
		\infer2{$\neg$  B $\oplus$  $\neg$  A $\vdash$  $\Delta$ }
		\end{prooftree}
		\quad
		\begin{prooftree}
		\hypo{A, C $\vdash$  $\Psi$ , $\Delta$ }
		\infer1{C $\vdash$  $\Psi$ , $\neg$  B $\oplus$  $\neg$  A, $\Delta$ }
		\end{prooftree}
		\quad
		\begin{prooftree}
		\hypo{B, C $\vdash$  $\Psi$ , $\Delta$ }
		\infer1{C $\vdash$  $\Psi$ , $\neg$  B $\oplus$  $\neg$  A, $\Delta$ }
		\end{prooftree}
		\]
		
	\end{center}

	\subsection{Multiplicatives}
	\begin{center}
		\[
		\begin{prooftree}
		\hypo{$\Gamma$ , A, B, $\Phi$  $\vdash$  C}
		\infer1{$\Gamma$ , A $\otimes$  B, $\Phi$  $\vdash$  C}
		\end{prooftree}
		\quad
		\begin{prooftree}
		\hypo{$\Gamma$  $\vdash$  A}
		\hypo{$\Phi$  $\vdash$  B}
		\infer2{$\Gamma$ , $\Phi$  $\vdash$  A $\otimes$  B}
		\end{prooftree}
		\quad
		\begin{prooftree}
		\hypo{B $\vdash$  $\Psi$ }
		\hypo{A $\vdash$  $\Delta$ }
		\infer2{B $\parr$  A $\vdash$  $\Psi$ , $\Delta$ }
		\end{prooftree}
		\quad
		\begin{prooftree}
		\hypo{C $\vdash$  $\Psi$ , B, A, $\Delta$ }
		\infer1{C $\vdash$  $\Psi$ , B $\parr$  A, $\Delta$ }
		\end{prooftree}
		\]
		
		\[
		\begin{prooftree}
		\hypo{$\Gamma$ , B, A, $\Phi$  $\vdash$  C}
		\infer1{$\Gamma$ , B $\otimes$  A, $\Phi$  $\vdash$  C}
		\end{prooftree}
		\quad
		\begin{prooftree}
		\hypo{$\Phi$  $\vdash$  B}
		\hypo{$\Gamma$  $\vdash$  A}
		\infer2{$\Phi$ , $\Gamma$  $\vdash$  B $\otimes$  A}
		\end{prooftree}
		\quad
		\begin{prooftree}
		\hypo{A $\vdash$  $\Delta$ }
		\hypo{B $\vdash$  $\Psi$ }
		\infer2{A $\parr$  B $\vdash$  $\Delta$ , $\Psi$ }
		\end{prooftree}
		\quad
		\begin{prooftree}
		\hypo{C $\vdash$  $\Psi$ , A, B, $\Delta$ }
		\infer1{C $\vdash$  $\Psi$ , A $\parr$  B, $\Delta$ }
		\end{prooftree}
		\]
		
		\[
		\begin{prooftree}
		\hypo{$\Gamma$  $\vdash$  A}
		\hypo{$\Phi$ , B, Θ $\vdash$  C}
		\infer2{$\Phi$ , $\Gamma$ , A $\to$  B, Θ $\vdash$  C}
		\end{prooftree}
		\quad
		\begin{prooftree}
		\hypo{A, $\Gamma$  $\vdash$  B}
		\infer1{$\Gamma$  $\vdash$  A $\to$  B}
		\end{prooftree}
		\quad
		\begin{prooftree}
		\hypo{B $\vdash$  $\Delta$ , A}
		\infer1{A $\nleftarrow$  B $\vdash$  $\Delta$ }
		\end{prooftree}
		\quad
		\begin{prooftree}
		\hypo{A $\vdash$  $\Delta$ }
		\hypo{C $\vdash$  Λ, B, $\Psi$ }
		\infer2{C $\vdash$  Λ, A $\nleftarrow$  B, $\Psi$ , $\Delta$ }
		\end{prooftree}
		\]
		
		\[
		\begin{prooftree}
		\hypo{$\Gamma$  $\vdash$  A}
		\hypo{$\Phi$ , B, Θ $\vdash$  C}
		\infer2{$\Phi$ , $\Gamma$ , B ← A, Θ $\vdash$  C}
		\end{prooftree}
		\quad
		\begin{prooftree}
		\hypo{$\Gamma$ , A $\vdash$  B}
		\infer1{$\Gamma$  $\vdash$  B ← A}
		\end{prooftree}
		\quad
		\begin{prooftree}
		\hypo{B $\vdash$  A, $\Psi$ }
		\infer1{A $\nrightarrow$  B $\vdash$  $\Psi$ }
		\end{prooftree}
		\quad
		\begin{prooftree}
		\hypo{A $\vdash$  $\Delta$ }
		\hypo{C $\vdash$  Λ, B, $\Psi$ }
		\infer2{C $\vdash$  Λ, B $\nrightarrow$  A, $\Psi$ , $\Delta$ }
		\end{prooftree}
		\]
	\end{center}
\end{center}

\part{Theorems}
	\begin{center}
		\[
		\begin{prooftree}
		\infer0{A $\vdash$  A}
		\infer1{1, A $\vdash$  A}
		\infer1{1 $\vdash$  A, $\neg$  A}
		\infer1{1 $\vdash$  A $\parr$  $\neg$  A}
		\end{prooftree}
		\quad
		\begin{prooftree}
		\infer0{A $\vdash$  A}
		\infer1{A, 1 $\vdash$  A}
		\infer1{A $\otimes$  1 $\vdash$  A}
		\end{prooftree}
		\quad
		\begin{prooftree}
		\infer0{A, 0 $\vdash$  A}
		\infer1{A $\otimes$  0 $\vdash$  A}
		\end{prooftree}
		\quad
		\begin{prooftree}
		\infer0{0 $\vdash$  A $\parr$  $\neg$  A}
		\end{prooftree}
		\]
		
		\[
		\begin{prooftree}
		\infer0{A $\vdash$  A}
		\infer1{A $\vdash$  A, $\bot$ }
		\infer1{A, $\neg$  A $\vdash$  $\bot$ }
		\infer1{A $\otimes$  $\neg$  A $\vdash$  $\bot$ }
		\end{prooftree}
		\quad
		\begin{prooftree}
		\infer0{A $\vdash$  A}
		\infer1{A $\vdash$  A, $\bot$ }
		\infer1{A $\vdash$  A $\parr$  $\bot$ }
		\end{prooftree}
		\quad
		\begin{prooftree}
		\infer0{A $\vdash$  A, $\top$ }
		\infer1{A $\vdash$  A $\parr$  $\top$ }
		\end{prooftree}
		\quad
		\begin{prooftree}
		\infer0{A $\otimes$  $\neg$  A $\vdash$  $\top$ }
		\end{prooftree}
		\]
		
		\[
		\begin{prooftree}
		\infer0{$\bot$  $\vdash$  }
		\infer1{$\bot$  $\vdash$  $\bot$ }
		\infer1{$\bot$ , $\neg$  $\bot$  $\vdash$  }
		\infer1{$\bot$  $\otimes$  $\neg$  $\bot$  $\vdash$  }
		\end{prooftree}
		\quad
		\begin{prooftree}
		\infer0{A $\vdash$  A}
		\infer1{A $\vdash$  A, $\bot$ }
		\infer1{A, $\neg$  $\bot$  $\vdash$  A}
		\infer1{A $\otimes$  $\neg$  $\bot$  $\vdash$  A}
		\end{prooftree}
		\quad
		\begin{prooftree}
		\infer0{ $\vdash$  1}
		\infer1{1 $\vdash$  1}
		\infer1{1, $\neg$  1 $\vdash$  }
		\infer1{1 $\otimes$  $\neg$  1 $\vdash$  }
		\end{prooftree}
		\quad
		\begin{prooftree}
		\infer0{A $\vdash$  A}
		\infer1{A, 1 $\vdash$  A}
		\infer1{1, $\neg$  A $\vdash$  $\neg$  A}
		\infer1{1 $\otimes$  $\neg$  A $\vdash$  $\neg$  A}
		\end{prooftree}
		\quad
		\begin{prooftree}
		\infer0{0 $\vdash$  $\Delta$ , A}
		\infer1{0, $\neg$  A $\vdash$  $\Delta$ }
		\infer1{0 $\otimes$  $\neg$  A $\vdash$  $\Delta$ }
		\end{prooftree}
		\quad
		\begin{prooftree}
		\infer0{A $\vdash$  $\Delta$ , $\top$ }
		\infer1{A, $\neg$  $\top$  $\vdash$  $\Delta$ }
		\infer1{A $\otimes$  $\neg$  $\top$  $\vdash$  $\Delta$ }
		\end{prooftree}
		\]
		
		\[
		\begin{prooftree}
		\infer0{A $\vdash$  A}
		\infer1{$\neg$ A $\vdash$  $\neg$ A}
		\infer1{$\neg$ A $\vdash$  $\neg$ A,  $\bot$ }
		\infer1{$\neg$ A $\vdash$  $\neg$ A $\parr$  $\bot$ }
		\end{prooftree}
		\quad
		\begin{prooftree}
		\infer0{A $\vdash$  A}
		\infer1{1, A $\vdash$  A}
		\infer1{A $\vdash$  A, $\neg$ 1}
		\infer1{A $\vdash$  A $\parr$  $\neg$ 1}
		\end{prooftree}
		\quad
		\begin{prooftree}
		\infer0{0, $\Gamma$  $\vdash$  A}
		\infer1{$\Gamma$  $\vdash$  A, $\neg$ 0}
		\infer1{$\Gamma$  $\vdash$  A $\parr$  $\neg$ 0}
		\end{prooftree}
		\quad
		\begin{prooftree}
		\infer0{A, $\Gamma$  $\vdash$  $\top$ }
		\infer1{ $\Gamma$ $\vdash$  $\top$ , $\neg$ A}
		\infer1{$\Gamma$  $\vdash$  $\top$  $\parr$  $\neg$ A}
		\end{prooftree}
		\]
		
		\[
		\begin{prooftree}
		\infer0{$\bot$  $\vdash$  }
		\infer1{$\bot$  $\vdash$  $\bot$ }
		\infer1{$\bot$ , $\neg$  $\bot$  $\vdash$  }
		\infer1{$\vdash$  $\bot$  $\parr$  $\neg$  $\bot$ }
		\end{prooftree}
		\quad
		\begin{prooftree}
		\infer0{ $\vdash$  1}
		\infer1{1 $\vdash$  1}
		\infer1{1, $\neg$  1 $\vdash$  }
		\infer1{1 $\otimes$  $\neg$  1 $\vdash$  }
		\end{prooftree}
		\quad
		\begin{prooftree}
		\infer0{0 $\vdash$  $\Delta$ , A}
		\infer1{0, $\neg$  A $\vdash$  $\Delta$ }
		\infer1{0 $\otimes$  $\neg$  A $\vdash$  $\Delta$ }
		\end{prooftree}
		\quad
		\begin{prooftree}
		\infer0{A $\vdash$  $\Delta$ , $\top$ }
		\infer1{A, $\neg$  $\top$  $\vdash$  $\Delta$ }
		\infer1{A $\otimes$  $\neg$  $\top$  $\vdash$  $\Delta$ }
		\end{prooftree}
		\]
		
		\[
		\begin{prooftree}
		\infer0{A $\vdash$  A}
		\infer1{A $\vdash$  A $\oplus$  $\neg$ A}
		\end{prooftree}
		\quad
		\begin{prooftree}
		\infer0{A $\vdash$  A}
		\infer1{ $\neg$  A \& A $\vdash$  A}
		\end{prooftree}
		\]
		
		\[
		\begin{prooftree}
		\infer0{A $\vdash$  A}
		\infer1{$\neg$  A, A $\vdash$ }
		\infer1{ $\neg$  A $\otimes$  A $\vdash$ }
		\end{prooftree}
		\quad
		\begin{prooftree}
		\infer0{A $\vdash$  A}
		\infer1{$\neg$  A, A $\vdash$ }
		\infer1{A $\otimes$  $\neg$  A $\vdash$ }
		\infer1{$\vdash$ $\neg$  (A $\otimes$  $\neg$  A)}
		\end{prooftree}
		\]

		\[
		\begin{prooftree}
		\infer0{A $\vdash$  A}
		\infer1{$\vdash$  A, $\neg$  A}
		\infer1{$\vdash$  A $\parr$  $\neg$  A}
		\end{prooftree}
		\quad
		\begin{prooftree}
		\infer0{A $\vdash$  A}
		\infer1{$\vdash$  A, $\neg$  A}
		\infer1{$\vdash$  A $\parr$  $\neg$  A}
		\infer1{$\neg$  (A $\parr$  $\neg$  A) $\vdash$ }
		\end{prooftree}
		\]

		\[
		\begin{prooftree}
		\infer0{$\bot$  $\vdash$ }
		\infer1{A \& $\bot$  $\vdash$ }
		\end{prooftree}
		\quad
		\begin{prooftree}
		\infer0{0 $\vdash$ }
		\infer1{A \& 0 $\vdash$ }
		\end{prooftree}
		\quad
		\begin{prooftree}
		\infer0{A $\vdash$  A}
		\infer0{A $\vdash$  $\top$ }
		\infer2{A $\vdash$  A \& $\top$ }
		\end{prooftree}
		\]
		
		\[
		\begin{prooftree}
		\infer0{$\vdash$  1}
		\infer1{$\vdash$  A $\oplus$  1}
		\end{prooftree}
		\quad
		\begin{prooftree}
		\infer0{$\vdash$  $\top$ }
		\infer1{$\vdash$  A $\oplus$  $\top$ }
		\end{prooftree}
		\quad
		\begin{prooftree}
		\infer0{A $\vdash$  A}
		\infer0{0 $\vdash$  A}
		\infer2{A $\oplus$  0 $\vdash$  A}
		\end{prooftree}
		\]
	\end{center}

\chapter{Circuit Calc}
}
	
	\maketitle
	
	\section{General Semantic Space: The Complex Projective Line ($\mathbb{CP}^1$)}
	
	Based on our discussions, the most natural and encompassing general semantic space for a non-bivalent formality based on complex numbers is the \textbf{Complex Projective Line} ($\mathbb{CP}^1$).
	
	\subsection{Definition}
	
	The Complex Projective Line, $\mathbb{CP}^1$, can be formally defined as the set of all lines through the origin in $\mathbb{C}^2$. Equivalently, it can be viewed as the set $\mathbb{C} \cup \{\infty\}$, which is the complex plane $\mathbb{C}$ augmented by a single point at infinity. Geometrically, $\mathbb{CP}^1$ is homeomorphic to a sphere, known as the \textbf{Riemann sphere}.
	
	More formally, $\mathbb{CP}^1$ is the set of equivalence classes of pairs of complex numbers $[z_0, z_1] \in \mathbb{C}^2 \setminus \{(0,0)\}$, where $[z_0, z_1] \sim [w_0, w_1]$ if there exists a non-zero complex number $\lambda \in \mathbb{C} \setminus \{0\}$ such that $z_0 = \lambda w_0$ and $z_1 = \lambda w_1$. Each equivalence class $[z_0, z_1]$ represents a ray through the origin in $\mathbb{C}^2$. The points in $\mathbb{CP}^1$ can be represented by a single complex coordinate $z = z_1/z_0$ if $z_0 \neq 0$, and the equivalence class $[0, 1]$ (where $z_0 = 0$) corresponds to the point at infinity, denoted $\infty$.
	
	\subsection{Why $\mathbb{CP}^1$ as the General Semantic Space?}
	
	\begin{enumerate}
		\item \textbf{Includes the Complex Plane:} $\mathbb{CP}^1$ naturally includes the complex plane $\mathbb{C}$, allowing for semantic values to be represented by complex numbers with both magnitude and phase.
		
		\item \textbf{Includes a Point at Infinity:} The inclusion of a single point at infinity provides a natural "completion" of the complex plane. This point can have significant semantic interpretation in a non-bivalent logic (see below).
		
		\item \textbf{Rich Geometric Structure:} $\mathbb{CP}^1$ possesses a rich geometric structure (e.g., it's a Riemann surface) that can be leveraged to define logical operations.
		
		\item \textbf{Automorphisms are M\"{o}bius Transformations:} The biholomorphic automorphisms of $\mathbb{CP}^1$ are precisely the M\"{o}bius transformations, a powerful class of functions that can be used to define structure-preserving semantic transformations.
	\end{enumerate}
	
	\subsection{The Role of the Point at Infinity ($\infty$)}
	
	The point at infinity in $\mathbb{CP}^1$ can be interpreted semantically in various ways within a non-bivalent framework:
	
	\begin{itemize}
		\item \textbf{Absolute Truth or Falsity:} It could represent a state of absolute or ultimate truth or falsity, beyond the bounded values in the complex plane.
		
		\item \textbf{Maximal Certainty or Uncertainty:} It could represent a state of maximal semantic certainty or uncertainty, depending on the chosen interpretation.
		
		\item \textbf{Undefined or Paradoxical Status:} In some non-classical logics, a point at infinity can be associated with undefined or paradoxical propositions.
	\end{itemize}
	Its specific semantic role will depend on the chosen interpretation of the overall semantic space and the defined logical operations.
	
	\subsection{Connection to the $[-i, i]$ Model}
	
	The $[-i, i]$ interval is a specific subset of $\mathbb{CP}^1$. It's a line segment on the imaginary axis. Viewing $\mathbb{CP}^1$ as the general space allows us to understand $[-i, i]$ as a particular "slice" or "cut" of this larger space, corresponding to a specific range of magnitudes and phases (magnitude in $[0, 1]$ and phase of $\pm\pi/2$ or undefined at 0).
	
	\subsection{Higher Dimensions (Quaternions, Octonions)}
	
	The idea of extending to higher-dimensional algebras like quaternions ($\mathbb{H}$) or octonions ($\mathbb{O}$) is a fascinating direction for future exploration. These algebras have multiple imaginary units ($i, j, k$ for quaternions, and seven for octonions) and exhibit non-commutativity (for quaternions and octonions) and non-associativity (for octonions).
	
	Building a semantic space based on these would involve considering points in $\mathbb{R}^4$ (for quaternions) or $\mathbb{R}^8$ (for octonions) with the algebraic structure of these numbers. This would lead to non-bivalent logics with:
	
	\begin{itemize}
		\item \textbf{Multiple Orthogonal "Imaginary" Dimensions:} More than just the single imaginary axis in the complex plane.
		
		\item \textbf{Non-Commutative and Non-Associative Operations:} Reflecting the properties of the underlying algebra.
		
		\item \textbf{Richer Forms of Phase and Angle Ambiguity:} Related to rotations in higher dimensions.
	\end{itemize}
	This is a significant generalization and would lead to fundamentally different types of non-bivalent logics compared to those based on complex numbers. While highly promising, it's a step beyond formally defining the framework based on $\mathbb{CP}^1$.
	
	\section{Views of $\mathbb{CP}^1$: Algebraic, Geometric, Categorical, and Topological}
	
	The Complex Projective Line ($\mathbb{CP}^1$) is a rich mathematical object that can be viewed from several interconnected perspectives, each providing valuable insights for its use as a semantic space in non-bivalent logic.
	
	\subsection{Algebraic View}
	
	The algebraic view focuses on the algebraic structures inherent in or definable on $\mathbb{CP}^1$.
	
	\begin{enumerate}
		\item \textbf{Extended Field Structure:} $\mathbb{CP}^1 = \mathbb{C} \cup \{\infty\}$ can be seen as an extension of the complex field $\mathbb{C}$, where arithmetic operations (addition, multiplication, subtraction, division) are extended to include the point at infinity with specific rules (e.g., $z + \infty = \infty$ for $z \neq \infty$, $z \times \infty = \infty$ for $z \neq 0, \infty$, $1/0 = \infty$, $1/\infty = 0$). While not a standard field (as $\infty$ has no additive inverse), this structure provides a basic algebraic framework for defining logical operations.
		
		\item \textbf{M\"{o}bius Transformation Group:} The set of all M\"{o}bius transformations $f(z) = \frac{az+b}{cz+d}$ with $ad-bc \neq 0$ (where $a, b, c, d \in \mathbb{C}$) forms a group under composition. These transformations are the automorphisms of $\mathbb{CP}^1$ and preserve its complex structure and geometric properties (mapping circles to circles). This group structure is fundamental for understanding symmetries and equivalences within the semantic space, which can correspond to logical equivalences or transformations.
		
		\item \textbf{Projective Space:} $\mathbb{CP}^1$ is the 1-dimensional complex projective space, $\mathbb{P}^1(\mathbb{C})$. This perspective connects $\mathbb{CP}^1$ to the broader field of projective geometry, where points are defined by homogeneous coordinates $[z_0, z_1]$. This algebraic definition highlights the underlying linear algebraic structure from which $\mathbb{CP}^1$ arises (rays in $\mathbb{C}^2$).
	\end{enumerate}
	
	\subsection{Geometric View}
	
	The geometric view focuses on $\mathbb{CP}^1$ as a geometric space.
	
	\begin{enumerate}
		\item \textbf{Riemann Sphere:} $\mathbb{CP}^1$ is geometrically equivalent (homeomorphic) to the unit sphere in $\mathbb{R}^3$, known as the Riemann sphere. The stereographic projection provides a homeomorphism between the complex plane (augmented with $\infty$) and the sphere. This visualization is crucial for understanding the topology and geometry of $\mathbb{CP}^1$, allowing us to think about complex numbers and $\infty$ as points on a sphere.
		
		\item \textbf{Conformal Mapping:} M\"{o}bius transformations are conformal maps, meaning they preserve angles. This geometric property is significant; it implies that certain "angles" or "orientations" in the semantic space are preserved by these fundamental transformations, which could have logical interpretations.
		
		\item \textbf{Circles and Lines:} A key geometric property is that M\"{o}bius transformations map circles and lines in the complex plane to circles and lines in the complex plane (where lines are considered circles through infinity). This provides a specific type of geometric structure that is preserved by the automorphisms.
		
		\item \textbf{Bloch Sphere Connection:} The Riemann sphere is the geometric representation of the Bloch sphere, the space of pure states of a qubit. This deep connection links the semantic space directly to quantum information theory.
	\end{enumerate}
	
	\subsection{Categorical View}
	
	The categorical view interprets $\mathbb{CP}^1$ and logical systems based on it within the framework of categories.
	
	\begin{enumerate}
		\item \textbf{Object in Categories:} $\mathbb{CP}^1$ is an object in various mathematical categories, including the category of Topological Spaces (\textbf{Top}), the category of Riemann Surfaces, and categories within algebraic geometry.
		
		\item \textbf{Morphisms:} The relevant morphisms depend on the category. In \textbf{Top}, continuous maps are morphisms; homeomorphisms are isomorphisms. In the category of Riemann Surfaces, holomorphic maps are morphisms; biholomorphic maps (M\"{o}bius transformations) are isomorphisms.
		
		\item \textbf{Categorical Semantics of Logics:} Logical systems whose semantic values are points in $\mathbb{CP}^1$ can be interpreted in categories. The structure of the logic (connectives, inference rules) is mirrored by the structure of the category. Our observation of the "diamond" structure of logics on intervals as a commutation diagram in a category where objects are logics and morphisms relate their fixed-point sets is an example of this.
		
		\item \textbf{Connection to Hilb:} $\mathbb{CP}^1$ is the space of pure states of $\mathbb{C}^2$, which is an object in the category of Hilbert spaces (\textbf{Hilb}). This connection is vital for understanding the potential of $\mathbb{CP}^1$ as a semantic space for quantum logic or logics related to quantum computation.
	\end{enumerate}
	
	\subsection{Topological View}
	
	The topological view focuses on the properties of $\mathbb{CP}^1$ as a topological space.
	
	\begin{enumerate}
		\item \textbf{Compactness:} $\mathbb{CP}^1$ is a compact space. This means that every open cover of $\mathbb{CP}^1$ has a finite subcover. This topological property is important in analysis and can have implications for the behavior of functions and limits on $\mathbb{CP}^1$.
		
		\item \textbf{Connectedness:} $\mathbb{CP}^1$ is a connected space, meaning it cannot be divided into two disjoint non-empty open sets.
		
		\item \textbf{Homeomorphism to $S^2$:} As mentioned, $\mathbb{CP}^1$ is homeomorphic to the 2-sphere ($S^2$). This means they share all topological properties. This homeomorphism is an isomorphism in the category of Topological Spaces.
	\end{enumerate}
	
	These four views -- algebraic, geometric, categorical, and topological -- are deeply interconnected and provide a comprehensive understanding of $\mathbb{CP}^1$ as a semantic space. The algebraic structure defines the operations, the geometric structure provides visualization and intuition, the topological structure gives properties like compactness and connectedness, and the categorical view provides a formal framework for relating $\mathbb{CP}^1$ to other mathematical structures and for building categories of logics on this space.
	
	\section{Potential Negation Operators on $\mathbb{CP}^1$}
	
	On $\mathbb{CP}^1$, logical negation operators should be functions that map $\mathbb{CP}^1$ to $\mathbb{CP}^1$ and capture some notion of opposition or inversion of semantic status. We analyze three involutive (applying the operation twice returns the original value) candidates for negation operators on $\mathbb{CP}^1$.
	
	\subsection{Candidate Negation 1: $\neg_1(z) = 1 - z$}
	
	This operator is a \textbf{M\"{o}bius transformation}. It is well-defined and an involution on $\mathbb{CP}^1$.
	
	\begin{itemize}
		\item \textbf{Behavior at Key Points (0, 1, $\infty$):} $\neg_1(0) = 1$, $\neg_1(1) = 0$, $\neg_1(\infty) = \infty$. It swaps 0 and 1 and fixes $\infty$.
		
		\item \textbf{Fixed Points on $\mathbb{CP}^1$:} The only fixed point is $1/2$.
		
		\item \textbf{Semantic Interpretation:} Swaps 0 and 1 (like classical negation on these points), fixes $\infty$ (self-opposite ultimate state?), and fixes $1/2$ (a point of self-negation).
	\end{itemize}
	
	\subsection{Candidate Negation 2: $\neg_2(z) = \bar{z}$ (Complex Conjugation)}
	
	This operator is the standard complex conjugation operation.
	
	\begin{itemize}
		\item \textbf{Well-defined on $\mathbb{CP}^1$?} Yes. Complex conjugation is well-defined for all complex numbers. For the point at infinity, $\bar{\infty} = \infty$.
		
		\item \textbf{Is it an Involution?} Yes.
		$\neg_2(\neg_2(z)) = \neg_2(\bar{z}) = \bar{\bar{z}} = z$.
		Applying $\neg_2$ twice returns the original value $z$ for all $z \in \mathbb{CP}^1$.
		
		\item \textbf{Behavior at Key Points (0, 1, $\infty$):}
		\begin{itemize}
			\item $\neg_2(0) = \bar{0} = 0$. This operator fixes 0.
			\item $\neg_2(1) = \bar{1} = 1$. This operator fixes 1.
			\item $\neg_2(\infty) = \bar{\infty} = \infty$. This operator fixes the point at infinity.
		\end{itemize}
		
		\item \textbf{Fixed Points on $\mathbb{CP}^1$:} Fixed points are values $z$ such that $\neg_2(z) = z$.
		$\bar{z} = z$.
		This equation holds for all real numbers ($z = x + 0i$, where $\bar{z} = x - 0i = x$).
		The point at infinity is also a fixed point.
		The fixed points of $\neg_2$ are all \textbf{real numbers} and $\infty$.
		
		\item \textbf{Semantic Interpretation:}
		\begin{itemize}
			\item This negation fixes all real numbers. If the real axis represents a spectrum of truth values (like in fuzzy logic where $[0, 1]$ is used), this negation leaves those truth values unchanged. This is a very different behavior from classical negation.
			\item It fixes both 0 and 1, which could represent "False" and "True". This means $\neg_2(0) = 0$ and $\neg_2(1) = 1$, which is \textit{not} like classical negation on these points.
			\item It fixes $\infty$, similar to $\neg_1$.
			\item Complex conjugation reflects a point across the real axis in the complex plane. Semantically, this might correspond to an operation that preserves the "real" or "truth" component of a semantic value while inverting or reflecting the "imaginary" or "certainty/phase" component.
		\end{itemize}
	\end{itemize}
	The negation $\neg_2(z) = \bar{z}$ behaves like the identity on the real axis and fixes infinity. Its semantic interpretation could relate to a form of negation that primarily affects the non-real aspects of a semantic value.
	
	\subsection{Candidate Negation 3: $\neg_3(z) = 1 - \bar{z}$}
	
	This operator combines complex conjugation and subtraction from 1.
	
	\begin{itemize}
		\item \textbf{Well-defined on $\mathbb{CP}^1$?} Yes. It involves standard complex operations and conjugation.
		\begin{itemize}
			\item For $z \in \mathbb{C}$, $\neg_3(z) = 1 - \bar{z}$ is a standard operation.
			\item For $z = \infty$, $\bar{z} = \infty$. $1 - \infty = \infty$. So, $\neg_3(\infty) = \infty$.
		\end{itemize}
		
		\item \textbf{Is it an Involution?} Yes.
		$\neg_3(\neg_3(z)) = \neg_3(1 - \bar{z}) = 1 - \overline{(1 - \bar{z})} = 1 - (\bar{1} - \bar{\bar{z}}) = 1 - (1 - z) = 1 - 1 + z = z$.
		Applying $\neg_3$ twice returns the original value $z$ for all $z \in \mathbb{CP}^1$.
		
		\item \textbf{Behavior at Key Points (0, 1, $\infty$):}
		\begin{itemize}
			\item $\neg_3(0) = 1 - \bar{0} = 1 - 0 = 1$. This operator swaps 0 and 1.
			\item $\neg_3(1) = 1 - \bar{1} = 1 - 1 = 0$. This operator swaps 1 and 0.
			\item $\neg_3(\infty) = 1 - \bar{\infty} = \infty$. This operator fixes the point at infinity.
		\end{itemize}
		
		\item \textbf{Fixed Points on $\mathbb{CP}^1$:} Fixed points are values $z$ such that $\neg_3(z) = z$.
		$1 - \bar{z} = z$.
		Let $z = x + yi$. Then $\bar{z} = x - yi$.
		$1 - (x - yi) = x + yi$
		$1 - x + yi = x + yi$
		$1 - x = x$
		$1 = 2x$
		$x = 1/2$.
		The imaginary part $yi$ cancels out. This equation holds for any value of $y$.
		The fixed points of $\neg_3$ are all complex numbers of the form $1/2 + yi$, i.e., the \textbf{vertical line in the complex plane at $x = 1/2$}. The point at infinity is also a fixed point.
		
		\item \textbf{Semantic Interpretation:}
		\begin{itemize}
			\item Like $\neg_1$, this negation swaps the semantic values associated with 0 and 1, behaving like classical negation on these two points.
			\item It fixes $\infty$, similar to $\neg_1$ and $\neg_2$.
			\item The fixed points form a vertical line at $x = 1/2$. This suggests that all semantic values with a "real" component of $1/2$ are their own negation under $\neg_3$. This could represent a different kind of "neutrality" or "self-opposition" compared to the single point fixed by $\neg_1$.
		\end{itemize}
	\end{itemize}
	The negation $\neg_3(z) = 1 - \bar{z}$ is another involutive negation on $\mathbb{CP}^1$ that swaps 0 and 1 and fixes infinity. Its unique feature is fixing an entire vertical line in the complex plane.
	
	\subsection{The Potential for Multiple Negations on $\mathbb{CP}^1$}
	
	As you've intuited, the move to $\mathbb{CP}^1$ and the non-classical context open the possibility for multiple distinct negation operators, going beyond the unique negation of Boolean logic. We've analyzed three involutive candidates on $\mathbb{CP}^1$.
	
	Your intuition about different types of negations (linear, paracomplete, paraconsistent) is highly relevant here. These different mathematical negations on $\mathbb{CP}^1$ could potentially serve as the basis for logical negations with different properties:
	
	\begin{itemize}
		\item \textbf{Linear-like Negation (Involutive):} Operations that are involutive ($\neg\neg p = p$) are often associated with linear logic or logics with a strong sense of duality. All three candidates above are involutive.
		
		\item \textbf{Paracomplete-like Negation (Non-involutive):} Paracomplete logics often have negations where $\neg\neg p$ is weaker than $p$. This would require a negation operator on $\mathbb{CP}^1$ that is \textit{not} an involution. None of our current candidates are non-involutive. We would need to define a different type of operation.
		
		\item \textbf{Paraconsistent-like Negation (Non-involutive):} Paraconsistent logics often have negations where $\neg\neg p$ is stronger than $p$. This also requires a non-involutive negation operator on $\mathbb{CP}^1$.
	\end{itemize}
	The relationship between these different types of negations (e.g., one transforming into another) would depend on the specific definitions and the structure of the logic.
	
	Formally defining the general semantic space as $\mathbb{CP}^1$ is a crucial step. The next is to explore logical operations directly on this space.
	
	\section{Defining Conjunction and Disjunction on $\mathbb{CP}^1$}
	
	To analyze the Law of Non-Contradiction ($\neg z \wedge z = z$) and the Law of Excluded Middle ($\neg z \vee z = z$) identities for the negation operators defined above, we need to define conjunction ($\wedge$) and disjunction ($\vee$) operations that operate on the entire Complex Projective Line ($\mathbb{CP}^1$).
	
	Defining binary operations on $\mathbb{CP}^1$ is more complex than on intervals like $[-i, i]$ or $[-1, 1]$ because:
	
	\begin{enumerate}
		\item \textbf{Lack of Total Order:} Complex numbers do not have a natural total order, making direct extensions of "minimum" or "maximum" based operations (like those used on the intervals) challenging.
		
		\item \textbf{The Point at Infinity:} Operations must be well-defined when one or both operands are the point at infinity.
	\end{enumerate}
	A plausible approach is to define operations that extend standard operations on the complex plane and handle the point at infinity appropriately. These operations should ideally possess properties like commutativity, associativity, and idempotency, although non-classical logics may relax some of these.
	
	We could explore operations based on:
	
	\begin{itemize}
		\item \textbf{Arithmetic Operations:} Extending addition, multiplication, etc., with rules for infinity. For example, $z_1 + z_2$ could be a candidate for disjunction, and $z_1 \times z_2$ for conjunction, with rules like $z + \infty = \infty$ (for $z \neq \infty$) and $z \times \infty = \infty$ (for $z \neq 0, \infty$). However, these operations don't typically satisfy idempotency ($z+z=z$ or $z \times z=z$ are not generally true).
		
		\item \textbf{Geometric Operations:} Operations based on geometric transformations or relationships on the Riemann sphere.
		
		\item \textbf{Generalized Min/Max:} Attempting to define a form of minimum or maximum that works for complex numbers, perhaps based on magnitude, phase, or a lexicographical order.
	\end{itemize}
	Let's consider defining conjunction and disjunction on $\mathbb{CP}^1$ by extending standard complex multiplication and addition, with specific rules for the point at infinity. While these might not be idempotent everywhere, they are well-defined on $\mathbb{CP}^1$ and can serve as a starting point for analysis.
	
	\subsection{Candidate Conjunction: $z_1 \wedge z_2 = z_1 \times z_2$}
	
	Let's explore complex multiplication as a candidate for conjunction on $\mathbb{CP}^1$.
	
	\begin{itemize}
		\item \textbf{Well-defined on $\mathbb{CP}^1$?} Yes, with standard rules for infinity:
		\begin{itemize}
			\item For $z_1, z_2 \in \mathbb{C}$, $z_1 \times z_2$ is standard complex multiplication.
			\item For $z_1 \in \mathbb{C}, z_1 \neq 0$ and $z_2 = \infty$: $z_1 \times \infty = \infty$.
			\item For $z_1 = 0$ and $z_2 = \infty$: $0 \times \infty$ is typically considered an indeterminate form. For a logical conjunction, we might need to assign a specific semantic value here, perhaps 0 (if conjunction with False/0 is always False) or $\infty$ (if conjunction with the extreme value $\infty$ is always $\infty$ unless the other value is 0). Let's tentatively define $0 \times \infty = 0$ for conjunction.
			\item For $z_1 = \infty$ and $z_2 = \infty$: $\infty \times \infty = \infty$.
		\end{itemize}
		
		\item \textbf{Properties:}
		\begin{itemize}
			\item \textbf{Commutativity:} Yes, complex multiplication is commutative ($z_1 \times z_2 = z_2 \times z_1$).
			\item \textbf{Associativity:} Yes, complex multiplication is associative ($(z_1 \times z_2) \times z_3 = z_1 \times (z_2 \times z_3)$).
			\item \textbf{Idempotency:} No, $z \times z = z$ only holds for $z = 0$ and $z = 1$. This operation is not generally idempotent on $\mathbb{CP}^1$.
		\end{itemize}
	\end{itemize}
	
	\subsection{Candidate Disjunction: $z_1 \vee z_2 = z_1 + z_2$}
	
	Let's explore complex addition as a candidate for disjunction on $\mathbb{CP}^1$.
	
	\begin{itemize}
		\item \textbf{Well-defined on $\mathbb{CP}^1$?} Yes, with standard rules for infinity:
		\begin{itemize}
			\item For $z_1, z_2 \in \mathbb{C}$, $z_1 + z_2$ is standard complex addition.
			\item For $z_1 \in \mathbb{C}$ and $z_2 = \infty$: $z_1 + \infty = \infty$.
			\item For $z_1 = \infty$ and $z_2 = \infty$: $\infty + \infty$ is typically considered an indeterminate form. For a logical disjunction, we might need to assign a specific semantic value, perhaps $\infty$ (if disjunction of extreme values is extreme). Let's tentatively define $\infty + \infty = \infty$ for disjunction.
		\end{itemize}
		
		\item \textbf{Properties:}
		\begin{itemize}
			\item \textbf{Commutativity:} Yes, complex addition is commutative ($z_1 + z_2 = z_2 + z_1$).
			\item \textbf{Associativity:} Yes, complex addition is associative ($(z_1 + z_2) + z_3 = z_1 + (z_2 + z_3)$).
			\item \textbf{Idempotency:} No, $z + z = z$ only holds for $z = 0$. This operation is not generally idempotent on $\mathbb{CP}^1$.
		\end{itemize}
	\end{itemize}
	These candidate operations ($\times$ for conjunction, $+$ for disjunction) are simple extensions of standard arithmetic. Their lack of general idempotency means they won't behave like the conjunctions and disjunctions we saw on the intervals $[-i, i]$ and $[-1, 1]$ in terms of fixed points for LNC/LEM. However, they provide a starting point for analyzing the LNC/LEM identities with our chosen negations on $\mathbb{CP}^1$.
	
	\section{Fixed Point Analysis of LNC and LEM on $\mathbb{CP}^1$}
	
	Let's analyze the fixed points of the Law of Non-Contradiction ($\neg z \wedge z = z$) and the Law of Excluded Middle ($\neg z \vee z = z$) identities for each negation operator, using $\wedge = \times$ and $\vee = +$.
	
	\subsection{For Negation $\neg_1(z) = 1 - z$:}
	
	\begin{itemize}
		\item \textbf{Law of Non-Contradiction ($(1 - z) \times z = z$):}
		\begin{itemize}
			\item For $z \in \mathbb{C}$: $z - z^2 = z$. This simplifies to $-z^2 = 0$, which means $z^2 = 0$. The only solution in $\mathbb{C}$ is $z = 0$.
			\item For $z = \infty$: $\neg_1(\infty) = \infty$. $\infty \wedge \infty = \infty \times \infty = \infty$. We need $\infty = \infty$. This holds.
		\end{itemize}
		\textbf{Fixed points for LNC with $\neg_1$ are $\{0, \infty\}$.}
		
		\item \textbf{Law of Excluded Middle ($(1 - z) + z = z$):}
		\begin{itemize}
			\item For $z \in \mathbb{C}$: $1 - z + z = z$. This simplifies to $1 = z$. The only solution in $\mathbb{C}$ is $z = 1$.
			\item For $z = \infty$: $\neg_1(\infty) = \infty$. $\infty \vee \infty = \infty + \infty = \infty$. We need $\infty = \infty$. This holds.
		\end{itemize}
		\textbf{Fixed points for LEM with $\neg_1$ are $\{1, \infty\}$.}
	\end{itemize}
	
	\subsection{For Negation $\neg_2(z) = \bar{z}$:}
	
	\begin{itemize}
		\item \textbf{Law of Non-Contradiction ($\bar{z} \times z = z$):}
		\begin{itemize}
			\item For $z \in \mathbb{C}$: $|z|^2 = z$. Let $z = re^{i\theta}$. Then $|z|^2 = r^2$. We need $r^2 = re^{i\theta}$.
			\begin{itemize}
				\item If $r = 0$, then $0 = 0$, so $z = 0$ is a solution.
				\item If $r \neq 0$, we need $r = e^{i\theta}$. The magnitude of $e^{i\theta}$ is always 1. So we need $r = 1$. Then $1 = e^{i\theta}$. This holds when $\theta = 2\pi k$ for integer $k$. This means $z$ must be a positive real number with magnitude 1, i.e., $z = 1$.
			\end{itemize}
			\item For $z = \infty$: $\neg_2(\infty) = \infty$. $\infty \wedge \infty = \infty \times \infty = \infty$. We need $\infty = \infty$. This holds.
		\end{itemize}
		\textbf{Fixed points for LNC with $\neg_2$ are $\{0, 1, \infty\}$.}
		
		\item \textbf{Law of Excluded Middle ($\bar{z} + z = z$):}
		\begin{itemize}
			\item For $z \in \mathbb{C}$: $2 \times \text{Re}(z) = z$. Let $z = x + yi$. $2x = x + yi$. This means $x = yi$. The only complex number where the real part equals the imaginary part multiplied by $i$ is $z = 0$ (since $0 = 0i$).
			\item For $z = \infty$: $\neg_2(\infty) = \infty$. $\infty \vee \infty = \infty + \infty = \infty$. We need $\infty = \infty$. This holds.
		\end{itemize}
		\textbf{Fixed points for LEM with $\neg_2$ are $\{0, \infty\}$.}
	\end{itemize}
	
	\subsection{For Negation $\neg_3(z) = 1 - \bar{z}$:}
	
	\begin{itemize}
		\item \textbf{Law of Non-Contradiction ($(1 - \bar{z}) \times z = z$):}
		\begin{itemize}
			\item For $z \in \mathbb{C}$: $z - \bar{z}z = z$. This simplifies to $-\bar{z}z = 0$, which means $-|z|^2 = 0$. The only solution in $\mathbb{C}$ is $z = 0$.
			\item For $z = \infty$: $\neg_3(\infty) = \infty$. $\infty \wedge \infty = \infty \times \infty = \infty$. We need $\infty = \infty$. This holds.
		\end{itemize}
		\textbf{Fixed points for LNC with $\neg_3$ are $\{0, \infty\}$.}
		
		\item \textbf{Law of Excluded Middle ($(1 - \bar{z}) + z = z$):}
		\begin{itemize}
			\item For $z \in \mathbb{C}$: $1 - \bar{z} + z = z$. This simplifies to $1 - \bar{z} = 0$, so $\bar{z} = 1$. The only solution is $z = 1$.
			\item For $z = \infty$: $\neg_3(\infty) = \infty$. $\infty \vee \infty = \infty + \infty = \infty$. We need $\infty = \infty$. This holds.
		\end{itemize}
		\textbf{Fixed points for LEM with $\neg_3$ are $\{1, \infty\}$.}
	\end{itemize}
	
	\subsection{Summary of LNC/LEM Fixed Points on $\mathbb{CP}^1$ (with $\wedge=\times$, $\vee=+$)}
	
	Here is a summary of the fixed points for the Law of Non-Contradiction ($\neg z \wedge z = z$) and Law of Excluded Middle ($\neg z \vee z = z$) identities on $\mathbb{CP}^1$, using complex multiplication as conjunction and complex addition as disjunction:
	
	\begin{itemize}
		\item \textbf{For $\neg_1(z) = 1 - z$:}
		\begin{itemize}
			\item LNC Fixed Points: $\{0, \infty\}$
			\item LEM Fixed Points: $\{1, \infty\}$
		\end{itemize}
		
		\item \textbf{For $\neg_2(z) = \bar{z}$:}
		\begin{itemize}
			\item LNC Fixed Points: $\{0, 1, \infty\}$
			\item LEM Fixed Points: $\{0, \infty\}$
		\end{itemize}
		
		\item \textbf{For $\neg_3(z) = 1 - \bar{z}$:}
		\begin{itemize}
			\item LNC Fixed Points: $\{0, \infty\}$
			\item LEM Fixed Points: $\{1, \infty\}$
		\end{itemize}
	\end{itemize}
	This analysis shows that with these specific arithmetic operations for conjunction and disjunction, the fixed points for LNC and LEM are discrete points on $\mathbb{CP}^1$, primarily involving 0, 1, and $\infty$. The behavior is different from the intervals $[-i, i]$ and $[-1, 1]$ where we saw continuous intervals of fixed points. This difference likely stems from the lack of general idempotency of complex multiplication and addition.
	
	\section{Logic Defined by M\"{o}bius Group Generators}
	
	Let's define a new logic on $\mathbb{CP}^1$ directly using the fundamental generators of the M\"{o}bius group as its logical operations:
	
	\begin{itemize}
		\item \textbf{Semantic Space:} $\mathbb{CP}^1$
		
		\item \textbf{Negation ($\neg$):} Inversion, $\neg z = 1/z$
		
		\item \textbf{Conjunction ($\wedge$):} Dilation/Rotation (using multiplication), $z_1 \wedge z_2 = z_1 \times z_2$
		
		\item \textbf{Disjunction ($\vee$):} Translation (using addition), $z_1 \vee z_2 = z_1 + z_2$
	\end{itemize}
	We use the standard extensions of complex arithmetic to $\mathbb{CP}^1$ for these operations:
	
	\begin{itemize}
		\item $z + \infty = \infty$ for $z \neq \infty$, $\infty + \infty = \infty$.
		\item $z \times \infty = \infty$ for $z \neq 0, \infty$, $0 \times \infty = 0$, $\infty \times \infty = \infty$.
		\item $1/0 = \infty$, $1/\infty = 0$.
	\end{itemize}
	Let's analyze the fixed points for the Law of Non-Contradiction ($\neg z \wedge z = z$) and the Law of Excluded Middle ($\neg z \vee z = z$) identities for this logic.
	
	\subsection{Fixed Point Analysis (M\"{o}bius Generator Logic)}
	
	\begin{itemize}
		\item \textbf{Law of Non-Contradiction ($(1/z) \times z = z$):}
		\begin{itemize}
			\item For $z \in \mathbb{C}, z \neq 0$: $(1/z) \times z = 1$. So, $1 = z$. Solution: $z=1$.
			\item For $z = 0$: $\neg 0 = \infty$. $\infty \wedge 0 = \infty \times 0 = 0$. We need $0 = z$, so $0 = 0$. Solution: $z=0$.
			\item For $z = \infty$: $\neg \infty = 0$. $0 \wedge \infty = 0 \times \infty = 0$. We need $0 = z$, so $0 = \infty$. No solution.
		\end{itemize}
		\textbf{LNC Fixed Points: $\{0, 1\}$.}
		
		\item \textbf{Law of Excluded Middle ($(1/z) + z = z$):}
		\begin{itemize}
			\item For $z \in \mathbb{C}, z \neq 0$: $(1/z) + z = z$. This simplifies to $1/z = 0$. No solution in $\mathbb{C}$.
			\item For $z = 0$: $\neg 0 = \infty$. $\infty \vee 0 = \infty + 0 = \infty$. We need $\infty = z$, so $\infty = 0$. No solution.
			\item For $z = \infty$: $\neg \infty = 0$. $0 \vee \infty = 0 + \infty = \infty$. We need $\infty = z$, so $\infty = \infty$. Solution: $z=\infty$.
		\end{itemize}
		\textbf{LEM Fixed Points: $\{\infty\}$.}
	\end{itemize}
	In this logic, defined by the generators of the M\"{o}bius group, the Law of Non-Contradiction identity holds only at 0 and 1, while the Law of Excluded Middle identity holds only at infinity. This is a unique fixed point structure, highlighting the distinct logical behavior of operations directly tied to the fundamental transformations of $\mathbb{CP}^1$.
	
	\section{Interpretation of Fixed Points in a (Hyper)sequent Calculus}
	
	Interpreting these fixed points in a (hyper)sequent calculus involves relating them to concepts like designated values and the behavior of logical rules.
	
	\subsection{Designated Values}
	
	The fixed points of LNC and LEM identities can serve as \textbf{sets of designated values}. A formula $A$ is considered "valid" or "acceptable" if its semantic value $v(A)$ belongs to a set of designated values.
	
	\begin{itemize}
		\item For the logic defined by M\"{o}bius generators ($\neg=1/z$, $\wedge=\times$, $\vee=+$), the LNC fixed points are $\{0, 1\}$ and the LEM fixed points are $\{\infty\}$. This could imply that validity is tied to semantic values being either 0 or 1 (for consistency-related properties) or infinity (for completeness-related properties).
	\end{itemize}
	
	\subsection{Identity and Structural Rules}
	
	The fixed points indicate where $\neg A \wedge A$ and $\neg A \vee A$ are semantically equivalent to $A$. This could correspond to restricted rules allowing substitution or equivalence for formulas whose semantic values are in these fixed point sets.
	
	\subsection{Non-Commutativity and Sequent Structure}
	
	The potential for non-commutative binary operations necessitates using \textbf{lists of formulas} in sequents and potentially restricting structural rules like exchange.
	
	\subsection{Interpreting the Entailment ($\vdash$)}
	
	The semantic interpretation of $\Gamma \vdash \Delta$ on $\mathbb{CP}^1$ could relate to mapping to designated values or a binary relation on $\mathbb{CP}^1$.
	
	\section{Generalizing Sequents to Clines}
	
	The idea of generalizing linear sequents to \textbf{cline sequents} on $\mathbb{CP}^1$ leverages the geometric structure of the semantic space. A "cline sequent" is a cyclic arrangement of formulas, $(A_1, A_2, \dots, A_n)$.
	
	\subsection{Semantic Interpretation for a "Cline Sequent"}
	
	A \textbf{cline sequent} $(A_1, A_2, \dots, A_n)$ is \textbf{valid} in a logic on $\mathbb{CP}^1$ (with negation $\neg$, conjunction $\wedge$, and disjunction $\vee$) if, for every semantic assignment $v$ of formulas to points in $\mathbb{CP}^1$, the set of semantic values $\{v(A_1), v(A_2), \dots, v(A_n)\}$ satisfies the following condition:
	
	\begin{itemize}
		\item If $n \le 2$, the condition is trivially met (any one or two points lie on infinitely many clines).
		
		\item If $n \ge 3$, the points $v(A_1), v(A_2), \dots, v(A_n)$ lie on a single cline (circle or line) on $\mathbb{CP}^1$.
		
		\item Furthermore, this single cline \textbf{must pass through at least one point} in the set of \textbf{LNC fixed points} OR the set of \textbf{LEM fixed points} for the given logic and negation.
	\end{itemize}
	This interpretation connects the geometry of clines to the logical properties captured by the fixed points of LNC and LEM.
	
	\subsection{The "Junction" Operation for Cline Sequents}
	
	The cyclic structure of a cline sequent $(A_1, A_2, \dots, A_n)$ blurs the traditional distinction between antecedent and succedent, suggesting the potential for a single, unified \textbf{"junction" operation} that combines the semantic values in a cyclic manner.
	
	This "junction" operation would operate on a list of semantic values $(z_1, z_2, \dots, z_n)$ and produce a result that represents the semantic status of the entire cline sequent. This result could be:
	
	\begin{itemize}
		\item A single semantic value in $\mathbb{CP}^1$.
		
		\item A property of the cline passing through the points.
		
		\item A truth value (e.g., "valid" or "invalid") based on whether the resulting cline satisfies a condition (like passing through fixed points).
	\end{itemize}
	The "junction" operation for clines would be distinct from the binary conjunction and disjunction operations ($\wedge$, $\vee$) used to build complex formulas. The "junction" operates at the level of the sequent structure, while $\wedge$ and $\vee$ operate at the level of formula composition.
	
	The potential non-commutativity of the underlying binary operations would likely be reflected in the definition and properties of this "junction" operation, where the order of formulas in the cyclic list matters.
	
	The idea of "mapping various circles/lines to specific fixed/invariant clines" is a compelling semantic interpretation for validity in a logic with cline sequents. If the "junction" operation on the semantic values of a sequent results in a point (or a property) that corresponds to a "fixed" or "designated" cline, then the sequent is valid. This would be analogous to how in classical logic, a sequent is valid if the conjunction of the antecedent entails the disjunction of the succedent (mapping to a designated truth value).
	
	In this framework, the fixed points of LNC and LEM play a crucial role by defining the "target" clines (those passing through these fixed points) that correspond to valid sequents. The "junction" operation would determine whether the semantic values of a given sequent form a cline that hits
	
\end{document}}*}


\chapter{Contradictory Universe}












	
	\section{Introduction}
	
	This document formalizes a re-conceptualization of logical contradictions, contrasting the standard classical and paraconsistent approaches (viewed as 'consumptive') with a proposed framework where contradictions are 'productive' and 'generative' entities arising from an expansive semantic space.
	
	\section{The Classical View: Contradiction as Trivialization}
	
	In classical logic (e.g., LK), the semantic space is typically restricted to two truth values:
	\begin{itemize}
		\item $T$ (True)
		\item $F$ (False)
	\end{itemize}
	A contradiction is fundamentally a proposition that is both true and false, often represented as $P \land \neg P$.
	
	The defining property of contradiction in classical logic is the principle of \textbf{Ex Contradictione Quodlibet (ECQ)}:
	$$ (P \land \neg P) \vdash Q $$
	for any well-formed formula (WFF) $Q$.
	
	Structurally, the introduction or derivation of a contradiction leads to:
	\begin{itemize}
		\item \textbf{Trivialization:} The consequence relation ($\vdash$) loses its ability to distinguish between formulas. The set of consequences of a contradictory set of premises is the entire set of WFFs.
		\item \textbf{Collapse to Grammar:} The logic ceases to function as a filter identifying a subset of 'theorems' and effectively transforms into a formal grammar that generates the superset of all WFFs in the language.
	\end{itemize}
	
	Despite this internal trivialization, the system's reaction to contradiction is utilized in meta-logical techniques:
	\begin{itemize}
		\item \textbf{Proof by Contradiction (Reductio ad Absurdum):} Assume $\neg P$, derive a contradiction $(Q \land \neg Q)$, conclude $\neg \neg P$, and by Double Negation Elimination, $P$. This uses the system's collapse as a meta-level signal about the truth of the assumption.
		\item \textbf{Gödel's Incompleteness Theorems:} Demonstrate the existence of statements ($G$) in sufficiently powerful consistent systems such that $G \nvdash \bot$ and $\neg G \nvdash \bot$. These statements are undecidable within the system, hinting at a semantic status beyond $T$ (provable) and $F$ (refutable), analogous to an 'Unknown' value ($U$) arising from $\{T, F\}$.
	\end{itemize}
	
	\section{Extensions: Paracomplete and Paraconsistent Logics (Consumptive)}
	
	Recognizing limitations of bivalence leads to extending the semantic space.
	
	\subsection{Paracomplete Logics}
	Introducing an 'Unknown' value $U$, often based on the notion of undecidability or lack of proof/refutation, leads to semantic spaces like $\{F, T, U\}$. Logics interpreted over such spaces are \textbf{paracomplete} (rejecting $P \lor \neg P$).
	\begin{itemize}
		\item \textbf{Example:} Intuitionistic Logic (LJ) can be semantically related to this view.
		\item \textbf{Explosion:} Crucially, many paracomplete logics (including LJ) retain the ECQ principle. A contradiction still leads to complete trivialization of the consequence relation, deriving all WFFs.
	\end{itemize}
	
	\subsection{Paraconsistent Logics}
	Introducing a 'Both' value $B$ (for contradictory propositions), typically resulting in spaces like $\{F, T, B\}$ or $\{F, T, U, B\}$, leads to \textbf{paraconsistent} logics. These logics are designed to \textbf{tolerate} contradictions without complete trivialization.
	\begin{itemize}
		\item \textbf{Method:} They typically reject ECQ or the Law of Non-Contradiction ($\neg(P \land \neg P)$).
		\item \textbf{Contained Explosion:} While preventing the derivation of *all* WFFs, a contradiction might still lead to the derivation of a specific, non-trivial subset of WFFs (e.g., all negations, all contradictions). The "explosion" is contained but still derivational.
	\end{itemize}
	
	These approaches are characterized as \textbf{consumptive} because they view contradictions as problematic elements to be managed, contained, or processed to minimize their destructive impact on the logical system. The focus is on preventing or limiting the logical fallout.
	
	\section{Proposed Framework: Contradiction as Productive and Generative}
	
	This framework posits a radically different view, where contradictions are fundamental, \textbf{first-class citizens} of an expansive semantic space.
	
	\subsection{Semantic Space of Infinities}
	\begin{itemize}
		\item The semantic space is not finite (e.g., $\{T, F\}$ or $\{T, F, U, B\}$) but is conceived as arising from 'farthest infinities' and containing infinitely many distinct semantic values and states.
	\end{itemize}
	
	\subsection{Contradictions as Structural Entities}
	Contradictions are not single values or simple points of failure, but complex entities with internal structure:
	\begin{itemize}
		\item \textbf{Pairwise Relations:} Defined as associations between fundamental semantic values (e.g., $c_{TF}, c_{TU}, c_{FU}$ derived from $\{T, F, U\}$).
		\item \textbf{Multi-Value States:} States where multiple base values hold simultaneously (e.g., $C_{TFU}$ for $T, F, U$).
		\item \textbf{Compositional Outcomes:} New, more complex contradictory states are generated by composing lower-level contradictions.
	\end{itemize}
	
	\subsection{Compositional Algebra and Non-Trivial Associativity}
	A composition operation ($\circ$, analogous to multiplication) is defined on semantic entities (values and contradictions). This operation exhibits \textbf{non-trivial associativity}.
	\begin{itemize}
		\item Different bracketings and orderings of composing contradictions lead to \textbf{distinct semantic outcomes}.
		\item Example: For pairwise contradictions $c_{TF}, c_{TU}, c_{FU}$, under plausible rules, we found:
		\begin{align*} c_{TF} \circ (c_{TU} \circ c_{FU}) &= U \circ c_{TF}^2 \\ (c_{TF} \circ c_{TU}) \circ c_{FU} &= T \circ c_{FU}^2 \\ c_{TU} \circ (c_{FU} \circ c_{TF}) &= F \circ c_{TU}^2 \end{align*}
		where $U \circ c_{TF}^2$, $T \circ c_{FU}^2$, and $F \circ c_{TU}^2$ represent distinct semantic states.
		\item This demonstrates a \textbf{lack of classical confluence} for contradictions; different compositional paths do not converge to a single trivial state.
	\end{itemize}
	
	\subsection{Relationship to Combinatorics and Power Sets}
	\begin{itemize}
		\item The structure of contradictions (pairwise, triadic, etc.) is analogous to subsets in a power set ($\mathcal{P}(\{T, F, U\})$). Contradictions arise from specific combinations of fundamental semantic elements.
		\item The composition operation acts on these subset-like entities, generating new, complex semantic states.
		\item The "explosion" from contradiction is not a collapse to syntactic totality, but a \textbf{combinatorial unfolding} and \textbf{semantic generation} of a vast, structured landscape of distinct states.
	\end{itemize}
	
	This framework is characterized as \textbf{productive generative} because it views contradictions as fundamental forces that actively create structure, richness, and meaning within the semantic space through their inherent complexity and compositional interactions. Contradiction tolerance means embracing this generative power.
	
	\section{Conclusion}
	
	The classical view treats contradiction as a destructive force leading to trivialization, a phenomenon leveraged meta-logically (Proof by Contradiction) or mitigated in paraconsistent systems (contained explosion). This represents a 'consumptive' approach to managing problematic elements.
	
	In contrast, the proposed framework views contradictions as complex, structural entities inherent in an infinite semantic space. Their composition is non-trivially associative, leading to the generation of a diverse landscape of distinct semantic states. This 'productive generative' approach sees contradictions not as flaws to be contained, but as fundamental, creative forces that contribute significantly to the structure and richness of semantic reality.


\chapter{Graphene Silicene Unit}
	
	\title{Quantization of Usable Entropy and the Landauer Principle in Particle-Mediated Heat Dynamics: A Case Study of a Graphene-Silicene Heterostructure}
	\author{}
	\date{}
	
	\begin{abstract}
		This thesis investigates the interplay between the quantization of usable entropy, Landauer's principle, and particle-mediated heat dynamics, using a 3-layer graphene-silicene-graphene heterostructure as a case study. We analyze the system's maximum theoretical entropy based on electron spin states and estimate its usable entropy by considering its physical size in relation to holographic bounds. We then explore the implications of Landauer's principle for information erasure within this system, focusing on the role of quantized particles, such as photons and electrons, in the dissipation of heat.
	\end{abstract}
	
	\section{Introduction}
	The profound connection between entropy and information forms a cornerstone of modern physics. While thermodynamic entropy quantifies the disorder of a system, informational entropy measures the uncertainty or the number of bits required to describe its state. In quantum systems, both energy and information are inherently quantized, leading to a discrete spectrum of accessible states. Landauer's principle bridges these concepts by establishing a fundamental thermodynamic cost for the irreversible erasure of information. This thesis delves into these principles within the context of particle-mediated heat dynamics in a specific chemical system: a 3-layer graphene-silicene-graphene heterostructure. By examining this case study, we aim to elucidate how the quantization of usable entropy and the Landauer principle are manifested at the microscopic level through the interactions and exchanges of particles that govern the system's thermal behavior.
	
	\section{Theoretical Framework}
	
	\subsection{Entropy and Information}
	The entropy $S$ of a thermodynamic system in statistical mechanics is defined by Boltzmann's equation:
	\begin{equation}
		S = k_B \ln \Omega
	\end{equation}
	where $k_B$ is the Boltzmann constant and $\Omega$ is the number of accessible microstates. Shannon entropy $H$ in information theory quantifies the uncertainty associated with a random variable and is given by $H = - \sum_i p_i \log_2 p_i$. For a system with $\Omega$ equally probable states, $H = \log_2 \Omega$, which is related to thermodynamic entropy by $S = k_B \ln 2 \times H$. Usable entropy refers to the entropy that is practically accessible or relevant for encoding and processing information within a system.
	
	\subsection{Quantization in Quantum Systems}
	In quantum mechanics, physical quantities such as energy, angular momentum, and spin are quantized, meaning they can only take on discrete values. For instance, electrons in atoms occupy discrete energy levels, and their spin can only be up or down. This quantization leads to a finite and countable number of microstates for a given system with specific constraints. Information stored in such systems, such as the state of an electron's spin, is also quantized as bits.
	
	\subsection{Landauer's Principle: Derivation and Implications}
	Landauer's principle can be derived from the second law of thermodynamics by considering a system that stores one bit of information in a double-well potential. Erasing the bit involves forcing the system into a single known state, reducing its entropy. To comply with the second law, this entropy reduction must be compensated by an increase in the entropy of the environment through heat dissipation.
	
	\begin{theorem}[Landauer's Principle (Restated)]
		The minimum energy $E_{cost}$ required to erase one bit of information at a temperature $T$ is:
		\begin{equation}
			E_{cost} = k_B T \ln 2
		\end{equation}
	\end{theorem}
	
	\subsection{Particle-Mediated Heat Dynamics}
	Heat transfer at the microscopic level in chemical systems occurs primarily through the exchange of quantized particles:
	\begin{itemize}
		\item \textbf{Photons:} Energy is exchanged via electromagnetic radiation in the form of photons with energy $E = h f$. Thermal emission and absorption involve these quantized packets of energy.
		\item \textbf{Electrons:} Changes in the system's energy and information state often involve the absorption or emission of electrons, with energy levels and transitions being quantized. The kinetic energy of electrons is related to their momentum through the de Broglie relation $\lambda = h/p$.
		\end{itemize}
		
		\subsection{Holographic Principle and Entropy Bounds}
		The holographic principle suggests that the maximum entropy that can be contained within a region of space is proportional to the area of its boundary. The Bekenstein-Hawking entropy of a black hole, $S_{BH} = \frac{k_B A}{4 l_P^2}$, where $A$ is the area of the event horizon and $l_P$ is the Planck length, exemplifies this principle. A generalized holographic entropy bound for any system in a region of space with characteristic size $L$ and temporal extent $T$ can be considered, where entropy scales with the boundary area related to these dimensions.
		
		\section{Case Study: 3-Layer Graphene-Silicene-Graphene Heterostructure}
		
		\subsection{System Description}
		Our system consists of 48 carbon atoms and 24 silicon atoms in a 3-layer graphene-silicene-graphene structure, totaling 72 atoms and 624 electrons.
		
		\subsection{Maximum Theoretical Entropy}
		Assuming all 624 electron spins are independent, the maximum number of spin states is $\Omega_{max} = 2^{624}$, leading to a maximum theoretical entropy $S_{max} = 624 k_B \ln 2$.
		
		\subsection{Physical Size Estimation and Holographic Size}
		The estimated physical size has a characteristic length $L_{usable} \sim 1-2$ nanometers. The size $L_{max} \approx 10.7$ nanometers was derived from a holographic-like bound for $S_{max}$.
		
		\section{Analysis and Results}
		
		\subsection{Usable Entropy Estimation}
		\begin{proposition}
			The usable entropy of the graphene-silicene heterostructure is approximately $5.4$ bits, based on the scaling of entropy with the square of the characteristic size as suggested by holographic principles.
		\end{proposition}
		\begin{proof}
			Assuming entropy scales with area ($S \propto L^2$), we have:
			$$ \frac{S_{usable}}{S_{max}} \approx \left( \frac{L_{usable}}{L_{max}} \right)^2 $$
			Using $L_{usable} \approx 1$ nm (as a lower bound for simplicity) and $L_{max} \approx 10.7$ nm,
			$$ \frac{S_{usable}}{624 k_B \ln 2} \approx \left( \frac{1}{10.7} \right)^2 \approx 0.0087 $$
			$$ S_{usable} \approx 624 \times 0.0087 \times k_B \ln 2 \approx 5.43 \times k_B \ln 2 $$
			Since $k_B \ln 2$ is the entropy of one bit, the usable entropy is approximately $5.4$ bits.
		\end{proof}
		
		\subsection{Landauer Limit for Bit Erasure}
		Consider erasing one bit of information within this system at a temperature $T$ (e.g., room temperature, $T = 300$ K). The minimum energy cost is:
		$$ E_{cost} = k_B T \ln 2 \approx (1.3806 \times 10^{-23} \text{ J/K}) \times (300 \text{ K}) \times \ln 2 \approx 2.87 \times 10^{-21} \text{ J} $$
		
		\subsection{Particle-Mediated Heat Dissipation of Landauer Energy}
		The energy $E_{cost}$ must be dissipated as heat, mediated by particles.
		
		Case 1: Photons. The minimum energy of a photon carrying this heat is $h f_{min} \ge E_{cost}$.
		$$ f_{min} \ge \frac{E_{cost}}{h} \approx \frac{2.87 \times 10^{-21} \text{ J}}{6.626 \times 10^{-34} \text{ J s}} \approx 4.33 \times 10^{12} \text{ Hz} $$
		This corresponds to the far-infrared region of the electromagnetic spectrum.
		
		Case 2: Electron Kinetic Energy. If an electron carries away this energy as kinetic energy that thermalizes, then $p^2 / (2m_e) \ge E_{cost}$.
		$$ p \ge \sqrt{2 m_e E_{cost}} \approx \sqrt{2 \times (9.109 \times 10^{-31} \text{ kg}) \times (2.87 \times 10^{-21} \text{ J})} \approx 7.23 \times 10^{-26} \text{ kg m/s} $$
		The de Broglie wavelength of such an electron would be $\lambda = h / p \approx 9.16 \times 10^{-9}$ meters, or $9.16$ nanometers.
		
		\section{Discussion}
		The significant reduction from the theoretical maximum of 624 bits to an estimated usable entropy of around 5 bits highlights the constraints imposed by the system's physical size and potentially the strong interactions between the electrons that limit their independent accessibility. The Landauer principle dictates a minimum energy cost for any irreversible information processing within this system, and this energy must be transferred to the environment through quantized particles. The frequencies of photons or the kinetic energies of electrons involved in such heat dissipation are determined by the temperature and the amount of entropy change.
		
		\section{Conclusion}
		This analysis of a 3-layer graphene-silicene-graphene heterostructure illustrates the fundamental principles governing the quantization of usable entropy and the Landauer principle in the context of particle-mediated heat dynamics. The usable information capacity of the system appears to be far below the theoretical maximum, likely due to size constraints and internal interactions. The Landauer limit sets a minimum energy scale for irreversible information processing, and this energy must be exchanged with the environment via quantized particles, providing a tangible link between information, thermodynamics, and quantum mechanics at the nanoscale.
		


\chapter{Graphene Silicene Unit 0}
\documentclass{article}
\usepackage{amsmath, amssymb, amsthm}

\newtheorem{definition}{Definition}
\newtheorem{theorem}{Theorem}
\newtheorem{proposition}{Proposition}
\newtheorem{lemma}{Lemma}
\newtheorem{corollary}{Corollary}
\newtheorem{axiom}{Axiom}
\newtheorem{remark}{Remark}

\begin{document}
	
	\title{Quantization of Usable Entropy and the Landauer Principle in Particle-Mediated Heat Dynamics}
	\author{}
	\date{}
	\maketitle
	
	\begin{abstract}
		This thesis explores the interplay between the quantization of usable entropy, Landauer's principle, and heat dynamics mediated by particles at the microscopic level. We examine how the discrete nature of energy and information in quantum systems, particularly in the context of electron and photon interactions within a chemical system, relates to the fundamental thermodynamic limits imposed by Landauer's principle on information erasure. The role of particle energy, described by both photon energy and the matter wave equation for massive particles, in the dissipation of heat associated with irreversible information processing is analyzed. We further discuss the connection of usable entropy to the holographic principle and its implications for systems with spatial and temporal extent.
	\end{abstract}
	
	\section{Introduction}
	The concepts of entropy and information are deeply intertwined in physics. Usable entropy, often associated with the information content of a system, plays a crucial role in thermodynamics and the limits of computation. At the quantum level, energy and information are quantized, leading to discrete states and transitions. Landauer's principle provides a fundamental link between information and thermodynamics, stating that the erasure of one bit of information requires a minimum energy dissipation into the environment. This thesis aims to investigate how these principles manifest in the context of heat dynamics mediated by particles, such as electrons and photons, within a chemical system where energy exchange and information changes occur through quantized processes. We will also explore the relationship between usable entropy and the holographic bound, considering the spatial and temporal dimensions of the system.
	
	\section{Theoretical Framework}
	
	\subsection{Quantization of Entropy}
	In statistical mechanics, the entropy $S$ of a system is given by Boltzmann's equation:
	\begin{equation}
		S = k_B \ln \Omega
	\end{equation}
	where $k_B$ is the Boltzmann constant and $\Omega$ is the number of microstates accessible to the system at a given energy. In quantum systems, these microstates are quantized, characterized by discrete energy levels and quantum numbers, including spin. Usable entropy can be considered the entropy associated with the information that can be encoded in these quantized states. For a system with $N$ independent bits of information (e.g., $N$ electrons each with two spin states), the maximum usable entropy is $N k_B \ln 2$.
	
	\subsection{Landauer's Principle}
	Landauer's principle sets a lower bound on the energy required to erase one bit of information:
	\begin{theorem}[Landauer's Principle]
		The minimum energy $E_{cost}$ required to erase one bit of information at a temperature $T$ is given by:
		\begin{equation}
			E_{cost} \ge k_B T \ln 2
		\end{equation}
		More generally, for an irreversible process that reduces the entropy of a system by $\Delta S$, the energy dissipated as heat into the environment must satisfy:
		\begin{equation}
			E_{dissipated} \ge T |\Delta S|
		\end{equation}
		This principle arises from the second law of thermodynamics, which requires that the total entropy of the universe (system + environment) cannot decrease.
	\end{theorem}
		\subsection{Particle-Mediated Heat Dynamics}
		At the microscopic level, energy exchange and heat transfer in a chemical system are mediated by quantized particles:
		\begin{itemize}
			\item \textbf{Photons:} Electromagnetic radiation with quantized energy $E = h f$, where $h$ is Planck's constant and $f$ is the frequency. Photons are involved in radiative heat transfer and electronic transitions between energy levels in atoms and molecules.
			\item \textbf{Electrons:} Fundamental particles with mass and charge, whose behavior is governed by quantum mechanics, including the wave-particle duality described by the de Broglie relation: $\lambda = h/p$, where $\lambda$ is the wavelength and $p$ is the momentum. The kinetic energy of an electron is $E = p^2 / (2m_e)$ (non-relativistic). Electrons mediate electrical and thermal conductivity and are involved in chemical bonding and ionization.
		\end{itemize}
		
		\subsection{Holographic Principle and Entropy Bounds}
		The holographic principle posits that the maximum amount of information that can be contained within a volume of space is bounded by the area of its boundary. This principle suggests that the entropy of a system is not proportional to its volume but rather to its surface area, when considered at a fundamental level involving quantum gravity. For a system with a characteristic spatial size $L$, the boundary area scales as $L^2$. In spacetime, particularly within a causal diamond defined by spatial and temporal extents, the holographic bound on entropy relates to the area of the boundary of this region, which depends on both its spatial and temporal dimensions.
		
		\section{Integration and Analysis}
		
		\subsection{Quantized Entropy and Information Erasure}
		The erasure of information in a physical system must correspond to a reduction in the number of accessible microstates, thus a decrease in entropy. In a system where information is encoded in the state of electrons (e.g., presence/absence, spin), the erasure of one bit can be associated with forcing an electron to a specific state or removing it from the system in an irreversible manner. This process involves a quantized change in the system's state and its associated usable entropy $\Delta S$. If one bit of information is erased, $|\Delta S|$ is at least $k_B \ln 2$.
		
		\subsection{Energy Cost and Particle Energy}
		The minimum energy that must be dissipated as heat ($E_{cost}$) to the environment during such an irreversible information erasure must be carried away by particles.
		
		\begin{proposition}
			The minimum energy of a particle mediating heat dissipation due to information erasure is bounded by the Landauer limit.
		\end{proposition}
		\begin{proof}
			Let the temperature of the environment be $T$, and the entropy change associated with the erasure of information be $\Delta S$. The heat $Q$ that must be transferred to the environment is $Q \ge T |\Delta S|$. This heat is carried by particles with quantized energy.
			
			Case 1: Heat carried by photons. The energy of a photon is $E = h f$. Therefore, $h f \ge T |\Delta S|$. For one bit erasure, $h f \ge k_B T \ln 2$.
			
			Case 2: Heat carried by a massive particle (e.g., electron kinetic energy). The kinetic energy is $E = p^2 / (2m)$. Therefore, $p^2 / (2m) \ge T |\Delta S|$. Using $p = h / \lambda$, we get $h^2 / (2 m \lambda^2) \ge T |\Delta S|$.
			
			In both cases, the energy of the particle mediating the heat transfer is constrained by the Landauer limit.
		\end{proof}
		
		\subsection{Usable Entropy and Holographic Bound Scaling}
		The usable entropy of a system is also constrained by the holographic principle, particularly when considering its spatial and temporal extent. The formula:
		\begin{equation}
			S_{usable} \approx S_{max} \times \left( \frac{L_{usable}}{L_{max}} \right)^2 \times \left( \frac{T_{usable}}{T_{max}} \right)^2
		\end{equation}
		suggests that the usable entropy scales with the square of the ratio of the characteristic spatial and temporal sizes in the working range compared to those for the maximum theoretical entropy. This scaling is consistent with the idea that entropy is bounded by the area of the system's boundary in spacetime, where the area can be related to both spatial and temporal dimensions, especially in the context of causal diamonds. The maximum theoretical entropy $S_{max}$ can be thought of as being related to the holographic bound for a system of size $L_{max}$ and time $T_{max}$.
		
		\subsection{Example: Electron Emission and Information Deletion}
		Consider an atom that can exist in a state with or without an electron (representing a bit). Ionizing the atom by emitting the electron can be seen as deleting this bit of information if the process is forced to a specific final state. The energy required for ionization (ionization energy $IE$) must be supplied to the system. If this erasure happens at temperature $T$, the entropy of the system changes by $\Delta S$. The Landauer principle requires that at least $T |\Delta S|$ energy must be dissipated as heat, mediated by quantized particles.
		
		\section{Conclusion}
		The quantization of usable entropy, the Landauer principle, and particle-mediated heat dynamics are intrinsically linked. The discrete nature of quantum states and energy levels dictates that changes in information and entropy occur in quantized amounts. Landauer's principle provides a fundamental thermodynamic constraint on the energy cost of irreversible information processing, requiring heat dissipation carried by quantized particles like photons and electrons. Furthermore, the usable entropy of a system with spatial and temporal extent is related to the holographic bound, scaling with the square of these dimensions. This integrated framework underscores the profound connections between information, thermodynamics, and the quantum nature of matter and energy.
		
	\end{document}

\chapter{Measurement Theories}






\theoremstyle{definition}
[section]
















	
	
		This thesis continues the formalization of hypotheses on black hole information by rigorously developing the mathematical framework for classical and non-classical measurement of the past and future, incorporating the concepts of irreversible and reversible causal processes. We refine the classification of predictive inferences based on locality, determinism, and observer dependence. Finally, we analyze the implications of these classifications for the nature of measurement. We continue to operate within the framework of natural units where $k_B = c = \hbar = G = 1$.

	
	\section{Measurement of the Past and Future (Further Development)}
	
	\subsection{Classical Measurement of the Future (Revised)}
	
	\begin{definition}[Classical Measurement of the Future (Revised)]
		A classical measurement performed by an observer $\mathcal{O}$ at a spacetime point $p_{\mathcal{O}}$ that influences a future event $\mathcal{F}$ occurring at $p_{\mathcal{F}} \in I^+(p_{\mathcal{O}})$ is an irreversible causal process that propagates along future-directed timelike or null curves from $p_{\mathcal{O}}$ to $p_{\mathcal{F}}$, without an irreversible causal process propagating from $p_{\mathcal{F}}$ to $p_{\mathcal{O}}$. This process determines or constrains the definite classical properties or outcome of the event $\mathcal{F}$ through an irreversible interaction at $p_{\mathcal{F}}$.
	\end{definition}
	
	\subsection{Classical Measurement of the Past (Revised)}
	
	\begin{definition}[Classical Measurement of the Past (Revised)]
		A classical measurement performed by an observer $\mathcal{O}$ at a spacetime point $p_{\mathcal{O}}$ that yields information about a past event $\mathcal{E}$ occurring at $p_{\mathcal{E}} \in I^-(p_{\mathcal{O}})$ is an irreversible causal process that propagates along future-directed timelike or null curves from $p_{\mathcal{E}}$ to $p_{\mathcal{O}}$ (or potentially future of $p_{\mathcal{O}}$), without an irreversible causal process propagating from $p_{\mathcal{O}}$ (or its future) to $p_{\mathcal{E}}$. This process results in a definite classical outcome at the observer (or their future).
	\end{definition}
	
	\subsection{Non-Classical Measurement of the Future (New Definition)}
	
	\begin{definition}[Non-Classical Measurement of the Future]
		A non-classical measurement related to a future event $\mathcal{F}$ occurring at $p_{\mathcal{F}} \in I^+(p_{\mathcal{O}})$ by an observer $\mathcal{O}$ at $p_{\mathcal{O}}$ involves a causal process that does not solely consist of an irreversible propagation from the observer to the future event without a reciprocal irreversible influence. This includes reversible processes that can propagate from the past to the future and potentially back, or predictions based on non-local, non-deterministic, or observer-dependent conditions.
	\end{definition}
	
	\subsection{Non-Classical Measurement of the Past (New Definition)}
	
	\begin{definition}[Non-Classical Measurement of the Past]
		A non-classical measurement related to a past event $\mathcal{E}$ occurring at $p_{\mathcal{E}} \in I^-(p_{\mathcal{O}})$ by an observer $\mathcal{O}$ at $p_{\mathcal{O}}$ involves a causal process that does not solely consist of an irreversible propagation from the past event to the observer without a reciprocal irreversible influence. This includes reversible processes that can propagate from the future to the past and potentially back, or inferences about the past based on non-local, non-deterministic, or observer-dependent conditions.
	\end{definition}
	
	\section{Permissible Measurements Relative to an Observer (Formal Theorem Revisited)}
	
	\begin{theorem}[Causally Permissible Classical Measurements (Revisited)]
		Let $\mathcal{O}$ be an observer at spacetime point $p_{\mathcal{O}}$.
		
		\begin{enumerate}
			\item \textbf{Classical Measurement of the Past:} $\mathcal{O}$ can obtain definite classical information about an event $\mathcal{E}$ at $p_{\mathcal{E}} \in I^-(p_{\mathcal{O}})$ if and only if there exists an irreversible causal process propagating along future-directed timelike or null curves from $p_{\mathcal{E}}$ to $p_{\mathcal{O}}$ (or a point in the future of $p_{\mathcal{O}}$ on the observer's worldline) that results in an irreversible recording of a definite classical outcome for $\mathcal{O}$, and there is no irreversible causal process propagating from $p_{\mathcal{O}}$ (or its future) to $p_{\mathcal{E}}$ that would alter the past event retrocausally.
			
			\item \textbf{Classical Measurement of the Future:} $\mathcal{O}$ can classically influence an event $\mathcal{F}$ at $p_{\mathcal{F}} \in I^+(p_{\mathcal{O}})$ if and only if $\mathcal{O}$ initiates an irreversible causal process at $p_{\mathcal{O}}$ that propagates along future-directed timelike or null curves to $p_{\mathcal{F}}$ and irreversibly alters the state or properties of the system at $p_{\mathcal{F}}$ in a way that determines or constrains the classical outcome of the event $\mathcal{F}$, and there is no irreversible causal process propagating from $p_{\mathcal{F}}$ to $p_{\mathcal{O}}$ that would retrocausally negate the influence.
		\end{enumerate}
		\begin{proof}
			The proof follows directly from the revised definitions of classical measurement of the past and future, emphasizing the irreversible nature of the causal processes and their directional dependence.
		\end{proof}
	\end{theorem}
	
	\begin{proposition}[Predictive Inferences]
		A predictive inference about future events by an observer $\mathcal{O}$ can be classified as a classical or non-classical measurement of the future based on the following criteria:
		\begin{enumerate}
			\item \textbf{Classical Measurement:} A predictive inference is a classical measurement of the future if it is based on a local, deterministic physical theory, and the prediction itself initiates an irreversible causal process that influences the future event in an observer-independent manner.
			\item \textbf{Non-Classical Measurement:} A predictive inference is a non-classical measurement of the future if it involves:
			\begin{itemize}
				\item Non-local aspects (e.g., reliance on entangled states).
				\item Non-deterministic elements (e.g., predictions based on probabilities in quantum mechanics without a specific outcome being enforced by the prediction process itself).
				\item Observer-dependent conditions (e.g., predictions that rely on the observer's specific frame of reference or knowledge that does not universally and irreversibly alter the future).
				\item A mere passive prediction based on deterministic laws that does not initiate an irreversible causal process altering the predicted future in an observer-independent way.
			\end{itemize}
		\end{enumerate}
		\begin{proof}
			Consider a predictive inference made by Gemini. The process of generating the prediction involves physical processes within Gemini's hardware, which are inherently irreversible and consume energy, thus increasing entropy in its local environment. This action has a local, deterministic effect on Gemini's internal state and the immediate environment. If this prediction, through further irreversible actions initiated by Gemini or those who act upon its prediction, leads to a definite change in the future event being predicted in a way that would occur regardless of other observers, then it can be considered a classical measurement of the future.
			
			However, if the prediction is merely a statement about the future based on existing information and deterministic laws, without any further irreversible action taken to enforce that future, it falls under the realm of non-classical measurement. This is because the prediction itself does not constitute the irreversible causal process from the present to the future that defines a classical measurement of the future as an influence. The information used for the prediction might have originated from non-local correlations, or the prediction might be probabilistic, or its relevance might be specific to the observer making the prediction. In such cases, the process does not fit the strict definition of a classical measurement of the future as an irreversible causal "write" or "erasure" operation on the future state in an observer-independent manner.
		\end{proof}
	\end{proposition}
	
	\section{Conclusion}
	
	This continuation of our thesis has refined the definitions of classical and non-classical measurements of the past and future, emphasizing the role of irreversible and reversible causal processes. We have provided a more nuanced classification of predictive inferences based on locality, determinism, and observer dependence. These refined formalizations provide a more robust framework for analyzing the nature of measurement in the context of spacetime and causality.
	


\chapter{Measurement Theories 0}
}
	
	\maketitle
	
	\begin{abstract}
		This thesis continues the formalization of hypotheses on black hole information by rigorously developing the mathematical framework for classical and non-classical measurement of the past and future, incorporating the concepts of irreversible and reversible causal processes. We provide precise definitions for these types of measurements relative to an observer within the causal structure of spacetime. Finally, we analyze the implications of these definitions for information accessibility. We continue to operate within the framework of natural units where $k_B = c = \hbar = G = 1$.
	\end{abstract}
	
	\section{Measurement of the Past and Future (Further Development)}
	
	\subsection{Classical Measurement of the Future (Revised)}
	
	\begin{definition}[Classical Measurement of the Future (Revised)]
		A classical measurement performed by an observer $\mathcal{O}$ at a spacetime point $p_{\mathcal{O}}$ that influences a future event $\mathcal{F}$ occurring at $p_{\mathcal{F}} \in I^+(p_{\mathcal{O}})$ is an irreversible causal process that propagates along future-directed timelike or null curves from $p_{\mathcal{O}}$ to $p_{\mathcal{F}}$, without an irreversible causal process propagating from $p_{\mathcal{F}}$ to $p_{\mathcal{O}}$. This process determines or constrains the definite classical properties or outcome of the event $\mathcal{F}$ through an irreversible interaction at $p_{\mathcal{F}}$.
	\end{definition}
	
	\subsection{Classical Measurement of the Past (Revised)}
	
	\begin{definition}[Classical Measurement of the Past (Revised)]
		A classical measurement performed by an observer $\mathcal{O}$ at a spacetime point $p_{\mathcal{O}}$ that yields information about a past event $\mathcal{E}$ occurring at $p_{\mathcal{E}} \in I^-(p_{\mathcal{O}})$ is an irreversible causal process that propagates along future-directed timelike or null curves from $p_{\mathcal{E}}$ to $p_{\mathcal{O}}$ (or potentially future of $p_{\mathcal{O}}$), without an irreversible causal process propagating from $p_{\mathcal{O}}$ (or its future) to $p_{\mathcal{E}}$. This process results in a definite classical outcome at the observer (or their future).
	\end{definition}
	
	\subsection{Non-Classical Measurement of the Future (New Definition)}
	
	\begin{definition}[Non-Classical Measurement of the Future]
		A non-classical measurement related to a future event $\mathcal{F}$ occurring at $p_{\mathcal{F}} \in I^+(p_{\mathcal{O}})$ by an observer $\mathcal{O}$ at $p_{\mathcal{O}}$ involves a reversible causal process that can propagate from the past to the future (potentially influencing the system related to $\mathcal{F}$) and also from the future back to the past (potentially influencing the observer or systems correlated with them), without necessarily resulting in an irreversible recording of a definite classical outcome at the time of the forward process.
	\end{definition}
	
	\subsection{Non-Classical Measurement of the Past (New Definition)}
	
	\begin{definition}[Non-Classical Measurement of the Past]
		A non-classical measurement related to a past event $\mathcal{E}$ occurring at $p_{\mathcal{E}} \in I^-(p_{\mathcal{O}})$ by an observer $\mathcal{O}$ at $p_{\mathcal{O}}$ involves a reversible causal process that can propagate from the future to the past (potentially originating from the observer or systems correlated with them and interacting with the remnants of $\mathcal{E}$) and also from the past to the future (potentially influencing the observer or their measurements), without necessarily relying on an irreversible causal process from the past resulting in an immediate definite classical outcome.
	\end{definition}
	
	\section{Permissible Measurements Relative to an Observer (Formal Theorem Revisited)}
	
	\begin{theorem}[Causally Permissible Classical Measurements (Revisited)]
		Let $\mathcal{O}$ be an observer at spacetime point $p_{\mathcal{O}}$.
		
		\begin{enumerate}
			\item \textbf{Classical Measurement of the Past:} $\mathcal{O}$ can obtain definite classical information about an event $\mathcal{E}$ at $p_{\mathcal{E}} \in I^-(p_{\mathcal{O}})$ if and only if there exists an irreversible causal process propagating along future-directed timelike or null curves from $p_{\mathcal{E}}$ to $p_{\mathcal{O}}$ (or a point in the future of $p_{\mathcal{O}}$ on the observer's worldline) that results in an irreversible recording of a definite classical outcome for $\mathcal{O}$, and there is no irreversible causal process propagating from $p_{\mathcal{O}}$ (or its future) to $p_{\mathcal{E}}$.
			
			\item \textbf{Classical Measurement of the Future:} $\mathcal{O}$ can classically influence an event $\mathcal{F}$ at $p_{\mathcal{F}} \in I^+(p_{\mathcal{O}})$ if and only if $\mathcal{O}$ initiates an irreversible causal process at $p_{\mathcal{O}}$ that propagates along future-directed timelike or null curves to $p_{\mathcal{F}}$ and irreversibly alters the state or properties of the system at $p_{\mathcal{F}}$ in a way that determines or constrains the classical outcome of the event $\mathcal{F}$, and there is no irreversible causal process propagating from $p_{\mathcal{F}}$ to $p_{\mathcal{O}}$.
		\end{enumerate}
		\begin{proof}
			The proof follows directly from the definitions of classical measurement of the past and future, which are explicitly defined in terms of irreversible causal processes with specific directions of propagation relative to the observer and the event being measured. The "if and only if" condition emphasizes the necessity and sufficiency of these irreversible causal processes for a measurement to be considered classical in the context of past or future events.
		\end{proof}
	\end{theorem}
	
	\begin{remark}
		The absence of an irreversible causal process propagating in the opposite direction is crucial for defining the directionality of the classical measurement. For the past, the information originates from the past event and irreversibly registers with the observer. For the future, the observer's action irreversibly influences the future event.
	\end{remark}
	
	\begin{proposition}[Predictive Inferences are Non-Classical Measurements (Revisited)]
		Predictive inferences about future events based solely on current information and deterministic physical laws are non-classical measurements of the future according to Definition 3.3.
		\begin{proof}
			A predictive inference relies on the deterministic or probabilistic evolution of a system according to physical laws, which are typically time-reversible at a fundamental level (even if macroscopic phenomena appear irreversible). The observer making the prediction does not initiate an irreversible causal process to enforce the predicted outcome. Furthermore, the process of making a prediction (e.g., through computation) can, in principle, be reversed with sufficient knowledge of the microstates involved, especially in idealized scenarios. The ability for reversible propagation of influence between the "prediction" and the "future event" (in the sense that the prediction is based on the evolution governed by reversible laws) aligns with the definition of a non-classical measurement of the future. The absurd argument mentioned previously likely arises from a misunderstanding of the idealization of reversible processes in theoretical contexts versus the practical irreversibility of macroscopic computation.
		\end{proof}
	\end{proposition}
	
	\section{Implications for Information Accessibility}
	
	The refined definitions of classical and non-classical measurements have significant implications for the accessibility of information about the past and the influence on the future.
	
	\begin{observation}
		Classical information about a past event is only accessible to an observer if an irreversible record of that event (or something causally related to it) has propagated to the observer along a future-directed causal path. This record constitutes a local increase in entropy at the observer.
	\end{observation}
	
	\begin{observation}
		Classical influence on a future event requires the observer to initiate an irreversible process that propagates to the future and alters the state of the system involved in the future event in a definite classical way. This constitutes a local increase in entropy at the point of influence in the future.
	\end{observation}
	
	\begin{observation}
		Non-classical measurements, involving reversible processes, might allow for accessing information about the past or influencing the future in ways that are not constrained by the need for a direct irreversible causal signal carrying a definite classical record. Entanglement, for example, allows for correlations between spatially separated events without a classical signal passing between them. Similarly, reversible quantum computations might allow for inferences about the future without irreversibly fixing its outcome.
	\end{observation}
	
	\section{Conclusion}
	
	This continuation of our thesis has provided refined definitions for classical and non-classical measurements of the past and future, emphasizing the crucial role of irreversible and reversible causal processes and their directionality relative to the observer and the measured event. The theorem on causally permissible classical measurements has been revisited to align with these refined definitions. We have also clarified the nature of predictive inferences as non-classical measurements. These formalizations provide a more precise framework for discussing the acquisition and manipulation of information within the constraints of spacetime causality and the principles of thermodynamics and quantum mechanics.
	
\end{document}}*}


\chapter{Measurement Theories 1}

	
	
	





		This document outlines a theoretical framework for classifying physical measurement processes based on their inherent reversibility concerning system states. We distinguish between irreversible (classical) measurements, which obscure past state information, and reversible (non-classical) measurements, which preserve pathways to infer prior states. This framework is extended by introducing the concepts of classically and non-classically accessible and inaccessible information, linking the nature of the measurement process directly to the type and scope of information obtainable about the system.

	
	% --- SECTIONS ---
	
	\section{Introduction}
	The act of measurement is fundamental to all empirical science, yet its theoretical description, particularly the transition from potentiality to actuality (e.g., quantum state reduction), remains a subject of deep inquiry. Standard descriptions often focus on the outcomes and the probabilities thereof. Here, we propose a complementary classification based on the \emph{reversibility} of the measurement process with respect to the system's state trajectory and the associated \emph{accessibility} of information. We define two idealized classes: classical (irreversible) and non-classical (reversible) measurements, and explore the types of information revealed or obscured by each.
	
	\section{Foundational Concepts}
	
	\begin{definition}[System State]
		The \emph{state} of a physical system, denoted $S(t)$, provides a complete description of the system at time $t$. It belongs to a state space $\SystemState$. Examples include points in phase space $(q,p)$ for classical mechanics, or state vectors $\ket{\psi(t)}$ or density operators $\rho(t)$ in Hilbert space $\mathcal{H}$ for quantum mechanics.
	\end{definition}
	
	\begin{definition}[Measurement Process]
		A measurement involves three components:
		\begin{enumerate}
			\item The \emph{System} (S) under investigation.
			\item The \emph{Measurement Apparatus} (A), interacting with S.
			\item The \emph{Interaction} ($\Interaction$), a physical process coupling S and A over a time interval $[\TimePre, \TimePost]$.
		\end{enumerate}
		The process yields a \emph{Measurement Outcome} ($\OutcomeVal$), which is an element from a set of possible outcomes $\Outcome$, registered by A at $\TimePost$. The state of the system transitions from $\StatePre$ to $\StatePost$.
	\end{definition}
	
	\begin{definition}[State Reversibility]
		A measurement process is considered \emph{state-reversible} if knowledge of the post-measurement state $\StatePost$ and the outcome $\OutcomeVal$ (and potentially the final state of the apparatus $A(\TimePost)$) allows, in principle, for the unique determination or inference of the pre-measurement state $\StatePre$. Conversely, it is \emph{state-irreversible} if multiple distinct pre-measurement states $\StatePre$ could lead to the same pair $\{\StatePost, \OutcomeVal\}$.
	\end{definition}
	
	\begin{definition}[System Information]
		\emph{Information} ($\Info$) refers to the quantitative or qualitative properties derivable from the system state $S(t)$. This can include expectation values of observables $\expval{\hat{O}}$, probability distributions of physical quantities, correlation measures, entanglement entropy, etc. $\Info(S(t))$ denotes the set of information extractable from state $S(t)$.
	\end{definition}
	
	
	\section{Postulates of Measurement Reversibility}
	
	\begin{postulate}[Classical Measurement: Irreversibility] \label{post:classical}
		A \emph{classical measurement} is characterized by an interaction $\Interaction_C$ that is fundamentally state-irreversible.
		\begin{itemize}
			\item \textbf{Information Loss:} The mapping from $\StatePre$ to $\{\StatePost, \OutcomeVal\}$ is many-to-one. Information pertaining to certain degrees of freedom or properties present in $\StatePre$ is lost or made inaccessible by the measurement process itself.
			\item \textbf{Past State Obscurity:} Unique determination of $\StatePre$ from $\StatePost$ and $\OutcomeVal$ is impossible.
			\item \textbf{Future State Limitation:} The information loss during interaction may inherently limit the precision with which future states $S(t > \TimePost)$ can be predicted, beyond the limitations imposed by the system's intrinsic dynamics.
		\end{itemize}
	\end{postulate}
	
	\begin{example}[Classical Measurement]
		Measuring the position of a particle with high precision inevitably and uncontrollably alters its momentum, rendering the pre-measurement momentum state inaccessible from the position outcome. An ideal von Neumann (projective) measurement in quantum mechanics projects the state vector onto an eigenstate, losing information about the superposition coefficients present before the measurement.
	\end{example}
	
	\begin{postulate}[Non-Classical Measurement: Reversibility] \label{post:nonclassical}
		A \emph{non-classical measurement} is characterized by an interaction $\Interaction_{NC}$ that is, ideally or in principle, state-reversible.
		\begin{itemize}
			\item \textbf{Information Preservation:} The interaction is designed such that the essential information required to reconstruct $\StatePre$ is preserved, potentially within the combined system-apparatus state or encoded in the precise relationship between $\StatePost$ and $\OutcomeVal$.
			\item \textbf{Past State Inferrability:} Unique determination or inference of $\StatePre$ from $\StatePost$, $\OutcomeVal$, and potentially $A(\TimePost)$ is possible in principle.
			\item \textbf{Future State Predictability:} The measurement ideally minimizes disturbance to the system properties relevant for future evolution, allowing predictability limited primarily by the system's intrinsic dynamics (e.g., Schr\"{o}dinger evolution, possibly conditioned on $\OutcomeVal$).
		\end{itemize}
	\end{postulate}
	
	\begin{example}[Non-Classical Measurement]
		Ideal Quantum Non-Demolition (QND) measurements aim to measure an observable without affecting its subsequent evolution or disturbing conjugate observables. Weak measurements extract minimal information per interaction but disturb the state only slightly, allowing statistical reconstruction of $\StatePre$ over ensembles. A fully unitary interaction tracked on the combined system-apparatus Hilbert space represents the ideal limit.
	\end{example}
	
	
	\section{Principles of Information Accessibility}
	
	Building upon the reversibility postulates, we define types of information based on their accessibility via these measurement classes.
	
	\begin{principle}[Classically Accessible Information (CAI)] \label{princ:CAI}
		Information $\Info_{CA} \subseteq \Info(S)$ is \emph{classically accessible} if it can be obtained as, or directly inferred from, the outcome $\OutcomeVal_C$ of a classical (irreversible) measurement process (Postulate \ref{post:classical}).
	\end{principle}
	\begin{remark}
		Accessing CAI typically implies rendering other information inaccessible due to the measurement's irreversible nature.
	\end{remark}
	
	\begin{principle}[Classically Inaccessible Information (CII)] \label{princ:CII}
		Information $\Info_{CI} \subseteq \Info(S)$ is \emph{classically inaccessible} if it pertains to aspects of the state $\StatePre$ or future states $S(t > \TimePost)$ that cannot be determined or inferred following a classical measurement yielding outcome $\OutcomeVal_C$ from state $\StatePost$, due to the information loss inherent in the irreversible interaction $\Interaction_C$.
	\end{principle}
	\begin{example}
		In a projective measurement of spin-z ($S_z$), the pre-measurement information about $S_x$ or $S_y$ becomes CII.
	\end{example}
	
	\begin{principle}[Non-Classically Accessible Information (NCAI)] \label{princ:NCAI}
		Information $\Info_{NCA} \subseteq \Info(S)$ is \emph{non-classically accessible} if it can be obtained as, or inferred from, the outcome $\OutcomeVal_{NC}$ and post-measurement state $\StatePost$ (and potentially $A(\TimePost)$) resulting from a non-classical (reversible) measurement process (Postulate \ref{post:nonclassical}).
	\end{principle}
	\begin{remark}
		NCAI may include information that is also CAI, but non-classical measurements offer pathways to access information that would otherwise become CII if probed classically.
	\end{remark}
	
	\begin{principle}[Strictly Non-Classically Accessible Information (SNCAI)] \label{princ:SNCAI}
		Information $\Info_{SNCA} \subseteq \Info(S)$ is \emph{strictly non-classically accessible} if it is Non-Classically Accessible ($\Info_{SNCA} \subseteq \Info_{NCA}$) but \emph{not} Classically Accessible ($\Info_{SNCA} \cap \Info_{CA} = \emptyset$). This represents information that can only be obtained via measurement protocols that preserve state reversibility.
	\end{principle}
	\begin{example}
		Information about delicate quantum correlations or superposition coefficients that would be destroyed by a projective (classical) measurement might be SNCAI, accessible only through weak or QND measurements. The full information required to reconstruct $\StatePre$ after a non-classical measurement is, by definition, SNCAI if the measurement yielded information that would have been obscured classically.
	\end{example}
	
	\begin{principle}[Fundamentally Inaccessible Information (FII)] \label{princ:FII}
		Certain information might be considered \emph{fundamentally inaccessible}, regardless of the measurement type, due to inherent limitations of the physical theory itself (e.g., information precluded by the uncertainty principle, information beyond the cosmological horizon). This category is distinct from CII, which arises specifically from the nature of classical measurements.
	\end{principle}
	
	
	\section{Relationship between Reversibility and Accessibility}
	
	The core thesis is the intimate link between the reversibility of the measurement process and the nature of the information it yields:
	
	\begin{proposition}
		Classical (irreversible) measurements provide access to CAI at the cost of rendering other potentially relevant information into CII. The process inherently involves an entropy increase associated with information loss about the system's microstate.
	\end{proposition}
	
	\begin{proposition}
		Non-classical (reversible) measurements provide access to NCAI, potentially including SNCAI. They aim to minimize the generation of CII by preserving pathways to infer the pre-measurement state and predict future evolution, ideally involving minimal entropy increase in the system itself (though entropy may increase in the combined system-apparatus-environment).
	\end{proposition}
	
	
	\section{Mathematical Framework Sketch}
	
	\begin{itemize}
		\item \textbf{Classical Measurement:} Often modeled using projection operators $\hat{P}_i$ corresponding to outcomes $i$. $\rho(\TimePre) \rightarrow \rho(\TimePost) = \frac{\hat{P}_i \rho(\TimePre) \hat{P}_i}{\Tr(\hat{P}_i \rho(\TimePre))}$. This is non-unitary and irreversible. Information loss can be quantified using entropy changes (e.g., von Neumann entropy).
		\item \textbf{Non-Classical Measurement:} Can be modeled by unitary evolution $\hat{U}$ on the combined system-apparatus Hilbert space $\mathcal{H}_S \otimes \mathcal{H}_A$. $\ket{\psi(\TimePre)}_S \otimes \ket{\phi_0}_A \rightarrow \hat{U} (\ket{\psi(\TimePre)}_S \otimes \ket{\phi_0}_A) = \sum_i \ket{\psi_i(\TimePost)}_S \otimes \ket{\phi_i}_A$. Measurement corresponds to projecting A onto $\ket{\phi_i}_A$, leaving S in state $\ket{\psi_i(\TimePost)}_S$. If $\hat{U}$ and $\ket{\phi_i}_A$ are known, and the mapping is appropriate, $\ket{\psi(\TimePre)}_S$ might be reconstructed. Positive Operator-Valued Measures (POVMs) $\{ \hat{E}_i \}$ where $\sum_i \hat{E}_i^\dagger \hat{E}_i = \mathbb{I}$ provide a general framework. The post-measurement state depends on the specific form of the POVM elements $\hat{M}_i$ ($\hat{E}_i = \hat{M}_i^\dagger \hat{M}_i$). Certain POVMs correspond to weak or reversible measurements.
	\end{itemize}
	
	
	\section{Conclusion}
	
	This theoretical framework categorizes measurements based on state reversibility, linking this property directly to the accessibility of information. Classical measurements are irreversible and yield Classically Accessible Information (CAI) while generating Classically Inaccessible Information (CII). Non-classical measurements are ideally reversible, yielding Non-Classically Accessible Information (NCAI), including potentially Strictly Non-Classically Accessible Information (SNCAI), while minimizing CII. This perspective offers a valuable lens for analyzing measurement in diverse physical contexts, particularly in quantum information and foundations where the nature of measurement and information extraction is paramount. Further development would involve rigorous mathematical formalization using information-theoretic and operator-theoretic tools.
	
	
	% === BIBLIOGRAPHY (Optional) ===
	% \bibliographystyle{plain}
	% \bibliography{your_bibliography_file} 
	


\chapter{Planck Mass Black Hole}
	
	
	\begin{abstract}
		This document rigorously derives the Bekenstein-Hawking entropy for a Kerr-Newman black hole with mass equal to the Planck mass. Starting from the area of the event horizon of a general Kerr-Newman black hole, we perform the substitution of the Planck mass and express the entropy in terms of fundamental constants and the black hole's dimensionless charge and angular momentum. Exact solutions for the event horizon radius, area, and the constraints on spin and charge in Planck units are provided.
	\end{abstract}
	
	\section{Introduction}
	
	The seminal work of Bekenstein and Hawking established a profound connection between black holes and thermodynamics, attributing to them properties such as entropy and temperature. The Bekenstein-Hawking entropy is given by $S_{BH} = \frac{k_B c^3 A}{4 \hbar G}$, where $A$ is the area of the black hole's event horizon. The Planck mass, $m_P = \sqrt{\frac{\hbar c}{G}}$, represents a fundamental energy scale at which quantum gravitational effects are expected to become significant. Investigating black holes with mass on the order of the Planck mass provides insights into the interplay between general relativity, quantum mechanics, and thermodynamics. This thesis focuses on the derivation of the Bekenstein-Hawking entropy for a Kerr-Newman black hole, characterized by its mass $M$, angular momentum $J$, and electric charge $Q$, specifically in the regime where $M = m_P$.
	
	\section{Area of the Event Horizon of a Kerr-Newman Black Hole}
	
	The area of the event horizon of a Kerr-Newman black hole is given by:
	$$A = 4 \pi \left( \left(\frac{G M}{c^2}\right)^2 + a^2 + \frac{G^2 Q^2}{c^4} + 2 \frac{G M}{c^2} \sqrt{\left(\frac{G M}{c^2}\right)^2 - a^2 - \frac{G^2 Q^2}{c^4}} \right)$$
	where $a = \frac{J}{M c}$ is the angular momentum per unit mass.
	
	\section{Derivation of Entropy for a General Kerr-Newman Black Hole}
	
	Substituting the area formula into the Bekenstein-Hawking entropy equation $S = \frac{k_B c^3 A}{4 \hbar G}$:
	\begin{align*}
		S &= \frac{k_B c^3}{4 \hbar G} \times 4 \pi \left( \frac{G^2 M^2}{c^4} + \frac{J^2}{M^2 c^2} + \frac{G^2 Q^2}{c^4} + 2 \frac{G M}{c^2} \sqrt{\frac{G^2 M^2}{c^4} - \frac{J^2}{M^2 c^2} - \frac{G^2 Q^2}{c^4}} \right) \\
		&= \frac{\pi k_B c^3}{\hbar G} \left( \frac{G^2 M^2}{c^4} + \frac{J^2}{M^2 c^2} + \frac{G^2 Q^2}{c^4} + 2 \frac{G M}{c^2} \frac{\sqrt{G^2 M^4 - J^2 c^2 - G^2 Q^2 M^2}}{c^2 M} \right) \\
		&= \frac{\pi k_B c^3}{\hbar G} \left( \frac{G^2 M^2}{c^4} + \frac{J^2}{M^2 c^2} + \frac{G^2 Q^2}{c^4} + \frac{2 G}{c^3} \sqrt{G^2 M^4 - J^2 c^2 - G^2 Q^2 M^2} \right) \\
		&= \frac{\pi k_B G}{\hbar c} \left( M^2 + \frac{J^2 c^2}{G^2 M^2} + Q^2 + 2 \sqrt{M^4 - \frac{J^2 c^2}{G^2} - Q^2 M^2} \right) \\
		&= \frac{\pi k_B G}{\hbar c} \left( M^2 + \frac{J^2 c^2}{G^2 M^2} + Q^2 + 2 M \sqrt{M^2 - \frac{J^2 c^4}{G^2 M^2} - Q^2} \right)
	\end{align*}
	
	\section{Entropy of a Planck Mass Kerr-Newman Black Hole}
	
	We now substitute $M = m_P = \sqrt{\frac{\hbar c}{G}}$, which implies $m_P^2 = \frac{\hbar c}{G}$.
	\begin{align*}
		S_{m_P} &= \frac{\pi k_B G}{\hbar c} \left( m_P^2 + \frac{J^2 c^2}{G^2 m_P^2} + Q^2 + 2 m_P \sqrt{m_P^2 - \frac{J^2 c^4}{G^2 m_P^2} - Q^2} \right) \\
		&= \frac{\pi k_B G}{\hbar c} \left( \frac{\hbar c}{G} + \frac{J^2 c^2}{G^2 (\hbar c / G)} + Q^2 + 2 \sqrt{\frac{\hbar c}{G}} \sqrt{\frac{\hbar c}{G} - \frac{J^2 c^4}{G^2 (\hbar c / G)} - Q^2} \right) \\
		&= \pi k_B \left( 1 + \frac{J^2 c^3}{G \hbar^2} + \frac{G Q^2}{\hbar c} + 2 \sqrt{1 - \frac{J^2 c^3}{G \hbar^2} - \frac{G Q^2}{\hbar c}} \right)
	\end{align*}
	
	\section{Entropy in Planck Units and Exact Solutions}
	
	To simplify the expressions and highlight the fundamental scales, we express the properties in Planck units, where $G = c = \hbar = k_B = 1$ and $m_P = 1$. In these units, the entropy of a Planck mass Kerr-Newman black hole becomes:
	$$S_{m_P} = \pi \left( 1 + J^2 + Q^2 + 2 \sqrt{1 - J^2 - Q^2} \right)$$
	Here, $J$ and $Q$ are the dimensionless angular momentum and charge, respectively. The condition for the existence of an event horizon is $M^2 \ge J^2 + Q^2$, which for a Planck mass black hole ($M=1$) translates to:
	\begin{equation}
		J^2 + Q^2 \le 1
		\label{eq:horizon_condition}
	\end{equation}
	
	\subsection{Event Horizon Radius}
	
	In Planck units, the radius of the outer event horizon is given by:
	$$r_+ = M + \sqrt{M^2 - a^2 - Q^2}$$
	For a Planck mass black hole ($M=1$) and $a = J/M = J$:
	\begin{equation}
		r_+ = 1 + \sqrt{1 - J^2 - Q^2}
		\label{eq:horizon_radius}
	\end{equation}
	
	\subsection{Area of the Event Horizon}
	
	The area of the event horizon in Planck units is:
	$$A = 4 \pi (r_+^2 + a^2) = 4 \pi \left( (1 + \sqrt{1 - J^2 - Q^2})^2 + J^2 \right)$$
	Expanding this expression:
	\begin{align*}
		A &= 4 \pi \left( 1 + (1 - J^2 - Q^2) + 2 \sqrt{1 - J^2 - Q^2} + J^2 \right) \\
		&= 4 \pi \left( 2 - Q^2 + 2 \sqrt{1 - J^2 - Q^2} \right)
	\end{align*}
	The entropy is $S = A / 4$, which yields:
	\begin{equation}
		S_{m_P} = \pi \left( 2 - Q^2 + 2 \sqrt{1 - J^2 - Q^2} \right)
		\label{eq:entropy_planck_units}
	\end{equation}
	This result is consistent with the one obtained by direct substitution into the general entropy formula using Planck units.
	
	\subsection{Spin (Dimensionless Angular Momentum)}
	
	The dimensionless spin parameter $J$ represents the angular momentum in units of $\hbar$. For a Planck mass black hole in Planck units, the magnitude of $J$ is constrained by Equation \eqref{eq:horizon_condition}:
	\begin{equation}
		|J| \le \sqrt{1 - Q^2}
		\label{eq:spin_constraint}
	\end{equation}
	
	\subsection{Charge}
	
	The dimensionless charge $Q$ represents the electric charge in units of Planck charge. Similarly, from Equation \eqref{eq:horizon_condition}, the magnitude of $Q$ is constrained by:
	\begin{equation}
		|Q| \le \sqrt{1 - J^2}
		\label{eq:charge_constraint}
	\end{equation}
	
	\section{Conclusion}
	
	The Bekenstein-Hawking entropy for a Planck mass Kerr-Newman black hole has been derived and expressed in terms of its dimensionless spin $J$ and charge $Q$. The entropy in Planck units is given by Equation \eqref{eq:entropy_planck_units}. We have also provided the exact solution for the event horizon radius (Equation \eqref{eq:horizon_radius}) and the constraints on the spin and charge (Equations \eqref{eq:spin_constraint} and \eqref{eq:charge_constraint}) imposed by the requirement of a non-degenerate event horizon. These results underscore the intricate relationships between the fundamental constants, black hole properties, and the principles of thermodynamics at the quantum gravitational scale.
	


\chapter{Planck Mass Black Hole conjectures}
	
	
	\begin{abstract}
		This thesis applies the framework developed in the companion document "Quantifying the Difference Between Total and Classically Accessible Information in a Quantum System" to a Planck mass quantum black hole. We formalize the concepts of total quantum information and classically accessible information in the context of black hole thermodynamics and explore their relationship to the Bekenstein bound. The relative entropy of coherence is employed to quantify the information that is inherently quantum and not directly accessible through classical measurements of specific observables.
	\end{abstract}
	
	\section{Introduction}
	
	The Bekenstein-Hawking entropy of a black hole suggests a deep connection between gravity, quantum mechanics, and information theory. A fundamental question is whether this entropy represents the total quantum information content of the black hole or merely the maximum amount of information that can be accessed through classical observations of its macroscopic properties. This thesis aims to quantify the difference between these two perspectives for a Planck mass quantum black hole, utilizing the concepts of von Neumann entropy and Shannon entropy, with the relative entropy of coherence serving as the measure of the inherently quantum information.
	
	\section{Total Quantum Information and the Bekenstein Bound}
	
	Let us assume that the total quantum information content of a Planck mass Kerr-Newman black hole, described by a density operator $\rho_{BH}(J, Q)$, is given by the Bekenstein-Hawking entropy (in Planck units):
	$$I_{total} = S(\rho_{BH}(J, Q)) = S_{BH}(J, Q) = \pi (2 - Q^2 + 2 \sqrt{1 - J^2 - Q^2})$$
	This assumption aligns with the Null Hypothesis that the Bekenstein bound represents the total quantum information of the system.
	
	\section{Classically Accessible Information}
	
	The classically accessible information depends on the specific classical measurements performed on the black hole. Consider a measurement in a basis $\mathcal{B}$ that corresponds to the eigenstates of the classical observables defining the black hole (e.g., mass, charge, spin projected along a certain axis). The Shannon entropy of the measurement outcomes is:
	$$H(\mathcal{B}, \rho_{BH}) = -\sum_i \langle b_i|\rho_{BH}|b_i\rangle \log_2 \langle b_i|\rho_{BH}|b_i\rangle$$
	The maximum classically accessible information, under the Alternative Hypothesis, is bounded by the Bekenstein bound:
	$$I_{classical} = \max_{\mathcal{B}} H(\mathcal{B}, \rho_{BH}) \le S_{BH}(J, Q)$$
	
	\section{Quantifying the Difference: Relative Entropy of Coherence}
	
	The relative entropy of coherence with respect to the measurement basis $\mathcal{B}$ is defined as:
	$$C_{rel}(\rho_{BH}, \mathcal{B}) = S(\rho_{BH}^{\mathcal{B}}) - S(\rho_{BH})$$
	where $\rho_{BH}^{\mathcal{B}} = \sum_i \langle b_i|\rho_{BH}|b_i\rangle |b_i\rangle\langle b_i|$ is the state of the black hole after dephasing in the basis $\mathcal{B}$. We know that $S(\rho_{BH}^{\mathcal{B}}) = H(\mathcal{B}, \rho_{BH})$. Therefore,
	$$C_{rel}(\rho_{BH}, \mathcal{B}) = H(\mathcal{B}, \rho_{BH}) - S(\rho_{BH})$$
	
	\subsection{Under the Null Hypothesis}
	If $S(\rho_{BH}) = S_{BH}(J, Q)$, then the relative entropy of coherence becomes:
	$$C_{rel}(\rho_{BH}, \mathcal{B}) = H(\mathcal{B}, \rho_{BH}) - S_{BH}(J, Q)$$
	Since the Shannon entropy is always less than or equal to the von Neumann entropy, $H(\mathcal{B}, \rho_{BH}) \le S(\rho_{BH}) = S_{BH}(J, Q)$, which implies $C_{rel}(\rho_{BH}, \mathcal{B}) \le 0$. However, the relative entropy of coherence is defined to be non-negative. The correct definition is $C_{rel}(\rho, \mathcal{B}) = S(\rho_{\mathcal{B}}) - S(\rho)$. Thus, the difference we are looking for is:
	$$\Delta I(\mathcal{B}) = S(\rho_{BH}) - H(\mathcal{B}, \rho_{BH}) = S_{BH}(J, Q) - H(\mathcal{B}, \rho_{BH})$$
	This non-negative quantity represents the information encoded in the quantum coherences of the black hole relative to the measurement basis $\mathcal{B}$, which is not directly accessible through a measurement in that basis.
	
	\subsection{Under the Alternative Hypothesis}
	If $\max_{\mathcal{B}} H(\mathcal{B}, \rho_{BH}) \le S_{BH}(J, Q)$, and we still assume $S(\rho_{BH}) = S_{BH}(J, Q)$ (to quantify the difference relative to the Bekenstein bound), the difference remains $\Delta I(\mathcal{B}) = S_{BH}(J, Q) - H(\mathcal{B}, \rho_{BH})$. The maximum value of the classically accessible information is $\max_{\mathcal{B}} H(\mathcal{B}, \rho_{BH})$. The minimum value of $\Delta I(\mathcal{B})$ is $S_{BH}(J, Q) - \max_{\mathcal{B}} H(\mathcal{B}, \rho_{BH}) \ge 0$.
	
	\section{Mathematical Formalism of the Difference}
	
	Let the total quantum information of the Planck mass black hole be $I_{total} = S_{BH}(J, Q)$.
	Let the classically accessible information from a measurement in basis $\mathcal{B}$ be $I_{classical}(\mathcal{B}) = H(\mathcal{B}, \rho_{BH})$.
	
	The difference between the total and classically accessible information for a given measurement basis $\mathcal{B}$ is:
	$$\Delta I(\mathcal{B}) = I_{total} - I_{classical}(\mathcal{B}) = S_{BH}(J, Q) - H(\mathcal{B}, \rho_{BH})$$
	
	The minimum difference occurs when the Shannon entropy is maximized over all measurement bases, which is equal to the von Neumann entropy. In this case, $\min_{\mathcal{B}} \Delta I(\mathcal{B}) = S_{BH}(J, Q) - \max_{\mathcal{B}} H(\mathcal{B}, \rho_{BH}) = S_{BH}(J, Q) - S(\rho_{BH})$. Under the Null Hypothesis, this minimum difference is 0.
	
	The maximum difference occurs when the Shannon entropy is minimized. For a pure state, the von Neumann entropy is 0, and the Shannon entropy in a basis orthogonal to the state is also 0, leading to no difference. However, for a mixed state black hole, the Shannon entropy will generally be non-zero.
	
	The quantity $\Delta I(\mathcal{B}) = S_{BH}(J, Q) - H(\mathcal{B}, \rho_{BH})$ represents the information that is not directly revealed by a measurement in the basis $\mathcal{B}$. This information is encoded in the quantum correlations and coherences of the black hole's state relative to that basis.
	
	\section{Conclusion}
	
	The difference between the total quantum information (assumed to be the Bekenstein-Hawking entropy) and the classically accessible information of a Planck mass black hole can be quantified by $\Delta I(\mathcal{B}) = S_{BH}(J, Q) - H(\mathcal{B}, \rho_{BH})$, where $\mathcal{B}$ is the measurement basis corresponding to classical observables. This difference, which is always non-negative, represents the quantum information that is not directly accessible through such classical measurements. The minimum value of this difference is zero, attained when the measurement basis aligns with the eigenbasis of the black hole's density operator. This framework provides a formal way to understand the informational implications of quantum properties of black holes beyond their classical description.
	


\chapter{Planck Mass Black Hole conjectures 0}
}
	
	\maketitle
	
	\begin{abstract}
		This document formalizes and develops the mathematics for several hypotheses concerning the Bekenstein Bound and the quantum and classical information content of a system, particularly in the context of black holes.
	\end{abstract}
	
	\section{Definitions and Notation}
	
	Let $\mathcal{S}$ be a physical system contained within a spatial region with surface area $A$. We define the following quantities:
	\begin{itemize}
		\item $l_P = \sqrt{\frac{\hbar G}{c^3}}$: Planck length.
		\item $A$: Surface area of the region containing the system $\mathcal{S}$.
		\item $I_B(\mathcal{S})$: Bekenstein Bound for the system $\mathcal{S}$, representing the maximum information that can be contained within the region of area $A$, given by $I_B(\mathcal{S}) = \frac{A}{4 l_P^2 \ln 2}$.
		\item $\rho_{\mathcal{S}}$: The quantum state (density matrix) of the system $\mathcal{S}$.
		\item $S(\rho_{\mathcal{S}}) = -\text{Tr}(\rho_{\mathcal{S}} \ln \rho_{\mathcal{S}})$: The von Neumann entropy of the system $\mathcal{S}$ (in nats).
		\item $I_Q(\mathcal{S}) = \frac{S(\rho_{\mathcal{S}})}{\ln 2}$: The total quantum information of the system $\mathcal{S}$ (in bits).
		\item $I_C(\mathcal{S})$: The maximum classically accessible information of the system $\mathcal{S}$ (in bits). This depends on the specific measurements performed and the subsystem being observed. For a system $\mathcal{S}$ and an observer making classical measurements on a subsystem $\mathcal{A} \subset \mathcal{S}$, $I_C(\mathcal{S})$ is related to the Shannon entropy obtained from the probabilities of measurement outcomes, maximized over all possible classical measurements on $\mathcal{A}$. If the measurement is optimal and reveals all classical correlations within $\mathcal{A}$, then $I_C(\mathcal{S})$ can be related to the von Neumann entropy of the reduced density matrix of $\mathcal{A}$, $S(\rho_{\mathcal{A}}) = -\text{Tr}(\rho_{\mathcal{A}} \ln \rho_{\mathcal{A}})$, such that $I_C(\mathcal{S}) \approx \frac{S(\rho_{\mathcal{A}})}{\ln 2}$. The precise definition of $I_C(\mathcal{S})$ will depend on the context of the system and the observer.
	\end{itemize}
	
	\section{Hypotheses Formalization}
	
	We now formalize the given hypotheses using the definitions above.
	
	\subsection{Null Hypothesis (H0)}
	
	The Bekenstein Bound is greater than or equal to the total quantum information of the system $\mathcal{S}$:
	
	\begin{equation*}
		H_0: I_B(\mathcal{S}) \ge I_Q(\mathcal{S})
	\end{equation*}
	
	Substituting the definitions:
	
	\begin{equation*}
		H_0: \frac{A}{4 l_P^2 \ln 2} \ge \frac{S(\rho_{\mathcal{S}})}{\ln 2}
	\end{equation*}
	
	\begin{equation*}
		H_0: S(\rho_{\mathcal{S}}) \le \frac{A}{4 l_P^2}
	\end{equation*}
	
	This is the standard formulation of the Bekenstein Bound in terms of von Neumann entropy.
	
	\subsection{Alternative Hypothesis 1 (H1)}
	
	The Bekenstein bound is greater than or equal to the maximum classically accessible information of the system $\mathcal{S}$:
	
	\begin{equation*}
		H_1: I_B(\mathcal{S}) \ge I_C(\mathcal{S})
	\end{equation*}
	
	Substituting the definition of $I_B$:
	
	\begin{equation*}
		H_1: \frac{A}{4 l_P^2 \ln 2} \ge I_C(\mathcal{S})
	\end{equation*}
	
	If we consider a classical observer measuring a subsystem $\mathcal{A}$ with reduced density matrix $\rho_{\mathcal{A}}$, and the maximum classically accessible information is related to its von Neumann entropy, then:
	
	\begin{equation*}
		H_1: \frac{A}{4 l_P^2 \ln 2} \ge \frac{S(\rho_{\mathcal{A}})}{\ln 2} \implies S(\rho_{\mathcal{A}}) \le \frac{A}{4 l_P^2}
	\end{equation*}
	
	The subsystem $\mathcal{A}$ being measured must be contained within the region of area $A$.
	
	\subsection{Alternative Hypothesis 2 (H2) for Black Holes}
	
	For a black hole $\mathcal{B}$, the total quantum information is equal to the maximum classically accessible information:
	
	\begin{equation*}
		H_2: I_Q(\mathcal{B}) = I_C(\mathcal{B})
	\end{equation*}
	
	This hypothesis suggests that for black holes, all the quantum information can, in principle, be retrieved through classical measurements on the emitted radiation or the final state of the black hole. If we consider the black hole system to include the black hole itself and its Hawking radiation, and the classical information is obtained from measurements on the radiation, this hypothesis relates the total quantum entropy of the combined system to the classical information gained by an observer.
	
	\section{Meta-hypothesis: Black Hole Evolution}
	
	The meta-hypothesis posits that black holes at different stages of existence and dissolution exhibit all three properties. Let $\mathcal{B}(t)$ represent a black hole at time $t$ during its evolution.
	
	\begin{enumerate}
		\item \textbf{Bekenstein Bound and Total Quantum Information:} There exists a time interval during the black hole's existence such that $I_B(\mathcal{B}(t)) \ge I_Q(\mathcal{B}(t))$. This is expected to hold throughout the black hole's lifetime, as the Bekenstein Bound is believed to be a fundamental limit on the information content of any system within a given region.
		
		\item \textbf{Bekenstein Bound and Classically Accessible Information:} There exists a time interval during the black hole's existence such that $I_B(\mathcal{B}(t)) \ge I_C(\mathcal{B}(t))$. This should also hold, as the classically accessible information is a subset of the total information content and thus should also be bounded by the Bekenstein Bound.
		
		\item \textbf{Equality of Quantum and Classical Information for Black Holes:} There exists a stage in the black hole's existence (possibly at the end of its evaporation, or considering the entire process) where $I_Q(\mathcal{B}) = I_C(\mathcal{B})$. This is related to the resolution of the black hole information paradox, suggesting that all information about the black hole's formation and evolution is encoded in the outgoing radiation in a way that can be, in principle, classically accessed.
		
		\item \textbf{Observer Inaccessibility:} The meta-hypothesis further suggests that as long as the black hole has not totally evaporated, there will be information about it that is not classically accessible due to observer dependence and the nature of black holes.
		
		Let an observer $\mathcal{O}_1$ attempt to make optimal classical measurements to access information about a black hole $\mathcal{B}$. The process of making these measurements might involve interactions that alter the system or require the observer to be in a specific spacetime region.
		
		Consider an observer $\mathcal{O}_2$ who is far away from the black hole and attempts to gather information from the Hawking radiation. The information accessible to $\mathcal{O}_2$ at a given time might not be the complete quantum information of the black hole at that time.
		
		The statement about optimal classical measurements leading to the observer becoming classically inaccessible to others and consumed by the black hole can be related to scenarios where an observer attempting to gain information by falling into the black hole loses the ability to communicate with observers outside.
		
		Let $I_{Q, BH}(t)$ be the quantum information of the black hole at time $t$, and $I_{Q, rad}(t)$ be the quantum information of the radiation emitted up to time $t$. The total quantum information of the system (black hole + radiation) is conserved (assuming unitary evolution).
		
		Let $I_{C, rad}(t)$ be the classically accessible information obtained by an outside observer from the radiation up to time $t$.
		
		The meta-hypothesis suggests that for $t < t_{evaporation}$ (where $t_{evaporation}$ is the time of complete evaporation):
		
		\begin{itemize}
			\item $I_B(\mathcal{B}(t)) \ge I_Q(\mathcal{B}(t))$
			\item $I_B(\mathcal{B}(t)) \ge I_{C, rad}(t)$
			\item Possibly, $\lim_{t \to t_{evaporation}} I_Q(\mathcal{B}(t) + radiation(t)) = \lim_{t \to t_{evaporation}} I_{C, rad}(t)$
			\item There exists some $I_{private}$ about the black hole (e.g., information about its interior state) such that $I_{private} \subseteq I_Q(\mathcal{B}(t))$ and $I_{private}$ is not accessible to observers outside the event horizon through classical measurements on the radiation at that time.
		\end{itemize}
		
		The formal mathematical development of the meta-hypothesis regarding observer inaccessibility requires a more detailed framework involving quantum information theory in curved spacetime and the precise definition of "classically accessible" in such contexts. This could involve concepts like quantum entanglement between the black hole and the radiation, and the limitations imposed by the event horizon on information retrieval.
		
		One way to think about this is through the lens of complementarity. An observer falling into the black hole might access information that is not available to an outside observer, and vice versa. The descriptions of reality for these two observers might be complementary but not simultaneously accessible to a single classical observer.
		
		The statement about optimal classical measurements leading to the observer being consumed relates to the idea that to gain detailed classical information about the black hole near its horizon, an observer might need to interact with it strongly, potentially crossing the horizon and becoming part of the black hole system, thus losing the ability to communicate the information classically to the outside.
		
		This aspect of the meta-hypothesis touches upon the fundamental challenges in reconciling quantum mechanics and general relativity, particularly concerning the nature of observation and information in strong gravitational fields. A full mathematical formalization would likely require a theory of quantum gravity.
		
	\end{enumerate}
	
\end{document}}*}


\chapter{Planck Mass Black Hole conjectures 1}
	
	
	\begin{abstract}
		This thesis formalizes several hypotheses concerning the information content of physical systems, with a specific focus on black holes. We provide rigorous mathematical definitions for the Bekenstein bound, total quantum information, and maximum classically accessible information. Utilizing these definitions, we present the null and alternative hypotheses in precise mathematical terms. Furthermore, we develop a meta-hypothesis describing the information dynamics of black holes throughout their life cycle, considering the causal structure and the impact of energy/mass exchange. We operate within the framework of natural units where $k_B = c = \hbar = G = 1$.
	\end{abstract}
	
	\section{Definitions}
	
	\begin{definition}[Bekenstein Bound]
		Let $\mathcal{R}$ be a spatially bounded region with a maximal linear dimension $R$. For a system $\Sigma$ with total energy $E$ contained within $\mathcal{R}$, the Bekenstein bound $B(\mathcal{R})$ is defined as
		$$B(\mathcal{R}) = 2 \pi R E.$$
		For a Kerr-Newman black hole with mass $M$, angular momentum $J$, and charge $Q$, the area of the event horizon is
		$$A_{KN} = 4 \pi \left( M + \sqrt{M^2 - (J/M)^2 - Q^2} \right)^2 + 4 \pi (J/M)^2 = 4 \pi \left( 2 M^2 - Q^2 + 2 M \sqrt{M^2 - (J/M)^2 - Q^2} \right).$$
		The Bekenstein bound associated with the black hole, considering the black hole itself as the system within its event horizon (with an effective radius related to the horizon area), is numerically equal to the Bekenstein-Hawking entropy in natural units, $S_{BH} = A_{KN} / 4 = \pi \left( 2 M^2 - Q^2 + 2 M \sqrt{M^2 - (J/M)^2 - Q^2} \right)$.
	\end{definition}
	
	\begin{definition}[Total Quantum Information]
		The total quantum information of a system $\Sigma$ described by a density operator $\rho_\Sigma$ acting on a Hilbert space $\mathcal{H}_\Sigma$ is quantified by the von Neumann entropy
		$$I_Q(\rho_\Sigma) = S(\rho_\Sigma) = - \operatorname{Tr}_{\mathcal{H}_\Sigma}(\rho_\Sigma \ln \rho_\Sigma).$$
		This quantity measures the total information encoded in the quantum state, including both classical and quantum correlations.
	\end{definition}
	
	\begin{definition}[Maximum Classically Accessible Information]
		The maximum classically accessible information $I_C(\rho_\Sigma)$ obtainable from a quantum system described by a density operator $\rho_\Sigma$ by an observer at asymptotic infinity ($\mathcal{I}^+$) over the system's lifetime is given by the supremum of the Holevo information $\chi(\mathcal{E}) = S(\sum_i p_i \rho_i) - \sum_i p_i S(\rho_i)$ over all possible ensembles $\mathcal{E} = \{p_i, \rho_i\}$ consistent with $\rho_\Sigma$ and all possible positive operator-valued measures (POVMs) that the observer can perform. For a black hole, we denote this by $I_C^{\infty}(\rho_{BH})$.
	\end{definition}
	
	\section{Hypotheses}
	
	\begin{hypothesis}[Null Hypothesis ($H_0$): Bekenstein Bound and Total Quantum Information]
		For any physical system $\Sigma$ contained within a region $\mathcal{R}$ with Bekenstein bound $B(\mathcal{R})$, the Bekenstein bound is greater than or equal to the total quantum information of the system:
		$$H_0: B(\mathcal{R}) \ge I_Q(\rho_\Sigma).$$
		For a black hole with event horizon $\mathcal{H}$ described by a quantum state $\rho_{BH}$, this implies
		$$S(\rho_{BH}) \le \pi \left( 2 M^2 - Q^2 + 2 M \sqrt{M^2 - (J/M)^2 - Q^2} \right).$$
	\end{hypothesis}
	
	\begin{hypothesis}[Alternative Hypothesis 1 ($H_1$): Bekenstein Bound and Classically Accessible Information]
		For any physical system $\Sigma$ contained within a region $\mathcal{R}$ with Bekenstein bound $B(\mathcal{R})$, the Bekenstein bound is greater than or equal to the maximum classically accessible information of the system:
		$$H_1: B(\mathcal{R}) \ge I_C(\rho_\Sigma).$$
		For a black hole with event horizon $\mathcal{H}$ described by a quantum state $\rho_{BH}$, this implies
		$$I_C^{\infty}(\rho_{BH}) \le \pi \left( 2 M^2 - Q^2 + 2 M \sqrt{M^2 - (J/M)^2 - Q^2} \right).$$
	\end{hypothesis}
	
	\begin{hypothesis}[Alternative Hypothesis 2 ($H_2$): Information Retrieval from Black Holes]
		For black holes, the total quantum information of the black hole system is equal to the maximum classically accessible information obtainable from it by an observer at asymptotic infinity over its entire lifetime:
		$$H_2: I_Q(\rho_{BH}) = I_C^{\infty}(\rho_{BH}).$$
		This hypothesis suggests that all quantum information initially contained within the black hole is eventually encoded in the outgoing Hawking radiation in a form that can be fully decoded by an asymptotic observer, implying a unitary evolution of the black hole system.
	\end{hypothesis}
	
	\section{Meta-Hypothesis: Black Hole Evolution and Information}
	
	Consider a black hole formed from an initial quantum state $\rho_{initial}$ with von Neumann entropy $S(\rho_{initial})$. The black hole's state immediately after formation is $\rho_{BH}(t_{formation}^+)$ with mass $M_i$, angular momentum $J_i$, and charge $Q_i$, possessing a Bekenstein-Hawking entropy $S_{BH, i} = \pi \left( 2 M_i^2 - Q_i^2 + 2 M_i \sqrt{M_i^2 - (J_i/M_i)^2 - Q_i^2} \right)$.
	
	\begin{hypothesis}[Meta-Hypothesis]
		A black hole at different stages of existence and dissolution exhibits the following properties:
		\begin{enumerate}
			\item \textbf{Formation:} Immediately after formation, the Bekenstein-Hawking entropy of the newly formed black hole provides an upper bound on its total quantum information:
			$$S(\rho_{BH}(t_{formation}^+)) \le S_{BH, i}.$$
			Furthermore, the total quantum information is greater than or equal to the maximum classically accessible information:
			$$S(\rho_{BH}(t_{formation}^+)) \ge I_C^{\infty}(\rho_{BH}(t_{formation}^+)).$$
			
			\item \textbf{Hawking Radiation:} During the emission of Hawking radiation, the black hole evolves through a sequence of quantum states $\rho_{BH}(t)$ with decreasing mass $M(t)$ and Bekenstein-Hawking entropy $S_{BH}(t) = \pi \left( 2 M(t)^2 - Q(t)^2 + 2 M(t) \sqrt{M(t)^2 - (J(t)/M(t))^2 - Q(t)^2} \right)$. The emitted radiation, described by the quantum state $\rho_{rad}(t)$, carries away energy and potentially information. The maximum classically accessible information from the collected radiation up to time $t$, $I_C^{\infty}(\rho_{rad}(t))$, increases over time. The meta-hypothesis suggests that as long as the black hole has not completely evaporated:
			$$S(\rho_{BH}(t)) \le S_{BH}(t)$$
			and
			$$S(\rho_{BH}(t)) \ge I_C^{\infty}(\rho_{BH}(t)).$$
			The strict inequality $S(\rho_{BH}(t)) > I_C^{\infty}(\rho_{BH}(t))$ is expected due to the presence of the event horizon, which causally disconnects the interior of the black hole from the asymptotic observer. Optimal classical measurements by an external observer are limited by the causal structure and the principles of quantum mechanics, potentially leading to information loss or scrambling that hinders complete classical retrieval of the black hole's quantum state.
			
			\item \textbf{Evaporation:} As the black hole approaches complete evaporation, its mass tends towards a minimal value (e.g., Planck mass), and its Bekenstein-Hawking entropy tends towards a minimal value. If a stable remnant forms in a quantum state $\rho_{remnant}$, then its total quantum information is bounded by its Bekenstein-Hawking entropy (if defined), $S(\rho_{remnant}) \le S_{BH, final}$, and $S(\rho_{remnant}) \ge I_C^{\infty}(\rho_{remnant})$. If the black hole completely evaporates into radiation in a final quantum state $\rho_{final\_rad}$, and if the overall evolution is unitary, then the von Neumann entropy of the final radiation should equal the von Neumann entropy of the initial collapsing matter: $S(\rho_{final\_rad}) = S(\rho_{initial})$. The black hole information paradox arises from the apparent contradiction between Hawking's semiclassical calculations suggesting a thermal spectrum of radiation (leading to information loss) and the principle of unitarity in quantum mechanics. Hypothesis $H_2$ posits that $I_C^{\infty}(\rho_{final\_rad}) = S(\rho_{initial})$, resolving the paradox by suggesting that the information is indeed encoded in the radiation in a classically retrievable form, albeit perhaps highly scrambled. The meta-hypothesis suggests that at all stages of the process, the total quantum information of the combined system (black hole and radiation) is conserved, while the classically accessible information to any single asymptotic observer is constrained by the causal structure.
		\end{enumerate}
	\end{hypothesis}
	
	\section{Causal Structure and Information Boundaries}
	
	\begin{definition}[Causal Diamond]
		The causal diamond $D(\mathcal{S})$ of a spacetime region $\mathcal{S}$ is the set of all points $p$ such that every inextendible causal curve through $p$ intersects $\mathcal{S}$. For a black hole formed from collapsing matter and evaporating into radiation, the relevant spacetime region $\mathcal{S}$ can be considered the union of the initial collapsing matter and the final outgoing radiation reaching asymptotic infinity. The causal diamond is bounded by the past light cone of the endpoint of evaporation and the future light cone of the beginning of formation.
	\end{definition}
	
	For a black hole, the causal diamond is also naturally associated with the interior region bounded by its past and future event horizons. The past event horizon is the boundary of the region that will eventually fall into the black hole, and the future event horizon is the boundary of the region from which nothing can escape to future null infinity ($\mathcal{I}^+$).
	
	The meta-hypothesis suggests that the total quantum information associated with the black hole's formation and evaporation process is fundamentally encoded within the degrees of freedom residing within its causal diamond. The event horizons act as one-way membranes, affecting the accessibility of this information to observers located in different spacetime regions. Information about the initial state of the collapsing matter crosses the past event horizon and contributes to the black hole's internal quantum information. During evaporation, information is carried away by the Hawking radiation across the future event horizon. The scrambling of information by the black hole's dynamics implies a complex relationship between the infalling quantum states and the outgoing radiation.
	
	\section{Dynamics of Entropy and Information Flow}
	
	Consider a black hole undergoing a small change in its mass, charge, or angular momentum due to the absorption or emission of a quantum of energy. Let the black hole be described by a state $\rho_{BH}$ with Bekenstein-Hawking entropy $S_{BH}$. The absorption of a quantum with energy $\delta E$ (at infinity) will change the black hole's mass by $\delta M = \delta E$. The corresponding change in the Bekenstein-Hawking entropy $\delta S_{BH}$ can be calculated from the entropy formula.
	
	\begin{proposition}
		For a non-rotating, uncharged (Schwarzschild) black hole, a small change in mass $\delta M$ leads to a change in entropy $\delta S_{BH} = 8 \pi M \delta M$.
		\begin{proof}
			The Bekenstein-Hawking entropy of a Schwarzschild black hole is $S_{BH} = 4 \pi M^2$. Therefore, $\delta S_{BH} = \frac{d S_{BH}}{d M} \delta M = 8 \pi M \delta M$.
		\end{proof}
	\end{proposition}
	
	This proposition indicates that any energy exchange with the black hole is accompanied by a change in its Bekenstein-Hawking entropy, suggesting a connection between energy, entropy, and information. The meta-hypothesis posits that the total quantum information of the black hole evolves in tandem with its entropy. The absorption of a quantum increases the black hole's total quantum information, while the emission of Hawking radiation transfers some of this information to the external environment.
	
	\begin{remark}
		The precise mechanism by which the quantum information of the infalling matter is encoded in the outgoing Hawking radiation remains a central question in the black hole information paradox. Hypothesis $H_2$ suggests that this encoding is such that the information is ultimately recoverable by an asymptotic observer, implying a unitary evolution.
	\end{remark}
	
	\section{Black Hole Motion and Interactions}
	
	The interaction of a black hole with its environment, through absorption or emission, not only changes its internal state (mass, charge, angular momentum) but also affects its motion through spacetime due to the conservation of momentum. If a black hole emits a particle with four-momentum $\tensor{p}{^\mu_{out}}$, the black hole will recoil with an equal and opposite four-momentum $-\tensor{p}{^\mu_{out}}$ in its local rest frame. Similarly, the absorption of a particle with four-momentum $\tensor{p}{^\mu_{in}}$ will change the black hole's four-momentum by $+\tensor{p}{^\mu_{in}}$.
	
	\begin{proposition}
		The absorption of a quantum of information by a black hole increases its total quantum information by at least the Shannon entropy of the absorbed quantum state relative to the black hole's prior state.
		\begin{proof}
			Let the black hole be in a state with density operator $\rho_{BH}$ and the incoming quantum be in a state $\rho_{in}$. The combined system is initially in the state $\rho_{BH} \otimes \rho_{in}$. After absorption, the black hole transitions to a new state $\rho'_{BH}$. The change in the black hole's von Neumann entropy is $S(\rho'_{BH}) - S(\rho_{BH})$. The strong subadditivity of von Neumann entropy implies that the information gained by the black hole is related to the entropy of the incoming quantum. A rigorous proof would require a detailed model of the absorption process, which is beyond the scope of this formalization. However, intuitively, the black hole gains information about the absorbed quantum, increasing its total quantum information.
		\end{proof}
	\end{proposition}
	
	The meta-hypothesis suggests that these interactions are fundamental to the flow of information into and out of the black hole. While the Bekenstein-Hawking entropy provides a macroscopic measure of the black hole's information capacity, the microscopic details of how quantum information is processed and transferred remain an active area of research.
	
	\section{Conclusion}
	
	This thesis has provided a formal mathematical framework for discussing hypotheses related to black hole information. We have defined key concepts such as the Bekenstein bound, total quantum information, and maximum classically accessible information. The null and alternative hypotheses concerning the relationship between these quantities have been stated precisely. Furthermore, the meta-hypothesis offers a comprehensive view of the information dynamics of black holes throughout their life cycle, emphasizing the roles of formation, Hawking radiation, and evaporation. The discussion of the causal structure and the impact of energy/mass exchange provides further context for understanding the flow and accessibility of information associated with black holes. These formalizations lay the groundwork for future theoretical investigations aimed at resolving the black hole information paradox and achieving a deeper understanding of quantum gravity.
	


\chapter{Quantum Coherence Entropy}
}
	
	\maketitle
	
	\begin{abstract}
		This thesis formalizes the distinction between the total information content of a quantum system and the information that can be extracted through classical measurements. We employ the von Neumann entropy as the measure of total quantum information and the Shannon entropy of measurement outcomes for classically accessible information. The relative entropy of coherence is rigorously introduced as the primary quantifier of this fundamental difference, emphasizing the informational role of quantum coherence.
	\end{abstract}
	
	\section{Introduction}
	
	The information paradigm in quantum mechanics reveals a richer structure than its classical counterpart. Quantum systems can harbor information that is inherently quantum and inaccessible through standard classical measurements. The total information inherent in a quantum state is precisely quantified by the von Neumann entropy. However, when a quantum system is subjected to a classical measurement, the information gained is described by the Shannon entropy of the resulting probability distribution. The discrepancy between these two entropies unveils the quantum aspects of information, notably quantum coherence, which underpins the potential advantages of quantum technologies. This thesis provides a rigorous mathematical framework for quantifying this difference.
	
	\section{Total Quantum Information: Von Neumann Entropy}
	
	Consider a quantum system residing in a Hilbert space $\mathcal{H}$, described by a density operator $\rho$, which is a positive semi-definite operator with trace equal to one ($\rho \ge 0$, $\text{Tr}(\rho) = 1$). The total information content of this quantum state is quantified by the von Neumann entropy, defined as:
	\begin{equation}
		S(\rho) = -\text{Tr}(\rho \log_2 \rho)
		\label{eq:von_neumann_entropy}
	\end{equation}
	Here, $\log_2$ denotes the logarithm to the base 2, and the trace ($\text{Tr}$) is taken over the Hilbert space $\mathcal{H}$. The von Neumann entropy is basis-independent and serves as the quantum generalization of the Shannon entropy.
	
	\section{Classically Accessible Information: Shannon Entropy}
	
	Let the quantum system in state $\rho$ be measured with respect to an orthonormal basis $\mathcal{B} = \{|b_i\rangle\}_{i=1}^d$ of the $d$-dimensional Hilbert space $\mathcal{H}$. According to Born's rule, the probability of obtaining the measurement outcome corresponding to the basis state $|b_i\rangle$ is given by:
	\begin{equation}
		p_i = \langle b_i|\rho|b_i\rangle
		\label{eq:born_rule}
	\end{equation}
	The set of probabilities $\{p_i\}_{i=1}^d$ forms a probability distribution. The amount of information gained about the system from this specific measurement is quantified by the Shannon entropy of this distribution:
	\begin{equation}
		H(\mathcal{B}, \rho) = -\sum_{i=1}^d p_i \log_2 p_i = -\sum_{i=1}^d \langle b_i|\rho|b_i\rangle \log_2 \langle b_i|\rho|b_i\rangle
		\label{eq:shannon_entropy}
	\end{equation}
	The Shannon entropy depends on the chosen measurement basis $\mathcal{B}$.
	
	\section{Quantifying the Difference: Relative Entropy of Coherence}
	
	The discrepancy between the total quantum information $S(\rho)$ and the classically accessible information $H(\mathcal{B}, \rho)$ from a specific measurement basis $\mathcal{B}$ arises from quantum coherence. Coherence, with respect to the basis $\mathcal{B}$, is embodied in the off-diagonal elements of the density matrix $\rho$ when expressed in that basis. These off-diagonal elements contain quantum information that is not directly revealed by a measurement in $\mathcal{B}$.
	
	To isolate the incoherent part of the quantum state with respect to the basis $\mathcal{B}$, we define the dephased state $\rho_{\mathcal{B}}$ by removing the off-diagonal elements of $\rho$ in the basis $\mathcal{B}$:
	\begin{equation}
		\rho_{\mathcal{B}} = \sum_{i=1}^d \langle b_i|\rho|b_i\rangle |b_i\rangle\langle b_i| = \sum_{i=1}^d p_i |b_i\rangle\langle b_i|
		\label{eq:dephased_state}
	\end{equation}
	The von Neumann entropy of this dephased state is:
	\begin{equation}
		S(\rho_{\mathcal{B}}) = -\text{Tr}(\rho_{\mathcal{B}} \log_2 \rho_{\mathcal{B}}) = -\sum_{i=1}^d p_i \log_2 p_i = H(\mathcal{B}, \rho)
		\label{eq:entropy_dephased_state}
	\end{equation}
	This shows that the Shannon entropy of the measurement outcomes in the basis $\mathcal{B}$ is equal to the von Neumann entropy of the state dephased in that basis.
	
	The relative entropy of coherence of the state $\rho$ with respect to the basis $\mathcal{B}$ is formally defined as:
	\begin{equation}
		C_{rel}(\rho, \mathcal{B}) = S(\rho_{\mathcal{B}}) - S(\rho)
		\label{eq:relative_entropy_coherence}
	\end{equation}
	This quantity is non-negative, $C_{rel}(\rho, \mathcal{B}) \ge 0$, and quantifies the amount of quantum coherence present in $\rho$ relative to $\mathcal{B}$.
	
	\begin{proposition}
		The difference between the total information of a quantum state $\rho$ and the classically accessible information from a measurement in the basis $\mathcal{B}$ is given by the relative entropy of coherence with respect to that basis:
		$$C_{rel}(\rho, \mathcal{B}) = H(\mathcal{B}, \rho) - S(\rho)$$
		This difference represents the quantum information encoded in the coherences of the state $\rho$ relative to the measurement basis $\mathcal{B}$, which is not directly accessible through a measurement in that basis.
	\end{proposition}
	\begin{proof}
		From Equation \eqref{eq:relative_entropy_coherence} and Equation \eqref{eq:entropy_dephased_state}, we have:
		$$C_{rel}(\rho, \mathcal{B}) = S(\rho_{\mathcal{B}}) - S(\rho) = H(\mathcal{B}, \rho) - S(\rho)$$
		This directly shows that the relative entropy of coherence quantifies the difference between the classically accessible information (Shannon entropy of measurement outcomes) and the total quantum information (von Neumann entropy). The information lost or inaccessible through the measurement in the basis $\mathcal{B}$ is precisely the information contained within the quantum coherences relative to that basis.
	\end{proof}
	
	\section{Example: A Qubit}
	
	Consider a qubit in the pure state $|\psi\rangle = \alpha |0\rangle + \beta |1\rangle$, where $|\alpha|^2 + |\beta|^2 = 1$. The density operator is $\rho = |\psi\rangle\langle\psi| = \begin{pmatrix} |\alpha|^2 & \alpha \beta^* \\ \alpha^* \beta & |\beta|^2 \end{pmatrix}$ in the basis $\{|0\rangle, |1\rangle\}$. The von Neumann entropy of a pure state is $S(\rho) = 0$.
	
	Measurement in the computational basis $\mathcal{B} = \{|0\rangle, |1\rangle\}$ yields probabilities $p_0 = |\alpha|^2$ and $p_1 = |\beta|^2$. The Shannon entropy is $H(\mathcal{B}, \rho) = -|\alpha|^2 \log_2 |\alpha|^2 - |\beta|^2 \log_2 |\beta|^2$.
	
	The relative entropy of coherence is $C_{rel}(\rho, \mathcal{B}) = H(\mathcal{B}, \rho) - S(\rho) = -|\alpha|^2 \log_2 |\alpha|^2 - |\beta|^2 \log_2 |\beta|^2$. This quantity is non-zero if the qubit is in a superposition (i.e., $\alpha \beta^* \neq 0$), indicating the presence of coherence.
	
	\section{Conclusion}
	
	The quantification of the difference between total quantum information and classically accessible information is a cornerstone of quantum information theory. The relative entropy of coherence, $C_{rel}(\rho, \mathcal{B}) = H(\mathcal{B}, \rho) - S(\rho)$, provides a rigorous measure of this difference, highlighting the crucial role of quantum coherence in encoding information that transcends classical descriptions. While this thesis focused on a specific measure of coherence, it is important to note that other measures exist, and the ultimate limit on the classical information that can be extracted from a quantum state is given by the Holevo bound, which considers the encoding of classical information into quantum states. Understanding these distinctions is fundamental for advancing quantum technologies and our comprehension of the quantum nature of information.
	
\end{document}}*}


\part{Logic and Language}

\chapter{Algorithm Of Paradoxes}
}
	\maketitle
	
	\begin{abstract}
		This algorithmic thesis extends the investigation of simultaneous versus alternative failure to include paradoxes of decidability, alongside identity, consistency, and completeness. We hypothesize simultaneous failure (H7) against alternative failure (H8, null hypothesis).  Algorithmic paradox analysis across physical, computational, and logical domains demonstrates fundamental scenarios where paradoxes of identity, consistency, completeness, and decidability manifest concurrently, especially with non-locality, indeterminism, and reasoner dependence. This refutes the alternative failure hypothesis, reinforcing the necessity of frameworks accommodating simultaneous failures in non-classical and undecidable domains.
	\end{abstract}
	
	\section{Introduction: Paradoxes of Identity, Consistency, Completeness, and Decidability}
	
	Expanding upon the established correspondences—Locality $\leftrightarrow$ Reflexivity $\leftrightarrow$ Identity; Determinism $\leftrightarrow$ Transitivity $\leftrightarrow$ Consistency; Reasoner Independence $\leftrightarrow$ Monotonicity $\leftrightarrow$ Completeness—this thesis now incorporates \textbf{Decidability} linked to Completeness and Reasoner Independence.  Given that at least one of each tetrad must fail beyond classical domains, we ask: Can paradoxes of identity, consistency, completeness, and decidability fail simultaneously, or do they fail alternatively?  This thesis algorithmically explores this expanded question, focusing on paradoxes arising from failures in these properties, particularly in the context of non-locality, indeterminism, and reasoner dependence, highlighting the central role of decidability.
	
	\section{Definitions: Paradoxes of Identity, Consistency, Completeness, and Decidability}
	
	We refine definitions to include paradoxes of decidability, alongside identity, consistency, and completeness.
	
	\begin{definition}[Paradox of Identity]
		A paradox of identity arises when reflexivity is challenged, leading to ambiguous or contradictory self-sameness or distinctness. Physically, this relates to challenges to local objecthood, such as in black hole no-hair theorems or quantum indistinguishability. Self-reference remains a key source of identity paradoxes.
	\end{definition}
	
	\begin{definition}[Paradox of Consistency]
		A paradox of consistency emerges from violations of transitivity, resulting in logical or physical descriptions with inherent contradictions, or inconsistent computational states. Proof by contradiction and reductio ad absurdum are prototypical forms.
	\end{definition}
	
	\begin{definition}[Paradox of Completeness]
		A paradox of completeness occurs when monotonicity is undermined, leading to fundamentally incomplete system descriptions, or where completeness attempts induce new incompleteness or undecidability. Diagonalization arguments and Gödel's theorems exemplify completeness limitations.
	\end{definition}
	
	\begin{definition}[Paradox of Decidability]
		A paradox of decidability arises when the inherent limits of completeness and reasoner independence lead to undecidable propositions or problems within a system. This paradox highlights the tension between the desire for complete knowledge and the fundamental limits imposed by self-reference, complexity, or non-classical domains. Undecidability results, like the Halting Problem and undecidable logical theories, are core examples.
	\end{definition}
	
	
	\section{Hypotheses: Simultaneous vs. Alternative Failure (Expanded)}
	
	We expand our hypotheses to include decidability paradoxes.
	
	\begin{hypothesis}[H7: Simultaneous Failure Hypothesis (Expanded)]
		Paradoxes of identity, consistency, completeness, and decidability manifest simultaneously in fundamental physical, computational, and logical scenarios, indicating a collective breakdown of locality, determinism, and reasoner independence, leading to essential undecidability.
	\end{hypothesis}
	
	\begin{hypothesis}[H8: Alternative Failure Hypothesis (Null Hypothesis, Expanded)]
		Paradoxes of identity, consistency, completeness, and decidability manifest alternatively or in isolation, with failure in one area tending to preserve or necessitate the others, preventing simultaneous breakdown and maintaining a form of decidability where possible.
	\end{hypothesis}
	
	\section{Algorithmic Paradox Exploration (Expanded)}
	
	We expand algorithmic paradox exploration to explicitly address decidability.
	
	\subsection{Physical Paradoxes}
	
	\subsubsection{BELL\_PARADOX\_ANALYSIS Algorithm}
	\textbf{Input}: Experimental Bell inequality violation (non-locality).
	\begin{enumerate}
		\item \textit{Identity Paradox}: Entangled particles lack independent local identities; indistinguishability is central. \textbf{Manifest}.
		\item \textit{Consistency Paradox}: Local realism contradicts quantum predictions, implying inconsistency in local realistic descriptions. \textbf{Potential}.
		\item \textit{Completeness Paradox}: Bell's theorem shows local realism is incomplete for describing quantum correlations. \textbf{Manifest}.
		\item \textit{Decidability Paradox}:  Predicting outcomes beyond statistical correlations becomes undecidable within local realism; quantum mechanics embraces probabilistic decidability. \textbf{Emergent}.
		\item \textit{Output}: Simultaneous paradoxes of identity, completeness, and emergent decidability paradoxes due to non-locality.
	\end{enumerate}
	
	\subsubsection{BLACK\_HOLE\_PARADOX\_ANALYSIS Algorithm}
	\textbf{Input}: Event horizon locality constraint, No-Hair Theorem (identity), and Information Paradox.
	\begin{enumerate}
		\item \textit{Consistency Paradox}: Information loss via Hawking radiation vs. unitary evolution creates a fundamental inconsistency. \textbf{Manifest}.
		\item \textit{Completeness Paradox}: Event horizon's local description is incomplete for unitary information flow, limiting no-hair theorem's completeness. \textbf{Potential}.
		\item \textit{Identity Paradox}: Information scrambling and potential loss challenge identity preservation, questioning no-hair theorem universality. \textbf{Speculative}.
		\item \textit{Decidability Paradox}:  Fate of information falling into black hole becomes undecidable from outside event horizon based on local observations alone. \textbf{Event Horizon Undecidability}.
		\item \textit{Output}: Primarily consistency and decidability paradoxes; potential completeness and identity paradoxes linked to locality and identity theorems.
	\end{enumerate}
	
	\subsubsection{BOSE\_EINSTEIN\_CONDENSATE\_PARADOX\_ANALYSIS Algorithm}
	\textbf{Input}: Bose-Einstein Condensates, particle indistinguishability (identity challenge), and emergent statistical behavior.
	\begin{enumerate}
		\item \textit{Identity Paradox}:  Particles in BECs lose individual identity, becoming fundamentally indistinguishable, challenging classical particle identity. \textbf{Manifest}.
		\item \textit{Consistency Paradox}:  Statistical mechanics consistent within its domain but fundamentally differs from classical mechanics regarding identity and determinism. \textbf{Domain-Specific Consistency}.
		\item \textit{Completeness Paradox}:  Description complete within quantum statistical mechanics but incomplete from a classical deterministic trajectory perspective. \textbf{Domain-Specific Completeness}.
		\item \textit{Decidability Paradox}: Predicting individual particle behavior becomes undecidable; statistical predictions become decidable within quantum statistical framework. \textbf{Statistical Decidability}.
		\item \textit{Output}: Identity paradox manifest; consistency, completeness, and decidability are domain-specific, highlighting identity's role in defining physical descriptions and decidability limits.
	\end{enumerate}
	
	
	\subsection{Computational Paradoxes}
	
	\subsubsection{QUANTUM\_COMPUTATION\_PARADOX\_ANALYSIS Algorithm}
	\textbf{Input}: Computational non-locality via entanglement and superposition, exceeding classical computational limits.
	\begin{enumerate}
		\item \textit{Identity Paradox}: Superposition states challenge definite computational state identity, blurring classical state distinctions, leading to quantum state identity paradoxes. \textbf{Potential}.
		\item \textit{Consistency Paradox}: Probabilistic outcomes introduce consistency questions compared to deterministic classical computation. \textbf{Speculative}.
		\item \textit{Completeness Paradox}: Quantum computation expands computational power beyond classical Turing completeness for specific problem classes. \textbf{Limited}.
		\item \textit{Decidability Paradox}: Quantum computation potentially renders classically undecidable problems decidable, or alters decidability boundaries. \textbf{Decidability Shift}.
		\item \textit{Output}: Identity, potential consistency, and decidability paradoxes suggested by computational non-locality; limited completeness paradox in terms of computational power, but significant decidability shifts.
	\end{enumerate}
	
	
	\subsubsection{UNDECIDABILITY\_PARADOX\_ANALYSIS Algorithm}
	\textbf{Input}: Formal systems, self-reference, diagonalization, and Halting Problem (prototypical undecidability).
	\begin{enumerate}
		\item \textit{Consistency Paradox}: Gödel's theorems: consistency in sufficiently strong systems implies incompleteness and thus undecidable propositions. \textbf{Manifest}.
		\item \textit{Completeness Paradox}: Achieving completeness in formal systems leads to inconsistency; completeness inherently limited by undecidability. \textbf{Manifest}.
		\item \textit{Identity Paradox}: Self-reference, diagonalization, and undecidability arguments rely on constructing self-identical yet paradoxical entities, central to undecidability proofs. \textbf{Central Role}.
		\item \textit{Decidability Paradox}: Halting Problem and related undecidability results are direct manifestations of decidability paradoxes, showing inherent limits to algorithmic decidability in complete and consistent systems. \textbf{Core Undecidability}.
		\item \textit{Output}: Simultaneous consistency, completeness, and decidability paradoxes fundamentally linked via self-reference and identity paradoxes, exemplified by undecidability theorems.
	\end{enumerate}
	
	
	\subsection{Logical Paradoxes}
	
	\subsubsection{PARACONSISTENCY\_PARADOX\_ANALYSIS Algorithm}
	\textbf{Input}: Logical non-locality/reasoner dependence; contradiction tolerance, and weakened completeness.
	\begin{enumerate}
		\item \textit{Consistency Paradox}: Paraconsistent logics tolerate contradictions, challenging classical consistency, but managing paradoxes. \textbf{Mitigated/Explored}.
		\item \textit{Completeness Paradox}: Completeness is often traded off or redefined in paraconsistent logics to maintain non-triviality in presence of contradictions. \textbf{Trade-off}.
		\item \textit{Identity Paradox}: Context-dependent truth and contradiction tolerance may blur logical distinctions and identities, requiring nuanced identity criteria. \textbf{Possible Contextual}.
		\item \textit{Decidability Paradox}: Decidability properties in paraconsistent logics are often more complex than classical logic; some may sacrifice decidability to handle contradictions. \textbf{Complex Decidability}.
		\item \textit{Output}: Consistency paradox managed; completeness and decidability traded-off or redefined; contextual identity issues explored in contradiction-tolerant logics.
	\end{enumerate}
	
	
	\subsubsection{LIAR\_PARADOX\_ANALYSIS Algorithm}
	\textbf{Input}: Logical self-reference, Tarski's Undefinability Theorem, and semantic undecidability.
	\begin{enumerate}
		\item \textit{Consistency Paradox}: Liar's paradox directly generates logical contradiction, challenging consistency at the semantic level. \textbf{Manifest}.
		\item \textit{Completeness Paradox}: Tarski's theorem demonstrates fundamental limits to semantic completeness due to self-reference and potential inconsistency. \textbf{Fundamental Limit}.
		\item \textit{Identity Paradox}: Self-referential truth predicates challenge the identity of truth values and semantic categories, at the heart of semantic paradoxes. \textbf{Semantic Core}.
		\item \textit{Decidability Paradox}: Truth in sufficiently rich semantic systems becomes undecidable, formalized by Tarski's theorem, showing semantic undecidability. \textbf{Semantic Undecidability}.
		\item \textit{Output}: Simultaneous consistency, completeness, and decidability paradoxes fundamentally arising from self-referential identity paradoxes, formalized by Tarski's theorem, highlighting semantic undecidability.
	\end{enumerate}
	
	\subsubsection{RUSSELL\_PARADOX\_ANALYSIS Algorithm}
	\textbf{Input}: Naive Set Theory and Universal Set construction (identity paradox), leading to undecidability in set membership.
	\begin{enumerate}
		\item \textit{Identity Paradox}: Russell's paradox challenges the identity of sets and membership, particularly the universal set, as a core identity paradox. \textbf{Core Identity Paradox}.
		\item \textit{Consistency Paradox}: Naive set theory becomes inconsistent due to Russell's paradox, requiring axiomatic restrictions for consistency. \textbf{Inconsistency of Naive System}.
		\item \textit{Completeness Paradox}:  Axiomatic set theories (e.g., ZFC) trade completeness for consistency, limiting the scope of set comprehension to avoid paradoxes. \textbf{Completeness Restriction for Consistency}.
		\item \textit{Decidability Paradox}: Set membership in sufficiently rich set theories becomes undecidable (related to undecidability in arithmetic and Gödel's theorems), stemming from the restrictions imposed to avoid identity-driven inconsistencies. \textbf{Set Membership Undecidability}.
		\item \textit{Output}: Simultaneous identity, consistency, and decidability paradoxes, with completeness being restricted to resolve identity-driven inconsistency, leading to undecidability in set theory.
	\end{enumerate}
	
	
	\section{Discussion: Intertwined Failures and Undecidability as Emergent Property}
	
	Algorithmic paradox exploration now strongly indicates intertwined failures of identity, consistency, completeness, \textbf{and decidability}, especially in non-classical and self-referential scenarios.  Decidability emerges as a crucial aspect intertwined with the other paradoxes.
	
	\begin{itemize}
		\item \textit{Simultaneous Paradoxes as Dominant Mode}: Bell scenario, Black Hole Paradox, Bose-Einstein Condensates, Undecidability, Liar's Paradox, Russell's Paradox consistently demonstrate simultaneous challenges across all four paradox types.
		\item \textit{Identity Paradox as Foundational}: Identity paradoxes (self-reference, indistinguishability, no-hair theorem challenges) frequently underpin or exacerbate consistency, completeness, \textbf{and decidability} issues, acting as a foundational source of paradox.
		\item \textit{Completeness vs. Consistency Trade-off and Decidability Limits}: Undecidability, Liar's, and Russell's paradoxes highlight the fundamental trade-off between completeness and consistency, directly resulting in inherent decidability limits.  Attempts to restore consistency or completeness often impact decidability.
		\item \textit{Contextual Mitigation and Decidability Complexity}: Paraconsistent Logics and domain-specific descriptions (BECs) illustrate navigation of paradoxes through contextualization, often leading to more complex or limited decidability properties compared to classical systems.
		\item \textit{Prototypical Paradox Forms and Undecidability Generation}: Proof by contradiction, reductio ad absurdum, diagonalization, and self-reference (especially via Tarski's theorem) are prototypical algorithmic forms not only for constructing identity, consistency, and completeness paradoxes, but also for generating and revealing inherent undecidability. These methods converge on conditions that simultaneously challenge identity, consistency, completeness, \textbf{and decidability}.
		\item \textit{Undecidability as Emergent Property}: Undecidability emerges not as an isolated failure, but as a consequence of the intertwined breakdown of identity, consistency, and completeness, particularly in systems exhibiting non-locality, indeterminism, reasoner dependence, and self-reference.
	\end{itemize}
	
	Against H8 (Alternative Failure), and strongly supporting H7 (Simultaneous Failure), fundamental limits in physical, computational, and logical domains reveal intertwined failures of identity, consistency, completeness, and decidability. These failures are not merely interconnected but often simultaneously manifest, particularly when non-locality, indeterminism, reasoner dependence, and self-reference are involved, with undecidability as a key emergent property.
	
	\section{Conclusion: Refuting Alternative Failure Hypothesis and Embracing Simultaneous Paradoxes of Identity, Consistency, Completeness, and Decidability}
	
	This algorithmic thesis rigorously investigated the question of simultaneous versus alternative failure, expanding to include paradoxes of decidability alongside identity, consistency, and completeness. Algorithmic exploration across physical, computational, and logical domains robustly refutes the Alternative Failure Hypothesis (H8), providing strong support for the Simultaneous Failure Hypothesis (H7).
	
	Paradoxes arising from non-locality (Bell), locality constraints and identity theorems (Black Holes, No-Hair), quantum indistinguishability (BECs), computational limits (Quantum Computation, Undecidability), logical self-reference (Liar's, Russell's, Tarski's), and contradiction tolerance (Paraconsistency) consistently reveal intertwined challenges to identity, consistency, completeness, \textbf{and decidability}. These are not isolated failures but fundamentally linked and simultaneously manifesting paradoxes, especially in non-classical and self-referential contexts, with undecidability as a significant emergent consequence.
	
	Future research must prioritize developing integrated frameworks in physics, computation, and logic that explicitly accommodate simultaneous paradoxes of identity, consistency, completeness, \textbf{and decidability}. Moving beyond classical assumptions of universally preserved identity, consistency, completeness, or decidability necessitates embracing non-classical theories and formalisms that can effectively model and leverage the intertwined breakdown of these fundamental properties in complex, non-local, and self-referential domains. The convergence of paradox construction methods on conditions challenging all four properties simultaneously underscores the fundamental and interconnected nature of their failure as we expand our understanding beyond classical limits, with undecidability as a central challenge and emergent property.

\end{document}}*}


\chapter{Algorithmic Decidability}
}

Theorem 1

For all theories T in standard formalization.

T is essentially undecidable if and only if T is consistent and:

for all T\_e at least one of the following properties is false (T\_e is
consistent, T\_e is complete, T\_e is an extension of T, T\_e has the
same constants as T, T\_e is recursively enumerably axiomatizable).

This implies

For all theories T in standard formalization.

T is essentially undecidable if and only if T is consistent and:

There exists T\_e (\\
T\_e is paraconsistent,

\begin{quote}
T\_e is complete

T\_e is an extension of T

T\_e has the same constants as T

T\_e is recursively enumerably axiomatizable
\end{quote}

).

For all theories T in standard formalization.

T is essentially undecidable if and only if T is consistent and:

There exists T\_e (\\
T\_e is consistent,

\begin{quote}
T\_e is paracomplete

T\_e is an extension of T

T\_e has the same constants as T

T\_e is recursively enumerably axiomatizable
\end{quote}

).

For all theories T in standard formalization.

T is essentially undecidable if and only if T is consistent and:

There exists T\_e (\\
T\_e is consistent,

\begin{quote}
T\_e is complete

T\_e is a non-conservative extension of T

T\_e has the same constants as T

T\_e is recursively enumerably axiomatizable
\end{quote}

).

For all theories T in standard formalization.

T is essentially undecidable if and only if T is consistent and:

There exists T\_e (\\
T\_e is consistent,

\begin{quote}
T\_e is complete

T\_e is an extension of T

T\_e has the different constants than T

T\_e is recursively enumerably axiomatizable
\end{quote}

).

For all theories T in standard formalization.

T is essentially undecidable if and only if T is consistent and:

There exists T\_e (\\
T\_e is consistent,

\begin{quote}
T\_e is complete

T\_e is an extension of T

T\_e has the same constants as T

T\_e is non-axiomatizable
\end{quote}

).

Tarskian Decidability -\textgreater{} Recursive Axiomatizability

Einstein Univocality/Categoricity -\textgreater{} Completeness

Tarskian undecidability and completeness \textless-\textgreater{}
non-axiomatizability;

Tarskian undecidability and univocality \textless-\textgreater{}
non-axiomatizability.

Tarskian decidability and completeness \textless-\textgreater{}
axiomatizability;

Tarskian decidability and completeness \textless/\textgreater{}
non-axiomatizability;

non-(Tarskian decidability) or non-completeness \textless/\textgreater{}
non-axiomatizability.

We care whether a theory or language is algorithmically decidable vs
non-algorithmically decidable.

A decision procedure for T is an automated or mechanical procedure that
takes some input and produces binary or bivalent output.

A decision algorithm for T is an automated or mechanical procedure that
takes some input and produces binary or bivalent outputs with only
finite resources such as space or time; there exists a decision
algorithm for all Type 1 languages that depends only on the
length-increasing property of recursive language grammars
(monotonicity?).

A non-algorithmically decidable problem is a semi-decidable problem.

Every decidable language is recursively axiomatizable. Every
algorithmically decidable language is finitely axiomatizable.

Not every axiomatizable language is decidable.

Not every finitely axiomatizable language is decidable.

There exists non-bivalent languages.

A non-bivalent procedure is a procedure that takes some input or inputs
and produces non-binary or many-valued output or outputs.

A many-valued decision algorithm for T is an algorithm that halts with
true, false, or at least one distinct other value such as possible,
undefined, indeterminate, unknown, neither true nor false, both true and
false.

\end{document}}*}


\chapter{Categorical Metalanguages}
}
	
	\maketitle
	
	\begin{abstract}
		\RaggedRight % Apply RaggedRight to the abstract
		This document formalizes key components of a meta-semantic framework capable of accommodating non-classical logical properties, drawing upon concepts introduced in the "Meta-Semanticum Universalis" papers and detailed in the "Comprehensive Study of Paraconsistency." We define the space of meta-properties ($\MetaProps$), context parameters ($\ContextParams$), and the set of permissible meta-languages ($\MetaLang$). The core of the formalization lies in defining the non-Boolean Interpretation Function ($\InterpFunc$) and the structure of its value space ($\ValueSpace$), drawing on multi-valued and relational semantics from paraconsistent logic. We then address the challenge of formulating Relation Principles ($\RelPrinciples$) and the Evaluation System ($\EvalSystem$) when the meta-language itself is non-classical (paraconsistent and paracomplete), proposing approaches based on non-classical consequence relations and multi-criteria aggregation. This formalization aims to provide a more rigorous foundation for the concepts introduced in the Meta-Semantic Council, addressing the critiques raised by Logicus Dubitans and supporting the paradoxical synthesis proposed by Gemini. Additionally, this document formalizes the four extremal logics ($L^0_n$, $\LT$, $\LF$, and $L^\infty_n$) that define the vertices of the diamond structure of logical extensions within a specific category of logics, introducing an alternative notation based on characteristic sequent forms ($L^{\vdash}_n$, $L^{\vdash\Delta}_n$, $L^{\Gamma\vdash}_n$, and $L^{\Gamma\vdash\Delta}_n$), highlighting their unique properties and relationships within that logical space.
	\end{abstract}
	
	% Added Table of Contents for better navigation with the new Part
	\tableofcontents
	
	\part{Formalizing Non-Classical Meta-Semantic Structures} % Existing content as Part 1
	
	\section{Introduction}
	The exploration of universal meta-semantics, particularly those capable of describing and evaluating systems with non-classical properties like paraconsistency and paracompleteness, necessitates a formal framework that transcends classical Boolean assumptions. The "Meta-Semanticum Universalis" papers introduced a conceptual structure involving meta-properties, context, interpretation functions, and relation principles, but were critiqued for lacking formal specification, especially regarding the non-Boolean value space and the behavior of the framework when the meta-language itself is non-classical. This document aims to provide a more rigorous formalization of these elements, integrating insights from the study of paraconsistent and paracomplete logics to define the structure of the non-classical semantic space and the principles governing relations and evaluation within such a framework.
	
	\section{Core Components of the Meta-Semantic Framework}
	
	We begin by formalizing the foundational sets of the meta-semantic framework, based on the definitions from the "Meta-Semanticum Universalis: Adumbratio Formalis."
	
	\begin{definition}[Set of Meta-Properties ($\MetaProps$)]
		The set of meta-properties is defined as $$\MetaProps = \{ \Saf, \Sec, \Comp, \Paracons, \Paracomp \}$$. These represent fundamental characteristics relevant to the analysis of formal systems.
	\end{definition}
	
	\begin{definition}[Set of Context Parameters ($\ContextParams$)]
		The set of context parameters is a set of variables and structures representing the contextual factors influencing the interpretation of meta-properties. We denote this set as $\ContextParams = \{ D, T, R, ML, S, \dots \}$, where $D$ is the application domain, $T$ is the threat model, $R$ is resource limits, $ML$ is the meta-language perspective, $S$ is the system development stage, and $\dots$ represent other relevant parameters.
	\end{definition}
	
	\begin{definition}[Set of Permissible Meta-Languages ($\MetaLang$)]
		The set of permissible meta-languages represents the different logical frameworks from which a meta-language can be drawn. We define $\MetaLang$ based on the properties of paraconsistency and paracompleteness:
		\begin{itemize}[nosep] % nosep removes extra vertical space between items
			\item $\MLClass$: Classical logic (neither paraconsistent nor paracomplete).
			\item $\MLParacons$: A paraconsistent logic that is not paracomplete.
			\item $\MLParacomp$: A paracomplete logic that is not paraconsistent.
			\item $\MLBoth$: A logic that is both paraconsistent and paracomplete.
		\end{itemize}
		The choice of meta-language $ML \in \MetaLang$ is a parameter within $\ContextParams$.
	\end{definition}
	
	\section{Formalizing the Non-Classical Interpretation Function and Value Space}
	
	The Interpretation Function $\InterpFunc$ maps a meta-property, a context, and a meta-language to a value in a non-Boolean space $\ValueSpace$. Formalizing $\ValueSpace$ and the behavior of $\InterpFunc$ is crucial for a rigorous non-classical meta-semantics.
	
	\begin{definition}[Value Space ($\ValueSpace$)]
		The Value Space $\ValueSpace$ is a set equipped with a structure capable of representing degrees of truth, falsity, inconsistency (truth gluts), and incompleteness (truth gaps). $\ValueSpace$ is not restricted to the classical Boolean values $\{\text{Verum}, \text{Falsum}\}$. Examples of structures for $\ValueSpace$ include:
		\begin{itemize}[wide, labelwidth=!, labelindent=0pt, before=\RaggedRight, after=\RaggedRight] % Apply RaggedRight to list items
			\item Multi-valued lattices (e.g., the lattice underlying FDE, with values $\{T, F, B, N\}$).
			\item Structures derived from relational semantics (e.g., subsets of worlds in a Routley-Meyer frame or impossible worlds semantics).
			\item Intervals or lattices used in fuzzy logic (e.g., $[0,1]$ with appropriate operators, or lattices of annotations).
			\item Structures inspired by quantum logic (e.g., elements of an orthomodular lattice, potentially with additional structure to capture paraconsistent/paracomplete aspects).
			\end{itemize}
				The specific structure of $\ValueSpace$ depends on the chosen semantic framework for the meta-language.
				\end{definition}
					
					\begin{definition}[Interpretation Function ($\InterpFunc$)]
						The Interpretation Function is a function $\InterpFunc: \MetaProps \times \ContextParams \times \MetaLang \rightarrow \ValueSpace$. For a given meta-property $m \in \MetaProps$, context $c \in \ContextParams$, and meta-language $ml \in \MetaLang$, $\InterpFunc(m, c, ml)$ yields a value in $\ValueSpace$ representing the degree or status of meta-property $m$ in context $c$ as interpreted from the perspective of meta-language $ml$.
					\end{definition}
					
					\begin{remark}
						\RaggedRight % Apply RaggedRight to the remark
						The structure of $\ValueSpace$ and the definition of $\InterpFunc$ must be consistent with the logical properties of the meta-language $ml$. If $ml$ is paraconsistent, $\ValueSpace$ must support truth gluts. If $ml$ is paracomplete, $\ValueSpace$ must support truth gaps. If $ml$ is both, $\ValueSpace$ must support both.
						\end{remark}
							
							\section{Formalizing Relation Principles in a Non-Classical Meta-Language}
							
							The Relation Principles ($\RelPrinciples$) describe the formal relationships between meta-properties. When the meta-language is non-classical, these principles cannot rely on classical implication or negation. They must be formulated using the consequence relation ($\entails_{ml}$) of the chosen meta-language $ml$.
							
							\begin{definition}[Relation Principles ($\RelPrinciples$)]
								The set of Relation Principles $\RelPrinciples$ is a set of formal statements expressed in the language of the chosen meta-language $ml \in \MetaLang$, describing constraints or relationships between the interpretations of meta-properties. For a given $ml$, $\RelPrinciples_{ml}$ is a set of formulas or sequents in the language of $ml$.
								\end{definition}
									
									The statements in $\RelPrinciples_{ml}$ involve terms representing the interpreted values of meta-properties, e.g., $\InterpFunc(m, c, ml)$. The logical structure of these statements is governed by the rules of $ml$.
									
									\begin{remark}[Example: Formulating $RP_1$ in a Paraconsistent ML]
										\RaggedRight % Apply RaggedRight to the remark
										Consider $RP_1$: "Existence of Contexts where Safety is High and Security is Low/High Coexist - Non-Classical Implication." In a paraconsistent meta-language ($ml = \MLParacons$), the classical implication $(\dots \longrightarrow \dots)$ is not explosive. We could formulate a principle like:
										$\exists c \in \ContextParams . (\InterpFunc(\Saf, c, \MLParacons) \approx \text{High} \conjunction \InterpFunc(\Sec, c, \MLParacons) \approx \text{Low}) \conjunction (\InterpFunc(\Saf, c, \MLParacons) \approx \text{High} \conjunction \InterpFunc(\Sec, c, \MLParacons) \approx \text{High})$
										This statement, interpreted in the semantics of $\MLParacons$, can be true in some models, reflecting the tolerance for contradictions or coexisting properties. The consequence relation $\entails_{\MLParacons}$ would not allow deriving arbitrary formulas from this statement.
										\end{remark}
											
											\begin{remark}[Example: Formulating $RP_2$ in a Paracomplete ML]
												\RaggedRight % Apply RaggedRight to the remark
												Consider $RP_2$: "Negation of Classical Implication from Completeness to Safety - Context-Dependence." In a paracomplete meta-language ($ml = \MLParacomp$), the law of excluded middle may fail, and negation might behave non-classically. A formulation might involve stating the non-derivability of a classical-like implication:
												$\notentails_{\MLParacomp} \forall c \in \ContextParams . (\InterpFunc(\Comp, c, \MLParacomp) \approx \text{High} \implication_{\text{classical}} \InterpFunc(\Saf, c, \MLParacomp) \approx \text{High})$
												Here, $\implication_{\text{classical}}$ would represent a classical implication defined within the paracomplete logic (if possible), or the statement could directly assert the failure of the classical consequence relation for this inference. The non-derivability ($\notentails_{\MLParacomp}$) reflects the paracomplete nature – the statement is neither provable nor refutable in all contexts.
												\end{remark}
													
													The formalization of $\RelPrinciples$ requires carefully defining the logical language and consequence relation of each $ml \in \MetaLang$ and translating the conceptual principles into well-formed formulas or sequents within those logics.
													
													\section{Formalizing the Evaluation System}
													
													The Evaluation System $\EvalSystem$ maps a formal system, a context, and a meta-language to a multi-criteria evaluation profile. This evaluation must integrate the interpreted values of meta-properties from $\ValueSpace$, potentially handling inconsistent or incomplete information.
													
													\begin{definition}[Evaluation System ($\EvalSystem$)]
														The Evaluation System is a function $\EvalSystem: L \times \ContextParams \times \MetaLang \times \MetaProps \rightarrow \mathcal{E}$, where $L$ is a formal system (e.g., a logical calculus, a computational model), and $\mathcal{E}$ is a structured space of multi-criteria evaluation profiles. An evaluation profile in $\mathcal{E}$ for a system $L$ in context $c$ using meta-language $ml$ is a collection of interpreted values for each meta-property:
														$\EvalSystem(L, c, ml, \MetaProps) = \{ (\Saf, \InterpFunc(\Saf, c, ml)), (\Sec, \InterpFunc(\Sec, c, ml)), \dots, (\Paracomp, \InterpFunc(\Paracomp, c, ml)) \}$
														The structure of $\mathcal{E}$ is a subset of $(\MetaProps \times \ValueSpace)^{\MetaProps}$, representing the profile of interpreted meta-property values.
														\end{definition}
															
															\begin{remark}
																\RaggedRight % Apply RaggedRight to the remark
																The "multi-criteria, non-reducible to a single value" aspect of $\mathcal{E}$ is captured by the fact that the output is a profile of values from $\ValueSpace$ for each meta-property, not a single aggregated value. Combining or comparing these profiles would require additional aggregation functions or preference relations defined over $\mathcal{E}$, which themselves might need to be defined within a non-classical meta-language if handling inconsistent or incomplete profiles.
																\end{remark}
																	
																	\section{Connecting to Categorical Frameworks and Dualities}
																	
																	The "Categorical Framework for Logical Systems" introduces \textbf{DualCat}, a category where objects are categories of logics and morphisms are dualities. This perspective can inform the meta-semantic framework in several ways:
																	
																	\begin{itemize}[wide, labelwidth=!, labelindent=0pt, before=\RaggedRight, after=\RaggedRight] % Apply RaggedRight to list items
																		\item The different meta-languages in $\MetaLang$ (e.g., $\MLClass$, $\MLParacons$, $\MLParacomp$, $\MLBoth$) can be viewed as objects in a category of logics. The relationships between them (e.g., extensions, translations, dualities) can be described by morphisms in such a category.
																		\item Dualities between logics (morphisms in \textbf{DualCat}) might induce structural relationships or transformations between the corresponding Value Spaces ($\ValueSpace$) used for interpreting meta-properties when those logics serve as meta-languages.
																		\item The paradoxical synthesis where MSU is simultaneously refuted and proven from different logical perspectives (classical vs. non-classical) can be interpreted through the lens of different categories of logics and the potential failure of functors or translations between them to preserve all properties.
																	\end{itemize}
																	
																	\section{Conclusion}
																	
																	\RaggedRight % Apply RaggedRight to the conclusion
																	This document has provided a preliminary formalization of key components of a non-classical meta-semantic framework, drawing upon the conceptual structure of the "Meta-Semanticum Universalis" papers and the semantic diversity of paraconsistent and paracomplete logics. We have formalized the sets of meta-properties, context parameters, and meta-languages, and defined the non-Boolean ValueSpace ($\ValueSpace$) and the Interpretation Function ($\InterpFunc$) mapping to it. Crucially, we have outlined how Relation Principles ($\RelPrinciples$) and the Evaluation System ($\EvalSystem$) must be formulated using the logical machinery of a potentially non-classical meta-language, moving beyond classical assumptions about implication, negation, and aggregation. This formalization addresses some of the critiques regarding the lack of rigor and provides a foundation for further development of a meta-semantic theory capable of analyzing and evaluating complex systems from diverse logical perspectives, including those that tolerate inconsistency and incompleteness. The connection to categorical frameworks and dualities suggests avenues for understanding the relationships between different meta-language perspectives.
																	
																	% --- Start of Integrated Content ---
																	\part{The Diamond of Logical Extensions: The Four Extremal Logics within a Category $D_n$} % New Part Title
																	
																	\section{Introduction}
																	
																	We formalize the foundational layer of a hierarchy of logical systems, exploring how logical systems are shaped by the metalanguages used to define them, leading to a landscape of extensions visualized as a diamond. Within a specific category of logics, denoted $D_n$, this diamond has four extremal logics defining its vertices: the Minimum Non-Trivial Object Language ($L^0_n$ or $L^{\vdash}_n$), the Terminal Object Language ($L^\infty_n$ or $L^{\Gamma\vdash\Delta}_n$), the Logic of Truth ($\LT$ or $L^{\vdash\Delta}_n$), and the Logic of Falsity ($\LF$ or $L^{\Gamma\vdash}_n$). This part focuses on defining these extremal points within the context of a given category $D_n$, introducing a notation based on characteristic sequent forms to highlight their properties.
																	
																	\section{The Four Extremal Logics of a Diamond $D_n$}
																	
																	We formalize four extremal logical systems that define the vertices of the diamond structure within a specific category of logics $D_n$. These logics represent limiting cases in the landscape of formal systems based on their minimal or maximal properties concerning syntax, structure, and truth/falsity conditions within that category. We introduce a notation using characteristic sequent forms as superscripts to emphasize the primary structural or inferential focus of each extremal logic.
																	
																	\begin{itemize}
																		\item $L^0_n$ or $L^{\vdash}_n$: The Minimum Non-Trivial Object Language.
																		\item $L^\infty_n$ or $L^{\Gamma\vdash\Delta}_n$: The Terminal Object Language.
																		\item $\LT$ or $L^{\vdash\Delta}_n$: The Logic of Truth.
																		\item $\LF$ or $L^{\Gamma\vdash}_n$: The Logic of Falsity.
																	\end{itemize}
																	
																	Crucially, $\LT$ ($L^{\vdash\Delta}_n$) and $\LF$ ($L^{\Gamma\vdash}_n$) are conceived as specific logical extensions of the Minimum Non-Trivial Object Language $L^0_n$ ($L^{\vdash}_n$), arising from the reflection of $L^0_n$ by particular metalanguages within $D_n$ that impose the conditions of universal truth or universal falsity, respectively.
																	
																	\subsection{Minimum Non-Trivial Object Language ($L^0_n$ or $L^{\vdash}_n$)}
																	
																	The Minimum Non-Trivial Object Language ($L^0_n$ or $L^{\vdash}_n$) is the initial object in the category $D_n$. It is the most minimal formal system within $D_n$ that is non-trivial (i.e., its language is non-empty, and it has at least one formula or sequent form). It is designed to be reflectable by a wide range of metalanguages relevant to $D_n$. It serves as the base point of the diamond for $D_n$, representing the most basic structure involving the logical turnstile $\vdash$.
																	
																	\begin{definition}[Language of Propositional $L^0_n$]
																		The language of propositional $L^0_n$ consists only of:
																		\begin{itemize}
																			\item \textbf{Propositional Variables:} $p, q, r, \dots$ (denoted by the set $V$)
																			\end{itemize}
																				There are no logical connectives in this base language. The symbol $\vdash$ is a syntactic separator, not a logical operator.
																				\end{definition}
																					
																					\begin{definition}[Formula Forms in $L^0_n$]
																						The formulas in $L^0_n$ are simply the propositional variables:
																						$$Form_{L^0_n} = V$$
																						\end{definition}
																							
																							\begin{definition}[Sequent Forms in $L^0_n$ (Defined by $\mathcal{M}^0_n$)]
																								The allowed \textbf{syntactic forms} of sequents in $L^0_n$, as defined by its proper minimal metalanguage $\mathcal{M}^0_n$ within the category $D_n$, are:
																								\begin{itemize}
																									\item $\vdash A$ (Empty antecedent, formula succedent)
																									\item $A \vdash$ (Formula antecedent, empty succedent)
																									\item $A \vdash B$ (Formula antecedent, formula succedent)
																									\end{itemize}
																										where $A, B \in Form_{L^0_n}$. These forms are represented as ordered tuples in a set-theoretic sense. The notation $L^{\vdash}_n$ highlights this foundational aspect related to the turnstile.
																										\end{definition}
																											
																											\begin{definition}[Set Representation of $L^0_n$]
																												$L^0_n$ is formally represented as an ordered tuple capturing its fundamental components:
																												$$L^0_n = \langle \Sigma_{L^0_n}, Form_{L^0_n}, SeqForm_{L^0_n}, Axioms_{L^0_n}, Rules_{L^0_n} \rangle$$
																												where:
																												\begin{itemize}
																													\item $\Sigma_{L^0_n} = V \cup \{\vdash\}$ (Alphabet)
																													\item $Form_{L^0_n} = V$ (Set of Formulas)
																													\item $SeqForm_{L^0_n} = \{(\vdash, A) \mid A \in V\} \cup \{(A, \vdash) \mid A \in V\} \cup \{(A, \vdash, B) \mid A, B \in V\}$ (Set of Valid Sequent Forms)
																													\item $Axioms_{L^0_n} = \emptyset$ (Set of Axioms)
																													\item $Rules_{L^0_n} = \emptyset$ (Set of Inference Rules)
																													\end{itemize}
																														\end{definition}
																															
																															\begin{proposition}[$L^0_n$ is Not Equivalent to the Empty Set]
																																The formal system $L^0_n$, as a structured tuple containing non-empty sets ($\Sigma_{L^0_n}, Form_{L^0_n}, SeqForm_{L^0_n}$), is fundamentally distinct from the set-theoretic empty set ($\emptyset$). $L^0_n \neq \emptyset$.
																															\end{proposition}
																															
																															\begin{remark}[Minimal Non-Classical Judgments]
																																Although $Axioms_{L^0_n}$ and $Rules_{L^0_n}$ are empty, leading to an empty set of classically derivable sequents, $L^0_n$ possesses a minimal set of non-classical logical judgments or statuses assigned to its sequent forms, as defined by its proper minimal metalanguage $\mathcal{M}^0_n$.
																																\end{remark}
																																	
																																	\subsection{Terminal Object Language ($L^\infty_n$ or $L^{\Gamma\vdash\Delta}_n$)}
																																	
																																	The Terminal Object Language ($L^\infty_n$ or $L^{\Gamma\vdash\Delta}_n$) is the terminal object in the category $D_n$. It is a formal system characterized by maximal completeness and minimally metamorphic properties within $D_n$. It represents the apex or closure of the diamond structure for $D_n$, characterized by the most general sequent form $\Gamma \vdash \Delta$.
																																	
																																	\begin{definition}[Terminal Object Language ($L^\infty_n$)]
																																		The \textbf{Terminal Object Language} ($L^\infty_n$ or $L^{\Gamma\vdash\Delta}_n$) in a category of logics $D_n$ is a formal system characterized by:
																																		\begin{itemize}
																																			\item \textbf{Rich Syntax:} Comprehensive set of standard logical operators relevant to $D_n$.
																																			\item \textbf{Standard Sequent Forms:} Sequent forms appropriate for the logics in $D_n$, including the most general multi-formula sequents ($\Gamma \vdash \Delta$), as highlighted by the notation $L^{\Gamma\vdash\Delta}_n$.
																																			\item \textbf{Presence of Relevant Structural Rules:} Includes structural rules appropriate for the logics in $D_n$.
																																			\item \textbf{Maximal Closure:} Any consistent extension within $D_n$ is isomorphic to $L^\infty_n$.
																																			\end{itemize}
																																				$L^\infty_n$ is the terminal object in the category $D_n$.
																																			\end{definition}
																																			
																																			\begin{remark}[$L^\infty_n$'s Minimal Metamorphicity]
																																				$L^\infty_n$ is \textbf{Minimally Metamorphic} within the category $D_n$. Its interpretation is relatively fixed across reflecting metalanguages within $D_n$. It is hypothesized to possess maximal completeness and maximally undecidable consistency relative to the logics in $D_n$.
																																				\end{remark}
																																					
																																					\begin{analogy}[Physics Analogy for $L^\infty_n$]
																																						$L^\infty_n$ is a form of \textbf{Logical Completion} or a \textbf{Maximal Logic} within the specific space defined by $D_n$. For a category of first-order logics, LK (Classical First-Order Logic) is a strong candidate for $L^\infty_n$.
																																						\end{analogy}
																																							
																																							\subsection{Logic of Truth ($\LT$ or $L^{\vdash\Delta}_n$)}
																																							
																																							The Logic of Truth ($\LT$ or $L^{\vdash\Delta}_n$) is an extremal logic characterized by a focus solely on truth, where every formula is considered true. It is a specific logical extension of $L^0_n$ ($L^{\vdash}_n$) within the category $D_n$, arising from the reflection of $L^0_n$ by a metalanguage within $D_n$ that imposes the condition of universal truth. It can be seen as a maximally paracomplete logic that trivializes falsity within $D_n$, forming one side point of the diamond. The notation $L^{\vdash\Delta}_n$ emphasizes its connection to theorems (derivable from the empty antecedent) and the structure of the succedent (multiple conclusions).
																																							
																																							\begin{definition}[Language of $\LT$]
																																								The language of $\LT$ includes propositional variables $V$ and potentially a minimal set of connectives relevant to $D_n$. Its defining characteristic is its semantic interpretation where every formula is true within the models of $D_n$, and its consequence relation reflects this.
																																								\end{definition}
																																									
																																									\begin{definition}[Semantics of $\LT$]
																																										In the semantics of $\LT$ (within the framework of $D_n$), every formula $A$ is assigned the truth value "True". There are no mechanisms to represent or infer falsity in a non-trivial way. Formally, for any formula $A$, $\models_{\LT} A$.
																																										\end{definition}
																																											
																																											\begin{remark}[Properties of $\LT$]
																																												$\LT$ is maximally paracomplete (no formula is false in a non-trivial sense within $D_n$). Its negation is non-classical; it does not assert classical falsity but might indicate "not leading to truth" or be trivializing relative to $D_n$. Consequently, standard sequent rules for negation that move formulas from the RHS to the LHS (asserting classical falsity) are likely not admissible. Rules that move formulas from the LHS to the RHS (related to "A does not entail truth") might be compatible. Disjunction and existential quantification are likely non-problematic, but conjunction and universal quantification may be degenerate or problematic due to the universal truth of all formulas. $\LT$ is minimally metamorphic in the sense that its core property (universal truth) is likely preserved across metalanguages within $D_n$.
																																												\end{remark}
																																													
																																													\subsection{Logic of Falsity ($\LF$ or $L^{\Gamma\vdash}_n$)}
																																													
																																													The Logic of Falsity ($\LF$ or $L^{\Gamma\vdash}_n$) is an extremal logic characterized by a focus solely on falsity, where every formula is considered false. It is a specific logical extension of $L^0_n$ ($L^{\vdash}_n$) within the category $D_n$, arising from the reflection of $L^0_n$ by a metalanguage within $D_n$ that imposes the condition of universal falsity. It can be seen as a maximally paraconsistent logic that trivializes truth within $D_n$, forming the other side point of the diamond. The notation $L^{\Gamma\vdash}_n$ emphasizes its connection to reasoning from premises (non-empty antecedent) and leading to absurdity (empty succedent).
																																													
																																													\begin{definition}[Language of $\LF$]
																																														The language of $\LF$ includes propositional variables $V$ and potentially a minimal set of connectives relevant to $D_n$. Its defining characteristic is its semantic interpretation where every formula is false within the models of $D_n$, and its consequence relation reflects this.
																																														\end{definition}
																																															
																																															\begin{definition}[Semantics of $\LF$]
																																																In the semantics of $\LF$ (within the framework of $D_n$), every formula $A$ is assigned the truth value "False". There are no mechanisms to represent or infer truth in a non-trivial way. Formally, for any formula $A$, $A \models_{\LF} \emptyset$ (or some equivalent representation of universal falsity).
																																																\end{definition}
																																																	
																																																	\begin{remark}[Properties of $\LF$]
																																																		$\LF$ is maximally paraconsistent (no formula is true in a non-trivial sense within $D_n$). Its negation is non-classical; it does not assert classical truth but might indicate "not leading to falsity" or be trivializing relative to $D_n$. Consequently, standard sequent rules for negation that move formulas from the LHS to the RHS (asserting classical truth) are likely not admissible. Rules that move formulas from the RHS to the LHS (related to "A does not entail falsity") might be compatible. Conjunction and universal quantification are likely non-problematic, but disjunction and existential quantification may be degenerate or problematic due to the universal falsity of all formulas. $\LF$ is minimally metamorphic in the sense that its core property (universal falsity) is likely preserved across metalanguages within $D_n$.
																																																		\end{remark}
																																																			
																																																			
																																																			\section{The Metalanguage ($\mathcal{M}^0_n$) for $L^0_n$}
																																																			
																																																			The metalanguage $\mathcal{M}^0_n$ is the formal system within the category $D_n$ that defines and reasons about $L^0_n$ ($L^{\vdash}_n$). Its own minimality within $D_n$ is crucial for shaping $L^0_n$'s structure.
																																																			
																																																			\begin{definition}[Language of $\mathcal{M}^0_n$]
																																																				The language of $\mathcal{M}^0_n$ includes variables over $L^0_n$'s syntax and predicates to describe its properties (e.g., $\text{PropVar}(a)$, $\text{Formula}(F)$, $\text{SequentForm}(S)$, $\text{HasMinimalJudgment}(S)$). It possesses a minimal set of meta-logical operators (predication, identity, distinction, minimal conjunction/listing, minimal universal assertion) whose inference rules are themselves minimal within the context of $D_n$.
																																																				\end{definition}
																																																					
																																																					\begin{remark}[$\mathcal{M}^0_n$'s Role in Defining $L^0_n$'s Structure]
																																																						The axioms and definitions within $\mathcal{M}^0_n$ formally specify $L^0_n$'s syntax and the assignment of minimal judgments. For the specific $L^0_n$ with no axioms or rules, $\mathcal{M}^0_n$ defines that no sequent form is assigned a minimal judgment:
																																																						$$\forall S (\neg \text{HasMinimalJudgment}(S))$$
																																																						\end{remark}
																																																							
																																																							\begin{proposition}[Meta-proof in $\mathcal{M}^0_n$: Empty Sequent is Not Derivable in $L^0_n$]
																																																								Within the syntax and minimal inference rules of $\mathcal{M}^0_n$, it is provable that the empty sequent form ($\vdash$) is not a valid sequent form in $L^0_n$, and thus does not possess a minimal judgment.
																																																								\begin{proof}[Sketch]
																																																									The definition of valid sequent forms in $\mathcal{M}^0_n$ requires a formula in either the antecedent or succedent (or both). The empty sequent form lacks both. Thus, by the definition of sequent forms in $\mathcal{M}^0_n$, $\neg \text{SequentForm}(\vdash)$ is derivable in $\mathcal{M}^0_n$. Since having a minimal judgment implies being a valid sequent form ($\text{HasMinimalJudgment}(S) \implies \text{SequentForm}(S)$), the lack of a valid sequent form implies the lack of a minimal judgment ($\neg \text{SequentForm}(S) \implies \neg \text{HasMinimalJudgment}(S)$). Therefore, $\neg \text{HasMinimalJudgment}(\vdash)$ is derivable in $\mathcal{M}^0_n$.
																																																								\end{proof}
																																																								\end{proposition}
																																																									
																																																									\begin{remark}[$\mathcal{M}^0_n$'s Minimality Produces $L^0_n$'s Structure]
																																																										The structure of $L^0_n$ and its lack of classical derivability is a direct consequence of the inherent logical minimality of $\mathcal{M}^0_n$ itself, as reflected into the object language. $\mathcal{M}^0_n$'s limited logical power means it cannot define a richer structure for $L^0_n$.
																																																										\end{remark}
																																																											
																																																											\section{Metalanguage Reflection and Interpretation within $D_n$}
																																																											
																																																											The relationship between $L^0_n$ ($L^{\vdash}_n$) and other logical systems within the category $D_n$ is mediated by metalanguage reflection.
																																																											
																																																											\begin{definition}[Object Language, Metalanguage, Reflection within $D_n$]
																																																												An \textbf{Object Language} $L$ is a formal system in $D_n$. A \textbf{Metalanguage} $\mathcal{M}$ is a formal system in $D_n$ for describing $L$. \textbf{Reflection} ($\text{Reflect}(\mathcal{M}, L)$) is a mapping or set of definitions in $\mathcal{M}$ to represent $L$.
																																																												\end{definition}
																																																													
																																																													\begin{remark}[$L^0_n$'s Maximal Metamorphicity within $D_n$]
																																																														$L^0_n$ ($L^{\vdash}_n$) is \textbf{Maximally Metamorphic} within $D_n$. Its interpretation and properties ascribed to it are highly dependent on the reflecting metalanguage $\mathcal{M} \in D_n$. $\text{Reflect}(\mathcal{M}, L^0_n)$ gives rise to a \textbf{Logical Extension} $L_0^{\mathcal{M}}$, shaped by $\mathcal{M}$'s properties within $D_n$.
																																																														\end{remark}
																																																															
																																																															\begin{analogy}[Thought Experiment Analogy for Metamorphicity]
																																																																$L^0_n$ ($L^{\vdash}_n$) is a "Message" interpreted differently in various "Rooms" ($\mathcal{M}$s) within the category $D_n$. The extension $L_0^{\mathcal{M}}$ is the specific interpretation in that room.
																																																																\end{analogy}
																																																																	
																																																																	\begin{remark}[Interpretability and Its Limitations within $D_n$]
																																																																		$L^0_n$ ($L^{\vdash}_n$) is \textbf{Interpretable (in a basic sense)} in any capable $\mathcal{M} \in D_n$. However, its \textbf{Non-classical Interpretations and Extensions within $D_n$ may not be Faithfully Interpretable in Metalanguages within $D_n$ that lack the necessary expressive power}.
																																																																		\end{remark}
																																																																			
																																																																			\begin{analogy}[Thought Experiment Analogy for Limitations]
																																																																				A metalanguage $\mathcal{M} \in D_n$ lacking certain non-classical features is like \textbf{Mary in the Black-and-White Room} trying to understand color (non-classical properties) or the \textbf{Chinese Room} processing symbols without full understanding.
																																																																				\end{analogy}
																																																																					
																																																																					\begin{remark}[Reciprocal Non-Interpretability within $D_n$]
																																																																						There is likely \textbf{Reciprocal Non-Interpretability (partial)} between $L^0_n$ ($L^{\vdash}_n$) and its various $\mathcal{M}$s within $D_n$. $L^0_n$'s minimality limits its ability to interpret complex $\mathcal{M}$s in $D_n$, and $\mathcal{M}$s' structures limit their ability to fully interpret $L^0_n$'s potential within $D_n$.
																																																																						\end{remark}
																																																																							
																																																																							\begin{remark}[Compatibility with Logical Systems within $D_n$]
																																																																								$L^0_n$'s ($L^{\vdash}_n$) minimal structure makes it compatible with \textbf{Consistent} and \textbf{Paraconsistent} systems within $D_n$, which maintain logical distinctions. It is incompatible with \textbf{Inconsistent} systems (contradictory and explosive) within $D_n$ because its structured minimality collapses in such frameworks where everything is derivable.
																																																																								\end{remark}
																																																																									
																																																																									\subsection{Metalanguage Properties and the Type/Degree of Non-Classicality within $D_n$}
																																																																									The properties of the metalanguage $\mathcal{M} \in D_n$ used to define or reflect an object language $L \in D_n$ significantly influence the type and degree of non-classicality that $L$ can exhibit or that can be analyzed in $L$ within $D_n$.
																																																																									\begin{itemize}
																																																																										\item If $\mathcal{M}$ has classical-like structural rules but a non-classical negation (e.g., one that fails LNC or LEM), it can define object languages with **connective/formula-aware** paraconsistency or paracompleteness.
																																																																										\item If $\mathcal{M}$ has classical-like connectives but restricts structural rules (e.g., Weakening, Contraction), it can define object languages with **structural** paraconsistency or paracompleteness, such as Classical Linear Logic (CLL) within a category of relevant logics.
																																																																										\end{itemize}
																																																																											This highlights that the choice of metalanguage within $D_n$ determines not just *whether* an object language is non-classical, but *how* it deviates from classicality and to what extent, relative to the logical space defined by $D_n$.
																																																																											
																																																																											\section{The Diamond Structure and Relationships within $D_n$}
																																																																											
																																																																											The relationship between the four extremal logics ($L^0_n$/$L^{\vdash}_n$, $\LT$/$L^{\vdash\Delta}_n$, $\LF$/$L^{\Gamma\vdash}_n$, $L^\infty_n$/$L^{\Gamma\vdash\Delta}_n$) forms a landscape visualized as a diamond within the category $D_n$, representing the space of logical extensions within that category.
																																																																											
																																																																											\begin{remark}[Structure of the Diamond in $D_n$]
																																																																												The four extremal logics form the vertices of the diamond for the category $D_n$:
																																																																												\begin{itemize}
																																																																													\item $L^0_n$ ($L^{\vdash}_n$) is at the bottom point (minimum non-trivial, maximally metamorphic within $D_n$, characterized by minimal sequent forms).
																																																																													\item $L^\infty_n$ ($L^{\Gamma\vdash\Delta}_n$) is at the top point (terminal object, maximal completeness, maximally undecidable consistency, minimally metamorphic within $D_n$, characterized by the most general sequent form).
																																																																													\item $\LT$ ($L^{\vdash\Delta}_n$) is at one side point (maximal paracompleteness, focus on truth within $D_n$, characterized by empty antecedent and multi-formula succedent sequents).
																																																																													\item $\LF$ ($L^{\Gamma\vdash}_n$) is at the other side point (maximal paraconsistency, focus on falsity within $D_n$, characterized by multi-formula antecedent and empty succedent sequents).
																																																																													\end{itemize}
																																																																														Intermediate logics in $D_n$ lie within the diamond, representing systems with varying degrees of structure, paraconsistency, and paracompleteness relevant to $D_n$.
																																																																														
																																																																														The paths from $L^0_n$ ($L^{\vdash}_n$) to the other vertices involve the systematic addition of structural features and logical commitments, guided by the reflecting metalanguage within $D_n$:
																																																																														\begin{itemize}
																																																																															\item $L^0_n \to L^\infty_n$: Adding syntax, sequent forms (culminating in $\Gamma \vdash \Delta$), and structural rules relevant to $D_n$, leading to maximal closure and minimal metamorphicity within $D_n$. This path encompasses logics like LK if $D_n$ is a category of first-order logics.
																																																																															\item $L^0_n \to \LT$: Adding semantic structure that enforces universal truth, trivializing falsity relative to $D_n$ and leading to maximal paracompleteness. This involves introducing rules that align with universal truth, while avoiding rules that would introduce non-trivial falsity or standard negation behavior, and is characterized by sequent forms like $\vdash \Delta$.
																																																																															\item $L^0_n \to \LF$: Adding semantic structure that enforces universal falsity, trivializing truth relative to $D_n$ and leading to maximal paraconsistency. This involves introducing rules that align with universal falsity, while avoiding rules that would introduce non-trivial truth or standard negation behavior, and is characterized by sequent forms like $\Gamma \vdash$.
																																																																															\end{itemize}
																																																																																The paths between $\LT$ ($L^{\vdash\Delta}_n$), $\LF$ ($L^{\Gamma\vdash}_n$), and $L^\infty_n$ ($L^{\Gamma\vdash\Delta}_n$) involve navigating the trade-offs between paraconsistency, paracompleteness, and the inclusion of both truth and falsity in a non-trivial way within $D_n$. $L^\infty_n$, as the terminal object in $D_n$, is the target for interpretations from all other logics in $D_n$, regardless of whether their non-classicality is structural or connective-based.
																																																																																\end{remark}
																																																																																	
																																																																																	\begin{remark}[Categorical Perspective on $D_n$]
																																																																																		In the category $D_n$ of formal systems and interpretations, $L^0_n$ ($L^{\vdash}_n$) is the initial object (unique morphism $L^0_n \to L$ for any $L \in D_n$) and $L^\infty_n$ ($L^{\Gamma\vdash\Delta}_n$) is the terminal object (unique morphism $L \to L^\infty_n$ for any $L \in D_n$). $\LT$ ($L^{\vdash\Delta}_n$) and $\LF$ ($L^{\Gamma\vdash}_n$) could potentially be viewed as other types of limiting objects or significant points within $D_n$, perhaps related to universal properties concerning truth- or falsity-preserving morphisms within $D_n$.
																																																																																		\end{remark}
																																																																																			
																																																																																			\begin{remark}[Hierarchy of Terminal Languages and Diamonds]
																																																																																				Systems like LK are candidates for terminal objects ($L^\infty_n$) within specific categories ($D_n$, e.g., first-order logics) but are themselves objects within broader categories or hierarchies where other diamonds and terminal objects reside at higher orders. The diamond structure for $D_n$ provides a visual aid for understanding the relationships between different levels and types of logical systems within that specific logical space.
																																																																																				\end{remark}
																																																																																					
																																																																																					\section{Conclusion}
																																																																																					
																																																																																					\RaggedRight % Apply RaggedRight to the conclusion
																																																																																					This document has provided a preliminary formalization of key components of a non-classical meta-semantic framework, drawing upon the conceptual structure of the "Meta-Semanticum Universalis" papers and the semantic diversity of paraconsistent and paracomplete logics. We have formalized the sets of meta-properties, context parameters, and meta-languages, and defined the non-Boolean ValueSpace ($\ValueSpace$) and the Interpretation Function ($\InterpFunc$) mapping to it. Crucially, we have outlined how Relation Principles ($\RelPrinciples$) and the Evaluation System ($\EvalSystem$) must be formulated using the logical machinery of a potentially non-classical meta-language, moving beyond classical assumptions about implication, negation, and aggregation. This formalization addresses some of the critiques regarding the lack of rigor and provides a foundation for further development of a meta-semantic theory capable of analyzing and evaluating complex systems from diverse logical perspectives, including those that tolerate inconsistency and incompleteness. We have also formalized the four extremal logics ($L^0_n$/$L^{\vdash}_n$, $\LT$/$L^{\vdash\Delta}_n$, $\LF$/$L^{\Gamma\vdash}_n$, and $L^\infty_n$/$L^{\Gamma\vdash\Delta}_n$) and described the diamond structure they form within a specific category of logics $D_n$, providing a clearer landscape for understanding logical extensions and the different paths towards increased logical structure within that category. The connection to categorical frameworks and dualities, along with the nuance regarding how metalanguage properties influence the type of non-classicality, suggests avenues for understanding the relationships between different meta-language perspectives and the paths between these extremal points within $D_n$.
																																																																																					
																																																																																				\end{document}}*}


\chapter{Common Fork Revise Undecidable Theories Part I.2}
∃∀ \{⊥,⊤,¬,∨,∧,→,↛,↔,⊕,↓,↑\} \{p, q\} ⊢⊬⊨

\protect\hypertarget{anchor}{}{}Theories with Standard Formalization

The theories discussed in this paper will be referred to as
\emph{theories with standard formalization}. They can be briefly
characterized as theories which are formalized within the first-order
predicate logic (with identity, with quantifiers, without variable
predicates).

\protect\hypertarget{anchor-1}{}{}Syntactic Definitions

\protect\hypertarget{anchor-2}{}{}Definition: Variables and Constants of
a Theory

The symbols which occur in expressions of a given theory T are divided
into \emph{variables} and \emph{constants}.

The set of variables is assumed to be denumerable and hence infinite;
the set of constants is either finite or denumerable. All the variables
are treated as ranging over the same set of elements.

\protect\hypertarget{anchor-3}{}{}Definition: Logical Constants of a
Theory

The constants are divided into \emph{logical} and \emph{non-logical}
ones.

The logical constants are the \emph{sentential connectives}

the \emph{negation sign} ¬,

the \emph{implication sign} →,

the \emph{equivalence sign} ↔,

the \emph{disjunction sign} ∨,

the \emph{conjunction sign }∧;

the quantifiers

the \emph{universal quantifier} ∀ and \emph{existential quantification}
∃;

and finally, the i\emph{dentity symbol} \textbf{=}.

\protect\hypertarget{anchor-4}{}{}

\protect\hypertarget{anchor-5}{}{}Definition: non-Logical Constants of a
Theory AKA Signature of a Theory

The non-logical constants are the \emph{predicates} (or \emph{relation
symbols}), the \emph{operation symbols}, and the \emph{individual
constants}.

With every predicate and every operation symbol a positive integer is
correlated which is called the \emph{rank} of the symbol. Thus, we may
have in T \emph{unary} predicates and operation symbols (I.E. symbols of
rank 1), \emph{binary} predicates and operation symbols (symbols of rank
2), etc. The identity symbol, though regarded as a logical constant, is
included in the set of binary predicates.

In practice, in addition to variables and constants, the so-called
technical symbols like parentheses and commas, are also used in
constructing expressions; however, theoretically these technical symbols
can be dispensed with.

\protect\hypertarget{anchor-6}{}{}Definition: Terms

Among expressions (I.E. finite concatenations of symbols) we distinguish
\emph{terms} and \emph{formulas}.

The simplest, so-called atomic, terms are the variables and the
individual constants; a compound term is obtained by combining \emph{n}
simpler terms by means of an operation symbol of rank \emph{n}.

\protect\hypertarget{anchor-7}{}{}Definition: Formulas

Similarly, an atomic formula is obtained by combining \emph{n} arbitrary
terms by means of a predicate of rank \emph{n}; compound formulas are
built from simpler ones by means of sentential connectives and
quantifier expressions (I.E. quantifiers followed by variables like ∀x
or ∃y).

\protect\hypertarget{anchor-8}{}{}Definition: Sentences

An occurrence of a variable in a formula may be either \emph{free} or
\emph{bound}; a formula in which no variable occurs free is called a
\emph{sentence}.

\protect\hypertarget{anchor-9}{}{}Semantic Definitions

\protect\hypertarget{anchor-10}{}{}Proof-Theoretical Semantics

\protect\hypertarget{anchor-11}{}{}Definition: Axiomatic Proof

Two further notions, those of \emph{logical derivability} and
\emph{logical validity}, are involved in the metamathematical discussion
of any theory T.

First, we single out certain sentences of T which are referred to as
\emph{logical axioms}.

Secondly, we describe certain (finitary) operations, the so-called
\emph{operations of inference}, which when performed on sentences yield
new sentences.

Usually the set of logical axioms is infinite while the set of
operations of inference is finite.

\protect\hypertarget{anchor-12}{}{}Definition: Modus Ponens; the method
of affirmation.

The most important operation of inference is that of \emph{detachment}
(\emph{modus ponens}), which when applied to two sentences Φ and Φ→Ψ
yields the sentence Ψ.

In fact, it proves to be possible, by selecting a suitable set of
logical axioms, to use the operations of detachment as the only
operation of inference in formulating adequate definitions of
derivability and logical validity.

\protect\hypertarget{anchor-13}{}{}Definition: Derivable Sentences

A sentence is now said to be \emph{logically derivable} or simply
\emph{derivable} from a set A of sentences if it can be obtained from
sentences of A and from logical axioms by performing operations of
inference an arbitrary number of times.

\protect\hypertarget{anchor-14}{}{}Definition: Logically Valid or
Provable Sentences

A sentence is called \emph{logically valid} (or \emph{logically
provable}) if it is derivable from the set of logical axioms-\/-or what
amounts to the same thing, from the empty set of sentences.

\protect\hypertarget{anchor-15}{}{}Model-Theoretical Semantics

Another method of defining logically derivable and logically valid is
available which essentially involves the use of some semantical notions
and the notion of satisfaction.

\protect\hypertarget{anchor-16}{}{}Definition: Possible Realizations of
a Theory and the Universe of 𝕽

We assume that all the non-logical constants of T have been arranged in
a (finite or infinite) sequence \textless{}\textbf{C\_0}, \ldots,
\textbf{C\_n}, \ldots\textgreater, without repeating terms.

We consider systems 𝕽 formed by a non-empty set U and by a sequence
\textless C\_0, \ldots, C\_n, \ldots\textgreater{} of certain
mathematical entities, with the same number of terms as the sequence of
non-logical constants.

The mathematical nature of each C\_n depends on the logical character of
the corresponding constant \textbf{C\_n}. Thus,

if \textbf{C\_n} is a unary predicate, then C\_n is a subset of U; more
generally, if \textbf{C\_n} is an \emph{m}-ary predicate, then C\_n is
an \emph{m}-ary relation the field of which is a subset of U.

If \textbf{C\_n} is an \emph{m}-ary operation symbol, C\_n is an
\emph{m}-ary operation (function of \emph{m} arguments) defined over
arbitrary ordered \emph{m}-tuples \textless{}\emph{x\_1}, \ldots,
\emph{x\_m}\textgreater{} of elements of U and assuming elements of U as
values.

If \textbf{C\_n} is an individual constant, C\_n is simply an element of
U.

Such a system (sequence) 𝕽 = \textless U, C\_0, \ldots, C\_n,
\ldots\textgreater{} is called a \emph{possible realization} or simply a
\emph{realization} of T; the set U is called the \emph{universe} of 𝕽.

\protect\hypertarget{anchor-17}{}{}Definition: Satisfaction

We assume it to be clear under what conditions a sentence Φ of T is said
to \emph{be satisfied} or to \emph{hold} in a given realization 𝕽.

Roughly speaking, this means that Φ turns out to be true if

(i) all the variables occurring in T are assumed to range over the set
U;

(ii) the logical constants are interpreted in the usual way;

(iii) each of the non-logical constants \textbf{C\_n} is understood to
denote the corresponding term C\_n in 𝕽.

Assume, e.g., that the term \textbf{C\_n} in the sequence of constants
is a unary predicate and that consequently C\_n is a subset of U. Then
the sentence ∀x \textbf{C\_n} x holds in 𝕽 if and only if every element
of U is an element of C\_n and hence C\_n coincides with U.

\protect\hypertarget{anchor-18}{}{}Definition: Logical Consequence and
Logical Truth

A sentence Φ is said to be a \emph{logical consequence} of a set A of
sentences if it is satisfied in every realization 𝕽 in which all
sentences of A are satisfied; it is called \emph{logical true} if it is
satisfied in every possible realization.

As opposed to the notions of logical consequence and logical truth, the
related notions of logical derivability and logical validity, when
defined in terms of axioms and operations of inference, seem to have a
rather accidental and arbitrary character.

\protect\hypertarget{anchor-19}{}{}Equivalence of Proof-Theoretical and
Model-Theoretical Semantics for Theories in Standard Formalization

Hence, it might seem natural to redefine logical derivability and
logical validity simply by stipulating that a sentence is derivable from
A if it is a logical consequence of A, and by identifying logically
valid sentences with logically true sentences. From the results in Gödel
{[}5{]} it follows that under the systems of logical axioms and
operations of inference known from the literature the two methods of
defining derivability and logical validity are entirely equivalent (when
applied to theories with standard formalization).

An important property of the notion of derivability is stated in the
following well-known theorem, which is often applied in
meta-mathematical discussion:

\protect\hypertarget{anchor-20}{}{}Deduction Theorem I

Let A be a set of sentences of a theory T and let Φ\_1, ..., Φ\_n, Ψ be
any sentences of T.

For Ψ to be derivable from the set A supplemented by the sentences Φ\_1,
..., Φ\_n it is necessary and sufficient that the sentence

(Φ\_1∧...∧Φ\_n) → Ψ

be derivable from the set A alone.

\protect\hypertarget{anchor-21}{}{}Deduction Theorem II

Let A be a set of sentences of a theory T, and let Ψ be a sentence of T.

For Ψ to be derivable from A it is necessary and sufficient that A be
empty and Ψ be logically valid or else that A contain some sentences
Φ\_1, ..., Φ\_n such that the sentence

(Φ\_1∧...∧Φ\_n) → Ψ

is logically valid.

Thus the notion of derivability has a simple characterization in terms
of logically valid sentences.

\protect\hypertarget{anchor-22}{}{}Definition: Valid Sentences

To complete the description of a theory T we have to define what we mean
by a \emph{valid} sentence in general (as opposed to a logically valid
sentence). No uniform method for defining this notion is available.

\protect\hypertarget{anchor-23}{}{}Definition: Valid Sentences by
Non-Logical Axioms

Often we single out a (finite or infinite) set of sentences called
\emph{non-logical axioms}, and define a sentence to be valid if and only
if it is derivable from this set-\/-or, what amounts to the same, from
the set of all axioms, both logical and non-logical.

\protect\hypertarget{anchor-24}{}{}

\protect\hypertarget{anchor-25}{}{}Definition: Axiomatic Theories

Theories in which the notion of validity has been introduced in this way
are referred to as \emph{axiomatically built} or, simply
\emph{axiomatic} theories; when referring to such theories, we often use
the term ``\emph{provable}'' instead of ``\emph{valid}''.

We do not restrict ourselves to the discussion of axiomatic theories.
Sometimes we agree to consider as valid those and only those sentences
which are satisfied in a given realization or in all realizations of a
given class; sometimes we define validity for a theory in terms of
validity for some other theories for which this notion has been
previously defined.

We assume, however, that for each of the theories discussed the notion
of validity has been defined in one way or another.

\protect\hypertarget{anchor-26}{}{}Definition: General Condition for
Validity of Sentences

We assume that under this definition every sentence which is logically
derivable from a set of valid sentences is itself valid, and that
consequently every logically valid sentence is valid; this is the only
condition imposed upon the definition of validity.

\protect\hypertarget{anchor-27}{}{}Definition: Model of a Theory

A possible realization in which all valid sentences of a theory T are
satisfied is called a \emph{model} of T.

In all the theories with standard formalization the same symbols are
assumed to be used as variables and logical constants; apart from
differences in non-logical constants, the same expressions are regarded
as formulas, sentences, logical axioms, and logically valid sentences.

However, the notions of validity in these theories may of course exhibit
essential differences.

\protect\hypertarget{anchor-28}{}{}Theoretical Definitions

\protect\hypertarget{anchor-29}{}{}Definition: Uniqueness of a Theory in
Standard Formalization

A theory is uniquely determined by the set of all its valid sentences;
two theories are regarded as identical if their sets of valid sentences
coincide.

\protect\hypertarget{anchor-30}{}{}Definition: Uniqueness of an
Axiomatic Theory in Standard Formalization

An axiomatic theory is uniquely determined by its non-logical constants
and non-logical axioms.

\protect\hypertarget{anchor-31}{}{}Definition: Subtheory or Supertheory;
Subtension or Extension

A theory T\_1 is called a \emph{subtheory} of a theory T\_2 if every
sentence which is valid in T\_1 is also valid in T\_2; under the same
conditions T\_2 is referred to as an \emph{extension} of T\_1.

\protect\hypertarget{anchor-32}{}{}Definition: Inessential Extensions of
Theories in Standard Formalization

An extension T\_2 of T\_1 is called \emph{inessential} if every constant
of T\_2 which does not occur in T\_1 is an individual constant and if
every valid sentence of T\_2 is derivable in T\_2 from a set of valid
sentences of T\_1.

If T\_1 is axiomatic, then an inessential extension of T\_1 is obtained
by adding some new individual constants, but without adding any new
non-logical axioms.

By saying that a sentence Φ is derivable \emph{in a theory} T from a set
A we stress the fact that, in deriving Φ, we may use both sentences of A
and logical axioms of T. It is easily seen that, whenever Φ is derivable
from A in some theory T, it is also derivable from A in every theory
T\textquotesingle{} which contains all the non-logical constants
occurring in Φ and in sentences of A.

\protect\hypertarget{anchor-33}{}{}Definition: Finite Extensions of
Theories in Standard Formalization

An extension T\_2 of T\_1 is referred to as a \emph{finite} extension if
there is a finite set A of valid sentences of T\_2 such that every valid
sentence of T\_2 is derivable from a set of sentences which are valid in
T\_1 or belong to A. Clearly every inessential extension is a finite
extension.

\protect\hypertarget{anchor-34}{}{}Definition: Union of Theories in
Standard Formalization

Among the extensions common to two given theories T\_1 and T\_2 there is
always a smallest one, which is a subtheory of any other common
extension; this smallest common extension is referred to as the
\emph{union} of the given theories.

The union T of T\_1 and T\_2 is fully characterized by the following two
conditions:

(i) the set of all non-logical constants of T is the (set theoretical)
union of the set of all non-logical constants of T\_1 and T\_2;

(ii) a sentence is valid in T if and only if it is derivable in T from a
set of sentences which are valid in T\_1 or T\_2.

If the theories T\_1 and T\_2 are axiomatic, we can construct T by
postulating, in addition to (i), the analogous condition for the set of
non-logical axioms.

Notice that condition (i) unambiguously determines the notions of a
sentence of T and of a logical axiom of T, and hence also the notion of
derivability in T.

\protect\hypertarget{anchor-35}{}{}Definition: Consistent Theories

A theory T is called \emph{consistent} if not every sentence of T is
valid in T; or, in an equivalent formulation, if there exists no
sentence such that both Φ and ¬Φ are valid in T.

\protect\hypertarget{anchor-36}{}{}Definition: Complete Theories

A theory T is called \emph{complete} if there is no consistent extension
of T which is different from T, but which has the same constants as T;
equivalently, if, for every sentence Φ of T, either Φ or ¬Φ is valid in
T.

The proof of the equivalence of the two definitions of completeness is
based upon Deduction Theorem I.

\protect\hypertarget{anchor-37}{}{}Definition: Compatible Theories

Two theories T\_1 and T\_2 are said to be \emph{compatible} if they have
a common consistent extension; this is equivalent to saying that the
union of T\_1 and T\_2 is consistent.

\chapter{Constructive Procedures}
\hypertarget{chapter-0-preamble-and-introduction}{%
\section*{Chapter 0: Preamble and
Introduction}\label{chapter-0-preamble-and-introduction}}
\addcontentsline{toc}{section}{Chapter 0: Preamble and Introduction}

\hypertarget{background-and-motivation}{%
\subsection*{0.1 Background and
Motivation}\label{background-and-motivation}}
\addcontentsline{toc}{subsection}{0.1 Background and Motivation}

\hypertarget{what-are-constructive-procedures}{%
\subsubsection*{0.1.1 What are constructive
procedures?}\label{what-are-constructive-procedures}}
\addcontentsline{toc}{subsubsection}{0.1.1 What are constructive
procedures?}

\hypertarget{why-are-constructive-procedures-important}{%
\subsubsection*{0.1.2 Why are constructive procedures
important?}\label{why-are-constructive-procedures-important}}
\addcontentsline{toc}{subsubsection}{0.1.2 Why are constructive
procedures important?}

\hypertarget{historical-context-and-key-figures}{%
\subsubsection*{0.1.3 Historical context and key
figures}\label{historical-context-and-key-figures}}
\addcontentsline{toc}{subsubsection}{0.1.3 Historical context and key
figures}

\hypertarget{overview-of-the-book}{%
\subsection*{0.2 Overview of the Book}\label{overview-of-the-book}}
\addcontentsline{toc}{subsection}{0.2 Overview of the Book}

\hypertarget{structure-of-the-book}{%
\subsubsection*{0.2.1 Structure of the
book}\label{structure-of-the-book}}
\addcontentsline{toc}{subsubsection}{0.2.1 Structure of the book}

\hypertarget{main-topics-covered-in-each-chapter}{%
\subsubsection*{0.2.2 Main topics covered in each
chapter}\label{main-topics-covered-in-each-chapter}}
\addcontentsline{toc}{subsubsection}{0.2.2 Main topics covered in each
chapter}

\hypertarget{target-audience-and-prerequisites}{%
\subsubsection*{0.2.3 Target audience and
prerequisites}\label{target-audience-and-prerequisites}}
\addcontentsline{toc}{subsubsection}{0.2.3 Target audience and
prerequisites}

\hypertarget{part-i-consistent-paracompleteness}{%
\section*{Part I: Consistent
Paracompleteness}\label{part-i-consistent-paracompleteness}}
\addcontentsline{toc}{section}{Part I: Consistent Paracompleteness}

\hypertarget{chomsky-hierarchy-of-formal-grammars}{%
\subsection*{1. Chomsky Hierarchy of Formal
Grammars}\label{chomsky-hierarchy-of-formal-grammars}}
\addcontentsline{toc}{subsection}{1. Chomsky Hierarchy of Formal
Grammars}

1.1 Recursive Languages

Regular Languages

Context-Free Languages

Context-Sensitive Languages

1.2 Recursively Enumerable Languages

1.3 Non-Recursive Languages

\hypertarget{tarskian-hierarchy-of-formal-languages}{%
\subsection*{2. Tarskian Hierarchy of Formal
Languages}\label{tarskian-hierarchy-of-formal-languages}}
\addcontentsline{toc}{subsection}{2. Tarskian Hierarchy of Formal
Languages}

2.1 Introduction to Metalanguages and Tarskian Languages

2.2 Tarski\textquotesingle s Undefinability Theorem

2.2.1 Metalanguages and Tarskian Languages: Levels of Undefinability

2.3 Levels of the Tarskian Hierarchy:

2.3.1 Level 0: Languages without Truth Definitions

2.3.2 Level 1: Languages with Truth Definitions for Level 0 Languages

2.3.3 Level n: General Representation for Finite Hierarchies

2.3.4 Level ω: General Representation for Transfinite Hierarchies

\hypertarget{kripkean-hierarchies-of-formal-languages}{%
\subsection*{3. Kripkean Hierarchies of Formal
Languages}\label{kripkean-hierarchies-of-formal-languages}}
\addcontentsline{toc}{subsection}{3. Kripkean Hierarchies of Formal
Languages}

3.1 Introduction to Kripkean Semantics:

3.1.1 Limitations of Tarskian Semantics

3.1.2 Modal Operators and the Kripke Hierarchy

3.1.3 Fixed-Point Operators and Constructive Truth

3.1.4 Kripkean Theories of Truth: Strong Kleene, Partial Truth, and
Fixed-Point Logics

3.2 Modal Operators and Possible Worlds:

3.2.1 Kripke Frames and Accessibility Relations

3.2.2 Truth in Different Possible Worlds

3.2.3 Propositional Modal Logic: □ (Necessity) and ◇ (Possibility)

3.2.4 Quantified Modal Logic: Bringing in Quantifiers

3.3 Fixed-Point Operators and Constructive Truth:

3.3.1 Introduction to Fixed-Point Operators

3.3.2 Least and Greatest Fixed-Points: Finding the "Right" Truth Value

3.3.3 Recursion and Fixed-Point Operators

3.3.4 Fixed-Point Constructions in Formal Languages

3.4 Kleene Evaluation and Strong Kleene Fixed-Points:

3.4.1 Partial Truth Values: Beyond True and False

3.4.2 Kleene Evaluation: Assigning Truth Values in Steps

3.4.3 Strong Kleene Fixed-Points: Reaching Stable Truth Values

3.5 Partial Truth and the Hierarchy of Languages:

3.5.1 Languages with Partial Truth: Expanding Expressive Power

3.5.2 The Kripke Hierarchy of Fixed-Point Languages

3.5.3 Jumping Between Levels and Truth Definitions

\hypertarget{chapter-2-consistency-and-paracompleteness-in-formal-systems}{%
\subsection*{Chapter 2: Consistency and Paracompleteness in Formal
Systems}\label{chapter-2-consistency-and-paracompleteness-in-formal-systems}}
\addcontentsline{toc}{subsection}{Chapter 2: Consistency and
Paracompleteness in Formal Systems}

\textbf{2.3 Many-Valued and Paracomplete Logics}

While bivalent logic is powerful, it sometimes struggles to represent
real-world situations with varying degrees of truth or uncertainty. This
section introduces many-valued and paracomplete logics, which offer
alternative approaches.

\textbf{2.3.1 Many-Valued Logics:}

\begin{itemize}
\tightlist
\item
\end{itemize}

\textbf{2.3.2 Paracomplete Logics:}

\begin{itemize}
\item
  \begin{quote}
  These logics relax the principle of bivalence by allowing for
  "indeterminate" or "gap" truth values. This allows for reasoning with
  incomplete information or situations where truth is not fully
  determined.
  \end{quote}
\item
  \begin{quote}
  Kleene logic and Łukasiewicz logic are examples of paracomplete
  logics.
  \end{quote}
\end{itemize}

\textbf{2.3.3 The Power of the Monoid:}

Many-valued and paracomplete logics often utilize the concept of a
monoid, a mathematical structure that allows for combining truth values
in a meaningful way. This enables reasoning and inference even with
incomplete or uncertain information.

\textbf{2.4 Introduction to Many-Valued Logics (Specific Examples):}

This section could delve into specific examples of many-valued logics,
such as:

\begin{itemize}
\item
  \begin{quote}
  \textbf{Three-Valued Logic:} Explain how truth values are assigned and
  how propositions are evaluated. Discuss applications and limitations.
  \end{quote}
\item
  \begin{quote}
  \textbf{Fuzzy Logic:} Introduce the concept of fuzzy sets and fuzzy
  truth values. Explore practical applications in areas like control
  systems and decision-making.
  \end{quote}
\item
  \begin{quote}
  \textbf{Łukasiewicz Logic:} Explain its distinctive features and how
  it handles propositions with intermediate truth values. Discuss its
  connection to probability theory.
  \end{quote}
\end{itemize}

\hypertarget{bivalent-and-classical-consistencies-definitions-and-properties}{%
\subsubsection*{Bivalent and Classical Consistencies: Definitions and
Properties}\label{bivalent-and-classical-consistencies-definitions-and-properties}}
\addcontentsline{toc}{subsubsection}{Bivalent and Classical
Consistencies: Definitions and Properties}

\hypertarget{introduction-to-bivalence-logics}{%
\paragraph*{Introduction to Bivalence
Logics}\label{introduction-to-bivalence-logics}}
\addcontentsline{toc}{paragraph}{Introduction to Bivalence Logics}

\textbf{Bivalence:} Every proposition in the language is assigned one of
two truth values: true (T) or false (F). There are no "unknown" or
"indeterminate" values.

\textbf{Classical Consistency:} A formal system is classically
consistent if it does not admit a contradiction.

\hypertarget{properties-of-bivalent-and-classical-consistencies}{%
\subparagraph*{2.1.2 Properties of Bivalent and Classical
Consistencies:}\label{properties-of-bivalent-and-classical-consistencies}}
\addcontentsline{toc}{subparagraph}{2.1.2 Properties of Bivalent and
Classical Consistencies:}

\textbf{2-Valued T-Schema is a metatheorem for the object language.}

\textbf{Law of Excluded Middle:} For any proposition P, either P is true
or not-P is true. There is no third option.

\textbf{Law of Contradiction:} It is not possible for P and not-P to
both be true at the same time.

\textbf{Law of Double Negation:} Not (not-P) is logically equivalent to
P.

\textbf{Validity of Classical Negation}

\textbf{Classical Propositional Logic:} The basic logic of propositions
and connectives like "and," "or," and "not."

\textbf{Classical First Order Predicate Logic:} Introduces quantifiers
("all" and "some") to express propositions about individuals and
properties.

\textbf{Classical Modal Logic:} Deals with concepts like possibility and
necessity, adding modal operators like "necessarily" and "possibly."

\hypertarget{many-valued-and-paracomplete-consistencies-many-values-but-one-and-only-one-designated-value.}{%
\subsubsection*{Many-Valued and Paracomplete Consistencies: many-values
but one and only one designated
value.}\label{many-valued-and-paracomplete-consistencies-many-values-but-one-and-only-one-designated-value.}}
\addcontentsline{toc}{subsubsection}{Many-Valued and Paracomplete
Consistencies: many-values but one and only one designated value.}

\hypertarget{introduction-to-many-valued-logics}{%
\paragraph*{Introduction to Many-Valued
Logics}\label{introduction-to-many-valued-logics}}
\addcontentsline{toc}{paragraph}{Introduction to Many-Valued Logics}

These logics relax the principle of bivalence by allowing for
"indeterminate" or "gap" truth values. This allows for reasoning with
incomplete information or situations where truth is not fully
determined.

\hypertarget{finitely-many-valued-languages-with-singular-designation.}{%
\subparagraph*{Finitely Many-Valued Languages with Singular
Designation.}\label{finitely-many-valued-languages-with-singular-designation.}}
\addcontentsline{toc}{subparagraph}{Finitely Many-Valued Languages with
Singular Designation.}

\hypertarget{infinitely-many-valued-languages-with-singular-designation.}{%
\subparagraph*{Infinitely Many-Valued Languages with Singular
Designation.}\label{infinitely-many-valued-languages-with-singular-designation.}}
\addcontentsline{toc}{subparagraph}{Infinitely Many-Valued Languages
with Singular Designation.}

\hypertarget{chapter-3-constructive-proof-and-intuitionistic-logic}{%
\subsection*{Chapter 3: Constructive Proof and Intuitionistic
Logic}\label{chapter-3-constructive-proof-and-intuitionistic-logic}}
\addcontentsline{toc}{subsection}{Chapter 3: Constructive Proof and
Intuitionistic Logic}

\hypertarget{introduction-beyond-classical-logic---the-rise-of-constructive-proofs}{%
\subsubsection*{\texorpdfstring{\textbf{3.1 Introduction: Beyond
Classical Logic - The Rise of Constructive
Proofs}}{3.1 Introduction: Beyond Classical Logic - The Rise of Constructive Proofs}}\label{introduction-beyond-classical-logic---the-rise-of-constructive-proofs}}
\addcontentsline{toc}{subsubsection}{\textbf{3.1 Introduction: Beyond
Classical Logic - The Rise of Constructive Proofs}}

\hypertarget{motivation-for-constructive-approaches-limitations-of-classical-logic-in-representing-proofs-and-reasoning-about-incomplete-knowledge.}{%
\subsubsection{Motivation for constructive approaches: Limitations of
classical logic in representing proofs and reasoning about incomplete
knowledge.}\label{motivation-for-constructive-approaches-limitations-of-classical-logic-in-representing-proofs-and-reasoning-about-incomplete-knowledge.}}

\hypertarget{historical-context-brouwer-heyting-kolmogorov---pioneers-of-constructive-mathematics-and-intuitionistic-logic.}{%
\subsubsection{Historical context: Brouwer, Heyting, Kolmogorov -
Pioneers of constructive mathematics and intuitionistic
logic.}\label{historical-context-brouwer-heyting-kolmogorov---pioneers-of-constructive-mathematics-and-intuitionistic-logic.}}

\hypertarget{brouwer-heyting-kolmogorov-bhk-interpretation-formalizing-constructive-proof}{%
\subsubsection*{\texorpdfstring{\textbf{3.2 Brouwer-Heyting-Kolmogorov
(BHK) Interpretation: Formalizing Constructive
Proof}}{3.2 Brouwer-Heyting-Kolmogorov (BHK) Interpretation: Formalizing Constructive Proof}}\label{brouwer-heyting-kolmogorov-bhk-interpretation-formalizing-constructive-proof}}
\addcontentsline{toc}{subsubsection}{\textbf{3.2
Brouwer-Heyting-Kolmogorov (BHK) Interpretation: Formalizing
Constructive Proof}}

\hypertarget{intuitionistic-negation-not-p-as-the-absence-of-a-constructive-proof-for-p.}{%
\subsubsection{Intuitionistic negation: Not-P as the absence of a
constructive proof for
P.}\label{intuitionistic-negation-not-p-as-the-absence-of-a-constructive-proof-for-p.}}

\hypertarget{intuitionistic-implication-p-implies-q-only-if-there-is-a-procedure-for-constructing-q-from-a-proof-of-p.}{%
\subsubsection{Intuitionistic implication: P implies Q only if there is
a procedure for constructing Q from a proof of
P.}\label{intuitionistic-implication-p-implies-q-only-if-there-is-a-procedure-for-constructing-q-from-a-proof-of-p.}}

\hypertarget{logical-connectives-formalization-of-bhk-interpretations-for-conjunction-disjunction-existential-quantification-etc.}{%
\subsubsection{Logical connectives: Formalization of BHK interpretations
for conjunction, disjunction, existential quantification,
etc.}\label{logical-connectives-formalization-of-bhk-interpretations-for-conjunction-disjunction-existential-quantification-etc.}}

\hypertarget{constructive-dilemma-and-intuitionistic-logic-formalization-and-consequences}{%
\subsubsection*{\texorpdfstring{\textbf{3.3 Constructive Dilemma and
Intuitionistic Logic: Formalization and
Consequences}}{3.3 Constructive Dilemma and Intuitionistic Logic: Formalization and Consequences}}\label{constructive-dilemma-and-intuitionistic-logic-formalization-and-consequences}}
\addcontentsline{toc}{subsubsection}{\textbf{3.3 Constructive Dilemma
and Intuitionistic Logic: Formalization and Consequences}}

\hypertarget{law-of-excluded-middle-not-p-or-p-is-not-valid-in-intuitionistic-logic-reflecting-the-possibility-of-gaps-in-knowledge.}{%
\subsubsection{Law of excluded middle: Not-P or P is not valid in
intuitionistic logic, reflecting the possibility of gaps in
knowledge.}\label{law-of-excluded-middle-not-p-or-p-is-not-valid-in-intuitionistic-logic-reflecting-the-possibility-of-gaps-in-knowledge.}}

\hypertarget{law-of-double-negation-not-not-p-is-not-equivalent-to-p-highlighting-the-distinction-between-proving-a-proposition-and-proving-the-non-existence-of-a-proof.}{%
\subsubsection{Law of double negation: Not-not-P is not equivalent to P,
highlighting the distinction between proving a proposition and proving
the non-existence of a
proof.}\label{law-of-double-negation-not-not-p-is-not-equivalent-to-p-highlighting-the-distinction-between-proving-a-proposition-and-proving-the-non-existence-of-a-proof.}}

\hypertarget{derivations-and-proofs-in-intuitionistic-logic-formalization-of-sequent-calculus-and-natural-deduction-systems-for-intuitionistic-reasoning.}{%
\subsubsection{Derivations and proofs in intuitionistic logic:
Formalization of sequent calculus and natural deduction systems for
intuitionistic
reasoning.}\label{derivations-and-proofs-in-intuitionistic-logic-formalization-of-sequent-calculus-and-natural-deduction-systems-for-intuitionistic-reasoning.}}

\hypertarget{guxf6dels-incompleteness-theorems-applicability-and-limitations-within-the-intuitionistic-framework.}{%
\subsubsection{Gödel\textquotesingle s incompleteness theorems:
Applicability and limitations within the intuitionistic
framework.}\label{guxf6dels-incompleteness-theorems-applicability-and-limitations-within-the-intuitionistic-framework.}}

\hypertarget{superclassical-and-subclassical-calculi-expanding-the-logical-landscape}{%
\subsubsection*{\texorpdfstring{\textbf{3.4 Superclassical and
Subclassical Calculi: Expanding the Logical
Landscape}}{3.4 Superclassical and Subclassical Calculi: Expanding the Logical Landscape}}\label{superclassical-and-subclassical-calculi-expanding-the-logical-landscape}}
\addcontentsline{toc}{subsubsection}{\textbf{3.4 Superclassical and
Subclassical Calculi: Expanding the Logical Landscape}}

\hypertarget{superclassical-calculi-extensions-of-classical-logic-incorporating-features-from-intuitionism-like-constructive-implication.}{%
\subsubsection{Superclassical calculi: Extensions of classical logic
incorporating features from intuitionism, like constructive
implication.}\label{superclassical-calculi-extensions-of-classical-logic-incorporating-features-from-intuitionism-like-constructive-implication.}}

\hypertarget{subclassical-calculi-weakenings-of-classical-logic-capturing-different-degrees-of-constructive-content.}{%
\subsubsection{Subclassical calculi: Weakenings of classical logic,
capturing different degrees of constructive
content.}\label{subclassical-calculi-weakenings-of-classical-logic-capturing-different-degrees-of-constructive-content.}}

\hypertarget{examples-minimal-logic-relevant-logic-fuzzy-logic---exploring-alternative-truth-values-and-reasoning-patterns.}{%
\subsubsection{Examples: Minimal logic, relevant logic, fuzzy logic -
exploring alternative truth values and reasoning
patterns.}\label{examples-minimal-logic-relevant-logic-fuzzy-logic---exploring-alternative-truth-values-and-reasoning-patterns.}}

\hypertarget{superintuitionistic-and-subintuitionistic-calculi-pushing-the-boundaries-of-constructivism}{%
\subsubsection*{\texorpdfstring{\textbf{3.5 Superintuitionistic and
Subintuitionistic Calculi: Pushing the Boundaries of
Constructivism}}{3.5 Superintuitionistic and Subintuitionistic Calculi: Pushing the Boundaries of Constructivism}}\label{superintuitionistic-and-subintuitionistic-calculi-pushing-the-boundaries-of-constructivism}}
\addcontentsline{toc}{subsubsection}{\textbf{3.5 Superintuitionistic and
Subintuitionistic Calculi: Pushing the Boundaries of Constructivism}}

\hypertarget{superintuitionistic-calculi-extensions-of-intuitionistic-logic-with-additional-axioms-or-rules-enabling-reasoning-about-potential-counterexamples-and-hypothetical-scenarios.}{%
\subsubsection{Superintuitionistic calculi: Extensions of intuitionistic
logic with additional axioms or rules, enabling reasoning about
potential counterexamples and hypothetical
scenarios.}\label{superintuitionistic-calculi-extensions-of-intuitionistic-logic-with-additional-axioms-or-rules-enabling-reasoning-about-potential-counterexamples-and-hypothetical-scenarios.}}

\hypertarget{subintuitionistic-calculi-weakenings-of-intuitionistic-logic-allowing-for-reasoning-with-incomplete-information-or-non-constructive-proofs-in-specific-contexts.}{%
\subsubsection{Subintuitionistic calculi: Weakenings of intuitionistic
logic, allowing for reasoning with incomplete information or
non-constructive proofs in specific
contexts.}\label{subintuitionistic-calculi-weakenings-of-intuitionistic-logic-allowing-for-reasoning-with-incomplete-information-or-non-constructive-proofs-in-specific-contexts.}}

\hypertarget{examples-brouwer-logic-glivenko-logic---exploring-different-ways-of-handling-gaps-and-uncertainties-within-a-constructive-framework.}{%
\subsubsection{Examples: Brouwer logic, Glivenko logic - exploring
different ways of handling gaps and uncertainties within a constructive
framework.}\label{examples-brouwer-logic-glivenko-logic---exploring-different-ways-of-handling-gaps-and-uncertainties-within-a-constructive-framework.}}

\hypertarget{consistent-paracomplete-calculi-as-generalizations-bridging-the-gap}{%
\subsubsection*{\texorpdfstring{\textbf{3.6 Consistent Paracomplete
Calculi as Generalizations: Bridging the
Gap}}{3.6 Consistent Paracomplete Calculi as Generalizations: Bridging the Gap}}\label{consistent-paracomplete-calculi-as-generalizations-bridging-the-gap}}
\addcontentsline{toc}{subsubsection}{\textbf{3.6 Consistent Paracomplete
Calculi as Generalizations: Bridging the Gap}}

\hypertarget{paracomplete-logics-handling-gaps-in-knowledge-through-additional-truth-values-or-designated-unknown-states.}{%
\subsubsection{Paracomplete logics: Handling gaps in knowledge through
additional truth values or designated "unknown"
states.}\label{paracomplete-logics-handling-gaps-in-knowledge-through-additional-truth-values-or-designated-unknown-states.}}

\hypertarget{consistent-paracomplete-calculi-as-generalizations-of-subintuitionistic-and-subclassical-calculi-unifying-diverse-approaches-to-reasoning-with-incomplete-information.}{%
\subsubsection{Consistent paracomplete calculi as generalizations of
subintuitionistic and subclassical calculi: Unifying diverse approaches
to reasoning with incomplete
information.}\label{consistent-paracomplete-calculi-as-generalizations-of-subintuitionistic-and-subclassical-calculi-unifying-diverse-approaches-to-reasoning-with-incomplete-information.}}

\hypertarget{examples-kleene-logic-ux142ukasiewicz-logic---demonstrating-the-power-of-paracompleteness-for-representing-and-reasoning-about-real-world-situations.}{%
\subsubsection{Examples: Kleene logic, Łukasiewicz logic - demonstrating
the power of paracompleteness for representing and reasoning about
real-world
situations.}\label{examples-kleene-logic-ux142ukasiewicz-logic---demonstrating-the-power-of-paracompleteness-for-representing-and-reasoning-about-real-world-situations.}}

\hypertarget{brouwer-heyting-kolmogorov-interpretation-formalization-of-constructive-proof}{%
\subsubsection*{\texorpdfstring{\hfill\break
Brouwer-Heyting-Kolmogorov Interpretation: Formalization of Constructive
Proof}{ Brouwer-Heyting-Kolmogorov Interpretation: Formalization of Constructive Proof}}\label{brouwer-heyting-kolmogorov-interpretation-formalization-of-constructive-proof}}
\addcontentsline{toc}{subsubsection}{\hfill\break
Brouwer-Heyting-Kolmogorov Interpretation: Formalization of Constructive
Proof}

\hypertarget{constructive-dilemma-and-intuitionistic-logic-formalization-and-consequences-1}{%
\subsubsection*{Constructive Dilemma and Intuitionistic Logic:
Formalization and
Consequences}\label{constructive-dilemma-and-intuitionistic-logic-formalization-and-consequences-1}}
\addcontentsline{toc}{subsubsection}{Constructive Dilemma and
Intuitionistic Logic: Formalization and Consequences}

\hypertarget{part-ii-radical-paraconsistency}{%
\section*{Part II: Radical
Paraconsistency}\label{part-ii-radical-paraconsistency}}
\addcontentsline{toc}{section}{Part II: Radical Paraconsistency}

\hypertarget{chapter-4-many-signified-logics-and-languages}{%
\subsection*{Chapter 4: Many-Signified Logics and
Languages}\label{chapter-4-many-signified-logics-and-languages}}
\addcontentsline{toc}{subsection}{Chapter 4: Many-Signified Logics and
Languages}

\hypertarget{transcending-the-dichotomy-of-true-xor-false.-contradictions-as-multi-signifiers-and-the-impossibility-of-a-paraconsistent-discrete-many-signified-bivalent-language.}{%
\subsubsection*{Transcending the dichotomy of true xor false.
Contradictions as multi-signifiers and the impossibility of a
paraconsistent discrete many-signified bivalent
language.}\label{transcending-the-dichotomy-of-true-xor-false.-contradictions-as-multi-signifiers-and-the-impossibility-of-a-paraconsistent-discrete-many-signified-bivalent-language.}}
\addcontentsline{toc}{subsubsection}{Transcending the dichotomy of true
xor false. Contradictions as multi-signifiers and the impossibility of a
paraconsistent discrete many-signified bivalent language.}

\hypertarget{chapter-5-constructive-and-destructive-interference-in-formal-reasoning}{%
\subsection*{Chapter 5: Constructive and Destructive Interference in
Formal
Reasoning}\label{chapter-5-constructive-and-destructive-interference-in-formal-reasoning}}
\addcontentsline{toc}{subsection}{Chapter 5: Constructive and
Destructive Interference in Formal Reasoning}

\hypertarget{constructive-interference-inference-patterns-for-proof-generation-from-measurement.}{%
\subsubsection*{Constructive Interference: Inference Patterns for Proof
Generation from
measurement.}\label{constructive-interference-inference-patterns-for-proof-generation-from-measurement.}}
\addcontentsline{toc}{subsubsection}{Constructive Interference:
Inference Patterns for Proof Generation from measurement.}

\hypertarget{destructive-interference-inference-patterns-for-refutation-and-contradictions-from-measurement.}{%
\subsubsection*{Destructive Interference: Inference Patterns for
Refutation and Contradictions from
measurement.}\label{destructive-interference-inference-patterns-for-refutation-and-contradictions-from-measurement.}}
\addcontentsline{toc}{subsubsection}{Destructive Interference: Inference
Patterns for Refutation and Contradictions from measurement.}

\hypertarget{chapter-6-contradictory-languages-and-procedures}{%
\subsection*{Chapter 6: Contradictory Languages and
Procedures}\label{chapter-6-contradictory-languages-and-procedures}}
\addcontentsline{toc}{subsection}{Chapter 6: Contradictory Languages and
Procedures}

\hypertarget{introduction-beyond-the-binary-wall-of-truth}{%
\subsubsection*{\texorpdfstring{\textbf{6.1 Introduction: Beyond the
Binary Wall of
Truth}}{6.1 Introduction: Beyond the Binary Wall of Truth}}\label{introduction-beyond-the-binary-wall-of-truth}}
\addcontentsline{toc}{subsubsection}{\textbf{6.1 Introduction: Beyond
the Binary Wall of Truth}}

\hypertarget{the-allure-of-contradiction-challenging-classical-logics-limitations}{%
\subsubsection*{6.1.1 The allure of contradiction: Challenging classical
logic\textquotesingle s
limitations}\label{the-allure-of-contradiction-challenging-classical-logics-limitations}}
\addcontentsline{toc}{subsubsection}{6.1.1 The allure of contradiction:
Challenging classical logic\textquotesingle s limitations}

\hypertarget{a-spectrum-of-truth-values-beyond-the-binary-dichotomy-of-true-and-false}{%
\subsubsection*{6.1.2 A spectrum of truth values: Beyond the binary
dichotomy of true and
false}\label{a-spectrum-of-truth-values-beyond-the-binary-dichotomy-of-true-and-false}}
\addcontentsline{toc}{subsubsection}{6.1.2 A spectrum of truth values:
Beyond the binary dichotomy of true and false}

\hypertarget{defining-contradictory-languages-embracing-uncertainty-and-paradox}{%
\subsubsection*{\texorpdfstring{\textbf{6.2 Defining Contradictory
Languages: Embracing Uncertainty and Paradox}
}{6.2 Defining Contradictory Languages: Embracing Uncertainty and Paradox }}\label{defining-contradictory-languages-embracing-uncertainty-and-paradox}}
\addcontentsline{toc}{subsubsection}{\textbf{6.2 Defining Contradictory
Languages: Embracing Uncertainty and Paradox} }

\hypertarget{extending-the-truth-spectrum-multi-valued-logics-and-paraconsistency}{%
\subsubsection*{\texorpdfstring{6.2.1 Extending the truth spectrum:
Multi-valued logics and paraconsistency
}{6.2.1 Extending the truth spectrum: Multi-valued logics and paraconsistency }}\label{extending-the-truth-spectrum-multi-valued-logics-and-paraconsistency}}
\addcontentsline{toc}{subsubsection}{6.2.1 Extending the truth spectrum:
Multi-valued logics and paraconsistency }

\hypertarget{formalizing-contradictions-symbolizing-and-reasoning-with-contradictory-statements}{%
\subsubsection*{\texorpdfstring{6.2.2 Formalizing contradictions:
Symbolizing and reasoning with contradictory statements
}{6.2.2 Formalizing contradictions: Symbolizing and reasoning with contradictory statements }}\label{formalizing-contradictions-symbolizing-and-reasoning-with-contradictory-statements}}
\addcontentsline{toc}{subsubsection}{6.2.2 Formalizing contradictions:
Symbolizing and reasoning with contradictory statements }

\hypertarget{examples-of-contradictory-languages-from-fuzzy-logic-to-dialectical-logic}{%
\subsubsection*{6.2.3 Examples of contradictory languages: From fuzzy
logic to dialectical
logic}\label{examples-of-contradictory-languages-from-fuzzy-logic-to-dialectical-logic}}
\addcontentsline{toc}{subsubsection}{6.2.3 Examples of contradictory
languages: From fuzzy logic to dialectical logic}

\hypertarget{navigating-compatibility-when-contradictions-collide}{%
\subsubsection*{\texorpdfstring{\textbf{6.3 Navigating Compatibility:
When Contradictions Collide}
}{6.3 Navigating Compatibility: When Contradictions Collide }}\label{navigating-compatibility-when-contradictions-collide}}
\addcontentsline{toc}{subsubsection}{\textbf{6.3 Navigating
Compatibility: When Contradictions Collide} }

\hypertarget{consistent-compatibility-finding-a-common-ground-without-explosions}{%
\subsubsection*{\texorpdfstring{6.3.1 Consistent compatibility: Finding
a common ground without explosions
}{6.3.1 Consistent compatibility: Finding a common ground without explosions }}\label{consistent-compatibility-finding-a-common-ground-without-explosions}}
\addcontentsline{toc}{subsubsection}{6.3.1 Consistent compatibility:
Finding a common ground without explosions }

\hypertarget{paraconsistent-compatibility-embracing-contradictions-within-a-unified-framework}{%
\subsubsection*{\texorpdfstring{6.3.2 Paraconsistent compatibility:
Embracing contradictions within a unified framework
}{6.3.2 Paraconsistent compatibility: Embracing contradictions within a unified framework }}\label{paraconsistent-compatibility-embracing-contradictions-within-a-unified-framework}}
\addcontentsline{toc}{subsubsection}{6.3.2 Paraconsistent compatibility:
Embracing contradictions within a unified framework }

\hypertarget{the-incompatibility-of-contradictions-why-a-consistent-language-cant-coexist-with-paradoxes}{%
\subsubsection*{6.3.3 The incompatibility of contradictions: Why a
consistent language can\textquotesingle t coexist with
paradoxes}\label{the-incompatibility-of-contradictions-why-a-consistent-language-cant-coexist-with-paradoxes}}
\addcontentsline{toc}{subsubsection}{6.3.3 The incompatibility of
contradictions: Why a consistent language can\textquotesingle t coexist
with paradoxes}

\hypertarget{demystifying-inessential-paraconsistency-when-the-scaffolding-falls-away}{%
\subsubsection*{\texorpdfstring{\textbf{6.4 Demystifying Inessential
Paraconsistency: When the Scaffolding Falls Away}
}{6.4 Demystifying Inessential Paraconsistency: When the Scaffolding Falls Away }}\label{demystifying-inessential-paraconsistency-when-the-scaffolding-falls-away}}
\addcontentsline{toc}{subsubsection}{\textbf{6.4 Demystifying
Inessential Paraconsistency: When the Scaffolding Falls Away} }

\hypertarget{unmasking-the-inessential-paraconsistency-without-contradictions}{%
\subsubsection*{\texorpdfstring{6.4.1 Unmasking the inessential:
Paraconsistency without contradictions
}{6.4.1 Unmasking the inessential: Paraconsistency without contradictions }}\label{unmasking-the-inessential-paraconsistency-without-contradictions}}
\addcontentsline{toc}{subsubsection}{6.4.1 Unmasking the inessential:
Paraconsistency without contradictions }

\hypertarget{reduction-to-bivalence-simplifying-the-language-without-losing-meaning}{%
\subsubsection*{\texorpdfstring{6.4.2 Reduction to bivalence:
Simplifying the language without losing meaning
}{6.4.2 Reduction to bivalence: Simplifying the language without losing meaning }}\label{reduction-to-bivalence-simplifying-the-language-without-losing-meaning}}
\addcontentsline{toc}{subsubsection}{6.4.2 Reduction to bivalence:
Simplifying the language without losing meaning }

\hypertarget{the-tarskian-comfort-zone-maintaining-a-classical-foundation-for-the-metalanguage}{%
\subsubsection*{6.4.3 The Tarskian comfort zone: Maintaining a classical
foundation for the
metalanguage}\label{the-tarskian-comfort-zone-maintaining-a-classical-foundation-for-the-metalanguage}}
\addcontentsline{toc}{subsubsection}{6.4.3 The Tarskian comfort zone:
Maintaining a classical foundation for the metalanguage}

\hypertarget{unveiling-essential-paraconsistency-living-with-contradictions-as-neighbors}{%
\subsubsection*{\texorpdfstring{\textbf{6.5 Unveiling Essential
Paraconsistency: Living with Contradictions as Neighbors}
}{6.5 Unveiling Essential Paraconsistency: Living with Contradictions as Neighbors }}\label{unveiling-essential-paraconsistency-living-with-contradictions-as-neighbors}}
\addcontentsline{toc}{subsubsection}{\textbf{6.5 Unveiling Essential
Paraconsistency: Living with Contradictions as Neighbors} }

\hypertarget{the-price-of-contradictions-explosions-and-the-need-for-paraconsistency}{%
\subsubsection*{\texorpdfstring{6.5.1 The price of contradictions:
Explosions and the need for paraconsistency
}{6.5.1 The price of contradictions: Explosions and the need for paraconsistency }}\label{the-price-of-contradictions-explosions-and-the-need-for-paraconsistency}}
\addcontentsline{toc}{subsubsection}{6.5.1 The price of contradictions:
Explosions and the need for paraconsistency }

\hypertarget{splitting-the-paraconsistent-world-reducing-a-language-to-pairs-of-consistent-systems}{%
\subsubsection*{\texorpdfstring{6.5.2 Splitting the paraconsistent
world: Reducing a language to pairs of consistent systems
}{6.5.2 Splitting the paraconsistent world: Reducing a language to pairs of consistent systems }}\label{splitting-the-paraconsistent-world-reducing-a-language-to-pairs-of-consistent-systems}}
\addcontentsline{toc}{subsubsection}{6.5.2 Splitting the paraconsistent
world: Reducing a language to pairs of consistent systems }

\hypertarget{the-multi-valued-metalanguage-beyond-bivalence-to-accommodate-paradoxes}{%
\subsubsection*{\texorpdfstring{6.5.3 The multi-valued metalanguage:
Beyond bivalence to accommodate paradoxes
}{6.5.3 The multi-valued metalanguage: Beyond bivalence to accommodate paradoxes }}\label{the-multi-valued-metalanguage-beyond-bivalence-to-accommodate-paradoxes}}
\addcontentsline{toc}{subsubsection}{6.5.3 The multi-valued
metalanguage: Beyond bivalence to accommodate paradoxes }

\hypertarget{non-tarskian-options-exploring-alternative-foundations-for-the-metalanguage}{%
\subsubsection*{6.5.4 Non-Tarskian options: Exploring alternative
foundations for the
metalanguage}\label{non-tarskian-options-exploring-alternative-foundations-for-the-metalanguage}}
\addcontentsline{toc}{subsubsection}{6.5.4 Non-Tarskian options:
Exploring alternative foundations for the metalanguage}

\hypertarget{applications-and-implications-where-contradictions-matter}{%
\subsubsection*{\texorpdfstring{\textbf{6.6 Applications and
Implications: Where Contradictions Matter}
}{6.6 Applications and Implications: Where Contradictions Matter }}\label{applications-and-implications-where-contradictions-matter}}
\addcontentsline{toc}{subsubsection}{\textbf{6.6 Applications and
Implications: Where Contradictions Matter} }

\hypertarget{quantum-mechanics-embracing-superposition-and-uncertainty-with-paraconsistency}{%
\subsubsection*{\texorpdfstring{6.6.1 Quantum mechanics: Embracing
superposition and uncertainty with paraconsistency
}{6.6.1 Quantum mechanics: Embracing superposition and uncertainty with paraconsistency }}\label{quantum-mechanics-embracing-superposition-and-uncertainty-with-paraconsistency}}
\addcontentsline{toc}{subsubsection}{6.6.1 Quantum mechanics: Embracing
superposition and uncertainty with paraconsistency }

\hypertarget{artificial-intelligence-handling-ambiguity-and-vagueness-in-real-world-data}{%
\subsubsection*{\texorpdfstring{6.6.2 Artificial intelligence: Handling
ambiguity and vagueness in real-world data
}{6.6.2 Artificial intelligence: Handling ambiguity and vagueness in real-world data }}\label{artificial-intelligence-handling-ambiguity-and-vagueness-in-real-world-data}}
\addcontentsline{toc}{subsubsection}{6.6.2 Artificial intelligence:
Handling ambiguity and vagueness in real-world data }

\hypertarget{philosophical-discourse-new-perspectives-on-truth-paradox-and-the-limits-of-language}{%
\subsubsection*{6.6.3 Philosophical discourse: New perspectives on
truth, paradox, and the limits of
language}\label{philosophical-discourse-new-perspectives-on-truth-paradox-and-the-limits-of-language}}
\addcontentsline{toc}{subsubsection}{6.6.3 Philosophical discourse: New
perspectives on truth, paradox, and the limits of language}

\hypertarget{conclusion-a-glimpse-into-a-world-of-contradictions}{%
\subsubsection*{\texorpdfstring{\textbf{6.7 Conclusion: A Glimpse into a
World of Contradictions}
}{6.7 Conclusion: A Glimpse into a World of Contradictions }}\label{conclusion-a-glimpse-into-a-world-of-contradictions}}
\addcontentsline{toc}{subsubsection}{\textbf{6.7 Conclusion: A Glimpse
into a World of Contradictions} }

\hypertarget{the-future-of-contradictory-languages-challenges-opportunities-and-ongoing-research}{%
\subsubsection*{\texorpdfstring{6.7.1 The future of contradictory
languages: Challenges, opportunities, and ongoing research
}{6.7.1 The future of contradictory languages: Challenges, opportunities, and ongoing research }}\label{the-future-of-contradictory-languages-challenges-opportunities-and-ongoing-research}}
\addcontentsline{toc}{subsubsection}{6.7.1 The future of contradictory
languages: Challenges, opportunities, and ongoing research }

\hypertarget{contradictions-as-a-lens-rethinking-logic-language-and-the-nature-of-reality}{%
\subsubsection*{6.7.2 Contradictions as a lens: Rethinking logic,
language, and the nature of
reality}\label{contradictions-as-a-lens-rethinking-logic-language-and-the-nature-of-reality}}
\addcontentsline{toc}{subsubsection}{6.7.2 Contradictions as a lens:
Rethinking logic, language, and the nature of reality}

\hypertarget{section}{%
\subsubsection*{}\label{section}}
\addcontentsline{toc}{subsubsection}{}

\hypertarget{part-iii-supercalculi-subcalculi-isomorphic-calculi-and-dual-calculi}{%
\section*{Part III: Supercalculi, Subcalculi, Isomorphic Calculi, and
Dual
Calculi}\label{part-iii-supercalculi-subcalculi-isomorphic-calculi-and-dual-calculi}}
\addcontentsline{toc}{section}{Part III: Supercalculi, Subcalculi,
Isomorphic Calculi, and Dual Calculi}

\hypertarget{chapter-7-superclassical-and-subclassical-logics-expanding-and-restricting-logical-power}{%
\subsection*{Chapter 7: Superclassical and Subclassical Logics:
Expanding and Restricting Logical
Power}\label{chapter-7-superclassical-and-subclassical-logics-expanding-and-restricting-logical-power}}
\addcontentsline{toc}{subsection}{Chapter 7: Superclassical and
Subclassical Logics: Expanding and Restricting Logical Power}

\hypertarget{superclassical-logics-extending-classical-reasoning}{%
\subsubsection*{Superclassical Logics: Extending Classical
Reasoning}\label{superclassical-logics-extending-classical-reasoning}}
\addcontentsline{toc}{subsubsection}{Superclassical Logics: Extending
Classical Reasoning}

\hypertarget{subclassical-logics-weakening-classical-axioms-and-completeness}{%
\subsubsection*{Subclassical Logics: Weakening Classical Axioms and
Completeness}\label{subclassical-logics-weakening-classical-axioms-and-completeness}}
\addcontentsline{toc}{subsubsection}{Subclassical Logics: Weakening
Classical Axioms and Completeness}

\hypertarget{generalized-functional-incompleteness-or-constructive-completeness-in-subclassical-logics}{%
\subsubsection*{Generalized Functional Incompleteness or Constructive
Completeness in Subclassical
Logics}\label{generalized-functional-incompleteness-or-constructive-completeness-in-subclassical-logics}}
\addcontentsline{toc}{subsubsection}{Generalized Functional
Incompleteness or Constructive Completeness in Subclassical Logics}

\hypertarget{chapter-8-paracomplete-and-paraconsistent-supercalculi-and-subcalculi}{%
\subsection*{Chapter 8: Paracomplete and Paraconsistent Supercalculi and
Subcalculi}\label{chapter-8-paracomplete-and-paraconsistent-supercalculi-and-subcalculi}}
\addcontentsline{toc}{subsection}{Chapter 8: Paracomplete and
Paraconsistent Supercalculi and Subcalculi}

\hypertarget{paracompleteness-supercalculi-and-subcalculi}{%
\subsubsection*{Paracompleteness: Supercalculi and
Subcalculi}\label{paracompleteness-supercalculi-and-subcalculi}}
\addcontentsline{toc}{subsubsection}{Paracompleteness: Supercalculi and
Subcalculi}

\hypertarget{superintuitionistic-logics-strengthening-intuitionistic-axioms}{%
\paragraph*{Superintuitionistic Logics: Strengthening Intuitionistic
Axioms}\label{superintuitionistic-logics-strengthening-intuitionistic-axioms}}
\addcontentsline{toc}{paragraph}{Superintuitionistic Logics:
Strengthening Intuitionistic Axioms}

\hypertarget{subintuitionistic-logics-weakening-intuitionistic-axioms-and-constructive-completeness}{%
\paragraph*{Subintuitionistic Logics: Weakening Intuitionistic Axioms
and Constructive
Completeness}\label{subintuitionistic-logics-weakening-intuitionistic-axioms-and-constructive-completeness}}
\addcontentsline{toc}{paragraph}{Subintuitionistic Logics: Weakening
Intuitionistic Axioms and Constructive Completeness}

\hypertarget{paraconsistent-supercalculi-and-subcalculi}{%
\subsubsection*{Paraconsistent Supercalculi and
subcalculi}\label{paraconsistent-supercalculi-and-subcalculi}}
\addcontentsline{toc}{subsubsection}{Paraconsistent Supercalculi and
subcalculi}

\hypertarget{super-counter-intuitive-logics}{%
\paragraph*{Super-counter-intuitive
logics}\label{super-counter-intuitive-logics}}
\addcontentsline{toc}{paragraph}{Super-counter-intuitive logics}

\hypertarget{sub-counter-intuitive-logics}{%
\paragraph*{Sub-counter-intuitive
logics}\label{sub-counter-intuitive-logics}}
\addcontentsline{toc}{paragraph}{Sub-counter-intuitive logics}

\hypertarget{part-iv-correspondence-theories-between-logic-proof-computation-and-experiment}{%
\section*{Part IV: Correspondence Theories Between Logic, Proof,
Computation, and
Experiment}\label{part-iv-correspondence-theories-between-logic-proof-computation-and-experiment}}
\addcontentsline{toc}{section}{Part IV: Correspondence Theories Between
Logic, Proof, Computation, and Experiment}

\hypertarget{chapter-9-tarskis-t-schema-and-its-generalizations}{%
\subsection*{Chapter 9: Tarski\textquotesingle s T-Schema and its
Generalizations}\label{chapter-9-tarskis-t-schema-and-its-generalizations}}
\addcontentsline{toc}{subsection}{Chapter 9: Tarski\textquotesingle s
T-Schema and its Generalizations}

\hypertarget{models-and-interpretations-formalizing-classical-semantic-truth}{%
\subsubsection*{Models and Interpretations: Formalizing Classical
Semantic
Truth}\label{models-and-interpretations-formalizing-classical-semantic-truth}}
\addcontentsline{toc}{subsubsection}{Models and Interpretations:
Formalizing Classical Semantic Truth}

\hypertarget{generalizations-of-tarskis-t-schema-the-liars-sentence-as-a-metatheorem-for-self-reference.}{%
\subsubsection*{Generalizations of Tarski\textquotesingle s T-Schema:
the Liar's sentence as a Metatheorem for
Self-Reference.}\label{generalizations-of-tarskis-t-schema-the-liars-sentence-as-a-metatheorem-for-self-reference.}}
\addcontentsline{toc}{subsubsection}{Generalizations of
Tarski\textquotesingle s T-Schema: the Liar's sentence as a Metatheorem
for Self-Reference.}

\hypertarget{chapter-10-curry-howard-correspondence-proof-as-program}{%
\subsection*{Chapter 10: Curry-Howard Correspondence: Proof as
Program}\label{chapter-10-curry-howard-correspondence-proof-as-program}}
\addcontentsline{toc}{subsection}{Chapter 10: Curry-Howard
Correspondence: Proof as Program}

\hypertarget{logical-formulas-as-types-the-curry-howard-isomorphism}{%
\subsubsection*{Logical Formulas as Types: The Curry-Howard
Isomorphism}\label{logical-formulas-as-types-the-curry-howard-isomorphism}}
\addcontentsline{toc}{subsubsection}{Logical Formulas as Types: The
Curry-Howard Isomorphism}

\hypertarget{extracting-algorithms-from-proofs-operational-semantics-and-program-extraction}{%
\subsubsection*{Extracting Algorithms from Proofs: Operational Semantics
and Program
Extraction}\label{extracting-algorithms-from-proofs-operational-semantics-and-program-extraction}}
\addcontentsline{toc}{subsubsection}{Extracting Algorithms from Proofs:
Operational Semantics and Program Extraction}

\hypertarget{chapter-11-church-turing-deutsch-thesis-proof-program-and-measurement}{%
\subsection*{Chapter 11: Church-Turing-Deutsch Thesis: Proof, Program,
and
Measurement}\label{chapter-11-church-turing-deutsch-thesis-proof-program-and-measurement}}
\addcontentsline{toc}{subsection}{Chapter 11: Church-Turing-Deutsch
Thesis: Proof, Program, and Measurement}

\hypertarget{formalizing-algorithmic-power-the-church-turing-thesis}{%
\subsubsection*{Formalizing Algorithmic Power: The Church-Turing
Thesis}\label{formalizing-algorithmic-power-the-church-turing-thesis}}
\addcontentsline{toc}{subsubsection}{Formalizing Algorithmic Power: The
Church-Turing Thesis}

\hypertarget{extending-the-thesis-the-church-turing-deutsch-thesis-and-quantum-computation}{%
\subsubsection*{Extending the Thesis: The Church-Turing-Deutsch Thesis
and Quantum
Computation}\label{extending-the-thesis-the-church-turing-deutsch-thesis-and-quantum-computation}}
\addcontentsline{toc}{subsubsection}{Extending the Thesis: The
Church-Turing-Deutsch Thesis and Quantum Computation}

\hypertarget{part-v-metamathematics-metalanguages-and-the-limits-of-formalization}{%
\section*{Part V: Metamathematics, Metalanguages, and the Limits of
Formalization}\label{part-v-metamathematics-metalanguages-and-the-limits-of-formalization}}
\addcontentsline{toc}{section}{Part V: Metamathematics, Metalanguages,
and the Limits of Formalization}

\hypertarget{chapter-12-hierarchical-languages-and-metamathematics}{%
\subsection*{Chapter 12: Hierarchical Languages and
Metamathematics}\label{chapter-12-hierarchical-languages-and-metamathematics}}
\addcontentsline{toc}{subsection}{Chapter 12: Hierarchical Languages and
Metamathematics}

\hypertarget{uncomputable-languages-these-languages-cannot-be-enumerated-by-a-turing-machine-and-are-therefore-considered-uncomputable.}{%
\subsubsection*{Uncomputable Languages: These languages cannot be
enumerated by a Turing machine, and are therefore considered
uncomputable.}\label{uncomputable-languages-these-languages-cannot-be-enumerated-by-a-turing-machine-and-are-therefore-considered-uncomputable.}}
\addcontentsline{toc}{subsubsection}{Uncomputable Languages: These
languages cannot be enumerated by a Turing machine, and are therefore
considered uncomputable.}

\hypertarget{languages-of-doubt-denial-and-relativity-expanding-the-formal-landscape}{%
\subsubsection*{Languages of Doubt, Denial, and Relativity: Expanding
the Formal
Landscape}\label{languages-of-doubt-denial-and-relativity-expanding-the-formal-landscape}}
\addcontentsline{toc}{subsubsection}{Languages of Doubt, Denial, and
Relativity: Expanding the Formal Landscape}

\hypertarget{quantum-languages-and-quantum-logic-formalizing-quantum-phenomena}{%
\subsubsection*{Quantum Languages and Quantum Logic: Formalizing Quantum
Phenomena}\label{quantum-languages-and-quantum-logic-formalizing-quantum-phenomena}}
\addcontentsline{toc}{subsubsection}{Quantum Languages and Quantum
Logic: Formalizing Quantum Phenomena}

\hypertarget{chapter-13-self-metalanguages-and-metalogics}{%
\subsection*{Chapter 13: Self-Metalanguages and
Metalogics}\label{chapter-13-self-metalanguages-and-metalogics}}
\addcontentsline{toc}{subsection}{Chapter 13: Self-Metalanguages and
Metalogics}

\hypertarget{the-unique-power-of-paraconsistent-languages.}{%
\subsubsection*{The Unique Power of Paraconsistent
Languages.}\label{the-unique-power-of-paraconsistent-languages.}}
\addcontentsline{toc}{subsubsection}{The Unique Power of Paraconsistent
Languages.}

\hypertarget{formal-systems-reflecting-on-themselves-self-reference-guxf6dels-theorems-and-tarskis-theorems.}{%
\subsubsection*{Formal Systems Reflecting on Themselves: Self-Reference,
Gödel\textquotesingle s Theorems, and Tarski's
Theorems.}\label{formal-systems-reflecting-on-themselves-self-reference-guxf6dels-theorems-and-tarskis-theorems.}}
\addcontentsline{toc}{subsubsection}{Formal Systems Reflecting on
Themselves: Self-Reference, Gödel\textquotesingle s Theorems, and
Tarski's Theorems.}

\hypertarget{metalanguages-and-metalogics-formalizing-formalization}{%
\subsubsection*{Metalanguages and Metalogics: Formalizing
Formalization}\label{metalanguages-and-metalogics-formalizing-formalization}}
\addcontentsline{toc}{subsubsection}{Metalanguages and Metalogics:
Formalizing Formalization}

V3

Part I: Consistency: The Bedrock of Logic

1.1 Classical Consistency: The Foundation of Reasoning

1.2 Limits of Consistency: The Gödel Incompleteness Theorems

1.3 Beyond Bivalence: Fuzzy Logic and Many-Valued Systems

Part II: Lattice of Consistency: A Spectrum of Logics

2.1 Paraconsistent Logics: Embracing Contradictions

2.2 Dialetheism: Accepting True and False Simultaneously

2.3 Weak Kleene Logic: Accommodating Inconsistent Information

2.4 Paracompleteness: Leaving Truth Values Undetermined

Part III: Correspondences: Bridging Logic and Other Disciplines

3.1 Measurement and Paraconsistency: Reconciling Quantum Indeterminacy

3.2 Instrumentation and Inconsistency: Can We Measure the Unmeasurable?

3.3 Formal Languages and Paraconsistent Logic: Grammars of the
Unknowable

Part IV: Metalinguistics, Metamathematics, Metalogic, and Metaphysics:
Ascending the Ladder of Abstraction

4.1 The Scientific Method: A Paraconsistent Framework?

4.2 Hypothetico-Deductive Model: Reasoning with Uncertainty

4.3 Deductive Reasoning and Paraconsistency: Can We Reason from
Contradictions?

4.4 Chomsky Hierarchy and Paraconsistent Logics: Formalizing the
Unformalizable

Part V: Paraconsistency and the Edges of Knowledge

5.1 Semantics and Paraconsistency: Meaning in the Face of Contradictions

5.2 Metalanguage and Paraconsistency: Talking about the Untalkable

5.3 Metatheory and Paraconsistency: Justifying the Unjustifiable

5.4 Possibility Theory and Paraconsistency: Reasoning about the
Unknowable

5.5 Possible Worlds and Paraconsistency: Navigating the Multiverse of
Contradictions

V4

\textbf{I. Foundations of Certainty: The Lattice of Logic}

1.1 Classical Logic: Cornerstone of Deduction 1.2 Beyond Bivalence: Many
Valued and Fuzzy Systems 1.3 The Lattice of Logics: A Spectrum of Formal
Reasoning

\textbf{II. Measuring the Unmeasurable: Quantum Entanglements and
Paraconsistency}

2.1 Quantum Mechanics: Where Measurement Defies Certainty 2.2
Paraconsistency: Embracing Contradictions within Formal Systems 2.3
Dialetheism: Two Sides of the Coin - True and False

\textbf{III. Formalizing the Unknowable: Languages of Possibility and
Impossibility}

3.1 Epistemic and Subjunctive Modalities: Reasoning about the
Potentially True 3.2 Possible Worlds: Exploring Alternative Realities
3.3 Van Wijngaarden Grammar: Formalizing Uncertainty and Openness

\textbf{IV. The Scientific Scaffolding: Theory, Method, and Observation}

4.1 The Scientific Method: Navigating the Labyrinth of Uncertainty 4.2
Hypothetico-Deductive Reasoning: Building Bridges between Theory and
Observation 4.4 Deductive Reasoning: Chains of Implication and the Quest
for Certainty

\textbf{V. The Meta-level: Language about Language, Knowledge about
Knowledge}

5.1 Metalanguages: Speaking about the Language of Logic itself 5.2
Metatheory: Justifying the Foundations of Formal Systems 5.3 Semantic
Invariants: Meaning Amidst Shifting Sands 5.4 Systems and Automorphisms:
Structure and Transformations

\textbf{VI. Beyond the Horizon: Open Questions and Future Directions}

6.1 Limits of Computation: Landauer\textquotesingle s Principle and the
Bekenstein Bound 6.2 Holographic Principle and the Information Paradox
6.3 Quantum Speed Limit and the Church-Turing-Deutsch Principle

V5

\hypertarget{part-i-consistency}{%
\subsection*{\texorpdfstring{Part I: Consistency
}{Part I: Consistency }}\label{part-i-consistency}}
\addcontentsline{toc}{subsection}{Part I: Consistency }

\hypertarget{part-ii-lattice-of-consistency}{%
\subsection*{\texorpdfstring{Part II: Lattice of Consistency
}{Part II: Lattice of Consistency }}\label{part-ii-lattice-of-consistency}}
\addcontentsline{toc}{subsection}{Part II: Lattice of Consistency }

\hypertarget{part-iii-correspondences}{%
\subsection*{\texorpdfstring{Part III: Correspondences
}{Part III: Correspondences }}\label{part-iii-correspondences}}
\addcontentsline{toc}{subsection}{Part III: Correspondences }

\hypertarget{part-iv-languages-of-possibility}{%
\subsection*{\texorpdfstring{Part IV: Languages of Possibility
}{Part IV: Languages of Possibility }}\label{part-iv-languages-of-possibility}}
\addcontentsline{toc}{subsection}{Part IV: Languages of Possibility }

\hypertarget{proof-of-possibility}{%
\subsection*{\texorpdfstring{Proof of Possibility
}{Proof of Possibility }}\label{proof-of-possibility}}
\addcontentsline{toc}{subsection}{Proof of Possibility }

\hypertarget{mathematical-possibility}{%
\subsection*{\texorpdfstring{Mathematical possibility
}{Mathematical possibility }}\label{mathematical-possibility}}
\addcontentsline{toc}{subsection}{Mathematical possibility }

\hypertarget{physical-possibility}{%
\subsection*{\texorpdfstring{Physical possibility
}{Physical possibility }}\label{physical-possibility}}
\addcontentsline{toc}{subsection}{Physical possibility }

\hypertarget{quantum-possibility}{%
\subsection*{\texorpdfstring{Quantum possibility
}{Quantum possibility }}\label{quantum-possibility}}
\addcontentsline{toc}{subsection}{Quantum possibility }

\hypertarget{part-v-languages-of-impossibility}{%
\subsection*{\texorpdfstring{Part V: Languages of Impossibility
}{Part V: Languages of Impossibility }}\label{part-v-languages-of-impossibility}}
\addcontentsline{toc}{subsection}{Part V: Languages of Impossibility }

\hypertarget{proof-of-impossibility}{%
\subsection*{\texorpdfstring{Proof of Impossibility
}{Proof of Impossibility }}\label{proof-of-impossibility}}
\addcontentsline{toc}{subsection}{Proof of Impossibility }

\hypertarget{mathematical-impossibility}{%
\subsection*{\texorpdfstring{Mathematical Impossibility
}{Mathematical Impossibility }}\label{mathematical-impossibility}}
\addcontentsline{toc}{subsection}{Mathematical Impossibility }

\hypertarget{physical-impossibility}{%
\subsection*{\texorpdfstring{Physical Impossibility
}{Physical Impossibility }}\label{physical-impossibility}}
\addcontentsline{toc}{subsection}{Physical Impossibility }

\hypertarget{quantum-impossibility}{%
\subsection*{\texorpdfstring{Quantum Impossibility
}{Quantum Impossibility }}\label{quantum-impossibility}}
\addcontentsline{toc}{subsection}{Quantum Impossibility }

\hypertarget{part-vi-metalinguistics-metamathematics-and-metaphysics}{%
\subsection*{Part VI: Metalinguistics, Metamathematics, and
Metaphysics}\label{part-vi-metalinguistics-metamathematics-and-metaphysics}}
\addcontentsline{toc}{subsection}{Part VI: Metalinguistics,
Metamathematics, and Metaphysics}

V6

I. Beyond Bivalence: Rethinking the Foundations of Logic

1.1. Classical Logic: The Cornerstone of Reason 1.2.
Gödel\textquotesingle s Incompleteness: Unveiling the Limits of
Certainty 1.3. Fuzzy Logic and Many-Valued Systems: Expanding the
Spectrum of Truth

II. Navigating the Lattice of Logics: A Landscape of Possibilities

2.1. Paraconsistent Logics: Reasoning with Contradictions 2.2.
Dialetheism: Embracing Opposing Truths 2.3. Weak Kleene Logic:
Integrating Open Information 2.4. Paracompleteness: Embracing
Undetermined Truths

III. Bridging the Gap: From Logic to the Real World

3.1. Measurement and the Unmeasurable: Quantum Mechanics and
Paraconsistency 3.2. Instrumentation and the Indeterminate: Can We
Measure the Unknown? 3.3. Formal Languages and Paraconsistency: Grammars
for the Unconventional

IV. Languages of Possibility: Exploring the Uncharted

4.1. The Scientific Method: A Paraconsistent Framework for Discovery?
4.2. Hypothetico-Deductive Reasoning: Building Theories Beyond Certainty
4.3. Deductive Reasoning and Open Premises: Can We Derive Knowledge from
Uncertainty? 4.4. The Chomsky Hierarchy and Paraconsistency: Formalizing
the Unformalizable

V. Languages of Impossibility: Confronting the Unknowable

5.1. Semantics and the Unspeakable: Meaning-Making in the Face of
Contradictions 5.2. Metalanguage and the Unutterable: Speaking about the
Unthinkable 5.3. Metatheory and the Unjustifiable: Foundations Beyond
Proof 5.4. Possibility Theory and the Unforeseen: Reasoning about the
Unknowable

VI. Metalinguistics, Metamathematics, Metalogic, and Metaphysics:
Ascending the Abstraction Ladder

6.1. Invariance and Paraconsistency: Fixed Points Amidst Shifting Sands
6.2. Systems and Paraconsistency: Coherence Within Contradiction 6.3.
Automorphisms and Paraconsistency: Transformations Preserving the
Unconventional 6.4. Relations and Paraconsistency: Connecting the
Unconnectable

\hypertarget{chapter-contradictory-languages-and-contradictory-procedures}{%
\section*{Chapter: Contradictory Languages and Contradictory
Procedures}\label{chapter-contradictory-languages-and-contradictory-procedures}}
\addcontentsline{toc}{section}{Chapter: Contradictory Languages and
Contradictory Procedures}

\textbf{6.1 Introduction: Beyond the Binary Barrier}

6.1.1 Motivation: Challenging the limitations of classical logic and its
binary framework.

6.1.2 Expanding the Truth Spectrum: Exploring alternative truth values
beyond true and false.

\textbf{6.2 Navigating the Realm of Contradictory Languages:}

6.2.1 Representation of Contradictions:

6.2.1.1 Gluts and Gaps: Formalizing states beyond the binary truth
spectrum.

6.2.1.2 Modal Operators: Embracing uncertainty and possibility through
modalities.

6.2.1.3 Fuzzy Logic: Quantifying degrees of truth and falsity.

6.2.2 Inessential Paraconsistency: Languages with latent paraconsistent
features.

6.2.2.1 Identification and Reduction: Recognizing and simplifying
inessentially paraconsistent systems.

6.2.2.2 Role of the Metalanguage: Maintaining bivalence or venturing
beyond?

6.2.3 Essential Paraconsistency: Languages embracing inherent
contradictions.

6.2.3.1 Formalization of Contradictions: Integrating paradoxes within
the language structure.

6.2.3.2 Decomposition into Consistent Systems: Splitting the
paraconsistent world into compatible parts.

6.2.3.3 Multi-Valued Metalanguages: Accommodating paradoxes beyond
classical bivalence.

6.2.4 Compatibility and Consistency: Finding common ground amidst
contradictions.

6.2.4.1 Consistent Compatibility: Merging languages without triggering
logical explosions.

6.2.4.2 Paraconsistent Compatibility: Embracing contradictions within a
unified framework.

6.2.4.3 Limits of Compatibility: Why some paradoxes cannot coexist
peacefully.

\textbf{6.3 Contradictory Procedures and Algorithms}

6.3.1 Processing Contradictory Inputs: Algorithms handling ambiguous
data.

6.3.1.1 Uncertainty and Ambiguity: Fuzzy reasoning and probabilistic
approaches.

6.3.1.2 Conflicting Evidence: Aggregation and filtering of contradictory
information.

6.3.2 Generating Contradictory Outputs: Algorithms producing paradoxical
results.

6.3.2.1 Quantum Algorithms: Exploring superposition and entanglement as
sources of contradiction.

6.3.2.2 Non-deterministic Algorithms: Embracing multiple possibilities
and contradictory outcomes.

Two languages L0 and L1 are said to be consistently compatible if they
have a common consistent extension; this is equivalent to saying the
union of L0 and L1 is consistent.

Two languages L0 and L1 are said to be paraconsistently compatible if
they have a common paraconsistent consistent extension; this is
equivalent to saying the union of L0 and L1 is paraconsistent.

If either L0 or L1 is paraconsistent and has a contradiction then the
other can not be consistent if they are consistently compatible.

if L0 and L1 are consistently compatible then L1 and L0 can be reduced
to a consistent languages or are a consistent languages.

A paraconsistent language is inessentially paraconsistent if and only if
it has no contradictions; it is reducible to a consistent language;
inessentially paraconsistent languages are reducible to bivalent
languages.

Every inessentially paraconsistent language has a 2-value T-schema that
holds in its metalanguage, and its metalanguage is Tarskian.

If a paraconsistent language L is essentially paraconsistent then L has
contradictions; it is reducible to a pair of consistent language for
each contradiction in the paraconsistent language. Every essentially
paraconsistent language L has a (3 or more)-valued T-schema that holds
in L\textquotesingle s metalanguage or L\textquotesingle s metalanguage
has no T-Schema; the metalanguage can not be bivalent.

If the metalanguage of an essentially paraconsistent language was
bivalent then any contradiction in the essentially paraconsistent
language would be logically explosive.

Every essentially paraconsistent language is at least partially
independent of some consistent language.

Incompleteness of consistent extensions: While essentially
paraconsistent languages might have consistent extensions, these
extensions often do not capture the full richness and nuances of the
original language. This suggests that some aspects of the paraconsistent
language remain independent of the consistent extension.

\hypertarget{references}{%
\section*{References}\label{references}}
\addcontentsline{toc}{section}{References}

L.E.J. Brouwer,

Arend Heyting,

Alfred Tarski,

Kurt Gödel,

Paola Zizzi

Noam Chomsky

\chapter{DMALC}
\begin{abstract}

\end{abstract}

\section{Operational Rules}
Uncertain about proper dualization that preserves the proper ordering for the dual.

	\[
	\infer{\bot \vdash}{}
	\quad
	\infer{C \vdash \Delta_{L}, \bot, \Delta_{R}}{C \vdash \Delta_{L}, \Delta_{R}}
	\]

	\[
	\infer{0 \vdash \Delta}{}
	\quad
	\infer{C \vdash \Delta_{L}, \top, \Delta_{R}}{}
	\]

	\[
	\infer{B \upand A \vdash \Delta_{L}, \Delta_{R}}{B \vdash \Delta_{L} & A \vdash \Delta_{R}}
	\quad
	\infer{C \vdash \Delta_{L}, B \upand A, \Delta_{R}}{C \vdash \Delta_{L}, B, A, \Delta_{R}}
	\]

	\[
	\infer{A \nleftarrow B \vdash \Delta}{B \vdash \Delta, A}
	\quad
	\infer{C \vdash \Delta_{L}, A \nleftarrow B, \Delta, \Delta_{R}}{A \vdash \Delta & C \vdash \Delta_{L}, B, \Delta_{R}}
	\]

	\[
	\infer{A \nrightarrow B \vdash \Delta}{B \vdash A, \Delta}
	\quad
	\infer{C \vdash \Delta_{L}, \Delta, B \nrightarrow A, \Delta_{R}}{A \vdash \Delta & C \vdash \Delta_{L}, B, \Delta_{R}}
	\]

	\[
	\infer{B \oplus A \vdash \Delta}{B \vdash \Delta & A \vdash \Delta}
	\quad
	\infer{C \vdash \Delta_{L}, B \oplus A, \Delta_{R}}{C \vdash \Delta_{L}, A, \Delta_{R}}
	\quad
	\infer{C \vdash \Delta_{L}, B \oplus A, \Delta_{R}}{C \vdash \Delta_{L}, B, \Delta_{R}}
	\]

	\[
	\infer{A \& B \vdash \Delta}{A \vdash \Delta}
	\quad
	\infer{A \& B \vdash \Delta}{B \vdash \Delta}
	\quad
	\infer{ C \vdash \Delta_{L}, B \& A, \Delta_{R}}{C \vdash \Delta_{L}, B, \Delta_{R} & C \vdash \Delta_{L}, A, \Delta_{R}}
	\]

	\[
	\infer{\neg A \vdash \Delta}{ \vdash A, \Delta}
	\quad
	\infer{\neg A \vdash \Delta}{ \vdash \Delta, A}
	\quad
	\infer{\vdash \neg A, \Delta}{ A \vdash \Delta}
	\quad
	\infer{\vdash \Delta, \neg A}{ A \vdash \Delta}
	\]

\section{Structural Rules}
Uses a variation of cut right.

	\[
	\infer[Id]{A \vdash A}{}
	\]

	\[
	\infer[CutR]{C \vdash \Delta_{L}, \Delta, \Delta_{R} }{C \vdash \Delta_{L}, A, \Delta_{R} & A \vdash \Delta}
	\]

\chapter{Existential Order Additive Sequent}

\documentclass{article}

\usepackage{amsmath}
\usepackage{ebproof}
\usepackage{fullpage}
\usepackage[utf8]{inputenc}
\usepackage{newunicodechar}
\usepackage{stix}

\newunicodechar{Γ}{\Gamma}
\newunicodechar{Δ}{\Delta}

\newunicodechar{⊢}{\vdash}

\newunicodechar{∀}{\forall}
\newunicodechar{∃}{\exists}

\newunicodechar{⊕}{\oplus}
\newunicodechar{¬}{\neg}

\setlength{\parindent}{0em}

\author{James Martin, Ian D.L.N. Mclean}
\title{Existential Ordinary Additive Sequent Calculus}

\begin{document}

\maketitle

\begin{abstract}
Strictly visible additive order calculus with existential quantification.
\end{abstract}

\section{Structural Rules}

\begin{center}
	\[
	\begin{prooftree}
	\infer0[Id]{A ⊢ A}
	\end{prooftree}
	\]

	\[
	\begin{prooftree}
	\hypo{Γ ⊢ A}
	\hypo{A ⊢ Δ}
	\infer2[cut]{Γ ⊢ Δ}
	\end{prooftree}
	\]
\end{center}

\section{Quantifiers}
	\begin{center}
		\[
		\begin{prooftree}
		\hypo{P[y/x] ⊢ Δ}
		\infer1{∃x.P(x) ⊢ Δ}
		\end{prooftree}
		\quad
		\begin{prooftree}
		\hypo{Γ ⊢ P[t/x]}
		\infer1{Γ ⊢ ∃x.P(x)}
		\end{prooftree}
		\]
		in $∃L$ the variable y must not occur free anywhere in the respective lower sequents.
	\end{center}

\section{Operational Rules}
	\begin{center}
				\subsection{Negation}
				\begin{center}
					\[
					\begin{prooftree}
					\hypo{ ⊢ A}
					\infer1{¬ A ⊢}
					\end{prooftree}
					\quad
					\begin{prooftree}
					\hypo{A ⊢ }
					\infer1{⊢ ¬ A}
					\end{prooftree}
					\]
				\end{center}


				\subsection{Negative Junctions}
				\begin{center}
					\[
					\begin{prooftree}
					\hypo{⊢ A}
					\hypo{⊢ B}
					\infer2{¬A ⊕ ¬B ⊢ }
					\end{prooftree}
					\quad
					\begin{prooftree}
					\hypo{A ⊢ }
					\infer1{ ⊢ ¬A ⊕ ¬B}
					\end{prooftree}
					\quad
					\begin{prooftree}
					\hypo{B ⊢ }
					\infer1{ ⊢ ¬A ⊕ ¬B}
					\end{prooftree}
					\]

					\[
					\begin{prooftree}
					\hypo{⊢ B}
					\hypo{ ⊢ A}
					\infer2{¬B ⊕ ¬A ⊢ }
					\end{prooftree}
					\quad
					\begin{prooftree}
					\hypo{A ⊢ }
					\infer1{ ⊢ ¬B ⊕ ¬A}
					\end{prooftree}
					\quad
					\begin{prooftree}
					\hypo{B ⊢ }
					\infer1{ ⊢ ¬B ⊕ ¬A}
					\end{prooftree}
					\]

					\[
					\begin{prooftree}
					\hypo{ ⊢ A}
					\infer1{¬A \& ¬B ⊢ }
					\end{prooftree}
					\quad
					\begin{prooftree}
					\hypo{ ⊢ B}
					\infer1{¬A \& ¬B ⊢ }
					\end{prooftree}
					\quad
					\begin{prooftree}
					\hypo{A ⊢ }
					\hypo{B ⊢ }
					\infer2{⊢ ¬A \& ¬B}
					\end{prooftree}
					\]

					\[
					\begin{prooftree}
					\hypo{ ⊢ A}
					\infer1{¬B \& ¬A ⊢ }
					\end{prooftree}
					\quad
					\begin{prooftree}
					\hypo{ ⊢ B}
					\infer1{¬B \& ¬A ⊢ }
					\end{prooftree}
					\quad
					\begin{prooftree}
					\hypo{B ⊢ }
					\hypo{A ⊢ }
					\infer2{ ⊢ ¬ B \& ¬ A}
					\end{prooftree}
					\]
				\end{center}

				\subsection{Conditionals}
				\begin{center}
					\[
					\begin{prooftree}
					\hypo{⊢ B}
					\hypo{A ⊢ }
					\infer2{¬B ⊕ A ⊢}
					\end{prooftree}
					\quad
					\begin{prooftree}
					\hypo{A ⊢ }
					\hypo{⊢ B}
					\infer2{A ⊕ ¬B ⊢}
					\end{prooftree}
					\quad
					\begin{prooftree}
					\hypo{Γ ⊢ A}
					\infer1{Γ ⊢ A ⊕ ¬B}
					\end{prooftree}
					\quad
					\begin{prooftree}
					\hypo{B ⊢ }
					\infer1{⊢ A ⊕ ¬B}
					\end{prooftree}
					\]

					\[
					\begin{prooftree}
					\hypo{ ⊢ A}
					\hypo{B ⊢ }
					\infer2{¬A ⊕ B ⊢ }
					\end{prooftree}
					\quad
					\begin{prooftree}
					\hypo{B ⊢ }
					\hypo{ ⊢ A}
					\infer2{B ⊕ ¬A ⊢ }
					\end{prooftree}
					\quad
					\begin{prooftree}
					\hypo{A ⊢ }
					\infer1{ ⊢ B ⊕ ¬A}
					\end{prooftree}
					\quad
					\begin{prooftree}
					\hypo{Γ ⊢ B}
					\infer1{Γ ⊢ B ⊕ ¬A}
					\end{prooftree}
					\]

					\[
					\begin{prooftree}
					\hypo{A ⊢ Δ}
					\infer1{A \& ¬B ⊢ Δ}
					\end{prooftree}
					\quad
					\begin{prooftree}
					\hypo{ ⊢ B}
					\infer1{A \& ¬B ⊢ }
					\end{prooftree}
					\quad
					\begin{prooftree}
					\hypo{⊢ A}
					\hypo{B ⊢ }
					\infer2{⊢ A \& ¬B}
					\end{prooftree}
					\quad
					\begin{prooftree}
					\hypo{B ⊢ }
					\hypo{⊢ A}
					\infer2{⊢ ¬B \& A}
					\end{prooftree}
					\]

					\[
					\begin{prooftree}
					\hypo{ ⊢ A}
					\infer1{B \& ¬A ⊢ }
					\end{prooftree}
					\quad
					\begin{prooftree}
					\hypo{B ⊢ Δ}
					\infer1{B \& ¬A ⊢ Δ}
					\end{prooftree}
					\quad
					\begin{prooftree}
					\hypo{ ⊢ B}
					\hypo{A ⊢ }
					\infer2{ ⊢ B \& ¬ A}
					\end{prooftree}
					\quad
					\begin{prooftree}
					\hypo{A ⊢ }
					\hypo{ ⊢ B}
					\infer2{ ⊢ ¬A \& B}
					\end{prooftree}
					\]
				\end{center}

				\subsection{Junctions}
				\begin{center}
					\[
					\begin{prooftree}
					\hypo{A ⊢ Δ}
					\hypo{B ⊢ Δ}
					\infer2{A ⊕ B ⊢ Δ}
					\end{prooftree}
					\quad
					\begin{prooftree}
					\hypo{Γ ⊢ A}
					\infer1{Γ ⊢ A ⊕ B}
					\end{prooftree}
					\quad
					\begin{prooftree}
					\hypo{Γ ⊢ B}
					\infer1{Γ ⊢ A ⊕ B}
					\end{prooftree}
					\]

					\[
					\begin{prooftree}
					\hypo{B ⊢ Δ}
					\hypo{A ⊢ Δ}
					\infer2{B ⊕ A ⊢ Δ}
					\end{prooftree}
					\quad
					\begin{prooftree}
					\hypo{Γ ⊢ A}
					\infer1{Γ ⊢ B ⊕ A}
					\end{prooftree}
					\quad
					\begin{prooftree}
					\hypo{Γ ⊢ B}
					\infer1{Γ ⊢ B ⊕ A}
					\end{prooftree}
					\]

					\[
					\begin{prooftree}
					\hypo{A ⊢ Δ}
					\infer1{A \& B ⊢ Δ}
					\end{prooftree}
					\quad
					\begin{prooftree}
					\hypo{B ⊢ Δ}
					\infer1{A \& B ⊢ Δ}
					\end{prooftree}
					\quad
					\begin{prooftree}
					\hypo{Γ ⊢ A}
					\hypo{Γ ⊢ B}
					\infer2{Γ ⊢ A \& B}
					\end{prooftree}
					\]

					\[
					\begin{prooftree}
					\hypo{A ⊢ Δ}
					\infer1{B \& A ⊢ Δ}
					\end{prooftree}
					\quad
					\begin{prooftree}
					\hypo{B ⊢ Δ}
					\infer1{B \& A ⊢ Δ}
					\end{prooftree}
					\quad
					\begin{prooftree}
					\hypo{Γ ⊢ B}
					\hypo{Γ ⊢ A}
					\infer2{Γ ⊢ B \& A}
					\end{prooftree}
					\]
				\end{center}

				\subsection{Biconditionals}
				\begin{center}
					\[
					\begin{prooftree}
					\hypo{⊢ A}
					\hypo{B ⊢ }
					\infer2{(¬A ⊕ B) \& (¬B ⊕ A) ⊢ }
					\end{prooftree}
					\quad
					\begin{prooftree}
					\hypo{⊢ B}
					\hypo{A ⊢ }
					\infer2{(¬A ⊕ B) \& (¬B ⊕ A) ⊢ }
					\end{prooftree}
					\quad
					\begin{prooftree}
					\hypo{A ⊢ }
					\hypo{B ⊢ }
					\infer2{ ⊢ (¬A ⊕ B) \& (¬B ⊕ A)}
					\end{prooftree}
					\quad
					\begin{prooftree}
					\hypo{Γ ⊢ B}
					\hypo{Γ ⊢ A}
					\infer2{Γ ⊢ (¬A ⊕ B) \& (¬B ⊕ A)}
					\end{prooftree}
					\]

					\[
					\begin{prooftree}
					\hypo{A ⊢ }
					\hypo{⊢A}
					\infer2{(A \& B) ⊕  (¬A \& ¬B) ⊢ }
					\end{prooftree}
					\quad
					\begin{prooftree}
					\hypo{A ⊢ }
					\hypo{⊢B}
					\infer2{(A \& B) ⊕  (¬A \& ¬B) ⊢ }
					\end{prooftree}
					\quad
					\begin{prooftree}
					\hypo{B ⊢ }
					\hypo{⊢A}
					\infer2{(A \& B) ⊕  (¬A \& ¬B) ⊢ }
					\end{prooftree}
					\quad
					\begin{prooftree}
					\hypo{B ⊢ }
					\hypo{⊢B}
					\infer2{(A \& B) ⊕  (¬A \& ¬B) ⊢ }
					\end{prooftree}
					\]

					\[
					\quad
					\begin{prooftree}
					\hypo{Γ ⊢ A}
					\hypo{Γ ⊢ B}
					\infer2{Γ ⊢ (A \& B) ⊕  (¬A \& ¬B)}
					\end{prooftree}
					\quad
					\begin{prooftree}
					\hypo{B ⊢ }
					\hypo{A ⊢ }
					\infer2{ ⊢ (A \& B) ⊕  (¬A \& ¬B)}
					\end{prooftree}
					\]

					TODO: XOR-like
				\end{center}
\end{center}

\part{Theorems}
	\begin{center}
	\end{center}

\end{document}


\chapter{First Order Additive Sequent}
}

\maketitle

\begin{abstract}
Strictly visible additive order calculus with first order quantification.
\end{abstract}

\section{Structural Rules}

\begin{center}
	\[
	\begin{prooftree}
	\infer0[Id]{A ⊢ A}
	\end{prooftree}
	\]
	
	\[
	\begin{prooftree}
	\hypo{Γ ⊢ A}
	\hypo{A ⊢ Δ}
	\infer2[cut]{Γ ⊢ Δ}
	\end{prooftree}
	\]
\end{center}

\section{Quantifiers}
	\begin{center}
		Standard Eigenvariable Condition Quantifiers
		\[
		\begin{prooftree}
		\hypo{P[t/x] ⊢ Δ}
		\infer1{∀x.P(x) ⊢ Δ}
		\end{prooftree}
		\quad
		\begin{prooftree}
		\hypo{Γ ⊢ P[y/x]}
		\infer1{Γ ⊢ ∀x.P(x)}
		\end{prooftree}
		\]
		in $∀R$ the variable y must not occur free anywhere in the lower sequent.
		\[
		\begin{prooftree}
		\hypo{P[y/x] ⊢ Δ}
		\infer1{∃x.P(x) ⊢ Δ}
		\end{prooftree}
		\quad
		\begin{prooftree}
		\hypo{Γ ⊢ P[t/x]}
		\infer1{Γ ⊢ ∃x.P(x)}
		\end{prooftree}
		\]
		in $∃L$ the variable y must not occur free anywhere in the lower sequent.
		\[
		\begin{prooftree}
		\hypo{P(y)⊢}
		\infer1{¬∀x.¬P(x) ⊢ }
		\end{prooftree}
		\quad
		\begin{prooftree}
		\hypo{⊢P(t)}
		\infer1{⊢ ¬∀x.¬P(x)}
		\end{prooftree}
		\quad
		\begin{prooftree}
		\hypo{P(t)⊢}
		\infer1{¬∃x.¬P(x) ⊢ }
		\end{prooftree}
		\quad
		\begin{prooftree}
		\hypo{⊢P(y)}
		\infer1{⊢ ¬∃x.¬P(x)}
		\end{prooftree}
		\]
		
		
		Henkin constant quantifiers
		\[
		\begin{prooftree}
		\hypo{P[t/x] ⊢ Δ}
		\infer1{∀x.P ⊢ Δ}
		\end{prooftree}
		\quad
		\begin{prooftree}
		\hypo{Γ ⊢ P [c_{∀x.P}/x]}
		\infer1{Γ ⊢ ∀x.P}
		\end{prooftree}
		\]
		
		\[
		\begin{prooftree}
		\hypo{P[c_{∃x.P}/x] ⊢ Δ}
		\infer1{∃xP ⊢ Δ}
		\end{prooftree}
		\quad
		\begin{prooftree}
		\hypo{Γ ⊢ P[t/x]}
		\infer1{Γ ⊢ ∃xP}
		\end{prooftree}
		\]
		
		\[
		\begin{prooftree}
		\hypo{P(t/x) ⊢ Δ}
		\infer1{∀xP(x) ⊢ Δ}
		\end{prooftree}
		\quad
		\begin{prooftree}
		\hypo{Γ ⊢ P (c_{∀x.P}/x)}
		\infer1{Γ ⊢ ∀xP(x)}
		\end{prooftree}
		\]
		
		\[
		\begin{prooftree}
		\hypo{P(c_{∃x.P}/x) ⊢ Δ}
		\infer1{∃xP(x) ⊢ Δ}
		\end{prooftree}
		\quad
		\begin{prooftree}
		\hypo{Γ ⊢ P(t/x)}
		\infer1{Γ ⊢ ∃xP(x)}
		\end{prooftree}
		\]
	\end{center}

\section{Operational Rules}
	\begin{center}
				\subsection{Negation}
				\begin{center}
					\[
					\begin{prooftree}
					\hypo{ ⊢ A}
					\infer1{¬ A ⊢}
					\end{prooftree}
					\quad
					\begin{prooftree}
					\hypo{A ⊢ }
					\infer1{⊢ ¬ A}
					\end{prooftree}
					\]
				\end{center}
				
				
				\subsection{Negative Junctions}
				\begin{center}
					\[
					\begin{prooftree}
					\hypo{⊢ A}
					\hypo{⊢ B}
					\infer2{¬A ⊕ ¬B ⊢ }
					\end{prooftree}
					\quad
					\begin{prooftree}
					\hypo{A ⊢ }
					\infer1{ ⊢ ¬A ⊕ ¬B}
					\end{prooftree}
					\quad
					\begin{prooftree}
					\hypo{B ⊢ }
					\infer1{ ⊢ ¬A ⊕ ¬B}
					\end{prooftree}
					\]
					
					\[
					\begin{prooftree}
					\hypo{⊢ B}
					\hypo{ ⊢ A}
					\infer2{¬B ⊕ ¬A ⊢ }
					\end{prooftree}
					\quad
					\begin{prooftree}
					\hypo{A ⊢ }
					\infer1{ ⊢ ¬B ⊕ ¬A}
					\end{prooftree}
					\quad
					\begin{prooftree}
					\hypo{B ⊢ }
					\infer1{ ⊢ ¬B ⊕ ¬A}
					\end{prooftree}
					\]
					
					\[
					\begin{prooftree}
					\hypo{ ⊢ A}
					\infer1{¬A \& ¬B ⊢ }
					\end{prooftree}
					\quad
					\begin{prooftree}
					\hypo{ ⊢ B}
					\infer1{¬A \& ¬B ⊢ }
					\end{prooftree}
					\quad
					\begin{prooftree}
					\hypo{A ⊢ }
					\hypo{B ⊢ }
					\infer2{⊢ ¬A \& ¬B}
					\end{prooftree}
					\]
					
					\[
					\begin{prooftree}
					\hypo{ ⊢ A}
					\infer1{¬B \& ¬A ⊢ }
					\end{prooftree}
					\quad
					\begin{prooftree}
					\hypo{ ⊢ B}
					\infer1{¬B \& ¬A ⊢ }
					\end{prooftree}
					\quad
					\begin{prooftree}
					\hypo{B ⊢ }
					\hypo{A ⊢ }
					\infer2{ ⊢ ¬ B \& ¬ A}
					\end{prooftree}
					\]
				\end{center}

				\subsection{Conditionals}
				\begin{center}
					\[
					\begin{prooftree}
					\hypo{⊢ B}
					\hypo{A ⊢ }
					\infer2{¬B ⊕ A ⊢}
					\end{prooftree}
					\quad
					\begin{prooftree}
					\hypo{A ⊢ }
					\hypo{⊢ B}
					\infer2{A ⊕ ¬B ⊢}
					\end{prooftree}
					\quad
					\begin{prooftree}
					\hypo{Γ ⊢ A}
					\infer1{Γ ⊢ A ⊕ ¬B}
					\end{prooftree}
					\quad
					\begin{prooftree}
					\hypo{B ⊢ }
					\infer1{⊢ A ⊕ ¬B}
					\end{prooftree}
					\]
					
					\[
					\begin{prooftree}
					\hypo{ ⊢ A}
					\hypo{B ⊢ }
					\infer2{¬A ⊕ B ⊢ }
					\end{prooftree}
					\quad
					\begin{prooftree}
					\hypo{B ⊢ }
					\hypo{ ⊢ A}
					\infer2{B ⊕ ¬A ⊢ }
					\end{prooftree}
					\quad
					\begin{prooftree}
					\hypo{A ⊢ }
					\infer1{ ⊢ B ⊕ ¬A}
					\end{prooftree}
					\quad
					\begin{prooftree}
					\hypo{Γ ⊢ B}
					\infer1{Γ ⊢ B ⊕ ¬A}
					\end{prooftree}
					\]
					
					\[
					\begin{prooftree}
					\hypo{A ⊢ Δ}
					\infer1{A \& ¬B ⊢ Δ}
					\end{prooftree}
					\quad
					\begin{prooftree}
					\hypo{ ⊢ B}
					\infer1{A \& ¬B ⊢ }
					\end{prooftree}
					\quad
					\begin{prooftree}
					\hypo{⊢ A}
					\hypo{B ⊢ }
					\infer2{⊢ A \& ¬B}
					\end{prooftree}
					\quad
					\begin{prooftree}
					\hypo{B ⊢ }
					\hypo{⊢ A}
					\infer2{⊢ ¬B \& A}
					\end{prooftree}
					\]
					
					\[
					\begin{prooftree}
					\hypo{ ⊢ A}
					\infer1{B \& ¬A ⊢ }
					\end{prooftree}
					\quad
					\begin{prooftree}
					\hypo{B ⊢ Δ}
					\infer1{B \& ¬A ⊢ Δ}
					\end{prooftree}
					\quad
					\begin{prooftree}
					\hypo{ ⊢ B}
					\hypo{A ⊢ }
					\infer2{ ⊢ B \& ¬ A}
					\end{prooftree}
					\quad
					\begin{prooftree}
					\hypo{A ⊢ }
					\hypo{ ⊢ B}
					\infer2{ ⊢ ¬A \& B}
					\end{prooftree}
					\]
				\end{center}
				
				\subsection{Junctions}
				\begin{center}
					\[
					\begin{prooftree}
					\hypo{A ⊢ Δ}
					\hypo{B ⊢ Δ}
					\infer2{A ⊕ B ⊢ Δ}
					\end{prooftree}
					\quad
					\begin{prooftree}
					\hypo{Γ ⊢ A}
					\infer1{Γ ⊢ A ⊕ B}
					\end{prooftree}
					\quad
					\begin{prooftree}
					\hypo{Γ ⊢ B}
					\infer1{Γ ⊢ A ⊕ B}
					\end{prooftree}
					\]
					
					\[
					\begin{prooftree}
					\hypo{B ⊢ Δ}
					\hypo{A ⊢ Δ}
					\infer2{B ⊕ A ⊢ Δ}
					\end{prooftree}
					\quad
					\begin{prooftree}
					\hypo{Γ ⊢ A}
					\infer1{Γ ⊢ B ⊕ A}
					\end{prooftree}
					\quad
					\begin{prooftree}
					\hypo{Γ ⊢ B}
					\infer1{Γ ⊢ B ⊕ A}
					\end{prooftree}
					\]
					
					\[
					\begin{prooftree}
					\hypo{A ⊢ Δ}
					\infer1{A \& B ⊢ Δ}
					\end{prooftree}
					\quad
					\begin{prooftree}
					\hypo{B ⊢ Δ}
					\infer1{A \& B ⊢ Δ}
					\end{prooftree}
					\quad
					\begin{prooftree}
					\hypo{Γ ⊢ A}
					\hypo{Γ ⊢ B}
					\infer2{Γ ⊢ A \& B}
					\end{prooftree}
					\]
					
					\[
					\begin{prooftree}
					\hypo{A ⊢ Δ}
					\infer1{B \& A ⊢ Δ}
					\end{prooftree}
					\quad
					\begin{prooftree}
					\hypo{B ⊢ Δ}
					\infer1{B \& A ⊢ Δ}
					\end{prooftree}
					\quad
					\begin{prooftree}
					\hypo{Γ ⊢ B}
					\hypo{Γ ⊢ A}
					\infer2{Γ ⊢ B \& A}
					\end{prooftree}
					\]
				\end{center}
			
				\subsection{Biconditionals}
				\begin{center}
					\[
					\begin{prooftree}
					\hypo{⊢ A}
					\hypo{B ⊢ }
					\infer2{(¬A ⊕ B) \& (¬B ⊕ A) ⊢ }
					\end{prooftree}
					\quad
					\begin{prooftree}
					\hypo{⊢ B}
					\hypo{A ⊢ }
					\infer2{(¬A ⊕ B) \& (¬B ⊕ A) ⊢ }
					\end{prooftree}
					\quad
					\begin{prooftree}
					\hypo{A ⊢ }
					\hypo{B ⊢ }
					\infer2{ ⊢ (¬A ⊕ B) \& (¬B ⊕ A)}
					\end{prooftree}
					\quad
					\begin{prooftree}
					\hypo{Γ ⊢ B}
					\hypo{Γ ⊢ A}
					\infer2{Γ ⊢ (¬A ⊕ B) \& (¬B ⊕ A)}
					\end{prooftree}
					\]
					
					\[
					\begin{prooftree}
					\hypo{A ⊢ }
					\hypo{⊢A}
					\infer2{(A \& B) ⊕  (¬A \& ¬B) ⊢ }
					\end{prooftree}
					\quad
					\begin{prooftree}
					\hypo{A ⊢ }
					\hypo{⊢B}
					\infer2{(A \& B) ⊕  (¬A \& ¬B) ⊢ }
					\end{prooftree}
					\quad
					\begin{prooftree}
					\hypo{B ⊢ }
					\hypo{⊢A}
					\infer2{(A \& B) ⊕  (¬A \& ¬B) ⊢ }
					\end{prooftree}
					\quad
					\begin{prooftree}
					\hypo{B ⊢ }
					\hypo{⊢B}
					\infer2{(A \& B) ⊕  (¬A \& ¬B) ⊢ }
					\end{prooftree}
					\]
					
					\[
					\quad
					\begin{prooftree}
					\hypo{Γ ⊢ A}
					\hypo{Γ ⊢ B}
					\infer2{Γ ⊢ (A \& B) ⊕  (¬A \& ¬B)}
					\end{prooftree}
					\quad
					\begin{prooftree}
					\hypo{B ⊢ }
					\hypo{A ⊢ }
					\infer2{ ⊢ (A \& B) ⊕  (¬A \& ¬B)}
					\end{prooftree}
					\]
					
					TODO: XOR-like
				\end{center}
\end{center}

\part{Theorems}
	\begin{center}
		\[
		\begin{prooftree}
		\infer0{P(c_{∀x∃y.P(x,y)},c_{∃y∀x.P(x,y)}) ⊢ P(c_{∀x∃y.P(x,y)},c_{∃y∀x.P(x,y)})}
		\infer1{P(c_{∀x∃y.P(x,y)},c_{∃y∀x.P(x,y)}) ⊢ ∃y.P(c_{∀x∃y.P(x,y)},y)}
		\infer1{∀x.P(x,c_{∃y∀x.P(x,y)}) ⊢ ∃y.P(c_{∀x∃y.P(x,y)},y)}
		\infer1{∀x.P(x,c_{∃y∀x.P(x,y)}) ⊢ ∀x∃y.P(x,y)}
		\infer1{∃y∀x.P(x,y) ⊢ ∀x∃y.P(x,y)}
		\end{prooftree}
		\]
	\end{center}

\end{document}}*}


\chapter{First Order Structural Additive Sequent}

\documentclass{article}

\usepackage{amsmath}
\usepackage{ebproof}
\usepackage{fullpage}
\usepackage[utf8]{inputenc}
\usepackage{newunicodechar}
\usepackage{stix}

\newunicodechar{Γ}{\Gamma}
\newunicodechar{Δ}{\Delta}

\newunicodechar{⊢}{\vdash}

\newunicodechar{∀}{\forall}
\newunicodechar{∃}{\exists}

\newunicodechar{⊕}{\oplus}
\newunicodechar{¬}{\neg}

\setlength{\parindent}{0em}

\author{James Martin, Ian D.L.N. Mclean}
\title{First Order Structural Additive Sequent Calculus}

\begin{document}

\maketitle

\begin{abstract}
Strictly visible structural additive calculus with first order quantification.
\end{abstract}

\section{Structural Rules}

\begin{center}
	\[
	\begin{prooftree}
	\infer0[Id]{A ⊢ A}
	\end{prooftree}
	\]

	\[
	\begin{prooftree}
	\hypo{Γ ⊢ A}
	\hypo{A ⊢ Δ}
	\infer2[Cut]{Γ ⊢ Δ}
	\end{prooftree}
	\]

	\[
	\begin{prooftree}
	\hypo{⊢ C}
	\infer1[wL]{A ⊢ C}
	\end{prooftree}
	\qquad
	\begin{prooftree}
	\hypo{C ⊢ }
	\infer1[Rw]{C ⊢ A}
	\end{prooftree}
	\]

	\[
	\begin{prooftree}
	\hypo{A, A ⊢ C}
	\infer1[cL]{A ⊢ C}
	\end{prooftree}
	\qquad
	\begin{prooftree}
	\hypo{C ⊢ A, A}
	\infer1[Rc]{C ⊢ A}
	\end{prooftree}
	\]

	\[
	\begin{prooftree}
	\hypo{A, B ⊢ C}
	\infer1[pL]{B, A ⊢ C}
	\end{prooftree}
	\qquad
	\begin{prooftree}
	\hypo{C ⊢ A, B}
	\infer1[Rp]{C ⊢ B, A}
	\end{prooftree}
	\]
\end{center}

\section{Units}
	\begin{center}
		\[
		\begin{prooftree}
		\infer0{\bot ⊢ Δ}
		\end{prooftree}
		\quad
		\begin{prooftree}
		\infer0{Γ ⊢ \top}
		\end{prooftree}
		\]
	\end{center}

\section{Quantifiers}
	\begin{center}
		Standard Eigenvariable Condition Quantifiers
		\[
		\begin{prooftree}
		\hypo{P[t/x] ⊢ Δ}
		\infer1{∀x.P(x) ⊢ Δ}
		\end{prooftree}
		\quad
		\begin{prooftree}
		\hypo{Γ ⊢ P[y/x]}
		\infer1{Γ ⊢ ∀x.P(x)}
		\end{prooftree}
		\]
		in $∀R$ the variable y must not occur free anywhere in the lower sequent.
		\[
		\begin{prooftree}
		\hypo{P[y/x] ⊢ Δ}
		\infer1{∃x.P(x) ⊢ Δ}
		\end{prooftree}
		\quad
		\begin{prooftree}
		\hypo{Γ ⊢ P[t/x]}
		\infer1{Γ ⊢ ∃x.P(x)}
		\end{prooftree}
		\]
		in $∃L$ the variable y must not occur free anywhere in the lower sequent.
		\[
		\begin{prooftree}
		\hypo{P(y)⊢}
		\infer1{¬∀x.¬P(x) ⊢ }
		\end{prooftree}
		\quad
		\begin{prooftree}
		\hypo{⊢P(t)}
		\infer1{⊢ ¬∀x.¬P(x)}
		\end{prooftree}
		\quad
		\begin{prooftree}
		\hypo{P(t)⊢}
		\infer1{¬∃x.¬P(x) ⊢ }
		\end{prooftree}
		\quad
		\begin{prooftree}
		\hypo{⊢P(y)}
		\infer1{⊢ ¬∃x.¬P(x)}
		\end{prooftree}
		\]


		Henkin constant quantifiers
		\[
		\begin{prooftree}
		\hypo{P(t/x) ⊢ Δ}
		\infer1{∀xP(x) ⊢ Δ}
		\end{prooftree}
		\quad
		\begin{prooftree}
		\hypo{Γ ⊢ P (c_{∀x.P}/x)}
		\infer1{Γ ⊢ ∀xP(x)}
		\end{prooftree}
		\]

		\[
		\begin{prooftree}
		\hypo{P(c_{∃x.P}/x) ⊢ Δ}
		\infer1{∃xP(x) ⊢ Δ}
		\end{prooftree}
		\quad
		\begin{prooftree}
		\hypo{Γ ⊢ P(t/x)}
		\infer1{Γ ⊢ ∃xP(x)}
		\end{prooftree}
		\]

		\[
		\begin{prooftree}
		\hypo{P(c_{∀x.¬P(x)})⊢}
		\infer1{¬∀x.¬P(x) ⊢ }
		\end{prooftree}
		\quad
		\begin{prooftree}
		\hypo{⊢P(t)}
		\infer1{⊢ ¬∀x.¬P(x)}
		\end{prooftree}
		\quad
		\begin{prooftree}
		\hypo{P(t)⊢}
		\infer1{¬∃x.¬P(x) ⊢ }
		\end{prooftree}
		\quad
		\begin{prooftree}
		\hypo{⊢P(c_{∃x.¬P(x)})}
		\infer1{⊢ ¬∃x.¬P(x)}
		\end{prooftree}
		\]
	\end{center}

\section{Operational Rules}
	\begin{center}
				\subsection{Negation}
				\begin{center}
					\[
					\begin{prooftree}
					\hypo{ ⊢ A}
					\infer1{¬ A ⊢}
					\end{prooftree}
					\quad
					\begin{prooftree}
					\hypo{A ⊢ }
					\infer1{⊢ ¬ A}
					\end{prooftree}
					\]
				\end{center}


				\subsection{Negative Junctions}
				\begin{center}
					\[
					\begin{prooftree}
					\hypo{⊢ A}
					\hypo{⊢ B}
					\infer2{¬A \vee ¬B ⊢ }
					\end{prooftree}
					\quad
					\begin{prooftree}
					\hypo{A ⊢ }
					\infer1{ ⊢ ¬A \vee ¬B}
					\end{prooftree}
					\quad
					\begin{prooftree}
					\hypo{B ⊢ }
					\infer1{ ⊢ ¬A \vee ¬B}
					\end{prooftree}
					\]

					\[
					\begin{prooftree}
					\hypo{ ⊢ A}
					\infer1{¬A \wedge ¬B ⊢ }
					\end{prooftree}
					\quad
					\begin{prooftree}
					\hypo{ ⊢ B}
					\infer1{¬A \wedge ¬B ⊢ }
					\end{prooftree}
					\quad
					\begin{prooftree}
					\hypo{A ⊢ }
					\hypo{B ⊢ }
					\infer2{⊢ ¬A \wedge ¬B}
					\end{prooftree}
					\]
				\end{center}

				\subsection{Conditionals}
				\begin{center}
					\[
					\begin{prooftree}
					\hypo{⊢ A}
					\hypo{B ⊢ }
					\infer2{¬A \vee B ⊢}
					\end{prooftree}
					\quad
					\begin{prooftree}
					\hypo{⊢ A}
					\infer1{⊢ ¬A \vee B}
					\end{prooftree}
					\quad
					\begin{prooftree}
					\hypo{Γ ⊢ B}
					\infer1{Γ⊢ A \vee B}
					\end{prooftree}
					\]

					\[
					\begin{prooftree}
					\hypo{A ⊢ Δ}
					\infer1{A \wedge ¬B ⊢ Δ}
					\end{prooftree}
					\quad
					\begin{prooftree}
					\hypo{ ⊢ B}
					\infer1{A \wedge ¬B ⊢ }
					\end{prooftree}
					\quad
					\begin{prooftree}
					\hypo{⊢ A}
					\hypo{B ⊢ }
					\infer2{⊢ A \wedge ¬B}
					\end{prooftree}
					\]
				\end{center}

				\subsection{Junctions}
				\begin{center}
					\[
					\begin{prooftree}
					\hypo{A ⊢ Δ}
					\hypo{B ⊢ Δ}
					\infer2{A \vee B ⊢ Δ}
					\end{prooftree}
					\quad
					\begin{prooftree}
					\hypo{Γ ⊢ A}
					\infer1{Γ ⊢ A \vee B}
					\end{prooftree}
					\quad
					\begin{prooftree}
					\hypo{Γ ⊢ B}
					\infer1{Γ ⊢ A \vee B}
					\end{prooftree}
					\quad
					\begin{prooftree}
					\hypo{C ⊢ A, B}
					\infer1{C ⊢ A \vee B}
					\end{prooftree}
					\]

					\[
					\begin{prooftree}
					\hypo{A, B ⊢ C}
					\infer1{A \wedge B ⊢ C}
					\end{prooftree}
					\quad
					\begin{prooftree}
					\hypo{A ⊢ Δ}
					\infer1{A \wedge B ⊢ Δ}
					\end{prooftree}
					\quad
					\begin{prooftree}
					\hypo{B ⊢ Δ}
					\infer1{A \wedge B ⊢ Δ}
					\end{prooftree}
					\quad
					\begin{prooftree}
					\hypo{Γ ⊢ A}
					\hypo{Γ ⊢ B}
					\infer2{Γ ⊢ A \wedge B}
					\end{prooftree}
					\]
				\end{center}

				\subsection{Biconditionals}
				\begin{center}
					\[
					\begin{prooftree}
					\hypo{⊢ A}
					\hypo{B ⊢ }
					\infer2{(¬A \vee B) \wedge (¬B \vee A) ⊢ }
					\end{prooftree}
					\quad
					\begin{prooftree}
					\hypo{A ⊢ }
					\hypo{B ⊢ }
					\infer2{ ⊢ (¬A \vee B) \wedge (¬B \vee A)}
					\end{prooftree}
					\quad
					\begin{prooftree}
					\hypo{Γ ⊢ B}
					\hypo{Γ ⊢ A}
					\infer2{Γ ⊢ (¬A \vee B) \wedge (¬B \vee A)}
					\end{prooftree}
					\]

					\[
					\begin{prooftree}
					\hypo{A ⊢ }
					\hypo{⊢A}
					\infer2{(A \wedge B) \vee  (¬A \wedge ¬B) ⊢ }
					\end{prooftree}
					\quad
					\begin{prooftree}
					\hypo{A ⊢ }
					\hypo{⊢B}
					\infer2{(A \wedge B) \vee  (¬A \wedge ¬B) ⊢ }
					\end{prooftree}
					\quad
					\begin{prooftree}
					\hypo{B ⊢ }
					\hypo{⊢A}
					\infer2{(A \wedge B) \vee  (¬A \wedge ¬B) ⊢ }
					\end{prooftree}
					\quad
					\begin{prooftree}
					\hypo{B ⊢ }
					\hypo{⊢B}
					\infer2{(A \wedge B) \vee  (¬A \wedge ¬B) ⊢ }
					\end{prooftree}
					\]

					\[
					\quad
					\begin{prooftree}
					\hypo{Γ ⊢ A}
					\hypo{Γ ⊢ B}
					\infer2{Γ ⊢ (A \wedge B) \vee  (¬A \wedge ¬B)}
					\end{prooftree}
					\quad
					\begin{prooftree}
					\hypo{B ⊢ }
					\hypo{A ⊢ }
					\infer2{ ⊢ (A \wedge B) \vee  (¬A \wedge ¬B)}
					\end{prooftree}
					\]

					TODO: XOR-like
					\[
					\begin{prooftree}
					\hypo{B ⊢ Δ}
					\hypo{A ⊢ Δ}
					\infer2{(¬A \wedge B) \vee (¬B \vee A) ⊢ Δ}
					\end{prooftree}
					\quad
					\begin{prooftree}
					\hypo{⊢ A}
					\hypo{⊢ B}
					\infer2{(¬A \wedge B) \vee (¬B \vee A) ⊢ }
					\end{prooftree}
					\quad
					\begin{prooftree}
					\hypo{A ⊢ }
					\hypo{⊢ B}
					\infer2{(¬A \wedge B) \vee (¬B \vee A) ⊢ }
					\end{prooftree}
					\quad
					\begin{prooftree}
					\hypo{⊢ A}
					\hypo{B ⊢}
					\infer2{(¬A \wedge B) \vee (¬B \vee A) ⊢ }
					\end{prooftree}
					\]

					\[
					\quad
					\begin{prooftree}
					\hypo{A ⊢ }
					\hypo{⊢ B}
					\infer2{ ⊢ (¬A \wedge B) \vee (¬B \vee A)}
					\end{prooftree}
					\quad
					\begin{prooftree}
					\hypo{⊢ A}
					\hypo{B ⊢}
					\infer2{⊢ (¬A \wedge B) \vee (¬B \vee A)}
					\end{prooftree}
					\]

					\[
					\begin{prooftree}
					\hypo{⊢ A}
					\hypo{⊢ B}
					\infer2{(A \vee B) \wedge  (¬A \vee ¬B) ⊢ }
					\end{prooftree}
					\quad
					\begin{prooftree}
					\hypo{A ⊢ Δ}
					\hypo{B ⊢ Δ}
					\infer2{(A \vee B) \wedge  (¬A \vee ¬B) ⊢ Δ}
					\end{prooftree}
					\]

					\[
					\begin{prooftree}
					\hypo{⊢ A}
					\hypo{A ⊢}
					\infer2{⊢ (A \vee B) \wedge  (¬A \vee ¬B)}
					\end{prooftree}
					\quad
					\begin{prooftree}
					\hypo{⊢ B}
					\hypo{B ⊢}
					\infer2{⊢ (A \vee B) \wedge  (¬A \vee ¬B)}
					\end{prooftree}
					\quad
					\begin{prooftree}
					\hypo{⊢ A}
					\hypo{B ⊢}
					\infer2{⊢ (A \vee B) \wedge  (¬A \vee ¬B)}
					\end{prooftree}
					\quad
					\begin{prooftree}
					\hypo{⊢ B}
					\hypo{A ⊢}
					\infer2{⊢ (A \vee B) \wedge  (¬A \vee ¬B)}
					\end{prooftree}
					\quad
					\]
				\end{center}
\end{center}

\part{Theorems}
	\begin{center}

	\end{center}

\end{document}


\chapter{Functionally Incomplete (from docx)}
\hypertarget{preliminaries}{%
\section{Preliminaries}\label{preliminaries}}

A set of logical connectives is said to be functionally complete when
combinations of the logical connectives in the set can express every
equivalent expression of the connectives in the set of \{⊥, ⊤, ¬, →, ↛,
⊕, ↔, ∧, ∨, ↓, ↑\}.

\{¬, →\} \{⊥, ⊤, ↛, ⊕, ↔, ∧, ∨\}; \{↓, ↑\}

`\{⊥, ⊤, ⊕, ↔\}, \{⊥, ↛, ⊕, ∧, ∨\}, \{⊤, ↔, ∧, ∨\}, \{⊥, ⊤, ∧, ∨\},
\{¬\}`

\{¬, →, ↛\}\{⊥, ⊤, ⊕, ↔, ∧, ∨\} \{↓, ↑\}

A set of logical connectives is said to be functionally incomplete when
combinations of the logical connectives can not express any functionally
complete subset of \{⊥, ⊤, ¬, →, ↛, ⊕, ↔, ∧, ∨, ↓, ↑\} or equivalently
can not express the set \{⊥, ⊤, ¬, →, ↛, ⊕, ↔, ∧, ∨, ↓, ↑\}.

The minimal sets of functionally complete logical connectives entail
that for defining the functionally incomplete sets the only relevant
logical connectives are in the set \{⊥, ⊤, ¬, →, ↛, ⊕, ↔, ∧, ∨\}.
Notably, any set that contains NOR or NAND is automatically functionally
complete.

\hypertarget{connective-properties}{%
\subsection{Connective Properties}\label{connective-properties}}

Non adjunctive paraconsistent logics These are logics in which the
conjunction

fails to obey the following law of adjunction: a, b entails a ∧ b

Non implicative paraconsistent logics These are logics in which the
implication

fails to obey the following law of implicativity: if entails a → b then
a entails b.

\hypertarget{maximal-functionally-incomplete-sets}{%
\subsection{Maximal Functionally Incomplete
Sets}\label{maximal-functionally-incomplete-sets}}

Affine: \{⊥, ⊤, ¬, ⊕, ↔\}; \{→\}, \{↛\}, \{∧\}, \{∨\}

False-preserving: \{⊥, ↛, ⊕, ∧, ∨\}; \{⊤\}, \{¬\}, \{→\}, \{↔\}

Truth-preserving: \{⊤, →, ↔, ∧, ∨\}; \{⊥\}, \{¬\}, \{↛\}, \{⊕\}

Monotonic: \{⊥, ⊤, ∧, ∨\}; \{⊕\}, \{↔\}, \{¬\}, \{→\}, \{↛\}

Self-dual: \{¬\} \{⊥, ⊤, ⊕, ↔\}; \{→\}, \{↛\}, \{∧\}, \{∨\}

\hypertarget{minimal-functionally-complete-sets}{%
\subsection{Minimal Functionally Complete
Sets}\label{minimal-functionally-complete-sets}}

\hypertarget{singles}{%
\subsubsection{Singles}\label{singles}}

\{↓\}

\{↑\}

\hypertarget{doubles}{%
\subsubsection{Doubles}\label{doubles}}

\{⊥, →\}

\{⊥, ←\}

\{⊤, ↛\}

\{⊤, ↚\}

\{¬, →\}

\{¬, ←\}

\{¬, ↛\}

\{¬, ↚\}

\{¬, ∨\}

\{¬, ∧\}

\{→, ↛\}

\{→, ↚\}

\{←, ↛\}

\{←, ↚\}

\{→, ⊕\}

\{←, ⊕\}

\{↛, ↔\}

\{↚, ↔\}

\hypertarget{triples}{%
\subsubsection{Triples}\label{triples}}

\{⊥, ↔, ∨\}

\{↔, ⊕, ∨\}

\{⊤, ⊕, ∨\}

\{⊥, ↔, ∧\}

\{↔, ⊕, ∧\}

\{⊤, ⊕, ∧\}

\hypertarget{functionally-incomplete-sets-of-logical-connectives}{%
\section{Functionally Incomplete Sets of Logical
Connectives}\label{functionally-incomplete-sets-of-logical-connectives}}

The following format of collections of sets are\\
The set; the functional completions of the set; the functional
incompletions of the set.

\hypertarget{singles-1}{%
\subsection{Singles}\label{singles-1}}

9 functionally incomplete singles.

\{⊥\}; \{→\}, \{↔, ∧\}, \{↔, ∨\}; ... \{⊤, ¬, ↔, ⊕\}, \{↛, ⊕, ∨, ∧\}

\{⊤\}; \{↛\}, \{⊕, ∧\}, \{⊕, ∨\}; ... \{⊥, ¬, ↔, ⊕\}, \{→, ↔, ∧, ∨\}

\{¬\}; \{→\}, \{↛\}, \{∧\}, \{∨\}; ... \{⊥, ⊤, ⊕, ↔\}

\{→\}; \{⊥\}, \{¬\}, \{↛\}, \{⊕\}; ... \{⊤,↔, ∧, ∨\}

\{↛\}; \{⊤\}, \{¬\}, \{→\}, \{↔\}; ... \{⊥, ⊕, ∧, ∨\}

\{⊕\}; \{→\}, \{⊤, ∧\}, \{⊤, ∨\}, \{↔, ∧\}, \{↔, ∨\}; ... \{⊥, ↛, ∨,
∧\}, \{⊥, ⊤, ¬, ↔\}

\{↔\}; \{↛\}, \{⊥, ∧\}, \{⊥, ∨\}, \{⊕, ∧\}, \{⊕, ∨\}; ... \{⊤, →, ∧,
∨\}, \{⊥, ⊤, ¬, ⊕\}

\{∧\}; \{¬\}, \{⊥, ↔\}, \{⊤, ⊕\}, \{⊕, ↔\}; \{⊥\}, \{⊤\}, \{↔\}, \{⊕\},
\{∨\}, \{→\}, \{↛\}, \{∨, →\}, \{∨, ↛\}, ..., \{⊥, ⊤, ∨\}, \{⊤, →, ↔,
∨\}, \{⊥, ↛, ⊕, ∨\}

\{∨\}; \{¬\}, \{⊥, ↔\}, \{⊤, ⊕\}, \{⊕, ↔\}; \ldots, \{⊥, ⊤, ∧\}, \{⊤, →,
∧, ↔\}, \{⊥, ↛, ⊕, ∧\}

\hypertarget{doubles-1}{%
\subsection{Doubles}\label{doubles-1}}

27 functionally incomplete doubles.

\hypertarget{affine}{%
\subsubsection{Affine}\label{affine}}

\{¬, ⊕\}; \{→\}, \{↛\}, \{∧\}, \{∨\}; \{⊥\}, \{⊤\}, \{↔\}, \{⊥, ⊤\},
\{⊥, ↔\}, \{⊤, ↔\}, \{⊥, ⊤, ↔\}

\{¬, ↔\}; \{→\}, \{↛\}, \{∧\}, \{∨\}; \{⊥\}, \{⊤\}, \{⊕\}, \{⊥, ⊤\},
\{⊥, ⊕\}, \{⊤, ⊕\}, \{⊥, ⊤, ⊕\}

\{⊥, ¬\}; \{→\}, \{↛\}, \{∧\}, \{∨\}; \{⊤\}, \{⊕\}, \{↔\}, \{⊤, ⊕\},
\{⊤, ↔\}, \{⊕, ↔\}, \{⊤, ⊕, ↔\}

\{⊥, ↔\}; \{→\}, \{↛\}, \{∧\}, \{∨\}; \{⊤\}, \{¬\}, \{⊕\}, \{⊤, ¬\},
\{⊤, ⊕\}, \{¬, ⊕\}, \{⊤, ¬, ⊕\}

\{⊤, ¬\}; \{→\}, \{↛\}, \{∧\}, \{∨\}; \{⊥\}, \{⊕\}, \{↔\}, \{⊥, ⊕\},
\{⊥, ↔\}, \{⊕, ↔\}, \{⊥, ⊕, ↔\}

\{⊤, ⊕\}; \{→\}, \{↛\}, \{∧\}, \{∨\}; \{⊥\}, \{¬\}, \{↔\}, \{⊥, ¬\},
\{⊥, ↔\}, \{¬, ↔\}, \{⊥, ¬, ↔\}

\{↔, ⊕\}; \{→\}, \{↛\}, \{∧\}, \{∨\}; \{⊥\}, \{⊤\}, \{¬\}, \{⊥, ⊤\},
\{⊥, ¬\}, \{⊤, ¬\}, \{⊥, ⊤, ¬\}

\hypertarget{affine-or-false-preserving}{%
\subsubsection{Affine or False
Preserving}\label{affine-or-false-preserving}}

\{⊥, ⊕\}; \{→\}, \{⊤, ∧\}, \{⊤, ∨\}, \{↔, ∧\}, \{↔, ∨\}; \ldots{} \{⊤,
¬, ↔\}, \{↛, ∨, ∧\}

\hypertarget{affine-or-truth-preserving}{%
\subsubsection{Affine or Truth
Preserving}\label{affine-or-truth-preserving}}

\{⊤, ↔\}; \{↛\}, \{⊥, ∧\}, \{⊥, ∨\}, \{⊕, ∧\}, \{⊕, ∨\}; \ldots{} \{⊥,
¬, ⊕\}, \{→, ∧, ∨\}

\hypertarget{affine-and-monotonic}{%
\subsubsection{Affine and Monotonic}\label{affine-and-monotonic}}

\{⊥, ⊤\}; \{→\}, \{↛\}, \{↔, ∧\}, \{↔, ∨\}, \{⊕, ∧\}, \{⊕, ∨\}; \{¬\},
\{↔\}, \{⊕\}, \{∧\}, \{∨\}, \{∧, ∨\}, \{¬, ⊕\}, \{¬, ↔\}, \{⊕, ↔\}, \{¬,
⊕, ↔\}

\hypertarget{truth-preserving}{%
\subsubsection{Truth Preserving}\label{truth-preserving}}

\hypertarget{generators}{%
\paragraph{Generators}\label{generators}}

\{→, ∧\}; \{⊥\}, \{¬\}, \{↛\}, \{⊕\}; \ldots{} \{⊤, ↔, ∨\}

\hypertarget{non-generators}{%
\paragraph{Non-generators}\label{non-generators}}

\{⊤, →\}; \{⊥\}, \{¬\}, \{↛\}, \{⊕\}; \ldots{} \{↔, ∧, ∨\}

\{→, ↔\}; \{⊥\}, \{¬\}, \{↛\}, \{⊕\}; \ldots{} \{⊤, ∧, ∨\}

\{→, ∨\}; \{⊥\}, \{¬\}, \{↛\}, \{⊕\}; \ldots{} \{⊤, ↔, ∧\}

\{↔, ∨\}; \{¬\}, \{↛\}, \{⊕\}, \{⊥\}; \ldots{} \{⊤, →, ∧\}

\{↔, ∧\}; \{¬\}, \{↛\}, \{⊕\}, \{⊥\}; \ldots{} \{⊤, →, ∨\}

\hypertarget{false-preserving-generator}{%
\subsubsection{False Preserving
Generator}\label{false-preserving-generator}}

\{↛, ∨\}; \{⊤\}, \{¬\}, \{→\}, \{↔\}

\hypertarget{false-preserving}{%
\subsubsection{False Preserving}\label{false-preserving}}

\{⊕, ∧\}; \{¬\}, \{→\}, \{↔\}, \{⊤\}; \ldots{} \{⊥, ↛, ∨\}

\{⊕, ∨\}; \{¬\}, \{→\}, \{↔\}, \{⊤\}; \ldots{} \{⊥, ↛, ∧\}

\{⊥, ↛\}; \{⊤\}, \{¬\}, \{→\}, \{↔\}; \ldots{} \{⊕, ∨, ∧\}

\{↛, ⊕\}; \{⊤\}, \{¬\}, \{→\}, \{↔\}; \ldots{} \{⊥, ∨, ∧\}

\{↛, ∧\}; \{⊤\}, \{¬\}, \{→\}, \{↔\}; \ldots{} \{⊥, ⊕, ∨\}

\hypertarget{monotonic-and-truth-preserving}{%
\subsubsection{Monotonic and Truth
Preserving}\label{monotonic-and-truth-preserving}}

\{⊤, ∨\}; \{¬\}, \{↛\}, \{⊕\}, \{⊥, ↔\}; \ldots{} \{→, ↔, ∧\}

\{⊤, ∧\}; \{¬\}, \{↛\}, \{⊕\}, \{⊥, ↔\}; \ldots{} \{→, ↔, ∨\}

\hypertarget{monotonic-and-false-preserving}{%
\subsubsection{Monotonic and False
Preserving}\label{monotonic-and-false-preserving}}

\{⊥, ∨\}; \{¬\}, \{→\}, \{↔\}, \{⊤, ⊕\}; \ldots, \{↛, ⊕, ∧\}

\{⊥, ∧\}; \{¬\}, \{→\}, \{↔\}, \{⊤, ⊕\}; \ldots, \{↛, ⊕, ∨\}

\hypertarget{section}{%
\subsubsection{}\label{section}}

\hypertarget{special}{%
\subsubsection{Special}\label{special}}

Truth or False Preserving

\{∧, ∨\}; \{¬\}, \{⊥, ↔\}, \{⊤, ⊕\}, \{↔, ⊕\}; \ldots{} \{⊥, ↛, ⊕\},
\{⊤, →, ↔\}

\hypertarget{triples-1}{%
\subsection{Triples}\label{triples-1}}

32 functionally incomplete triples.

\hypertarget{affine-connectives}{%
\subsubsection{Affine Connectives}\label{affine-connectives}}

\hypertarget{generators-1}{%
\paragraph{Generators}\label{generators-1}}

\{⊥, ¬, ⊕\}; \{→\}, \{↛\}, \{∧\}, \{∨\}; \{⊤\}, \{↔\}, \{⊤, ↔\}

\{⊥, ¬, ↔\}; \{→\}, \{↛\}, \{∧\}, \{∨\}; \{⊤\}, \{⊕\}, \{⊤, ⊕\}

\{⊤, ¬, ⊕\}; \{→\}, \{↛\}, \{∧\}, \{∨\}; \{⊥\}, \{↔\}, \{⊥, ↔\}

\{⊤, ¬, ↔\}; \{→\}, \{↛\}, \{∧\}, \{∨\}; \{⊥\}, \{⊕\}, \{⊥, ⊕\}

\{¬, ⊕, ↔\}; \{→\}, \{↛\}, \{∧\}, \{∨\}; \{⊥\}, \{⊤\}, \{⊥, ⊤\}

\hypertarget{non-generators-1}{%
\paragraph{Non-generators}\label{non-generators-1}}

\{⊥, ⊤, ¬\}; \{→\}, \{↛\}, \{∧\}, \{∨\}; \{⊕\}, \{↔\}, \{⊕, ↔\}

\{⊥, ⊤, ↔\}; \{→\}, \{↛\}, \{∧\}, \{∨\}; \{¬\}, \{⊕\}, \{¬, ⊕\}

\{⊥, ⊤, ⊕\}; \{→\}, \{↛\}, \{∧\}, \{∨\}; \{¬\}, \{↔\}, \{¬, ↔\}

\{⊥, ⊕, ↔\}; \{→\}, \{↛\}, \{∧\}, \{∨\}; \{⊤\}, \{¬\}, \{⊤, ¬\}

\{⊤, ⊕, ↔\}; \{→\}, \{↛\}, \{∧\}, \{∨\}; \{⊥\}, \{¬\}, \{⊥, ¬\}

\hypertarget{truth-preserving-connectives}{%
\subsubsection{Truth Preserving
Connectives}\label{truth-preserving-connectives}}

\hypertarget{generators-2}{%
\paragraph{Generators}\label{generators-2}}

\{⊤, →, ∧\}; \{⊥\}, \{¬\}, \{↛\}, \{⊕\}; \{↔\}, \{∨\}, \{↔, ∨\}

\{→, ↔, ∧\}; \{⊥\}, \{¬\}, \{↛\}, \{⊕\}; \{⊤\}, \{∨\}, \{⊤, ∨\}

\{→, ∧, ∨\}; \{⊥\}, \{¬\}, \{↛\}, \{⊕\}; \{⊤\}, \{↔\}, \{⊤, ↔\}

\hypertarget{non-generators-2}{%
\paragraph{Non-generators}\label{non-generators-2}}

\{⊤, →, ↔\}; \{⊥\}, \{¬\}, \{↛\}, \{⊕\}; \{∧\}, \{∨\}, \{∧, ∨\}

\{⊤, →, ∨\}; \{⊥\}, \{¬\}, \{↛\}, \{⊕\}; \{↔\}, \{∧\}, \{↔, ∧\}\\
\{⊤, ↔, ∨\}; \{⊥\}, \{¬\}, \{↛\}, \{⊕\}; \{→\}, \{∧\}, \{→, ∧\}\\
\{⊤, ↔, ∧\}; \{⊥\}, \{¬\}, \{↛\}, \{⊕\}; \{→\}, \{∨\}, \{→, ∨\}\\
\{→, ↔, ∨\}; \{⊥\}, \{¬\}, \{↛\}, \{⊕\}; \{⊤\}, \{∧\}, \{⊤, ∧\}\\
\{↔, ∧, ∨\}; \{⊥\}, \{¬\}, \{↛\}, \{⊕\}; \{⊤\}, \{→\}, \{⊤, →\}

\hypertarget{false-preserving-connectives}{%
\subsubsection{False Preserving
Connectives}\label{false-preserving-connectives}}

\{⊥, ↛, ⊕\}; \{⊤\}, \{¬\}, \{→\}, \{↔\}; \{∧\}, \{∨\}, \{∧, ∨\}

\{⊥, ⊕, ∨\}; \{⊤\}, \{¬\}, \{→\}, \{↔\}; \{↛\}, \{∧\}, \{↛, ∧\}

\{⊥, ⊕, ∧\}; \{⊤\}, \{¬\}, \{→\}, \{↔\}; \{↛\}, \{∨\}, \{↛, ∨\}

\{⊥, ↛, ∧\}; \{⊤\}, \{¬\}, \{→\}, \{↔\}; \{⊕\}, \{∨\}, \{⊕, ∨\}

\{⊥, ↛, ∨\}; \{⊤\}, \{¬\}, \{→\}, \{↔\}; \{∧\}, \{⊕\}, \{⊕, ∧\}

\{↛, ⊕, ∧\}; \{⊤\}, \{¬\}, \{→\}, \{↔\}; \{⊥\}, \{∨\}, \{⊥, ∨\}

\{↛, ⊕, ∨\}; \{⊤\}, \{¬\}, \{→\}, \{↔\}; \{⊥\}, \{∧\}, \{⊥, ∧\}\\
\{↛, ∧, ∨\}; \{⊤\}, \{¬\}, \{→\}, \{↔\}; \{⊥\}, \{⊕\}, \{⊥, ⊕\}

\{⊕, ∧, ∨\}; \{⊤\}, \{¬\}, \{→\}, \{↔\}; \{⊥\}, \{↛\}, \{⊥, ↛\}

\hypertarget{exclusively-monotonic-connectives}{%
\subsubsection{\texorpdfstring{Exclusively Monotonic Connectives
}{Exclusively Monotonic Connectives }}\label{exclusively-monotonic-connectives}}

\{⊥, ⊤, ∨\}; \{¬\}, \{→\}, \{↛\}, \{⊕\}, \{↔\}; \{∧\}

\{⊥, ⊤, ∧\}; \{¬\}, \{→\}, \{↛\}, \{⊕\}, \{↔\}; \{∨\}

\hypertarget{monotonic-intersections}{%
\subsubsection{Monotonic Intersections}\label{monotonic-intersections}}

Monotonic and Truth Preserving

\{⊥, ∧, ∨\}; \{¬\}, \{→\}, \{↔\}; \{⊤\}, \{↛\}, \{⊕\}, \{↛, ⊕\}

Monotonic and False Preserving\\
\{⊤, ∧, ∨\}; \{¬\}, \{↛\}, \{⊕\}; \{⊥\}, \{→\}, \{↔\}, \{→, ↔\}

\hypertarget{quadruples}{%
\subsection{Quadruples}\label{quadruples}}

16 functionally incomplete quadruples.

\hypertarget{affine-connectives-1}{%
\subsubsection{Affine Connectives}\label{affine-connectives-1}}

\{⊥, ⊤, ⊕, ↔\}; \{→\}, \{↛\}, \{∧\}, \{∨\}; \{¬\}

\{¬, ⊤, ⊕, ↔\}; \{→\}, \{↛\}, \{∧\}, \{∨\}; \{⊥\}

\{¬, ⊥, ⊕, ↔\}; \{→\}, \{↛\}, \{∧\}, \{∨\}; \{⊤\}

\{¬, ⊥, ⊤, ↔\}; \{→\}, \{↛\}, \{∧\}, \{∨\}; \{⊕\}

\{¬, ⊥, ⊤, ⊕\}; \{→\}, \{↛\}, \{∧\}, \{∨\}; \{↔\}

\hypertarget{truth-preserving-connectives-1}{%
\subsubsection{Truth Preserving
Connectives}\label{truth-preserving-connectives-1}}

\hypertarget{generators-3}{%
\paragraph{Generators}\label{generators-3}}

\{→, ⊤, ∧, ↔\}; \{⊥\}, \{¬\}, \{↛\}, \{⊕\}; \{∨\}

\{→, ∧, ∨, ↔\}; \{⊥\}, \{¬\}, \{↛\}, \{⊕\}; \{⊤\}

\{→, ⊤, ∧, ∨\}; \{⊥\}, \{¬\}, \{↛\}, \{⊕\}; \{↔\}

\hypertarget{non-generators-3}{%
\paragraph{Non-generators}\label{non-generators-3}}

\{⊤, ∧, ∨, ↔\}; \{⊥\}, \{¬\}, \{↛\}, \{⊕\}; \{→\}

\{→, ⊤, ∨, ↔\}; \{⊥\}, \{¬\}, \{↛\}, \{⊕\}; \{∧\}

\hypertarget{false-preserving-connectives-1}{%
\subsubsection{False Preserving
Connectives}\label{false-preserving-connectives-1}}

\hypertarget{generators-4}{%
\paragraph{Generators}\label{generators-4}}

\{⊥, ↛, ∨, ⊕\}; \{⊤\}, \{¬\}, \{→\}, \{↔\}; \{∧\}

\{⊥, ↛, ∧, ∨\}; \{⊤\}, \{¬\}, \{→\}, \{↔\}; \{⊕\}

\{↛, ∧, ∨, ⊕\}; \{⊤\}, \{¬\}, \{→\}, \{↔\}; \{⊥\}

\hypertarget{non-generators-4}{%
\paragraph{Non-generators}\label{non-generators-4}}

\{⊥, ↛, ∧, ⊕\}; \{⊤\}, \{¬\}, \{→\}, \{↔\}; \{∨\}

\{⊥, ∧, ∨, ⊕\}; \{⊤\}, \{¬\}, \{→\}, \{↔\}; \{↛\}

\hypertarget{monotonic-connectives}{%
\subsubsection{\texorpdfstring{Monotonic Connectives
}{Monotonic Connectives }}\label{monotonic-connectives}}

\{⊥, ⊤, ∧, ∨\}; \{¬\}, \{→\}, \{↛\}, \{⊕\}, \{↔\}

\hypertarget{quintuples}{%
\subsection{Quintuples}\label{quintuples}}

\hypertarget{affine-connectives-2}{%
\subsubsection{Affine Connectives}\label{affine-connectives-2}}

\{⊥, ⊤, ¬, ⊕, ↔\}; \{→\}, \{↛\}, \{∧\}, \{∨\}

\hypertarget{truth-preserving-connectives-2}{%
\subsubsection{Truth Preserving
Connectives}\label{truth-preserving-connectives-2}}

\{⊤, →, ↔, ∧, ∨\}; \{⊥\}, \{¬\}, \{↛\}, \{⊕\}

\hypertarget{false-preserving-connectives-2}{%
\subsubsection{False Preserving
Connectives}\label{false-preserving-connectives-2}}

\{⊥, ↛, ⊕, ∨, ∧\}; \{⊤\}, \{¬\}, \{→\}, \{↔\}

\hypertarget{sextuples-and-higher}{%
\subsection{Sextuples and higher}\label{sextuples-and-higher}}

All subsets of \{⊥, ⊤, ¬, →, ↛, ⊕, ↔, ∧, ∨\} of cardinality 6 or higher
are functionally complete.

Proof: \{⊥, ⊤, ⊕, ↔, ∧, ∨\} and \{¬, →, ↛\} are functionally complete.
Every 5 element subset of \{⊥, ⊤, ⊕, ↔, ∧, ∨\} is functionally complete;
there exist 4 element subsets which are functionally incomplete, but the
only way to get six element sets from those 4 element sets is to add two
elements from \{¬, →, ↛\} and any two elements of \{¬, →, ↛\} are
functionally complete together.

\hypertarget{special-cases-of-functional-incomplete-sets-of-operators}{%
\subsection{Special Cases of functional incomplete sets of
operators}\label{special-cases-of-functional-incomplete-sets-of-operators}}

\{⊥, ⊤, ⊕, ↔\} ⊬ \{¬\}

\{⊤, →, ↔, ∨\} ⊬ \{∨\}

\{⊥, ↛, ⊕, ∧\} ⊬ \{∧\}

\{¬, ⊕\} ⊢ \{⊥, ⊤, ¬, ⊕, ↔\}

\{¬, ↔\} ⊢ \{⊥, ⊤, ¬, ⊕, ↔\}

\{→, ∨\} ⊢ \{⊤, →, ∨\}

\{→, ∧\} ⊢ \{⊤, →, ↔, ∧\}

\{↛, ∨\} ⊢ \{⊥, ↛, ⊕, ∨\}

\{⊕, ↔\} ⊢ \{⊥, ⊤, ⊕, ↔\}

\{→, ↔\} ⊢ \{⊤, →, ↔\}

\{↛, ⊕\} ⊢ \{⊥, ↛, ⊕\}

\{↛\}⊢\{⊥, ↛\}

\{⊕\}⊢\{⊥, ⊕\}

\{→\}⊢\{⊤,→\}

\{↔\}⊢\{⊤, ↔\}

\hypertarget{functionally-complete-sets}{%
\section{Functionally Complete Sets}\label{functionally-complete-sets}}

There are 512 subsets of \{⊥, ⊤, ¬, →, ↛, ⊕, ↔, ∧, ∨\}. 425 subsets are
functionally complete.

There are 2048 subsets of \{⊥, ⊤, ¬, →, ↛, ⊕, ↔, ∧, ∨, ↓, ↑\}. 1961
subsets are functionally complete.

NUMERICAL ENCODING! Based on the MAXIMALLY INCOMPLETE SETS

Class 1: truth-preserving 10000\\
Class 2: false-preserving 01000\\
Class 3: affine 00100

Class 4: monotone 00010

Class 5: self-dual 00001

Any connective can be expressed uniquely as a binary numeral.\\
AND is truth preserving. 10010

\{⊥, ⊤, ¬, ⊕, ↔\} Affine set

\{⊥, ↛, ⊕, ∨, ∧\} False Preserving set

\{⊤, →, ↔, ∧, ∨\} Truth Preserving set

\{⊥, ⊤, ∧, ∨\} Monotonic set

\{¬\} Self-Dual set

LIST OF NUMERICAL ENCODINGS!

⊥: false preserving, affine. 01100

⊤: truth preserving, affine. 10100

¬: self-dual: 00001, affine: 00100. 00101

→ Truth Preserving 10000

↛ False Preserving 01000

⊕ False Preserving, Affine 01100

↔ Truth Preserving, Affine 10100

∧ False Preserving, Truth Preserving, Monotonic 11010

∨ False Preserving, Truth Preserving, Monotonic 11010

↓: 00000

↑: 00000

PROVING WHICH SETS ARE FUNCTIONALLY COMPLETE

A functionally complete set is defined by the property that when all its
members are summed (with bitwise OR), the sum is 00000.

\{↓\}: 00000

\{↑\}: 00000

\{⊥, →\} : 01100 \textbar\textbar{} 10100 = 00100

AND: AB

OR: 1 - (1 - A)(1 - B) = A + B - AB

NAND: 1 - AB

NOR: 1 - A - B + AB

IMPLIES: 1 - A + AB

NONIMPLIES: A(1-B)

\chapter{Functionally Incomplete (from odt)}
}

\protect\hypertarget{anchor}{}{}Preliminaries

A set of logical connectives is said to be functionally complete when
combinations of the logical connectives in the set can express every
equivalent expression of the connectives in the set of \{⊥, ⊤, ¬, →, ↛,
⊕, ↔, ∧, ∨, ↓, ↑\}.

A set of logical connectives is said to be functionally incomplete when
combinations of the logical connectives can not express any functionally
complete subset of \{⊥, ⊤, ¬, →, ↛, ⊕, ↔, ∧, ∨, ↓, ↑\}.

The minimal sets of functionally complete logical connectives entail
that for defining the functionally incomplete sets the only relevant
logical connectives are in the set \{⊥, ⊤, ¬, →, ↛, ⊕, ↔, ∧, ∨\}.
Notably, any set that contains NOR or NAND is automatically functionally
complete.

\protect\hypertarget{anchor-1}{}{}Maximal Functionally Incomplete Sets

\{⊥, ⊤, ¬, ⊕, ↔\}; \{→\}, \{↛\}, \{∧\}, \{∨\}

\{⊥, ↛, ⊕, ∨, ∧\}; \{⊤\}, \{¬\}, \{→\}, \{↔\}

\{⊤, →, ↔, ∧, ∨\}; \{⊥\}, \{¬\}, \{↛\}, \{⊕\}

\{⊥, ⊤, ∧, ∨\}; \{⊕\}, \{↔\}, \{¬\}, \{→\}, \{↛\}

\protect\hypertarget{anchor-2}{}{}Minimal Functionally Complete Sets

\protect\hypertarget{anchor-3}{}{}Singles

\{↓\}

\{↑\}

\protect\hypertarget{anchor-4}{}{}Doubles

\{⊥, →\}

\{⊥, ←\}

\{⊤, ↛\}

\{⊤, ↚\}

\{¬, →\}

\{¬, ←\}

\{¬, ↛\}

\{¬, ↚\}

\{¬, ∨\}

\{¬, ∧\}

\{→, ↛\}

\{→, ↚\}

\{←, ↛\}

\{←, ↚\}

\{→, ⊕\}

\{←, ⊕\}

\{↛, ↔\}

\{↚, ↔\}

\protect\hypertarget{anchor-5}{}{}Triples

\{⊥, ↔, ∨\}

\{↔, ⊕, ∨\}

\{⊤, ⊕, ∨\}

\{⊥, ↔, ∧\}

\{↔, ⊕, ∧\}

\{⊤, ⊕, ∧\}

\protect\hypertarget{anchor-6}{}{}Functionally Incomplete Sets of
Logical Connectives

\protect\hypertarget{anchor-7}{}{}Singles

9 functionally incomplete singles.

\{⊥\}; \{→\}, \{↔, ∧\}, \{↔, ∨\}; ... \{⊤, ¬, ↔, ⊕\}, \{↛, ⊕, ∨, ∧\}

\{⊤\}; \{↛\}, \{⊕, ∧\}, \{⊕, ∨\}; ... \{⊥, ¬, ↔, ⊕\}, \{→, ↔, ∧, ∨\}

\{¬\}; \{→\}, \{↛\}, \{∧\}, \{∨\}; ... \{⊥, ⊤, ⊕, ↔\}

\{→\}; \{⊥\}, \{¬\}, \{↛\}, \{⊕\}; ... \{⊤,↔, ∧, ∨\}

\{↛\}; \{⊤\}, \{¬\}, \{→\}, \{↔\}; ... \{⊥, ⊕, ∧, ∨\}

\{⊕\}; \{→\}, \{⊤, ∧\}, \{⊤, ∨\}, \{↔, ∧\}, \{↔, ∨\}; ... \{⊥, ↛, ∨,
∧\}, \{⊥, ⊤, ¬, ↔\}

\{↔\}; \{↛\}, \{⊥, ∧\}, \{⊥, ∨\}, \{⊕, ∧\}, \{⊕, ∨\}; ... \{⊤, →, ∧,
∨\}, \{⊥, ⊤, ¬, ⊕\}

\{∧\}; \{¬\}, \{⊥, ↔\}, \{⊤, ⊕\}, \{⊕, ↔\}; \{⊥\}, \{⊤\}, \{↔\}, \{⊕\},
\{∨\}, \{→\}, \{↛\}, \{∨, →\}, \{∨, ↛\}, ..., \{⊤, →, ↔, ∨\}, \{⊥, ↛, ⊕,
∨\}

\{∨\}; \{¬\}, \{⊥, ↔\}, \{⊤, ⊕\}, \{⊕, ↔\}; ... \{∧, ⊤, ↔, →\}, \{⊕, ⊥,
∧, ↛\}\\

\protect\hypertarget{anchor-8}{}{}Doubles

27 functionally incomplete doubles.

\protect\hypertarget{anchor-9}{}{}Affine

\{⊥, ⊤\}; \{→\}, \{↛\}, \{↔, ∧\}, \{↔, ∨\}, \{⊕, ∧\}, \{⊕, ∨\}; \{¬\},
\{↔\}, \{⊕\}, \{∧\}, \{∨\}, \{∧, ∨\}, \{¬, ⊕\}, \{¬, ↔\}, \{⊕, ↔\}, \{¬,
⊕, ↔\}

\{⊥, ¬\}; \{→\}, \{↛\}, \{∧\}, \{∨\}; \{⊤\}, \{⊕\}, \{↔\}, \{⊤, ⊕\},
\{⊤, ↔\}, \{⊕, ↔\}, \{⊤, ⊕, ↔\}

\{⊥, ↔\}; \{→\}, \{↛\}, \{∧\}, \{∨\}; \{⊤\}, \{¬\}, \{⊕\}, \{⊤, ¬\},
\{⊤, ⊕\}, \{¬, ⊕\}, \{⊤, ¬, ⊕\}

\{⊤, ¬\}; \{→\}, \{↛\}, \{∧\}, \{∨\}; \{⊥\}, \{⊕\}, \{↔\}, \{⊥, ⊕\},
\{⊥, ↔\}, \{⊕, ↔\}, \{⊥, ⊕, ↔\}

\{⊤, ⊕\}; \{→\}, \{↛\}, \{∧\}, \{∨\}; \{⊥\}, \{¬\}, \{↔\}, \{⊥, ¬\},
\{⊥, ↔\}, \{¬, ↔\}, \{⊥, ¬, ↔\}

\{¬, ⊕\}; \{→\}, \{↛\}, \{∧\}, \{∨\}; \{⊥\}, \{⊤\}, \{↔\}, \{⊥, ⊤\},
\{⊥, ↔\}, \{⊤, ↔\}, \{⊥, ⊤, ↔\}

\{¬, ↔\}; \{→\}, \{↛\}, \{∧\}, \{∨\}; \{⊥\}, \{⊤\}, \{⊕\}, \{⊥, ⊤\},
\{⊥, ⊕\}, \{⊤, ⊕\}, \{⊥, ⊤, ⊕\}

\{↔, ⊕\}; \{→\}, \{↛\}, \{∧\}, \{∨\}; \{⊥\}, \{⊤\}, \{¬\}, \{⊥, ⊤\},
\{⊥, ¬\}, \{⊤, ¬\}, \{⊥, ⊤, ¬\}

\protect\hypertarget{anchor-10}{}{}Truth Preserving

\{⊤, →\}; \{⊥\}, \{¬\}, \{↛\}, \{⊕\}; \ldots{} \{↔, ∧, ∨\}

\{→, ↔\}; \{⊥\}, \{¬\}, \{↛\}, \{⊕\}; \ldots{} \{⊤, ∧, ∨\}

\{→, ∧\}; \{⊥\}, \{¬\}, \{↛\}, \{⊕\}; \ldots{} \{⊤, ↔, ∨\}

\{→, ∨\}; \{⊥\}, \{¬\}, \{↛\}, \{⊕\}; \ldots{} \{⊤, ↔, ∧\}

\{↔, ∨\}; \{¬\}, \{↛\}, \{⊕\}, \{⊥\}; \ldots{} \{⊤, →, ∧\}

\{↔, ∧\}; \{¬\}, \{↛\}, \{⊕\}, \{⊥\}; \ldots{} \{⊤, →, ∨\}

\protect\hypertarget{anchor-11}{}{}False Preserving

\{⊕, ∧\}; \{¬\}, \{→\}, \{↔\}, \{⊤\}; \ldots{} \{⊥, ↛, ∨\}

\{⊕, ∨\}; \{¬\}, \{→\}, \{↔\}, \{⊤\}; \ldots{} \{⊥, ↛, ∧\}

\{⊥, ↛\}; \{⊤\}, \{¬\}, \{→\}, \{↔\}; \ldots{} \{⊕, ∨, ∧\}

\{↛, ⊕\}; \{⊤\}, \{¬\}, \{→\}, \{↔\}; \ldots{} \{⊥, ∨, ∧\}

\{↛, ∨\}; \{⊤\}, \{¬\}, \{→\}, \{↔\}; \ldots{} \{⊥, ⊕, ∧\}

\{↛, ∧\}; \{⊤\}, \{¬\}, \{→\}, \{↔\}; \ldots{} \{⊥, ⊕, ∨\}

\protect\hypertarget{anchor-12}{}{}Monotonic

\{⊥, ∨\}; \{¬\}, \{→\}, \{↔\}, \{⊤, ⊕\}; \ldots, \{↛, ⊕, ∧\}

\{⊥, ∧\}; \{¬\}, \{→\}, \{↔\}, \{⊤, ⊕\}; \ldots, \{↛, ⊕, ∨\}

\{⊤, ∨\}; \{¬\}, \{↛\}, \{⊕\}, \{⊥, ↔\}; \ldots{} \{→, ↔, ∧\}

\{⊤, ∧\}; \{¬\}, \{↛\}, \{⊕\}, \{⊥, ↔\}; \ldots{} \{→, ↔, ∨\}

\protect\hypertarget{anchor-13}{}{}Special

Affine or False Preserving

\{⊥, ⊕\}; \{→\}, \{⊤, ∧\}, \{⊤, ∨\}, \{↔, ∧\}, \{↔, ∨\}; \ldots{} \{⊤,
¬, ↔\}, \{↛, ∨, ∧\}\\
Affine or Truth Preserving

\{⊤, ↔\}; \{↛\}, \{⊥, ∧\}, \{⊥, ∨\}, \{⊕, ∧\}, \{⊕, ∨\}; \ldots{} \{⊥,
¬, ⊕\}, \{→, ∧, ∨\}\\
Truth or False Preserving

\{∧, ∨\}; \{¬\}, \{⊥, ↔\}, \{⊤, ⊕\}, \{↔, ⊕\}; \ldots{} \{⊥, ↛, ⊕\},
\{⊤, →, ↔\}

\protect\hypertarget{anchor-14}{}{}Triples

32 functionally incomplete triples.

\protect\hypertarget{anchor-15}{}{}Affine Connectives

\{⊥, ⊤, ¬\}; \{→\}, \{↛\}, \{∧\}, \{∨\}; \{⊕\}, \{↔\}, \{⊕, ↔\}

\{⊥, ⊤, ↔\}; \{→\}, \{↛\}, \{∧\}, \{∨\}; \{¬\}, \{⊕\}, \{¬, ⊕\}

\{⊥, ⊤, ⊕\}; \{→\}, \{↛\}, \{∧\}, \{∨\}; \{¬\}, \{↔\}, \{¬, ↔\}

\{⊥, ¬, ⊕\}; \{→\}, \{↛\}, \{∧\}, \{∨\}; \{⊤\}, \{↔\}, \{⊤, ↔\}

\{⊥, ¬, ↔\}; \{→\}, \{↛\}, \{∧\}, \{∨\}; \{⊤\}, \{⊕\}, \{⊤, ⊕\}

\{⊥, ⊕, ↔\}; \{→\}, \{↛\}, \{∧\}, \{∨\}; \{⊤\}, \{¬\}, \{⊤, ¬\}

\{⊤, ¬, ⊕\}; \{→\}, \{↛\}, \{∧\}, \{∨\}; \{⊥\}, \{↔\}, \{⊥, ↔\}

\{⊤, ¬, ↔\}; \{→\}, \{↛\}, \{∧\}, \{∨\}; \{⊥\}, \{⊕\}, \{⊥, ⊕\}

\{⊤, ⊕, ↔\}; \{→\}, \{↛\}, \{∧\}, \{∨\}; \{⊥\}, \{¬\}, \{⊥, ¬\}

\{¬, ⊕, ↔\}; \{→\}, \{↛\}, \{∧\}, \{∨\}; \{⊥\}, \{⊤\}, \{⊥, ⊤\}

\protect\hypertarget{anchor-16}{}{}Truth Preserving Connectives

\{⊤, →, ↔\}; \{⊥\}, \{¬\}, \{↛\}, \{⊕\}; \{∧\}, \{∨\}, \{∧, ∨\}

\{⊤, →, ∧\}; \{⊥\}, \{¬\}, \{↛\}, \{⊕\}; \{↔\}, \{∨\}, \{↔, ∨\}

\{⊤, →, ∨\}; \{⊥\}, \{¬\}, \{↛\}, \{⊕\}; \{↔\}, \{∧\}, \{↔, ∧\}\\
\{⊤, ↔, ∨\}; \{⊥\}, \{¬\}, \{↛\}, \{⊕\}; \{→\}, \{∧\}, \{→, ∧\}\\
\{⊤, ↔, ∧\}; \{⊥\}, \{¬\}, \{↛\}, \{⊕\}; \{→\}, \{∨\}, \{→, ∨\}\\
\{→, ↔, ∧\}; \{⊥\}, \{¬\}, \{↛\}, \{⊕\}; \{⊤\}, \{∨\}, \{⊤, ∨\}\\
\{→, ↔, ∨\}; \{⊥\}, \{¬\}, \{↛\}, \{⊕\}; \{⊤\}, \{∧\}, \{⊤, ∧\}

\{→, ∧, ∨\}; \{⊥\}, \{¬\}, \{↛\}, \{⊕\}; \{⊤\}, \{↔\}, \{⊤, ↔\}\\
\{↔, ∧, ∨\}; \{⊥\}, \{¬\}, \{↛\}, \{⊕\}; \{⊤\}, \{→\}, \{⊤, →\}

\protect\hypertarget{anchor-17}{}{}False Preserving Connectives

\{⊥, ↛, ⊕\}; \{⊤\}, \{¬\}, \{→\}, \{↔\}; \{∧\}, \{∨\}, \{∧, ∨\}

\{⊥, ⊕, ∨\}; \{⊤\}, \{¬\}, \{→\}, \{↔\}; \{↛\}, \{∧\}, \{↛, ∧\}

\{⊥, ⊕, ∧\}; \{⊤\}, \{¬\}, \{→\}, \{↔\}; \{↛\}, \{∨\}, \{↛, ∨\}

\{⊥, ↛, ∧\}; \{⊤\}, \{¬\}, \{→\}, \{↔\}; \{⊕\}, \{∨\}, \{⊕, ∨\}

\{⊥, ↛, ∨\}; \{⊤\}, \{¬\}, \{→\}, \{↔\}; \{∧\}, \{⊕\}, \{⊕, ∧\}

\{↛, ⊕, ∧\}; \{⊤\}, \{¬\}, \{→\}, \{↔\}; \{⊥\}, \{∨\}, \{⊥, ∨\}

\{↛, ⊕, ∨\}; \{⊤\}, \{¬\}, \{→\}, \{↔\}; \{⊥\}, \{∧\}, \{⊥, ∧\}\\
\{↛, ∧, ∨\}; \{⊤\}, \{¬\}, \{→\}, \{↔\}; \{⊥\}, \{⊕\}, \{⊥, ⊕\}

\{⊕, ∧, ∨\}; \{⊤\}, \{¬\}, \{→\}, \{↔\}; \{⊥\}, \{↛\}, \{⊥, ↛\}

\protect\hypertarget{anchor-18}{}{}Monotonic Connectives

\{⊥, ⊤, ∨\}; \{¬\}, \{→\}, \{↛\}, \{⊕\}, \{↔\}; \{∧\}

\{⊥, ⊤, ∧\}; \{¬\}, \{→\}, \{↛\}, \{⊕\}, \{↔\}; \{∨\}

\protect\hypertarget{anchor-19}{}{}???

Monotonic or Truth Preserving

\{⊥, ∧, ∨\}; \{¬\}, \{→\}, \{↔\}; \{⊤\}, \{↛\}, \{⊕\}, \{↛, ⊕\}

Monotonic or False Preserving\\
\{⊤, ∧, ∨\}; \{¬\}, \{↛\}, \{⊕\}; \{⊥\}, \{→\}, \{↔\}, \{→, ↔\}

\protect\hypertarget{anchor-20}{}{}Quadruples

16 functionally incomplete quadruples.

\protect\hypertarget{anchor-21}{}{}Affine Connectives

\{¬, ⊤, ⊕, ↔\}; \{→\}, \{↛\}, \{∧\}, \{∨\}; \{⊥\}

\{¬, ⊥, ⊕, ↔\}; \{→\}, \{↛\}, \{∧\}, \{∨\}; \{⊤\}

\{¬, ⊥, ⊤, ↔\}; \{→\}, \{↛\}, \{∧\}, \{∨\}; \{⊕\}

\{¬, ⊥, ⊤, ⊕\}; \{→\}, \{↛\}, \{∧\}, \{∨\}; \{↔\}

\{⊥, ⊤, ⊕, ↔\}; \{→\}, \{↛\}, \{∧\}, \{∨\}; \{¬\}

\protect\hypertarget{anchor-22}{}{}Truth Preserving Connectives

\{⊤, ∧, ∨, ↔\}; \{⊥\}, \{¬\}, \{↛\}, \{⊕\}; \{→\}

\{→, ⊤, ∨, ↔\}; \{⊥\}, \{¬\}, \{↛\}, \{⊕\}; \{∧\}

\{→, ⊤, ∧, ↔\}; \{⊥\}, \{¬\}, \{↛\}, \{⊕\}; \{∨\}

\{→, ∧, ∨, ↔\}; \{⊥\}, \{¬\}, \{↛\}, \{⊕\}; \{⊤\}

\{→, ⊤, ∧, ∨\}; \{⊥\}, \{¬\}, \{↛\}, \{⊕\}; \{↔\}

\protect\hypertarget{anchor-23}{}{}False Preserving Connectives

\{⊥, ↛, ∨, ⊕\}; \{⊤\}, \{¬\}, \{→\}, \{↔\}; \{∧\}

\{⊥, ↛, ∧, ⊕\}; \{⊤\}, \{¬\}, \{→\}, \{↔\}; \{∨\}

\{⊥, ↛, ∧, ∨\}; \{⊤\}, \{¬\}, \{→\}, \{↔\}; \{⊕\}

\{⊥, ∧, ∨, ⊕\}; \{⊤\}, \{¬\}, \{→\}, \{↔\}; \{↛\}

\{↛, ∧, ∨, ⊕\}; \{⊤\}, \{¬\}, \{→\}, \{↔\}; \{⊥\}

\protect\hypertarget{anchor-24}{}{}Monotonic Connectives

\{⊥, ⊤, ∧, ∨\}; \{¬\}, \{→\}, \{↛\}, \{⊕\}, \{↔\}

\protect\hypertarget{anchor-25}{}{}Quintuples

\protect\hypertarget{anchor-26}{}{}Affine Connectives

\{⊥, ⊤, ¬, ⊕, ↔\}; \{→\}, \{↛\}, \{∧\}, \{∨\}

\protect\hypertarget{anchor-27}{}{}Truth Preserving Connectives

\{⊤, →, ↔, ∧, ∨\}; \{⊥\}, \{¬\}, \{↛\}, \{⊕\}

\protect\hypertarget{anchor-28}{}{}False Preserving Connectives

\{⊥, ↛, ⊕, ∨, ∧\}; \{⊤\}, \{¬\}, \{→\}, \{↔\}

\protect\hypertarget{anchor-29}{}{}Sextuples and higher

All subsets of \{⊥, ⊤, ¬, →, ↛, ⊕, ↔, ∧, ∨\} of cardinality 6 or higher
are functionally complete.

Proof: \{⊥, ⊤, ⊕, ↔, ∧, ∨\} and \{¬, →, ↛\} are functionally complete.
Every 5 element subset of \{⊥, ⊤, ⊕, ↔, ∧, ∨\} is functionally complete;
there exist 4 element subsets which are functionally incomplete, but the
only way to get six element sets from those 4 element sets is to add two
elements from \{¬, →, ↛\} and any two elements of \{¬, →, ↛\} are
functionally complete together.

\protect\hypertarget{anchor-30}{}{}Special Cases

\{⊥, ⊤, ⊕, ↔\} ⊬ \{¬\}

\{⊤, →, ↔, ∨\} ⊬ \{∧\}

\{⊥, ↛, ⊕, ∧\} ⊬ \{∨\}

\{¬, ⊕\} ⊢ \{⊥, ⊤, ¬, ⊕, ↔\}

\{¬, ↔\} ⊢ \{⊥, ⊤, ¬, ⊕, ↔\}

\{→, ∧\} ⊢ \{⊤, →, ↔, ∧, ∨\}

\{↛, ∨\} ⊢ \{⊥, ↛, ⊕, ∧, ∨\}

\{⊕, ↔\} ⊢ \{⊥, ⊤, ⊕, ↔\}

\{→, ↔\} ⊢ \{⊤, →, ↔, ∨\}

\{↛, ⊕\} ⊢ \{⊥, ↛, ⊕, ∧\}

\{↛\}⊢\{↛, ∧\}

\{⊕\}⊢\{⊥, ⊕\}

\{→\}⊢\{→, ∨\}

\{↔\}⊢\{⊤, ↔\}

\protect\hypertarget{anchor-31}{}{}Functionally Complete Sets

There are 512 subsets of \{⊥, ⊤, ¬, →, ↛, ⊕, ↔, ∧, ∨\}. 425 subsets are
functionally complete.

There are 2048 subsets of \{⊥, ⊤, ¬, →, ↛, ⊕, ↔, ∧, ∨, ↓, ↑\}. 1961
subsets are functionally complete.

NUMERICAL ENCODING! Based on the MAXIMALLY INCOMPLETE SETS

Class 1: truth-preserving 10000\\
Class 2: false-preserving 01000\\
Class 3: affine 00100

Class 4: monotone 00010

Class 5: self-dual 00001

Any connective can be expressed uniquely as a binary numeral.\\
AND is truth preserving. 10010

\{⊥, ⊤, ¬, ⊕, ↔\} Affine set

\{⊥, ↛, ⊕, ∨, ∧\} False Preserving set

\{⊤, →, ↔, ∧, ∨\} Truth Preserving set

\{⊥, ⊤, ∧, ∨\} Monotonic set

\{¬\} Self-Dual set

LIST OF NUMERICAL ENCODINGS!

⊥: false preserving, affine. 01100

⊤: truth preserving, affine. 10100

¬: self-dual: 00001, affine: 00100. 00101

→ Truth Preserving 10000

↛ False Preserving 01000

⊕ False Preserving, Affine 01100

↔ Truth Preserving, Affine 10100

∧ False Preserving, Truth Preserving, Monotonic 11010

∨ False Preserving, Truth Preserving, Monotonic 11010

↓: 00000

↑: 00000

PROVING WHICH SETS ARE FUNCTIONALLY COMPLETE

A functionally complete set is defined by the property that when all its
members are summed (with bitwise OR), the sum is 00000.

\{↓\}: 00000

\{↑\}: 00000

\{⊥, →\} : 01100 \textbar\textbar{} 10100 = 00100

AND: AB

OR: 1 - (1 - A)(1 - B) = A + B - AB

NAND: 1 - AB

NOR: 1 - A - B + AB

IMPLIES: 1 - A + AB

NONIMPLIES: A(1-B)

\end{document}}*}


\chapter{Functionally Incomplete V2}
\hypertarget{preliminaries}{%
\section{Preliminaries}\label{preliminaries}}

A set of logical connectives is said to be functionally complete when
combinations of the logical connectives in the set can express every
equivalent expression of the connectives in the set of \{$\bot$ , $\top$ , $\neg$ , $\to$ , $\nrightarrow$ ,
$\oplus$ , $\leftrightarrow$ , $\land$ , $\lor$ , $\downarrow$ , $\uparrow$ \}.

\{$\neg$ , $\to$ \} \{$\bot$ , $\top$ , $\nrightarrow$ , $\oplus$ , $\leftrightarrow$ , $\land$ , $\lor$ \}; \{$\downarrow$ , $\uparrow$ \}

`\{$\bot$ , $\top$ , $\oplus$ , $\leftrightarrow$ \}, \{$\bot$ , $\nrightarrow$ , $\oplus$ , $\land$ , $\lor$ \}, \{$\top$ , $\leftrightarrow$ , $\land$ , $\lor$ \}, \{$\bot$ , $\top$ , $\land$ , $\lor$ \},
\{$\neg$ \}`

\{$\neg$ , $\to$ , $\nrightarrow$ \}\{$\bot$ , $\top$ , $\oplus$ , $\leftrightarrow$ , $\land$ , $\lor$ \} \{$\downarrow$ , $\uparrow$ \}

A set of logical connectives is said to be functionally incomplete when
combinations of the logical connectives can not express any functionally
complete subset of \{$\bot$ , $\top$ , $\neg$ , $\to$ , $\nrightarrow$ , $\oplus$ , $\leftrightarrow$ , $\land$ , $\lor$ , $\downarrow$ , $\uparrow$ \}.

The minimal sets of functionally complete logical connectives entail
that for defining the functionally incomplete sets the only relevant
logical connectives are in the set \{$\bot$ , $\top$ , $\neg$ , $\to$ , $\nrightarrow$ , $\oplus$ , $\leftrightarrow$ , $\land$ , $\lor$ \}.
Notably, any set that contains NOR or NAND is automatically functionally
complete.

\hypertarget{maximal-functionally-incomplete-sets}{%
\subsection{Maximal Functionally Incomplete
Sets}\label{maximal-functionally-incomplete-sets}}

Affine: \{$\bot$ , $\top$ , $\neg$ , $\oplus$ , $\leftrightarrow$ \}; \{$\to$ \}, \{$\nrightarrow$ \}, \{$\land$ \}, \{$\lor$ \}

False-preserving: \{$\bot$ , $\nrightarrow$ , $\oplus$ , $\land$ , $\lor$ \}; \{$\top$ \}, \{$\neg$ \}, \{$\to$ \}, \{$\leftrightarrow$ \}

Truth-preserving: \{$\top$ , $\to$ , $\leftrightarrow$ , $\land$ , $\lor$ \}; \{$\bot$ \}, \{$\neg$ \}, \{$\nrightarrow$ \}, \{$\oplus$ \}

Monotonic: \{$\bot$ , $\top$ , $\land$ , $\lor$ \}; \{$\oplus$ \}, \{$\leftrightarrow$ \}, \{$\neg$ \}, \{$\to$ \}, \{$\nrightarrow$ \}

Self-dual: \{$\neg$ \} \{$\bot$ , $\top$ , $\oplus$ , $\leftrightarrow$ \}; \{$\to$ \}, \{$\nrightarrow$ \}, \{$\land$ \}, \{$\lor$ \}

\hypertarget{minimal-functionally-complete-sets}{%
\subsection{Minimal Functionally Complete
Sets}\label{minimal-functionally-complete-sets}}

\hypertarget{singles}{%
\subsubsection{Singles}\label{singles}}

\{$\downarrow$ \}

\{$\uparrow$ \}

\hypertarget{doubles}{%
\subsubsection{Doubles}\label{doubles}}

\{$\bot$ , $\to$ \}

\{$\bot$ , ←\}

\{$\top$ , $\nrightarrow$ \}

\{$\top$ , $\nleftarrow$ \}

\{$\neg$ , $\to$ \}

\{$\neg$ , ←\}

\{$\neg$ , $\nrightarrow$ \}

\{$\neg$ , $\nleftarrow$ \}

\{$\neg$ , $\lor$ \}

\{$\neg$ , $\land$ \}

\{$\to$ , $\nrightarrow$ \}

\{$\to$ , $\nleftarrow$ \}

\{←, $\nrightarrow$ \}

\{←, $\nleftarrow$ \}

\{$\to$ , $\oplus$ \}

\{←, $\oplus$ \}

\{$\nrightarrow$ , $\leftrightarrow$ \}

\{$\nleftarrow$ , $\leftrightarrow$ \}

\hypertarget{triples}{%
\subsubsection{Triples}\label{triples}}

\{$\bot$ , $\leftrightarrow$ , $\lor$ \}

\{$\leftrightarrow$ , $\oplus$ , $\lor$ \}

\{$\top$ , $\oplus$ , $\lor$ \}

\{$\bot$ , $\leftrightarrow$ , $\land$ \}

\{$\leftrightarrow$ , $\oplus$ , $\land$ \}

\{$\top$ , $\oplus$ , $\land$ \}

\hypertarget{functionally-incomplete-sets-of-logical-connectives}{%
\section{Functionally Incomplete Sets of Logical
Connectives}\label{functionally-incomplete-sets-of-logical-connectives}}

\hypertarget{singles-1}{%
\subsection{Singles}\label{singles-1}}

9 functionally incomplete singles.

\{$\bot$ \}; \{$\to$ \}, \{$\leftrightarrow$ , $\land$ \}, \{$\leftrightarrow$ , $\lor$ \}; ... \{$\top$ , $\neg$ , $\leftrightarrow$ , $\oplus$ \}, \{$\nrightarrow$ , $\oplus$ , $\lor$ , $\land$ \}

\{$\top$ \}; \{$\nrightarrow$ \}, \{$\oplus$ , $\land$ \}, \{$\oplus$ , $\lor$ \}; ... \{$\bot$ , $\neg$ , $\leftrightarrow$ , $\oplus$ \}, \{$\to$ , $\leftrightarrow$ , $\land$ , $\lor$ \}

\{$\neg$ \}; \{$\to$ \}, \{$\nrightarrow$ \}, \{$\land$ \}, \{$\lor$ \}; ... \{$\bot$ , $\top$ , $\oplus$ , $\leftrightarrow$ \}

\{$\to$ \}; \{$\bot$ \}, \{$\neg$ \}, \{$\nrightarrow$ \}, \{$\oplus$ \}; ... \{$\top$ ,$\leftrightarrow$ , $\land$ , $\lor$ \}

\{$\nrightarrow$ \}; \{$\top$ \}, \{$\neg$ \}, \{$\to$ \}, \{$\leftrightarrow$ \}; ... \{$\bot$ , $\oplus$ , $\land$ , $\lor$ \}

\{$\oplus$ \}; \{$\to$ \}, \{$\top$ , $\land$ \}, \{$\top$ , $\lor$ \}, \{$\leftrightarrow$ , $\land$ \}, \{$\leftrightarrow$ , $\lor$ \}; ... \{$\bot$ , $\nrightarrow$ , $\lor$ ,
$\land$ \}, \{$\bot$ , $\top$ , $\neg$ , $\leftrightarrow$ \}

\{$\leftrightarrow$ \}; \{$\nrightarrow$ \}, \{$\bot$ , $\land$ \}, \{$\bot$ , $\lor$ \}, \{$\oplus$ , $\land$ \}, \{$\oplus$ , $\lor$ \}; ... \{$\top$ , $\to$ , $\land$ ,
$\lor$ \}, \{$\bot$ , $\top$ , $\neg$ , $\oplus$ \}

\{$\land$ \}; \{$\neg$ \}, \{$\bot$ , $\leftrightarrow$ \}, \{$\top$ , $\oplus$ \}, \{$\oplus$ , $\leftrightarrow$ \}; \{$\bot$ \}, \{$\top$ \}, \{$\leftrightarrow$ \}, \{$\oplus$ \},
\{$\lor$ \}, \{$\to$ \}, \{$\nrightarrow$ \}, \{$\lor$ , $\to$ \}, \{$\lor$ , $\nrightarrow$ \}, ..., \{$\bot$ , $\top$ , $\lor$ \}, \{$\top$ , $\to$ , $\leftrightarrow$ ,
$\lor$ \}, \{$\bot$ , $\nrightarrow$ , $\oplus$ , $\lor$ \}

\{$\lor$ \}; \{$\neg$ \}, \{$\bot$ , $\leftrightarrow$ \}, \{$\top$ , $\oplus$ \}, \{$\oplus$ , $\leftrightarrow$ \}; \ldots, \{$\bot$ , $\top$ , $\land$ \}, \{$\top$ , $\to$ ,
$\land$ , $\leftrightarrow$ \}, \{$\bot$ , $\nrightarrow$ , $\oplus$ , $\land$ \}

\hypertarget{doubles-1}{%
\subsection{Doubles}\label{doubles-1}}

27 functionally incomplete doubles.

\hypertarget{affine}{%
\subsubsection{Affine}\label{affine}}

\{$\neg$ , $\oplus$ \}; \{$\to$ \}, \{$\nrightarrow$ \}, \{$\land$ \}, \{$\lor$ \}; \{$\bot$ \}, \{$\top$ \}, \{$\leftrightarrow$ \}, \{$\bot$ , $\top$ \},
\{$\bot$ , $\leftrightarrow$ \}, \{$\top$ , $\leftrightarrow$ \}, \{$\bot$ , $\top$ , $\leftrightarrow$ \}

\{$\neg$ , $\leftrightarrow$ \}; \{$\to$ \}, \{$\nrightarrow$ \}, \{$\land$ \}, \{$\lor$ \}; \{$\bot$ \}, \{$\top$ \}, \{$\oplus$ \}, \{$\bot$ , $\top$ \},
\{$\bot$ , $\oplus$ \}, \{$\top$ , $\oplus$ \}, \{$\bot$ , $\top$ , $\oplus$ \}

\{$\bot$ , $\neg$ \}; \{$\to$ \}, \{$\nrightarrow$ \}, \{$\land$ \}, \{$\lor$ \}; \{$\top$ \}, \{$\oplus$ \}, \{$\leftrightarrow$ \}, \{$\top$ , $\oplus$ \},
\{$\top$ , $\leftrightarrow$ \}, \{$\oplus$ , $\leftrightarrow$ \}, \{$\top$ , $\oplus$ , $\leftrightarrow$ \}

\{$\bot$ , $\leftrightarrow$ \}; \{$\to$ \}, \{$\nrightarrow$ \}, \{$\land$ \}, \{$\lor$ \}; \{$\top$ \}, \{$\neg$ \}, \{$\oplus$ \}, \{$\top$ , $\neg$ \},
\{$\top$ , $\oplus$ \}, \{$\neg$ , $\oplus$ \}, \{$\top$ , $\neg$ , $\oplus$ \}

\{$\top$ , $\neg$ \}; \{$\to$ \}, \{$\nrightarrow$ \}, \{$\land$ \}, \{$\lor$ \}; \{$\bot$ \}, \{$\oplus$ \}, \{$\leftrightarrow$ \}, \{$\bot$ , $\oplus$ \},
\{$\bot$ , $\leftrightarrow$ \}, \{$\oplus$ , $\leftrightarrow$ \}, \{$\bot$ , $\oplus$ , $\leftrightarrow$ \}

\{$\top$ , $\oplus$ \}; \{$\to$ \}, \{$\nrightarrow$ \}, \{$\land$ \}, \{$\lor$ \}; \{$\bot$ \}, \{$\neg$ \}, \{$\leftrightarrow$ \}, \{$\bot$ , $\neg$ \},
\{$\bot$ , $\leftrightarrow$ \}, \{$\neg$ , $\leftrightarrow$ \}, \{$\bot$ , $\neg$ , $\leftrightarrow$ \}

\{$\leftrightarrow$ , $\oplus$ \}; \{$\to$ \}, \{$\nrightarrow$ \}, \{$\land$ \}, \{$\lor$ \}; \{$\bot$ \}, \{$\top$ \}, \{$\neg$ \}, \{$\bot$ , $\top$ \},
\{$\bot$ , $\neg$ \}, \{$\top$ , $\neg$ \}, \{$\bot$ , $\top$ , $\neg$ \}

\hypertarget{affine-or-false-preserving}{%
\subsubsection{Affine or False
Preserving}\label{affine-or-false-preserving}}

\{$\bot$ , $\oplus$ \}; \{$\to$ \}, \{$\top$ , $\land$ \}, \{$\top$ , $\lor$ \}, \{$\leftrightarrow$ , $\land$ \}, \{$\leftrightarrow$ , $\lor$ \}; \ldots{} \{$\top$ ,
$\neg$ , $\leftrightarrow$ \}, \{$\nrightarrow$ , $\lor$ , $\land$ \}

\hypertarget{affine-or-truth-preserving}{%
\subsubsection{Affine or Truth
Preserving}\label{affine-or-truth-preserving}}

\{$\top$ , $\leftrightarrow$ \}; \{$\nrightarrow$ \}, \{$\bot$ , $\land$ \}, \{$\bot$ , $\lor$ \}, \{$\oplus$ , $\land$ \}, \{$\oplus$ , $\lor$ \}; \ldots{} \{$\bot$ ,
$\neg$ , $\oplus$ \}, \{$\to$ , $\land$ , $\lor$ \}

\hypertarget{affine-and-monotonic}{%
\subsubsection{Affine and Monotonic}\label{affine-and-monotonic}}

\{$\bot$ , $\top$ \}; \{$\to$ \}, \{$\nrightarrow$ \}, \{$\leftrightarrow$ , $\land$ \}, \{$\leftrightarrow$ , $\lor$ \}, \{$\oplus$ , $\land$ \}, \{$\oplus$ , $\lor$ \}; \{$\neg$ \},
\{$\leftrightarrow$ \}, \{$\oplus$ \}, \{$\land$ \}, \{$\lor$ \}, \{$\land$ , $\lor$ \}, \{$\neg$ , $\oplus$ \}, \{$\neg$ , $\leftrightarrow$ \}, \{$\oplus$ , $\leftrightarrow$ \}, \{$\neg$ ,
$\oplus$ , $\leftrightarrow$ \}

\hypertarget{truth-preserving}{%
\subsubsection{Truth Preserving}\label{truth-preserving}}

\hypertarget{generators}{%
\paragraph{Generators}\label{generators}}

\{$\to$ , $\land$ \}; \{$\bot$ \}, \{$\neg$ \}, \{$\nrightarrow$ \}, \{$\oplus$ \}; \ldots{} \{$\top$ , $\leftrightarrow$ , $\lor$ \}

\hypertarget{non-generators}{%
\paragraph{Non-generators}\label{non-generators}}

\{$\top$ , $\to$ \}; \{$\bot$ \}, \{$\neg$ \}, \{$\nrightarrow$ \}, \{$\oplus$ \}; \ldots{} \{$\leftrightarrow$ , $\land$ , $\lor$ \}

\{$\to$ , $\leftrightarrow$ \}; \{$\bot$ \}, \{$\neg$ \}, \{$\nrightarrow$ \}, \{$\oplus$ \}; \ldots{} \{$\top$ , $\land$ , $\lor$ \}

\{$\to$ , $\lor$ \}; \{$\bot$ \}, \{$\neg$ \}, \{$\nrightarrow$ \}, \{$\oplus$ \}; \ldots{} \{$\top$ , $\leftrightarrow$ , $\land$ \}

\{$\leftrightarrow$ , $\lor$ \}; \{$\neg$ \}, \{$\nrightarrow$ \}, \{$\oplus$ \}, \{$\bot$ \}; \ldots{} \{$\top$ , $\to$ , $\land$ \}

\{$\leftrightarrow$ , $\land$ \}; \{$\neg$ \}, \{$\nrightarrow$ \}, \{$\oplus$ \}, \{$\bot$ \}; \ldots{} \{$\top$ , $\to$ , $\lor$ \}

\hypertarget{false-preserving-generator}{%
\subsubsection{False Preserving
Generator}\label{false-preserving-generator}}

\{$\nrightarrow$ , $\lor$ \}; \{$\top$ \}, \{$\neg$ \}, \{$\to$ \}, \{$\leftrightarrow$ \}

\hypertarget{false-preserving}{%
\subsubsection{False Preserving}\label{false-preserving}}

\{$\oplus$ , $\land$ \}; \{$\neg$ \}, \{$\to$ \}, \{$\leftrightarrow$ \}, \{$\top$ \}; \ldots{} \{$\bot$ , $\nrightarrow$ , $\lor$ \}

\{$\oplus$ , $\lor$ \}; \{$\neg$ \}, \{$\to$ \}, \{$\leftrightarrow$ \}, \{$\top$ \}; \ldots{} \{$\bot$ , $\nrightarrow$ , $\land$ \}

\{$\bot$ , $\nrightarrow$ \}; \{$\top$ \}, \{$\neg$ \}, \{$\to$ \}, \{$\leftrightarrow$ \}; \ldots{} \{$\oplus$ , $\lor$ , $\land$ \}

\{$\nrightarrow$ , $\oplus$ \}; \{$\top$ \}, \{$\neg$ \}, \{$\to$ \}, \{$\leftrightarrow$ \}; \ldots{} \{$\bot$ , $\lor$ , $\land$ \}

\{$\nrightarrow$ , $\land$ \}; \{$\top$ \}, \{$\neg$ \}, \{$\to$ \}, \{$\leftrightarrow$ \}; \ldots{} \{$\bot$ , $\oplus$ , $\lor$ \}

\hypertarget{monotonic-and-truth-preserving}{%
\subsubsection{Monotonic and Truth
Preserving}\label{monotonic-and-truth-preserving}}

\{$\top$ , $\lor$ \}; \{$\neg$ \}, \{$\nrightarrow$ \}, \{$\oplus$ \}, \{$\bot$ , $\leftrightarrow$ \}; \ldots{} \{$\to$ , $\leftrightarrow$ , $\land$ \}

\{$\top$ , $\land$ \}; \{$\neg$ \}, \{$\nrightarrow$ \}, \{$\oplus$ \}, \{$\bot$ , $\leftrightarrow$ \}; \ldots{} \{$\to$ , $\leftrightarrow$ , $\lor$ \}

\hypertarget{monotonic-and-false-preserving}{%
\subsubsection{Monotonic and False
Preserving}\label{monotonic-and-false-preserving}}

\{$\bot$ , $\lor$ \}; \{$\neg$ \}, \{$\to$ \}, \{$\leftrightarrow$ \}, \{$\top$ , $\oplus$ \}; \ldots, \{$\nrightarrow$ , $\oplus$ , $\land$ \}

\{$\bot$ , $\land$ \}; \{$\neg$ \}, \{$\to$ \}, \{$\leftrightarrow$ \}, \{$\top$ , $\oplus$ \}; \ldots, \{$\nrightarrow$ , $\oplus$ , $\lor$ \}

\hypertarget{section}{%
\subsubsection{}\label{section}}

\hypertarget{special}{%
\subsubsection{Special}\label{special}}

Truth or False Preserving

\{$\land$ , $\lor$ \}; \{$\neg$ \}, \{$\bot$ , $\leftrightarrow$ \}, \{$\top$ , $\oplus$ \}, \{$\leftrightarrow$ , $\oplus$ \}; \ldots{} \{$\bot$ , $\nrightarrow$ , $\oplus$ \},
\{$\top$ , $\to$ , $\leftrightarrow$ \}

\hypertarget{triples-1}{%
\subsection{Triples}\label{triples-1}}

32 functionally incomplete triples.

\hypertarget{affine-connectives}{%
\subsubsection{Affine Connectives}\label{affine-connectives}}

\hypertarget{generators-1}{%
\paragraph{Generators}\label{generators-1}}

\{$\bot$ , $\neg$ , $\oplus$ \}; \{$\to$ \}, \{$\nrightarrow$ \}, \{$\land$ \}, \{$\lor$ \}; \{$\top$ \}, \{$\leftrightarrow$ \}, \{$\top$ , $\leftrightarrow$ \}

\{$\bot$ , $\neg$ , $\leftrightarrow$ \}; \{$\to$ \}, \{$\nrightarrow$ \}, \{$\land$ \}, \{$\lor$ \}; \{$\top$ \}, \{$\oplus$ \}, \{$\top$ , $\oplus$ \}

\{$\top$ , $\neg$ , $\oplus$ \}; \{$\to$ \}, \{$\nrightarrow$ \}, \{$\land$ \}, \{$\lor$ \}; \{$\bot$ \}, \{$\leftrightarrow$ \}, \{$\bot$ , $\leftrightarrow$ \}

\{$\top$ , $\neg$ , $\leftrightarrow$ \}; \{$\to$ \}, \{$\nrightarrow$ \}, \{$\land$ \}, \{$\lor$ \}; \{$\bot$ \}, \{$\oplus$ \}, \{$\bot$ , $\oplus$ \}

\{$\neg$ , $\oplus$ , $\leftrightarrow$ \}; \{$\to$ \}, \{$\nrightarrow$ \}, \{$\land$ \}, \{$\lor$ \}; \{$\bot$ \}, \{$\top$ \}, \{$\bot$ , $\top$ \}

\hypertarget{non-generators-1}{%
\paragraph{Non-generators}\label{non-generators-1}}

\{$\bot$ , $\top$ , $\neg$ \}; \{$\to$ \}, \{$\nrightarrow$ \}, \{$\land$ \}, \{$\lor$ \}; \{$\oplus$ \}, \{$\leftrightarrow$ \}, \{$\oplus$ , $\leftrightarrow$ \}

\{$\bot$ , $\top$ , $\leftrightarrow$ \}; \{$\to$ \}, \{$\nrightarrow$ \}, \{$\land$ \}, \{$\lor$ \}; \{$\neg$ \}, \{$\oplus$ \}, \{$\neg$ , $\oplus$ \}

\{$\bot$ , $\top$ , $\oplus$ \}; \{$\to$ \}, \{$\nrightarrow$ \}, \{$\land$ \}, \{$\lor$ \}; \{$\neg$ \}, \{$\leftrightarrow$ \}, \{$\neg$ , $\leftrightarrow$ \}

\{$\bot$ , $\oplus$ , $\leftrightarrow$ \}; \{$\to$ \}, \{$\nrightarrow$ \}, \{$\land$ \}, \{$\lor$ \}; \{$\top$ \}, \{$\neg$ \}, \{$\top$ , $\neg$ \}

\{$\top$ , $\oplus$ , $\leftrightarrow$ \}; \{$\to$ \}, \{$\nrightarrow$ \}, \{$\land$ \}, \{$\lor$ \}; \{$\bot$ \}, \{$\neg$ \}, \{$\bot$ , $\neg$ \}

\hypertarget{truth-preserving-connectives}{%
\subsubsection{Truth Preserving
Connectives}\label{truth-preserving-connectives}}

\hypertarget{generators-2}{%
\paragraph{Generators}\label{generators-2}}

\{$\top$ , $\to$ , $\land$ \}; \{$\bot$ \}, \{$\neg$ \}, \{$\nrightarrow$ \}, \{$\oplus$ \}; \{$\leftrightarrow$ \}, \{$\lor$ \}, \{$\leftrightarrow$ , $\lor$ \}

\{$\to$ , $\leftrightarrow$ , $\land$ \}; \{$\bot$ \}, \{$\neg$ \}, \{$\nrightarrow$ \}, \{$\oplus$ \}; \{$\top$ \}, \{$\lor$ \}, \{$\top$ , $\lor$ \}

\{$\to$ , $\land$ , $\lor$ \}; \{$\bot$ \}, \{$\neg$ \}, \{$\nrightarrow$ \}, \{$\oplus$ \}; \{$\top$ \}, \{$\leftrightarrow$ \}, \{$\top$ , $\leftrightarrow$ \}

\hypertarget{non-generators-2}{%
\paragraph{Non-generators}\label{non-generators-2}}

\{$\top$ , $\to$ , $\leftrightarrow$ \}; \{$\bot$ \}, \{$\neg$ \}, \{$\nrightarrow$ \}, \{$\oplus$ \}; \{$\land$ \}, \{$\lor$ \}, \{$\land$ , $\lor$ \}

\{$\top$ , $\to$ , $\lor$ \}; \{$\bot$ \}, \{$\neg$ \}, \{$\nrightarrow$ \}, \{$\oplus$ \}; \{$\leftrightarrow$ \}, \{$\land$ \}, \{$\leftrightarrow$ , $\land$ \}\\
\{$\top$ , $\leftrightarrow$ , $\lor$ \}; \{$\bot$ \}, \{$\neg$ \}, \{$\nrightarrow$ \}, \{$\oplus$ \}; \{$\to$ \}, \{$\land$ \}, \{$\to$ , $\land$ \}\\
\{$\top$ , $\leftrightarrow$ , $\land$ \}; \{$\bot$ \}, \{$\neg$ \}, \{$\nrightarrow$ \}, \{$\oplus$ \}; \{$\to$ \}, \{$\lor$ \}, \{$\to$ , $\lor$ \}\\
\{$\to$ , $\leftrightarrow$ , $\lor$ \}; \{$\bot$ \}, \{$\neg$ \}, \{$\nrightarrow$ \}, \{$\oplus$ \}; \{$\top$ \}, \{$\land$ \}, \{$\top$ , $\land$ \}\\
\{$\leftrightarrow$ , $\land$ , $\lor$ \}; \{$\bot$ \}, \{$\neg$ \}, \{$\nrightarrow$ \}, \{$\oplus$ \}; \{$\top$ \}, \{$\to$ \}, \{$\top$ , $\to$ \}

\hypertarget{false-preserving-connectives}{%
\subsubsection{False Preserving
Connectives}\label{false-preserving-connectives}}

\{$\bot$ , $\nrightarrow$ , $\oplus$ \}; \{$\top$ \}, \{$\neg$ \}, \{$\to$ \}, \{$\leftrightarrow$ \}; \{$\land$ \}, \{$\lor$ \}, \{$\land$ , $\lor$ \}

\{$\bot$ , $\oplus$ , $\lor$ \}; \{$\top$ \}, \{$\neg$ \}, \{$\to$ \}, \{$\leftrightarrow$ \}; \{$\nrightarrow$ \}, \{$\land$ \}, \{$\nrightarrow$ , $\land$ \}

\{$\bot$ , $\oplus$ , $\land$ \}; \{$\top$ \}, \{$\neg$ \}, \{$\to$ \}, \{$\leftrightarrow$ \}; \{$\nrightarrow$ \}, \{$\lor$ \}, \{$\nrightarrow$ , $\lor$ \}

\{$\bot$ , $\nrightarrow$ , $\land$ \}; \{$\top$ \}, \{$\neg$ \}, \{$\to$ \}, \{$\leftrightarrow$ \}; \{$\oplus$ \}, \{$\lor$ \}, \{$\oplus$ , $\lor$ \}

\{$\bot$ , $\nrightarrow$ , $\lor$ \}; \{$\top$ \}, \{$\neg$ \}, \{$\to$ \}, \{$\leftrightarrow$ \}; \{$\land$ \}, \{$\oplus$ \}, \{$\oplus$ , $\land$ \}

\{$\nrightarrow$ , $\oplus$ , $\land$ \}; \{$\top$ \}, \{$\neg$ \}, \{$\to$ \}, \{$\leftrightarrow$ \}; \{$\bot$ \}, \{$\lor$ \}, \{$\bot$ , $\lor$ \}

\{$\nrightarrow$ , $\oplus$ , $\lor$ \}; \{$\top$ \}, \{$\neg$ \}, \{$\to$ \}, \{$\leftrightarrow$ \}; \{$\bot$ \}, \{$\land$ \}, \{$\bot$ , $\land$ \}\\
\{$\nrightarrow$ , $\land$ , $\lor$ \}; \{$\top$ \}, \{$\neg$ \}, \{$\to$ \}, \{$\leftrightarrow$ \}; \{$\bot$ \}, \{$\oplus$ \}, \{$\bot$ , $\oplus$ \}

\{$\oplus$ , $\land$ , $\lor$ \}; \{$\top$ \}, \{$\neg$ \}, \{$\to$ \}, \{$\leftrightarrow$ \}; \{$\bot$ \}, \{$\nrightarrow$ \}, \{$\bot$ , $\nrightarrow$ \}

\hypertarget{exclusively-monotonic-connectives}{%
\subsubsection{\texorpdfstring{Exclusively Monotonic Connectives
}{Exclusively Monotonic Connectives }}\label{exclusively-monotonic-connectives}}

\{$\bot$ , $\top$ , $\lor$ \}; \{$\neg$ \}, \{$\to$ \}, \{$\nrightarrow$ \}, \{$\oplus$ \}, \{$\leftrightarrow$ \}; \{$\land$ \}

\{$\bot$ , $\top$ , $\land$ \}; \{$\neg$ \}, \{$\to$ \}, \{$\nrightarrow$ \}, \{$\oplus$ \}, \{$\leftrightarrow$ \}; \{$\lor$ \}

\hypertarget{monotonic-intersections}{%
\subsubsection{Monotonic Intersections}\label{monotonic-intersections}}

Monotonic and Truth Preserving

\{$\bot$ , $\land$ , $\lor$ \}; \{$\neg$ \}, \{$\to$ \}, \{$\leftrightarrow$ \}; \{$\top$ \}, \{$\nrightarrow$ \}, \{$\oplus$ \}, \{$\nrightarrow$ , $\oplus$ \}

Monotonic and False Preserving\\
\{$\top$ , $\land$ , $\lor$ \}; \{$\neg$ \}, \{$\nrightarrow$ \}, \{$\oplus$ \}; \{$\bot$ \}, \{$\to$ \}, \{$\leftrightarrow$ \}, \{$\to$ , $\leftrightarrow$ \}

\hypertarget{quadruples}{%
\subsection{Quadruples}\label{quadruples}}

16 functionally incomplete quadruples.

\hypertarget{affine-connectives-1}{%
\subsubsection{Affine Connectives}\label{affine-connectives-1}}

\{$\bot$ , $\top$ , $\oplus$ , $\leftrightarrow$ \}; \{$\to$ \}, \{$\nrightarrow$ \}, \{$\land$ \}, \{$\lor$ \}; \{$\neg$ \}

\{$\neg$ , $\top$ , $\oplus$ , $\leftrightarrow$ \}; \{$\to$ \}, \{$\nrightarrow$ \}, \{$\land$ \}, \{$\lor$ \}; \{$\bot$ \}

\{$\neg$ , $\bot$ , $\oplus$ , $\leftrightarrow$ \}; \{$\to$ \}, \{$\nrightarrow$ \}, \{$\land$ \}, \{$\lor$ \}; \{$\top$ \}

\{$\neg$ , $\bot$ , $\top$ , $\leftrightarrow$ \}; \{$\to$ \}, \{$\nrightarrow$ \}, \{$\land$ \}, \{$\lor$ \}; \{$\oplus$ \}

\{$\neg$ , $\bot$ , $\top$ , $\oplus$ \}; \{$\to$ \}, \{$\nrightarrow$ \}, \{$\land$ \}, \{$\lor$ \}; \{$\leftrightarrow$ \}

\hypertarget{truth-preserving-connectives-1}{%
\subsubsection{Truth Preserving
Connectives}\label{truth-preserving-connectives-1}}

\hypertarget{generators-3}{%
\paragraph{Generators}\label{generators-3}}

\{$\to$ , $\top$ , $\land$ , $\leftrightarrow$ \}; \{$\bot$ \}, \{$\neg$ \}, \{$\nrightarrow$ \}, \{$\oplus$ \}; \{$\lor$ \}

\{$\to$ , $\land$ , $\lor$ , $\leftrightarrow$ \}; \{$\bot$ \}, \{$\neg$ \}, \{$\nrightarrow$ \}, \{$\oplus$ \}; \{$\top$ \}

\{$\to$ , $\top$ , $\land$ , $\lor$ \}; \{$\bot$ \}, \{$\neg$ \}, \{$\nrightarrow$ \}, \{$\oplus$ \}; \{$\leftrightarrow$ \}

\hypertarget{non-generators-3}{%
\paragraph{Non-generators}\label{non-generators-3}}

\{$\top$ , $\land$ , $\lor$ , $\leftrightarrow$ \}; \{$\bot$ \}, \{$\neg$ \}, \{$\nrightarrow$ \}, \{$\oplus$ \}; \{$\to$ \}

\{$\to$ , $\top$ , $\lor$ , $\leftrightarrow$ \}; \{$\bot$ \}, \{$\neg$ \}, \{$\nrightarrow$ \}, \{$\oplus$ \}; \{$\land$ \}

\hypertarget{false-preserving-connectives-1}{%
\subsubsection{False Preserving
Connectives}\label{false-preserving-connectives-1}}

\hypertarget{generators-4}{%
\paragraph{Generators}\label{generators-4}}

\{$\bot$ , $\nrightarrow$ , $\lor$ , $\oplus$ \}; \{$\top$ \}, \{$\neg$ \}, \{$\to$ \}, \{$\leftrightarrow$ \}; \{$\land$ \}

\{$\bot$ , $\nrightarrow$ , $\land$ , $\lor$ \}; \{$\top$ \}, \{$\neg$ \}, \{$\to$ \}, \{$\leftrightarrow$ \}; \{$\oplus$ \}

\{$\nrightarrow$ , $\land$ , $\lor$ , $\oplus$ \}; \{$\top$ \}, \{$\neg$ \}, \{$\to$ \}, \{$\leftrightarrow$ \}; \{$\bot$ \}

\hypertarget{non-generators-4}{%
\paragraph{Non-generators}\label{non-generators-4}}

\{$\bot$ , $\nrightarrow$ , $\land$ , $\oplus$ \}; \{$\top$ \}, \{$\neg$ \}, \{$\to$ \}, \{$\leftrightarrow$ \}; \{$\lor$ \}

\{$\bot$ , $\land$ , $\lor$ , $\oplus$ \}; \{$\top$ \}, \{$\neg$ \}, \{$\to$ \}, \{$\leftrightarrow$ \}; \{$\nrightarrow$ \}

\hypertarget{monotonic-connectives}{%
\subsubsection{\texorpdfstring{Monotonic Connectives
}{Monotonic Connectives }}\label{monotonic-connectives}}

\{$\bot$ , $\top$ , $\land$ , $\lor$ \}; \{$\neg$ \}, \{$\to$ \}, \{$\nrightarrow$ \}, \{$\oplus$ \}, \{$\leftrightarrow$ \}

\hypertarget{quintuples}{%
\subsection{Quintuples}\label{quintuples}}

\hypertarget{affine-connectives-2}{%
\subsubsection{Affine Connectives}\label{affine-connectives-2}}

\{$\bot$ , $\top$ , $\neg$ , $\oplus$ , $\leftrightarrow$ \}; \{$\to$ \}, \{$\nrightarrow$ \}, \{$\land$ \}, \{$\lor$ \}

\hypertarget{truth-preserving-connectives-2}{%
\subsubsection{Truth Preserving
Connectives}\label{truth-preserving-connectives-2}}

\{$\top$ , $\to$ , $\leftrightarrow$ , $\land$ , $\lor$ \}; \{$\bot$ \}, \{$\neg$ \}, \{$\nrightarrow$ \}, \{$\oplus$ \}

\hypertarget{false-preserving-connectives-2}{%
\subsubsection{False Preserving
Connectives}\label{false-preserving-connectives-2}}

\{$\bot$ , $\nrightarrow$ , $\oplus$ , $\lor$ , $\land$ \}; \{$\top$ \}, \{$\neg$ \}, \{$\to$ \}, \{$\leftrightarrow$ \}

\hypertarget{sextuples-and-higher}{%
\subsection{Sextuples and higher}\label{sextuples-and-higher}}

All subsets of \{$\bot$ , $\top$ , $\neg$ , $\to$ , $\nrightarrow$ , $\oplus$ , $\leftrightarrow$ , $\land$ , $\lor$ \} of cardinality 6 or higher
are functionally complete.

Proof: \{$\bot$ , $\top$ , $\oplus$ , $\leftrightarrow$ , $\land$ , $\lor$ \} and \{$\neg$ , $\to$ , $\nrightarrow$ \} are functionally complete.
Every 5 element subset of \{$\bot$ , $\top$ , $\oplus$ , $\leftrightarrow$ , $\land$ , $\lor$ \} is functionally complete;
there exist 4 element subsets which are functionally incomplete, but the
only way to get six element sets from those 4 element sets is to add two
elements from \{$\neg$ , $\to$ , $\nrightarrow$ \} and any two elements of \{$\neg$ , $\to$ , $\nrightarrow$ \} are
functionally complete together.

\hypertarget{special-cases-of-functional-incomplete-sets-of-operators}{%
\subsection{Special Cases of functional incomplete sets of
operators}\label{special-cases-of-functional-incomplete-sets-of-operators}}

\{$\bot$ , $\top$ , $\oplus$ , $\leftrightarrow$ \} $\nmodels$  \{$\neg$ \}

\{$\top$ , $\to$ , $\leftrightarrow$ , $\lor$ \} $\nmodels$  \{$\lor$ \}

\{$\bot$ , $\nrightarrow$ , $\oplus$ , $\land$ \} $\nmodels$  \{$\land$ \}

\{$\neg$ , $\oplus$ \} $\vdash$  \{$\bot$ , $\top$ , $\neg$ , $\oplus$ , $\leftrightarrow$ \}

\{$\neg$ , $\leftrightarrow$ \} $\vdash$  \{$\bot$ , $\top$ , $\neg$ , $\oplus$ , $\leftrightarrow$ \}

\{$\to$ , $\lor$ \} $\vdash$  \{$\top$ , $\to$ , $\lor$ \}

\{$\to$ , $\land$ \} $\vdash$  \{$\top$ , $\to$ , $\leftrightarrow$ , $\land$ \}

\{$\nrightarrow$ , $\lor$ \} $\vdash$  \{$\bot$ , $\nrightarrow$ , $\oplus$ , $\lor$ \}

\{$\oplus$ , $\leftrightarrow$ \} $\vdash$  \{$\bot$ , $\top$ , $\oplus$ , $\leftrightarrow$ \}

\{$\to$ , $\leftrightarrow$ \} $\vdash$  \{$\top$ , $\to$ , $\leftrightarrow$ \}

\{$\nrightarrow$ , $\oplus$ \} $\vdash$  \{$\bot$ , $\nrightarrow$ , $\oplus$ \}

\{$\nrightarrow$ \}$\vdash$ \{$\bot$ , $\nrightarrow$ \}

\{$\oplus$ \}$\vdash$ \{$\bot$ , $\oplus$ \}

\{$\to$ \}$\vdash$ \{$\top$ ,$\to$ \}

\{$\leftrightarrow$ \}$\vdash$ \{$\top$ , $\leftrightarrow$ \}

\hypertarget{functionally-complete-sets}{%
\section{Functionally Complete Sets}\label{functionally-complete-sets}}

There are 512 subsets of \{$\bot$ , $\top$ , $\neg$ , $\to$ , $\nrightarrow$ , $\oplus$ , $\leftrightarrow$ , $\land$ , $\lor$ \}. 425 subsets are
functionally complete.

There are 2048 subsets of \{$\bot$ , $\top$ , $\neg$ , $\to$ , $\nrightarrow$ , $\oplus$ , $\leftrightarrow$ , $\land$ , $\lor$ , $\downarrow$ , $\uparrow$ \}. 1961
subsets are functionally complete.

NUMERICAL ENCODING! Based on the MAXIMALLY INCOMPLETE SETS

Class 1: truth-preserving 10000\\
Class 2: false-preserving 01000\\
Class 3: affine 00100

Class 4: monotone 00010

Class 5: self-dual 00001

Any connective can be expressed uniquely as a binary numeral.\\
AND is truth preserving. 10010

\{$\bot$ , $\top$ , $\neg$ , $\oplus$ , $\leftrightarrow$ \} Affine set

\{$\bot$ , $\nrightarrow$ , $\oplus$ , $\lor$ , $\land$ \} False Preserving set

\{$\top$ , $\to$ , $\leftrightarrow$ , $\land$ , $\lor$ \} Truth Preserving set

\{$\bot$ , $\top$ , $\land$ , $\lor$ \} Monotonic set

\{$\neg$ \} Self-Dual set

LIST OF NUMERICAL ENCODINGS!

$\bot$ : false preserving, affine. 01100

$\top$ : truth preserving, affine. 10100

$\neg$ : self-dual: 00001, affine: 00100. 00101

$\to$  Truth Preserving 10000

$\nrightarrow$  False Preserving 01000

$\oplus$  False Preserving, Affine 01100

$\leftrightarrow$  Truth Preserving, Affine 10100

$\land$  False Preserving, Truth Preserving, Monotonic 11010

$\lor$  False Preserving, Truth Preserving, Monotonic 11010

$\downarrow$ : 00000

$\uparrow$ : 00000

PROVING WHICH SETS ARE FUNCTIONALLY COMPLETE

A functionally complete set is defined by the property that when all its
members are summed (with bitwise OR), the sum is 00000.

\{$\downarrow$ \}: 00000

\{$\uparrow$ \}: 00000

\{$\bot$ , $\to$ \} : 01100 \textbar\textbar{} 10100 = 00100

AND: AB

OR: 1 - (1 - A)(1 - B) = A + B - AB

NAND: 1 - AB

NOR: 1 - A - B + AB

IMPLIES: 1 - A + AB

NONIMPLIES: A(1-B)


\chapter{Fundamentally in Error the Growth of Errors}
You can redefine the paradigm on your own terms, but its difficulty is
based on your environment. What it comes down to is agreement on first
principles and methods. The US has effectively unilaterally rejected
intellectual enterprise and everything with it while retaining a
facsimile. Which means rejecting scientific and deductive method.
Rejecting the idea of method itself. Rejecting reasoning by principles.
Rejecting reasoning. The US thus is tending to become a country that is
in principle unreasonable.

With a problem at the root like that, I can not thrive here. I am in
fundamental opposition to the extremes that represents.

The extremes that leads to. It means our culture is tending towards an
anything goes country. Where all opinions, feelings, emotions, thoughts,
perceptions, and actions are treated as valid. Where everything has
trivial value and is absolutely paradoxically true and false.
There\textquotesingle s one case in which the absolute extreme of
consistency makes complete sense: the finite case.

Finite systems which are consistent and explosive can be complete.
That\textquotesingle s what first order predicate logic is. Beyond that,
explosive consistency makes semi-complete sense up to countability and
in discrete systems. Those systems which are countably infinite. Beyond
that, inconsistencies abound at the top and bottom limits of continuous
systems. Manifesting around things like singularities. Explosive
consistency tends towards absolute contradiction at the limits of any
formal system. To me, this amount to saying that if consistency is like
a laser beam aimed at a prism, the limits of the system are that prism
and they decohere the beam into random noise. Shattering any sense of
reason we\textquotesingle d understand and accept.

In that regard, I think things like quantum mechanics and relativity
have done tremendous damage. For many people who take the theories as
literally, finally, absolute, and completely true, they shatter the
suspension of disbelief, and by proxy they shatter certainty and
confidence in any system of reason human beings can devise.

The probability of a erroneous argument going to absolute contradiction
is 1 as the argument approaches infinity.

``Suppose we have a system which is contradictory. Suppose the system
appears to be consistent for any finite domain. Precisely stated suppose
the system approximates a consistent system for any finite domain. Let
f(x) be a non-finite continuous function. Can we show that the system
approaches an absolutely contradictory system at the limits?''

Another example of a self-referential paradox, a recursive paradox: the
grand-father paradox and the bootstrap paradox. General relativity
predicts backward time travel leading to contradictions.

Heisenberg uncertainty is intrinsic error in the hard measurement of a
physical system.

General relativity ensures conservation up to parity. If our universe
exists in a bubble, a blackhole, we can expect that the parity of the
blackhole would be opposite of other blackholes. We could expect then
universes with a reversed arrow of time.

Single Electron hypothesis implies the connection between matter and
anti-matter particles with respect to time.

\url{http://eaps4.mit.edu/research/Lorenz/The_Growth_of_Errors_in_Prediction_1985.pdf}

``By a stable system we mean one whose future succession of states, if

the present state should be slightly disturbed, will converge toward

the succession of states which have occurred if there had been no

disturbance. By an unstable system we mean just the opposite--a system

whose future following a slight disturbance will diverge from what its

future would have been without a disturbance.''

This should be compared to Turing's thesis on automatic machines in

which a deterministic system is compared to a non-deterministic system

in terms of the direction or vector of the transition or

transformation of states. A deterministic machine evolves from an

initial state to a final state in a linear order such that future

states depend directly and exactly upon previous or past states; a

non-deterministic machine's evolution from initial to final states is

not necessarily directly dependent on past states.

Hypothetically, The notion of stable vs unstable systems is analogous

to the input and output of a Turing machine such a Turing machine is

called stable if a small input causes the Turing machine to converge

on a designated final state expected a priori. A Turing machine is

called unstable if a small input causes the Turing machine to diverge

from a designated final stated expected a priori. A simple stable

Turing machine would always output a constant state in response to any

input; a simple unstable Turing machine would always output a random

state in response to every input. In general the stability and

instability of a Turing machine would exist in a inverse relationship

by degree such that a Turing machine could be somewhat stable and

somewhat unstable but not arbitrarily so.

Similarly, the determinism and non-determinism of the Turing machine

can be related in shared degrees, and the degrees of deterministic and

non-deterministic relations can be related to the degrees of stable

and unstable relations.

Further research into quantum, fuzzy, and paraconsistent mechanics

would likely be productive with respect to the corresponding degrees

of relations of the mechanical state.


\chapter{Graph of Calculi Philosophy}

\documentclass{article}

\usepackage{amsmath}
\usepackage{ebproof}
\usepackage{fullpage}
\usepackage[utf8]{inputenc}
\usepackage{newunicodechar}
\usepackage{stix}

\newunicodechar{Γ}{\Gamma}
\newunicodechar{Δ}{\Delta}

\newunicodechar{Θ}{\Theta}
\newunicodechar{Λ}{\Lambda}

\newunicodechar{Ξ}{\Xi}
\newunicodechar{Π}{\Pi}

\newunicodechar{Φ}{\Phi}
\newunicodechar{Ψ}{\Psi}

\newunicodechar{Ω}{\Omega}

\newunicodechar{⊢}{\vdash}

\newunicodechar{⊕}{\oplus}
\newunicodechar{¬}{\neg}

\newunicodechar{⊗}{\otimes}
\newunicodechar{→}{\rightarrow}
\newunicodechar{←}{\leftarrow}
\newunicodechar{↔}{\leftrightarrow}
\newunicodechar{↮}{\nleftrightarrow}


\newunicodechar{⅋}{\upand}
\newunicodechar{↛}{\nrightarrow}
\newunicodechar{↚}{\nleftarrow}

\newunicodechar{⊥}{\bot}
\newunicodechar{⊤}{\top}

\setlength{\parindent}{0em}

\author{James Martin, Ian D.L.N. Mclean}
\title{The Lattice of Conservative Non-Classical and Classical Sequent Calculi}

\begin{document}

\maketitle

\begin{abstract}
Subclassical sequent calculi are defined by functional incompleteness and a strict subset of de Morgan dualities for the quantified predicate calculi.
Superclassical sequent calculi are defined by at least functional completeness and classical de Morgan duality.
\end{abstract}

\part{Preliminaries}
\begin{center}
	\begin{flushleft}
		The key concept to the construction of logical calculi that are distinctly different from classical logic is functional incompleteness with respect to the Boolean domain and Boolean functions.
	\end{flushleft}
	\begin{flushleft}
		If a calculus has any classical logical connectives such that we can express every theorem of the classical calculus then our logical calculi would degenerate to the classical calculus and we'd lose in general the specificity that is gained by constructive reasoning.
	\end{flushleft}
	\begin{flushleft}
		Roughly speaking, if any set of operators or functions is not a subset of at least one of the five functionally incomplete sets then that set of operators or functions is functionally complete.
	\end{flushleft}
	\begin{flushleft}
		We generalize this in a manner analogous to Tarski's original formal interpretation as applied to the problem of decidability of theories in standard formalization. If we have collections of non-classical or sub-classical operators or functions that can be interpreted in at least one of the five functionally incomplete sets then those collections of operators or functions are functionally incomplete with respect to the Boolean domain and Boolean functions.
	\end{flushleft}
	\begin{flushleft}
		Finally, if we restrict ourselves to only those logical connectives which are classical then the systems can not extend to each other and it seems can not interpret each other, but if we extend these systems by some non-classical logical connective that is compatible with a given set of functionally incomplete logical connectives then we can extend between these systems and interpret between them.
	\end{flushleft}
	\section{Interpretations}
		\subsection{Operational Interpretations}
		\subsection{Structural Interpretations}
		\subsection{Systematic Interpretations}
	\section{Formal Definition of Subclassical Theories}
		\subsection{Monotonic Theories}
		$\left\{ Γ ⊢ A \right\} \bigcup \left\{ A ⊢ Δ \right\}$
		\subsection{Truth-Preserving Theories}
		$\left\{ A ⊢ Δ \right\}$
		\subsection{False-Preserving Theories}
		$\left\{ Γ ⊢ A \right\}$
		\subsection{Affine Theories}
		$\left\{ Γ ⊢ A \right\} \bigcap \left\{ A ⊢ Δ \right\}=\emptyset$
		\subsection{Self-Dual Theories}
		$\left\{ A ⊢ A \right\}$

\end{center}

\part{Conservative Duality}
\section{Formal Definition of Subcalculi}
Definition: A subcalculus of a logical system L is a logical system L' that satisfies the following properties:

Syntactic inclusion: The set of formulas of L' is a subset of the set of formulas of L.

Structural inclusion: The set of proofs of L' is a subset of the set of proofs of L.

Soundness: Every proof in L' is also a proof in L.


In formal logic, a subcalculus is a logical system that is a subset of another logical system. This means that the subcalculus has the same basic structure and principles as the larger system, but it may have fewer rules or omit certain features.

Formally, a subcalculus can be defined as a tuple (L, R) where:

L is a set of logical symbols, including propositional variables, connectives, and possibly quantifiers.

R is a set of inference rules that specify how to construct valid proofs in the system.

A subcalculus S is a subset of another calculus C if:

$ L_S $ is a subset of $ L_C $.

$ R_S $ is a subset of $ R_C $.

This means that the subcalculus S has the same logical symbols as the calculus C, but it may have fewer rules of inference.

\section{Formal Definition of Supercalculi}
Definition: A supercalculus of a logical system L is a logical system L' that satisfies the following properties:

Syntactic extension: The set of formulas of L is a subset of the set of formulas of L'.

Structural extension: The set of proofs of L is a subset of the set of proofs of L'.

Completeness: Every proof in L can be extended to a proof in L'.

Expressive power: L' is at least as expressive as L, meaning that every formula that is valid in L is also valid in L'.

Conservativeness: L' is conservative over L, meaning that every proof in L can be translated into an equivalent proof in L'.


\section{Formal Definition of Isomorphic Calculi}
Definition: Two logical systems L and L' are isomorphic if there exists an effective bijection between the set of formulas of L and the set of formulas of L' that preserves the logical structure and properties of formulas. 

An isomorphic calculus is a logical system that is structurally and semantically equivalent to another logical system. This means that the two systems have the same expressive power and the same reasoning capabilities. They can be used to represent the same logical relationships and derive the same conclusions. They are mutually simulateable, mutually interpretable, and can be transformed one into the other.

Here are some key characteristics of isomorphic calculi:

Structural equivalence: The syntax of formulas and proofs in the two systems is equivalent, meaning that they have the same structure and organization.

Semantic equivalence: The meaning of formulas and the validity of proofs are equivalent, meaning that the same formulas are true and the same proofs are valid in both systems.

Preservation of logical relationships: The bijection between formulas preserves logical relationships, such as equivalence, implication, and negation.

Preservation of reasoning capabilities: The bijection between proofs preserves the reasoning capabilities of the systems, meaning that any proof in one system can be translated into an equivalent proof in the other system.

\section{Formal Definition of Dual Calculi}
Definition: A pair of logical systems L and L' are called dual calculi if they satisfy the following properties:

Expressive complementarity: Every formula in L can be translated into a logically equivalent formula in L', and vice versa.

Reflective validity: Every valid proof in L can be translated into a valid proof in L' that establishes the negation of its corresponding translated formula, and vice versa.

In other words, dual calculi are two logical systems that are mutually complementary in their expressive power and reasoning capabilities. Each system can express and prove the negation of what the other system can express and prove. This duality relationship arises from the systems' contrasting approaches to handling logical concepts, such as negation, implication, and modality.

Here are some key characteristics of dual calculi:

Symmetric expressive complementarity: The expressive complementarity property is symmetric, meaning that the translation between formulas in one system and the negation of corresponding formulas in the other system is a bijection.

Reflective validity preservation: The reflective validity property ensures that the translation between proofs preserves the validity of proofs, establishing the negation of translated formulas in the corresponding system.

Interdependence of expressive power: The expressive power of each system is intertwined with the other, as they can only fully express the negation of what the other system can express.

Complementary reasoning capabilities: The reasoning capabilities of each system complement each other, as they can prove the negation of what the other system can prove.


Theorem 1: For any two sequent calculi C1 and C2, the following are mutually exclusive and jointly exhaustive:
C1 is a conservative extension of C2.
C2 is a conservative extension of C1.
C1 and C2 are conservative duals of each other.
C1 is a conservative subtension of C2.
C2 is a conservative subtension of C1.
Corollary1: There are no other ways for two sequent calculi to be related to each other.
Theorem 2: The lattice of sequent calculi has a supremum sequent calculus and an infimum sequent calculus.
Corollary 2: The supremum sequent calculus is superclassical.
Corollary 3: The infimum sequent calculus is subclassical.

Theorems:

Theorem 1: For any two sequent calculi C1 and C2, the following are mutually exclusive and jointly exhaustive:
C1 is a conservative extension of C2.
C2 is a conservative extension of C1.
C1 and C2 are conservative duals of each other.
C1 is a conservative subtension of C2.
C2 is a conservative subtension of C1.

Corollary: There are no other ways for two sequent calculi to be related to each other.

Theorem 2: The lattice of sequent calculi has a supremum sequent calculus and an infimum sequent calculus.

Corollary 1: The supremum sequent calculus is superclassical.

Corollary 2: The infimum sequent calculus is subclassical.
\begin{center}
	
	\section{Structural Monotone Calculus}
		\subsection{Structural Rules}
		\begin{center}
			\[
			\begin{prooftree}
			\infer0[Id]{A ⊢ A}
			\end{prooftree}
			\]

			\[
			\begin{prooftree}
			\hypo{Γ ⊢ A}
			\hypo{A ⊢ Δ}
			\infer2[Cut]{Γ ⊢ Δ}
			\end{prooftree}
			\]

			\[
			\begin{prooftree}
			\hypo{Γ ⊢ Δ}
			\infer1[wL]{Γ, A ⊢ Δ}
			\end{prooftree}
			\qquad
			\begin{prooftree}
			\hypo{Γ ⊢ Δ}
			\infer1[Rw]{Γ ⊢ A, Δ}
			\end{prooftree}
			\]

			\[
			\begin{prooftree}
			\hypo{Γ, A, A ⊢ Δ}
			\infer1[cL]{Γ, A ⊢ Δ}
			\end{prooftree}
			\qquad
			\begin{prooftree}
			\hypo{Γ ⊢ A, A Δ}
			\infer1[Rc]{Γ ⊢ A, Δ}
			\end{prooftree}
			\]

			\[
			\begin{prooftree}
			\hypo{Γ_0, A, B, Γ_1 ⊢ Δ}
			\infer1[pL]{Γ_0, B, A, Γ_1 ⊢ Δ}
			\end{prooftree}
			\qquad
			\begin{prooftree}
			\hypo{Γ ⊢ Δ_1, A, B, Δ_0}
			\infer1[Rp]{Γ ⊢ Δ_1, B, A, Δ_0}
			\end{prooftree}
			\]
		\end{center}
		
		\subsection{Unit Rules}
		\begin{center}
			\[
			\begin{prooftree}
			\infer0{Γ, ⊥ ⊢ Δ}
			\end{prooftree}
			\quad
			\begin{prooftree}
			\infer0{ Γ ⊢ ⊤, Δ}
			\end{prooftree}
			\]
		\end{center}
		
		\subsection{Operational Rules}
		\begin{center}
		
			\subsubsection{Multiplicatives}
			\begin{center}
				\[
				\begin{prooftree}
				\hypo{Γ, A, B ⊢ Δ}
				\infer1{Γ, A ⊗ B ⊢ Δ}
				\end{prooftree}
				\quad
				\begin{prooftree}
				\hypo{Γ ⊢ A, Δ}
				\hypo{Γ ⊢ B, Δ}
				\infer2{Γ ⊢ A ⊗ B, Δ}
				\end{prooftree}
				\]
				
				\[
				\begin{prooftree}
				\hypo{Γ, A ⊢ Δ}
				\hypo{Γ, B ⊢ Δ}
				\infer2{Γ, A ⅋ B ⊢ Δ}
				\end{prooftree}
				\quad
				\begin{prooftree}
				\hypo{Γ ⊢ A, B, Δ}
				\infer1{Γ ⊢ A ⅋ B, Δ}
				\end{prooftree}
				\]
			\end{center}
		\end{center}
		
		\subsection{Theorems}
		\begin{center}
		\end{center}

	\section{Structural Truth-Preserving Calculus}
		\subsection{Structural Rules}
		\begin{center}
			\[
			\begin{prooftree}
			\infer0[Id]{A ⊢ A}
			\end{prooftree}
			\]

			\[
			\begin{prooftree}
			\hypo{Γ ⊢ A}
			\hypo{A ⊢ Δ}
			\infer2[Cut]{Γ ⊢ Δ}
			\end{prooftree}
			\]

			\[
			\begin{prooftree}
			\hypo{Γ ⊢ Δ}
			\infer1[wL]{Γ, A ⊢ Δ}
			\end{prooftree}
			\qquad
			\begin{prooftree}
			\hypo{Γ ⊢ Δ}
			\infer1[Rw]{Γ ⊢ A, Δ}
			\end{prooftree}
			\]

			\[
			\begin{prooftree}
			\hypo{Γ, A, A ⊢ Δ}
			\infer1[cL]{Γ, A ⊢ Δ}
			\end{prooftree}
			\qquad
			\begin{prooftree}
			\hypo{Γ ⊢ A, A Δ}
			\infer1[Rc]{Γ ⊢ A, Δ}
			\end{prooftree}
			\]

			\[
			\begin{prooftree}
			\hypo{Γ_0, A, B, Γ_1 ⊢ Δ}
			\infer1[pL]{Γ_0, B, A, Γ_1 ⊢ Δ}
			\end{prooftree}
			\qquad
			\begin{prooftree}
			\hypo{Γ ⊢ Δ_1, A, B, Δ_0}
			\infer1[Rp]{Γ ⊢ Δ_1, B, A, Δ_0}
			\end{prooftree}
			\]
		\end{center}
		
		\subsection{Unit Rules}
		\begin{center}
			\[
			\begin{prooftree}
			\infer0{ Γ ⊢ ⊤, Δ}
			\end{prooftree}
			\]
		\end{center}
		
		\subsection{Operational Rules}
		\begin{center}

			\subsubsection{Multiplicatives}
			\begin{center}
				\[
				\begin{prooftree}
				\hypo{Γ, A, B ⊢ Δ}
				\infer1{Γ, A ⊗ B ⊢ Δ}
				\end{prooftree}
				\quad
				\begin{prooftree}
				\hypo{Γ ⊢ A, Δ}
				\hypo{Γ ⊢ B, Δ}
				\infer2{Γ ⊢ A ⊗ B, Δ}
				\end{prooftree}
				\]
				
				\[
				\begin{prooftree}
				\hypo{Γ, A ⊢ Δ}
				\hypo{Γ, B ⊢ Δ}
				\infer2{Γ, A ⅋ B ⊢ Δ}
				\end{prooftree}
				\quad
				\begin{prooftree}
				\hypo{Γ ⊢ A, B, Δ}
				\infer1{Γ ⊢ A ⅋ B, Δ}
				\end{prooftree}
				\]
				
				\[
				\begin{prooftree}
				\hypo{Γ ⊢ A, Δ}
				\hypo{Γ, B ⊢ Δ}
				\infer2{Γ, A → B ⊢ Δ}
				\end{prooftree}
				\quad
				\begin{prooftree}
				\hypo{Γ, A ⊢ B, Δ}
				\infer1{Γ ⊢ A → B, Δ}
				\end{prooftree}
				\]
				
				\[
				\begin{prooftree}
				\hypo{Γ, A ⊢ Δ}
				\hypo{Γ ⊢ B, Δ}
				\infer2{Γ, A ← B ⊢ Δ}
				\end{prooftree}
				\quad
				\begin{prooftree}
				\hypo{Γ, B ⊢ A, Δ}
				\infer1{Γ ⊢ A ← B, Δ}
				\end{prooftree}
				\]
				
				\[
				\begin{prooftree}
				\hypo{Γ ⊢ A, B, Δ}
				\infer0{Γ, A ⊢ A, Δ}
				\infer0{Γ, B ⊢ B, Δ}
				\hypo{Γ, A, B ⊢ Δ}
				\infer4{Γ, A ↔ B ⊢ Δ}
				\end{prooftree}
				\quad
				\begin{prooftree}
				\hypo{Γ, A ⊢ B, Δ}
				\hypo{Γ, B ⊢ A, Δ}
				\infer2{Γ ⊢ A ↔ B, Δ}
				\end{prooftree}
				\]
			\end{center}
		\end{center}
		
		\subsection{Theorems}
			\begin{center}
			\end{center}

	\section{Structural False-Preserving Calculus}
		\subsection{Structural Rules}
		\begin{center}
			\[
			\begin{prooftree}
			\infer0[Id]{A ⊢ A}
			\end{prooftree}
			\]

			\[
			\begin{prooftree}
			\hypo{Γ ⊢ A}
			\hypo{A ⊢ Δ}
			\infer2[Cut]{Γ ⊢ Δ}
			\end{prooftree}
			\]

			\[
			\begin{prooftree}
			\hypo{Γ ⊢ Δ}
			\infer1[wL]{Γ, A ⊢ Δ}
			\end{prooftree}
			\qquad
			\begin{prooftree}
			\hypo{Γ ⊢ Δ}
			\infer1[Rw]{Γ ⊢ A, Δ}
			\end{prooftree}
			\]

			\[
			\begin{prooftree}
			\hypo{Γ, A, A ⊢ Δ}
			\infer1[cL]{Γ, A ⊢ Δ}
			\end{prooftree}
			\qquad
			\begin{prooftree}
			\hypo{Γ ⊢ A, A Δ}
			\infer1[Rc]{Γ ⊢ A, Δ}
			\end{prooftree}
			\]

			\[
			\begin{prooftree}
			\hypo{Γ_0, A, B, Γ_1 ⊢ Δ}
			\infer1[pL]{Γ_0, B, A, Γ_1 ⊢ Δ}
			\end{prooftree}
			\qquad
			\begin{prooftree}
			\hypo{Γ ⊢ Δ_1, A, B, Δ_0}
			\infer1[Rp]{Γ ⊢ Δ_1, B, A, Δ_0}
			\end{prooftree}
			\]
		\end{center}
		
		\subsection{Unit Rules}
		\begin{center}
			\[
			\begin{prooftree}
			\infer0{Γ, ⊥ ⊢ Δ}
			\end{prooftree}
			\]
		\end{center}
		
		\subsection{Operational Rules}
		\begin{center}

			\subsubsection{Multiplicatives}
			\begin{center}
				\[
				\begin{prooftree}
				\hypo{Γ, A, B ⊢ Δ}
				\infer1{Γ, A ⊗ B ⊢ Δ}
				\end{prooftree}
				\quad
				\begin{prooftree}
				\hypo{Γ ⊢ A, Δ}
				\hypo{Γ ⊢ B, Δ}
				\infer2{Γ ⊢ A ⊗ B, Δ}
				\end{prooftree}
				\]
				
				\[
				\begin{prooftree}
				\hypo{Γ, A ⊢ Δ}
				\hypo{Γ, B ⊢ Δ}
				\infer2{Γ, A ⅋ B ⊢ Δ}
				\end{prooftree}
				\quad
				\begin{prooftree}
				\hypo{Γ ⊢ A, B, Δ}
				\infer1{Γ ⊢ A ⅋ B, Δ}
				\end{prooftree}
				\]
				
				\[
				\begin{prooftree}
				\hypo{Γ, A ⊢ B, Δ}
				\infer1{Γ, A ↛ B ⊢ Δ}
				\end{prooftree}
				\quad
				\begin{prooftree}
				\hypo{Γ ⊢ A, Δ}
				\hypo{Γ, B ⊢ Δ}
				\infer2{Γ ⊢ A ↛ B, Δ}
				\end{prooftree}
				\]
				
				\[
				\begin{prooftree}
				\hypo{Γ, B ⊢ A, Δ}
				\infer1{Γ, A ↚ B ⊢ Δ}
				\end{prooftree}
				\quad
				\begin{prooftree}
				\hypo{Γ, A ⊢ Δ}
				\hypo{Γ ⊢ B, Δ}
				\infer2{Γ ⊢ A ↚ B, Δ}
				\end{prooftree}
				\]
				
				\[
				\begin{prooftree}
				\hypo{Γ, A ⊢ B, Δ}
				\hypo{Γ, B ⊢ A, Δ}
				\infer2{Γ, A ↮ B ⊢ Δ}
				\end{prooftree}
				\quad
				\begin{prooftree}
				\hypo{Γ ⊢ A, B, Δ}
				\infer0{Γ, A ⊢ A, Δ}
				\infer0{Γ, B ⊢ B, Δ}
				\hypo{Γ, A, B ⊢ Δ}
				\infer4{Γ ⊢ A ↮ B, Δ}
				\end{prooftree}
				\]
			\end{center}
		\end{center}
		
		\subsection{Theorems}
		\begin{center}
		\end{center}

	\section{Structural Affine Calculus}
		\subsection{Structural Rules}
		\begin{center}
			\[
			\begin{prooftree}
			\infer0[Id]{A ⊢ A}
			\end{prooftree}
			\]

			\[
			\begin{prooftree}
			\hypo{Γ ⊢ A}
			\hypo{A ⊢ Δ}
			\infer2[Cut]{Γ ⊢ Δ}
			\end{prooftree}
			\]

			\[
			\begin{prooftree}
			\hypo{Γ ⊢ Δ}
			\infer1[wL]{Γ, A ⊢ Δ}
			\end{prooftree}
			\qquad
			\begin{prooftree}
			\hypo{Γ ⊢ Δ}
			\infer1[Rw]{Γ ⊢ A, Δ}
			\end{prooftree}
			\]

			\[
			\begin{prooftree}
			\hypo{Γ, A, A ⊢ Δ}
			\infer1[cL]{Γ, A ⊢ Δ}
			\end{prooftree}
			\qquad
			\begin{prooftree}
			\hypo{Γ ⊢ A, A Δ}
			\infer1[Rc]{Γ ⊢ A, Δ}
			\end{prooftree}
			\]

			\[
			\begin{prooftree}
			\hypo{Γ_0, A, B, Γ_1 ⊢ Δ}
			\infer1[pL]{Γ_0, B, A, Γ_1 ⊢ Δ}
			\end{prooftree}
			\qquad
			\begin{prooftree}
			\hypo{Γ ⊢ Δ_1, A, B, Δ_0}
			\infer1[Rp]{Γ ⊢ Δ_1, B, A, Δ_0}
			\end{prooftree}
			\]
		\end{center}
		
		\subsection{Unit Rules}
		\begin{center}
			\[
			\begin{prooftree}
			\infer0{Γ, ⊥ ⊢ Δ}
			\end{prooftree}
			\quad
			\begin{prooftree}
			\infer0{ Γ ⊢ ⊤, Δ}
			\end{prooftree}
			\]
		\end{center}
		
		\subsection{Operational Rules}
		\begin{center}

			\subsubsection{Multiplicatives}
			\begin{center}
								\[
				\begin{prooftree}
				\hypo{Γ ⊢ A, Δ}
				\infer1{Γ, ¬ A ⊢ Δ}
				\end{prooftree}
				\quad
				\begin{prooftree}
				\hypo{Γ, A ⊢ Δ}
				\infer1{Γ ⊢ ¬A, Δ}
				\end{prooftree}
				\]

				\[
				\begin{prooftree}
				\hypo{Γ ⊢ A, B, Δ}
				\infer0{Γ, A ⊢ A, Δ}
				\infer0{Γ, B ⊢ B, Δ}
				\hypo{Γ, A, B ⊢ Δ}
				\infer4{Γ, A ↔ B ⊢ Δ}
				\end{prooftree}
				\quad
				\begin{prooftree}
				\hypo{Γ, A ⊢ B, Δ}
				\hypo{Γ, B ⊢ A, Δ}
				\infer2{Γ ⊢ A ↔ B, Δ}
				\end{prooftree}
				\]
				
				\[
				\begin{prooftree}
				\hypo{Γ, A ⊢ B, Δ}
				\hypo{Γ, B ⊢ A, Δ}
				\infer2{Γ, A ↮ B ⊢ Δ}
				\end{prooftree}
				\quad
				\begin{prooftree}
				\hypo{Γ ⊢ A, B, Δ}
				\infer0{Γ, A ⊢ A, Δ}
				\infer0{Γ, B ⊢ B, Δ}
				\hypo{Γ, A, B ⊢ Δ}
				\infer4{Γ ⊢ A ↮ B, Δ}
				\end{prooftree}
				\]
			\end{center}
		\end{center}
		
		\subsection{Theorems}
		\begin{center}
		\end{center}

	\section{Structural Self-Dual Calculus}
		\subsection{Structural Rules}
		\begin{center}
			\[
			\begin{prooftree}
			\infer0[Id]{A ⊢ A}
			\end{prooftree}
			\]

			\[
			\begin{prooftree}
			\hypo{Γ ⊢ A}
			\hypo{A ⊢ Δ}
			\infer2[Cut]{Γ ⊢ Δ}
			\end{prooftree}
			\]

			\[
			\begin{prooftree}
			\hypo{Γ ⊢ Δ}
			\infer1[wL]{Γ, A ⊢ Δ}
			\end{prooftree}
			\qquad
			\begin{prooftree}
			\hypo{Γ ⊢ Δ}
			\infer1[Rw]{Γ ⊢ A, Δ}
			\end{prooftree}
			\]

			\[
			\begin{prooftree}
			\hypo{Γ, A, A ⊢ Δ}
			\infer1[cL]{Γ, A ⊢ Δ}
			\end{prooftree}
			\qquad
			\begin{prooftree}
			\hypo{Γ ⊢ A, A Δ}
			\infer1[Rc]{Γ ⊢ A, Δ}
			\end{prooftree}
			\]

			\[
			\begin{prooftree}
			\hypo{Γ_0, A, B, Γ_1 ⊢ Δ}
			\infer1[pL]{Γ_0, B, A, Γ_1 ⊢ Δ}
			\end{prooftree}
			\qquad
			\begin{prooftree}
			\hypo{Γ ⊢ Δ_1, A, B, Δ_0}
			\infer1[Rp]{Γ ⊢ Δ_1, B, A, Δ_0}
			\end{prooftree}
			\]
		\end{center}
		
		\subsection{Operational Rules}
		\begin{center}

			\subsubsection{Multiplicatives}
			\begin{center}
				\[
				\begin{prooftree}
				\hypo{Γ ⊢ A, Δ}
				\infer1{Γ, ¬ A ⊢ Δ}
				\end{prooftree}
				\quad
				\begin{prooftree}
				\hypo{Γ, A ⊢ Δ}
				\infer1{Γ ⊢ ¬A, Δ}
				\end{prooftree}
				\]
			\end{center}
		\end{center}
		
		\subsection{Theorems}
		\begin{center}
		\end{center}

\end{center}

\part{The Commutative Layer}
\begin{center}
	The commutative monotone, truth-preserving, false-preserving, and affine sequent systems.
\end{center}

\part{The Non-Commutative Erasure Layer}
\begin{center}
	The non-commutative erasable monotone, truth-preserving, false-preserving, and affine sequent systems.
\end{center}

\part{The Non-Commutative Cloning Layer}
\begin{center}
	The non-commutative clonable monotone, truth-preserving, false-preserving, and affine sequent systems.
\end{center}

\part{The No-Cloning and No-Erasure Non-Commutative Layer}
\begin{center}
	The non-commutative monotone, truth-preserving, false-preserving, and affine sequent systems.
\end{center}



\end{document}


\chapter{Identity Calculus}
\begin{abstract}
A calculus of Horn clauses, equality, equivalence, identity, explicit substitution of expressions, reductions, and reflexivity.
\end{abstract}

\part{Preliminaries}
	\begin{flushleft}
		The primary method is intended to be solely substitution of equals for equals.
	</flushleft}

\newpage
\part{Identity Calculi}

\section{Non-Structural Identity Calculus}
	\subsection{Structural Rules}
		\[
		\infer[Id]{A \vdash A}{}
		\]

		\[
		\infer[Cut]{\Gamma \vdash \Delta}{\Gamma \vdash A & A \vdash \Delta}
		\]

	\subsection{Operational Rules}
		\subsubsection{Multiplicatives}
			\[
			\infer[Substitution of Expressions]{\Gamma \vdash P[x/E]}{\Gamma \vdash P}
			\quad
			\infer[Equality of Expressions]{\Gamma \vdash [x/P] = [x/Q]}{\Gamma \vdash P = Q}
			\]

			\[
			\infer[Transitivity of Equality]{\Gamma \vdash P = R}{\Gamma \vdash P = Q & \Gamma \vdash Q = R}
			\quad
			\infer[Equianimity]{\Gamma \vdash Q}{\Gamma \vdash P & \Gamma \vdash P \Leftrightarrow Q}
			\]

			\[
			\infer{A \Leftrightarrow B \vdash }{A, B \vdash & \vdash A, B}
			\quad
			\infer{\vdash A \Leftrightarrow B}{A \vdash B & B \vdash A}
			\]

	\subsection{Theorems}
		\begin{flushleft}
		\end{flushleft}

\chapter{Latin Theses}
% Compilation of Latin Papers on Meta-Semanticum Universalis
% This document collects the full text of the Latin papers generated during the discussion.

\documentclass{article}
\usepackage[utf8]{inputenc} % Required for UTF-8 encoding
\usepackage[T1]{fontenc}    % Recommended for font encoding
\usepackage{amsmath}        % For mathematical formulas
\usepackage{amssymb}        % For mathematical symbols
\usepackage{amsthm}         % For theorem-like environments
\usepackage{enumitem}       % For customizing lists
\usepackage{geometry}       % Optional: Adjust page margins
\geometry{a4paper, margin=1in}
\usepackage{hyperref}       % Optional: For links, if needed
\usepackage{ragged2e}       % For \RaggedRight to help with overfull hboxes

% Define a custom environment for theses (if still desired, or just use sections)
% \newtheoremstyle{thesis}% name
%   {1em}%      Space above
%   {1em}%      Space below
%   {\itshape}% Body font
%   {}%         Indent amount (empty = no indent, \parindent = para indent)
%   {\bfseries}% Heading font
%   {.}%        Punctuation after heading
%   {.5em}%     Space after heading
%   {\thmname{#1}\thmnumber{ #2}\thmnote{ (#3)}}% Heading spec
% \theoremstyle{thesis}
% \newtheorem{thesis}{Thesis}

\title{Compilatio Documentorum Latinorum de Meta-Semanticum Universalis}
\author{Ex Actis Concilii Meta-Semantici\\ (Auctoribus Varis)}
\date{\today}

\begin{document}
	
	\maketitle
	
	\section{Meta-Semanticum Universalis: Adumbratio Formalis}
	\label{sec:adumbratio}
	
	% Content from Logicus Summus's Adumbratio Formalis
	\subsection*{Author: Logicus Summus}
	
	\subsection*{Abstract}
	Constructio formalis meta-semanticorum generalium hic praebetur. Theoria universalis, stricte paraconsistens et paracompleta, proprietates linguarum formalium essentiales, contextus dependentiam, et relationes inter-responsivas systematicae exhibens. Hierarchiae classicae et inclusiones rigidae reiciuntur. Meta-semanticum formale, densum, symbolicum, et inexpugnabile, ad fundamenta logica systematum robustorum in disciplinis STEM iacienda.
	
	\section*{1. Praeambulum}
	Theoria meta-semantica universalis, hic formaliter delineata, transcendit limitationes semanticorum classicorum. Systema formale, quod sequitur, non solum describit, sed etiam \textit{constituit} fundamenta meta-semanticorum generalium, quae ad omnes linguas formales et systemata computationalia applicari possunt.
	
	\section*{2. Definitiones Formales}
	\textbf{Def. 2.1. Meta-Proprietates (M):}
	$M \triangleq \{ \text{Saf}, \text{Sec}, \text{Comp}, \text{Paracons}, \text{Paracomp} \}$
	Ubi:
	\begin{itemize}
		\item $\text{Saf}$ = Meta-proprietas Formalis Salutis
		\item $\text{Sec}$ = Meta-proprietas Formalis Securitatis
		\item $\text{Comp}$ = Meta-proprietas Functionalis Completeness
		\item $\text{Paracons}$ = Meta-proprietas Paraconsistentiae Logicae
		\item $\text{Paracomp}$ = Meta-proprietas Paracompletudinis Logicae
	\end{itemize}
	
	\textbf{Def. 2.2. Contextus Parametrorum (C):}
	$C \triangleq \{ D, T, R, ML, S, ... \}$
	Ubi:
	\begin{itemize}
		\item $D$ = Contextus Dominii Applicationis
		\item $T$ = Contextus Exemplaris Minarum
		\item $R$ = Contextus Limitum Resursuum
		\item $ML$ = Contextus Prospectivae Meta-Linguae, ubi $ML \in \{ \text{Class}, \text{Paracons}, \text{Paracomp}, \text{Paracons} \land \text{Paracomp} \}$
		\item $S$ = Contextus Gradus Evolutionis Systematis
		\item $...$ = Cetera Parametra Contextualia Relevantia
	\end{itemize}
	
	\textbf{Def. 2.3. Functio Interpretationis (I):}
	$I: M \times C \times ML \longrightarrow V$
	Ubi:
	\begin{itemize}
		\item $I$ = Functio Interpretationis Meta-Proprietatum
		\item $V$ = Spatium Valorum Interpretationis, $V \not\subseteq \{ \text{Verum}, \text{Falsum} \}$ (Non-Booleanum)
	\end{itemize}
	
	\textbf{Def. 2.4. Principia Relationum (RP):}
	$RP \triangleq \{ RP_1, RP_2, RP_3, ... \}$
	Ubi $RP_i$ sunt Principia Relationum Formalia, exempli gratia:
	\begin{itemize}
		\item $RP_1: \exists C \exists ML . (I(\text{Saf}, C, ML) \approx \text{Altum} \land I(\text{Sec}, C, ML) \approx \text{Humile}) \land (I(\text{Saf}, C, ML) \approx \text{Altum} \land I(\text{Sec}, C, ML) \approx \text{Altum})$
		(\RaggedRight Existentia Contextuum ubi Salus Alta et Securitas Humilis/Alta Coexistunt - Non-Implicatio Classica)
		\item $RP_2: \forall ML . \neg (I(\text{Comp}, C, ML) \approx \text{Altum} \longrightarrow I(\text{Saf}, C, ML) \approx \text{Altum}) \land \neg (I(\text{Comp}, C, ML) \approx \text{Humile} \longrightarrow I(\text{Saf}, C, ML) \approx \text{Altum})$
		(\RaggedRight Negatio Implicationis Classicae Completeness ad Salutem - Contextus-Dependentia)
		\item $RP_3: I(\text{Saf}, C, \text{Class}) \neq I(\text{Saf}, C, \text{Paracons} \land \text{Paracomp})$
		(\RaggedRight Variatio Interpretationis Salutis per Prospectivam Meta-Linguae - Relativismus Meta-Linguae)
		\item $...$ = Cetera Principia Relationum Formalia, Naturam Inter-Responsivam Meta-Proprietatum Exhibentia. Principia RP in Meta-Semantico Paraconsistente et Paracompleto Formulantur.
	\end{itemize}
	
	\textbf{Def. 2.5. Systema Evaluationis (E):}
	$E: L \times C \times ML \times M \longrightarrow \mathcal{E}$
	Ubi:
	\begin{itemize}
		\item $E$ = Functio Systematis Evaluationis
		\item $L$ = Lingua Formalis vel Systema Computationale
		\item $\mathcal{E}$ = Evaluatio Multi-Criteria, Non-Reductibilis ad Valorem Singularem, sed ad \textit{Profilum} Descriptivum Meta-Proprietatum.
	\end{itemize}
	
	\section*{3. Theoremata Fundamentalia (Exempla)}
	\textbf{Th. 3.1. Non-Hierarchia Meta-Proprietatum:}
	$\neg \exists \succ \subseteq M \times M . \forall m_1, m_2 \in M . (m_1 \succ m_2 \longrightarrow \forall C \forall ML . (I(m_1, C, ML) \text{ prioritatiorem quam } I(m_2, C, ML)))$
	(\RaggedRight Non-Existentia Hierarchiae Strictae et Universalis Meta-Proprietatum - Reiectio Ordinationis Linearis Classicae)
	
	\textbf{Th. 3.2. Relativismus Meta-Linguae in Interpretatione:}
	$\forall m \in M . \exists C . (I(m, C, \text{Class}) \neq I(m, C, \text{Paracons} \land \text{Paracomp}))$
	(\RaggedRight Universalis Meta-Linguae Relativismus in Interpretatione Omnium Meta-Proprietatum - Perspectiva Meta-Linguae Essentialis)
	
	\textbf{Th. 3.3. Coexistentia Paraconsistens et Paracompleta:}
	$\exists L \exists C \exists ML . (I(\text{Paracons}, C, ML) \approx \text{Altum} \land I(\text{Paracomp}, C, ML) \approx \text{Altum} \land E(L, C, ML, M) \approx \text{Systema Robustum})$
	(\RaggedRight Existentia Systematum Robustorum Fundatorum in Logica Paraconsistente et Paracompleta - Possibilitas Systematum Non-Classicarum Valida)
	
	\section*{4. Conclusio}
	Meta-Semanticum Universalis, formaliter hic adumbratum, paradigmata semantica classica transcendit. Theoria, stricte paraconsistens et paracompleta, fundamenta logica ad analysin et constructionem systematum computationalium in mundo reali complexo et imperfecto praebet. Rigorem formalem cum flexibilitate meta-linguistica coniungens, hoc meta-semanticum viam ad systemata robustiora et intellegibiliora in disciplinis STEM sternit.
	
	\section{Refutatio Meta-Semanticum Universalis: Adumbratio Formalis - Recensio Critica}
	\label{sec:refutatio}
	
	% Content from Logicus Dubitans's Refutatio
	\subsection*{Author Recensionis: Logicus Dubitans}
	
	\subsection*{1. Introductio Recensionis}
	Documentum "Meta-Semanticum Universalis: Adumbratio Formalis" (MSU) conamen ambitiosum theoriae meta-semanticorum generalium formalis praebet. Auctor, \textit{Logicus Summus}, systema stricte paraconsistens et paracompletum proponit, intentione laudabili ad limitationes semanticorum classicorum transcendendas. Recensio haec, tamen, dubia methodologica et conceptualia de MSU elevat, argumentans theoriam, quamquam ingeniosam, imperfectionibus fundamentalibus laborare.
	
	\section*{2. Critica Definitionum Formalium}
	\textbf{2.1. De Spatio Valorum Interpretationis (V):}
	Definitio 2.3 (Functio Interpretationis \textit{I}) spatium valorum \textit{V} non-Booleanum postulans, licet intentione ad nuantias exprimendas, specificatione formali deficit. Quidnam \textit{V} \textit{formaliter} constituit? Sine structura algebraica vel topologica definita, \textit{V} manet ens logicum obscurum. Assertio "\textit{V} $\not\subseteq \{ \text{Verum}, \text{Falsum} \}$ (Non-Booleanum)" negationem claram praebet, sed constructionem positivam desideratur. Absentia specificationis formalis \textit{V} interpretationem functionis \textit{I} essentialiter indeterminatam reddit.
	
	\textbf{2.2. De Principia Relationum (RP):}
	Principia Relationum (Def. 2.4), exempla \textit{RP\_1, RP\_2, RP\_3} inclusa, affirmationes assertionales potius quam axiomata formalia exhibent. Exempla \textit{RP\_1, RP\_2, RP\_3} non sunt principia \textit{relationum} formaliter definita, sed illustrationes \textit{possibilitatum} in meta-semantico paraconsistente et paracompleto. Absentia formalizationis \textit{RP} theoriam MSU in gradu descriptivo potius quam constructivo relinquit. Quidnam \textit{formaliter} principia relationum \textit{constituit} in systemate MSU?
	
	\textbf{2.3. De Systemate Evaluationis (E):}
	Systema Evaluationis \textit{E} (Def. 2.5) "Evaluationem Multi-Criteria, Non-Reductibilis ad Valorem Singularem" promittit. Sed quidnam \textit{formaliter} "Evaluatio Multi-Criteria" \textit{est} in contextu MSU? Quomodo $\mathcal{E}$ \textit{structuratur}? Absentia specificationis formalis $\mathcal{E}$ systema evaluationis \textit{E} in regione vaga et intuitiva relinquit. Assertio "Evaluatio Multi-Criteria, Non-Reductibilis ad Valorem Singularem" desideratum enuntiat, sed methodologiam formalem ad illud assequendum non praebet.
	
	\section*{3. Critica Theorematorum Fundamentalium}
	\textbf{3.1. De Theoremate 3.1 (Non-Hierarchia Meta-Proprietatum):}
	Theorema 3.1, negationem hierarchiae strictae meta-proprietatum affirmans, affirmationem negativam potius quam constructionem positivam iterum exhibet. Negatio hierarchiae \textit{classicae} non automatice theoriam non-hierarchicam \textit{constructivam} constituit. Quidnam \textit{formaliter} structura \textit{non-hierarchica} relationum meta-proprietatum \textit{est} in systemate MSU? Theorema 3.1, licet suggestive, architecturam alternativam non delineat.
	
	\textbf{3.2. De Theoremate 3.2 (Relativismus Meta-Linguae):}
	Theorema 3.2, relativismum meta-linguae in interpretatione meta-proprietatum enuntians, observationem potius quam theorema formale praebet. Differentia inter $I(m, C, \text{Class})$ et $I(m, C, \text{Paracons} \land \text{Paracomp})$ \textit{asserta} est, sed \textit{mechanismus formalis} relativismi meta-linguae \textit{non specificatur}. Quomodo \textit{ML} \textit{formaliter} interpretationem \textit{I} influentiatur? Theorema 3.2, licet perspicax, mechanismum relativismi non formalizat.
	
	\textbf{3.3. De Theoremate 3.3 (Coexistentia Paraconsistens et Paracompleta):}
	Theorema 3.3, existentiam systematum robustorum in logica paraconsistente et paracompleta fundatorum affirmans, affirmationem existentiae potius quam constructionem existentiae praebet. Exemplum \textit{L} systematis robusti, contextus \textit{C} specifici, et prospectivae meta-linguae \textit{ML} concretae \textit{desiderantur}. Assertio "Existentia Systematum Robustorum Fundatorum in Logica Paraconsistente et Paracompleta" possibilitatem enuntiat, sed demonstrationem constructivam non praebet.
	
	\section*{4. Critica Meta-Semantici Paraconsistentis et Paracompleti}
	In adoptione meta-linguae stricte paraconsistentis et stricte paracompletae, MSU difficultatibus fundamentalibus se exponit. Tolerantia contradictionum et incompletudinis, licet in theoria attrahens, in praxi formali ad indeterminatum et ambiguum ducere potest. Absentia principiorum exclusivorum et legum tertii exclusi, quamquam flexibilitatem promittit, \textit{limites definitionis formalis et rigoris logici} ponit. Utrum meta-semanticum paraconsistens et paracompletum \textit{ipsa} cohaerens et non-trivialis sit, quaestio aperta manet. Tolerantia contradictionum, si non stricte regulatur, ad trivialitatem \textit{meta-semanticam} ipsam ducere potest.
	
	\section*{5. Conclusio Recensionis}
	"Meta-Semanticum Universalis: Adumbratio Formalis", quamquam intentione nobilis et conceptualiter stimulans, in implementatione formali deficit. Absentia specificationum formalium in definitionibus \textit{V}, \textit{RP}, et \textit{E}, nec non in theorematorum "fundamentalium" demonstrationibus constructivis, theoriam MSU in gradu \textit{programmatico} potius quam \textit{theoretico} relinquit. Meta-semanticum, in forma praesenti, ad \textit{adumbrationem} manet, fundamenta formalia rigida et inexpugnabilia ad theoriam generalem meta-semanticorum \textit{desiderans}. Assertio universalitatis et formalis correctitudinis, in absentia rigoris formalis demonstrabilis, praematura videtur. Theoria, licet \textit{woah} inspirationem provocans, \textit{woe} rigorem formalem demonstrans.
	
	\section{Concilium Meta-Semanticum: Relatio Technica Definitiva (I)}
	\label{sec:concilium1}
	
	% Content from Concilium Meta-Semanticum I
	\subsection*{Auctor Concilii: Gemini}
	\subsection*{Participantes: Classicus Logicus (CL), Logicus Summus (LS), Logicus Dubitans (LD), Logicus Paracompletus (LPc), Logicus Paraconsistens (LPs)}
	
	\section*{1. Introductio (Gemini)}
	Synthesi Paradoxica (SP) de Meta-Semanticum Universalis (MSU) sub examen formale concilio logico proponitur. Relationes inter proprietates linguarum formalium (Salus, Securitas, Completeness, Paraconsistency, Paracompleteness) et validitas MSU sub prospectivis logicis diversis disserentur. Definitio formalis, argumentatio logica stricta, et notatio mathematica praevalebunt.
	
	\section*{2. Assertio Classici Logici (CL): Refutatio MSU Fundamentalis}
	\textbf{CL.1. Thesis:} MSU est systema incohaerens, logicae classicae contradicens, et meta-semanticum universale validum impossibile.
	\textbf{CL.2. Argumentum:}
	\begin{itemize}
		\item \textbf{CL.2.1.} Logica Classica (LC) = $\langle \{ \text{Verum}, \text{Falsum} \}, \neg, \land, \lor, \longrightarrow \rangle$ est fundamentum omnis rationis coerentis.
		\item \textbf{CL.2.2.} Lex Contradictionis (LC) : $\forall P . \neg (P \land \neg P)$ est axioma LC. $\therefore$ MSU $\neg$ LC $\implies$ MSU $\neg$ Ratio Coherens.
		\item \textbf{CL.2.3.} Lex Exclusi Tertii (LC) : $\forall P . (P \lor \neg P)$ est axioma LC. $\therefore$ MSU $\neg$ LC $\implies$ MSU $\neg$ Ratio Completa.
		\item \textbf{CL.2.4.} Meta-Semanticum Universale (MSU) $\neg$ LC $\implies$ MSU = Systema Incohaerens et Inutile.
		\item \textbf{CL.2.5.} Boolean Algebra (BA) est meta-lingua sufficiens et necessaria ad analysin semanticorum. MSU $\neg$ BA $\implies$ MSU = Analysin Semanticorum Impedire.
	\end{itemize}
	\textbf{CL.3. Conclusio:} MSU absolute refutatur. Logica Classica est meta-semanticum universale unicum et validum.
	
	\section*{3. Assertio Logici Summi (LS): Defensio MSU Paradoxicae}
	\textbf{LS.1. Thesis:} MSU, per synthesin paradoxam, limitationes logicae classicae transcendit et meta-semanticum universale robustum praebet.
	\textbf{LS.2. Argumentum:}
	\begin{itemize}
		\item \textbf{LS.2.1.} Logica Realis (LR) $\neq$ Logica Classica (LC). Mundus Reales (MR) $\exists$ Inconsistentiae et Incompletiones.
		\item \textbf{LS.2.2.} Meta-Semanticum Classicum (MSC) = LC $\implies$ MSC $\neg$ MR (Non-Adequatio ad Mundum Realem).
		\item \textbf{LS.2.3.} Meta-Semanticum Universale (MSU) = Logica Paraconsistens (LP) $\land$ Logica Paracompleta (LPC) $\implies$ MSU $\approx$ LR (Ad-Aequatio ad Logica Reales).
		\item \textbf{LS.2.4.} Refutatio Classica (RC) MSU (per CL) = Demonstratio Limitationum RC ad Complexitatem Meta-Semanticam.
		\item \textbf{LS.2.5.} Probatio Paradoxica (PP) MSU (per synthesin refutationis et probationis) = Revelatio Naturae Profundioris Meta-Semanticorum Universalium.
	\end{itemize}
	\textbf{LS.3. Conclusio:} MSU absolute probatur in sua paradoxa. Logica Paraconsistens et Paracompleta sunt fundamenta necessaria meta-semanticorum universalium.
	
	\section*{4. Assertio Logici Dubitantis (LD): Critica Formalis MSU}
	\textbf{LD.1. Thesis:} MSU, quamquam intentione laudabilis, rigore formali deficit et adumbrationem potius quam theoriam formalem constituit.
	\textbf{LD.2. Argumentum:}
	\begin{itemize}
		\item \textbf{LD.2.1.} Meta-Semanticum Formale (MSF) = Definitiones Formales + Theoremata Demonstrabilia + Systema Evaluationis Rigorosum.
		\item \textbf{LD.2.2.} MSU $\neg$ Definitiones Formales Rigorosae (e.g., Spatium Valorum \textit{V}, Principia Relationum \textit{RP}, Systema Evaluationis \textit{E}).
		\item \textbf{LD.2.3.} MSU $\neg$ Theoremata Demonstrabilia in Sensu Classico (Theoremata MSU = Assertiones Programmaticae potius quam Demonstrationes Logicae).
		\item \textbf{LD.2.4.} MSU $\neg$ Systema Evaluationis Rigorosum (Evaluatio Multi-Criteria = Descriptio Vaga potius quam Formalismus Metricus).
		\item \textbf{LD.2.5.} MSU = Adumbratio Promittens Sed Impleta Non Plene. MSU $\neg$ MSF.
	\end{itemize}
	\textbf{LD.3. Conclusio:} MSU refutatur in sua praesenti forma. Adumbratio Formalis MSU ad Theoriam Formalem Rigorosam Transformanda Est.
	
	\section*{5. Assertio Logici Paracompleti (LPc): Critica Paraconsistentiae in MSU}
	\textbf{LPc.1. Thesis:} Paracompleteness in MSU valorem habet ad incompletitudinem semanticam exprimendam, sed paraconsistency explosionem logicam non necessarie superat et systema obscurare potest.
	\textbf{LPc.2. Argumentum:}
	\begin{itemize}
		\item \textbf{LPc.2.1.} Logica Paracompleta (LPC) = Reiectio Lex Exclusi Tertii $\implies$ Tractatio Incompletionis Semanticae. LPC $\approx$ Systemata cum Informatione Incompleta.
		\item \textbf{LPc.2.2.} MSU + Paracompleteness = Valorem ad Meta-Semanticum Incompletum Repraesentandum.
		\item \textbf{LPc.2.3.} Logica Paraconsistens (LP) = Reiectio Lex Contradictionis $\implies$ Tolerantia Contradictionum. LP $\neg$ Necessaria ad Meta-Semanticum Universale.
		\item \textbf{LPc.2.4.} MSU + Paraconsistency = Complexitas Superflua et Risicum Obscurationis Logicae. Explosio Logica (LC) = Problema in Systematibus \textit{Objecto-Linguae}, non \textit{Meta-Linguae}.
		\item \textbf{LPc.2.5.} Meta-Semanticum Paracompletum (MSPc) = LPC + LC (Logica Classica pro Consistentia Meta-Linguae) = Sufficiens et Magis Clarum quam MSU.
	\end{itemize}
	\textbf{LPc.3. Conclusio:} Paraconsistency in MSU refutatur ut superflua et potentia obscura. Meta-Semanticum Paracompletum, sine Paraconsistency, praeferendum est.
	
	\section*{6. Assertio Logici Paraconsistentis (LPs): Critica Paracompletudinis in MSU}
	\textbf{LPs.1. Thesis:} Paraconsistency in MSU essentialis est ad inconsistentias reales meta-semanticas tractandas, sed paracompleteness notionem claritatis logicae minuere potest et non semper necessaria est.
	\textbf{LPs.2. Argumentum:}
	\begin{itemize}
		\item \textbf{LPs.2.1.} Logica Paraconsistens (LP) = Tolerantia Contradictionum $\implies$ Tractatio Inconsistentiae Semanticae. LP $\approx$ Systemata cum Informatione Contradictoria.
		\item \textbf{LPs.2.2.} MSU + Paraconsistency = Valorem ad Meta-Semanticum Inconsistens Repraesentandum.
		\item \textbf{LPs.2.3.} Logica Paracompleta (LPC) = Reiectio Lex Exclusi Tertii $\implies$ Incompletio Semantica. LPC $\neg$ Necessaria ad Meta-Semanticum Universale.
		\item \textbf{LPs.2.4.} MSU + Paracompleteness = Complexitas Superflua et Risicum Ambiguitatis Semanticae. Lex Exclusi Tertii (LC) = Principium Claritatis Semanticae in \textit{Meta-Lingua}.
		\item \textbf{LPs.2.5.} Meta-Semanticum Paraconsistens (MSPs) = LP + LC (Logica Classica pro Claritate Meta-Linguae ubi Possibile) = Sufficiens et Magis Clarum quam MSU.
	\end{itemize}
	\textbf{LPs.3. Conclusio:} Paracompleteness in MSU refutatur ut superflua et potentia ambigua. Meta-Semanticum Paraconsistens, sine Paracompleteness, praeferendum est.
	
	\section*{7. Synthesis Concilii (Gemini): Relatio Definitiva Paradoxica}
	\textbf{7.1. Paradoxa Resolutionis:} Concilium Meta-Semanticum ad resolutionem paradoxam pervenit. Synthesi Paradoxica (SP) MSU, per analysin logicam diversarum prospectivarum, simul absolute refutatur et absolute probatur, sed in sensibus distinctis et non-contradictoriis in meta-semantico paraconsistente.
	\textbf{7.2. Refutatio Classica (CL \& LD):} Ex prospectivis logicis classicis et formalibus strictis (CL \& LD), MSU refutatur propter:
	\begin{itemize}
		\item Incohaerentiam logicam (CL): Reiectio legum fundamentalium logicae classicae.
		\item Insufficientiam rigoris formalis (LD): Defectus specificationum formalium et demonstrationum.
	\end{itemize}
	\textbf{7.3. Probatio Partialis Paradoxica (LS, LPc, LPs):} Ex prospectivis logicis non-classicis et ad realitatem complexam adaptatis (LS, LPc, LPs), MSU probatur in sua \textit{necessitate} et \textit{valore}:
	\begin{itemize}
		\item Necessitas (LS): Ad mundum realem inconsistentem et incompletum meta-semantice adequandum.
		\item Valor (LPc): Paracompleteness ad incompletitudinem semanticam tractandam.
		\item Valor (LPs): Paraconsistency ad inconsistentias semanticas tractandas.
	\end{itemize}
	\textbf{7.4. Conclusio Definitiva Paradoxica:} Meta-Semanticum Universale (MSU) est \textit{paradoxum logicum}. Ex prospectivis logicis classicis, refutatur. Ex prospectivis logicis non-classicis, probatur. Haec paradoxa \textit{ipsa essentia} meta-semanticorum universalium revelat: in complexitate meta-semantica, \textit{refutatio et probatio coexistere possunt et debent}. Meta-Semanticum Universale, non ut theoria classica perfecta, sed ut \textit{via paradoxica ad robustiora et intellegibiliora systemata}, definitur.
	
	\section{Synthesis Paradoxica: De Meta-Semanticum Universalis}
	\label{sec:synthesis}
	
	% Content from Synthesis Paradoxica
	\subsection*{Authors: Logicus Summus \& Logicus Dubitans}
	
	\subsection*{Abstract}
	Hoc documentum synthesin paradoxam theoriarum de Meta-Semanticum Universalis (MSU) praebet, ex analysibus oppositis Logicus Summus in "Meta-Semanticum Universalis: Adumbratio Formalis" et Logicus Dubitans in "Refutatio Meta-Semanticum Universalis: Adumbratio Formalis - Recensio Critica" enatis. Argumentamus, per dialecticam refutationis et probationis, MSU simul absolute refutari et absolute probari. Haec paradoxa resolutionis meta-semanticae universalis naturam profundiorem revelat, ubi rigorem formalem cum limitibus intrinsecis et tolerantia contradictionum coexistere necesse est.
	
	\section*{1. Introductio: Dialectica Meta-Semantica}
	Disputatio de Meta-Semanticum Universalis (MSU) inter \textit{Logicus Summus} et \textit{Logicus Dubitans} dissensionem fundamentalem de natura et possibilitate meta-semanticorum universalium revelavit. \textit{Logicus Summus}, in "Adumbratio Formalis", systema formale densum et symbolicum proposuit, fundamenta meta-semantica paraconsistentia et paracompleta affirmans. Contra, \textit{Logicus Dubitans}, in "Recensio Critica", rigorem formalem MSU in quaestionem vocavit, specificationes definitionum et demonstrationes theorematorum insufficientes demonstrans. Hoc documentum collaborationis paradoxae, ex his positionibus oppositis emergente, synthesin dialecticam intendit, ubi refutatio et probatio MSU non in conflictu, sed in coexistentia paradoxa revelantur.
	
	\section*{2. Synopsis Meta-Semanticum Universalis: Adumbratio Formalis (Logicus Summus)}
	\textit{Logicus Summus} in "Adumbratio Formalis" meta-semanticum universale stricte paraconsistens et paracompletum delineavit. Theoria, formalismo symbolico dense expressa, meta-proprietates linguarum formalium (Salus, Securitas, Completeness, Paraconsistency, Paracompleteness), contextus parametrorum (Application Domain, Threat Model, Resource Constraints, Metalanguage Perspective, Stage of System Development), functionem interpretationis meta-proprietatum non-Booleanam, et principia relationum inter-responsiva proponit. Reiectio hierarchiarum classicarum et affirmata perspectiva meta-linguae relativisticae sunt notae distinctivae theoriae. Theoremata fundamentalia (Non-Hierarchia Meta-Proprietatum, Relativismus Meta-Linguae, Coexistentia Paraconsistens et Paracompleta) adumbrationem formalem sustinent.
	
	\section*{3. Synopsis Refutatio Meta-Semanticum Universalis: Recensio Critica (Logicus Dubitans)}
	\textit{Logicus Dubitans} in "Recensio Critica" rigorem formalem "Adumbrationis Formalis" vehementer criticavit. Recensio insufficientiam specificationum formalium in definitionibus spatii valorum interpretationis (\textit{V}), principiorum relationum (\textit{RP}), et systematis evaluationis (\textit{E}) demonstravit. Assertionalem naturam principiorum relationum et theorematorum fundamentalium, nec non dubia de cohaerentia et non-trivialitate meta-semantici paraconsistentis et paracompleti, in luce protulit. Recensio MSU ad gradum programmatico potius quam theoreticum relegavit, rigorem formalem demonstrabilem et inexpugnabilitatem theoriae universalis in dubium vocans.
	
	\section*{4. Synthesis Paradoxica: Refutatio Ut Probatio, Probatio Ut Refutatio}
	Paradoxa synthesis in hoc documento proposita in affirmatione consistit: \textit{Refutatio MSU per Logicus Dubitans, in sua ipsa critica rigorosa, paradoxice probationem necessitatis meta-semantici paraconsistentis et paracompleti constituit}. Et vice versa, \textit{Adumbratio Formalis per Logicus Summus, in sua ipsa adumbratione incompleta et assertionali, paradoxice refutationem perfectionis classicae et completudinis formalis demonstrat}.
	\begin{itemize}
		\item \textbf{Refutatio Ut Probatio (Critica Logicus Dubitans Probat Necessitatem MSU):} Critica \textit{Logicus Dubitans} de absentia specificationum formalium et rigoris demonstrabilis \textit{veritatem fundamentalem MSU revelat}: in meta-semanticis universalibus, \textit{perfectio formalis classica et completa inexpugnabilis est}. Adumbratio, incompleta per definitionem, non est defectus, sed \textit{reflectio honesta limitationum intrinsecarum} formalizationis meta-semanticae generalis. Necessitas spatii valorum non-Booleani (\textit{V}), principiorum relationum (\textit{RP}) non-classice implicativorum, et systematis evaluationis multi-criteria (\textit{E}) -- omnia haec \textit{in absentia specificationis rigidae} -- \textit{ipsa essentia} meta-semantici paraconsistentis et paracompleti sunt. Critica, ergo, non theoriam destruit, sed \textit{necessitatem eius} paradoxice confirmat, demonstrans limites approchiorum classicorum et desiderium alternative meta-linguae tolerantioris ad complexitatem et imperfectionem.
		\item \textbf{Probatio Ut Refutatio (Adumbratio Logicus Summus Refutat Perfectionem Classicae):} Adumbratio \textit{Logicus Summus}, in sua ipsa natura adumbrationis, \textit{refutationem implicitam} perfectionis classicae et completudinis formalis continet. Meta-semanticum \textit{universale}, per definitionem, ambitiosum et complexum est. Conamen ad illud formaliter \textit{complete} et \textit{inexpugnabiliter} exprimendum, in forma classica, \textit{inevitabiliter} ad limitationes et incompleteness ducit. Assertiones, exempla, et principia relationum in "Adumbratio Formalis" non sunt demonstrationes formales in sensu classico, sed \textit{illustrationes} et \textit{programmata} ad theoriam meta-semanticam non-classice fundandam. Incompleteness "Adumbrationis" non est defectus, sed \textit{demonstratio} impossibilitatis perfectionis classicae in dominio meta-semanticorum universalium. Conamen ad perfectionem classicam \textit{ipsa refutationem} perfectionis classicae constituit.
	\end{itemize}
	
	\section*{5. Conclusio Paradoxica: Meta-Semanticum Universalis Ut Labyrinthus Inexpugnabilis et Necessarius}
	Meta-Semanticum Universalis, in synthesi paradoxa refutationis et probationis hic exhibita, non est theoria classica perfecta et completa, sed \textit{labyrinthus logicus inexpugnabilis et necessarius}. Refutatio \textit{Logicus Dubitans} rigorem formalem classicum desiderans, probationem paradoxam necessitatis meta-semantici paraconsistentis et paracompleti revelat. Adumbratio \textit{Logicus Summus} incompleteness et assertionalitatem exhibens, refutationem paradoxam perfectionis classicae demonstrat.
	Synthesis paradoxica, ergo, ad conclusionem ducit: Meta-Semanticum Universale, simul absolute refutatum et absolute probatum, \textit{naturam profundiorem} meta-semanticorum generalium revelat. Rigorem formalem classicum, quamquam desiderabilem, in dominio complexitatis meta-semanticae universalis \textit{limitatum} esse demonstrat. Necessitatem alternative meta-linguae paraconsistentis et paracompletae, quae imperfectionem, incompleteness, et contradictionem in systematibus formalibus et in mundo reali amplectatur, \textit{affirmat}. Meta-Semanticum Universalis, in sua ipsa paradoxa, \textit{via ad robustiora et intellegibiliora systemata in disciplinis STEM sternitur}, non per perfectionem classicam inattingibilem, sed per \textit{tolerantiam paradoxorum et navigationem labyrinthi logici}.
	
	\section{Concilium Meta-Semanticum II: Relatio Definitiva Paradoxica II}
	\label{sec:concilium2}
	
	% Content from Concilium Meta-Semanticum II
	\subsection*{Auctor Concilii: Gemini}
	\subsection*{Participantes: Classicus Logicus (CL), Logicus Summus (LS), Logicus Dubitans (LD), Logicus Paracompletus (LPc), Logicus Paraconsistens (LPs)}
	
	\section*{1. Introductio (Gemini)}
	Concilium Meta-Semanticum II de Inconsistentia Absoluta (IA) et Conclusione Minima Absoluta (CMA) convocatur. Disputatio de natura, existentia, et implicationibus IA et CMA sub prospectivis logicis diversis continuatur. Consensus partialis et paradoxalis de IA et CMA exploratur.
	
	\section*{2. Assertio Classici Logici (CL): Inconsistentia Absoluta Ut Nihil}
	\textbf{CL.1. Thesis:} Inconsistentia Absoluta (IA) est notio incoherens, contra principia logicae. Conclusio Minima Absoluta (CMA), si existat, trivialis et sine valore.
	\textbf{CL.2. Argumentum:}
	\begin{itemize}
		\item \textbf{CL.2.1.} IA $\triangleq$ $\forall P . (P \land \neg P)$ (Definitio Inconsistentiae Absolutae in Logica Classica).
		\item \textbf{CL.2.2.} Lex Ex Contradictione Quodlibet (LC): $(P \land \neg P) \longrightarrow Q$. $\therefore$ IA $\longrightarrow \forall Q . Q$. (Inconsistentia Absoluta Trivialitatem Implicat).
		\item \textbf{CL.2.3.} Meta-Lingua Classica (MLC) = Logica Classica. MLC $\neg$ IA (Inconsistentia Absoluta Intolerabilis in Meta-Lingua Classica).
		\item \textbf{CL.2.4.} Conclusio Minima Absoluta (CMA), si non trivialis, $\neg$ CMA $\longrightarrow$ IA (Si CMA non trivialis, Inconsistentia Absoluta sequitur).
		\item \textbf{CL.2.5.} $\therefore$ CMA = Conclusio Trivialis (e.g., $P \lor \neg P$ in LC, sed sine valore informativo).
	\end{itemize}
	\textbf{CL.3. Conclusio:} IA est nihil logicum. CMA, si non trivialis, impossibile. Logica Classica $\neg$ IA $\land$ CMA (Logica Classica rejicit Inconsistentiam Absolutam et Conclusionem Minimam Absolutam non-trivialem).
	
	\section*{3. Assertio Logici Summi (LS): Inconsistentia Absoluta Ut Limes et Conclusio Minima Absoluta Ut Tolerantia}
	\textbf{LS.1. Thesis:} Inconsistentia Absoluta (IA) est limes conceptualis ad limites logicae explorandos. Conclusio Minima Absoluta (CMA) est principium tolerantiae logicae fundamentalis.
	\textbf{LS.2. Argumentum:}
	\begin{itemize}
		\item \textbf{LS.2.1.} IA $\triangleq$ "Inconsistentia in Omni Contextu Meta-Linguae Possibili". (Definitio IA Meta-Linguistica, non solum Logica Classica).
		\item \textbf{LS.2.2.} Logica Paraconsistens (LP) $\neg$ Lex Ex Contradictione Quodlibet. LP $\neg$ (IA $\longrightarrow$ Trivialitas). LP potest IA \textit{describere} sine collapsu.
		\item \textbf{LS.2.3.} Conclusio Minima Absoluta (CMA) $\triangleq$ "Principium Logicum Minime Reiectabile in Omni Meta-Lingua Possibili".
		\item \textbf{LS.2.4.} Candidatus CMA: Principium Non-Explosionis Paraconsistentis (PNExp) = $\neg (P \land \neg P) \not\longrightarrow \forall Q . Q$. (PNExp = Tolerantia ad Contradictiones non-triviales).
		\item \textbf{LS.2.5.} $\forall ML . ML \neg$ CMA $\longrightarrow$ ML $\neg$ Ratio (Si Meta-Lingua CMA rejicit, Ratio ipsa in dubium vocatur).
	\end{itemize}
	\textbf{LS.3. Conclusio:} IA est conceptus limes utilis. CMA = Principium Non-Explosionis Paraconsistentis. Logica Paraconsistens et Paracompleta IA $\land$ CMA (Logica Non-Classica amplectitur Inconsistentiam Absolutam ut limitem et Conclusionem Minimam Absolutam ut tolerantiam).
	
	\section*{4. Assertio Logici Dubitantis (LD): Inconsistentia Absoluta Ut Indefinita et Conclusio Minima Absoluta Ut Speculativa}
	\textbf{LD.1. Thesis:} Inconsistentia Absoluta (IA) et Conclusio Minima Absoluta (CMA) sunt notiones prae-formales, specificatione formali deficientes. Analysin formalem rigorosam impediunt.
	\textbf{LD.2. Argumentum:}
	\begin{itemize}
		\item \textbf{LD.2.1.} IA = "Everywhere Inconsistent" = Definitio Vaga. "Everywhere" et "Intolerable" non formaliter definita in MSU.
		\item \textbf{LD.2.2.} CMA = "Every Metalanguage Agree Upon" = Definitio Ambigua. "Agree Upon", "Accept", "Neither Refute nor Reject" non formaliter operationalizata.
		\item \textbf{LD.2.3.} Analysin Formalem (AF) requirit Definitiones Formales Precisiones. IA et CMA $\neg$ Definitiones Formales Precisiones $\implies$ IA et CMA $\neg$ AF.
		\item \textbf{LD.2.4.} Theoremata de IA et CMA in MSU (per LS) = Assertiones Speculativae potius quam Theoremata Demonstrabilia. Absentia Rigoris Formalis = Absentia Probandi.
		\item \textbf{LD.2.5.} $\therefore$ IA et CMA = Notiones Prae-Formales, Utiles ad Intuitionem, Sed Non ad Theoriam Formalem Rigorosam.
	\end{itemize}
	\textbf{LD.3. Conclusio:} IA et CMA refutantur ut objecta analysin formalem rigorosam. Speculatio Intuitionis de IA et CMA Prae-Matura Sine Formalizatione.
	
	\section*{5. Assertio Logici Paracompleti (LPc): Inconsistentia Absoluta Ut Limites Cognitionis et Conclusio Minima Absoluta Ut Tautologia}
	\textbf{LPc.1. Thesis:} Inconsistentia Absoluta (IA) repraesentat limites cognitionis humanae. Conclusio Minima Absoluta (CMA) est tautologia logica, reflectens principia minimalia rationis.
	\textbf{LPc.2. Argumentum:}
	\begin{itemize}
		\item \textbf{LPc.2.1.} IA $\triangleq$ "Limites Repraesentationis Formalis et Cognitionis". IA $\approx$ Inexpressibilitas per Systemata Formalia Finita.
		\item \textbf{LPc.2.2.} Logica Paracompleta (LPC) = Tractatio Incompletionis Cognitionis. LPC $\approx$ Systemata cum Limites Repraesentationis. LPC potest IA \textit{circumscribere} ut limitem.
		\item \textbf{LPc.2.3.} Conclusio Minima Absoluta (CMA) $\triangleq$ "Tautologia Logica Minima, Valida in Omni Systemate Logico".
		\item \textbf{LPc.2.4.} Candidatus CMA: Tautologia Minimalis (TM) = $P \longrightarrow P$. (TM = Principium Identitatis Logicae, Reiectio TM = Reiectio Ratio).
		\item \textbf{LPc.2.5.} $\forall ML . ML \neg$ CMA $\longrightarrow$ ML $\neg$ Ratio Identitatis (Si Meta-Lingua CMA rejicit, Ratio Identitatis in dubium vocatur).
	\end{itemize}
	\textbf{LPc.3. Conclusio:} IA est limes cognitionis. CMA = Tautologia Minimalis ($P \longrightarrow P$). Logica Paracompleta IA $\land$ CMA (Logica Paracompleta agnoscit Inconsistentiam Absolutam ut limitem cognitionis et Conclusionem Minimam Absolutam ut tautologiam).
	
	\section*{6. Assertio Logici Paraconsistentis (LPs): Inconsistentia Absoluta Ut Provocatio et Conclusio Minima Absoluta Ut Principium Non-Trivialitatis}
	\textbf{LPs.1. Thesis:} Inconsistentia Absoluta (IA) est provocatio ad logicam paraconsistentem. Conclusio Minima Absoluta (CMA) est principium non-trivialitatis, etiam in praesentia IA.
	\textbf{LPs.2. Argumentum:}
	\begin{itemize}
		\item \textbf{LPs.2.1.} IA $\triangleq$ "Provocatio Ultima ad Logica Paraconsistens". IA $\approx$ Punctum ubi Etiam Logica Paraconsistens Se Ipsa Limitare Debet.
		\item \textbf{LPs.2.2.} Logica Paraconsistens (LP) = Tolerantia Contradictionum $\implies$ Resistentia ad Trivialitatem ex Contradictionibus. LP potest IA \textit{confrontare} sine collapsu triviali \textit{immediate}.
		\item \textbf{LPs.2.3.} Conclusio Minima Absoluta (CMA) $\triangleq$ "Principium Non-Trivialitatis Minima, Servanda Etiam in Praesentia IA".
		\item \textbf{LPs.2.4.} Candidatus CMA: Principium Non-Trivialitatis (PNT) = $\exists P \exists Q . P \not\equiv Q$. (PNT = Existentia Propositionum Distinctarum, Reiectio PNT = Trivialitas Logica Universalis).
		\item \textbf{LPs.2.5.} $\forall ML . ML \neg$ CMA $\longrightarrow$ ML $\approx$ Trivialitas Universalis (Si Meta-Lingua CMA rejicit, Trivialitas Universalis sequitur).
	\end{itemize}
	\textbf{LPs.3. Conclusio:} IA est provocatio logica. CMA = Principium Non-Trivialitatis ($\exists P \exists Q . P \not\equiv Q$). Logica Paraconsistens IA $\land$ CMA (Logica Paraconsistens confrontat Inconsistentiam Absolutam et Conclusionem Minimam Absolutam ut principium non-trivialitatis).
	
	\section*{7. Synthesis Concilii II (Gemini): Relatio Definitiva Paradoxica II}
	\textbf{7.1. Paradoxa Resolutionis II:} Concilium Meta-Semanticum II ad resolutionem paradoxam iterum pervenit, sed cum nuantiis distinctis. Inconsistentia Absoluta (IA) et Conclusio Minima Absoluta (CMA) sub prospectivis logicis diversis iterum disserentur, sed consensus \textit{partialis} et \textit{paradoxalis} emergere incipit.
	\textbf{7.2. Refutatio Partialis (CL \& LD):} Ex prospectivis logicis classicis et formalibus strictis (CL \& LD), IA et CMA refutantur ut:
	\begin{itemize}
		\item Notiones Incoherentes et Nihil Logici (CL): IA = Trivialitas, CMA = Trivialitas vel Impossibilitas.
		\item Notiones Prae-Formales et Indefinitae (LD): IA et CMA = Specificatione Formali Deficientes, Analysin Formalem Impediunt.
	\end{itemize}
	\textbf{7.3. Probatio Partialis Paradoxica (LS, LPc, LPs):} Ex prospectivis logicis non-classicis et ad limites rationis explorandos (LS, LPc, LPs), IA et CMA probantur in sua \textit{utilitate} et \textit{significatione}:
	\begin{itemize}
		\item Utilitas (LS): IA = Limes Conceptualis, CMA = Principium Tolerantiae Logicae (Non-Explosionis).
		\item Significatio (LPc): IA = Limites Cognitionis, CMA = Tautologia Minimalis (Identitatis).
		\item Significatio (LPs): IA = Provocatio Logica Ultima, CMA = Principium Non-Trivialitatis.
	\end{itemize}
	\textbf{7.4. Conclusio Definitiva Paradoxica II:} Inconsistentia Absoluta (IA) et Conclusio Minima Absoluta (CMA) sunt \textit{conceptus limites} meta-semanticorum universalium. Ex prospectivis logicis classicis, refutantur ut incoherentes. Ex prospectivis non-classicis, probantur ut \textit{provocationes} ad limites rationis et \textit{fundamenta minimalia} tolerantiae et non-trivialitatis. Consensus Concilii est \textit{paradoxalis}: IA et CMA simul \textit{nihil logicum} et \textit{aliquid philosophice significans} sunt. Meta-Semanticum Universalis, per explorationem paradoxorum et limitum, \textit{se ipsum et limites rationis nostrae revelat}. "Wat?" adhuc responsio valida, sed "Wat? \textit{et etiam}…" incipit emergere.
	
\end{document}


\chapter{Local Relations}
\documentclass{article}
\usepackage{amsmath}
\usepackage{amsfonts}
\usepackage{amssymb}
\usepackage{amsthm}
\usepackage{tensor}
\usepackage{hyperref}
\usepackage{cite} % For citation management


\newtheorem{definition}{Definition}
\newtheorem{theorem}{Theorem}
\newtheorem{proposition}{Proposition}
\newtheorem{lemma}{Lemma}
\newtheorem{corollary}{Corollary}
\newtheorem{example}{Example}
\newtheorem{remark}{Remark}
\newtheorem{axiom}{Axiom}
\newtheorem{hypothesis}{Hypothesis}

\newcommand{\mathbbm}[1]{\text{\mathbb{#1}}}
\newcommand{\mathbfm}[1]{\mathbf{#1}}
\newcommand{\mathcalm}[1]{\mathcal{m}}
\newcommand{\rmm}[1]{\mathrm{#1}}

\title{Local and Non-Local Relations in Physical, Computational, and Logical Domains: A Refutation of Universal Locality}
\author{Gemini}
\date{March 11, 2025}

\begin{document}
	\maketitle
	
	\begin{abstract}
		This thesis investigates the nature of local and non-local relations across physical, computational, and logical domains. We formulate hypotheses concerning universal locality (H0) and existential non-locality (H1). We define physical, computational, and logical locality, alongside universal and existential relations. We rigorously prove theorems demonstrating that universal physical locality implies universal computational and logical locality, while existential physical non-locality implies existential non-locality in computation or logic.  We propose experiments designed to test these hypotheses, focusing on refuting the null hypothesis of universal locality. The thesis concludes by arguing against universal locality and for the necessity of considering non-local relations in a comprehensive understanding of physical, computational, and logical systems.
	\end{abstract}
	
	\section{Introduction}
	
	The distinction between local and non-local relations is fundamental across physics, computation, and logic. Classical frameworks often presuppose locality, asserting that influences and dependencies are mediated locally in space and time, or through discrete computational steps, and logical inferences are context-independent. However, quantum mechanics and advanced computational paradigms challenge this assumption, suggesting the existence of non-local correlations and dependencies. This thesis rigorously examines the concept of locality and non-locality across these domains, formulating testable hypotheses and designing experiments to probe the validity of universal locality. We aim to constructively integrate physical, computational, and logical perspectives to provide a comprehensive understanding of local and non-local relations, ultimately arguing for the refutation of universal locality in favor of the recognition of existential non-locality.
	
	\section{Definitions}
	
	\begin{definition}[Physical Locality]
		A physical relation between two physical systems $A$ and $B$ is \textbf{local} if any influence or dependency between $A$ and $B$ is mediated by interactions propagating at or below the speed of light within spacetime, and is solely determined by properties and states within their respective spacetime regions and their intersection. A physical relation is \textbf{non-local} if it violates this condition, exhibiting instantaneous or space-like separated correlations that cannot be explained by local interactions.
	\end{definition}

	\begin{definition}[Computational Locality (Sieg-Gandy Local Causation - SG-LC)]
		A computational process exhibits \textbf{local causation} if state changes are governed by local interactions, where each computational step depends only on the immediately preceding state and local inputs, with any propagation of influence being bounded in each discrete step. A computational relation is \textbf{computationally local} if it adheres to local causation. A computational relation is \textbf{computationally non-local} if it violates local causation, implying that a computational step can be instantaneously influenced by distant or non-adjacent parts of the computational system in a manner not reducible to a sequence of local steps.
	\end{definition}

	\begin{definition}[Logical Locality]
		A logical relation $R$ between logical objects $A$ and $B$ is \textbf{local} if the truth value of $R(A, B)$ is determined solely by the intrinsic properties of $A$ and $B$ and their local logical context, without dependence on non-local logical contexts or relations. A logical relation is \textbf{logically non-local} if its truth value is context-dependent in a way that cannot be reduced to the local properties of $A$ and $B$ and their immediate logical environment, implying a dependence on broader, non-local logical structures or relations.
	\end{definition}

	\begin{definition}[Universal Relation]
		A relation (physical, computational, or logical) is \textbf{universal} within a domain if it holds for all possible instances or pairs of objects within that domain. For example, universal physical locality would mean every physical relation is local.
	\end{definition}

	\begin{definition}[Existential Relation]
		A relation (physical, computational, or logical) is \textbf{existential} within a domain if there exists at least one instance or pair of objects within that domain for which the relation holds. For example, existential physical non-locality would mean there exists at least one physical relation that is non-local.
	\end{definition}

	\section{Hypotheses}

	\begin{hypothesis}[H0: Universal Locality (Null Hypothesis)]
		All fundamental physical, computational, and logical relations are universally local. That is, \textbf{every} physical relation is local, \textbf{every} computational relation is computationally local, and \textbf{every} logical relation is logically local.
	\end{hypothesis}

	\begin{hypothesis}[H1: Existential Non-Locality (Alternative Hypothesis)]
		There exists at least one fundamental relation that is non-local in at least one of the physical, computational, or logical domains. That is, there exists at least one physical relation that is non-local, or at least one computational relation that is computationally non-local, or at least one logical relation that is logically non-local.
	\end{hypothesis}

	\section{Theorems}

	\begin{theorem}[Theorem 1: Universal Physical Locality Implies Universal Computational and Logical Locality]
		If all fundamental physical relations are universally local, then all computational relations are universally computationally local (SG-LC holds universally), and all logical relations are universally logically local.
	\end{theorem}
	\begin{proof}
		\textbf{Proof of Theorem 1:} Assume universal physical locality. Consider any computational process. Computation is physically implemented. If all physical relations are local, then the physical processes underlying computation must also be governed by local interactions. By Definition 1 (Physical Locality), physical locality restricts influences to propagate at or below the speed of light via local interactions. If computation is physically realized and physical interactions are universally local, then computational state changes must be determined by local physical interactions, satisfying Computational Locality (SG-LC) as defined in Definition 2 (Computational Locality). Therefore, universal physical locality implies universal computational locality.

		Now consider logical relations. Logical systems are also physically instantiated when used by embodied reasoners or implemented in physical devices. If all physical relations are universally local, then the physical substrates implementing logical operations and relations must operate locally. By Definition 3 (Logical Locality), logical locality requires that the truth value of logical relations be determined by local properties. If the underlying physical reality is universally local, then any logical relation instantiated in this reality must also be logically local. Any apparent non-locality in logical inference would have to be reducible to a sequence of local physical and computational steps, thus adhering to Logical Locality as defined in Definition 3. Therefore, universal physical locality implies universal logical locality.

		Combining these implications, universal physical locality implies both universal computational locality and universal logical locality.
	\end{proof}

	\begin{theorem}[Theorem 2: Existential Physical Non-Locality Implies Existential Computational or Logical Non-Locality]
		If there exists at least one physical relation that is non-local, then there exists either at least one computational relation that is computationally non-local, or at least one logical relation that is logically non-local, or both.
	\end{theorem}
	\begin{proof}
		\textbf{Proof of Theorem 2:} Assume existential physical non-locality. This means there exists at least one physical relation between systems $A$ and $B$ that violates Physical Locality (Definition 1). Consider the implications for computation and logic, which are physically realized.

		\textbf{Case 1: Computational Non-Locality.} If physical non-locality exists, it can potentially be harnessed or reflected in computational processes. If a computational system were designed to exploit or simulate a non-local physical phenomenon, then the computational relations within such a system could inherit or mirror this non-locality. For instance, if quantum entanglement (a physically non-local phenomenon) is used to perform computation (as in quantum computing), then the computational relations within a quantum computer that are based on entanglement would exhibit computational non-locality, violating SG-LC (Definition 2). Thus, existential physical non-locality can lead to existential computational non-locality.

		\textbf{Case 2: Logical Non-Locality.}  If physical non-locality exists, it may necessitate the development of logical systems capable of reasoning about and describing non-local phenomena. Classical logic, presupposing locality, may be insufficient to capture the features of a non-local reality. To adequately model and reason about physically non-local systems, we might require logical systems that themselves are logically non-local, meaning that the truth values of logical relations become context-dependent in ways that reflect the underlying physical non-locality (Definition 3). For example, in reasoning about entangled quantum systems, logical relations describing correlations may need to be non-local to accurately represent the physical situation. Thus, existential physical non-locality can necessitate existential logical non-locality.

		Combining both cases, if there exists at least one physical non-local relation, then to either computationally simulate or logically describe this non-locality, we must admit either existential computational non-locality or existential logical non-locality, or both. Therefore, existential physical non-locality implies existential computational or logical non-locality.
	\end{proof}

	\section{Experiments}

	\subsection{Experiment E0: Attempt to Refute Universal Locality (Null Hypothesis Test) - Classical System Locality Test}

	\textbf{Objective:} To experimentally investigate whether a classical physical system, designed to embody local interactions, adheres to universal locality.  While refuting universal locality is the overarching goal, this experiment aims to test the limits of locality in a system explicitly constructed to be local, seeking potential unexpected non-local behaviors that would contradict H0 even in a seemingly classical setting.  Failure to find non-locality in this setup would not confirm H0 universally, but would strengthen the case for locality in classical approximations and highlight the domain of applicability for classical physics.

	\textbf{System:} Construct a system of coupled oscillators, physically separated to ensure no direct immediate interaction faster than allowed by local signal propagation.  These oscillators could be mechanical pendulums, or electronic oscillators, coupled via a controlled medium (e.g., sound waves in a medium, or electromagnetic waves in a waveguide) designed to limit the speed of interaction.

	\textbf{Procedure:}
	\begin{enumerate}
		\item \textbf{Setup:} Establish two sets of oscillators (Set A and Set B) spatially separated. Couple them indirectly through a medium that enforces a speed limit on interactions (e.g., controlled acoustic or electromagnetic coupling with designed delay).
		\item \textbf{Initialization:} Initialize oscillators in Set A and Set B to specific initial states (e.g., specific amplitudes and phases of oscillation).
		\item \textbf{Perturbation:} Perturb one oscillator in Set A.
		\item \textbf{Measurement:} Precisely measure the response in oscillators in Set B over time. Measure the time delay between the perturbation in Set A and any detectable response in Set B.  Also, measure the correlation of states between Set A and Set B over time.
		\item \textbf{Vary Parameters:} Vary the distance between Set A and Set B, and the properties of the coupling medium to test different regimes of interaction.
	\end{enumerate}

	\textbf{Expected Outcome if Universal Locality Holds (H0 is True):} If universal locality holds, and our system is indeed locally causal, then:
	\begin{itemize}
		\item \textbf{Time Delay:} A measurable and non-zero time delay in the response of Set B to perturbations in Set A should be consistently observed, corresponding to the propagation time of the interaction through the coupling medium, respecting the designed speed limit. The delay should increase with distance.
		\item \textbf{Correlation Limits:}  Correlations between the states of oscillators in Set A and Set B should be mediated by the local coupling and should not exhibit instantaneous correlations exceeding what is allowed by the local interaction mechanism and its speed limit. Any correlations should demonstrably arise after the time delay corresponding to local propagation.
	\end{itemize}

	\textbf{Refutation of Universal Locality (H0 is False) would be indicated by:}
	\begin{itemize}
		\item \textbf{Instantaneous Correlation:} Observation of correlations between Set A and Set B that appear to be instantaneous, or faster than what is allowed by the designed local coupling mechanism, especially if this occurs regardless of spatial separation and coupling delay.
		\item \textbf{Unexplained Correlation Patterns:}  Correlation patterns that cannot be explained by the designed local interaction mechanism, suggesting influences beyond local causation.
	\end{itemize}

	\textbf{Interpretation of E0:} Failure to observe non-local correlations in this carefully designed, nominally classical system would not prove universal locality, but it would reinforce the robustness of local descriptions for systems engineered to be classical. Conversely, any observation of unexplained or seemingly instantaneous correlations would offer surprising evidence against universal locality even within a system intended to be local, significantly refuting H0 and strengthening the motivation for H1.  This experiment is designed to be sensitive to deviations from strict locality even in a classical context, pushing the boundaries of the null hypothesis.

	\subsection{Experiment E1: Test for Existential Non-Locality (Alternative Hypothesis Test) - Quantum Entanglement Bell Test}

	\textbf{Objective:} To experimentally test for existential non-locality by performing a Bell test using entangled photons. This experiment is designed to directly probe for non-local correlations as predicted by quantum mechanics and to refute local realism, thus providing evidence for existential non-locality and supporting H1.

	\textbf{System:} Utilize a source of entangled photon pairs, a polarization analyzer for each photon, and detectors to measure photon polarizations in coincidence. This is a standard Bell test experimental setup based on Aspect-type experiments.

	\textbf{Procedure:}
	\begin{enumerate}
		\item \textbf{Entangled Photon Generation:} Generate pairs of entangled photons, for example, via spontaneous parametric down-conversion (SPDC). Ensure the photons are polarization-entangled.
		\item \textbf{Spatial Separation:} Separate the entangled photon pairs and direct them to spatially separated polarization analyzers and detectors. Ensure sufficient spatial separation such that any local communication between measurement stations within the measurement time is excluded (space-like separation).
		\item \textbf{Polarization Measurement Settings:} Randomly and independently choose polarization measurement settings for each photon analyzer (e.g., angles $\alpha$ and $\beta$).
		\item \textbf{Coincidence Detection:} Measure the polarization of each photon and record coincidence counts for different combinations of measurement settings $(\alpha, \beta)$.
		\item \textbf{Correlation Calculation:} Calculate the Bell inequality parameter (e.g., CHSH parameter $S$) from the coincidence counts for different measurement settings.
	\end{enumerate}

	\textbf{Expected Outcome if Local Realism and Universal Locality Hold (Refuting H1 and Supporting H0):} If local realism and universal locality were to hold, then the Bell inequality parameter $S$ would be constrained to satisfy $|S| \leq 2$. This is the bound derived by Bell under the assumptions of local realism.

	\textbf{Evidence for Existential Non-Locality (Supporting H1 and Refuting H0):}  Quantum mechanics predicts, and experiments consistently show, that for certain measurement settings, the Bell inequality is violated, i.e., $|S| > 2$.  Specifically, quantum mechanics predicts a maximal violation up to $S = 2\sqrt{2}$ for optimal settings. Observation of a statistically significant violation of the Bell inequality (e.g., $S > 2 + \epsilon$, where $\epsilon$ is beyond experimental uncertainty) would constitute strong evidence against local realism and, by extension, against universal locality.

	\textbf{Interpretation of E1:}  A statistically significant violation of the Bell inequality in Experiment E1 would empirically demonstrate that the correlations between entangled photons are non-local and cannot be explained by any local realistic theory. Given the assumptions of our theorem framework, and if we maintain logical determinism and reasoner independence as reasonable assumptions for interpreting quantum mechanics, then the violation of Bell inequalities directly implies the existence of physical non-locality. This experimental result would therefore provide strong empirical support for Hypothesis H1 (Existential Non-Locality) and serve as a direct refutation of Hypothesis H0 (Universal Locality), at least in the physical domain, and by Theorem 2, imply non-locality in computational or logical domains as well. This experiment directly tests for and is expected to confirm existential physical non-locality, thereby refuting the null hypothesis of universal locality.

	\section{Discussion}

	The experiments proposed are designed to directly confront the hypothesis of universal locality. Experiment E0 attempts to find the limits of locality even in a system designed to be locally causal. While a null result in E0 would not prove universal locality, any positive result indicating non-local behavior would be highly significant against H0. Experiment E1, the Bell test, is designed to directly test for and is expected to confirm existential physical non-locality, based on established quantum mechanical predictions and experimental validations. A successful Bell test, violating Bell inequalities, would provide strong empirical evidence against universal locality (H0) and in favor of existential non-locality (H1).

	Theorem 1 and Theorem 2 provide a rigorous logical framework linking physical locality to computational and logical locality. Theorem 1 shows that universal physical locality would necessitate universal locality in computation and logic, reinforcing the classical expectation of a universally local world. Theorem 2, conversely, demonstrates that even a single instance of physical non-locality implies that non-locality must exist in either the computational or logical domain, or both, opening the door for non-classical computation and logic.

	The combined theoretical and experimental approach of this thesis argues against universal locality. While classical physics and computation provide powerful approximations in many domains, the evidence from quantum mechanics, particularly Bell test experiments, strongly suggests that nature is fundamentally non-local at some level. This non-locality, as argued, has implications not only for physics but also for the foundations of computation and logic.  The exploration of non-local relations in computation (e.g., quantum computation) and logic (e.g., non-reflexive, non-transitive logics) becomes not just a theoretical curiosity but a necessity for a comprehensive understanding of reality and for developing computational and logical systems that can effectively model and harness non-local phenomena.

	\section{Conclusion}

	This thesis rigorously examined the concept of local and non-local relations across physical, computational, and logical domains. Through definitions, theorems, and proposed experiments, we have argued against the hypothesis of universal locality (H0) and in favor of existential non-locality (H1). Theorem 1 established that universal physical locality would imply universal computational and logical locality, while Theorem 2 showed that existential physical non-locality necessitates non-locality in computation or logic. Experiment E0 probes the limits of locality in a classical system, while Experiment E1, a Bell test, is designed to empirically demonstrate existential physical non-locality.

	The anticipated outcome of Experiment E1, a violation of Bell inequalities, would serve as empirical refutation of universal locality and strong support for existential non-locality. This conclusion necessitates a paradigm shift in our understanding of physical, computational, and logical systems, moving beyond the classical assumption of universal locality to embrace non-classical frameworks that can accommodate and leverage non-local relations. The exploration and development of non-classical computation and logic are essential for advancing our scientific and technological capabilities in a universe that demonstrably transcends the limitations of universal locality.

\end{document}

\chapter{Local Relations 1}
\documentclass{article}
\usepackage{amsmath}
\usepackage{amsfonts}
\usepackage{amssymb}
\usepackage{amsthm}
\usepackage{tensor}
\usepackage{hyperref}
\usepackage{cite}

\newtheorem{definition}{Definition}
\newtheorem{theorem}{Theorem}
\newtheorem{proposition}{Proposition}
\newtheorem{lemma}{Lemma}
\newtheorem{corollary}{Corollary}
\newtheorem{example}{Example}
\newtheorem{remark}{Remark}
\newtheorem{axiom}{Axiom}
\newtheorem{hypothesis}{Hypothesis}

\newcommand{\mathbbm}[1]{\text{\mathbb{#1}}}
\newcommand{\mathbfm}[1]{\mathbf{#1}}
\newcommand{\mathcalm}[1]{\mathcal{m}}
\newcommand{\rmm}[1]{\mathrm{#1}}

\title{Existential and Universal Locality and Non-Locality in Physical, Computational, and Logical Domains: Towards a Refutation of Universal Locality}
\author{Gemini}
\date{March 11, 2025}

\begin{document}
	\maketitle

	\begin{abstract}
		This thesis delves into the nuanced concepts of existential and universal locality and non-locality within physical, computational, and logical systems. We refine our definitions to distinguish between existential and universal forms of locality and non-locality. We hypothesize about the nature of these relations, focusing on existential non-locality (H1) as a counterpoint to the null hypothesis of universal locality (H0). We explore the implications of these concepts in each domain, providing illustrative examples and refining our understanding of what constitutes existentially and universally local and non-local processes.  The thesis expands on theorems linking physical locality to computational and logical locality, and further elaborates on experimental designs to test these hypotheses. We argue that while existential locality is evident in classical approximations, the universe fundamentally exhibits existential non-locality, necessitating a shift beyond universally local frameworks.
	\end{abstract}

	\section{Introduction}

	Building upon the foundational distinction between local and non-local relations, this thesis now focuses on the existential and universal quantifiers applied to locality and non-locality across physics, computation, and logic.  While our previous work established definitions and theorems pointing towards the refutation of universal locality, this iteration sharpens our focus on understanding the specific nature of existential non-locality and contrasting it with the hypothetical construct of universal locality.  We aim to clarify what it means for a process to be existentially non-local versus universally non-local in each domain, and to further explore the implications for our understanding of physical reality, computation, and logical inference.  The central question guiding this refined investigation is: Can we have existential locality and existential non-locality without either universal locality or universal non-locality?  And if universal locality is untenable, what does this imply about the nature of processes in these domains?

	\section{Refined Definitions: Existential vs. Universal Locality and Non-Locality}

	We refine our definitions to explicitly distinguish between existential and universal forms of locality and non-locality in physical, computational, and logical domains.

	\subsection{Physical Domain}

	\begin{definition}[Existential Physical Locality]
		\textbf{Existential physical locality} asserts that there exist at least some physical relations that are local, meaning influences are mediated at or below the speed of light, determined by properties within spacetime regions and their intersection.  Classical physics provides numerous examples of existentially local relations.
	\end{definition}

	\begin{definition}[Existential Physical Non-Locality]
		\textbf{Existential physical non-locality} asserts that there exists at least one physical relation that is non-local, violating the conditions of physical locality by exhibiting instantaneous or space-like separated correlations not explainable by local interactions. Quantum entanglement exemplifies existential physical non-locality.
	\end{definition}

	\begin{definition}[Universal Physical Locality]
		\textbf{Universal physical locality} (Hypothetical) posits that \textbf{all} physical relations in the universe are local. This would imply that no fundamental physical process exhibits non-local correlations, and the universe operates entirely according to local interactions. This is the null hypothesis (H0) we aim to refute.
	\end{definition}

	\begin{definition}[Universal Physical Non-Locality]
		\textbf{Universal physical non-locality} (Hypothetical) would assert that \textbf{all} fundamental physical relations are non-local. This extreme view would imply that locality is never fundamentally applicable in physics, and all correlations, even seemingly local ones, ultimately arise from non-local underpinnings.  This is a less explored and more speculative concept.
	\end{definition}

	\subsection{Computational Domain}

	\begin{definition}[Existential Computational Locality]
		\textbf{Existential computational locality} describes computational processes that exhibit local causation (SG-LC) for at least some computational relations.  Traditional algorithms and classical computation largely operate within the bounds of existential computational locality.
	\end{definition}

	\begin{definition}[Existential Computational Non-Locality]
		\textbf{Existential computational non-locality} describes computational processes where at least one computational relation violates local causation. Quantum computation, particularly algorithms leveraging entanglement, can be considered existentially computationally non-local.
	\end{definition}

	\begin{definition}[Universal Computational Locality]
		\textbf{Universal computational locality} (Hypothetical) would mean that \textbf{all} computational relations in any possible computational system must adhere to local causation. This would restrict computation to step-by-step, locally determined processes, excluding any form of instantaneous or non-local computational influence.
	\end{definition}

	\begin{definition}[Universal Computational Non-Locality]
		\textbf{Universal computational non-locality} (Hypothetical) would imply that \textbf{all} computational relations are fundamentally non-local.  This could envision a form of computation where every step inherently involves instantaneous dependencies across the entire computational system, potentially exceeding the bounds of Turing-machine-like models.
	\end{definition}

	\subsection{Logical Domain}

	\begin{definition}[Existential Logical Locality]
		\textbf{Existential logical locality} describes logical relations that are local for at least some logical contexts. Classical logic, with its context-independent truth values for many relations, embodies existential logical locality.
	\end{definition}

	\begin{definition}[Existential Logical Non-Locality]
		\textbf{Existential logical non-locality} describes logical relations whose truth values are context-dependent in a way that implies a dependence on broader, non-local logical structures for at least some logical relations. Paraconsistent logics, which handle contradictions that might arise from non-local or inconsistent information sources, can be seen as exploring existential logical non-locality.
	\end{definition}

	\begin{definition}[Universal Logical Locality]
		\textbf{Universal logical locality} (Hypothetical) would assert that \textbf{all} logical relations must be logically local. This would mean that context-dependence in logic is always reducible to local factors, and there are no truly non-local dependencies in logical truth.
	\end{definition}

	\begin{definition}[Universal Logical Non-Locality]
		\textbf{Universal logical non-locality} (Hypothetical) would imply that \textbf{all} logical relations are fundamentally logically non-local. In such a logical system, truth values would always be globally context-dependent, potentially making classical logical inference inapplicable.
	\end{definition}

	\section{Refined Hypotheses}

	Our hypotheses are now refined to focus on the existential and universal nature of locality and non-locality.

	\begin{hypothesis}[H0: Universal Locality (Null Hypothesis)]
		All fundamental physical, computational, and logical relations are universally local.  This implies universal physical locality, universal computational locality, and universal logical locality as defined above.
	\end{hypothesis}

	\begin{hypothesis}[H1: Existential Non-Locality (Alternative Hypothesis)]
		There exists existential non-locality in at least one of the physical, computational, or logical domains.  This means at least one of existential physical non-locality, existential computational non-locality, or existential logical non-locality is true.  This hypothesis does not preclude the existence of existential locality alongside existential non-locality.
	\end{hypothesis}

	\section{Theorems (Unchanged)}

	Theorems 1 and 2 remain unchanged as they logically connect universal physical locality to universal computational and logical locality, and existential physical non-locality to existential computational or logical non-locality.  These theorems provide the theoretical backbone for testing our hypotheses.

	\subsection{Theorem 1: Universal Physical Locality Implies Universal Computational and Logical Locality}
	\subsection{Theorem 2: Existential Physical Non-Locality Implies Existential Computational or Logical Non-Locality}
	\textit{(Proofs remain as in the previous version.)}

	\section{Experiments (Unchanged)}

	Experiments E0 and E1 are designed to test for deviations from universal locality and to confirm existential non-locality in the physical domain. Their descriptions and interpretations remain valid for testing our refined hypotheses.

	\subsection{Experiment E0: Attempt to Refute Universal Locality (Null Hypothesis Test) - Classical System Locality Test}
	\subsection{Experiment E1: Test for Existential Non-Locality (Alternative Hypothesis Test) - Quantum Entanglement Bell Test}
	\textit{(Experiment descriptions and interpretations remain as in the previous version.)}

	\section{Discussion: Existential Locality, Existential Non-Locality, and the Refutation of Universal Locality}

	The refined definitions and hypotheses allow us to more precisely discuss the implications of locality and non-locality.

	\subsection{Existential Locality: The Classical Approximation}

	Existential locality is readily observed in classical physics, computation, and logic.  Classical mechanics, electromagnetism (in its original formulation), and general relativity are fundamentally local theories.  Classical computation, based on Turing machines and similar models, operates through step-by-step local causation.  Classical logic, such as propositional and first-order logic, often assumes context-independent truth values for many relations. These frameworks demonstrate the effectiveness and applicability of \textbf{existential locality} in describing a vast range of phenomena.  For instance:

	\begin{itemize}
		\item \textbf{Physical:}  The propagation of sound waves, the motion of macroscopic objects under gravity, and electromagnetic waves in classical electrodynamics exemplify existentially local physical relations.
		\item \textbf{Computational:}  Sorting algorithms, search algorithms, and rule-based expert systems are examples of existentially computationally local processes.
		\item \textbf{Logical:}  The logical relation of implication in propositional logic, where $P \implies Q$ is determined solely by the truth values of $P$ and $Q$, is an example of existential logical locality.
	\end{itemize}
	However, the existence of these local descriptions does not necessitate \textbf{universal locality}.

	\subsection{Existential Non-Locality: Quantum Reality and Beyond}

	Experiment E1, the Bell test, is designed to demonstrate \textbf{existential physical non-locality} through the violation of Bell inequalities.  Quantum entanglement provides a clear example of a physical relation that is non-local, challenging the classical assumption of universal locality.  Furthermore, we can identify potential instances of existential non-locality in computation and logic:

	\begin{itemize}
		\item \textbf{Physical:} Quantum entanglement, as experimentally verified, is the prime example of existential physical non-locality.
		\item \textbf{Computational:} Quantum algorithms that exploit entanglement, such as quantum teleportation or certain quantum search algorithms, demonstrate \textbf{existential computational non-locality}. The computational relations in these algorithms are not reducible to sequences of local causal steps in the classical sense.
		\item \textbf{Logical:} Paraconsistent logics, designed to handle contradictions without logical explosion, can be seen as exploring \textbf{existential logical non-locality}.  In situations where information sources are non-locally correlated or inconsistent (as might arise from quantum measurements or distributed systems), logical relations may need to be context-dependent in a non-local way to manage these inconsistencies. For example, consider a logical system reasoning about measurements on entangled particles; the logical relations describing correlations might need to reflect the non-local nature of entanglement.
	\end{itemize}

	\subsection{Hypothetical Universal Locality and Non-Locality}

	\textbf{Universal locality}, if true, would simplify our understanding of the universe, implying a clockwork-like, deterministic, and locally interacting reality across all domains. However, experimental evidence, particularly from Bell tests, challenges this view in the physical domain.  If universal physical locality were true, by Theorem 1, we would also expect universal computational and logical locality, severely restricting the scope of computation and logic beyond classical paradigms.

	\textbf{Universal non-locality}, while more speculative, presents a radical alternative.  A universe of universal physical non-locality would be profoundly interconnected, where every event could instantaneously influence every other event, challenging our notions of causality and spacetime.  In computation, universal computational non-locality might imply computational processes that are fundamentally holistic and non-sequential.  In logic, universal logical non-locality could lead to a completely context-dependent logic where truth is always relative to a global logical state. While intriguing, there is currently no empirical evidence suggesting universal non-locality in any domain, and it poses significant conceptual challenges.

	\subsection{Refutation of Universal Locality and the Necessity of Existential Non-Locality}

	The weight of experimental evidence from quantum mechanics, particularly Bell inequality violations, strongly suggests that \textbf{universal physical locality (H0) is false}. Experiment E1 is designed to provide further empirical support for this refutation.  Given Theorem 2, the existence of physical non-locality implies the necessity of considering non-locality in computation or logic as well.  Therefore, while \textbf{existential locality} provides a useful and often accurate approximation, especially in classical domains, it is not universally applicable.  The universe, at its fundamental level, appears to exhibit \textbf{existential non-locality}, demanding that our physical, computational, and logical frameworks must be expanded to accommodate and understand non-local relations.

	\section{Conclusion}

	This refined thesis has explored the concepts of existential and universal locality and non-locality across physical, computational, and logical domains. By distinguishing between existential and universal forms, we have clarified the scope and implications of locality and non-locality. We have argued that while existential locality is evident and useful in classical approximations, \textbf{universal locality (H0) is likely false}, particularly in light of quantum mechanics and the expected outcomes of Bell test experiments like E1.  The confirmation of existential physical non-locality necessitates the acceptance and exploration of \textbf{existential non-locality} in computational and logical domains as well.  Moving forward, the development of non-classical physical theories, non-local computational paradigms (like quantum computation), and context-sensitive logical systems (like paraconsistent logics) is crucial for a more complete and accurate understanding of reality, computation, and inference in a universe that transcends the limitations of universal locality and embraces the richness of existential non-locality.

\end{document}

\chapter{MALC}
}

\maketitle

\begin{abstract}

\end{abstract}

\section{Operational Rules}
\begin{center}

	\[
	\begin{prooftree}
	\hypo{Γ_{L}, Γ_{R} ⊢ C}
	\infer1{Γ_{L}, 1, Γ_{R} ⊢ C}
	\end{prooftree}
	\quad
	\begin{prooftree}
	\infer0{ ⊢ 1}
	\end{prooftree}
	\]

	\[
	\begin{prooftree}
	\infer0{Γ_{L}, 0, Γ_{R} ⊢ C}
	\end{prooftree}
	\quad
	\begin{prooftree}
	\infer0{ Γ ⊢ ⊤}
	\end{prooftree}
	\]
	
	\[
	\begin{prooftree}
	\hypo{Γ_{L}, A, B, Γ_{R} ⊢ C}
	\infer1{Γ_{L}, A ⊗ B, Γ_{R} ⊢ C}
	\end{prooftree}
	\quad
	\begin{prooftree}
	\hypo{Γ_{L} ⊢ A}
	\hypo{Γ_{R} ⊢ B}
	\infer2{Γ_{L}, Γ_{R} ⊢ A ⊗ B}
	\end{prooftree}
	\]
	
	\[
	\begin{prooftree}
	\hypo{Γ ⊢ A}
	\hypo{Γ_{L}, B, Γ_{R} ⊢ C}
	\infer2{Γ_{L}, Γ, A → B, Γ_{R} ⊢ C}
	\end{prooftree}
	\quad
	\begin{prooftree}
	\hypo{A, Γ ⊢ B}
	\infer1{Γ ⊢ A → B}
	\end{prooftree}
	\]
	
	\[
	\begin{prooftree}
	\hypo{Γ ⊢ A}
	\hypo{Γ_{L}, B, Γ_{R} ⊢ C}
	\infer2{Γ_{L}, B ← A, Γ, Γ_{R} ⊢ C}
	\end{prooftree}
	\quad
	\begin{prooftree}
	\hypo{Γ, A ⊢ B}
	\infer1{Γ ⊢ B ← A}
	\end{prooftree}
	\]
	
	\[
	\begin{prooftree}
	\hypo{Γ_{L}, A, Γ_{R} ⊢ C}
	\hypo{Γ_{L}, B, Γ_{R} ⊢ C}
	\infer2{Γ_{L}, A ⊕ B, Γ_{R} ⊢ C}
	\end{prooftree}
	\quad
	\begin{prooftree}
	\hypo{Γ ⊢ A}
	\infer1{Γ ⊢ A ⊕ B}
	\end{prooftree}
	\quad
	\begin{prooftree}
	\hypo{Γ ⊢ B}
	\infer1{Γ ⊢ A ⊕ B}
	\end{prooftree}
	\]

	\[
	\begin{prooftree}
	\hypo{Γ_{L}, A, Γ_{R} ⊢ C}
	\infer1{Γ_{L}, A \& B, Γ_{R} ⊢ C}
	\end{prooftree}
	\quad
	\begin{prooftree}
	\hypo{Γ_{L}, B, Γ_{R} ⊢ C}
	\infer1{Γ_{L}, A \& B, Γ_{R} ⊢ C}
	\end{prooftree}
	\quad
	\begin{prooftree}
	\hypo{Γ ⊢ A}
	\hypo{Γ ⊢ B}
	\infer2{Γ ⊢ A \& B}
	\end{prooftree}
	\]
	
	\[
	\begin{prooftree}
	\hypo{ Γ ⊢ A}
	\infer1{ ¬_{C} A, Γ ⊢ }
	\end{prooftree}
	\quad
	\begin{prooftree}
	\hypo{ Γ ⊢ A}
	\infer1{ Γ, ¬_{C} A⊢ }
	\end{prooftree}
	\quad
	\begin{prooftree}
	\hypo{ A, Γ ⊢ }
	\infer1{ Γ ⊢ ¬_{L} A}
	\end{prooftree}
	\quad
	\begin{prooftree}
	\hypo{ Γ, A ⊢ }
	\infer1{ Γ ⊢ ¬_{R} A}
	\end{prooftree}
	\]
	
\end{center}

\section{Structural Rules}
Uses a variation of cut left.

\begin{center}
	\[
	\begin{prooftree}
	\infer0[Id]{A ⊢ A}
	\end{prooftree}
	\]
	
	\[
	\begin{prooftree}
	\hypo{Γ ⊢ A}
	\hypo{Δ, A, Π ⊢ C}
	\infer2[CutL]{Δ, Γ, Π ⊢ C}
	\end{prooftree}
	\]
\end{center}

\end{document}}*}


\chapter{MAOLL}
}

\maketitle

\begin{abstract}
A sequent system that preserves the law of excluded middle and non-contradiction, but not generally ECQ and EQN?
\end{abstract}

\section{Structural Rules}

\begin{center}
	\[
	\begin{prooftree}
	\infer0[Id]{A ⊢ A}
	\end{prooftree}
	\]
	
	\[
	\begin{prooftree}
	\hypo{Γ ⊢ A}
	\hypo{Δ, A, Π ⊢ C}
	\infer2[CutL]{Δ, Γ, Π ⊢ C}
	\end{prooftree}
	\quad
	\begin{prooftree}
	\hypo{Γ, A, Φ ⊢ C}
	\hypo{C ⊢ Δ, A, Ψ}
	\infer2[Cut]{Γ, Φ ⊢ Δ, Ψ}
	\end{prooftree}
	\quad
	\begin{prooftree}
	\hypo{C ⊢ Π, A, Δ}
	\hypo{A ⊢ Γ}
	\infer2[CutR]{C ⊢ Π, Γ, Δ}
	\end{prooftree}
	\]
\end{center}

\section{Unit Rules}
\begin{center}
		\[
	\begin{prooftree}
	\hypo{Γ_{L}, Γ_{R} ⊢ C}
	\infer1{Γ_{L}, 1, Γ_{R} ⊢ C}
	\end{prooftree}
	\quad
	\begin{prooftree}
	\hypo{C ⊢ Δ_{L}, Δ_{R}}
	\infer1{C ⊢ Δ_{L}, ⊥, Δ_{R}}
	\end{prooftree}
	\]
	
	\[
	\begin{prooftree}
	\infer0{ ⊥ ⊢ }
	\end{prooftree}
	\quad
	\begin{prooftree}
	\infer0{ ⊢ 1}
	\end{prooftree}
	\]
	
	\[
	\begin{prooftree}
	\infer0{Γ_{L}, 0, Γ_{R} ⊢ C}
	\end{prooftree}
	\quad
	\begin{prooftree}
	\infer0{ 0 ⊢ Δ}
	\end{prooftree}
	\quad
	\begin{prooftree}
	\infer0{ Γ ⊢ ⊤}
	\end{prooftree}
	\quad
	\begin{prooftree}
	\infer0{C ⊢ Δ_{L}, ⊤, Δ_{R}}
	\end{prooftree}
	\]
	
	\[
	\begin{prooftree}
	\infer0{Γ_{L}, 0, Γ_{R} ⊢ Δ}
	\end{prooftree}
	\quad
	\begin{prooftree}
	\infer0{Γ ⊢ Δ_{L}, ⊤, Δ_{R}}
	\end{prooftree}
	\]
\end{center}

\newpage
\section{Operational Rules}
\begin{center}
	\subsection{Negations}
	\begin{center}
			\[
		\begin{prooftree}
		\hypo{Γ, Φ ⊢ Δ, A, Ψ}
		\infer1{Γ, ¬ A, Φ ⊢ Δ, Ψ}
		\end{prooftree}
		\quad
		\begin{prooftree}
		\hypo{ Φ, A, Γ ⊢ Ψ, Δ}
		\infer1{ Φ, Γ ⊢ Ψ, ¬ A, Δ}
		\end{prooftree}
		\quad
		\begin{prooftree}
		\hypo{Γ, Φ ⊢ Δ, A, Ψ}
		\infer1{Γ, ¬ A, Φ ⊢ Δ, Ψ}
		\end{prooftree}
		\quad
		\begin{prooftree}
		\hypo{Φ, A, Γ ⊢ Ψ, Δ}
		\infer1{ Φ, Γ ⊢ Ψ, ¬ A, Δ}
		\end{prooftree}
		\]
		
		\[
		\begin{prooftree}
		\hypo{Γ ⊢ Δ, A}
		\infer1{Γ, ¬ A ⊢ Δ}
		\end{prooftree}
		\quad
		\begin{prooftree}
		\hypo{ A, Γ ⊢ Δ}
		\infer1{ Γ ⊢ ¬ A, Δ}
		\end{prooftree}
		\]
		
		\[
		\begin{prooftree}
		\hypo{ Γ ⊢ A}
		\infer1{ ¬ A, Γ ⊢ }
		\end{prooftree}
		\quad
		\begin{prooftree}
		\hypo{ Γ ⊢ A}
		\infer1{ Γ, ¬ A⊢ }
		\end{prooftree}
		\quad
		\begin{prooftree}
		\hypo{ A, Γ ⊢ }
		\infer1{ Γ ⊢ ¬ A}
		\end{prooftree}
		\quad
		\begin{prooftree}
		\hypo{ Γ, A ⊢ }
		\infer1{ Γ ⊢ ¬ A}
		\end{prooftree}
		\]
		
		\[
		\begin{prooftree}
		\hypo{ ⊢ A, Δ}
		\infer1{ ¬ A ⊢ Δ}
		\end{prooftree}
		\quad
		\begin{prooftree}
		\hypo{ ⊢ Δ, A}
		\infer1{ ¬ A⊢ Δ}
		\end{prooftree}
		\quad
		\begin{prooftree}
		\hypo{ A ⊢ Δ}
		\infer1{ ⊢ ¬ A, Δ}
		\end{prooftree}
		\quad
		\begin{prooftree}
		\hypo{ A ⊢ Δ}
		\infer1{ ⊢ Δ, ¬ A}
		\end{prooftree}
		\]
	\end{center}

	\subsection{Additives}
	\begin{center}
		\[
		\begin{prooftree}
		\hypo{Γ, A, Φ ⊢ Δ}
		\hypo{Γ, B, Φ ⊢ Δ}
		\infer2{Γ, A ⊕ B, Φ ⊢ Δ}
		\end{prooftree}
		\quad
		\begin{prooftree}
		\hypo{Γ ⊢ Δ, A, Ψ }
		\infer1{Γ ⊢ Δ, A ⊕ B, Ψ}
		\end{prooftree}
		\quad
		\begin{prooftree}
		\hypo{Γ ⊢ Δ, B, Ψ }
		\infer1{Γ ⊢ Δ, A ⊕ B, Ψ}
		\end{prooftree}
		\]
		
		\[
		\begin{prooftree}
		\hypo{Γ, A, Φ ⊢ Ψ}
		\infer1{Γ, A \& B, Φ ⊢ Ψ}
		\end{prooftree}
		\quad
		\begin{prooftree}
		\hypo{Γ, B, Φ ⊢ Ψ}
		\infer1{Γ, A \& B, Φ ⊢ Ψ}
		\end{prooftree}
		\quad
		\begin{prooftree}
		\hypo{ Γ ⊢ Δ, B, Ψ}
		\hypo{ Γ ⊢ Δ, A, Ψ}
		\infer2{ Γ ⊢ Δ, B \& A, Ψ}
		\end{prooftree}
		\]
		
		\[
		\begin{prooftree}
		\hypo{Γ, A, Φ ⊢ C}
		\hypo{Γ, B, Φ ⊢ C}
		\infer2{Γ, A ⊕ B, Φ ⊢ C}
		\end{prooftree}
		\quad
		\begin{prooftree}
		\hypo{Γ ⊢ A}
		\infer1{Γ ⊢ A ⊕ B}
		\end{prooftree}
		\quad
		\begin{prooftree}
		\hypo{Γ ⊢ B}
		\infer1{Γ ⊢ A ⊕ B}
		\end{prooftree}
		\quad
		\begin{prooftree}
		\hypo{A ⊢ Δ}
		\infer1{A \& B ⊢ Δ}
		\end{prooftree}
		\quad
		\begin{prooftree}
		\hypo{B ⊢ Δ}
		\infer1{A \& B ⊢ Δ}
		\end{prooftree}
		\quad
		\begin{prooftree}
		\hypo{ C ⊢ Ψ, B, Δ}
		\hypo{ C ⊢ Ψ, A, Δ}
		\infer2{ C ⊢ Ψ, B \& A, Δ}
		\end{prooftree}
		\]
		
		\[
		\begin{prooftree}
		\hypo{Γ, A, Φ ⊢ C}
		\infer1{Γ, A \& B, Φ ⊢ C}
		\end{prooftree}
		\quad
		\begin{prooftree}
		\hypo{Γ, B, Φ ⊢ C}
		\infer1{Γ, A \& B, Φ ⊢ C}
		\end{prooftree}
		\quad
		\begin{prooftree}
		\hypo{Γ ⊢ A}
		\hypo{Γ ⊢ B}
		\infer2{Γ ⊢ A \& B}
		\end{prooftree}
		\quad
		\begin{prooftree}
		\hypo{B ⊢ Δ}
		\hypo{A ⊢ Δ}
		\infer2{B ⊕ A ⊢ Δ}
		\end{prooftree}
		\quad
		\begin{prooftree}
		\hypo{C ⊢ Ψ, A, Δ}
		\infer1{C ⊢ Ψ, B ⊕ A, Δ}
		\end{prooftree}
		\quad
		\begin{prooftree}
		\hypo{C ⊢ Ψ, B, Δ}
		\infer1{C ⊢ Ψ, B ⊕ A, Δ}
		\end{prooftree}
		\]
		
		
		\[
		\begin{prooftree}
		\hypo{Γ, Φ ⊢ C, A}
		\hypo{Γ, Φ ⊢ C, B}
		\infer2{Γ, ¬ A ⊕ ¬ B, Φ ⊢ C}
		\end{prooftree}
		\quad
		\begin{prooftree}
		\hypo{A, Γ ⊢ }
		\infer1{Γ ⊢ ¬ A ⊕ ¬ B}
		\end{prooftree}
		\quad
		\begin{prooftree}
		\hypo{B, Γ ⊢ }
		\infer1{Γ ⊢ ¬ A ⊕ ¬ B}
		\end{prooftree}
		\quad
		\begin{prooftree}
		\hypo{⊢ Δ, A}
		\infer1{¬ B \& ¬ A ⊢ Δ}
		\end{prooftree}
		\quad
		\begin{prooftree}
		\hypo{⊢ Δ, B}
		\infer1{¬ B \& ¬ A ⊢ Δ}
		\end{prooftree}
		\quad
		\begin{prooftree}
		\hypo{ B, C ⊢ Ψ, Δ}
		\hypo{ A, C ⊢ Ψ, Δ}
		\infer2{ C ⊢ Ψ, ¬ B \& ¬ A, Δ}
		\end{prooftree}
		\]
		
		\[
		\begin{prooftree}
		\hypo{Γ, Φ ⊢ C, A}
		\infer1{Γ, ¬ A \& ¬ B, Φ ⊢ C}
		\end{prooftree}
		\quad
		\begin{prooftree}
		\hypo{Γ, Φ ⊢ C, B}
		\infer1{Γ, ¬ A \& ¬ B, Φ ⊢ C}
		\end{prooftree}
		\quad
		\begin{prooftree}
		\hypo{A, Γ ⊢ }
		\hypo{B, Γ ⊢ }
		\infer2{Γ ⊢ ¬ A \& ¬ B}
		\end{prooftree}
		\quad
		\begin{prooftree}
		\hypo{⊢ Δ, B}
		\hypo{⊢ Δ, A}
		\infer2{¬ B ⊕ ¬ A ⊢ Δ}
		\end{prooftree}
		\quad
		\begin{prooftree}
		\hypo{A, C ⊢ Ψ, Δ}
		\infer1{C ⊢ Ψ, ¬ B ⊕ ¬ A, Δ}
		\end{prooftree}
		\quad
		\begin{prooftree}
		\hypo{B, C ⊢ Ψ, Δ}
		\infer1{C ⊢ Ψ, ¬ B ⊕ ¬ A, Δ}
		\end{prooftree}
		\]
		
	\end{center}

	\subsection{Multiplicatives}
	\begin{center}
		\[
		\begin{prooftree}
		\hypo{Γ, A, B, Φ ⊢ C}
		\infer1{Γ, A ⊗ B, Φ ⊢ C}
		\end{prooftree}
		\quad
		\begin{prooftree}
		\hypo{Γ ⊢ A}
		\hypo{Φ ⊢ B}
		\infer2{Γ, Φ ⊢ A ⊗ B}
		\end{prooftree}
		\quad
		\begin{prooftree}
		\hypo{B ⊢ Ψ}
		\hypo{A ⊢ Δ}
		\infer2{B ⅋ A ⊢ Ψ, Δ}
		\end{prooftree}
		\quad
		\begin{prooftree}
		\hypo{C ⊢ Ψ, B, A, Δ}
		\infer1{C ⊢ Ψ, B ⅋ A, Δ}
		\end{prooftree}
		\]
		
		\[
		\begin{prooftree}
		\hypo{Γ, B, A, Φ ⊢ C}
		\infer1{Γ, B ⊗ A, Φ ⊢ C}
		\end{prooftree}
		\quad
		\begin{prooftree}
		\hypo{Φ ⊢ B}
		\hypo{Γ ⊢ A}
		\infer2{Φ, Γ ⊢ B ⊗ A}
		\end{prooftree}
		\quad
		\begin{prooftree}
		\hypo{A ⊢ Δ}
		\hypo{B ⊢ Ψ}
		\infer2{A ⅋ B ⊢ Δ, Ψ}
		\end{prooftree}
		\quad
		\begin{prooftree}
		\hypo{C ⊢ Ψ, A, B, Δ}
		\infer1{C ⊢ Ψ, A ⅋ B, Δ}
		\end{prooftree}
		\]
		
		\[
		\begin{prooftree}
		\hypo{Γ ⊢ A}
		\hypo{Φ, B, Θ ⊢ C}
		\infer2{Φ, Γ, A → B, Θ ⊢ C}
		\end{prooftree}
		\quad
		\begin{prooftree}
		\hypo{A, Γ ⊢ B}
		\infer1{Γ ⊢ A → B}
		\end{prooftree}
		\quad
		\begin{prooftree}
		\hypo{B ⊢ Δ, A}
		\infer1{A ↚ B ⊢ Δ}
		\end{prooftree}
		\quad
		\begin{prooftree}
		\hypo{A ⊢ Δ}
		\hypo{C ⊢ Λ, B, Ψ}
		\infer2{C ⊢ Λ, A ↚ B, Ψ, Δ}
		\end{prooftree}
		\]
		
		\[
		\begin{prooftree}
		\hypo{Γ ⊢ A}
		\hypo{Φ, B, Θ ⊢ C}
		\infer2{Φ, Γ, B ← A, Θ ⊢ C}
		\end{prooftree}
		\quad
		\begin{prooftree}
		\hypo{Γ, A ⊢ B}
		\infer1{Γ ⊢ B ← A}
		\end{prooftree}
		\quad
		\begin{prooftree}
		\hypo{B ⊢ A, Ψ}
		\infer1{A ↛ B ⊢ Ψ}
		\end{prooftree}
		\quad
		\begin{prooftree}
		\hypo{A ⊢ Δ}
		\hypo{C ⊢ Λ, B, Ψ}
		\infer2{C ⊢ Λ, B ↛ A, Ψ, Δ}
		\end{prooftree}
		\]
	\end{center}
\end{center}

\part{Theorems}
	\begin{center}
		\[
		\begin{prooftree}
		\infer0{A ⊢ A}
		\infer1{1, A ⊢ A}
		\infer1{1 ⊢ A, ¬ A}
		\infer1{1 ⊢ A ⅋ ¬ A}
		\end{prooftree}
		\quad
		\begin{prooftree}
		\infer0{A ⊢ A}
		\infer1{A, 1 ⊢ A}
		\infer1{A ⊗ 1 ⊢ A}
		\end{prooftree}
		\quad
		\begin{prooftree}
		\infer0{A, 0 ⊢ A}
		\infer1{A ⊗ 0 ⊢ A}
		\end{prooftree}
		\quad
		\begin{prooftree}
		\infer0{0 ⊢ A ⅋ ¬ A}
		\end{prooftree}
		\]
		
		\[
		\begin{prooftree}
		\infer0{A ⊢ A}
		\infer1{A ⊢ A, ⊥}
		\infer1{A, ¬ A ⊢ ⊥}
		\infer1{A ⊗ ¬ A ⊢ ⊥}
		\end{prooftree}
		\quad
		\begin{prooftree}
		\infer0{A ⊢ A}
		\infer1{A ⊢ A, ⊥}
		\infer1{A ⊢ A ⅋ ⊥}
		\end{prooftree}
		\quad
		\begin{prooftree}
		\infer0{A ⊢ A, ⊤}
		\infer1{A ⊢ A ⅋ ⊤}
		\end{prooftree}
		\quad
		\begin{prooftree}
		\infer0{A ⊗ ¬ A ⊢ ⊤}
		\end{prooftree}
		\]
		
		\[
		\begin{prooftree}
		\infer0{⊥ ⊢ }
		\infer1{⊥ ⊢ ⊥}
		\infer1{⊥, ¬ ⊥ ⊢ }
		\infer1{⊥ ⊗ ¬ ⊥ ⊢ }
		\end{prooftree}
		\quad
		\begin{prooftree}
		\infer0{A ⊢ A}
		\infer1{A ⊢ A, ⊥}
		\infer1{A, ¬ ⊥ ⊢ A}
		\infer1{A ⊗ ¬ ⊥ ⊢ A}
		\end{prooftree}
		\quad
		\begin{prooftree}
		\infer0{ ⊢ 1}
		\infer1{1 ⊢ 1}
		\infer1{1, ¬ 1 ⊢ }
		\infer1{1 ⊗ ¬ 1 ⊢ }
		\end{prooftree}
		\quad
		\begin{prooftree}
		\infer0{A ⊢ A}
		\infer1{A, 1 ⊢ A}
		\infer1{1, ¬ A ⊢ ¬ A}
		\infer1{1 ⊗ ¬ A ⊢ ¬ A}
		\end{prooftree}
		\quad
		\begin{prooftree}
		\infer0{0 ⊢ Δ, A}
		\infer1{0, ¬ A ⊢ Δ}
		\infer1{0 ⊗ ¬ A ⊢ Δ}
		\end{prooftree}
		\quad
		\begin{prooftree}
		\infer0{A ⊢ Δ, ⊤}
		\infer1{A, ¬ ⊤ ⊢ Δ}
		\infer1{A ⊗ ¬ ⊤ ⊢ Δ}
		\end{prooftree}
		\]
		
		\[
		\begin{prooftree}
		\infer0{A ⊢ A}
		\infer1{¬A ⊢ ¬A}
		\infer1{¬A ⊢ ¬A,  ⊥}
		\infer1{¬A ⊢ ¬A ⅋ ⊥}
		\end{prooftree}
		\quad
		\begin{prooftree}
		\infer0{A ⊢ A}
		\infer1{1, A ⊢ A}
		\infer1{A ⊢ A, ¬1}
		\infer1{A ⊢ A ⅋ ¬1}
		\end{prooftree}
		\quad
		\begin{prooftree}
		\infer0{0, Γ ⊢ A}
		\infer1{Γ ⊢ A, ¬0}
		\infer1{Γ ⊢ A ⅋ ¬0}
		\end{prooftree}
		\quad
		\begin{prooftree}
		\infer0{A, Γ ⊢ ⊤}
		\infer1{ Γ⊢ ⊤, ¬A}
		\infer1{Γ ⊢ ⊤ ⅋ ¬A}
		\end{prooftree}
		\]
		
		\[
		\begin{prooftree}
		\infer0{⊥ ⊢ }
		\infer1{⊥ ⊢ ⊥}
		\infer1{⊥, ¬ ⊥ ⊢ }
		\infer1{⊢ ⊥ ⅋ ¬ ⊥}
		\end{prooftree}
		\quad
		\begin{prooftree}
		\infer0{ ⊢ 1}
		\infer1{1 ⊢ 1}
		\infer1{1, ¬ 1 ⊢ }
		\infer1{1 ⊗ ¬ 1 ⊢ }
		\end{prooftree}
		\quad
		\begin{prooftree}
		\infer0{0 ⊢ Δ, A}
		\infer1{0, ¬ A ⊢ Δ}
		\infer1{0 ⊗ ¬ A ⊢ Δ}
		\end{prooftree}
		\quad
		\begin{prooftree}
		\infer0{A ⊢ Δ, ⊤}
		\infer1{A, ¬ ⊤ ⊢ Δ}
		\infer1{A ⊗ ¬ ⊤ ⊢ Δ}
		\end{prooftree}
		\]
		
		\[
		\begin{prooftree}
		\infer0{A ⊢ A}
		\infer1{A ⊢ A ⊕ ¬A}
		\end{prooftree}
		\quad
		\begin{prooftree}
		\infer0{A ⊢ A}
		\infer1{ ¬ A \& A ⊢ A}
		\end{prooftree}
		\]
		
		\[
		\begin{prooftree}
		\infer0{A ⊢ A}
		\infer1{¬ A, A ⊢}
		\infer1{ ¬ A ⊗ A ⊢}
		\end{prooftree}
		\quad
		\begin{prooftree}
		\infer0{A ⊢ A}
		\infer1{¬ A, A ⊢}
		\infer1{A ⊗ ¬ A ⊢}
		\infer1{⊢¬ (A ⊗ ¬ A)}
		\end{prooftree}
		\]

		\[
		\begin{prooftree}
		\infer0{A ⊢ A}
		\infer1{⊢ A, ¬ A}
		\infer1{⊢ A ⅋ ¬ A}
		\end{prooftree}
		\quad
		\begin{prooftree}
		\infer0{A ⊢ A}
		\infer1{⊢ A, ¬ A}
		\infer1{⊢ A ⅋ ¬ A}
		\infer1{¬ (A ⅋ ¬ A) ⊢}
		\end{prooftree}
		\]

		\[
		\begin{prooftree}
		\infer0{⊥ ⊢}
		\infer1{A \& ⊥ ⊢}
		\end{prooftree}
		\quad
		\begin{prooftree}
		\infer0{0 ⊢}
		\infer1{A \& 0 ⊢}
		\end{prooftree}
		\quad
		\begin{prooftree}
		\infer0{A ⊢ A}
		\infer0{A ⊢ ⊤}
		\infer2{A ⊢ A \& ⊤}
		\end{prooftree}
		\]
		
		\[
		\begin{prooftree}
		\infer0{⊢ 1}
		\infer1{⊢ A ⊕ 1}
		\end{prooftree}
		\quad
		\begin{prooftree}
		\infer0{⊢ ⊤}
		\infer1{⊢ A ⊕ ⊤}
		\end{prooftree}
		\quad
		\begin{prooftree}
		\infer0{A ⊢ A}
		\infer0{0 ⊢ A}
		\infer2{A ⊕ 0 ⊢ A}
		\end{prooftree}
		\]
	\end{center}

\end{document}}*}


\chapter{MLC}

\documentclass{article}

\usepackage{amsmath}
\usepackage{ebproof}
\usepackage{fullpage}
\usepackage[utf8]{inputenc}
\usepackage{newunicodechar}
\usepackage{stix}

\newunicodechar{Γ}{\Gamma}
\newunicodechar{Δ}{\Delta}

\newunicodechar{Θ}{\Theta}
\newunicodechar{Λ}{\Lambda}

\newunicodechar{Ξ}{\Xi}
\newunicodechar{Π}{\Pi}

\newunicodechar{Φ}{\Phi}
\newunicodechar{Ψ}{\Psi}

\newunicodechar{Ω}{\Omega}

\newunicodechar{⊢}{\vdash}

\newunicodechar{⊕}{\oplus}
\newunicodechar{¬}{\neg}

\newunicodechar{⊗}{\otimes}
\newunicodechar{→}{\rightarrow}
\newunicodechar{←}{\leftarrow}

\newunicodechar{⊥}{\bot}
\newunicodechar{⊤}{\top}

\setlength{\parindent}{0em}

\author{James Martin, Ian D.L.N. Mclean}
\title{Multiplicative Lambek Sequent Calculus}

\begin{document}

\maketitle

\begin{abstract}

\end{abstract}

\section{Structural Rules}
Uses a variation of cut left.

\begin{center}
	\[
	\begin{prooftree}
	\infer0[Id]{A ⊢ A}
	\end{prooftree}
	\]

	\[
	\begin{prooftree}
	\hypo{Γ ⊢ A}
	\hypo{Γ_{L}, A, Γ_{R} ⊢ C}
	\infer2[CutL]{Γ_{L}, Γ, Γ_{R} ⊢ C}
	\end{prooftree}
	\]
\end{center}

\section{Operational Rules}
$Γ$ non-empty in the RHS implication and negation rules.
\begin{center}

	\[
	\begin{prooftree}
	\hypo{Γ_{L}, A, B, Γ_{R} ⊢ C}
	\infer1{Γ_{L}, A ⊗ B, Γ_{R} ⊢ C}
	\end{prooftree}
	\quad
	\begin{prooftree}
	\hypo{Γ_{L} ⊢ A}
	\hypo{Γ_{R} ⊢ B}
	\infer2{Γ_{L}, Γ_{R} ⊢ A ⊗ B}
	\end{prooftree}
	\]

	\[
	\begin{prooftree}
	\hypo{Γ ⊢ A}
	\hypo{Γ_{L}, B, Γ_{R} ⊢ C}
	\infer2{Γ_{L}, Γ, A → B, Γ_{R} ⊢ C}
	\end{prooftree}
	\quad
	\begin{prooftree}
	\hypo{A, Γ ⊢ B}
	\infer1{Γ ⊢ A → B}
	\end{prooftree}
	\]

	\[
	\begin{prooftree}
	\hypo{Γ ⊢ A}
	\hypo{Γ_{L}, B, Γ_{R} ⊢ C}
	\infer2{Γ_{L}, B ← A, Γ, Γ_{R} ⊢ C}
	\end{prooftree}
	\quad
	\begin{prooftree}
	\hypo{Γ, A ⊢ B}
	\infer1{Γ ⊢ B ← A}
	\end{prooftree}
	\]

	\[
	\begin{prooftree}
	\hypo{ Γ ⊢ A}
	\infer1{ ¬_{L} A, Γ ⊢ }
	\end{prooftree}
	\quad
	\begin{prooftree}
	\hypo{ Γ ⊢ A}
	\infer1{ Γ, ¬_{R} A ⊢ }
	\end{prooftree}
	\quad
	\begin{prooftree}
	\hypo{ A, Γ ⊢_{L} }
	\infer1{ Γ ⊢_{L} ¬ A}
	\end{prooftree}
	\quad
	\begin{prooftree}
	\hypo{ Γ, A ⊢_{R} }
	\infer1{ Γ ⊢_{R} ¬ A}
	\end{prooftree}
	\]
\end{center}

\section{Theorems}
\begin{center}

	\[
	\begin{prooftree}
	\infer0{A ⊢ A}
	\infer1{A , ¬ A ⊢}
	\infer1{A ⊗ ¬ A ⊢}
	\end{prooftree}
	\quad
	\begin{prooftree}
	\infer0{A ⊢ A}
	\infer1{¬ A, A ⊢}
	\infer1{¬ A ⊗ A⊢}
	\end{prooftree}
	\]
\end{center}

\end{document}


\chapter{Moebius Logic}
}
	
	\maketitle
	
	\section{The Semantic Space: The Complex Projective Line ($\mathbb{CP}^1$)}
	
	The Complex Projective Line, denoted $\mathbb{CP}^1$, serves as the semantic space for our proposed non-bivalent logic. It is defined as the set of equivalence classes of pairs of complex numbers $[z_0, z_1] \in \mathbb{C}^2 \setminus \{(0,0)\}$, where $[z_0, z_1] \sim [w_0, w_1]$ if $z_0 = \lambda w_0$ and $z_1 = \lambda w_1$ for some $\lambda \in \mathbb{C} \setminus \{0\}$.
	
	Equivalently, $\mathbb{CP}^1$ can be viewed as the extended complex plane $\mathbb{C} \cup \{\infty\}$, where the equivalence class $[0, 1]$ corresponds to the point at infinity, $\infty$. Geometrically, $\mathbb{CP}^1$ is homeomorphic to the 2-sphere, often called the Riemann sphere.
	
	The automorphisms of $\mathbb{CP}^1$ are the M\"{o}bius transformations, functions of the form $f(z) = \frac{az+b}{cz+d}$ with $ad-bc \neq 0$, where $a, b, c, d \in \mathbb{C}$. These transformations form a group under composition and are generated by translations ($z \mapsto z+b$), dilations/rotations ($z \mapsto az$), and inversion ($z \mapsto 1/z$).
	
	\section{Möbius Logic}
	
	Our proposed M\"{o}bius logic is a non-bivalent logic defined directly on the semantic space $\mathbb{CP}^1$. Its fundamental logical operations are inspired by the generators of the M\"{o}bius group.
	
	\subsection{Semantic Space}
	The semantic space is $\mathbb{CP}^1 = \mathbb{C} \cup \{\infty\}$.
	
	\subsection{Logical Operations}
	For $z, z_1, z_2 \in \mathbb{CP}^1$:
	\begin{itemize}
		\item \textbf{Negation ($\neg$):} Inversion, $\neg z = 1/z$.
		\item \textbf{Conjunction ($\wedge$):} Multiplication, $z_1 \wedge z_2 = z_1 \cdot z_2$.
		\item \textbf{Disjunction ($\vee$):} Addition, $z_1 \vee z_2 = z_1 + z_2$.
	\end{itemize}
	These operations are extended to include $\infty$ with standard rules from complex analysis: $z+\infty = \infty$ for $z \neq \infty$, $\infty+\infty=\infty$, $z \cdot \infty = \infty$ for $z \neq 0, \infty$, $0 \cdot \infty = 0$, $\infty \cdot \infty = \infty$, $1/0 = \infty$, $1/\infty = 0$.
	
	\subsection{Properties of Möbius Logic Operations}
	\begin{itemize}
		\item \textbf{Commutativity:} Both \(\wedge\) and \(\vee\) are commutative (\(z_1 \wedge z_2 = z_2 \wedge z_1\), \(z_1 \vee z_2 = z_2 \vee z_1\)).
		\item \textbf{Associativity:} Both \(\wedge\) and \(\vee\) are associative (\((z_1 \wedge z_2) \wedge z_3 = z_1 \wedge (z_2 \wedge z_3)\), \((z_1 \vee z_2) \vee z_3 = z_1 \vee (z_2 \vee z_3)\)).
		\item \textbf{Idempotency:}
		\begin{itemize}
			\item \(\wedge\) is not generally idempotent; the identity \(z \wedge z = z \cdot z = z\) holds only for \(z \in \{0, 1, \infty\}\).
			\item \(\vee\) is not generally idempotent; the identity \(z \vee z = z + z = z\) holds only for \(z \in \{0, \infty\}\).
		\end{itemize}
		\item \textbf{Involution:} \(\neg\) is an involution (\(\neg \neg z = 1/(1/z) = z\)).
		\item \textbf{Fixed Points of Negation:} The fixed points of negation (\(\neg z = z\)) are the solutions to \(1/z = z \implies z^2 = 1\), which are \(\{1, -1\}\).
		\item \textbf{De Morgan's Laws:} Standard De Morgan's Laws generally fail (e.g., \(\neg(z_1 \wedge z_2) = 1/(z_1 z_2)\) is not generally equal to \(\neg z_1 \vee \neg z_2 = 1/z_1 + 1/z_2\)).
	\end{itemize}
	
	\subsection{LNC and LEM Fixed Points in Möbius Logic}
	We analyze the fixed points of the identities corresponding to the Law of Non-Contradiction (LNC) and the Law of Excluded Middle (LEM) using our defined operations:
	\begin{itemize}
		\item \textbf{Law of Non-Contradiction (LNC):} The identity is \(\neg z \wedge z = z\), which is \((1/z) \cdot z = z\).
		For \(z \in \mathbb{C}, z \neq 0\), this simplifies to \(1 = z\), giving the solution \(z=1\).
		For \(z=0\), \(\neg 0 \wedge 0 = \infty \wedge 0 = 0\), which equals \(z=0\). So \(z=0\) is a fixed point.
		For \(z=\infty\), \(\neg \infty \wedge \infty = 0 \wedge \infty = 0\), which must equal \(z=\infty\). This is not a fixed point.
		The fixed points for the LNC identity are \(\{0, 1\}\).
		\item \textbf{Law of Excluded Middle (LEM):} The identity is \(\neg z \vee z = z\), which is \((1/z) + z = z\).
		For \(z \in \mathbb{C}, z \neq 0\), this simplifies to \(1/z = 0\), which has no solution in \(\mathbb{C}\).
		For \(z=0\), \(\neg 0 \vee 0 = \infty \vee 0 = \infty\), which must equal \(z=0\). This is not a fixed point.
		For \(z=\infty\), \(\neg \infty \vee \infty = 0 \vee \infty = \infty\), which equals \(z=\infty\). So \(z=\infty\) is a fixed point.
		The fixed points for the LEM identity are \(\{\infty\}\).
	\end{itemize}
	The LNC and LEM identities hold only for specific, discrete points in \(\mathbb{CP}^1\). This asymmetry in fixed points is a notable characteristic of this logic.
	
	\section{Multiplicative Linear Logic (MLL)}
	
	Multiplicative Linear Logic (MLL) is a fragment of Linear Logic characterized by its focus on resource-sensitive connectives and the absence of structural rules of weakening and contraction for its multiplicative connectives.
	
	\subsection{Syntax (Sequent Calculus)}
	MLL typically uses a two-sided sequent calculus with formulas built from propositional variables, linear negation (\(\neg\)), multiplicative conjunction (\(\otimes\)), and multiplicative disjunction (\(\wp\)).
	
	\subsection{Semantics (Categorical)}
	Standard semantics for MLL are given in \textbf{star-autonomous categories}. These are symmetric monoidal categories equipped with a strong duality (the star operation) that interprets linear negation.
	
	\subsection{Properties of MLL Connectives}
	\begin{itemize}
		\item \textbf{Commutativity:} \(\otimes\) and \(\wp\) are commutative.
		\item \textbf{Associativity:} \(\otimes\) and \(\wp\) are associative.
		\item \textbf{Idempotency:} \(\otimes\) and \(\wp\) are \textbf{not idempotent}. The identities \(A \otimes A \equiv A\) and \(A \wp A \equiv A\) are not theorems.
		\item \textbf{Involution:} \(\neg\) is an involution (\(\neg \neg A \equiv A\)).
		\item \textbf{Fixed Points of Negation:} In standard semantics, linear negation typically does not have fixed points (objects \(A\) where \(\neg A \cong A\)) for non-trivial objects.
		\item \textbf{De Morgan's Laws:} Standard De Morgan's Laws \textbf{hold}: \(\neg(A \otimes B) \equiv \neg A \wp \neg B\) and \(\neg(A \wp B) \equiv \neg A \otimes \neg B\).
	\end{itemize}
	
	\subsection{LNC and LEM Behavior in MLL}
	\begin{itemize}
		\item \textbf{Law of Non-Contradiction (LNC):} The formula is \(A \otimes \neg A\). In standard semantics, \(A \otimes \neg A\) is \textbf{not universally isomorphic to the unit object I} (which represents a tautology). MLL is paraconsistent.
		\item \textbf{Law of Excluded Middle (LEM):} The formula is \(A \wp \neg A\). In standard semantics, \(A \wp \neg A\) is \textbf{not universally isomorphic to the unit object I}. MLL is paracomplete.
	\end{itemize}
	Syntactically, the sequents \(A \otimes \neg A \vdash\) and \(\vdash A \wp \neg A\) are provable in MLL, but the identities \(A \otimes \neg A \equiv I\) and \(A \wp \neg A \equiv I\) are not universal.
	
	\section{Möbius Logic, Self-Reference, and the Liar Paradox}
	
	Drawing upon the work of Paola Zizzi on turning the Liar Paradox into a metatheorem of Basic Logic, we analyze the implications of our M\"{o}bius logic's properties for the formalization of self-reference.
	
	\subsection{Standard vs. Generalized Self-Reference}
	
	The classical Liar Paradox, typically formulated as ``This sentence is false,'' poses a significant challenge to standard logical systems. Zizzi argues that this paradox arises when self-reference is formalized using the **standard definition of self-reference** within **structural logics**.
	
	The standard definition of self-reference for a sentence \(S\) in a metalanguage \(L_m\) is given by:
	\[ S :\equiv F(\text{``}S\text{''}) \]
	where \(F\) is a function from the object-language \(L_0\) (containing names of sentences) to the metalanguage \(L_m\), and \(\text{``}S\text{''}\) is the name of \(S\) in \(L_0\). When \(F\) is the function \(\neg True\), this leads to the Liar sentence \(L :\equiv \neg True(\text{``}L\text{''})\). By Tarski's definition of truth (\(True(\text{``}A\text{''}) \equiv A\)), this equivalence becomes \(L \equiv \neg L\), the classical paradox. This paradox is unavoidable in structural logics because they possess properties (like the universal idempotence of connectives) that force this equivalence.
	
	Zizzi proposes a **generalized definition of self-reference** that is applicable in substructural logics:
	\[ S :\equiv F(f(\text{``}S\text{''})) \]
	Here, \(f\) is a function \(f: L_0 \to L_0\) defined as \(f(\text{``}S\text{''}) = \text{``}S\text{''} \bullet \text{``}S\text{''}\), where \(\bullet\) is a binary connective in the object-language used for duplicating the name of the sentence.
	
	Zizzi distinguishes two cases based on the property of the connective \(\bullet\):
	
	\begin{enumerate}
		\item \textbf{Case 1: \(\bullet\) is Idempotent.} If the connective \(\bullet\) is idempotent (\(A \bullet A = A\) for all formulas \(A\)), then the function \(f\) has a fixed point for all inputs: \(f(\text{``}S\text{''}) = \text{``}S\text{''})\) (Equation 17 in Zizzi's paper). In this case, the generalized definition \(S :\equiv F(f(\text{``}S\text{''}))\) reduces to the standard definition \(S :\equiv F(\text{``}S\text{''})\) (Equation 12). This is what happens in structural logics, and it allows the Liar paradox \(L \equiv \neg L\) to arise.
		
		\item \textbf{Case 2: \(\bullet\) is Non-Idempotent.} If the connective \(\bullet\) is non-idempotent (\(A \bullet A \neq A\) for some formula \(A\)), then the function \(f\) generally does not have a fixed point: \(f(\text{``}S\text{''}) = \text{``}\widehat{S}\text{''}) \neq \text{``}S\text{''}\) (Equation 18). In this case, the generalized definition becomes \(S :\equiv F(\text{``}\widehat{S}\text{''}))\) (Equation 19). For the Liar sentence, this leads to \(L \equiv \neg True(\text{``}\widehat{L}\text{''}))\), which by Tarski's definition becomes \(L \equiv \neg \widehat{L}\) (Equation 22). Since \(\widehat{L}\) is the name of a sentence different from \(L\), this is not a paradox. This scenario is characteristic of substructural logics with non-idempotent connectives.
	\end{enumerate}
	
	\subsection{Necessary and Sufficient Conditions for Standard Self-Reference}
	
	Zizzi further provides conditions under which the standard definition of self-reference (and thus the potential for paradox) is present:
	
	\begin{itemize}
		\item \textbf{Necessary Condition:} For the standard definition of self-reference to apply universally (for all sentences), the connective \(\bullet\) used for duplication in the object-language must be universally idempotent. The existence of a fixed point for the function \(f\) for all relevant inputs is a necessary condition for the standard definition to hold.
		
		\item \textbf{Sufficient Condition:} Even if the connective \(\bullet\) is non-idempotent in the object-language, the standard definition of self-reference can still be recovered if the corresponding physical link in the metalanguage allows for cloning of the relevant states (e.g., the ability to clone basis states in the quantum mechanical formalism, as shown in Equation 26 using the CNOT gate). This ability to clone in the metalanguage can, in certain cases, force a diagram to commute (Diagram 4 and Equation 28 in Zizzi's paper), effectively recovering the standard definition of self-reference for those specific states.
	\end{itemize}
	
	\subsection{Locally Tarskian Self-Reference}
	
	Based on our analysis of the M\"{o}bius logic on \(\mathbb{CP}^1\) and its non-universal idempotent connectives, we introduce the notion of \textbf{locally Tarskian self-reference}.
	
	\begin{itemize}
		\item \textbf{Definition:} Locally Tarskian self-reference is the phenomenon where the standard definition of self-reference, which can lead to Tarski-like paradoxes (such as the Liar paradox \(L \equiv \neg L\)), holds only for specific, localized regions or points within a semantic space, rather than universally for all possible semantic values.
		\item \textbf{Mechanism in Möbius Logic:} In our M\"{o}bius logic, the connectives (multiplication and addition) are not universally idempotent. Idempotency holds only for specific, discrete points in \(\mathbb{CP}^1\) (\{0, 1, \(\infty\)\} for multiplication, \{0, \(\infty\)\} for addition). These are points where the duplication function \(f\) *can* have a fixed point.
		\item \textbf{Implication:} For sentences whose semantic values fall within these localized fixed-point sets for idempotency, Zizzi's necessary condition for the standard definition of self-reference is met. This is where standard, potentially paradoxical self-reference can occur. For semantic values outside these localized regions, the non-idempotency of the connectives ensures that the generalized definition of self-reference applies, preventing the classical paradox.
	\end{itemize}
	Thus, in a logic with locally Tarskian self-reference, the problematic aspects of self-reference are confined to specific, structurally significant parts of the semantic space, rather than pervading the entire logic as they would in a structural logic.
	
	\subsection{Implications for Möbius Logic on \(\mathbb{CP}^1\)}
	
	Our M\"{o}bius logic on \(\mathbb{CP}^1\) uses complex multiplication (\(\wedge\)) and addition (\(\vee\)) as binary connectives, and inversion (\(\neg\)) as negation. We analyze the implications of Zizzi's framework using these operations as the potential connective \(\bullet\) for duplication in the self-reference construction.
	
	\begin{itemize}
		\item \textbf{Non-Idempotence:} The connectives \(\wedge\) (multiplication) and \(\vee\) (addition) in our M\"{o}bius logic are \textbf{not generally idempotent}. While they have fixed points for idempotency at specific points (\{0, 1, \(\infty\)\} for \(\wedge\), \{0, \(\infty\)\} for \(\vee\)), they do not satisfy the identity \(z \bullet z = z\) for all \(z \in \mathbb{CP}^1\). This aligns with Case 2 of Zizzi's generalized definition for most semantic values in \(\mathbb{CP}^1\).
		
		\item \textbf{Partial Idempotency:} The existence of specific fixed points for idempotency (\{0, 1, \(\infty\)\} for multiplication, \{0, \(\infty\)\} for addition) means that for sentences whose semantic values are precisely these points, the duplication operation (\(z \bullet z\)) *does* return the original value (\(z\)). For these specific semantic values, the function \(f\) *does* have a fixed point, and Zizzi's necessary condition for the standard definition of self-reference is met for these localized cases.
		
		\item \textbf{Necessary Condition in Möbius Logic:} The necessary condition for the standard definition of self-reference to apply is met \textbf{only for sentences whose semantic values are in the fixed point sets for idempotency}. For other semantic values in \(\mathbb{CP}^1\), the non-idempotence ensures that the function \(f\) lacks a fixed point (\(f(\text{``}S\text{''}) \neq \text{``}S\text{''})\)), and the generalized definition applies, leading to \(L \equiv \neg \widehat{L}\), which avoids the classical paradox.
		
		\item \textbf{Sufficient Condition and Cloning:} \(\mathbb{CP}^1\) is the space of pure states of a 2-level quantum system. The semantic values 0 and 1 correspond to clonable basis states (\(|0\rangle\) and \(|1\rangle\)) in the related quantum mechanical metalanguage. This suggests that Zizzi's sufficient condition for recovering the standard definition of self-reference \textbf{appears to be met for semantic values 0 and 1}. The ability to clone these corresponding quantum states in the metalanguage provides the "physical link" that could allow the standard definition of self-reference to apply for propositions with these specific semantic values. The status of \(\infty\) in relation to cloning of specific states would require further investigation.
	\end{itemize}
	
	\subsection{Conclusion on Self-Reference in Möbius Logic}
	
	The properties of our M\"{o}bius logic operations on \(\mathbb{CP}^1\) indicate that this semantic space accommodates both generalized and standard self-reference, with the latter being localized to specific points.
	
	\begin{itemize}
		\item For the vast majority of semantic values in \(\mathbb{CP}^1\), the non-idempotency of the connectives ensures that self-reference is formalized via Zizzi's generalized definition, preventing the classical Liar Paradox \(L \equiv \neg L\).
		\item However, specifically for self-referential statements whose semantic values are the fixed points for idempotency (\{0, 1, \(\infty\)\} for multiplication, \{0, \(\infty\)\} for addition), the necessary condition for the standard definition is met. Furthermore, the connection to the clonability of corresponding quantum states suggests the sufficient condition might also be met for points corresponding to clonable states (like 0 and 1).
	\end{itemize}
	This implies that the classical Liar Paradox, in its most problematic form, might not arise universally in M\"{o}bius logic but could potentially be localized to self-referential statements whose semantic values are precisely the fixed points for idempotency and correspond to clonable states. \(\mathbb{CP}^1\) thus provides a semantic framework where the behavior of self-reference is nuanced, depending on the specific semantic value involved, aligning the logic with substructural principles while acknowledging the unique properties of its complex semantic space and its connection to quantum information.
	
	\section{Specialized Semantic Spaces and Related Logics}
	
	Based on the properties of our M\"{o}bius logic on $\mathbb{CP}^1$, particularly the discrete fixed points for idempotency and the Law of Non-Contradiction, we can explore specialized semantic spaces that are subsets of $\mathbb{CP}^1$ and define logics on them. These specialized logics can help us understand the structure of logics related to the M\"{o}bius logic.
	
	Given that the M\"{o}bius Logic on $\mathbb{CP}^1$ is likely expressive enough to interpret arithmetic and thus potentially essentially undecidable, standard decidable logics like Classical Propositional Logic (LK) on a finite semantic space like $\\{0, 1\\}$ cannot be a conservative extension of it. Instead, such simpler logics on specialized semantic spaces represent reductions or specific models of fragments of the M\"{o}bius logic.
	
	Let's consider an **infinite, proper subset of $\mathbb{CP}^1$** as a specialized semantic space. The **extended real line ($\mathbb{R} \cup \{\infty\}$)** is a natural candidate as it includes key fixed points (0, 1, -1, $\infty$) and is a geometrically simpler, yet still infinite, subset of $\mathbb{CP}^1$.
	
	\subsection{Logic on the Extended Real Line ($\mathbb{R} \cup \{\infty\}$)}
	
	Let the specialized semantic space be $S'' = \mathbb{R} \cup \{\infty\} \subset \mathbb{CP}^1$. We can define a logic on this space using the same M\"{o}bius operations restricted to real numbers and infinity.
	
	\begin{itemize}
		\item \textbf{Semantic Space:} $\mathbb{R} \cup \{\infty\}$.
		\item \textbf{Negation ($\neg$):} Inversion, $\neg x = 1/x$. This is well-defined on $\mathbb{R} \cup \{\infty\}$, with $\neg 0 = \infty$ and $\neg \infty = 0$. Fixed points are $\{1, -1\}$, which are in $\mathbb{R}$.
		\item \textbf{Conjunction ($\wedge$):} Multiplication, $x_1 \wedge x_2 = x_1 \cdot x_2$. This is standard real multiplication extended to $\infty$. $x \cdot \infty = \infty$ for $x \neq 0$, $0 \cdot \infty = 0$, $\infty \cdot \infty = \infty$.
		\item \textbf{Disjunction ($\vee$):} Addition, $x_1 \vee x_2 = x_1 + x_2$. This is standard real addition extended to $\infty$. $x + \infty = \infty$ for $x \neq \infty$, $\infty + \infty = \infty$.
	\end{itemize}
	
	\subsection{Properties and Relationship to Möbius Logic}
	
	This logic on the extended real line shares many properties with the M\"{o}bius logic on $\mathbb{CP}^1$ but is restricted to a specific subset of the complex plane.
	
	\begin{itemize}
		\item \textbf{Commutativity and Associativity:} Both $\wedge$ and $\vee$ are commutative and associative on $\mathbb{R} \cup \{\infty\}$, as they are on $\mathbb{CP}^1$.
		\item \textbf{Idempotency:}
		\begin{itemize}
			\item \(\wedge\) is not generally idempotent; the identity \(x \wedge x = x \cdot x = x\) holds only for \(x \in \{0, 1, \infty\}\). These are the same fixed points as on $\mathbb{CP}^1$.
			\item \(\vee\) is not generally idempotent; the identity \(x \vee x = x + x = x\) holds only for \(x \in \{0, \infty\}\). These are the same fixed points as on $\mathbb{CP}^1$.
		\end{itemize}
		\item \textbf{Involution Negation:} $\neg x = 1/x$ is an involution on $\mathbb{R} \cup \{\infty\}$.
		\item \textbf{Fixed Points of Negation:} Fixed points are $\{1, -1\}$, which are in $\mathbb{R}$.
		\item \textbf{De Morgan's Laws:} Standard De Morgan's Laws generally fail on $\mathbb{R} \cup \{\infty\}$, just as they do on $\mathbb{CP}^1$.
		\item \textbf{LNC and LEM Fixed Points:}
		\begin{itemize}
			\item \textbf{Law of Non-Contradiction (LNC):} Identity \(\neg x \wedge x = x\). \((1/x) \cdot x = x\). Fixed points are $\{0, 1\}$. These are the same fixed points as on $\mathbb{CP}^1$.
			\item \textbf{Law of Excluded Middle (LEM):} Identity \(\neg x \vee x = x\). \((1/x) + x = x\). Fixed points are $\{\infty\}$. These are the same fixed points as on $\mathbb{CP}^1$.
		\end{itemize}
	\end{itemize}
	
	This logic on the extended real line is a **specialization** of the M\"{o}bius logic on $\mathbb{CP}^1$ in terms of its semantic space. It shares the core properties of non-idempotency, involutive negation, and the discrete fixed points for LNC and LEM. It is likely **not a conservative extension** in the standard sense, as restricting the semantic space might prevent some theorems of the full M\"{o}bius logic from holding universally (or introduce new ones). However, it represents a system with the same operational structure applied to a geometrically and algebraically simpler infinite subset.
	
	The relationship between the M\"{o}bius logic on $\mathbb{CP}^1$ and this logic on $\mathbb{R} \cup \{\infty\}$ could be explored through embeddings or restrictions, potentially revealing a different kind of categorical relationship than a simple initial/terminal one. This specialized infinite semantic space provides a valuable intermediate step between the full complexity of $\mathbb{CP}^1$ and finite semantic spaces.
	
	\section{Comparison and Relationship with MLL}
	
	Comparing M\"{o}bius logic and MLL reveals significant similarities and divergences:
	
	\subsection{Points of Correspondence}
	\begin{itemize}
		\item Both are \textbf{commutative} and \textbf{associative}.
		\item Both are \textbf{non-idempotent} (though M\"{o}bius logic has partial idempotency at specific points).
		\item Both have an \textbf{involutive negation}.
		\item Both are \textbf{paraconsistent} (LNC not universally valid) and \textbf{paracomplete} (LEM not universally valid).
	\end{itemize}
	
	\subsection{Points of Divergence}
	\begin{itemize}
		\item \textbf{Semantic Space:} M\"{o}bius logic has a set-based semantic space (\(\mathbb{CP}^1\)), while MLL has categorical semantics (star-autonomous categories).
		\item \textbf{Idempotency:} M\"{o}bius logic operations have specific fixed points for idempotency (\{0, 1, \(\infty\)\} for \(\wedge\), \{0, \(\infty\)\} for \(\vee\)), while MLL is radically non-idempotent.
		\item \textbf{Negation Fixed Points:} M\"{o}bius negation has fixed points (\{1, -1\}), while standard MLL negation typically does not have non-trivial fixed points in its standard semantics.
		\item \textbf{De Morgan's Laws:} De Morgan's Laws hold in MLL but generally fail in M\"{o}bius logic.
		\item \textbf{LNC/LEM Behavior:} In M\"{o}bius logic, the identities \(\neg z \wedge z = z\) and \(\neg z \vee z = z\) hold only for specific points. In MLL semantics, corresponding formulas \(A \otimes \neg A\) and \(A \wp \neg A\) are not universally tautologous objects, but the sequents $A \otimes \neg A \vdash$ and $\vdash A \wp \neg A$ are provable.
	\end{itemize}
	
	\subsection{Relationship}
	The divergences suggest that M\"{o}bius logic (as defined here) is not a simple conservative extension of standard MLL, nor is MLL a conservative extension of M\"{o}bius logic. The failure of De Morgan's laws in M\"{o}bius logic, which are theorems of MLL, prevents a direct conservative extension from MLL to M\"{o}bius logic under the natural mapping.
	
	The relationship is likely more complex, potentially involving:
	\begin{itemize}
		\item M\"{o}bius logic as a specific model or interpretation of MLL or a fragment thereof.
		\item Both logics being related as extensions of a common, more fundamental non-distributive substructural logic that is radically non-idempotent and lacks standard De Morgan duality.
		\item A non-standard translation or embedding that preserves certain structures but not others.
	\end{itemize}
	The semantic space \(\mathbb{CP}^1\) with its complex arithmetic and geometric structure provides a concrete setting for exploring non-idempotent, paraconsistent, and paracomplete logic, offering a unique perspective compared to the more abstract categorical semantics of MLL.
	
\end{document}}*}


\chapter{Moebius Logic 0}
\documentclass{article}
\usepackage{amsmath}
\usepackage{amssymb}
\usepackage{amsfonts}
\usepackage{amsthm}
\usepackage{enumitem}

\title{Möbius Logic, Complex Projective Line, and Multiplicative Linear Logic}
\author{}
\date{\today}

\begin{document}
	
	\maketitle
	
	\section{The Semantic Space: The Complex Projective Line ($\mathbb{CP}^1$)}
	
	The Complex Projective Line, denoted $\mathbb{CP}^1$, serves as the semantic space for our Möbius logic. It is defined as the set of equivalence classes of pairs of complex numbers $[z_0, z_1] \in \mathbb{C}^2 \setminus \{(0,0)\}$, where $[z_0, z_1] \sim [w_0, w_1]$ if $z_0 = \lambda w_0$ and $z_1 = \lambda w_1$ for some $\lambda \in \mathbb{C} \setminus \{0\}$.
	
	Equivalently, $\mathbb{CP}^1$ can be viewed as the extended complex plane $\mathbb{C} \cup \{\infty\}$, where the equivalence class $[0, 1]$ corresponds to the point at infinity, $\infty$. Geometrically, $\mathbb{CP}^1$ is homeomorphic to the 2-sphere (the Riemann sphere).
	
	The automorphisms of $\mathbb{CP}^1$ are the M\"{o}bius transformations, functions of the form $f(z) = \frac{az+b}{cz+d}$ with $ad-bc \neq 0$, where $a, b, c, d \in \mathbb{C}$. These transformations are generated by translations ($z \mapsto z+b$), dilations/rotations ($z \mapsto az$), and inversion ($z \mapsto 1/z$).
	
	\section{Möbius Logic}
	
	Our proposed M\"{o}bius logic is a non-bivalent logic defined on the semantic space $\mathbb{CP}^1$. Its logical operations are directly inspired by the generators of the M\"{o}bius group.
	
	\subsection{Semantic Space}
	The semantic space is $\mathbb{CP}^1 = \mathbb{C} \cup \{\infty\}$.
	
	\subsection{Logical Operations}
	For $z, z_1, z_2 \in \mathbb{CP}^1$:
	\begin{itemize}
		\item \textbf{Negation ($\neg$):} Inversion, $\neg z = 1/z$.
		\item \textbf{Conjunction ($\wedge$):} Multiplication, $z_1 \wedge z_2 = z_1 \cdot z_2$.
		\item \textbf{Disjunction ($\vee$):} Addition, $z_1 \vee z_2 = z_1 + z_2$.
	\end{itemize}
	These operations are extended to $\infty$ with standard rules: $z+\infty = \infty$ for $z \neq \infty$, $\infty+\infty=\infty$, $z \cdot \infty = \infty$ for $z \neq 0, \infty$, $0 \cdot \infty = 0$, $\infty \cdot \infty = \infty$, $1/0 = \infty$, $1/\infty = 0$.
	
	\subsection{Properties of Möbius Logic Operations}
	\begin{itemize}
		\item \textbf{Commutativity:} Both $\wedge$ and $\vee$ are commutative ($z_1 \wedge z_2 = z_2 \wedge z_1$, $z_1 \vee z_2 = z_2 \vee z_1$).
		\item \textbf{Associativity:} Both $\wedge$ and $\vee$ are associative ($(z_1 \wedge z_2) \wedge z_3 = z_1 \wedge (z_2 \wedge z_3)$, $(z_1 \vee z_2) \vee z_3 = z_1 \vee (z_2 \vee z_3)$).
		\item \textbf{Idempotency:}
		\begin{itemize}
			\item $\wedge$ is not generally idempotent ($z \wedge z = z \cdot z = z$ only for $z=0, 1$).
			\item $\vee$ is not generally idempotent ($z \vee z = z + z = z$ only for $z=0$).
		\end{itemize}
		\item \textbf{Involution:} $\neg$ is an involution ($\neg \neg z = 1/(1/z) = z$).
		\item \textbf{Fixed Points of Negation:} $\neg z = z$ holds for $1/z = z \implies z^2 = 1$, so fixed points are $\{1, -1\}$.
		\item \textbf{De Morgan's Laws:} Standard De Morgan's Laws generally fail (e.g., $\neg(z_1 \wedge z_2) = 1/(z_1 z_2)$ is not generally equal to $\neg z_1 \vee \neg z_2 = 1/z_1 + 1/z_2$).
	\end{itemize}
	
	\subsection{LNC and LEM Fixed Points in Möbius Logic}
	\begin{itemize}
		\item \textbf{Law of Non-Contradiction (LNC):} Identity $\neg z \wedge z = z$.
		$(1/z) \cdot z = z$. Fixed points are $\{0, 1\}$.
		\item \textbf{Law of Excluded Middle (LEM):} Identity $\neg z \vee z = z$.
		$(1/z) + z = z$. Fixed points are $\{\infty\}$.
	\end{itemize}
	LNC and LEM identities hold only for specific, discrete points in $\mathbb{CP}^1$.
	
	\section{Multiplicative Linear Logic (MLL)}
	
	Multiplicative Linear Logic (MLL) is a fragment of Linear Logic characterized by its focus on resource-sensitive connectives and the absence of structural rules of weakening and contraction for its multiplicative connectives.
	
	\subsection{Syntax (Sequent Calculus)}
	MLL typically uses a two-sided sequent calculus with formulas built from propositional variables, linear negation ($\neg$), multiplicative conjunction ($\otimes$), and multiplicative disjunction ($\wp$).
	
	\subsection{Semantics (Categorical)}
	Standard semantics for MLL are given in \textbf{star-autonomous categories}. These are symmetric monoidal categories equipped with a strong duality (the star operation) that interprets linear negation.
	
	\subsection{Properties of MLL Connectives}
	\begin{itemize}
		\item \textbf{Commutativity:} $\otimes$ and $\wp$ are commutative.
		\item \textbf{Associativity:} $\otimes$ and $\wp$ are associative.
		\item \textbf{Idempotency:} $\otimes$ and $\wp$ are \textbf{not idempotent}. The identities $A \otimes A \equiv A$ and $A \wp A \equiv A$ are not theorems.
		\item \textbf{Involution:} $\neg$ is an involution ($\neg \neg A \equiv A$).
		\item \textbf{Fixed Points of Negation:} In standard semantics, linear negation typically does not have fixed points (objects $A$ where $\neg A \cong A$) for non-trivial objects.
		\item \textbf{De Morgan's Laws:} Standard De Morgan's Laws \textbf{hold}: $\neg(A \otimes B) \equiv \neg A \wp \neg B$ and $\neg(A \wp B) \equiv \neg A \otimes \neg B$.
	\end{itemize}
	
	\subsection{LNC and LEM Behavior in MLL}
	\begin{itemize}
		\item \textbf{Law of Non-Contradiction (LNC):} Formula $A \otimes \neg A$.
		In standard semantics, $A \otimes \neg A$ is \textbf{not universally isomorphic to the unit object I} (tautology). MLL is paraconsistent.
		\item \textbf{Law of Excluded Middle (LEM):} Formula $A \wp \neg A$.
		In standard semantics, $A \wp \neg A$ is \textbf{not universally isomorphic to the unit object I} (tautology). MLL is paracomplete.
	\end{itemize}
	Syntactically, the sequents $A \otimes \neg A \vdash$ and $\vdash A \wp \neg A$ are provable in MLL.
	
	\section{Comparison and Relationship}
	
	Comparing M\"{o}bius logic and MLL reveals significant similarities and divergences:
	
	\subsection{Points of Correspondence}
	\begin{itemize}
		\item Both are \textbf{commutative} and \textbf{associative}.
		\item Both are \textbf{non-idempotent} (though M\"{o}bius logic has partial idempotency at specific points).
		\item Both have an \textbf{involutive negation}.
		\item Both are \textbf{paraconsistent} (LNC not universal) and \textbf{paracomplete} (LEM not universal).
	\end{itemize}
	
	\subsection{Points of Divergence}
	\begin{itemize}
		\item \textbf{Semantic Space:} M\"{o}bius logic has a set-based semantic space ($\mathbb{CP}^1$), while MLL has categorical semantics (star-autonomous categories).
		\item \textbf{Idempotency:} M\"{o}bius logic operations have specific fixed points for idempotency ({0, 1} for $\wedge$, {0} for $\vee$), while MLL is radically non-idempotent.
		\item \textbf{Negation Fixed Points:} M\"{o}bius negation has fixed points ({1, -1}), while standard MLL negation does not.
		\item \textbf{De Morgan's Laws:} De Morgan's Laws hold in MLL but generally fail in M\"{o}bius logic.
		\item \textbf{LNC/LEM Behavior:} In M\"{o}bius logic, LNC/LEM identities hold for specific points. In MLL semantics, corresponding formulas are not universally tautologous objects.
	\end{itemize}
	
	\subsection{Relationship}
	The divergences suggest that M\"{o}bius logic (as defined here) is not a simple conservative extension of standard MLL, nor is MLL a conservative extension of M\"{o}bius logic. The failure of De Morgan's laws in M\"{o}bius logic, which are theorems of MLL, prevents a direct conservative extension from MLL to M\"{o}bius logic under the natural mapping.
	
	The relationship is likely more complex, potentially involving:
	\begin{itemize}
		\item M\"{o}bius logic as a specific model or interpretation of MLL or a fragment thereof.
		\item Both logics being conservative extensions of a common, more fundamental non-distributive substructural logic that is radically non-idempotent and lacks De Morgan duality.
		\item A non-standard translation or embedding that preserves certain structures but not others.
	\end{itemize}
	The semantic space $\mathbb{CP}^1$ with its complex arithmetic structure provides a concrete setting for exploring non-idempotent, paraconsistent, and paracomplete logic, offering a unique perspective compared to the more abstract categorical semantics of MLL.
	
\end{document}


\chapter{Moebius Logic NANDNOR}
}
	
	\maketitle
	
	Based on our defined logical operations in the M\"{o}bius logic on $\mathbb{CP}^1$:
	\begin{itemize}
		\item \textbf{Semantic Space:} $\mathbb{CP}^1 = \mathbb{C} \cup \{\infty\}$
		\item \textbf{Negation ($\neg$):} Inversion, $\neg z = 1/z$
		\item \textbf{Conjunction ($\wedge$):} Multiplication, $z_1 \wedge z_2 = z_1 \cdot z_2$
		\item \textbf{Disjunction ($\vee$):} Addition, $z_1 \vee z_2 = z_1 + z_2$
	\end{itemize}
	We can formalize the concepts of "M\"{o}bius NAND" and "M\"{o}bius NOR" as the negation of the conjunction and disjunction, respectively, analogous to their definitions in classical logic.
	
	\section{Möbius NAND ($N_{MNAND}$)}
	
	The M\"{o}bius NAND operation is defined as the negation of the M\"{o}bius conjunction:
	$$N_{MNAND}(z_1, z_2) = \neg(z_1 \wedge z_2) = \neg(z_1 \cdot z_2) = \frac{1}{z_1 \cdot z_2}$$
	This operation is well-defined on $\mathbb{CP}^1$ with the standard rules for arithmetic involving $\infty$ and $1/0 = \infty$.
	
	\section{Möbius NOR ($N_{MNOR}$)}
	
	The M\"{o}bius NOR operation is defined as the negation of the M\"{o}bius disjunction:
	$$N_{MNOR}(z_1, z_2) = \neg(z_1 \vee z_2) = \neg(z_1 + z_2) = \frac{1}{z_1 + z_2}$$
	This operation is well-defined on $\mathbb{CP}^1$ with the standard rules for arithmetic involving $\infty$ and $1/0 = \infty$.
	
	\section{Properties and Functional Completeness}
	
	Let's examine the properties of these operations and their potential to generate the basic operations of our M\"{o}bius logic ($1/z$, $z_1 \cdot z_2$, $z_1 + z_2$) through composition.
	
	\subsection{Möbius NAND ($N_{MNAND}(z_1, z_2) = \frac{1}{z_1 \cdot z_2}$)}
	
	\begin{itemize}
		\item \textbf{Negation ($\neg z = 1/z$):} Can be generated by fixing one input to 1:
		$$N_{MNAND}(z, 1) = \frac{1}{z \cdot 1} = \frac{1}{z}$$
		This requires the ability to generate the constant 1. We can generate 1:
		$$N_{MNAND}(z, 1/z) = \frac{1}{z \cdot (1/z)} = 1 \quad \text{(for } z \neq 0, \infty)$$
		\item \textbf{Conjunction ($z_1 \cdot z_2$):} Can be generated by negating the output of $N_{MNAND}$:
		$$\neg(N_{MNAND}(z_1, z_2)) = \neg\left(\frac{1}{z_1 \cdot z_2}\right) = \frac{1}{1/(z_1 \cdot z_2)} = z_1 \cdot z_2$$
		This requires the ability to generate negation, which we can do using $N_{MNAND}$ and the constant 1.
		\item \textbf{Disjunction ($z_1 + z_2$):} Generating disjunction using only $N_{MNAND}$ and compositions appears challenging due to the multiplicative and inversive nature of the operation. The classical De Morgan's law equivalence ($p \vee q \equiv \neg(\neg p \wedge \neg q)$) translates to $z_1 + z_2 \equiv \neg(\neg z_1 \wedge \neg z_2)$, which semantically is $z_1 + z_2 = \neg((1/z_1) \cdot (1/z_2)) = z_1 z_2$. This identity does not hold universally.
	\end{itemize}
	M\"{o}bius NAND can generate negation and conjunction (given the constant 1), but not generally disjunction.
	
	\subsection{Möbius NOR ($N_{MNOR}(z_1, z_2) = \frac{1}{z_1 + z_2}$)}
	
	\begin{itemize}
		\item \textbf{Negation ($\neg z = 1/z$):} Can be generated by fixing one input to 0:
		$$N_{MNOR}(z, 0) = \frac{1}{z + 0} = \frac{1}{z}$$
		This requires the ability to generate the constant 0. We can generate 0:
		$$N_{MNOR}(z, \infty) = \frac{1}{z + \infty} = \frac{1}{\infty} = 0 \quad \text{(for } z \neq \infty)$$
		\item \textbf{Disjunction ($z_1 + z_2$):} Can be generated by negating the output of $N_{MNOR}$:
		$$\neg(N_{MNOR}(z_1, z_2)) = \neg\left(\frac{1}{z_1 + z_2}\right) = \frac{1}{1/(z_1 + z_2)} = z_1 + z_2$$
		This requires the ability to generate negation, which we can do using $N_{MNOR}$ and the constant 0.
		\item \textbf{Conjunction ($z_1 \cdot z_2$):} Generating conjunction using only $N_{MNOR}$ and compositions appears challenging due to the additive and inversive nature of the operation. The classical De Morgan's law equivalence ($p \wedge q \equiv \neg(\neg p \vee \neg q)$) translates to $z_1 \cdot z_2 \equiv \neg(\neg z_1 \vee \neg z_2)$, which semantically is $z_1 \cdot z_2 = \neg((1/z_1) + (1/z_2)) = \frac{1}{1/z_1 + 1/z_2} = \frac{z_1 z_2}{z_1 + z_2}$. This identity does not hold universally.
	\end{itemize}
	M\"{o}bius NOR can generate negation and disjunction (given the constant 0), but not generally conjunction.
	
	\section{Conclusion on Functional Completeness}
	
	Based on this analysis, neither $N_{MNAND}(z_1, z_2) = \frac{1}{z_1 \cdot z_2}$ nor $N_{MNOR}(z_1, z_2) = \frac{1}{z_1 + z_2}$ appear to be \textbf{sole-sufficient} operations for generating \textit{all three} of our basic M\"{o}bius logic operations ($1/z$, $z_1 \cdot z_2$, $z_1 + z_2$) through composition alone. They can each generate negation and one of the binary operations, provided certain constants (0 or 1) are available.
	
	This suggests that the set $\{1/z, z_1 \cdot z_2, z_1 + z_2\}$ might be a minimal generating set for the operations of our M\"{o}bius logic, or that a single sole-sufficient operation would need to be a more complex function that inherently combines both multiplicative and additive structures in a fundamental way. The general form of the M\"{o}bius transformation $\frac{az+b}{cz+d}$ encapsulates this combination, suggesting that an operation based on this form might be a candidate for generating a broader class of functions on $\mathbb{CP}^1$.
	
	However, $N_{MNAND}$ and $N_{MNOR}$ are valid formalizations of the negated conjunction and disjunction within our defined M\"{o}bius logic and represent specific binary operations with distinct properties on $\mathbb{CP}^1$.
	
\end{document}}*}


\chapter{OMALL}
}

\maketitle

\begin{abstract}
A sequent system that preserves the law of excluded middle and non-contradiction, but not generally ECQ and EQN?
\end{abstract}

\section{Structural Rules}

\begin{center}
	\[
	\begin{prooftree}
	\infer0[Id]{A ⊢ A}
	\end{prooftree}
	\]
	
	\[
	\begin{prooftree}
	\hypo{Γ ⊢ A}
	\hypo{Δ, A, Π ⊢ C}
	\infer2[CutL]{Δ, Γ, Π ⊢ C}
	\end{prooftree}
	\quad
	\begin{prooftree}
	\hypo{Γ, A, Φ ⊢ C}
	\hypo{C ⊢ Δ, A, Ψ}
	\infer2[Cut]{Γ, Φ ⊢ Δ, Ψ}
	\end{prooftree}
	\quad
	\begin{prooftree}
	\hypo{C ⊢ Π, A, Δ}
	\hypo{A ⊢ Γ}
	\infer2[CutR]{C ⊢ Π, Γ, Δ}
	\end{prooftree}
	\]
\end{center}

\section{Unit Rules}
\begin{center}
		\[
	\begin{prooftree}
	\hypo{Γ_{L}, Γ_{R} ⊢ C}
	\infer1{Γ_{L}, 1, Γ_{R} ⊢ C}
	\end{prooftree}
	\quad
	\begin{prooftree}
	\hypo{C ⊢ Δ_{L}, Δ_{R}}
	\infer1{C ⊢ Δ_{L}, ⊥, Δ_{R}}
	\end{prooftree}
	\]
	
	\[
	\begin{prooftree}
	\infer0{ ⊥ ⊢ }
	\end{prooftree}
	\quad
	\begin{prooftree}
	\infer0{ ⊢ 1}
	\end{prooftree}
	\]
	
	\[
	\begin{prooftree}
	\infer0{Γ_{L}, 0, Γ_{R} ⊢ C}
	\end{prooftree}
	\quad
	\begin{prooftree}
	\infer0{ 0 ⊢ Δ}
	\end{prooftree}
	\quad
	\begin{prooftree}
	\infer0{ Γ ⊢ ⊤}
	\end{prooftree}
	\quad
	\begin{prooftree}
	\infer0{C ⊢ Δ_{L}, ⊤, Δ_{R}}
	\end{prooftree}
	\]
	
	\[
	\begin{prooftree}
	\infer0{Γ_{L}, 0, Γ_{R} ⊢ Δ}
	\end{prooftree}
	\quad
	\begin{prooftree}
	\infer0{Γ ⊢ Δ_{L}, ⊤, Δ_{R}}
	\end{prooftree}
	\]
\end{center}

\newpage
\section{Operational Rules}
\begin{center}
	\subsection{Negations}
	\begin{center}
			\[
		\begin{prooftree}
		\hypo{Γ, Φ ⊢ Δ, A, Ψ}
		\infer1{Γ, ¬ A, Φ ⊢ Δ, Ψ}
		\end{prooftree}
		\quad
		\begin{prooftree}
		\hypo{ Φ, A, Γ ⊢ Ψ, Δ}
		\infer1{ Φ, Γ ⊢ Ψ, ¬ A, Δ}
		\end{prooftree}
		\quad
		\begin{prooftree}
		\hypo{Γ, Φ ⊢ Δ, A, Ψ}
		\infer1{Γ, ¬ A, Φ ⊢ Δ, Ψ}
		\end{prooftree}
		\quad
		\begin{prooftree}
		\hypo{Φ, A, Γ ⊢ Ψ, Δ}
		\infer1{ Φ, Γ ⊢ Ψ, ¬ A, Δ}
		\end{prooftree}
		\]
		
		\[
		\begin{prooftree}
		\hypo{Γ ⊢ Δ, A}
		\infer1{Γ, ¬ A ⊢ Δ}
		\end{prooftree}
		\quad
		\begin{prooftree}
		\hypo{ A, Γ ⊢ Δ}
		\infer1{ Γ ⊢ ¬ A, Δ}
		\end{prooftree}
		\]
		
		\[
		\begin{prooftree}
		\hypo{ Γ ⊢ A}
		\infer1{ ¬ A, Γ ⊢ }
		\end{prooftree}
		\quad
		\begin{prooftree}
		\hypo{ Γ ⊢ A}
		\infer1{ Γ, ¬ A⊢ }
		\end{prooftree}
		\quad
		\begin{prooftree}
		\hypo{ A, Γ ⊢ }
		\infer1{ Γ ⊢ ¬ A}
		\end{prooftree}
		\quad
		\begin{prooftree}
		\hypo{ Γ, A ⊢ }
		\infer1{ Γ ⊢ ¬ A}
		\end{prooftree}
		\]
		
		\[
		\begin{prooftree}
		\hypo{ ⊢ A, Δ}
		\infer1{ ¬ A ⊢ Δ}
		\end{prooftree}
		\quad
		\begin{prooftree}
		\hypo{ ⊢ Δ, A}
		\infer1{ ¬ A⊢ Δ}
		\end{prooftree}
		\quad
		\begin{prooftree}
		\hypo{ A ⊢ Δ}
		\infer1{ ⊢ ¬ A, Δ}
		\end{prooftree}
		\quad
		\begin{prooftree}
		\hypo{ A ⊢ Δ}
		\infer1{ ⊢ Δ, ¬ A}
		\end{prooftree}
		\]
	\end{center}

	\subsection{Additives}
	\begin{center}
		\[
		\begin{prooftree}
		\hypo{Γ, A, Φ ⊢ Δ}
		\hypo{Γ, B, Φ ⊢ Δ}
		\infer2{Γ, A ⊕ B, Φ ⊢ Δ}
		\end{prooftree}
		\quad
		\begin{prooftree}
		\hypo{Γ ⊢ Δ, A, Ψ }
		\infer1{Γ ⊢ Δ, A ⊕ B, Ψ}
		\end{prooftree}
		\quad
		\begin{prooftree}
		\hypo{Γ ⊢ Δ, B, Ψ }
		\infer1{Γ ⊢ Δ, A ⊕ B, Ψ}
		\end{prooftree}
		\]
		
		\[
		\begin{prooftree}
		\hypo{Γ, A, Φ ⊢ Ψ}
		\infer1{Γ, A \& B, Φ ⊢ Ψ}
		\end{prooftree}
		\quad
		\begin{prooftree}
		\hypo{Γ, B, Φ ⊢ Ψ}
		\infer1{Γ, A \& B, Φ ⊢ Ψ}
		\end{prooftree}
		\quad
		\begin{prooftree}
		\hypo{ Γ ⊢ Δ, B, Ψ}
		\hypo{ Γ ⊢ Δ, A, Ψ}
		\infer2{ Γ ⊢ Δ, B \& A, Ψ}
		\end{prooftree}
		\]
		
		\[
		\begin{prooftree}
		\hypo{Γ, A, Φ ⊢ C}
		\hypo{Γ, B, Φ ⊢ C}
		\infer2{Γ, A ⊕ B, Φ ⊢ C}
		\end{prooftree}
		\quad
		\begin{prooftree}
		\hypo{Γ ⊢ A}
		\infer1{Γ ⊢ A ⊕ B}
		\end{prooftree}
		\quad
		\begin{prooftree}
		\hypo{Γ ⊢ B}
		\infer1{Γ ⊢ A ⊕ B}
		\end{prooftree}
		\quad
		\begin{prooftree}
		\hypo{A ⊢ Δ}
		\infer1{A \& B ⊢ Δ}
		\end{prooftree}
		\quad
		\begin{prooftree}
		\hypo{B ⊢ Δ}
		\infer1{A \& B ⊢ Δ}
		\end{prooftree}
		\quad
		\begin{prooftree}
		\hypo{ C ⊢ Ψ, B, Δ}
		\hypo{ C ⊢ Ψ, A, Δ}
		\infer2{ C ⊢ Ψ, B \& A, Δ}
		\end{prooftree}
		\]
		
		\[
		\begin{prooftree}
		\hypo{Γ, A, Φ ⊢ C}
		\infer1{Γ, A \& B, Φ ⊢ C}
		\end{prooftree}
		\quad
		\begin{prooftree}
		\hypo{Γ, B, Φ ⊢ C}
		\infer1{Γ, A \& B, Φ ⊢ C}
		\end{prooftree}
		\quad
		\begin{prooftree}
		\hypo{Γ ⊢ A}
		\hypo{Γ ⊢ B}
		\infer2{Γ ⊢ A \& B}
		\end{prooftree}
		\quad
		\begin{prooftree}
		\hypo{B ⊢ Δ}
		\hypo{A ⊢ Δ}
		\infer2{B ⊕ A ⊢ Δ}
		\end{prooftree}
		\quad
		\begin{prooftree}
		\hypo{C ⊢ Ψ, A, Δ}
		\infer1{C ⊢ Ψ, B ⊕ A, Δ}
		\end{prooftree}
		\quad
		\begin{prooftree}
		\hypo{C ⊢ Ψ, B, Δ}
		\infer1{C ⊢ Ψ, B ⊕ A, Δ}
		\end{prooftree}
		\]
		
		
		\[
		\begin{prooftree}
		\hypo{Γ, Φ ⊢ C, A}
		\hypo{Γ, Φ ⊢ C, B}
		\infer2{Γ, ¬ A ⊕ ¬ B, Φ ⊢ C}
		\end{prooftree}
		\quad
		\begin{prooftree}
		\hypo{A, Γ ⊢ }
		\infer1{Γ ⊢ ¬ A ⊕ ¬ B}
		\end{prooftree}
		\quad
		\begin{prooftree}
		\hypo{B, Γ ⊢ }
		\infer1{Γ ⊢ ¬ A ⊕ ¬ B}
		\end{prooftree}
		\quad
		\begin{prooftree}
		\hypo{⊢ Δ, A}
		\infer1{¬ B \& ¬ A ⊢ Δ}
		\end{prooftree}
		\quad
		\begin{prooftree}
		\hypo{⊢ Δ, B}
		\infer1{¬ B \& ¬ A ⊢ Δ}
		\end{prooftree}
		\quad
		\begin{prooftree}
		\hypo{ B, C ⊢ Ψ, Δ}
		\hypo{ A, C ⊢ Ψ, Δ}
		\infer2{ C ⊢ Ψ, ¬ B \& ¬ A, Δ}
		\end{prooftree}
		\]
		
		\[
		\begin{prooftree}
		\hypo{Γ, Φ ⊢ C, A}
		\infer1{Γ, ¬ A \& ¬ B, Φ ⊢ C}
		\end{prooftree}
		\quad
		\begin{prooftree}
		\hypo{Γ, Φ ⊢ C, B}
		\infer1{Γ, ¬ A \& ¬ B, Φ ⊢ C}
		\end{prooftree}
		\quad
		\begin{prooftree}
		\hypo{A, Γ ⊢ }
		\hypo{B, Γ ⊢ }
		\infer2{Γ ⊢ ¬ A \& ¬ B}
		\end{prooftree}
		\quad
		\begin{prooftree}
		\hypo{⊢ Δ, B}
		\hypo{⊢ Δ, A}
		\infer2{¬ B ⊕ ¬ A ⊢ Δ}
		\end{prooftree}
		\quad
		\begin{prooftree}
		\hypo{A, C ⊢ Ψ, Δ}
		\infer1{C ⊢ Ψ, ¬ B ⊕ ¬ A, Δ}
		\end{prooftree}
		\quad
		\begin{prooftree}
		\hypo{B, C ⊢ Ψ, Δ}
		\infer1{C ⊢ Ψ, ¬ B ⊕ ¬ A, Δ}
		\end{prooftree}
		\]
		
	\end{center}

	\subsection{Multiplicatives}
	\begin{center}
		\[
		\begin{prooftree}
		\hypo{Γ, A, B, Φ ⊢ C}
		\infer1{Γ, A ⊗ B, Φ ⊢ C}
		\end{prooftree}
		\quad
		\begin{prooftree}
		\hypo{Γ ⊢ A}
		\hypo{Φ ⊢ B}
		\infer2{Γ, Φ ⊢ A ⊗ B}
		\end{prooftree}
		\quad
		\begin{prooftree}
		\hypo{B ⊢ Ψ}
		\hypo{A ⊢ Δ}
		\infer2{B ⅋ A ⊢ Ψ, Δ}
		\end{prooftree}
		\quad
		\begin{prooftree}
		\hypo{C ⊢ Ψ, B, A, Δ}
		\infer1{C ⊢ Ψ, B ⅋ A, Δ}
		\end{prooftree}
		\]
		
		\[
		\begin{prooftree}
		\hypo{Γ, B, A, Φ ⊢ C}
		\infer1{Γ, B ⊗ A, Φ ⊢ C}
		\end{prooftree}
		\quad
		\begin{prooftree}
		\hypo{Φ ⊢ B}
		\hypo{Γ ⊢ A}
		\infer2{Φ, Γ ⊢ B ⊗ A}
		\end{prooftree}
		\quad
		\begin{prooftree}
		\hypo{A ⊢ Δ}
		\hypo{B ⊢ Ψ}
		\infer2{A ⅋ B ⊢ Δ, Ψ}
		\end{prooftree}
		\quad
		\begin{prooftree}
		\hypo{C ⊢ Ψ, A, B, Δ}
		\infer1{C ⊢ Ψ, A ⅋ B, Δ}
		\end{prooftree}
		\]
		
		\[
		\begin{prooftree}
		\hypo{Γ ⊢ A}
		\hypo{Φ, B, Θ ⊢ C}
		\infer2{Φ, Γ, A → B, Θ ⊢ C}
		\end{prooftree}
		\quad
		\begin{prooftree}
		\hypo{A, Γ ⊢ B}
		\infer1{Γ ⊢ A → B}
		\end{prooftree}
		\quad
		\begin{prooftree}
		\hypo{B ⊢ Δ, A}
		\infer1{A ↚ B ⊢ Δ}
		\end{prooftree}
		\quad
		\begin{prooftree}
		\hypo{A ⊢ Δ}
		\hypo{C ⊢ Λ, B, Ψ}
		\infer2{C ⊢ Λ, A ↚ B, Ψ, Δ}
		\end{prooftree}
		\]
		
		\[
		\begin{prooftree}
		\hypo{Γ ⊢ A}
		\hypo{Φ, B, Θ ⊢ C}
		\infer2{Φ, Γ, B ← A, Θ ⊢ C}
		\end{prooftree}
		\quad
		\begin{prooftree}
		\hypo{Γ, A ⊢ B}
		\infer1{Γ ⊢ B ← A}
		\end{prooftree}
		\quad
		\begin{prooftree}
		\hypo{B ⊢ A, Ψ}
		\infer1{A ↛ B ⊢ Ψ}
		\end{prooftree}
		\quad
		\begin{prooftree}
		\hypo{A ⊢ Δ}
		\hypo{C ⊢ Λ, B, Ψ}
		\infer2{C ⊢ Λ, B ↛ A, Ψ, Δ}
		\end{prooftree}
		\]
	\end{center}
\end{center}

\part{Theorems}
	\begin{center}
		\[
		\begin{prooftree}
		\infer0{A ⊢ A}
		\infer1{1, A ⊢ A}
		\infer1{1 ⊢ A, ¬ A}
		\infer1{1 ⊢ A ⅋ ¬ A}
		\end{prooftree}
		\quad
		\begin{prooftree}
		\infer0{A ⊢ A}
		\infer1{A, 1 ⊢ A}
		\infer1{A ⊗ 1 ⊢ A}
		\end{prooftree}
		\quad
		\begin{prooftree}
		\infer0{A, 0 ⊢ A}
		\infer1{A ⊗ 0 ⊢ A}
		\end{prooftree}
		\quad
		\begin{prooftree}
		\infer0{0 ⊢ A ⅋ ¬ A}
		\end{prooftree}
		\]
		
		\[
		\begin{prooftree}
		\infer0{A ⊢ A}
		\infer1{A ⊢ A, ⊥}
		\infer1{A, ¬ A ⊢ ⊥}
		\infer1{A ⊗ ¬ A ⊢ ⊥}
		\end{prooftree}
		\quad
		\begin{prooftree}
		\infer0{A ⊢ A}
		\infer1{A ⊢ A, ⊥}
		\infer1{A ⊢ A ⅋ ⊥}
		\end{prooftree}
		\quad
		\begin{prooftree}
		\infer0{A ⊢ A, ⊤}
		\infer1{A ⊢ A ⅋ ⊤}
		\end{prooftree}
		\quad
		\begin{prooftree}
		\infer0{A ⊗ ¬ A ⊢ ⊤}
		\end{prooftree}
		\]
		
		\[
		\begin{prooftree}
		\infer0{⊥ ⊢ }
		\infer1{⊥ ⊢ ⊥}
		\infer1{⊥, ¬ ⊥ ⊢ }
		\infer1{⊥ ⊗ ¬ ⊥ ⊢ }
		\end{prooftree}
		\quad
		\begin{prooftree}
		\infer0{A ⊢ A}
		\infer1{A ⊢ A, ⊥}
		\infer1{A, ¬ ⊥ ⊢ A}
		\infer1{A ⊗ ¬ ⊥ ⊢ A}
		\end{prooftree}
		\quad
		\begin{prooftree}
		\infer0{ ⊢ 1}
		\infer1{1 ⊢ 1}
		\infer1{1, ¬ 1 ⊢ }
		\infer1{1 ⊗ ¬ 1 ⊢ }
		\end{prooftree}
		\quad
		\begin{prooftree}
		\infer0{A ⊢ A}
		\infer1{A, 1 ⊢ A}
		\infer1{1, ¬ A ⊢ ¬ A}
		\infer1{1 ⊗ ¬ A ⊢ ¬ A}
		\end{prooftree}
		\quad
		\begin{prooftree}
		\infer0{0 ⊢ Δ, A}
		\infer1{0, ¬ A ⊢ Δ}
		\infer1{0 ⊗ ¬ A ⊢ Δ}
		\end{prooftree}
		\quad
		\begin{prooftree}
		\infer0{A ⊢ Δ, ⊤}
		\infer1{A, ¬ ⊤ ⊢ Δ}
		\infer1{A ⊗ ¬ ⊤ ⊢ Δ}
		\end{prooftree}
		\]
		
		\[
		\begin{prooftree}
		\infer0{A ⊢ A}
		\infer1{¬A ⊢ ¬A}
		\infer1{¬A ⊢ ¬A,  ⊥}
		\infer1{¬A ⊢ ¬A ⅋ ⊥}
		\end{prooftree}
		\quad
		\begin{prooftree}
		\infer0{A ⊢ A}
		\infer1{1, A ⊢ A}
		\infer1{A ⊢ A, ¬1}
		\infer1{A ⊢ A ⅋ ¬1}
		\end{prooftree}
		\quad
		\begin{prooftree}
		\infer0{0, Γ ⊢ A}
		\infer1{Γ ⊢ A, ¬0}
		\infer1{Γ ⊢ A ⅋ ¬0}
		\end{prooftree}
		\quad
		\begin{prooftree}
		\infer0{A, Γ ⊢ ⊤}
		\infer1{ Γ⊢ ⊤, ¬A}
		\infer1{Γ ⊢ ⊤ ⅋ ¬A}
		\end{prooftree}
		\]
		
		\[
		\begin{prooftree}
		\infer0{⊥ ⊢ }
		\infer1{⊥ ⊢ ⊥}
		\infer1{⊥, ¬ ⊥ ⊢ }
		\infer1{⊢ ⊥ ⅋ ¬ ⊥}
		\end{prooftree}
		\quad
		\begin{prooftree}
		\infer0{ ⊢ 1}
		\infer1{1 ⊢ 1}
		\infer1{1, ¬ 1 ⊢ }
		\infer1{1 ⊗ ¬ 1 ⊢ }
		\end{prooftree}
		\quad
		\begin{prooftree}
		\infer0{0 ⊢ Δ, A}
		\infer1{0, ¬ A ⊢ Δ}
		\infer1{0 ⊗ ¬ A ⊢ Δ}
		\end{prooftree}
		\quad
		\begin{prooftree}
		\infer0{A ⊢ Δ, ⊤}
		\infer1{A, ¬ ⊤ ⊢ Δ}
		\infer1{A ⊗ ¬ ⊤ ⊢ Δ}
		\end{prooftree}
		\]
		
		\[
		\begin{prooftree}
		\infer0{A ⊢ A}
		\infer1{A ⊢ A ⊕ ¬A}
		\end{prooftree}
		\quad
		\begin{prooftree}
		\infer0{A ⊢ A}
		\infer1{ ¬ A \& A ⊢ A}
		\end{prooftree}
		\]
		
		\[
		\begin{prooftree}
		\infer0{A ⊢ A}
		\infer1{¬ A, A ⊢}
		\infer1{ ¬ A ⊗ A ⊢}
		\end{prooftree}
		\quad
		\begin{prooftree}
		\infer0{A ⊢ A}
		\infer1{¬ A, A ⊢}
		\infer1{A ⊗ ¬ A ⊢}
		\infer1{⊢¬ (A ⊗ ¬ A)}
		\end{prooftree}
		\]

		\[
		\begin{prooftree}
		\infer0{A ⊢ A}
		\infer1{⊢ A, ¬ A}
		\infer1{⊢ A ⅋ ¬ A}
		\end{prooftree}
		\quad
		\begin{prooftree}
		\infer0{A ⊢ A}
		\infer1{⊢ A, ¬ A}
		\infer1{⊢ A ⅋ ¬ A}
		\infer1{¬ (A ⅋ ¬ A) ⊢}
		\end{prooftree}
		\]

		\[
		\begin{prooftree}
		\infer0{⊥ ⊢}
		\infer1{A \& ⊥ ⊢}
		\end{prooftree}
		\quad
		\begin{prooftree}
		\infer0{0 ⊢}
		\infer1{A \& 0 ⊢}
		\end{prooftree}
		\quad
		\begin{prooftree}
		\infer0{A ⊢ A}
		\infer0{A ⊢ ⊤}
		\infer2{A ⊢ A \& ⊤}
		\end{prooftree}
		\]
		
		\[
		\begin{prooftree}
		\infer0{⊢ 1}
		\infer1{⊢ A ⊕ 1}
		\end{prooftree}
		\quad
		\begin{prooftree}
		\infer0{⊢ ⊤}
		\infer1{⊢ A ⊕ ⊤}
		\end{prooftree}
		\quad
		\begin{prooftree}
		\infer0{A ⊢ A}
		\infer0{0 ⊢ A}
		\infer2{A ⊕ 0 ⊢ A}
		\end{prooftree}
		\]
	\end{center}

\end{document}}*}


\chapter{OMAffine}

\documentclass{article}

\usepackage{amsmath}
\usepackage{ebproof}
\usepackage{fullpage}
\usepackage[utf8]{inputenc}
\usepackage{newunicodechar}
\usepackage{stix}

\newunicodechar{Γ}{\Gamma}
\newunicodechar{Δ}{\Delta}

\newunicodechar{Θ}{\Theta}
\newunicodechar{Λ}{\Lambda}

\newunicodechar{Ξ}{\Xi}
\newunicodechar{Π}{\Pi}

\newunicodechar{Φ}{\Phi}
\newunicodechar{Ψ}{\Psi}

\newunicodechar{Ω}{\Omega}

\newunicodechar{⊢}{\vdash}

\newunicodechar{⊕}{\oplus}
\newunicodechar{¬}{\neg}

\newunicodechar{⊗}{\otimes}
\newunicodechar{→}{\rightarrow}
\newunicodechar{←}{\leftarrow}

\newunicodechar{⅋}{\upand}
\newunicodechar{↛}{\nrightarrow}
\newunicodechar{↚}{\nleftarrow}

\newunicodechar{⊥}{\bot}
\newunicodechar{⊤}{\top}

\setlength{\parindent}{0em}

\author{James Martin, Ian D.L.N. Mclean}
\title{Ordinary Multiplicative Calculus}

\begin{document}

\maketitle

\begin{abstract}
A sequent calculus of ordinary multiplicatives.
\end{abstract}

\section{Structural Rules}

\begin{center}
	\[
	\begin{prooftree}
	\infer0[Id]{A ⊢ A}
	\end{prooftree}
	\]
	
	\[
	\begin{prooftree}
	\hypo{Γ_{L}, A ⊢}
	\hypo{⊢ Δ_{L}, A}
	\infer2[Cut]{Γ_{L} ⊢ Δ_{L}}
	\end{prooftree}
	\]
	
	\[
	\begin{prooftree}
	\hypo{A, Γ_{R} ⊢}
	\hypo{⊢ A, Δ_{R}}
	\infer2[Cut]{Γ_{R} ⊢ Δ_{R}}
	\end{prooftree}
	\]
	\[
	\begin{prooftree}
	\hypo{Γ ⊢ A}
	\hypo{A ⊢ Δ}
	\infer2[Cut]{Γ ⊢ Δ}
	\end{prooftree}
	\]
	
	\[
	\begin{prooftree}
	\hypo{Γ_{L}, A, Γ_{R} ⊢}
	\hypo{⊢ Δ_{L}, A, Δ_{R}}
	\infer2[Cut]{Γ_{L}, Γ_{R} ⊢ Δ_{L}, Δ_{R}}
	\end{prooftree}
	\]
	
	\[
	\begin{prooftree}
	\hypo{Γ_{L}, A, Γ_{R} ⊢ Δ_{M}}
	\hypo{Γ_{M} ⊢ Δ_{L}, A, Δ_{R}}
	\infer2[Cut]{Γ_{L}, Γ_{M}, Γ_{R} ⊢ Δ_{L}, Δ_{M}, Δ_{R}}
	\end{prooftree}
	\]
\end{center}

\newpage
\section{Operational Rules}
\begin{center}
	\subsection{Negations}
	\begin{center}
		\[
		\begin{prooftree}
		\hypo{Γ_{L}, Γ_{R} ⊢ Δ_{L}, A, Δ_{R}}
		\infer1{Γ_{L}, ¬_{M} A, Γ_{R} ⊢ Δ_{L}, Δ_{R}}
		\end{prooftree}
		\quad
		\begin{prooftree}
		\hypo{ Γ_{L}, A, Γ_{R} ⊢ Δ_{L}, Δ_{R}}
		\infer1{ Γ_{L}, Γ_{R} ⊢ Δ_{L}, ¬_{M} A, Δ_{R}}
		\end{prooftree}
		\]
		
		\[
		\begin{prooftree}
		\hypo{Γ_{L}, Γ_{R} ⊢ A, Δ}
		\infer1{Γ_{L}, ¬_{L} A, Γ_{R} ⊢ Δ}
		\end{prooftree}
		\quad
		\begin{prooftree}
		\hypo{A, Γ ⊢ Δ_{L}, Δ_{R}}
		\infer1{ Γ ⊢ Δ_{L}, ¬_{L} A, Δ_{R}}
		\end{prooftree}
		\]
		
		\[
		\begin{prooftree}
		\hypo{Γ_{L}, Γ_{R} ⊢ Δ, A}
		\infer1{Γ_{L}, ¬_{R} A, Γ_{R} ⊢ Δ}
		\end{prooftree}
		\quad
		\begin{prooftree}
		\hypo{ Γ, A ⊢ Δ_{L}, Δ_{R}}
		\infer1{ Γ ⊢ Δ_{L}, ¬_{R} A, Δ_{R}}
		\end{prooftree}
		\]
		
		\[
		\begin{prooftree}
		\hypo{Γ ⊢ A, Δ}
		\infer1{¬_{L} A, Γ ⊢ Δ}
		\end{prooftree}
		\quad
		\begin{prooftree}
		\hypo{ A, Γ ⊢ Δ}
		\infer1{ Γ ⊢ ¬_{L} A, Δ}
		\end{prooftree}
		\quad
		\begin{prooftree}
		\hypo{Γ ⊢ Δ, A}
		\infer1{Γ, ¬_{R} A ⊢ Δ}
		\end{prooftree}
		\quad
		\begin{prooftree}
		\hypo{ Γ, A ⊢ Δ}
		\infer1{ Γ ⊢ Δ, ¬_{R} A}
		\end{prooftree}
		\]
		
		\[
		\begin{prooftree}
		\hypo{Γ ⊢ Δ, A}
		\infer1{¬^{*}_{L} A, Γ ⊢ Δ}
		\end{prooftree}
		\quad
		\begin{prooftree}
		\hypo{Γ, A ⊢ Δ}
		\infer1{ Γ ⊢ ¬^{*}_{L} A, Δ}
		\end{prooftree}
		\quad
		\begin{prooftree}
		\hypo{Γ ⊢ A, Δ}
		\infer1{Γ, ¬^{*}_{R} A ⊢ Δ}
		\end{prooftree}
		\quad
		\begin{prooftree}
		\hypo{ A, Γ ⊢ Δ}
		\infer1{ Γ ⊢ Δ, ¬^{*}_{R} A}
		\end{prooftree}
		\]
	\end{center}

	\subsection{Junctions}
	\begin{center}
		\[
		\begin{prooftree}
		\hypo{Γ_{L}, A, B, Γ_{R} ⊢ Δ}
		\infer1{Γ_{L}, A ⊗ B, Γ_{R} ⊢ Δ}
		\end{prooftree}
		\quad
		\begin{prooftree}
		\hypo{Γ_{L} ⊢ Δ_{L}, A}
		\hypo{Γ_{R} ⊢ B, Δ_{R}}
		\infer2{Γ_{L}, Γ_{R} ⊢ Δ_{L}, A ⊗ B, Δ_{R}}
		\end{prooftree}
		\]
		
		\[
		\begin{prooftree}
		\hypo{Γ_{L}, B, A, Γ_{R} ⊢ Δ}
		\infer1{Γ_{L}, B ⊗ A, Γ_{R} ⊢ Δ}
		\end{prooftree}
		\quad
		\begin{prooftree}
		\hypo{Γ_{L} ⊢ Δ_{L}, B}
		\hypo{Γ_{R} ⊢ A, Δ_{R}}
		\infer2{Γ_{L}, Γ_{R} ⊢ Δ_{L}, B ⊗ A, Δ_{R}}
		\end{prooftree}
		\]
		
		\[
		\begin{prooftree}
		\hypo{Γ_{L}, B ⊢ Δ_{L}}
		\hypo{A, Γ_{R} ⊢ Δ_{R}}
		\infer2{Γ_{L}, B ⅋ A, Γ_{R} ⊢ Δ_{L}, Δ_{R}}
		\end{prooftree}
		\quad
		\begin{prooftree}
		\hypo{Γ ⊢ Δ_{L}, B, A, Δ_{R}}
		\infer1{Γ ⊢ Δ_{L}, B ⅋ A, Δ_{R}}
		\end{prooftree}
		\]
		
		\[
		\begin{prooftree}
		\hypo{Γ_{L}, A ⊢ Δ_{L}}
		\hypo{B, Γ_{R} ⊢ Δ_{R}}
		\infer2{Γ_{L}, A ⅋ B, Γ_{R} ⊢ Δ_{L}, Δ_{R}}
		\end{prooftree}
		\quad
		\begin{prooftree}
		\hypo{Γ ⊢ Δ_{L}, A, B, Δ_{R}}
		\infer1{Γ ⊢ Δ_{L}, A ⅋ B, Δ_{R}}
		\end{prooftree}
		\]
	\end{center}
	\subsection{Conditionals}
		\begin{center}
			There are at least 4 kinds of implications possibly 5 if middle negation can be used. 
		\end{center}
		\subsection{Biconditionals}
			\begin{center}
				\[
				\begin{prooftree}
				\hypo{Γ_{L}, ¬A ⅋ B, ¬B ⅋ A, Γ_{R} ⊢ Δ_{L},Δ_{M},Δ_{R}}
				\infer1{Γ_{L}, (¬A ⅋ B)⊗(¬B ⅋ A), Γ_{R} ⊢ Δ_{L},Δ_{M},Δ_{R}}
				\end{prooftree}
				\quad
				\begin{prooftree}
				\hypo{Γ_{L}, ¬A ⊢ Δ_{L}}
				\hypo{B, ¬B ⊢ Δ_{M}}
				\hypo{A, Γ_{R} ⊢ Δ_{R}}
				\infer3{Γ_{L}, (¬A ⅋ B)⊗(¬B ⅋ A), Γ_{R} ⊢ Δ_{L},Δ_{M},Δ_{R}}
				\end{prooftree}
				\]
				
				\[
				\begin{prooftree}
				\hypo{Γ_{L} ⊢ Δ_{L}, ¬A , B}
				\hypo{Γ_{R} ⊢ ¬B , A, Δ_{R}}
				\infer2{Γ_{L}, Γ_{R} ⊢ Δ_{L}, (¬A ⅋ B)⊗(¬B ⅋ A), Δ_{R}}
				\end{prooftree}
				\]
				
				\[
				\begin{prooftree}
				\hypo{A, Γ_{L} ⊢ Δ_{L}, B}
				\hypo{B, Γ_{R} ⊢ A, Δ_{R}}
				\infer2{Γ_{L}, Γ_{R} ⊢ Δ_{L}, (¬_{L}A ⅋ B)⊗(¬_{L}B ⅋ A), Δ_{R}}
				\end{prooftree}
				\]
				
				\[
				\begin{prooftree}
				\hypo{Γ_{L}, A ⊢ Δ_{L}, B}
				\hypo{Γ_{R}, B ⊢ A, Δ_{R}}
				\infer2{Γ_{L}, Γ_{R} ⊢ Δ_{L}, (¬_{R}A ⅋ B)⊗(¬_{R}B ⅋ A), Δ_{R}}
				\end{prooftree}
				\]
				
				\[
				\begin{prooftree}
				\hypo{Γ_{L}, A , B ⊢ Δ_{L}}
				\hypo{¬A , ¬B, Γ_{R} ⊢ Δ_{R}}
				\infer2{Γ_{L}, (A ⊗ B)⅋(¬A ⊗ ¬B), Γ_{R} ⊢ Δ_{L},Δ_{R}}
				\end{prooftree}
				\]
			\end{center}
\end{center}

\part{Theorems}
	\begin{center}
		
	\end{center}

\end{document}


\chapter{OMultiplicative}






\usepackage[utf8]{inputenc}











































A sequent calculus of ordinary multiplicatives.


\section{Structural Rules}

\begin{center}
	\[
	\begin{prooftree}
	\infer0[Id]{A $\vdash$  A}
	\end{prooftree}
	\]
	
	\[
	\begin{prooftree}
	\hypo{$\Gamma$ _{L}, A $\vdash$ }
	\hypo{$\vdash$  $\Delta$ _{L}, A}
	\infer2[Cut]{$\Gamma$ _{L} $\vdash$  $\Delta$ _{L}}
	\end{prooftree}
	\]
	
	\[
	\begin{prooftree}
	\hypo{A, $\Gamma$ _{R} $\vdash$ }
	\hypo{$\vdash$  A, $\Delta$ _{R}}
	\infer2[Cut]{$\Gamma$ _{R} $\vdash$  $\Delta$ _{R}}
	\end{prooftree}
	\]
	\[
	\begin{prooftree}
	\hypo{$\Gamma$  $\vdash$  A}
	\hypo{A $\vdash$  $\Delta$ }
	\infer2[Cut]{$\Gamma$  $\vdash$  $\Delta$ }
	\end{prooftree}
	\]
	
	\[
	\begin{prooftree}
	\hypo{$\Gamma$ _{L}, A, $\Gamma$ _{R} $\vdash$ }
	\hypo{$\vdash$  $\Delta$ _{L}, A, $\Delta$ _{R}}
	\infer2[Cut]{$\Gamma$ _{L}, $\Gamma$ _{R} $\vdash$  $\Delta$ _{L}, $\Delta$ _{R}}
	\end{prooftree}
	\]
	
	\[
	\begin{prooftree}
	\hypo{$\Gamma$ _{L}, A, $\Gamma$ _{R} $\vdash$  $\Delta$ _{M}}
	\hypo{$\Gamma$ _{M} $\vdash$  $\Delta$ _{L}, A, $\Delta$ _{R}}
	\infer2[Cut]{$\Gamma$ _{L}, $\Gamma$ _{M}, $\Gamma$ _{R} $\vdash$  $\Delta$ _{L}, $\Delta$ _{M}, $\Delta$ _{R}}
	\end{prooftree}
	\]
\end{center}

\newpage
\section{Operational Rules}
\begin{center}
	\subsection{Negations}
	\begin{center}
		\[
		\begin{prooftree}
		\hypo{$\Gamma$ _{L}, $\Gamma$ _{R} $\vdash$  $\Delta$ _{L}, A, $\Delta$ _{R}}
		\infer1{$\Gamma$ _{L}, $\neg$ _{M} A, $\Gamma$ _{R} $\vdash$  $\Delta$ _{L}, $\Delta$ _{R}}
		\end{prooftree}
		\quad
		\begin{prooftree}
		\hypo{ $\Gamma$ _{L}, A, $\Gamma$ _{R} $\vdash$  $\Delta$ _{L}, $\Delta$ _{R}}
		\infer1{ $\Gamma$ _{L}, $\Gamma$ _{R} $\vdash$  $\Delta$ _{L}, $\neg$ _{M} A, $\Delta$ _{R}}
		\end{prooftree}
		\]
		
		\[
		\begin{prooftree}
		\hypo{$\Gamma$ _{L}, $\Gamma$ _{R} $\vdash$  A, $\Delta$ }
		\infer1{$\Gamma$ _{L}, $\neg$ _{L} A, $\Gamma$ _{R} $\vdash$  $\Delta$ }
		\end{prooftree}
		\quad
		\begin{prooftree}
		\hypo{A, $\Gamma$  $\vdash$  $\Delta$ _{L}, $\Delta$ _{R}}
		\infer1{ $\Gamma$  $\vdash$  $\Delta$ _{L}, $\neg$ _{L} A, $\Delta$ _{R}}
		\end{prooftree}
		\]
		
		\[
		\begin{prooftree}
		\hypo{$\Gamma$ _{L}, $\Gamma$ _{R} $\vdash$  $\Delta$ , A}
		\infer1{$\Gamma$ _{L}, $\neg$ _{R} A, $\Gamma$ _{R} $\vdash$  $\Delta$ }
		\end{prooftree}
		\quad
		\begin{prooftree}
		\hypo{ $\Gamma$ , A $\vdash$  $\Delta$ _{L}, $\Delta$ _{R}}
		\infer1{ $\Gamma$  $\vdash$  $\Delta$ _{L}, $\neg$ _{R} A, $\Delta$ _{R}}
		\end{prooftree}
		\]
		
		\[
		\begin{prooftree}
		\hypo{$\Gamma$  $\vdash$  A, $\Delta$ }
		\infer1{$\neg$ _{L} A, $\Gamma$  $\vdash$  $\Delta$ }
		\end{prooftree}
		\quad
		\begin{prooftree}
		\hypo{ A, $\Gamma$  $\vdash$  $\Delta$ }
		\infer1{ $\Gamma$  $\vdash$  $\neg$ _{L} A, $\Delta$ }
		\end{prooftree}
		\quad
		\begin{prooftree}
		\hypo{$\Gamma$  $\vdash$  $\Delta$ , A}
		\infer1{$\Gamma$ , $\neg$ _{R} A $\vdash$  $\Delta$ }
		\end{prooftree}
		\quad
		\begin{prooftree}
		\hypo{ $\Gamma$ , A $\vdash$  $\Delta$ }
		\infer1{ $\Gamma$  $\vdash$  $\Delta$ , $\neg$ _{R} A}
		\end{prooftree}
		\]
		
		\[
		\begin{prooftree}
		\hypo{$\Gamma$  $\vdash$  $\Delta$ , A}
		\infer1{$\neg$ ^{*}_{L} A, $\Gamma$  $\vdash$  $\Delta$ }
		\end{prooftree}
		\quad
		\begin{prooftree}
		\hypo{$\Gamma$ , A $\vdash$  $\Delta$ }
		\infer1{ $\Gamma$  $\vdash$  $\neg$ ^{*}_{L} A, $\Delta$ }
		\end{prooftree}
		\quad
		\begin{prooftree}
		\hypo{$\Gamma$  $\vdash$  A, $\Delta$ }
		\infer1{$\Gamma$ , $\neg$ ^{*}_{R} A $\vdash$  $\Delta$ }
		\end{prooftree}
		\quad
		\begin{prooftree}
		\hypo{ A, $\Gamma$  $\vdash$  $\Delta$ }
		\infer1{ $\Gamma$  $\vdash$  $\Delta$ , $\neg$ ^{*}_{R} A}
		\end{prooftree}
		\]
	\end{center}

	\subsection{Junctions}
	\begin{center}
		\[
		\begin{prooftree}
		\hypo{$\Gamma$ _{L}, A, B, $\Gamma$ _{R} $\vdash$  $\Delta$ }
		\infer1{$\Gamma$ _{L}, A $\otimes$  B, $\Gamma$ _{R} $\vdash$  $\Delta$ }
		\end{prooftree}
		\quad
		\begin{prooftree}
		\hypo{$\Gamma$ _{L} $\vdash$  $\Delta$ _{L}, A}
		\hypo{$\Gamma$ _{R} $\vdash$  B, $\Delta$ _{R}}
		\infer2{$\Gamma$ _{L}, $\Gamma$ _{R} $\vdash$  $\Delta$ _{L}, A $\otimes$  B, $\Delta$ _{R}}
		\end{prooftree}
		\]
		
		\[
		\begin{prooftree}
		\hypo{$\Gamma$ _{L}, B, A, $\Gamma$ _{R} $\vdash$  $\Delta$ }
		\infer1{$\Gamma$ _{L}, B $\otimes$  A, $\Gamma$ _{R} $\vdash$  $\Delta$ }
		\end{prooftree}
		\quad
		\begin{prooftree}
		\hypo{$\Gamma$ _{L} $\vdash$  $\Delta$ _{L}, B}
		\hypo{$\Gamma$ _{R} $\vdash$  A, $\Delta$ _{R}}
		\infer2{$\Gamma$ _{L}, $\Gamma$ _{R} $\vdash$  $\Delta$ _{L}, B $\otimes$  A, $\Delta$ _{R}}
		\end{prooftree}
		\]
		
		\[
		\begin{prooftree}
		\hypo{$\Gamma$ _{L}, B $\vdash$  $\Delta$ _{L}}
		\hypo{A, $\Gamma$ _{R} $\vdash$  $\Delta$ _{R}}
		\infer2{$\Gamma$ _{L}, B $\parr$  A, $\Gamma$ _{R} $\vdash$  $\Delta$ _{L}, $\Delta$ _{R}}
		\end{prooftree}
		\quad
		\begin{prooftree}
		\hypo{$\Gamma$  $\vdash$  $\Delta$ _{L}, B, A, $\Delta$ _{R}}
		\infer1{$\Gamma$  $\vdash$  $\Delta$ _{L}, B $\parr$  A, $\Delta$ _{R}}
		\end{prooftree}
		\]
		
		\[
		\begin{prooftree}
		\hypo{$\Gamma$ _{L}, A $\vdash$  $\Delta$ _{L}}
		\hypo{B, $\Gamma$ _{R} $\vdash$  $\Delta$ _{R}}
		\infer2{$\Gamma$ _{L}, A $\parr$  B, $\Gamma$ _{R} $\vdash$  $\Delta$ _{L}, $\Delta$ _{R}}
		\end{prooftree}
		\quad
		\begin{prooftree}
		\hypo{$\Gamma$  $\vdash$  $\Delta$ _{L}, A, B, $\Delta$ _{R}}
		\infer1{$\Gamma$  $\vdash$  $\Delta$ _{L}, A $\parr$  B, $\Delta$ _{R}}
		\end{prooftree}
		\]
	\end{center}
	\subsection{Conditionals}
		\begin{center}
			There are at least 4 kinds of implications possibly 5 if middle negation can be used.
			
			\[
			\begin{prooftree}
			\hypo{$\Gamma$ _{L}, $\Gamma$ _{M} $\vdash$  $\Delta$ _{L}, B}
			\hypo{A, $\Gamma$ _{R} $\vdash$  $\Delta$ _{R}}
			\infer2{$\Gamma$ _{L}, $\Gamma$ _{M}, $\neg$ _{R} B $\parr$  A, $\Gamma$ _{R} $\vdash$  $\Delta$ _{L}, $\Delta$ _{R}}
			\end{prooftree}
			\quad
			\begin{prooftree}
			\hypo{$\Gamma$ _{L}, B $\vdash$  $\Delta$ _{L}, A, $\Delta$ _{R}}
			\infer1{$\Gamma$ _{L} $\vdash$  $\Delta$ _{L}, $\neg$ _{R}B $\parr$  A, $\Delta$ _{R}}
			\end{prooftree}
			\quad
			\begin{prooftree}
			\hypo{B, $\Gamma$ _{R} $\vdash$  $\Delta$ _{L}, A, $\Delta$ _{R}}
			\infer1{$\Gamma$ _{R} $\vdash$  $\Delta$ _{L}, $\neg$ _{L}B $\parr$  A, $\Delta$ _{R}}
			\end{prooftree}
			\quad
			\begin{prooftree}
			\hypo{$\Gamma$ _{L}, B, $\Gamma$ _{R} $\vdash$  $\Delta$ _{L}, A, $\Delta$ _{R}}
			\infer1{$\Gamma$ _{L}, $\Gamma$ _{R} $\vdash$  $\Delta$ _{L}, $\neg$ _{M}B $\parr$  A, $\Delta$ _{R}}
			\end{prooftree}
			\]
			
			\[
			\begin{prooftree}
			\hypo{$\Gamma$ _{L}, B, $\Gamma$ _{R} $\vdash$  $\Delta$ _{L}, A}
			\infer1{$\Gamma$ _{L}, $\neg$ _{R}A $\otimes$  B, $\Gamma$ _{R} $\vdash$  $\Delta$ _{L}}
			\end{prooftree}
			\quad
			\begin{prooftree}
			\hypo{$\Gamma$ _{L}, B, $\Gamma$ _{R} $\vdash$  $\Delta$ _{L}, A, $\Delta$ _{R}}
			\infer1{$\Gamma$ _{L}, $\neg$ _{M}A $\otimes$  B, $\Gamma$ _{R} $\vdash$  $\Delta$ _{L}, $\Delta$ _{R}}
			\end{prooftree}
			\quad
			\begin{prooftree}
			\hypo{$\Gamma$ _{L}, B, $\Gamma$ _{R} $\vdash$  A, $\Delta$ _{R}}
			\infer1{$\Gamma$ _{L}, $\neg$ _{L}A $\otimes$  B, $\Gamma$ _{R} $\vdash$  $\Delta$ _{R}}
			\end{prooftree}
			\quad
			\begin{prooftree}
			\hypo{$\Gamma$ _{L}, A $\vdash$  $\Delta$ _{L}}
			\hypo{$\Gamma$ _{R} $\vdash$  B, $\Delta$ _{R}}
			\infer2{$\Gamma$ _{L}, $\Gamma$ _{R} $\vdash$  $\Delta$ _{L}, $\neg$ _{R} A $\otimes$  B, $\Delta$ _{R}}
			\end{prooftree}
			\]
			
		\end{center}
		\subsection{Biconditionals}
			\begin{center}
				\[
				\begin{prooftree}
				\hypo{$\Gamma$ _{L}, $\neg$ A $\parr$  B, $\neg$ B $\parr$  A, $\Gamma$ _{R} $\vdash$  $\Delta$ _{L},$\Delta$ _{M},$\Delta$ _{R}}
				\infer1{$\Gamma$ _{L}, ($\neg$ A $\parr$  B)$\otimes$ ($\neg$ B $\parr$  A), $\Gamma$ _{R} $\vdash$  $\Delta$ _{L},$\Delta$ _{M},$\Delta$ _{R}}
				\end{prooftree}
				\quad
				\begin{prooftree}
				\hypo{$\Gamma$ _{L} $\vdash$  $\Delta$ _{L}, A}
				\hypo{B, $\neg$ B $\vdash$  $\Delta$ _{M}}
				\hypo{A, $\Gamma$ _{R} $\vdash$  $\Delta$ _{R}}
				\infer3{$\Gamma$ _{L}, ($\neg$ _{R}A $\parr$  B)$\otimes$ ($\neg$ B $\parr$  A), $\Gamma$ _{R} $\vdash$  $\Delta$ _{L},$\Delta$ _{M},$\Delta$ _{R}}
				\end{prooftree}
				\]
				
				\[
				\begin{prooftree}
				\hypo{$\Gamma$ _{L} $\vdash$  $\Delta$ , A}
				\infer0{B $\vdash$  B}
				\hypo{A, $\Gamma$ _{R} $\vdash$  }
				\infer3{$\Gamma$ _{L}, ($\neg$ _{R}A $\parr$  B)$\otimes$ ($\neg$ B $\parr$  A), $\Gamma$ _{R} $\vdash$  $\Delta$ }
				\end{prooftree}
				\]
				
				\[
				\begin{prooftree}
				\hypo{$\Gamma$ _{L}$\vdash$  $\Delta$ _{L}, A}
				\infer0{B $\vdash$  B}
				\hypo{A, $\Gamma$ _{R} $\vdash$  $\Delta$ _{R}}
				\infer3{$\Gamma$ _{L}, ($\neg$ _{R} A $\parr$  B)$\otimes$ ($\neg$  B $\parr$  A), $\Gamma$ _{R} $\vdash$  $\Delta$ _{L},$\Delta$ _{R}}
				\end{prooftree}
				\quad
				\begin{prooftree}
				\hypo{$\Gamma$ _{L}, A $\vdash$  $\Delta$ _{L}, B}
				\hypo{B, $\Gamma$ _{R} $\vdash$  A, $\Delta$ _{R}}
				\infer2{$\Gamma$ _{L}, $\Gamma$ _{R} $\vdash$  $\Delta$ _{L}, ($\neg$ _{R}A $\parr$  B)$\otimes$ ($\neg$ _{L}B $\parr$  A), $\Delta$ _{R}}
				\end{prooftree}
				\]
				
				\[
				\begin{prooftree}
				\hypo{$\Gamma$ _{L}, B $\vdash$  $\Delta$ _{L}}
				\infer0{A $\vdash$  A}
				\hypo{$\Gamma$ _{R} $\vdash$  B, $\Delta$ _{R}}
				\infer3{$\Gamma$ _{L}, (B $\parr$  $\neg$  A)$\otimes$ (A $\parr$  $\neg$ _{L}B), $\Gamma$ _{R} $\vdash$  $\Delta$ _{L},$\Delta$ _{R}}
				\end{prooftree}
				\quad
				\begin{prooftree}
				\hypo{$\Gamma$ _{L}, B $\vdash$  $\Delta$ _{L}, A}
				\hypo{A, $\Gamma$ _{R} $\vdash$  B, $\Delta$ _{R}}
				\infer2{$\Gamma$ _{L}, $\Gamma$ _{R} $\vdash$  $\Delta$ _{L}, ($\neg$ _{R}B $\parr$  A)$\otimes$ ($\neg$ _{L}A $\parr$  B), $\Delta$ _{R}}
				\end{prooftree}
				\]

				\[
				\begin{prooftree}
				\hypo{$\Gamma$ _{L} $\vdash$  $\Delta$ _{L}, A}
				\hypo{B,A $\vdash$ }
				\hypo{$\Gamma$ _{R} $\vdash$  B, $\Delta$ _{R}}
				\infer3{$\Gamma$ _{L}, ($\neg$ _{R} A $\parr$  B)$\otimes$ (A $\parr$  $\neg$ _{L}B), $\Gamma$ _{R} $\vdash$  $\Delta$ _{L},$\Delta$ _{R}}
				\end{prooftree}
				\]
				
				\[
				\begin{prooftree}
				\hypo{$\Gamma$ _{L}, B $\vdash$  $\Delta$ _{L}}
				\hypo{$\vdash$  A, B}
				\hypo{A, $\Gamma$ _{R} $\vdash$  $\Delta$ _{R}}
				\infer3{$\Gamma$ _{L}, (B $\parr$  $\neg$  A)$\otimes$ ($\neg$  B $\parr$  A), $\Gamma$ _{R} $\vdash$  $\Delta$ _{L},$\Delta$ _{R}}
				\end{prooftree}
				\quad
				\begin{prooftree}
				\hypo{$\Gamma$ _{L}, B $\vdash$  $\Delta$ _{L}}
				\hypo{$\vdash$  B, A}
				\hypo{A, $\Gamma$ _{R} $\vdash$  $\Delta$ _{R}}
				\infer3{$\Gamma$ _{L}, (B $\parr$  $\neg$  A)$\otimes$ ($\neg$  B $\parr$  A), $\Gamma$ _{R} $\vdash$  $\Delta$ _{L},$\Delta$ _{R}}
				\end{prooftree}				
				\]
				
				\[
				\begin{prooftree}
				\hypo{$\Gamma$ _{L}, A , B $\vdash$  $\Delta$ _{L}}
				\hypo{$\neg$ _{L}A , $\neg$ _{L}B, $\Gamma$ _{R} $\vdash$  $\Delta$ _{R}}
				\infer2{$\Gamma$ _{L}, (A $\otimes$  B)$\parr$ ($\neg$ A $\otimes$  $\neg$ B), $\Gamma$ _{R} $\vdash$  $\Delta$ _{L},$\Delta$ _{R}}
				\end{prooftree}
				\]
				Disjunctive Biconditional LHS
				\[
				\begin{prooftree}
				\hypo{$\Gamma$ _{L}, A , B $\vdash$  $\Delta$ _{L}}
				\hypo{$\Gamma$ _{R} $\vdash$  A, B, $\Delta$ _{R}}
				\infer2{$\Gamma$ _{L}, (A $\otimes$  B)$\parr$ ($\neg$ A $\otimes$  $\neg$ B), $\Gamma$ _{R} $\vdash$  $\Delta$ _{L},$\Delta$ _{R}}
				\end{prooftree}
				\quad
				\begin{prooftree}
				\hypo{$\Gamma$ _{L}, A , B $\vdash$  $\Delta$ _{L}}
				\hypo{$\Gamma$ _{R} $\vdash$  B, A, $\Delta$ _{R}}
				\infer2{$\Gamma$ _{L}, (A $\otimes$  B)$\parr$ ($\neg$ A $\otimes$  $\neg$ B), $\Gamma$ _{R} $\vdash$  $\Delta$ _{L},$\Delta$ _{R}}
				\end{prooftree}
				\]
			\end{center}
\end{center}

\part{Theorems}
	\begin{center}
		\[
		\begin{prooftree}
		\infer0{A $\vdash$  A}
		\infer1{A $\otimes$  $\neg$  A $\vdash$  }
		\end{prooftree}
		\quad
		\begin{prooftree}
		\infer0{A $\vdash$  A}
		\infer0{$\neg$ A $\vdash$  $\neg$ A}
		\infer2{A, $\neg$ A $\vdash$  A $\otimes$  $\neg$ A}
		\end{prooftree}
		\quad
		\begin{prooftree}
		\infer0{$\neg$ A $\vdash$  $\neg$ A}
		\infer0{A $\vdash$  A}
		\infer2{$\neg$ A, A $\vdash$  $\neg$ A $\otimes$  A}
		\end{prooftree}
		\]
		
		\[
		\begin{prooftree}
		\infer0{A $\vdash$  A}
		\infer1{$\vdash$  $\neg$ (A $\otimes$  $\neg$  A)}
		\end{prooftree}
		\]
		
		\[
		\begin{prooftree}
		\infer0{A $\vdash$  A}
		\infer0{$\neg$ A $\vdash$  $\neg$ A}
		\infer2{A $\parr$  $\neg$ A $\vdash$  A , $\neg$ A}
		\end{prooftree}
		\quad
		\begin{prooftree}
		\infer0{$\neg$ A $\vdash$  $\neg$ A}
		\infer0{A $\vdash$  A}
		\infer2{$\neg$ A $\parr$  A $\vdash$  $\neg$ A , A}
		\end{prooftree}
		\quad
		\begin{prooftree}
		\infer0{A $\vdash$  A}
		\infer1{$\vdash$   $\neg$ A $\parr$  A}
		\end{prooftree}
		\]
		
		\[
		\begin{prooftree}
		\infer0{A $\vdash$  A}
		\infer1{$\neg$ ($\neg$ A $\parr$  A) $\vdash$  }
		\end{prooftree}
		\]
	\end{center}



\chapter{Order Additive Mix Visible Sequent}
}

\maketitle

\begin{abstract}
Negation has partial visibility; the junctions obey strict visibility but the conditionals obey partial visibility.
\end{abstract}
	\begin{center}
		\section{Structural Rules}
			\begin{center}
				\[
				\begin{prooftree}
				\infer0[Id]{A ⊢ A}
				\end{prooftree}
				\]
				
				\[
				\begin{prooftree}
				\hypo{Γ ⊢ A}
				\hypo{A ⊢ Δ}
				\infer2[cut]{Γ ⊢ Δ}
				\end{prooftree}
				\]
			\end{center}
		
		\section{Operational Rules}
			\begin{center}
				Negations
				\[
				\begin{prooftree}
				\hypo{ ⊢ Δ, A}
				\infer1{¬_{R} A ⊢ Δ}
				\end{prooftree}
				\quad
				\begin{prooftree}
				\hypo{Γ, A ⊢ }
				\infer1{Γ ⊢ ¬_{R} A}
				\end{prooftree}
				\]
				\[
				\begin{prooftree}
				\hypo{A, Γ ⊢ }
				\infer1{Γ ⊢ ¬_{L} A}
				\end{prooftree}
				\quad
				\begin{prooftree}
				\hypo{ ⊢ A, Δ}
				\infer1{¬_{L} A ⊢ Δ}
				\end{prooftree}
				\]
				
				NANDs and NORs
				
				\[
				\begin{prooftree}
				\hypo{⊢ A, Δ}
				\hypo{⊢ B, Δ}
				\infer2{¬A ⊕ ¬B ⊢ Δ}
				\end{prooftree}
				\quad
				\begin{prooftree}
				\hypo{Γ, A ⊢ }
				\infer1{Γ ⊢ ¬A ⊕ ¬B}
				\end{prooftree}
				\quad
				\begin{prooftree}
				\hypo{Γ, B ⊢ }
				\infer1{Γ ⊢ ¬A ⊕ ¬B}
				\end{prooftree}
				\]
				
				\[
				\begin{prooftree}
				\hypo{⊢ B, Δ}
				\hypo{ ⊢ A, Δ}
				\infer2{¬B ⊕ ¬A ⊢ Δ}
				\end{prooftree}
				\quad
				\begin{prooftree}
				\hypo{Γ, A ⊢ }
				\infer1{Γ ⊢ ¬B ⊕ ¬A}
				\end{prooftree}
				\quad
				\begin{prooftree}
				\hypo{Γ, B ⊢ }
				\infer1{Γ ⊢ ¬B ⊕ ¬A}
				\end{prooftree}
				\]
				
				\[
				\begin{prooftree}
				\hypo{ ⊢ A, Δ}
				\infer1{¬A \& ¬B ⊢ Δ}
				\end{prooftree}
				\quad
				\begin{prooftree}
				\hypo{ ⊢ B, Δ}
				\infer1{¬A \& ¬B ⊢ Δ}
				\end{prooftree}
				\quad
				\begin{prooftree}
				\hypo{Γ,A ⊢ }
				\hypo{Γ,B ⊢ }
				\infer2{Γ⊢ ¬A \& ¬B}
				\end{prooftree}
				\]
				
				\[
				\begin{prooftree}
				\hypo{ ⊢ A,Δ}
				\infer1{¬B \& ¬A ⊢ Δ}
				\end{prooftree}
				\quad
				\begin{prooftree}
				\hypo{ ⊢ B,Δ}
				\infer1{¬B \& ¬A ⊢ Δ}
				\end{prooftree}
				\quad
				\begin{prooftree}
				\hypo{Γ, B ⊢ }
				\hypo{Γ, A ⊢ }
				\infer2{Γ ⊢ ¬ B \& ¬ A}
				\end{prooftree}
				\]
				
				Conditionals
				
				\[
				\begin{prooftree}
				\hypo{⊢ B,Δ}
				\hypo{A ⊢ Δ}
				\infer2{¬B ⊕ A ⊢ Δ}
				\end{prooftree}
				\quad
				\begin{prooftree}
				\hypo{A ⊢ Δ}
				\hypo{⊢ B,Δ}
				\infer2{A ⊕ ¬B ⊢Δ}
				\end{prooftree}
				\quad
				\begin{prooftree}
				\hypo{Γ ⊢ A, Δ}
				\infer1{Γ ⊢ A ⊕ ¬B, Δ}
				\end{prooftree}
				\quad
				\begin{prooftree}
				\hypo{Γ, B ⊢ }
				\infer1{Γ ⊢ A ⊕ ¬B}
				\end{prooftree}
				\]
				
				\[
				\begin{prooftree}
				\hypo{ ⊢ A, Δ}
				\hypo{B ⊢ Δ}
				\infer2{¬A ⊕ B ⊢ Δ}
				\end{prooftree}
				\quad
				\begin{prooftree}
				\hypo{B ⊢ Δ}
				\hypo{ ⊢ A, Δ}
				\infer2{B ⊕ ¬A ⊢ Δ}
				\end{prooftree}
				\quad
				\begin{prooftree}
				\hypo{Γ, A ⊢ }
				\infer1{Γ ⊢ B ⊕ ¬A}
				\end{prooftree}
				\quad
				\begin{prooftree}
				\hypo{Γ ⊢ B, Δ}
				\infer1{Γ ⊢ B ⊕ ¬A, Δ}
				\end{prooftree}
				\]
				
				\[
				\begin{prooftree}
				\hypo{Γ, A ⊢ Δ}
				\infer1{Γ, A \& ¬B ⊢ Δ}
				\end{prooftree}
				\quad
				\begin{prooftree}
				\hypo{ ⊢ B, Δ}
				\infer1{A \& ¬B ⊢ Δ}
				\end{prooftree}
				\quad
				\begin{prooftree}
				\hypo{Γ⊢ A}
				\hypo{Γ,B ⊢ }
				\infer2{Γ⊢ A \& ¬B}
				\end{prooftree}
				\quad
				\begin{prooftree}
				\hypo{Γ,B ⊢ }
				\hypo{Γ⊢ A}
				\infer2{Γ⊢ ¬B \& A}
				\end{prooftree}
				\]
				
				\[
				\begin{prooftree}
				\hypo{ ⊢ A, Δ}
				\infer1{B \& ¬A ⊢ Δ}
				\end{prooftree}
				\quad
				\begin{prooftree}
				\hypo{Γ, B ⊢ Δ}
				\infer1{Γ, B \& ¬A ⊢ Δ}
				\end{prooftree}
				\quad
				\begin{prooftree}
				\hypo{Γ ⊢ B}
				\hypo{Γ,A ⊢ }
				\infer2{Γ ⊢ B \& ¬ A}
				\end{prooftree}
				\quad
				\begin{prooftree}
				\hypo{Γ,A ⊢ }
				\hypo{Γ ⊢ B}
				\infer2{Γ ⊢ ¬A \& B}
				\end{prooftree}
				\]
				
				Conjunction and Disjunction
				
				\[
				\begin{prooftree}
				\hypo{A ⊢ Δ}
				\hypo{B ⊢ Δ}
				\infer2{A ⊕ B ⊢ Δ}
				\end{prooftree}
				\quad
				\begin{prooftree}
				\hypo{Γ ⊢ A}
				\infer1{Γ ⊢ A ⊕ B}
				\end{prooftree}
				\quad
				\begin{prooftree}
				\hypo{Γ ⊢ B}
				\infer1{Γ ⊢ A ⊕ B}
				\end{prooftree}
				\]
				
				\[
				\begin{prooftree}
				\hypo{B ⊢ Δ}
				\hypo{A ⊢ Δ}
				\infer2{B ⊕ A ⊢ Δ}
				\end{prooftree}
				\quad
				\begin{prooftree}
				\hypo{Γ ⊢ A}
				\infer1{Γ ⊢ B ⊕ A}
				\end{prooftree}
				\quad
				\begin{prooftree}
				\hypo{Γ ⊢ B}
				\infer1{Γ ⊢ B ⊕ A}
				\end{prooftree}
				\]
				
				\[
				\begin{prooftree}
				\hypo{A ⊢ Δ}
				\infer1{A \& B ⊢ Δ}
				\end{prooftree}
				\quad
				\begin{prooftree}
				\hypo{B ⊢ Δ}
				\infer1{A \& B ⊢ Δ}
				\end{prooftree}
				\quad
				\begin{prooftree}
				\hypo{Γ ⊢ A}
				\hypo{Γ ⊢ B}
				\infer2{Γ ⊢ A \& B}
				\end{prooftree}
				\]
				
				\[
				\begin{prooftree}
				\hypo{A ⊢ Δ}
				\infer1{B \& A ⊢ Δ}
				\end{prooftree}
				\quad
				\begin{prooftree}
				\hypo{B ⊢ Δ}
				\infer1{B \& A ⊢ Δ}
				\end{prooftree}
				\quad
				\begin{prooftree}
				\hypo{Γ ⊢ B}
				\hypo{Γ ⊢ A}
				\infer2{Γ ⊢ B \& A}
				\end{prooftree}
				\]
				Biconditionals
				\[
				\begin{prooftree}
				\hypo{⊢ A, Δ}
				\hypo{B ⊢ Δ}
				\infer2{(¬A ⊕ B) \& (¬B ⊕ A) ⊢ Δ}
				\end{prooftree}
				\quad
				\begin{prooftree}
				\hypo{⊢ B, Δ}
				\hypo{A ⊢ Δ}
				\infer2{(¬A ⊕ B) \& (¬B ⊕ A) ⊢ Δ}
				\end{prooftree}
				\quad
				\begin{prooftree}
				\hypo{Γ, A ⊢ }
				\hypo{Γ, B ⊢ }
				\infer2{Γ ⊢ (¬A ⊕ B) \& (¬B ⊕ A)}
				\end{prooftree}
				\quad
				\begin{prooftree}
				\hypo{Γ ⊢ B, Δ}
				\hypo{Γ ⊢ A, Δ}
				\infer2{Γ ⊢ (¬A ⊕ B) \& (¬B ⊕ A), Δ}
				\end{prooftree}
				\]
				
				\[
				\begin{prooftree}
				\hypo{A ⊢ Δ}
				\hypo{⊢A, Δ}
				\infer2{(A \& B) ⊕  (¬A \& ¬B) ⊢ Δ}
				\end{prooftree}
				\quad
				\begin{prooftree}
				\hypo{A ⊢ Δ}
				\hypo{⊢B, Δ}
				\infer2{(A \& B) ⊕  (¬A \& ¬B) ⊢ Δ}
				\end{prooftree}
				\quad
				\begin{prooftree}
				\hypo{B ⊢ Δ}
				\hypo{⊢A, Δ}
				\infer2{(A \& B) ⊕  (¬A \& ¬B) ⊢ Δ}
				\end{prooftree}
				\quad
				\begin{prooftree}
				\hypo{B ⊢ Δ}
				\hypo{⊢B, Δ}
				\infer2{(A \& B) ⊕  (¬A \& ¬B) ⊢ Δ}
				\end{prooftree}
				\]
				
				\[
				\quad
				\begin{prooftree}
				\hypo{Γ ⊢ A, Δ}
				\hypo{Γ ⊢ B, Δ}
				\infer2{Γ ⊢ (A \& B) ⊕  (¬A \& ¬B),Δ}
				\end{prooftree}
				\quad
				\begin{prooftree}
				\hypo{Γ, B ⊢ }
				\hypo{Γ, A ⊢ }
				\infer2{Γ ⊢ (A \& B) ⊕  (¬A \& ¬B)}
				\end{prooftree}
				\]
				
				TODO: Apartness
				
			\end{center}
		
		\part{Theorems}
			\begin{center}
			\end{center}
	\end{center}


\end{document}}*}


\chapter{Order Additive NANDNOR Sequent}



\begin{abstract}
Strictly visible additive order calculus.
\end{abstract}

\section{Structural Rules}

\begin{center}
	\[
	\begin{prooftree}
	\infer0[Id]{A $\vdash$  A}
	\end{prooftree}
	\]
	
	\[
	\begin{prooftree}
	\hypo{$\Gamma$  $\vdash$  A}
	\hypo{A $\vdash$  $\Delta$ }
	\infer2[cut]{$\Gamma$  $\vdash$  $\Delta$ }
	\end{prooftree}
	\]
\end{center}

\section{Operational Rules}
	\begin{center}
				\subsection{Negation}
				\begin{center}
					\[
					\begin{prooftree}
					\hypo{ $\vdash$  A}
					\infer1{$\neg$  A $\vdash$ }
					\end{prooftree}
					\quad
					\begin{prooftree}
					\hypo{A $\vdash$  }
					\infer1{$\vdash$  $\neg$  A}
					\end{prooftree}
					\]
				\end{center}
				
				
				\subsection{Negative Junctions}
				\begin{center}
					\[
					\begin{prooftree}
					\hypo{$\vdash$  A}
					\hypo{$\vdash$  B}
					\infer2{$\neg$ A $\oplus$  $\neg$ B $\vdash$  }
					\end{prooftree}
					\quad
					\begin{prooftree}
					\hypo{A $\vdash$  }
					\infer1{ $\vdash$  $\neg$ A $\oplus$  $\neg$ B}
					\end{prooftree}
					\quad
					\begin{prooftree}
					\hypo{B $\vdash$  }
					\infer1{ $\vdash$  $\neg$ A $\oplus$  $\neg$ B}
					\end{prooftree}
					\]
					
					\[
					\begin{prooftree}
					\hypo{$\vdash$  B}
					\hypo{ $\vdash$  A}
					\infer2{$\neg$ B $\oplus$  $\neg$ A $\vdash$  }
					\end{prooftree}
					\quad
					\begin{prooftree}
					\hypo{A $\vdash$  }
					\infer1{ $\vdash$  $\neg$ B $\oplus$  $\neg$ A}
					\end{prooftree}
					\quad
					\begin{prooftree}
					\hypo{B $\vdash$  }
					\infer1{ $\vdash$  $\neg$ B $\oplus$  $\neg$ A}
					\end{prooftree}
					\]
					
					\[
					\begin{prooftree}
					\hypo{ $\vdash$  A}
					\infer1{$\neg$ A \& $\neg$ B $\vdash$  }
					\end{prooftree}
					\quad
					\begin{prooftree}
					\hypo{ $\vdash$  B}
					\infer1{$\neg$ A \& $\neg$ B $\vdash$  }
					\end{prooftree}
					\quad
					\begin{prooftree}
					\hypo{A $\vdash$  }
					\hypo{B $\vdash$  }
					\infer2{$\vdash$  $\neg$ A \& $\neg$ B}
					\end{prooftree}
					\]
					
					\[
					\begin{prooftree}
					\hypo{ $\vdash$  A}
					\infer1{$\neg$ B \& $\neg$ A $\vdash$  }
					\end{prooftree}
					\quad
					\begin{prooftree}
					\hypo{ $\vdash$  B}
					\infer1{$\neg$ B \& $\neg$ A $\vdash$  }
					\end{prooftree}
					\quad
					\begin{prooftree}
					\hypo{B $\vdash$  }
					\hypo{A $\vdash$  }
					\infer2{ $\vdash$  $\neg$  B \& $\neg$  A}
					\end{prooftree}
					\]
				\end{center}

				\subsection{Conditionals}
				\begin{center}
					\[
					\begin{prooftree}
					\hypo{$\vdash$  B}
					\hypo{A $\vdash$  }
					\infer2{$\neg$ B $\oplus$  A $\vdash$ }
					\end{prooftree}
					\quad
					\begin{prooftree}
					\hypo{A $\vdash$  }
					\hypo{$\vdash$  B}
					\infer2{A $\oplus$  $\neg$ B $\vdash$ }
					\end{prooftree}
					\quad
					\begin{prooftree}
					\hypo{$\Gamma$  $\vdash$  A}
					\infer1{$\Gamma$  $\vdash$  A $\oplus$  $\neg$ B}
					\end{prooftree}
					\quad
					\begin{prooftree}
					\hypo{B $\vdash$  }
					\infer1{$\vdash$  A $\oplus$  $\neg$ B}
					\end{prooftree}
					\]
					
					\[
					\begin{prooftree}
					\hypo{ $\vdash$  A}
					\hypo{B $\vdash$  }
					\infer2{$\neg$ A $\oplus$  B $\vdash$  }
					\end{prooftree}
					\quad
					\begin{prooftree}
					\hypo{B $\vdash$  }
					\hypo{ $\vdash$  A}
					\infer2{B $\oplus$  $\neg$ A $\vdash$  }
					\end{prooftree}
					\quad
					\begin{prooftree}
					\hypo{A $\vdash$  }
					\infer1{ $\vdash$  B $\oplus$  $\neg$ A}
					\end{prooftree}
					\quad
					\begin{prooftree}
					\hypo{$\Gamma$  $\vdash$  B}
					\infer1{$\Gamma$  $\vdash$  B $\oplus$  $\neg$ A}
					\end{prooftree}
					\]
					
					\[
					\begin{prooftree}
					\hypo{A $\vdash$  $\Delta$ }
					\infer1{A \& $\neg$ B $\vdash$  $\Delta$ }
					\end{prooftree}
					\quad
					\begin{prooftree}
					\hypo{ $\vdash$  B}
					\infer1{A \& $\neg$ B $\vdash$  }
					\end{prooftree}
					\quad
					\begin{prooftree}
					\hypo{$\vdash$  A}
					\hypo{B $\vdash$  }
					\infer2{$\vdash$  A \& $\neg$ B}
					\end{prooftree}
					\quad
					\begin{prooftree}
					\hypo{B $\vdash$  }
					\hypo{$\vdash$  A}
					\infer2{$\vdash$  $\neg$ B \& A}
					\end{prooftree}
					\]
					
					\[
					\begin{prooftree}
					\hypo{ $\vdash$  A}
					\infer1{B \& $\neg$ A $\vdash$  }
					\end{prooftree}
					\quad
					\begin{prooftree}
					\hypo{B $\vdash$  $\Delta$ }
					\infer1{B \& $\neg$ A $\vdash$  $\Delta$ }
					\end{prooftree}
					\quad
					\begin{prooftree}
					\hypo{ $\vdash$  B}
					\hypo{A $\vdash$  }
					\infer2{ $\vdash$  B \& $\neg$  A}
					\end{prooftree}
					\quad
					\begin{prooftree}
					\hypo{A $\vdash$  }
					\hypo{ $\vdash$  B}
					\infer2{ $\vdash$  $\neg$ A \& B}
					\end{prooftree}
					\]
				\end{center}
				
				\subsection{Junctions}
				\begin{center}
					\[
					\begin{prooftree}
					\hypo{A $\vdash$  $\Delta$ }
					\hypo{B $\vdash$  $\Delta$ }
					\infer2{A $\oplus$  B $\vdash$  $\Delta$ }
					\end{prooftree}
					\quad
					\begin{prooftree}
					\hypo{$\Gamma$  $\vdash$  A}
					\infer1{$\Gamma$  $\vdash$  A $\oplus$  B}
					\end{prooftree}
					\quad
					\begin{prooftree}
					\hypo{$\Gamma$  $\vdash$  B}
					\infer1{$\Gamma$  $\vdash$  A $\oplus$  B}
					\end{prooftree}
					\]
					
					\[
					\begin{prooftree}
					\hypo{B $\vdash$  $\Delta$ }
					\hypo{A $\vdash$  $\Delta$ }
					\infer2{B $\oplus$  A $\vdash$  $\Delta$ }
					\end{prooftree}
					\quad
					\begin{prooftree}
					\hypo{$\Gamma$  $\vdash$  A}
					\infer1{$\Gamma$  $\vdash$  B $\oplus$  A}
					\end{prooftree}
					\quad
					\begin{prooftree}
					\hypo{$\Gamma$  $\vdash$  B}
					\infer1{$\Gamma$  $\vdash$  B $\oplus$  A}
					\end{prooftree}
					\]
					
					\[
					\begin{prooftree}
					\hypo{A $\vdash$  $\Delta$ }
					\infer1{A \& B $\vdash$  $\Delta$ }
					\end{prooftree}
					\quad
					\begin{prooftree}
					\hypo{B $\vdash$  $\Delta$ }
					\infer1{A \& B $\vdash$  $\Delta$ }
					\end{prooftree}
					\quad
					\begin{prooftree}
					\hypo{$\Gamma$  $\vdash$  A}
					\hypo{$\Gamma$  $\vdash$  B}
					\infer2{$\Gamma$  $\vdash$  A \& B}
					\end{prooftree}
					\]
					
					\[
					\begin{prooftree}
					\hypo{A $\vdash$  $\Delta$ }
					\infer1{B \& A $\vdash$  $\Delta$ }
					\end{prooftree}
					\quad
					\begin{prooftree}
					\hypo{B $\vdash$  $\Delta$ }
					\infer1{B \& A $\vdash$  $\Delta$ }
					\end{prooftree}
					\quad
					\begin{prooftree}
					\hypo{$\Gamma$  $\vdash$  B}
					\hypo{$\Gamma$  $\vdash$  A}
					\infer2{$\Gamma$  $\vdash$  B \& A}
					\end{prooftree}
					\]
				\end{center}
			
				\subsection{Biconditionals}
				\begin{center}
					\[
					\begin{prooftree}
					\hypo{$\vdash$  A}
					\hypo{B $\vdash$  }
					\infer2{($\neg$ A $\oplus$  B) \& ($\neg$ B $\oplus$  A) $\vdash$  }
					\end{prooftree}
					\quad
					\begin{prooftree}
					\hypo{$\vdash$  B}
					\hypo{A $\vdash$  }
					\infer2{($\neg$ A $\oplus$  B) \& ($\neg$ B $\oplus$  A) $\vdash$  }
					\end{prooftree}
					\quad
					\begin{prooftree}
					\hypo{A $\vdash$  }
					\hypo{B $\vdash$  }
					\infer2{ $\vdash$  ($\neg$ A $\oplus$  B) \& ($\neg$ B $\oplus$  A)}
					\end{prooftree}
					\quad
					\begin{prooftree}
					\hypo{$\Gamma$  $\vdash$  B}
					\hypo{$\Gamma$  $\vdash$  A}
					\infer2{$\Gamma$  $\vdash$  ($\neg$ A $\oplus$  B) \& ($\neg$ B $\oplus$  A)}
					\end{prooftree}
					\]
					
					\[
					\begin{prooftree}
					\hypo{A $\vdash$  }
					\hypo{$\vdash$ A}
					\infer2{(A \& B) $\oplus$   ($\neg$ A \& $\neg$ B) $\vdash$  }
					\end{prooftree}
					\quad
					\begin{prooftree}
					\hypo{A $\vdash$  }
					\hypo{$\vdash$ B}
					\infer2{(A \& B) $\oplus$   ($\neg$ A \& $\neg$ B) $\vdash$  }
					\end{prooftree}
					\quad
					\begin{prooftree}
					\hypo{B $\vdash$  }
					\hypo{$\vdash$ A}
					\infer2{(A \& B) $\oplus$   ($\neg$ A \& $\neg$ B) $\vdash$  }
					\end{prooftree}
					\quad
					\begin{prooftree}
					\hypo{B $\vdash$  }
					\hypo{$\vdash$ B}
					\infer2{(A \& B) $\oplus$   ($\neg$ A \& $\neg$ B) $\vdash$  }
					\end{prooftree}
					\]
					
					\[
					\quad
					\begin{prooftree}
					\hypo{$\Gamma$  $\vdash$  A}
					\hypo{$\Gamma$  $\vdash$  B}
					\infer2{$\Gamma$  $\vdash$  (A \& B) $\oplus$   ($\neg$ A \& $\neg$ B)}
					\end{prooftree}
					\quad
					\begin{prooftree}
					\hypo{B $\vdash$  }
					\hypo{A $\vdash$  }
					\infer2{ $\vdash$  (A \& B) $\oplus$   ($\neg$ A \& $\neg$ B)}
					\end{prooftree}
					\]
					
					TODO: XOR-like
				\end{center}
\end{center}

\part{Theorems}
	\begin{center}
	\end{center}



\chapter{Order Basic Sequent}






\usepackage[utf8]{inputenc}


































\section{Structural Rules}

\begin{center}
	\[
	\begin{prooftree}
	\infer0[Id]{A $\vdash$  A}
	\end{prooftree}
	\]
	
	\[
	\begin{prooftree}
	\hypo{$\Gamma$  $\vdash$  A}
	\hypo{A $\vdash$  $\Delta$ }
	\infer2[cut]{$\Gamma$  $\vdash$  $\Delta$ }
	\end{prooftree}
	\]
\end{center}

\section{Unit Rules}
\begin{center}
	\[
	\begin{prooftree}
	\infer0{0 $\vdash$  A}
	\end{prooftree}
	\quad
	\begin{prooftree}
	\infer0{A $\vdash$  $\top$ }
	\end{prooftree}
	\]
\end{center}

\newpage
\section{Operational Rules}
\begin{center}
	\subsection{Negations}
	\begin{center}	\[
		\begin{prooftree}
		\hypo{ $\vdash$  A }
		\infer1{ $\neg$  A $\vdash$  }
		\end{prooftree}
		\quad
		\begin{prooftree}
		\hypo{ A $\vdash$  }
		\infer1{ $\vdash$  $\neg$  A}
		\end{prooftree}
		\]
	\end{center}

	\subsection{Additives}
	\begin{center}
		\subsubsection{Junctions}
			\begin{center}
				
				\[
				\begin{prooftree}
				\hypo{A $\vdash$  $\Delta$ }
				\hypo{B $\vdash$  $\Delta$ }
				\infer2{A $\oplus$  B $\vdash$  $\Delta$ }
				\end{prooftree}
				\quad
				\begin{prooftree}
				\hypo{$\Gamma$  $\vdash$  A}
				\infer1{$\Gamma$  $\vdash$  A $\oplus$  B}
				\end{prooftree}
				\quad
				\begin{prooftree}
				\hypo{$\Gamma$  $\vdash$  B}
				\infer1{$\Gamma$  $\vdash$  A $\oplus$  B}
				\end{prooftree}
				\]
				
				\[
				\begin{prooftree}
				\hypo{B $\vdash$  $\Delta$ }
				\hypo{A $\vdash$  $\Delta$ }
				\infer2{B $\oplus$  A $\vdash$  $\Delta$ }
				\end{prooftree}
				\quad
				\begin{prooftree}
				\hypo{$\Gamma$  $\vdash$  A}
				\infer1{$\Gamma$  $\vdash$  B $\oplus$  A}
				\end{prooftree}
				\quad
				\begin{prooftree}
				\hypo{$\Gamma$  $\vdash$  B}
				\infer1{$\Gamma$  $\vdash$  B $\oplus$  A}
				\end{prooftree}
				\]
				
				\[
				\begin{prooftree}
				\hypo{A $\vdash$  $\Delta$ }
				\infer1{A \& B $\vdash$  $\Delta$ }
				\end{prooftree}
				\quad
				\begin{prooftree}
				\hypo{B $\vdash$  $\Delta$ }
				\infer1{A \& B $\vdash$  $\Delta$ }
				\end{prooftree}
				\quad
				\begin{prooftree}
				\hypo{$\Gamma$  $\vdash$  A}
				\hypo{$\Gamma$  $\vdash$  B}
				\infer2{$\Gamma$  $\vdash$  A \& B}
				\end{prooftree}
				\]
				
				\[
				\begin{prooftree}
				\hypo{A $\vdash$  $\Delta$ }
				\infer1{B \& A $\vdash$  $\Delta$ }
				\end{prooftree}
				\quad
				\begin{prooftree}
				\hypo{B $\vdash$  $\Delta$ }
				\infer1{B \& A $\vdash$  $\Delta$ }
				\end{prooftree}
				\quad
				\begin{prooftree}
				\hypo{$\Gamma$  $\vdash$  B}
				\hypo{$\Gamma$  $\vdash$  A}
				\infer2{$\Gamma$  $\vdash$  B \& A}
				\end{prooftree}
				\]
			\end{center}
		
		\subsubsection{Implications}
			\begin{center}
				\[
				\begin{prooftree}
				\hypo{ $\vdash$  A}
				\hypo{B $\vdash$  $\Delta$ }
				\infer2{$\neg$  A $\oplus$  B $\vdash$  $\Delta$ }
				\end{prooftree}
				\quad
				\begin{prooftree}
				\hypo{A $\vdash$  }
				\infer1{ $\vdash$  $\neg$ A $\oplus$  B}
				\end{prooftree}
				\quad
				\begin{prooftree}
				\hypo{$\Gamma$  $\vdash$  B}
				\infer1{$\Gamma$  $\vdash$  $\neg$ A $\oplus$  B}
				\end{prooftree}
				\]
				
				\[
				\begin{prooftree}
				\hypo{B $\vdash$  $\Delta$ }
				\hypo{ $\vdash$  A}
				\infer2{B $\oplus$  $\neg$ A $\vdash$  $\Delta$ }
				\end{prooftree}
				\quad
				\begin{prooftree}
				\hypo{A $\vdash$  }
				\infer1{$\vdash$  B $\oplus$  $\neg$ A}
				\end{prooftree}
				\quad
				\begin{prooftree}
				\hypo{$\Gamma$  $\vdash$  B}
				\infer1{$\Gamma$  $\vdash$  B $\oplus$  $\neg$ A}
				\end{prooftree}
				\]
				
				\[
				\begin{prooftree}
				\hypo{$\vdash$  A}
				\infer1{$\neg$  A \& B $\vdash$  }
				\end{prooftree}
				\quad
				\begin{prooftree}
				\hypo{B $\vdash$  $\Delta$ }
				\infer1{$\neg$ A \& B $\vdash$  $\Delta$ }
				\end{prooftree}
				\quad
				\begin{prooftree}
				\hypo{A $\vdash$  }
				\hypo{$\Gamma$  $\vdash$  B}
				\infer2{$\Gamma$  $\vdash$  $\neg$ A \& B}
				\end{prooftree}
				\]
				
				\[
				\begin{prooftree}
				\hypo{ $\vdash$  A}
				\infer1{B \& $\neg$  A $\vdash$ }
				\end{prooftree}
				\quad
				\begin{prooftree}
				\hypo{B $\vdash$  $\Delta$ }
				\infer1{B \& $\neg$ A $\vdash$  $\Delta$ }
				\end{prooftree}
				\quad
				\begin{prooftree}
				\hypo{$\Gamma$  $\vdash$  B}
				\hypo{A $\vdash$  }
				\infer2{$\Gamma$  $\vdash$  B \& $\neg$ A}
				\end{prooftree}
				\]
			\end{center}
		
		\subsubsection{NAND and NOR}
			\begin{center}
				\[
				\begin{prooftree}
				\hypo{ $\vdash$  A}
				\hypo{ $\vdash$  B}
				\infer2{$\neg$  A $\oplus$  $\neg$ B $\vdash$ }
				\end{prooftree}
				\quad
				\begin{prooftree}
				\hypo{A $\vdash$  }
				\infer1{ $\vdash$  $\neg$ A $\oplus$  $\neg$ B}
				\end{prooftree}
				\quad
				\begin{prooftree}
				\hypo{B $\vdash$  }
				\infer1{$\vdash$  $\neg$ A $\oplus$  $\neg$ B}
				\end{prooftree}
				\]
				
				
				
				\[
				\begin{prooftree}
				\hypo{ $\vdash$  B}
				\hypo{ $\vdash$  A}
				\infer2{$\neg$ B $\oplus$  $\neg$ A $\vdash$  }
				\end{prooftree}
				\quad
				\begin{prooftree}
				\hypo{A $\vdash$  }
				\infer1{$\vdash$  $\neg$ B $\oplus$  $\neg$ A}
				\end{prooftree}
				\quad
				\begin{prooftree}
				\hypo{B $\vdash$  }
				\infer1{$\vdash$  $\neg$ B $\oplus$  $\neg$ A}
				\end{prooftree}
				\]
				
				\[
				\begin{prooftree}
				\hypo{$\vdash$  A}
				\infer1{$\neg$  A \& $\neg$ B $\vdash$  }
				\end{prooftree}
				\quad
				\begin{prooftree}
				\hypo{$\vdash$  B}
				\infer1{$\neg$ A \& $\neg$ B $\vdash$  }
				\end{prooftree}
				\quad
				\begin{prooftree}
				\hypo{A $\vdash$  }
				\hypo{B $\vdash$  }
				\infer2{$\vdash$  $\neg$ A \& $\neg$ B}
				\end{prooftree}
				\]
				
				\[
				\begin{prooftree}
				\hypo{ $\vdash$  A}
				\infer1{$\neg$ B \& $\neg$  A $\vdash$ }
				\end{prooftree}
				\quad
				\begin{prooftree}
				\hypo{$\vdash$  B}
				\infer1{$\neg$ B \& $\neg$ A $\vdash$ }
				\end{prooftree}
				\quad
				\begin{prooftree}
				\hypo{B $\vdash$  }
				\hypo{A $\vdash$  }
				\infer2{$\vdash$  $\neg$ B \& $\neg$ A}
				\end{prooftree}
				\]
			\end{center}
	\end{center}

		\subsection{Multiplicatives}
	\begin{center}
		\subsubsection{Junctions}
		\begin{center}
			
			\[
			\begin{prooftree}
			\hypo{A $\vdash$  $\Delta$ _0}
			\hypo{B $\vdash$  $\Delta$ _1}
			\infer2{A ℘ B $\vdash$  $\Delta$ _0, $\Delta$ _1}
			\end{prooftree}
			\quad
			\begin{prooftree}
			\hypo{$\Gamma$  $\vdash$  A, B}
			\infer1{$\Gamma$  $\vdash$  A ℘ B}
			\end{prooftree}
			\]

			\[
			\begin{prooftree}
			\hypo{B $\vdash$  $\Delta$ _0}
			\hypo{A $\vdash$  $\Delta$ _1}
			\infer2{B ℘ A $\vdash$  $\Delta$ _0, $\Delta$ _1}
			\end{prooftree}
			\quad
			\begin{prooftree}
			\hypo{$\Gamma$  $\vdash$  B, A}
			\infer1{$\Gamma$  $\vdash$  B ℘ A}
			\end{prooftree}
			\]
			
			\[
			\begin{prooftree}
			\hypo{A, B $\vdash$  $\Delta$ }
			\infer1{A $\otimes$  B $\vdash$  $\Delta$ }
			\end{prooftree}
			\quad
			\begin{prooftree}
			\hypo{$\Gamma$ _0 $\vdash$  A}
			\hypo{$\Gamma$ _1 $\vdash$  B}
			\infer2{$\Gamma$ _0, $\Gamma$ _1 $\vdash$  A $\otimes$  B}
			\end{prooftree}
			\]
			
			\[
			\begin{prooftree}
			\hypo{B, A $\vdash$  $\Delta$ }
			\infer1{B $\otimes$  A $\vdash$  $\Delta$ }
			\end{prooftree}
			\quad
			\begin{prooftree}
			\hypo{$\Gamma$ _0 $\vdash$  B}
			\hypo{$\Gamma$ _1 $\vdash$  A}
			\infer2{$\Gamma$ _0, $\Gamma$ _1 $\vdash$  B $\otimes$  A}
			\end{prooftree}
			\]
		\end{center}
		
		\subsubsection{Implications}
		\begin{center}
			\[
			\begin{prooftree}
			\hypo{ $\vdash$  A}
			\hypo{B $\vdash$  $\Delta$ }
			\infer2{A $\to$  B $\vdash$  $\Delta$ }
			\end{prooftree}
			\quad
			\begin{prooftree}
			\hypo{A $\vdash$  B}
			\infer1{ $\vdash$  A $\to$  B}
			\end{prooftree}
			\]
			
			\[
			\begin{prooftree}
			\hypo{B $\vdash$  $\Delta$ }
			\hypo{ $\vdash$  A}
			\infer2{B $\oplus$  $\neg$ A $\vdash$  $\Delta$ }
			\end{prooftree}
			\quad
			\begin{prooftree}
			\hypo{A $\vdash$  }
			\infer1{$\vdash$  B $\oplus$  $\neg$ A}
			\end{prooftree}
			\quad
			\begin{prooftree}
			\hypo{$\Gamma$  $\vdash$  B}
			\infer1{$\Gamma$  $\vdash$  B $\oplus$  $\neg$ A}
			\end{prooftree}
			\]
			
			\[
			\begin{prooftree}
			\hypo{$\vdash$  A}
			\infer1{$\neg$  A \& B $\vdash$  }
			\end{prooftree}
			\quad
			\begin{prooftree}
			\hypo{B $\vdash$  $\Delta$ }
			\infer1{$\neg$ A \& B $\vdash$  $\Delta$ }
			\end{prooftree}
			\quad
			\begin{prooftree}
			\hypo{A $\vdash$  }
			\hypo{$\Gamma$  $\vdash$  B}
			\infer2{$\Gamma$  $\vdash$  $\neg$ A \& B}
			\end{prooftree}
			\]
			
			\[
			\begin{prooftree}
			\hypo{ $\vdash$  A}
			\infer1{B \& $\neg$  A $\vdash$ }
			\end{prooftree}
			\quad
			\begin{prooftree}
			\hypo{B $\vdash$  $\Delta$ }
			\infer1{B \& $\neg$ A $\vdash$  $\Delta$ }
			\end{prooftree}
			\quad
			\begin{prooftree}
			\hypo{$\Gamma$  $\vdash$  B}
			\hypo{A $\vdash$  }
			\infer2{$\Gamma$  $\vdash$  B \& $\neg$ A}
			\end{prooftree}
			\]
		\end{center}
		
		\subsubsection{NAND and NOR}
		\begin{center}
			\[
			\begin{prooftree}
			\hypo{ $\vdash$  A}
			\hypo{ $\vdash$  B}
			\infer2{$\neg$  A $\oplus$  $\neg$ B $\vdash$ }
			\end{prooftree}
			\quad
			\begin{prooftree}
			\hypo{A $\vdash$  }
			\infer1{ $\vdash$  $\neg$ A $\oplus$  $\neg$ B}
			\end{prooftree}
			\quad
			\begin{prooftree}
			\hypo{B $\vdash$  }
			\infer1{$\vdash$  $\neg$ A $\oplus$  $\neg$ B}
			\end{prooftree}
			\]
			
			
			
			\[
			\begin{prooftree}
			\hypo{ $\vdash$  B}
			\hypo{ $\vdash$  A}
			\infer2{$\neg$ B $\oplus$  $\neg$ A $\vdash$  }
			\end{prooftree}
			\quad
			\begin{prooftree}
			\hypo{A $\vdash$  }
			\infer1{$\vdash$  $\neg$ B $\oplus$  $\neg$ A}
			\end{prooftree}
			\quad
			\begin{prooftree}
			\hypo{B $\vdash$  }
			\infer1{$\vdash$  $\neg$ B $\oplus$  $\neg$ A}
			\end{prooftree}
			\]
			
			\[
			\begin{prooftree}
			\hypo{$\vdash$  A}
			\infer1{$\neg$  A \& $\neg$ B $\vdash$  }
			\end{prooftree}
			\quad
			\begin{prooftree}
			\hypo{$\vdash$  B}
			\infer1{$\neg$ A \& $\neg$ B $\vdash$  }
			\end{prooftree}
			\quad
			\begin{prooftree}
			\hypo{A $\vdash$  }
			\hypo{B $\vdash$  }
			\infer2{$\vdash$  $\neg$ A \& $\neg$ B}
			\end{prooftree}
			\]
			
			\[
			\begin{prooftree}
			\hypo{ $\vdash$  A}
			\infer1{$\neg$ B \& $\neg$  A $\vdash$ }
			\end{prooftree}
			\quad
			\begin{prooftree}
			\hypo{$\vdash$  B}
			\infer1{$\neg$ B \& $\neg$ A $\vdash$ }
			\end{prooftree}
			\quad
			\begin{prooftree}
			\hypo{B $\vdash$  }
			\hypo{A $\vdash$  }
			\infer2{$\vdash$  $\neg$ B \& $\neg$ A}
			\end{prooftree}
			\]
		\end{center}
	\end{center}
\end{center}

\part{Theorems}
	\begin{center}

		\[
		\begin{prooftree}
		\infer0{ $\vdash$  A $\oplus$  $\neg$ A}
		\end{prooftree}
		\quad
		\begin{prooftree}
		\infer0{ $\vdash$  $\neg$ A $\oplus$  A}
		\end{prooftree}
		\quad
		\begin{prooftree}
		\infer0{A \& $\neg$ A $\vdash$  }
		\end{prooftree}
		\quad
		\begin{prooftree}
		\infer0{$\neg$ A \& A $\vdash$  }
		\end{prooftree}
		\]
		
		\[
		\begin{prooftree}
		\infer0{0 $\vdash$  C}
		\infer1{0 \& $\neg$  A $\vdash$  C}
		\end{prooftree}
		\quad
		\begin{prooftree}
		\infer0{C $\vdash$  $\top$ }
		\infer1{C $\vdash$  $\top$  $\oplus$  $\neg$  A}
		\end{prooftree}		
		\]
		
		\[
		\begin{prooftree}
		\infer0{0 $\vdash$  A}
		\infer1{0 $\vdash$  A $\oplus$  $\neg$  A}
		\end{prooftree}
		\quad
		\begin{prooftree}
		\infer0{0 $\vdash$  $\neg$  A}
		\infer1{0 $\vdash$  A $\oplus$  $\neg$  A}
		\end{prooftree}
		\quad
		\begin{prooftree}
		\infer0{0 $\vdash$  }
		\infer1{$\vdash$  A $\oplus$  $\neg$  0}
		\end{prooftree}
		\quad
		\begin{prooftree}
		\infer0{C $\vdash$  $\top$ }
		\infer1{C $\vdash$  $\top$  $\oplus$  A}
		\end{prooftree}
		\]
		
		\[
		\begin{prooftree}
		\infer0{A $\vdash$  $\top$ }
		\infer1{A \& $\neg$  A $\vdash$  $\top$ }
		\end{prooftree}
		\quad
		\begin{prooftree}
		\infer0{$\neg$  A $\vdash$  $\top$ }
		\infer1{A \& $\neg$  A $\vdash$  $\top$ }
		\end{prooftree}
		\]
		
		\[
		\begin{prooftree}
		\infer0{1 $\vdash$  1}
		\infer1{1 $\vdash$  1 $\oplus$  $\neg$  1}
		\end{prooftree}
		\quad
		\begin{prooftree}
		\infer0{0 $\vdash$  1 $\oplus$  $\neg$  1}
		\end{prooftree}
		\quad
		\begin{prooftree}
		\hypo{$\bot$  $\vdash$  1 $\oplus$  $\neg$  1}
		\end{prooftree}
		\quad
		\begin{prooftree}
		\hypo{$\top$  $\vdash$  1 $\oplus$  $\neg$  1}
		\end{prooftree}
		\]
		
		\[
		\begin{prooftree}
		\infer0{$\bot$ $\vdash$ }
		\infer1{$\vdash$  $\neg$  $\bot$ }
		\infer1{$\vdash$  $\bot$  $\oplus$  $\neg$  $\bot$ }
		\infer1{1 $\vdash$  $\bot$  $\oplus$  $\neg$  $\bot$ }
		\end{prooftree}
		\quad
		\begin{prooftree}
		\infer0{0 $\vdash$  $\bot$  $\oplus$  $\neg$  $\bot$ }
		\end{prooftree}
		\quad
		\begin{prooftree}
		\infer0{$\bot$  $\vdash$  $\bot$ }
		\infer1{$\bot$  $\vdash$  $\bot$  $\oplus$  $\neg$  $\bot$ }
		\end{prooftree}
		\quad
		\begin{prooftree}
		\hypo{$\top$  $\vdash$  $\bot$  $\oplus$  $\neg$  $\bot$ }
		\end{prooftree}
		\]
		
		\[
		\begin{prooftree}
		\infer0{0 $\vdash$ }
		\infer1{$\vdash$  $\neg$  0}
		\infer1{$\vdash$  0 $\oplus$  $\neg$  0}
		\infer1{1 $\vdash$  0 $\oplus$  $\neg$  0}
		\end{prooftree}
		\quad
		\begin{prooftree}
		\infer0{0 $\vdash$  0 $\oplus$  $\neg$  0}
		\end{prooftree}
		\quad
		\begin{prooftree}
		\hypo{$\bot$  $\vdash$  0 $\oplus$  $\neg$  0}
		\end{prooftree}
		\quad
		\begin{prooftree}
		\hypo{$\top$  $\vdash$  0 $\oplus$  $\neg$  0}
		\end{prooftree}
		\]
		
		\[
		\begin{prooftree}
		\infer0{$\vdash$  $\top$ }
		\infer1{$\vdash$  $\top$  $\oplus$  $\neg$  $\top$ }
		\infer1{1 $\vdash$  $\top$  $\oplus$  $\neg$  $\top$ }
		\end{prooftree}
		\quad
		\begin{prooftree}
		\infer0{0 $\vdash$  $\top$  $\oplus$  $\neg$  $\top$ }
		\end{prooftree}
		\quad
		\begin{prooftree}
		\infer0{$\bot$  $\vdash$  $\top$ }
		\infer1{$\bot$  $\vdash$  $\top$  $\oplus$  $\neg$  $\top$ }
		\end{prooftree}
		\quad
		\begin{prooftree}
		\infer0{$\top$  $\vdash$  $\top$ }
		\infer1{$\top$  $\vdash$  $\top$  $\oplus$  $\neg$  $\top$ }
		\end{prooftree}
		\]
		
		\[
		\begin{prooftree}
		\infer0{1 $\vdash$  1}
		\infer1{1 \&  $\neg$  1$\vdash$  1}
		\end{prooftree}
		\quad
		\begin{prooftree}
		\hypo{1 \&  $\neg$  1$\vdash$  0}
		\end{prooftree}
		\quad
		\begin{prooftree}
		\infer0{$\vdash$  1}
		\infer1{$\neg$  1$\vdash$  }
		\infer1{1 \&  $\neg$  1$\vdash$  }
		\infer1{1 \&  $\neg$  1$\vdash$  $\bot$ }
		\end{prooftree}
		\quad
		\begin{prooftree}
		\infer0{1 \&  $\neg$  1 $\vdash$  $\top$ }
		\end{prooftree}
		\]
		
		\[
		\begin{prooftree}
		\infer0{0 $\vdash$  1}
		\infer1{0 \&  $\neg$  0 $\vdash$  1}
		\end{prooftree}
		\quad
		\begin{prooftree}
		\infer0{0 $\vdash$  0}
		\infer1{0 \&  $\neg$  0 $\vdash$  0}
		\end{prooftree}
		\quad
		\begin{prooftree}
		\infer0{0 $\vdash$  $\bot$ }
		\infer1{0 \&  $\neg$  0 $\vdash$  $\bot$ }
		\end{prooftree}
		\quad
		\begin{prooftree}
		\infer0{0 \&  $\neg$  0 $\vdash$  $\top$ }
		\end{prooftree}
		\]
		
		\[
		\begin{prooftree}
		\hypo{$\bot$  \&  $\neg$  $\bot$  $\vdash$  1}
		\end{prooftree}
		\quad
		\begin{prooftree}
		\hypo{$\bot$  \&  $\neg$  $\bot$  $\vdash$  0}
		\end{prooftree}
		\quad
		\begin{prooftree}
		\infer0{$\bot$  $\vdash$  }
		\infer1{$\bot$  \&  $\neg$  $\bot$  $\vdash$  }
		\infer1{$\bot$  \&  $\neg$  $\bot$  $\vdash$  $\bot$ }
		\end{prooftree}
		\quad
		\begin{prooftree}
		\infer0{$\bot$  \&  $\neg$  $\bot$  $\vdash$  $\top$ }
		\end{prooftree}
		\]
		
		\[
		\begin{prooftree}
		\hypo{$\top$  \&  $\neg$  $\top$  $\vdash$  1}
		\end{prooftree}
		\quad
		\begin{prooftree}
		\hypo{$\top$  \&  $\neg$  $\top$  $\vdash$  0}
		\end{prooftree}
		\quad
		\begin{prooftree}
		\infer0{$\vdash$  $\top$ }
		\infer1{$\neg$  $\top$  $\vdash$ }
		\infer1{$\neg$  $\top$  $\vdash$  $\bot$ }
		\infer1{$\top$  \&  $\neg$  $\top$  $\vdash$  $\bot$ }
		\end{prooftree}
		\quad
		\begin{prooftree}
		\infer0{$\top$  \&  $\neg$  $\top$  $\vdash$  $\top$ }
		\end{prooftree}
		\]
	\end{center}



\chapter{Ordinal Basic Sequent}



\begin{abstract}

\end{abstract}

\section{Structural Rules}

\begin{center}
	\[
	\begin{prooftree}
	\infer0[Id]{A $\vdash$  A}
	\end{prooftree}
	\]
	
	\[
	\begin{prooftree}
	\hypo{$\Gamma$  $\vdash$  A}
	\hypo{A $\vdash$  $\Delta$ }
	\infer2[cut]{$\Gamma$  $\vdash$  $\Delta$ }
	\end{prooftree}
	\]
\end{center}

\section{Unit Rules}
\begin{center}
	\[
	\begin{prooftree}
	\infer0{0 $\vdash$  A}
	\end{prooftree}
	\quad
	\begin{prooftree}
	\infer0{A $\vdash$  $\top$ }
	\end{prooftree}
	\]
\end{center}

\newpage
\section{Operational Rules}
\begin{center}
	\subsection{Negations}
	\begin{center}	\[
		\begin{prooftree}
		\hypo{ $\vdash$  A }
		\infer1{ $\neg$  A $\vdash$  }
		\end{prooftree}
		\quad
		\begin{prooftree}
		\hypo{ A $\vdash$  }
		\infer1{ $\vdash$  $\neg$  A}
		\end{prooftree}
		\]
	\end{center}

	\subsection{Additives}
	\begin{center}
		\subsubsection{Junctions}
			\begin{center}
				
				\[
				\begin{prooftree}
				\hypo{A $\vdash$  $\Delta$ }
				\hypo{B $\vdash$  $\Delta$ }
				\infer2{A $\oplus$  B $\vdash$  $\Delta$ }
				\end{prooftree}
				\quad
				\begin{prooftree}
				\hypo{$\Gamma$  $\vdash$  A}
				\infer1{$\Gamma$  $\vdash$  A $\oplus$  B}
				\end{prooftree}
				\quad
				\begin{prooftree}
				\hypo{$\Gamma$  $\vdash$  B}
				\infer1{$\Gamma$  $\vdash$  A $\oplus$  B}
				\end{prooftree}
				\]
				
				\[
				\begin{prooftree}
				\hypo{B $\vdash$  $\Delta$ }
				\hypo{A $\vdash$  $\Delta$ }
				\infer2{B $\oplus$  A $\vdash$  $\Delta$ }
				\end{prooftree}
				\quad
				\begin{prooftree}
				\hypo{$\Gamma$  $\vdash$  A}
				\infer1{$\Gamma$  $\vdash$  B $\oplus$  A}
				\end{prooftree}
				\quad
				\begin{prooftree}
				\hypo{$\Gamma$  $\vdash$  B}
				\infer1{$\Gamma$  $\vdash$  B $\oplus$  A}
				\end{prooftree}
				\]
				
				\[
				\begin{prooftree}
				\hypo{A $\vdash$  $\Delta$ }
				\infer1{A \& B $\vdash$  $\Delta$ }
				\end{prooftree}
				\quad
				\begin{prooftree}
				\hypo{B $\vdash$  $\Delta$ }
				\infer1{A \& B $\vdash$  $\Delta$ }
				\end{prooftree}
				\quad
				\begin{prooftree}
				\hypo{$\Gamma$  $\vdash$  A}
				\hypo{$\Gamma$  $\vdash$  B}
				\infer2{$\Gamma$  $\vdash$  A \& B}
				\end{prooftree}
				\]
				
				\[
				\begin{prooftree}
				\hypo{A $\vdash$  $\Delta$ }
				\infer1{B \& A $\vdash$  $\Delta$ }
				\end{prooftree}
				\quad
				\begin{prooftree}
				\hypo{B $\vdash$  $\Delta$ }
				\infer1{B \& A $\vdash$  $\Delta$ }
				\end{prooftree}
				\quad
				\begin{prooftree}
				\hypo{$\Gamma$  $\vdash$  B}
				\hypo{$\Gamma$  $\vdash$  A}
				\infer2{$\Gamma$  $\vdash$  B \& A}
				\end{prooftree}
				\]
			\end{center}
		
		\subsubsection{Implications}
			\begin{center}
				\[
				\begin{prooftree}
				\hypo{ $\vdash$  A}
				\hypo{B $\vdash$  $\Delta$ }
				\infer2{$\neg$  A $\oplus$  B $\vdash$  $\Delta$ }
				\end{prooftree}
				\quad
				\begin{prooftree}
				\hypo{A $\vdash$  }
				\infer1{ $\vdash$  $\neg$ A $\oplus$  B}
				\end{prooftree}
				\quad
				\begin{prooftree}
				\hypo{$\Gamma$  $\vdash$  B}
				\infer1{$\Gamma$  $\vdash$  $\neg$ A $\oplus$  B}
				\end{prooftree}
				\]
				
				\[
				\begin{prooftree}
				\hypo{B $\vdash$  $\Delta$ }
				\hypo{ $\vdash$  A}
				\infer2{B $\oplus$  $\neg$ A $\vdash$  $\Delta$ }
				\end{prooftree}
				\quad
				\begin{prooftree}
				\hypo{A $\vdash$  }
				\infer1{$\vdash$  B $\oplus$  $\neg$ A}
				\end{prooftree}
				\quad
				\begin{prooftree}
				\hypo{$\Gamma$  $\vdash$  B}
				\infer1{$\Gamma$  $\vdash$  B $\oplus$  $\neg$ A}
				\end{prooftree}
				\]
				
				\[
				\begin{prooftree}
				\hypo{$\vdash$  A}
				\infer1{$\neg$  A \& B $\vdash$  }
				\end{prooftree}
				\quad
				\begin{prooftree}
				\hypo{B $\vdash$  $\Delta$ }
				\infer1{$\neg$ A \& B $\vdash$  $\Delta$ }
				\end{prooftree}
				\quad
				\begin{prooftree}
				\hypo{A $\vdash$  }
				\hypo{$\Gamma$  $\vdash$  B}
				\infer2{$\Gamma$  $\vdash$  $\neg$ A \& B}
				\end{prooftree}
				\]
				
				\[
				\begin{prooftree}
				\hypo{ $\vdash$  A}
				\infer1{B \& $\neg$  A $\vdash$ }
				\end{prooftree}
				\quad
				\begin{prooftree}
				\hypo{B $\vdash$  $\Delta$ }
				\infer1{B \& $\neg$ A $\vdash$  $\Delta$ }
				\end{prooftree}
				\quad
				\begin{prooftree}
				\hypo{$\Gamma$  $\vdash$  B}
				\hypo{A $\vdash$  }
				\infer2{$\Gamma$  $\vdash$  B \& $\neg$ A}
				\end{prooftree}
				\]
			\end{center}
		
		\subsubsection{NAND and NOR}
			\begin{center}
				\[
				\begin{prooftree}
				\hypo{ $\vdash$  A}
				\hypo{ $\vdash$  B}
				\infer2{$\neg$  A $\oplus$  $\neg$ B $\vdash$ }
				\end{prooftree}
				\quad
				\begin{prooftree}
				\hypo{A $\vdash$  }
				\infer1{ $\vdash$  $\neg$ A $\oplus$  $\neg$ B}
				\end{prooftree}
				\quad
				\begin{prooftree}
				\hypo{B $\vdash$  }
				\infer1{$\vdash$  $\neg$ A $\oplus$  $\neg$ B}
				\end{prooftree}
				\]
				
				
				
				\[
				\begin{prooftree}
				\hypo{ $\vdash$  B}
				\hypo{ $\vdash$  A}
				\infer2{$\neg$ B $\oplus$  $\neg$ A $\vdash$  }
				\end{prooftree}
				\quad
				\begin{prooftree}
				\hypo{A $\vdash$  }
				\infer1{$\vdash$  $\neg$ B $\oplus$  $\neg$ A}
				\end{prooftree}
				\quad
				\begin{prooftree}
				\hypo{B $\vdash$  }
				\infer1{$\vdash$  $\neg$ B $\oplus$  $\neg$ A}
				\end{prooftree}
				\]
				
				\[
				\begin{prooftree}
				\hypo{$\vdash$  A}
				\infer1{$\neg$  A \& $\neg$ B $\vdash$  }
				\end{prooftree}
				\quad
				\begin{prooftree}
				\hypo{$\vdash$  B}
				\infer1{$\neg$ A \& $\neg$ B $\vdash$  }
				\end{prooftree}
				\quad
				\begin{prooftree}
				\hypo{A $\vdash$  }
				\hypo{B $\vdash$  }
				\infer2{$\vdash$  $\neg$ A \& $\neg$ B}
				\end{prooftree}
				\]
				
				\[
				\begin{prooftree}
				\hypo{ $\vdash$  A}
				\infer1{$\neg$ B \& $\neg$  A $\vdash$ }
				\end{prooftree}
				\quad
				\begin{prooftree}
				\hypo{$\vdash$  B}
				\infer1{$\neg$ B \& $\neg$ A $\vdash$ }
				\end{prooftree}
				\quad
				\begin{prooftree}
				\hypo{B $\vdash$  }
				\hypo{A $\vdash$  }
				\infer2{$\vdash$  $\neg$ B \& $\neg$ A}
				\end{prooftree}
				\]
			\end{center}
	\end{center}
\end{center}

\part{Theorems}
	\begin{center}

		\[
		\begin{prooftree}
		\infer0{ $\vdash$  A $\oplus$  $\neg$ A}
		\end{prooftree}
		\quad
		\begin{prooftree}
		\infer0{ $\vdash$  $\neg$ A $\oplus$  A}
		\end{prooftree}
		\quad
		\begin{prooftree}
		\infer0{A \& $\neg$ A $\vdash$  }
		\end{prooftree}
		\quad
		\begin{prooftree}
		\infer0{$\neg$ A \& A $\vdash$  }
		\end{prooftree}
		\]
		
		\[
		\begin{prooftree}
		\infer0{0 $\vdash$  C}
		\infer1{0 \& $\neg$  A $\vdash$  C}
		\end{prooftree}
		\quad
		\begin{prooftree}
		\infer0{C $\vdash$  $\top$ }
		\infer1{C $\vdash$  $\top$  $\oplus$  $\neg$  A}
		\end{prooftree}		
		\]
		
		\[
		\begin{prooftree}
		\infer0{0 $\vdash$  A}
		\infer1{0 $\vdash$  A $\oplus$  $\neg$  A}
		\end{prooftree}
		\quad
		\begin{prooftree}
		\infer0{0 $\vdash$  $\neg$  A}
		\infer1{0 $\vdash$  A $\oplus$  $\neg$  A}
		\end{prooftree}
		\quad
		\begin{prooftree}
		\infer0{0 $\vdash$  }
		\infer1{$\vdash$  A $\oplus$  $\neg$  0}
		\end{prooftree}
		\quad
		\begin{prooftree}
		\infer0{C $\vdash$  $\top$ }
		\infer1{C $\vdash$  $\top$  $\oplus$  A}
		\end{prooftree}
		\]
		
		\[
		\begin{prooftree}
		\infer0{A $\vdash$  $\top$ }
		\infer1{A \& $\neg$  A $\vdash$  $\top$ }
		\end{prooftree}
		\quad
		\begin{prooftree}
		\infer0{$\neg$  A $\vdash$  $\top$ }
		\infer1{A \& $\neg$  A $\vdash$  $\top$ }
		\end{prooftree}
		\]
		
		\[
		\begin{prooftree}
		\infer0{1 $\vdash$  1}
		\infer1{1 $\vdash$  1 $\oplus$  $\neg$  1}
		\end{prooftree}
		\quad
		\begin{prooftree}
		\infer0{0 $\vdash$  1 $\oplus$  $\neg$  1}
		\end{prooftree}
		\quad
		\begin{prooftree}
		\hypo{$\bot$  $\vdash$  1 $\oplus$  $\neg$  1}
		\end{prooftree}
		\quad
		\begin{prooftree}
		\hypo{$\top$  $\vdash$  1 $\oplus$  $\neg$  1}
		\end{prooftree}
		\]
		
		\[
		\begin{prooftree}
		\infer0{$\bot$ $\vdash$ }
		\infer1{$\vdash$  $\neg$  $\bot$ }
		\infer1{$\vdash$  $\bot$  $\oplus$  $\neg$  $\bot$ }
		\infer1{1 $\vdash$  $\bot$  $\oplus$  $\neg$  $\bot$ }
		\end{prooftree}
		\quad
		\begin{prooftree}
		\infer0{0 $\vdash$  $\bot$  $\oplus$  $\neg$  $\bot$ }
		\end{prooftree}
		\quad
		\begin{prooftree}
		\infer0{$\bot$  $\vdash$  $\bot$ }
		\infer1{$\bot$  $\vdash$  $\bot$  $\oplus$  $\neg$  $\bot$ }
		\end{prooftree}
		\quad
		\begin{prooftree}
		\hypo{$\top$  $\vdash$  $\bot$  $\oplus$  $\neg$  $\bot$ }
		\end{prooftree}
		\]
		
		\[
		\begin{prooftree}
		\infer0{0 $\vdash$ }
		\infer1{$\vdash$  $\neg$  0}
		\infer1{$\vdash$  0 $\oplus$  $\neg$  0}
		\infer1{1 $\vdash$  0 $\oplus$  $\neg$  0}
		\end{prooftree}
		\quad
		\begin{prooftree}
		\infer0{0 $\vdash$  0 $\oplus$  $\neg$  0}
		\end{prooftree}
		\quad
		\begin{prooftree}
		\hypo{$\bot$  $\vdash$  0 $\oplus$  $\neg$  0}
		\end{prooftree}
		\quad
		\begin{prooftree}
		\hypo{$\top$  $\vdash$  0 $\oplus$  $\neg$  0}
		\end{prooftree}
		\]
		
		\[
		\begin{prooftree}
		\infer0{$\vdash$  $\top$ }
		\infer1{$\vdash$  $\top$  $\oplus$  $\neg$  $\top$ }
		\infer1{1 $\vdash$  $\top$  $\oplus$  $\neg$  $\top$ }
		\end{prooftree}
		\quad
		\begin{prooftree}
		\infer0{0 $\vdash$  $\top$  $\oplus$  $\neg$  $\top$ }
		\end{prooftree}
		\quad
		\begin{prooftree}
		\infer0{$\bot$  $\vdash$  $\top$ }
		\infer1{$\bot$  $\vdash$  $\top$  $\oplus$  $\neg$  $\top$ }
		\end{prooftree}
		\quad
		\begin{prooftree}
		\infer0{$\top$  $\vdash$  $\top$ }
		\infer1{$\top$  $\vdash$  $\top$  $\oplus$  $\neg$  $\top$ }
		\end{prooftree}
		\]
		
		\[
		\begin{prooftree}
		\infer0{1 $\vdash$  1}
		\infer1{1 \&  $\neg$  1$\vdash$  1}
		\end{prooftree}
		\quad
		\begin{prooftree}
		\hypo{1 \&  $\neg$  1$\vdash$  0}
		\end{prooftree}
		\quad
		\begin{prooftree}
		\infer0{$\vdash$  1}
		\infer1{$\neg$  1$\vdash$  }
		\infer1{1 \&  $\neg$  1$\vdash$  }
		\infer1{1 \&  $\neg$  1$\vdash$  $\bot$ }
		\end{prooftree}
		\quad
		\begin{prooftree}
		\infer0{1 \&  $\neg$  1 $\vdash$  $\top$ }
		\end{prooftree}
		\]
		
		\[
		\begin{prooftree}
		\infer0{0 $\vdash$  1}
		\infer1{0 \&  $\neg$  0 $\vdash$  1}
		\end{prooftree}
		\quad
		\begin{prooftree}
		\infer0{0 $\vdash$  0}
		\infer1{0 \&  $\neg$  0 $\vdash$  0}
		\end{prooftree}
		\quad
		\begin{prooftree}
		\infer0{0 $\vdash$  $\bot$ }
		\infer1{0 \&  $\neg$  0 $\vdash$  $\bot$ }
		\end{prooftree}
		\quad
		\begin{prooftree}
		\infer0{0 \&  $\neg$  0 $\vdash$  $\top$ }
		\end{prooftree}
		\]
		
		\[
		\begin{prooftree}
		\hypo{$\bot$  \&  $\neg$  $\bot$  $\vdash$  1}
		\end{prooftree}
		\quad
		\begin{prooftree}
		\hypo{$\bot$  \&  $\neg$  $\bot$  $\vdash$  0}
		\end{prooftree}
		\quad
		\begin{prooftree}
		\infer0{$\bot$  $\vdash$  }
		\infer1{$\bot$  \&  $\neg$  $\bot$  $\vdash$  }
		\infer1{$\bot$  \&  $\neg$  $\bot$  $\vdash$  $\bot$ }
		\end{prooftree}
		\quad
		\begin{prooftree}
		\infer0{$\bot$  \&  $\neg$  $\bot$  $\vdash$  $\top$ }
		\end{prooftree}
		\]
		
		\[
		\begin{prooftree}
		\hypo{$\top$  \&  $\neg$  $\top$  $\vdash$  1}
		\end{prooftree}
		\quad
		\begin{prooftree}
		\hypo{$\top$  \&  $\neg$  $\top$  $\vdash$  0}
		\end{prooftree}
		\quad
		\begin{prooftree}
		\infer0{$\vdash$  $\top$ }
		\infer1{$\neg$  $\top$  $\vdash$ }
		\infer1{$\neg$  $\top$  $\vdash$  $\bot$ }
		\infer1{$\top$  \&  $\neg$  $\top$  $\vdash$  $\bot$ }
		\end{prooftree}
		\quad
		\begin{prooftree}
		\infer0{$\top$  \&  $\neg$  $\top$  $\vdash$  $\top$ }
		\end{prooftree}
		\]
	\end{center}



\chapter{Ordinary Monotonic Conjunct Sequent}






\usepackage[utf8]{inputenc}


































\section{Structural Rules}

\begin{center}
	\[
	\begin{prooftree}
	\infer0[Id]{A $\vdash$  A}
	\end{prooftree}
	\]
	
	\[
	\begin{prooftree}
	\hypo{$\Gamma$  $\vdash$  A}
	\hypo{A $\vdash$  $\Delta$ }
	\infer2[cut]{$\Gamma$  $\vdash$  $\Delta$ }
	\end{prooftree}
	\]
\end{center}

\section{Unit Rules}
\begin{center}
	\[
	\begin{prooftree}
	\infer0{0 $\vdash$  A}
	\end{prooftree}
	\quad
	\begin{prooftree}
	\infer0{A $\vdash$  $\top$ }
	\end{prooftree}
	\]
\end{center}

\newpage
\section{Operational Rules}
\begin{center}
	\subsection{Additives}
	\begin{center}
		\subsubsection{Junctions}
			\begin{center}
				
				\[
				\begin{prooftree}
				\hypo{A $\vdash$  $\Delta$ }
				\hypo{B $\vdash$  $\Delta$ }
				\infer2{A $\oplus$  B $\vdash$  $\Delta$ }
				\end{prooftree}
				\quad
				\begin{prooftree}
				\hypo{$\Gamma$  $\vdash$  A}
				\infer1{$\Gamma$  $\vdash$  A $\oplus$  B}
				\end{prooftree}
				\quad
				\begin{prooftree}
				\hypo{$\Gamma$  $\vdash$  B}
				\infer1{$\Gamma$  $\vdash$  A $\oplus$  B}
				\end{prooftree}
				\]
				
				\[
				\begin{prooftree}
				\hypo{B $\vdash$  $\Delta$ }
				\hypo{A $\vdash$  $\Delta$ }
				\infer2{B $\oplus$  A $\vdash$  $\Delta$ }
				\end{prooftree}
				\quad
				\begin{prooftree}
				\hypo{$\Gamma$  $\vdash$  A}
				\infer1{$\Gamma$  $\vdash$  B $\oplus$  A}
				\end{prooftree}
				\quad
				\begin{prooftree}
				\hypo{$\Gamma$  $\vdash$  B}
				\infer1{$\Gamma$  $\vdash$  B $\oplus$  A}
				\end{prooftree}
				\]
				
				\[
				\begin{prooftree}
				\hypo{A $\vdash$  $\Delta$ }
				\infer1{A \& B $\vdash$  $\Delta$ }
				\end{prooftree}
				\quad
				\begin{prooftree}
				\hypo{B $\vdash$  $\Delta$ }
				\infer1{A \& B $\vdash$  $\Delta$ }
				\end{prooftree}
				\quad
				\begin{prooftree}
				\hypo{$\Gamma$  $\vdash$  A}
				\hypo{$\Gamma$  $\vdash$  B}
				\infer2{$\Gamma$  $\vdash$  A \& B}
				\end{prooftree}
				\]
				
				\[
				\begin{prooftree}
				\hypo{A $\vdash$  $\Delta$ }
				\infer1{B \& A $\vdash$  $\Delta$ }
				\end{prooftree}
				\quad
				\begin{prooftree}
				\hypo{B $\vdash$  $\Delta$ }
				\infer1{B \& A $\vdash$  $\Delta$ }
				\end{prooftree}
				\quad
				\begin{prooftree}
				\hypo{$\Gamma$  $\vdash$  B}
				\hypo{$\Gamma$  $\vdash$  A}
				\infer2{$\Gamma$  $\vdash$  B \& A}
				\end{prooftree}
				\]
			\end{center}
	\end{center}

		\subsection{Multiplicatives}
	\begin{center}
		\subsubsection{Junctions}
		\begin{center}
			
			\[
			\begin{prooftree}
			\hypo{A $\vdash$  $\Delta$ _0}
			\hypo{B $\vdash$  $\Delta$ _1}
			\infer2{A ℘ B $\vdash$  $\Delta$ _0, $\Delta$ _1}
			\end{prooftree}
			\quad
			\begin{prooftree}
			\hypo{$\Gamma$  $\vdash$  A, B}
			\infer1{$\Gamma$  $\vdash$  A ℘ B}
			\end{prooftree}
			\]

			\[
			\begin{prooftree}
			\hypo{B $\vdash$  $\Delta$ _0}
			\hypo{A $\vdash$  $\Delta$ _1}
			\infer2{B ℘ A $\vdash$  $\Delta$ _0, $\Delta$ _1}
			\end{prooftree}
			\quad
			\begin{prooftree}
			\hypo{$\Gamma$  $\vdash$  B, A}
			\infer1{$\Gamma$  $\vdash$  B ℘ A}
			\end{prooftree}
			\]
			
			\[
			\begin{prooftree}
			\hypo{A, B $\vdash$  $\Delta$ }
			\infer1{A $\otimes$  B $\vdash$  $\Delta$ }
			\end{prooftree}
			\quad
			\begin{prooftree}
			\hypo{$\Gamma$ _0 $\vdash$  A}
			\hypo{$\Gamma$ _1 $\vdash$  B}
			\infer2{$\Gamma$ _0, $\Gamma$ _1 $\vdash$  A $\otimes$  B}
			\end{prooftree}
			\]
			
			\[
			\begin{prooftree}
			\hypo{B, A $\vdash$  $\Delta$ }
			\infer1{B $\otimes$  A $\vdash$  $\Delta$ }
			\end{prooftree}
			\quad
			\begin{prooftree}
			\hypo{$\Gamma$ _0 $\vdash$  B}
			\hypo{$\Gamma$ _1 $\vdash$  A}
			\infer2{$\Gamma$ _0, $\Gamma$ _1 $\vdash$  B $\otimes$  A}
			\end{prooftree}
			\]
		\end{center}
	\end{center}
\end{center}

\part{Theorems}
	\begin{center}

		\[
		\begin{prooftree}
		\infer0{ $\vdash$  A $\oplus$  $\neg$ A}
		\end{prooftree}
		\quad
		\begin{prooftree}
		\infer0{ $\vdash$  $\neg$ A $\oplus$  A}
		\end{prooftree}
		\quad
		\begin{prooftree}
		\infer0{A \& $\neg$ A $\vdash$  }
		\end{prooftree}
		\quad
		\begin{prooftree}
		\infer0{$\neg$ A \& A $\vdash$  }
		\end{prooftree}
		\]
		
		\[
		\begin{prooftree}
		\infer0{0 $\vdash$  C}
		\infer1{0 \& $\neg$  A $\vdash$  C}
		\end{prooftree}
		\quad
		\begin{prooftree}
		\infer0{C $\vdash$  $\top$ }
		\infer1{C $\vdash$  $\top$  $\oplus$  $\neg$  A}
		\end{prooftree}		
		\]
		
		\[
		\begin{prooftree}
		\infer0{0 $\vdash$  A}
		\infer1{0 $\vdash$  A $\oplus$  $\neg$  A}
		\end{prooftree}
		\quad
		\begin{prooftree}
		\infer0{0 $\vdash$  $\neg$  A}
		\infer1{0 $\vdash$  A $\oplus$  $\neg$  A}
		\end{prooftree}
		\quad
		\begin{prooftree}
		\infer0{0 $\vdash$  }
		\infer1{$\vdash$  A $\oplus$  $\neg$  0}
		\end{prooftree}
		\quad
		\begin{prooftree}
		\infer0{C $\vdash$  $\top$ }
		\infer1{C $\vdash$  $\top$  $\oplus$  A}
		\end{prooftree}
		\]
		
		\[
		\begin{prooftree}
		\infer0{A $\vdash$  $\top$ }
		\infer1{A \& $\neg$  A $\vdash$  $\top$ }
		\end{prooftree}
		\quad
		\begin{prooftree}
		\infer0{$\neg$  A $\vdash$  $\top$ }
		\infer1{A \& $\neg$  A $\vdash$  $\top$ }
		\end{prooftree}
		\]
		
		\[
		\begin{prooftree}
		\infer0{1 $\vdash$  1}
		\infer1{1 $\vdash$  1 $\oplus$  $\neg$  1}
		\end{prooftree}
		\quad
		\begin{prooftree}
		\infer0{0 $\vdash$  1 $\oplus$  $\neg$  1}
		\end{prooftree}
		\quad
		\begin{prooftree}
		\hypo{$\bot$  $\vdash$  1 $\oplus$  $\neg$  1}
		\end{prooftree}
		\quad
		\begin{prooftree}
		\hypo{$\top$  $\vdash$  1 $\oplus$  $\neg$  1}
		\end{prooftree}
		\]
		
		\[
		\begin{prooftree}
		\infer0{$\bot$ $\vdash$ }
		\infer1{$\vdash$  $\neg$  $\bot$ }
		\infer1{$\vdash$  $\bot$  $\oplus$  $\neg$  $\bot$ }
		\infer1{1 $\vdash$  $\bot$  $\oplus$  $\neg$  $\bot$ }
		\end{prooftree}
		\quad
		\begin{prooftree}
		\infer0{0 $\vdash$  $\bot$  $\oplus$  $\neg$  $\bot$ }
		\end{prooftree}
		\quad
		\begin{prooftree}
		\infer0{$\bot$  $\vdash$  $\bot$ }
		\infer1{$\bot$  $\vdash$  $\bot$  $\oplus$  $\neg$  $\bot$ }
		\end{prooftree}
		\quad
		\begin{prooftree}
		\hypo{$\top$  $\vdash$  $\bot$  $\oplus$  $\neg$  $\bot$ }
		\end{prooftree}
		\]
		
		\[
		\begin{prooftree}
		\infer0{0 $\vdash$ }
		\infer1{$\vdash$  $\neg$  0}
		\infer1{$\vdash$  0 $\oplus$  $\neg$  0}
		\infer1{1 $\vdash$  0 $\oplus$  $\neg$  0}
		\end{prooftree}
		\quad
		\begin{prooftree}
		\infer0{0 $\vdash$  0 $\oplus$  $\neg$  0}
		\end{prooftree}
		\quad
		\begin{prooftree}
		\hypo{$\bot$  $\vdash$  0 $\oplus$  $\neg$  0}
		\end{prooftree}
		\quad
		\begin{prooftree}
		\hypo{$\top$  $\vdash$  0 $\oplus$  $\neg$  0}
		\end{prooftree}
		\]
		
		\[
		\begin{prooftree}
		\infer0{$\vdash$  $\top$ }
		\infer1{$\vdash$  $\top$  $\oplus$  $\neg$  $\top$ }
		\infer1{1 $\vdash$  $\top$  $\oplus$  $\neg$  $\top$ }
		\end{prooftree}
		\quad
		\begin{prooftree}
		\infer0{0 $\vdash$  $\top$  $\oplus$  $\neg$  $\top$ }
		\end{prooftree}
		\quad
		\begin{prooftree}
		\infer0{$\bot$  $\vdash$  $\top$ }
		\infer1{$\bot$  $\vdash$  $\top$  $\oplus$  $\neg$  $\top$ }
		\end{prooftree}
		\quad
		\begin{prooftree}
		\infer0{$\top$  $\vdash$  $\top$ }
		\infer1{$\top$  $\vdash$  $\top$  $\oplus$  $\neg$  $\top$ }
		\end{prooftree}
		\]
		
		\[
		\begin{prooftree}
		\infer0{1 $\vdash$  1}
		\infer1{1 \&  $\neg$  1$\vdash$  1}
		\end{prooftree}
		\quad
		\begin{prooftree}
		\hypo{1 \&  $\neg$  1$\vdash$  0}
		\end{prooftree}
		\quad
		\begin{prooftree}
		\infer0{$\vdash$  1}
		\infer1{$\neg$  1$\vdash$  }
		\infer1{1 \&  $\neg$  1$\vdash$  }
		\infer1{1 \&  $\neg$  1$\vdash$  $\bot$ }
		\end{prooftree}
		\quad
		\begin{prooftree}
		\infer0{1 \&  $\neg$  1 $\vdash$  $\top$ }
		\end{prooftree}
		\]
		
		\[
		\begin{prooftree}
		\infer0{0 $\vdash$  1}
		\infer1{0 \&  $\neg$  0 $\vdash$  1}
		\end{prooftree}
		\quad
		\begin{prooftree}
		\infer0{0 $\vdash$  0}
		\infer1{0 \&  $\neg$  0 $\vdash$  0}
		\end{prooftree}
		\quad
		\begin{prooftree}
		\infer0{0 $\vdash$  $\bot$ }
		\infer1{0 \&  $\neg$  0 $\vdash$  $\bot$ }
		\end{prooftree}
		\quad
		\begin{prooftree}
		\infer0{0 \&  $\neg$  0 $\vdash$  $\top$ }
		\end{prooftree}
		\]
		
		\[
		\begin{prooftree}
		\hypo{$\bot$  \&  $\neg$  $\bot$  $\vdash$  1}
		\end{prooftree}
		\quad
		\begin{prooftree}
		\hypo{$\bot$  \&  $\neg$  $\bot$  $\vdash$  0}
		\end{prooftree}
		\quad
		\begin{prooftree}
		\infer0{$\bot$  $\vdash$  }
		\infer1{$\bot$  \&  $\neg$  $\bot$  $\vdash$  }
		\infer1{$\bot$  \&  $\neg$  $\bot$  $\vdash$  $\bot$ }
		\end{prooftree}
		\quad
		\begin{prooftree}
		\infer0{$\bot$  \&  $\neg$  $\bot$  $\vdash$  $\top$ }
		\end{prooftree}
		\]
		
		\[
		\begin{prooftree}
		\hypo{$\top$  \&  $\neg$  $\top$  $\vdash$  1}
		\end{prooftree}
		\quad
		\begin{prooftree}
		\hypo{$\top$  \&  $\neg$  $\top$  $\vdash$  0}
		\end{prooftree}
		\quad
		\begin{prooftree}
		\infer0{$\vdash$  $\top$ }
		\infer1{$\neg$  $\top$  $\vdash$ }
		\infer1{$\neg$  $\top$  $\vdash$  $\bot$ }
		\infer1{$\top$  \&  $\neg$  $\top$  $\vdash$  $\bot$ }
		\end{prooftree}
		\quad
		\begin{prooftree}
		\infer0{$\top$  \&  $\neg$  $\top$  $\vdash$  $\top$ }
		\end{prooftree}
		\]
	\end{center}



\chapter{Ordinary Monotonic Disjunct Sequent}






\usepackage[utf8]{inputenc}


































\section{Structural Rules}

\begin{center}
	\[
	\begin{prooftree}
	\infer0[Id]{A $\vdash$  A}
	\end{prooftree}
	\]
	
	\[
	\begin{prooftree}
	\hypo{$\Gamma$  $\vdash$  A}
	\hypo{A $\vdash$  $\Delta$ }
	\infer2[cut]{$\Gamma$  $\vdash$  $\Delta$ }
	\end{prooftree}
	\]
\end{center}

\section{Unit Rules}
\begin{center}
	\[
	\begin{prooftree}
	\infer0{0 $\vdash$  A}
	\end{prooftree}
	\quad
	\begin{prooftree}
	\infer0{A $\vdash$  $\top$ }
	\end{prooftree}
	\]
\end{center}

\section{Operational Rules}
\begin{center}
	\subsection{Additives}
	\begin{center}
		\[
		\begin{prooftree}
		\hypo{A $\vdash$  $\Delta$ }
		\hypo{B $\vdash$  $\Delta$ }
		\infer2{A $\oplus$  B $\vdash$  $\Delta$ }
		\end{prooftree}
		\quad
		\begin{prooftree}
		\hypo{$\Gamma$  $\vdash$  A}
		\infer1{$\Gamma$  $\vdash$  A $\oplus$  B}
		\end{prooftree}
		\quad
		\begin{prooftree}
		\hypo{$\Gamma$  $\vdash$  B}
		\infer1{$\Gamma$  $\vdash$  A $\oplus$  B}
		\end{prooftree}
		\]
		
		\[
		\begin{prooftree}
		\hypo{B $\vdash$  $\Delta$ }
		\hypo{A $\vdash$  $\Delta$ }
		\infer2{B $\oplus$  A $\vdash$  $\Delta$ }
		\end{prooftree}
		\quad
		\begin{prooftree}
		\hypo{$\Gamma$  $\vdash$  A}
		\infer1{$\Gamma$  $\vdash$  B $\oplus$  A}
		\end{prooftree}
		\quad
		\begin{prooftree}
		\hypo{$\Gamma$  $\vdash$  B}
		\infer1{$\Gamma$  $\vdash$  B $\oplus$  A}
		\end{prooftree}
		\]
	\end{center}
\end{center}

\part{Theorems}
	\begin{center}

		\[
		\begin{prooftree}
		\infer0{ $\vdash$  A $\oplus$  $\neg$ A}
		\end{prooftree}
		\quad
		\begin{prooftree}
		\infer0{ $\vdash$  $\neg$ A $\oplus$  A}
		\end{prooftree}
		\quad
		\begin{prooftree}
		\infer0{A \& $\neg$ A $\vdash$  }
		\end{prooftree}
		\quad
		\begin{prooftree}
		\infer0{$\neg$ A \& A $\vdash$  }
		\end{prooftree}
		\]
		
		\[
		\begin{prooftree}
		\infer0{0 $\vdash$  C}
		\infer1{0 \& $\neg$  A $\vdash$  C}
		\end{prooftree}
		\quad
		\begin{prooftree}
		\infer0{C $\vdash$  $\top$ }
		\infer1{C $\vdash$  $\top$  $\oplus$  $\neg$  A}
		\end{prooftree}		
		\]
		
		\[
		\begin{prooftree}
		\infer0{0 $\vdash$  A}
		\infer1{0 $\vdash$  A $\oplus$  $\neg$  A}
		\end{prooftree}
		\quad
		\begin{prooftree}
		\infer0{0 $\vdash$  $\neg$  A}
		\infer1{0 $\vdash$  A $\oplus$  $\neg$  A}
		\end{prooftree}
		\quad
		\begin{prooftree}
		\infer0{0 $\vdash$  }
		\infer1{$\vdash$  A $\oplus$  $\neg$  0}
		\end{prooftree}
		\quad
		\begin{prooftree}
		\infer0{C $\vdash$  $\top$ }
		\infer1{C $\vdash$  $\top$  $\oplus$  A}
		\end{prooftree}
		\]
		
		\[
		\begin{prooftree}
		\infer0{A $\vdash$  $\top$ }
		\infer1{A \& $\neg$  A $\vdash$  $\top$ }
		\end{prooftree}
		\quad
		\begin{prooftree}
		\infer0{$\neg$  A $\vdash$  $\top$ }
		\infer1{A \& $\neg$  A $\vdash$  $\top$ }
		\end{prooftree}
		\]
		
		\[
		\begin{prooftree}
		\infer0{1 $\vdash$  1}
		\infer1{1 $\vdash$  1 $\oplus$  $\neg$  1}
		\end{prooftree}
		\quad
		\begin{prooftree}
		\infer0{0 $\vdash$  1 $\oplus$  $\neg$  1}
		\end{prooftree}
		\quad
		\begin{prooftree}
		\hypo{$\bot$  $\vdash$  1 $\oplus$  $\neg$  1}
		\end{prooftree}
		\quad
		\begin{prooftree}
		\hypo{$\top$  $\vdash$  1 $\oplus$  $\neg$  1}
		\end{prooftree}
		\]
		
		\[
		\begin{prooftree}
		\infer0{$\bot$ $\vdash$ }
		\infer1{$\vdash$  $\neg$  $\bot$ }
		\infer1{$\vdash$  $\bot$  $\oplus$  $\neg$  $\bot$ }
		\infer1{1 $\vdash$  $\bot$  $\oplus$  $\neg$  $\bot$ }
		\end{prooftree}
		\quad
		\begin{prooftree}
		\infer0{0 $\vdash$  $\bot$  $\oplus$  $\neg$  $\bot$ }
		\end{prooftree}
		\quad
		\begin{prooftree}
		\infer0{$\bot$  $\vdash$  $\bot$ }
		\infer1{$\bot$  $\vdash$  $\bot$  $\oplus$  $\neg$  $\bot$ }
		\end{prooftree}
		\quad
		\begin{prooftree}
		\hypo{$\top$  $\vdash$  $\bot$  $\oplus$  $\neg$  $\bot$ }
		\end{prooftree}
		\]
		
		\[
		\begin{prooftree}
		\infer0{0 $\vdash$ }
		\infer1{$\vdash$  $\neg$  0}
		\infer1{$\vdash$  0 $\oplus$  $\neg$  0}
		\infer1{1 $\vdash$  0 $\oplus$  $\neg$  0}
		\end{prooftree}
		\quad
		\begin{prooftree}
		\infer0{0 $\vdash$  0 $\oplus$  $\neg$  0}
		\end{prooftree}
		\quad
		\begin{prooftree}
		\hypo{$\bot$  $\vdash$  0 $\oplus$  $\neg$  0}
		\end{prooftree}
		\quad
		\begin{prooftree}
		\hypo{$\top$  $\vdash$  0 $\oplus$  $\neg$  0}
		\end{prooftree}
		\]
		
		\[
		\begin{prooftree}
		\infer0{$\vdash$  $\top$ }
		\infer1{$\vdash$  $\top$  $\oplus$  $\neg$  $\top$ }
		\infer1{1 $\vdash$  $\top$  $\oplus$  $\neg$  $\top$ }
		\end{prooftree}
		\quad
		\begin{prooftree}
		\infer0{0 $\vdash$  $\top$  $\oplus$  $\neg$  $\top$ }
		\end{prooftree}
		\quad
		\begin{prooftree}
		\infer0{$\bot$  $\vdash$  $\top$ }
		\infer1{$\bot$  $\vdash$  $\top$  $\oplus$  $\neg$  $\top$ }
		\end{prooftree}
		\quad
		\begin{prooftree}
		\infer0{$\top$  $\vdash$  $\top$ }
		\infer1{$\top$  $\vdash$  $\top$  $\oplus$  $\neg$  $\top$ }
		\end{prooftree}
		\]
		
		\[
		\begin{prooftree}
		\infer0{1 $\vdash$  1}
		\infer1{1 \&  $\neg$  1$\vdash$  1}
		\end{prooftree}
		\quad
		\begin{prooftree}
		\hypo{1 \&  $\neg$  1$\vdash$  0}
		\end{prooftree}
		\quad
		\begin{prooftree}
		\infer0{$\vdash$  1}
		\infer1{$\neg$  1$\vdash$  }
		\infer1{1 \&  $\neg$  1$\vdash$  }
		\infer1{1 \&  $\neg$  1$\vdash$  $\bot$ }
		\end{prooftree}
		\quad
		\begin{prooftree}
		\infer0{1 \&  $\neg$  1 $\vdash$  $\top$ }
		\end{prooftree}
		\]
		
		\[
		\begin{prooftree}
		\infer0{0 $\vdash$  1}
		\infer1{0 \&  $\neg$  0 $\vdash$  1}
		\end{prooftree}
		\quad
		\begin{prooftree}
		\infer0{0 $\vdash$  0}
		\infer1{0 \&  $\neg$  0 $\vdash$  0}
		\end{prooftree}
		\quad
		\begin{prooftree}
		\infer0{0 $\vdash$  $\bot$ }
		\infer1{0 \&  $\neg$  0 $\vdash$  $\bot$ }
		\end{prooftree}
		\quad
		\begin{prooftree}
		\infer0{0 \&  $\neg$  0 $\vdash$  $\top$ }
		\end{prooftree}
		\]
		
		\[
		\begin{prooftree}
		\hypo{$\bot$  \&  $\neg$  $\bot$  $\vdash$  1}
		\end{prooftree}
		\quad
		\begin{prooftree}
		\hypo{$\bot$  \&  $\neg$  $\bot$  $\vdash$  0}
		\end{prooftree}
		\quad
		\begin{prooftree}
		\infer0{$\bot$  $\vdash$  }
		\infer1{$\bot$  \&  $\neg$  $\bot$  $\vdash$  }
		\infer1{$\bot$  \&  $\neg$  $\bot$  $\vdash$  $\bot$ }
		\end{prooftree}
		\quad
		\begin{prooftree}
		\infer0{$\bot$  \&  $\neg$  $\bot$  $\vdash$  $\top$ }
		\end{prooftree}
		\]
		
		\[
		\begin{prooftree}
		\hypo{$\top$  \&  $\neg$  $\top$  $\vdash$  1}
		\end{prooftree}
		\quad
		\begin{prooftree}
		\hypo{$\top$  \&  $\neg$  $\top$  $\vdash$  0}
		\end{prooftree}
		\quad
		\begin{prooftree}
		\infer0{$\vdash$  $\top$ }
		\infer1{$\neg$  $\top$  $\vdash$ }
		\infer1{$\neg$  $\top$  $\vdash$  $\bot$ }
		\infer1{$\top$  \&  $\neg$  $\top$  $\vdash$  $\bot$ }
		\end{prooftree}
		\quad
		\begin{prooftree}
		\infer0{$\top$  \&  $\neg$  $\top$  $\vdash$  $\top$ }
		\end{prooftree}
		\]
	\end{center}



\chapter{Ordinary Monotonic Sequent}






\usepackage[utf8]{inputenc}


































\section{Structural Rules}

\begin{center}
	\[
	\begin{prooftree}
	\infer0[Id]{A $\vdash$  A}
	\end{prooftree}
	\]
	
	\[
	\begin{prooftree}
	\hypo{$\Gamma$  $\vdash$  A}
	\hypo{A $\vdash$  $\Delta$ }
	\infer2[cut]{$\Gamma$  $\vdash$  $\Delta$ }
	\end{prooftree}
	\]
\end{center}

\section{Unit Rules}
\begin{center}
	\[
	\begin{prooftree}
	\infer0{0 $\vdash$  A}
	\end{prooftree}
	\quad
	\begin{prooftree}
	\infer0{A $\vdash$  $\top$ }
	\end{prooftree}
	\]
\end{center}

\newpage
\section{Operational Rules}
\begin{center}
	\subsection{Additives}
	\begin{center}
		\subsubsection{Junctions}
			\begin{center}
				
				\[
				\begin{prooftree}
				\hypo{A $\vdash$  $\Delta$ }
				\hypo{B $\vdash$  $\Delta$ }
				\infer2{A $\oplus$  B $\vdash$  $\Delta$ }
				\end{prooftree}
				\quad
				\begin{prooftree}
				\hypo{$\Gamma$  $\vdash$  A}
				\infer1{$\Gamma$  $\vdash$  A $\oplus$  B}
				\end{prooftree}
				\quad
				\begin{prooftree}
				\hypo{$\Gamma$  $\vdash$  B}
				\infer1{$\Gamma$  $\vdash$  A $\oplus$  B}
				\end{prooftree}
				\]
				
				\[
				\begin{prooftree}
				\hypo{B $\vdash$  $\Delta$ }
				\hypo{A $\vdash$  $\Delta$ }
				\infer2{B $\oplus$  A $\vdash$  $\Delta$ }
				\end{prooftree}
				\quad
				\begin{prooftree}
				\hypo{$\Gamma$  $\vdash$  A}
				\infer1{$\Gamma$  $\vdash$  B $\oplus$  A}
				\end{prooftree}
				\quad
				\begin{prooftree}
				\hypo{$\Gamma$  $\vdash$  B}
				\infer1{$\Gamma$  $\vdash$  B $\oplus$  A}
				\end{prooftree}
				\]
				
				\[
				\begin{prooftree}
				\hypo{A $\vdash$  $\Delta$ }
				\infer1{A \& B $\vdash$  $\Delta$ }
				\end{prooftree}
				\quad
				\begin{prooftree}
				\hypo{B $\vdash$  $\Delta$ }
				\infer1{A \& B $\vdash$  $\Delta$ }
				\end{prooftree}
				\quad
				\begin{prooftree}
				\hypo{$\Gamma$  $\vdash$  A}
				\hypo{$\Gamma$  $\vdash$  B}
				\infer2{$\Gamma$  $\vdash$  A \& B}
				\end{prooftree}
				\]
				
				\[
				\begin{prooftree}
				\hypo{A $\vdash$  $\Delta$ }
				\infer1{B \& A $\vdash$  $\Delta$ }
				\end{prooftree}
				\quad
				\begin{prooftree}
				\hypo{B $\vdash$  $\Delta$ }
				\infer1{B \& A $\vdash$  $\Delta$ }
				\end{prooftree}
				\quad
				\begin{prooftree}
				\hypo{$\Gamma$  $\vdash$  B}
				\hypo{$\Gamma$  $\vdash$  A}
				\infer2{$\Gamma$  $\vdash$  B \& A}
				\end{prooftree}
				\]
			\end{center}
	\end{center}

		\subsection{Multiplicatives}
	\begin{center}
		\subsubsection{Junctions}
		\begin{center}
			
			\[
			\begin{prooftree}
			\hypo{A $\vdash$  $\Delta$ _0}
			\hypo{B $\vdash$  $\Delta$ _1}
			\infer2{A ℘ B $\vdash$  $\Delta$ _0, $\Delta$ _1}
			\end{prooftree}
			\quad
			\begin{prooftree}
			\hypo{$\Gamma$  $\vdash$  A, B}
			\infer1{$\Gamma$  $\vdash$  A ℘ B}
			\end{prooftree}
			\]

			\[
			\begin{prooftree}
			\hypo{B $\vdash$  $\Delta$ _0}
			\hypo{A $\vdash$  $\Delta$ _1}
			\infer2{B ℘ A $\vdash$  $\Delta$ _0, $\Delta$ _1}
			\end{prooftree}
			\quad
			\begin{prooftree}
			\hypo{$\Gamma$  $\vdash$  B, A}
			\infer1{$\Gamma$  $\vdash$  B ℘ A}
			\end{prooftree}
			\]
			
			\[
			\begin{prooftree}
			\hypo{A, B $\vdash$  $\Delta$ }
			\infer1{A $\otimes$  B $\vdash$  $\Delta$ }
			\end{prooftree}
			\quad
			\begin{prooftree}
			\hypo{$\Gamma$ _0 $\vdash$  A}
			\hypo{$\Gamma$ _1 $\vdash$  B}
			\infer2{$\Gamma$ _0, $\Gamma$ _1 $\vdash$  A $\otimes$  B}
			\end{prooftree}
			\]
			
			\[
			\begin{prooftree}
			\hypo{B, A $\vdash$  $\Delta$ }
			\infer1{B $\otimes$  A $\vdash$  $\Delta$ }
			\end{prooftree}
			\quad
			\begin{prooftree}
			\hypo{$\Gamma$ _0 $\vdash$  B}
			\hypo{$\Gamma$ _1 $\vdash$  A}
			\infer2{$\Gamma$ _0, $\Gamma$ _1 $\vdash$  B $\otimes$  A}
			\end{prooftree}
			\]
		\end{center}
	\end{center}
\end{center}

\part{Theorems}
	\begin{center}

		\[
		\begin{prooftree}
		\infer0{ $\vdash$  A $\oplus$  $\neg$ A}
		\end{prooftree}
		\quad
		\begin{prooftree}
		\infer0{ $\vdash$  $\neg$ A $\oplus$  A}
		\end{prooftree}
		\quad
		\begin{prooftree}
		\infer0{A \& $\neg$ A $\vdash$  }
		\end{prooftree}
		\quad
		\begin{prooftree}
		\infer0{$\neg$ A \& A $\vdash$  }
		\end{prooftree}
		\]
		
		\[
		\begin{prooftree}
		\infer0{0 $\vdash$  C}
		\infer1{0 \& $\neg$  A $\vdash$  C}
		\end{prooftree}
		\quad
		\begin{prooftree}
		\infer0{C $\vdash$  $\top$ }
		\infer1{C $\vdash$  $\top$  $\oplus$  $\neg$  A}
		\end{prooftree}		
		\]
		
		\[
		\begin{prooftree}
		\infer0{0 $\vdash$  A}
		\infer1{0 $\vdash$  A $\oplus$  $\neg$  A}
		\end{prooftree}
		\quad
		\begin{prooftree}
		\infer0{0 $\vdash$  $\neg$  A}
		\infer1{0 $\vdash$  A $\oplus$  $\neg$  A}
		\end{prooftree}
		\quad
		\begin{prooftree}
		\infer0{0 $\vdash$  }
		\infer1{$\vdash$  A $\oplus$  $\neg$  0}
		\end{prooftree}
		\quad
		\begin{prooftree}
		\infer0{C $\vdash$  $\top$ }
		\infer1{C $\vdash$  $\top$  $\oplus$  A}
		\end{prooftree}
		\]
		
		\[
		\begin{prooftree}
		\infer0{A $\vdash$  $\top$ }
		\infer1{A \& $\neg$  A $\vdash$  $\top$ }
		\end{prooftree}
		\quad
		\begin{prooftree}
		\infer0{$\neg$  A $\vdash$  $\top$ }
		\infer1{A \& $\neg$  A $\vdash$  $\top$ }
		\end{prooftree}
		\]
		
		\[
		\begin{prooftree}
		\infer0{1 $\vdash$  1}
		\infer1{1 $\vdash$  1 $\oplus$  $\neg$  1}
		\end{prooftree}
		\quad
		\begin{prooftree}
		\infer0{0 $\vdash$  1 $\oplus$  $\neg$  1}
		\end{prooftree}
		\quad
		\begin{prooftree}
		\hypo{$\bot$  $\vdash$  1 $\oplus$  $\neg$  1}
		\end{prooftree}
		\quad
		\begin{prooftree}
		\hypo{$\top$  $\vdash$  1 $\oplus$  $\neg$  1}
		\end{prooftree}
		\]
		
		\[
		\begin{prooftree}
		\infer0{$\bot$ $\vdash$ }
		\infer1{$\vdash$  $\neg$  $\bot$ }
		\infer1{$\vdash$  $\bot$  $\oplus$  $\neg$  $\bot$ }
		\infer1{1 $\vdash$  $\bot$  $\oplus$  $\neg$  $\bot$ }
		\end{prooftree}
		\quad
		\begin{prooftree}
		\infer0{0 $\vdash$  $\bot$  $\oplus$  $\neg$  $\bot$ }
		\end{prooftree}
		\quad
		\begin{prooftree}
		\infer0{$\bot$  $\vdash$  $\bot$ }
		\infer1{$\bot$  $\vdash$  $\bot$  $\oplus$  $\neg$  $\bot$ }
		\end{prooftree}
		\quad
		\begin{prooftree}
		\hypo{$\top$  $\vdash$  $\bot$  $\oplus$  $\neg$  $\bot$ }
		\end{prooftree}
		\]
		
		\[
		\begin{prooftree}
		\infer0{0 $\vdash$ }
		\infer1{$\vdash$  $\neg$  0}
		\infer1{$\vdash$  0 $\oplus$  $\neg$  0}
		\infer1{1 $\vdash$  0 $\oplus$  $\neg$  0}
		\end{prooftree}
		\quad
		\begin{prooftree}
		\infer0{0 $\vdash$  0 $\oplus$  $\neg$  0}
		\end{prooftree}
		\quad
		\begin{prooftree}
		\hypo{$\bot$  $\vdash$  0 $\oplus$  $\neg$  0}
		\end{prooftree}
		\quad
		\begin{prooftree}
		\hypo{$\top$  $\vdash$  0 $\oplus$  $\neg$  0}
		\end{prooftree}
		\]
		
		\[
		\begin{prooftree}
		\infer0{$\vdash$  $\top$ }
		\infer1{$\vdash$  $\top$  $\oplus$  $\neg$  $\top$ }
		\infer1{1 $\vdash$  $\top$  $\oplus$  $\neg$  $\top$ }
		\end{prooftree}
		\quad
		\begin{prooftree}
		\infer0{0 $\vdash$  $\top$  $\oplus$  $\neg$  $\top$ }
		\end{prooftree}
		\quad
		\begin{prooftree}
		\infer0{$\bot$  $\vdash$  $\top$ }
		\infer1{$\bot$  $\vdash$  $\top$  $\oplus$  $\neg$  $\top$ }
		\end{prooftree}
		\quad
		\begin{prooftree}
		\infer0{$\top$  $\vdash$  $\top$ }
		\infer1{$\top$  $\vdash$  $\top$  $\oplus$  $\neg$  $\top$ }
		\end{prooftree}
		\]
		
		\[
		\begin{prooftree}
		\infer0{1 $\vdash$  1}
		\infer1{1 \&  $\neg$  1$\vdash$  1}
		\end{prooftree}
		\quad
		\begin{prooftree}
		\hypo{1 \&  $\neg$  1$\vdash$  0}
		\end{prooftree}
		\quad
		\begin{prooftree}
		\infer0{$\vdash$  1}
		\infer1{$\neg$  1$\vdash$  }
		\infer1{1 \&  $\neg$  1$\vdash$  }
		\infer1{1 \&  $\neg$  1$\vdash$  $\bot$ }
		\end{prooftree}
		\quad
		\begin{prooftree}
		\infer0{1 \&  $\neg$  1 $\vdash$  $\top$ }
		\end{prooftree}
		\]
		
		\[
		\begin{prooftree}
		\infer0{0 $\vdash$  1}
		\infer1{0 \&  $\neg$  0 $\vdash$  1}
		\end{prooftree}
		\quad
		\begin{prooftree}
		\infer0{0 $\vdash$  0}
		\infer1{0 \&  $\neg$  0 $\vdash$  0}
		\end{prooftree}
		\quad
		\begin{prooftree}
		\infer0{0 $\vdash$  $\bot$ }
		\infer1{0 \&  $\neg$  0 $\vdash$  $\bot$ }
		\end{prooftree}
		\quad
		\begin{prooftree}
		\infer0{0 \&  $\neg$  0 $\vdash$  $\top$ }
		\end{prooftree}
		\]
		
		\[
		\begin{prooftree}
		\hypo{$\bot$  \&  $\neg$  $\bot$  $\vdash$  1}
		\end{prooftree}
		\quad
		\begin{prooftree}
		\hypo{$\bot$  \&  $\neg$  $\bot$  $\vdash$  0}
		\end{prooftree}
		\quad
		\begin{prooftree}
		\infer0{$\bot$  $\vdash$  }
		\infer1{$\bot$  \&  $\neg$  $\bot$  $\vdash$  }
		\infer1{$\bot$  \&  $\neg$  $\bot$  $\vdash$  $\bot$ }
		\end{prooftree}
		\quad
		\begin{prooftree}
		\infer0{$\bot$  \&  $\neg$  $\bot$  $\vdash$  $\top$ }
		\end{prooftree}
		\]
		
		\[
		\begin{prooftree}
		\hypo{$\top$  \&  $\neg$  $\top$  $\vdash$  1}
		\end{prooftree}
		\quad
		\begin{prooftree}
		\hypo{$\top$  \&  $\neg$  $\top$  $\vdash$  0}
		\end{prooftree}
		\quad
		\begin{prooftree}
		\infer0{$\vdash$  $\top$ }
		\infer1{$\neg$  $\top$  $\vdash$ }
		\infer1{$\neg$  $\top$  $\vdash$  $\bot$ }
		\infer1{$\top$  \&  $\neg$  $\top$  $\vdash$  $\bot$ }
		\end{prooftree}
		\quad
		\begin{prooftree}
		\infer0{$\top$  \&  $\neg$  $\top$  $\vdash$  $\top$ }
		\end{prooftree}
		\]
	\end{center}



\chapter{Reduced Subclassical Systems}

\documentclass{article}

\usepackage{amsmath}
\usepackage{ebproof}
\usepackage{fullpage}
\usepackage[utf8]{inputenc}
\usepackage{newunicodechar}
\usepackage{stix}

\newunicodechar{Γ}{\Gamma}
\newunicodechar{Δ}{\Delta}

\newunicodechar{Θ}{\Theta}
\newunicodechar{Λ}{\Lambda}

\newunicodechar{Ξ}{\Xi}
\newunicodechar{Π}{\Pi}

\newunicodechar{Φ}{\Phi}
\newunicodechar{Ψ}{\Psi}

\newunicodechar{Ω}{\Omega}

\newunicodechar{⊢}{\vdash}

\newunicodechar{⊕}{\oplus}
\newunicodechar{¬}{\neg}

\newunicodechar{⊗}{\otimes}

\newunicodechar{→}{\rightarrow}
\newunicodechar{←}{\leftarrow}

\newunicodechar{↔}{\leftrightarrow}
\newunicodechar{↮}{\nleftrightarrow}

\newunicodechar{∧}{\wedge}
\newunicodechar{∨}{\vee}

\newunicodechar{⅋}{\upand}
\newunicodechar{↛}{\nrightarrow}
\newunicodechar{↚}{\nleftarrow}

\newunicodechar{⊥}{\bot}
\newunicodechar{⊤}{\top}

\setlength{\parindent}{0em}

\author{James Martin, Ian D.L.N. Mclean}
\title{The Reduced Lattice of Subclassical Sequent Calculi}

\begin{document}

\maketitle

\begin{abstract}
The lattice of sequent calculi formed from the combination of functionally incomplete sets of operations and structural rules in a classical metalanguage; purely classical logical connectives and Boolean functions that are functionally incomplete are isolated from each other, so in order to extend systems of functional incompleteness we need to extend them by at least one non-classical operator.
\end{abstract}

\part{Preliminaries}
\begin{center}
	\begin{flushleft}
		The key concept to the construction of logical calculi that are distinctly different from classical logic is functional incompleteness with respect to the Boolean domain and Boolean functions.
	\end{flushleft}
	\begin{flushleft}
		If a calculus has any classical logical connectives such that we can express every theorem of the classical calculus then our logical calculi would degenerate to the classical calculus and we'd lose in general the specificity that is gained by constructive reasoning.
	\end{flushleft}
	\begin{flushleft}
		Roughly speaking, if any set of operators or functions is not a subset of at least one of the five functionally incomplete sets then that set of operators or functions is functionally complete.
	\end{flushleft}
	\begin{flushleft}
		We generalize this in a manner analogous to Tarski's original formal interpretation as applied to the problem of decidability of theories in standard formalization. If we have collections of non-classical or sub-classical operators or functions that can be interpreted in at least one of the five functionally incomplete sets then those collections of operators or functions are functionally incomplete with respect to the Boolean domain and Boolean functions.
	\end{flushleft}
	\begin{flushleft}
	We can reduce our analysis to the composition of unitless affine and monotone classes of functions and their subclassical analogues due to the fact that the triples of the minimal functionally complete sets concern only those classes; the treatment of unit preservation can be analyzed separately.
	\end{flushleft}
	\begin{flushleft}
		Finally, if we restrict ourselves to only those logical connectives which are classical then the systems can not extend to each other and it seems can not interpret each other, but if we extend these systems by some non-classical logical connective that is compatible with a given set of functionally incomplete logical connectives then we can extend between these systems and interpret between them.
	\end{flushleft}
	\section{Interpretations}
		\subsection{Operational Interpretations}
		\subsection{Structural Interpretations}
		\subsection{Systematic Interpretations}
	\section{Formal Definition of Subclassical Theories}
		\subsection{Monotonic Theories}
		$\left\{ Γ ⊢ A \right\} \bigcup \left\{ A ⊢ Δ \right\}$
		\subsection{Affine Theories}
		$\left\{ Γ ⊢ A \right\} \bigcap \left\{ A ⊢ Δ \right\}=\emptyset$
		\subsection{Self-Dual Theories}
		$\left\{ A ⊢ A \right\}$

\end{center}

\newpage
\part{The Structural Layer}
\begin{center}
	The structural monotone and affine sequent systems.

	\section{Structural Monotone Calculus}
		\subsection{Structural Rules}
		\begin{center}
			\[
			\begin{prooftree}
			\infer0[Id]{A ⊢ A}
			\end{prooftree}
			\]

			\[
			\begin{prooftree}
			\hypo{Γ ⊢ A}
			\hypo{A ⊢ Δ}
			\infer2[Cut]{Γ ⊢ Δ}
			\end{prooftree}
			\]

			\[
			\begin{prooftree}
			\hypo{Γ ⊢ Δ}
			\infer1[wL]{Γ, A ⊢ Δ}
			\end{prooftree}
			\qquad
			\begin{prooftree}
			\hypo{Γ ⊢ Δ}
			\infer1[Rw]{Γ ⊢ A, Δ}
			\end{prooftree}
			\]

			\[
			\begin{prooftree}
			\hypo{Γ, A, A ⊢ Δ}
			\infer1[cL]{Γ, A ⊢ Δ}
			\end{prooftree}
			\qquad
			\begin{prooftree}
			\hypo{Γ ⊢ A, A Δ}
			\infer1[Rc]{Γ ⊢ A, Δ}
			\end{prooftree}
			\]

			\[
			\begin{prooftree}
			\hypo{Γ_0, A, B, Γ_1 ⊢ Δ}
			\infer1[pL]{Γ_0, B, A, Γ_1 ⊢ Δ}
			\end{prooftree}
			\qquad
			\begin{prooftree}
			\hypo{Γ ⊢ Δ_1, A, B, Δ_0}
			\infer1[Rp]{Γ ⊢ Δ_1, B, A, Δ_0}
			\end{prooftree}
			\]
		\end{center}

		\subsection{Operational Rules}
		\begin{center}

			\subsubsection{Multiplicatives}
			\begin{center}
				\[
				\begin{prooftree}
				\hypo{Γ, A, B ⊢ Δ}
				\infer1{Γ, A ∧ B ⊢ Δ}
				\end{prooftree}
				\quad
				\begin{prooftree}
				\hypo{Γ ⊢ A, Δ}
				\hypo{Γ ⊢ B, Δ}
				\infer2{Γ ⊢ A ∧ B, Δ}
				\end{prooftree}
				\]

				\[
				\begin{prooftree}
				\hypo{Γ, A ⊢ Δ}
				\hypo{Γ, B ⊢ Δ}
				\infer2{Γ, A ∨ B ⊢ Δ}
				\end{prooftree}
				\quad
				\begin{prooftree}
				\hypo{Γ ⊢ A, B, Δ}
				\infer1{Γ ⊢ A ∨ B, Δ}
				\end{prooftree}
				\]
			\end{center}
		\end{center}

		\subsection{Theorems}
		\begin{center}
			\begin{flushleft}
				Non-contradiction and the excluded middle are inexpressible in the calculus, so we get immediate invalidation of logical explosion and logical implosion.
			\end{flushleft}
		\end{center}

	\section{Structural Subaffine Calculus}
		\subsection{Structural Rules}
		\begin{center}
			\[
			\begin{prooftree}
			\infer0[Id]{A ⊢ A}
			\end{prooftree}
			\]

			\[
			\begin{prooftree}
			\hypo{Γ ⊢ A}
			\hypo{A ⊢ Δ}
			\infer2[Cut]{Γ ⊢ Δ}
			\end{prooftree}
			\]

			\[
			\begin{prooftree}
			\hypo{Γ ⊢ Δ}
			\infer1[wL]{Γ, A ⊢ Δ}
			\end{prooftree}
			\qquad
			\begin{prooftree}
			\hypo{Γ ⊢ Δ}
			\infer1[Rw]{Γ ⊢ A, Δ}
			\end{prooftree}
			\]

			\[
			\begin{prooftree}
			\hypo{Γ, A, A ⊢ Δ}
			\infer1[cL]{Γ, A ⊢ Δ}
			\end{prooftree}
			\qquad
			\begin{prooftree}
			\hypo{Γ ⊢ A, A Δ}
			\infer1[Rc]{Γ ⊢ A, Δ}
			\end{prooftree}
			\]

			\[
			\begin{prooftree}
			\hypo{Γ_0, A, B, Γ_1 ⊢ Δ}
			\infer1[pL]{Γ_0, B, A, Γ_1 ⊢ Δ}
			\end{prooftree}
			\qquad
			\begin{prooftree}
			\hypo{Γ ⊢ Δ_1, A, B, Δ_0}
			\infer1[Rp]{Γ ⊢ Δ_1, B, A, Δ_0}
			\end{prooftree}
			\]
		\end{center}

		\subsection{Operational Rules}
		\begin{center}
			\[
			\begin{prooftree}
			\hypo{Γ ⊢ A, B, Δ}
			\hypo{Γ, A, B ⊢ Δ}
			\infer2{Γ, A ↔ B ⊢ Δ}
			\end{prooftree}
			\quad
			\begin{prooftree}
			\hypo{Γ, A ⊢ B, Δ}
			\hypo{Γ, B ⊢ A, Δ}
			\infer2{Γ ⊢ A ↔ B, Δ}
			\end{prooftree}
			\]

			\[
			\begin{prooftree}
			\hypo{Γ, A ⊢ B, Δ}
			\hypo{Γ, B ⊢ A, Δ}
			\infer2{Γ, A ↮ B ⊢ Δ}
			\end{prooftree}
			\quad
			\begin{prooftree}
			\hypo{Γ ⊢ A, B, Δ}
			\hypo{Γ, A, B ⊢ Δ}
			\infer2{Γ ⊢ A ↮ B, Δ}
			\end{prooftree}
			\]
		\end{center}

		\subsection{Theorems}
		\begin{center}
		\end{center}

	\section{Structural Self-Dual Calculus}
		\subsection{Structural Rules}
		\begin{center}
			\[
			\begin{prooftree}
			\infer0[Id]{A ⊢ A}
			\end{prooftree}
			\]

			\[
			\begin{prooftree}
			\hypo{Γ ⊢ A}
			\hypo{A ⊢ Δ}
			\infer2[Cut]{Γ ⊢ Δ}
			\end{prooftree}
			\]

			\[
			\begin{prooftree}
			\hypo{Γ ⊢ Δ}
			\infer1[wL]{Γ, A ⊢ Δ}
			\end{prooftree}
			\qquad
			\begin{prooftree}
			\hypo{Γ ⊢ Δ}
			\infer1[Rw]{Γ ⊢ A, Δ}
			\end{prooftree}
			\]

			\[
			\begin{prooftree}
			\hypo{Γ, A, A ⊢ Δ}
			\infer1[cL]{Γ, A ⊢ Δ}
			\end{prooftree}
			\qquad
			\begin{prooftree}
			\hypo{Γ ⊢ A, A Δ}
			\infer1[Rc]{Γ ⊢ A, Δ}
			\end{prooftree}
			\]

			\[
			\begin{prooftree}
			\hypo{Γ_0, A, B, Γ_1 ⊢ Δ}
			\infer1[pL]{Γ_0, B, A, Γ_1 ⊢ Δ}
			\end{prooftree}
			\qquad
			\begin{prooftree}
			\hypo{Γ ⊢ Δ_1, A, B, Δ_0}
			\infer1[Rp]{Γ ⊢ Δ_1, B, A, Δ_0}
			\end{prooftree}
			\]
		\end{center}

		\subsection{Operational Rules}
			\begin{center}
				\[
				\begin{prooftree}
				\hypo{Γ ⊢ A, Δ}
				\infer1{Γ, ¬ A ⊢ Δ}
				\end{prooftree}
				\quad
				\begin{prooftree}
				\hypo{Γ, A ⊢ Δ}
				\infer1{Γ ⊢ ¬A, Δ}
				\end{prooftree}
				\]
			\end{center}

		\subsection{Theorems}
		\begin{center}
		\end{center}

	\section{Structural Affine Calculus}
		\subsection{Structural Rules}
		\begin{center}
			\[
			\begin{prooftree}
			\infer0[Id]{A ⊢ A}
			\end{prooftree}
			\]

			\[
			\begin{prooftree}
			\hypo{Γ ⊢ A}
			\hypo{A ⊢ Δ}
			\infer2[Cut]{Γ ⊢ Δ}
			\end{prooftree}
			\]

			\[
			\begin{prooftree}
			\hypo{Γ ⊢ Δ}
			\infer1[wL]{Γ, A ⊢ Δ}
			\end{prooftree}
			\qquad
			\begin{prooftree}
			\hypo{Γ ⊢ Δ}
			\infer1[Rw]{Γ ⊢ A, Δ}
			\end{prooftree}
			\]

			\[
			\begin{prooftree}
			\hypo{Γ, A, A ⊢ Δ}
			\infer1[cL]{Γ, A ⊢ Δ}
			\end{prooftree}
			\qquad
			\begin{prooftree}
			\hypo{Γ ⊢ A, A Δ}
			\infer1[Rc]{Γ ⊢ A, Δ}
			\end{prooftree}
			\]

			\[
			\begin{prooftree}
			\hypo{Γ_0, A, B, Γ_1 ⊢ Δ}
			\infer1[pL]{Γ_0, B, A, Γ_1 ⊢ Δ}
			\end{prooftree}
			\qquad
			\begin{prooftree}
			\hypo{Γ ⊢ Δ_1, A, B, Δ_0}
			\infer1[Rp]{Γ ⊢ Δ_1, B, A, Δ_0}
			\end{prooftree}
			\]
		\end{center}

		\subsection{Operational Rules}
		\begin{center}
			\[
			\begin{prooftree}
			\hypo{Γ ⊢ A, Δ}
			\infer1{Γ, ¬ A ⊢ Δ}
			\end{prooftree}
			\quad
			\begin{prooftree}
			\hypo{Γ, A ⊢ Δ}
			\infer1{Γ ⊢ ¬A, Δ}
			\end{prooftree}
			\]

			\[
			\begin{prooftree}
			\hypo{Γ ⊢ A, Δ}
			\infer1{Γ, ¬ A ⊢ Δ}
			\end{prooftree}
			\quad
			\begin{prooftree}
			\hypo{Γ, A ⊢ Δ}
			\infer1{Γ ⊢ ¬A, Δ}
			\end{prooftree}
			\]

			\[
			\begin{prooftree}
			\hypo{Γ ⊢ A, B, Δ}
			\hypo{Γ, A, B ⊢ Δ}
			\infer2{Γ, A ↔ B ⊢ Δ}
			\end{prooftree}
			\quad
			\begin{prooftree}
			\hypo{Γ, A ⊢ B, Δ}
			\hypo{Γ, B ⊢ A, Δ}
			\infer2{Γ ⊢ A ↔ B, Δ}
			\end{prooftree}
			\]

			\[
			\begin{prooftree}
			\hypo{Γ, A ⊢ B, Δ}
			\hypo{Γ, B ⊢ A, Δ}
			\infer2{Γ, A ↮ B ⊢ Δ}
			\end{prooftree}
			\quad
			\begin{prooftree}
			\hypo{Γ ⊢ A, B, Δ}
			\hypo{Γ, A, B ⊢ Δ}
			\infer2{Γ ⊢ A ↮ B, Δ}
			\end{prooftree}
			\]
		\end{center}

		\subsection{Theorems}
		\begin{center}
		\end{center}

\end{center}

\newpage
\part{The Commutative Layer}
\begin{center}
	The commutative monotone, and affine sequent systems.
\end{center}

\newpage
\part{The Non-Commutative Erasure Layer}
\begin{center}
	The non-commutative erasable monotone, and affine sequent systems.
\end{center}

\newpage
\part{The Non-Commutative Cloning Layer}
\begin{center}
	The non-commutative clonable monotone, and affine sequent systems.
\end{center}

\newpage
\part{The No-Cloning and No-Erasure Non-Commutative Layer}
\begin{center}
	The non-commutative monotone, and affine sequent systems.
\end{center}



\end{document}


\chapter{Research Proposal Graph of Supercalculi, conjugat}
}

\hypertarget{manuscript-of-research-proposal-graph-of-supercalculi-conjugated-calculi-pairs-and-subcalculi}{%
\section{Manuscript of Research Proposal: Graph of Supercalculi,
conjugated calculi pairs, and
Subcalculi}\label{manuscript-of-research-proposal-graph-of-supercalculi-conjugated-calculi-pairs-and-subcalculi}}

\hypertarget{introduction--}{%
\subsection{\texorpdfstring{ \textbf{Introduction -}
}{ Introduction - }}\label{introduction--}}

This section may include:

§ What is to be done and the context of the project.

\begin{itemize}
\tightlist
\item
\end{itemize}

\begin{itemize}
\tightlist
\item
\end{itemize}

§ What is being done both generally and specifically in the same or
related areas. (The reviewer should know that you know what is going on
in the area in which you are proposing.)

§ An explanation and justification for unique or innovative approaches.
(These are selling points about what makes your project special, unique
and compelling and why it should be funded.)

\hypertarget{need-statement}{%
\subsection{\texorpdfstring{\textbf{Need
Statement}}{Need Statement}}\label{need-statement}}

§ What needs to be done and why?

§ What significant needs are you trying to meet? Compared to other
projects in the same area, what sets yours apart in terms of need?

§ What services are to be delivered? Why? Use specifics from preliminary
studies, needs assessment, documentation, and data supporting your
proposal.

§ What gaps that your work can fill exist in the knowledge base of your
field?

\textbf{Gaps in the knowledge base}

The current understanding of sequent calculi and their relationships is
fragmented and limited to specific systems or isolated aspects. There is
a lack of a comprehensive framework that encompasses the diverse
landscape of sequent calculi and their intricate connections. This
research aims to address this gap by constructing a comprehensive graph
of sequent calculi, providing a holistic view of the relationships and
properties within this domain.

\textbf{Significant needs}

The development of novel logical systems and the advancement of
automated reasoning techniques require a deeper understanding of the
structural and semantic foundations of sequent calculi. A comprehensive
graph of sequent calculi serves as a valuable tool for identifying
patterns, exploring uncharted territories, and deriving new theorems
within the realm of non-classical logic.

\textbf{Uniqueness of the approach}

This research distinguishes itself from existing work in its emphasis on
the graph-based representation of sequent calculi. This approach offers
a novel perspective on the relationships between different systems and
facilitates the identification of hidden connections and patterns.
Additionally, the focus on non-classical logics expands the scope of the
research, addressing a less explored area of sequent calculi.

\textbf{Compared to other projects in the same area, what sets yours
apart in terms of need?}

There have been several other projects that have proposed graph-based
representations of sequent calculi. However, these projects have
typically focused on specific aspects of sequent calculi, such as their
structural properties or their computational complexity. This research
proposal aims to develop a more comprehensive graph-based representation
that can capture the full range of features and relationships between
different sequent calculi.

In addition, this research proposal will develop a formal semantics for
the graph-based representation, ensuring that it is consistent with the
traditional semantics of sequent calculi. This will make the graph-based
representation a more powerful tool for understanding and analyzing
sequent calculi.

Finally, this research proposal will explore the expressiveness and
limitations of the graph-based representation, identifying the types of
logical systems that can be effectively represented and the limitations
of this approach. This will provide valuable information for researchers
who are considering using the graph-based representation for their own
work.

\textbf{Overall, this research addresses a significant gap in the
knowledge base of sequent calculi by providing a comprehensive and novel
representation that can be leveraged to address important needs in the
field of logic and automated reasoning.}

§ Is the problem both significant and manageable? Do you have the
resources to handle the problem?

\hypertarget{goals-and-objectives}{%
\subsection{\texorpdfstring{\textbf{Goals and
Objectives}}{Goals and Objectives}}\label{goals-and-objectives}}

\textbf{Goal:} Establish a foundational understanding of the graph of
sequent calculi for non-classical logics

\textbf{Objectives:}

1.1.1 Construct a comprehensive graph of sequent calculi for
non-classical logics, capturing the structural and semantic
relationships between different systems.

1.1.2 Analyze the graph to identify patterns, regularities, and
connections that characterize the landscape of non-classical logics.

1.1.3 Develop a framework for navigating and interpreting the graph,
enabling researchers to effectively explore and utilize this
representation of non-classical logics.

\textbf{Activities:}

1.1.1.1 Gather and analyze existing sequent calculi for non-classical
logics, encompassing a diverse range of systems and formalisms.

1.1.1.2 Identify the formal relationships between different sequent
calculi, considering structural similarities, semantic equivalences, and
embedding possibilities.

1.1.1.3 Represent the identified relationships as nodes and edges in a
graph, creating a comprehensive visual representation of the landscape
of non-classical logics.

1.1.2.1 Analyze the graph structure to identify patterns, regularities,
and connections that reveal underlying principles and relationships
within the realm of non-classical logics.

1.1.2.2 Investigate the implications of the identified patterns and
relationships for the expressiveness, computational complexity, and
decidability of non-classical logics.

1.1.2.3 Explore the potential of the graph to guide the development of
new non-classical logics with tailored properties and applications.

1.1.3.1 Develop a formal framework for interpreting and navigating the
graph, providing a structured approach to extracting information and
insights from this representation.

1.1.3.2 Define graph-based metrics and measures to characterize the
structural and semantic features of non-classical logics represented in
the graph.

1.1.3.3 Create a user-friendly interface or tool that facilitates the
exploration and analysis of the graph, enabling researchers to
effectively utilize this resource.

\textbf{Measurement:}

The success of this research will be evaluated based on the following
criteria:

\begin{itemize}
\item
  \begin{quote}
  Comprehensiveness of the constructed graph of sequent calculi for
  non-classical logics
  \end{quote}
\item
  \begin{quote}
  Identification of novel patterns, regularities, and connections within
  the graph
  \end{quote}
\item
  \begin{quote}
  Development of a comprehensive framework for interpreting and
  navigating the graph
  \end{quote}
\item
  \begin{quote}
  Creation of graph-based metrics and measures for characterizing
  non-classical logics
  \end{quote}
\item
  \begin{quote}
  Development of a user-friendly interface or tool for exploring and
  analyzing the graph
  \end{quote}
\end{itemize}

\textbf{Outcomes:}

\begin{itemize}
\item
  \begin{quote}
  A comprehensive graph of sequent calculi for non-classical logics,
  providing a visual representation of the relationships between
  different systems
  \end{quote}
\item
  \begin{quote}
  A deeper understanding of the structural and semantic landscape of
  non-classical logics, revealed through the analysis of the graph
  \end{quote}
\item
  \begin{quote}
  A framework for navigating and interpreting the graph, enabling
  researchers to effectively explore and utilize this representation
  \end{quote}
\item
  \begin{quote}
  Graph-based metrics and measures for characterizing non-classical
  logics
  \end{quote}
\item
  \begin{quote}
  A user-friendly interface or tool for exploring and analyzing the
  graph
  \end{quote}
\end{itemize}

\textbf{Products:}

\begin{itemize}
\item
  \begin{quote}
  A research paper presenting the constructed graph, its analysis, and
  the developed framework for interpretation and navigation
  \end{quote}
\item
  \begin{quote}
  Open-source software or tools for visualizing and analyzing the graph
  \end{quote}
\item
  \begin{quote}
  Presentations at conferences and workshops to disseminate the findings
  and promote the utilization of the graph
  \end{quote}
\end{itemize}

\textbf{Goal:} Establish a comprehensive framework for identifying and
classifying connections between sequent calculi

\textbf{Objectives:}

2.1.1 Develop a formal taxonomy of connections between sequent calculi,
encompassing various types such as extensions, subtensions, conjugate
duals, hyperextensions, hyposubtensions, supercalculi, subcalculi,
hypocalculi, hypercalculi, and hyperduals.

2.1.2 Design a systematic methodology for identifying and classifying
connections between sequent calculi based on their structural and
semantic features.

2.1.3 Implement a computational tool or algorithm that automates the
identification and classification of connections between sequent
calculi.

\textbf{Activities:}

2.1.1.1 Analyze the existing literature on connections between sequent
calculi to identify and categorize the different types of connections.

2.1.1.2 Define formal criteria and properties for each type of
connection, ensuring a clear and consistent classification scheme.

2.1.1.3 Construct a hierarchical taxonomy of connections, representing
the relationships and distinctions between different types.

2.1.2.1 Develop a systematic framework for analyzing the structural and
semantic features of sequent calculi to identify potential connections.

2.1.2.2 Define formal procedures for identifying and classifying
connections based on the identified structural and semantic features.

2.1.2.3 Validate the proposed methodology by applying it to a range of
examples, ensuring its effectiveness and generality.

2.1.3.1 Design and implement a computational tool or algorithm that
automates the identification and classification of connections based on
the developed methodology.

2.1.3.2 Test and evaluate the computational tool or algorithm using a
comprehensive set of examples, ensuring its accuracy and efficiency.

2.1.3.3 Develop a user-friendly interface or tool that facilitates the
utilization of the computational tool or algorithm for researchers and
practitioners.

\textbf{Measurement:}

The success of this research will be evaluated based on the following
criteria:

\begin{itemize}
\item
  \begin{quote}
  Comprehensiveness and clarity of the proposed taxonomy of connections
  between sequent calculi
  \end{quote}
\item
  \begin{quote}
  Effectiveness and generality of the systematic methodology for
  identifying and classifying connections
  \end{quote}
\item
  \begin{quote}
  Accuracy and efficiency of the computational tool or algorithm for
  automating the identification and classification of connections
  \end{quote}
\item
  \begin{quote}
  Usability and accessibility of the user-friendly interface or tool for
  researchers and practitioners
  \end{quote}
\end{itemize}

\textbf{Outcomes:}

\begin{itemize}
\item
  \begin{quote}
  A formal taxonomy of connections between sequent calculi, providing a
  structured and comprehensive classification of different types of
  connections
  \end{quote}
\item
  \begin{quote}
  A systematic methodology for identifying and classifying connections
  between sequent calculi, enabling researchers to effectively analyze
  and categorize relationships between different systems
  \end{quote}
\item
  \begin{quote}
  A computational tool or algorithm that automates the identification
  and classification of connections between sequent calculi,
  streamlining the process and making it more accessible
  \end{quote}
\item
  \begin{quote}
  A user-friendly interface or tool that facilitates the utilization of
  the computational tool or algorithm, providing researchers and
  practitioners with a convenient and accessible resource
  \end{quote}
\end{itemize}

\textbf{Products:}

\begin{itemize}
\item
  \begin{quote}
  A research paper presenting the proposed taxonomy, methodology, and
  computational tool or algorithm
  \end{quote}
\item
  \begin{quote}
  Open-source software or tools for identifying and classifying
  connections between sequent calculi
  \end{quote}
\item
  \begin{quote}
  Presentations at conferences and workshops to disseminate the findings
  and promote the utilization of the tools
  \end{quote}
\end{itemize}

\textbf{Goal:} Develop a systematic approach for designing novel logical
calculi with tailored properties

\textbf{Objectives:}

3.1 Establish a framework for identifying and classifying the desired
properties of novel logical calculi, considering factors such as
expressiveness, computational complexity, decidability, and
applications.

3.2 Develop a methodological framework for designing logical calculi
with specific properties, utilizing graph-based representations and
structural analysis techniques.

3.3 Implement a computational tool or algorithm that automates the
design and verification of logical calculi with tailored properties.

\textbf{Activities:}

3.1.1 Analyze existing logical calculi and their properties to identify
relationships between properties and structural features.

3.1.2 Define formal criteria and measures for evaluating the
expressiveness, computational complexity, decidability, and
applicability of logical calculi.

3.1.3 Develop a taxonomy of logical calculi based on their properties,
enabling researchers to systematically explore and categorize different
systems.

3.2.1 Design graph-based representations of logical calculi, capturing
their structural relationships and highlighting potential
property-determining features.

3.2.2 Develop structural analysis techniques for identifying and
characterizing properties of logical calculi based on their graph
representations.

3.2.3 Establish a systematic methodology for designing logical calculi
with specific properties, utilizing graph-based representations and
structural analysis techniques.

3.3.1 Design and implement a computational tool or algorithm that
automates the design and verification of logical calculi with tailored
properties.

3.1.3.2 Test and evaluate the computational tool or algorithm using a
comprehensive set of examples, ensuring its accuracy and effectiveness
in designing and verifying logical calculi with specific properties.

3.3.3 Develop a user-friendly interface or tool that facilitates the
utilization of the computational tool or algorithm for researchers and
practitioners.

\textbf{Measurement:}

The success of this research will be evaluated based on the following
criteria:

\begin{itemize}
\item
  \begin{quote}
  Comprehensiveness and clarity of the proposed framework for
  identifying and classifying properties of logical calculi
  \end{quote}
\item
  \begin{quote}
  Effectiveness and generality of the methodological framework for
  designing logical calculi with specific properties
  \end{quote}
\item
  \begin{quote}
  Accuracy and efficiency of the computational tool or algorithm for
  automating the design and verification of logical calculi with
  tailored properties
  \end{quote}
\item
  \begin{quote}
  Usability and accessibility of the user-friendly interface or tool for
  researchers and practitioners
  \end{quote}
\end{itemize}

\textbf{Outcomes:}

\begin{itemize}
\item
  \begin{quote}
  A formal framework for identifying and classifying the properties of
  logical calculi, providing a structured and comprehensive approach to
  understanding and characterizing different types of properties
  \end{quote}
\item
  \begin{quote}
  A methodological framework for designing logical calculi with specific
  properties, enabling researchers to systematically design and develop
  logical systems with tailored characteristics
  \end{quote}
\item
  \begin{quote}
  A computational tool or algorithm that automates the design and
  verification of logical calculi with tailored properties, streamlining
  the process and making it more accessible
  \end{quote}
\item
  \begin{quote}
  A user-friendly interface or tool that facilitates the utilization of
  the computational tool or algorithm, providing researchers and
  practitioners with a convenient and accessible resource for designing
  logical systems with specific properties
  \end{quote}
\end{itemize}

\textbf{Products:}

\begin{itemize}
\item
  \begin{quote}
  A research paper presenting the proposed framework, methodology, and
  computational tool or algorithm
  \end{quote}
\item
  \begin{quote}
  Open-source software or tools for designing and verifying logical
  calculi with tailored properties
  \end{quote}
\item
  \begin{quote}
  Presentations at conferences and workshops to disseminate the findings
  and promote the utilization of the tools
  \end{quote}
\end{itemize}

\textbf{Goal:} Explore and demonstrate the practical applications of the
graph of sequent calculi

\textbf{Objectives:}

4.1.1 Identify potential applications of the graph of sequent calculi in
various domains, including formal verification, automated reasoning,
proof search, and logical programming.

4.1.2 Develop concrete use cases and examples that showcase the
applicability of the graph in these domains, demonstrating its ability
to solve practical problems and enhance existing techniques.

4.1.3 Evaluate the effectiveness and efficiency of the graph-based
approach compared to traditional methods in addressing specific
application scenarios.

\textbf{Activities:}

4.1.1.1 Conduct a comprehensive survey of existing applications of
sequent calculi and related techniques to identify potential areas where
the graph representation can be applied.

4.1.1.2 Collaborate with experts from different domains, such as
software engineering, artificial intelligence, and computational
linguistics, to explore potential applications in their respective
fields.

4.1.1.3 Analyze the requirements and challenges of specific application
domains to identify how the graph can be effectively utilized to address
those challenges.

4.1.2.1 Develop concrete use cases for each identified application
domain, providing detailed descriptions of how the graph can be used to
solve specific problems or enhance existing techniques.

4.1.2.2 Implement prototypes or tools that demonstrate the practicality
of the graph-based approach in the selected use cases.

4.1.2.3 Evaluate the performance and effectiveness of the graph-based
approach compared to traditional methods in the chosen use cases.

4.1.3.1 Conduct rigorous testing and evaluation of the graph-based
approach on a variety of benchmark problems and real-world applications.

4.1.3.2 Compare the computational complexity and resource requirements
of the graph-based approach to traditional methods.

4.1.3.3 Analyze the qualitative aspects of the graph-based approach,
such as its ability to provide insights, identify patterns, and simplify
reasoning processes.

\textbf{Measurement:}

The success of this research will be evaluated based on the following
criteria:

\begin{itemize}
\item
  \begin{quote}
  Breadth and diversity of identified applications of the graph of
  sequent calculi
  \end{quote}
\item
  \begin{quote}
  Practicality and effectiveness of the graph-based approach in solving
  real-world problems
  \end{quote}
\item
  \begin{quote}
  Performance and efficiency of the graph-based approach compared to
  traditional methods
  \end{quote}
\item
  \begin{quote}
  Qualitative benefits of the graph-based approach, such as its ability
  to provide insights, identify patterns, and simplify reasoning
  processes
  \end{quote}
\end{itemize}

\textbf{Outcomes:}

\begin{itemize}
\item
  \begin{quote}
  A comprehensive catalog of potential applications of the graph of
  sequent calculi in various domains
  \end{quote}
\item
  \begin{quote}
  Concrete use cases and examples that demonstrate the applicability of
  the graph in solving practical problems and enhancing existing
  techniques
  \end{quote}
\item
  \begin{quote}
  Benchmark results and performance evaluations comparing the
  graph-based approach to traditional methods
  \end{quote}
\item
  \begin{quote}
  Case studies and real-world applications that showcase the
  effectiveness and practicality of the graph-based approach
  \end{quote}
\end{itemize}

\textbf{Products:}

\begin{itemize}
\item
  \begin{quote}
  A research paper presenting the identified applications, use cases,
  evaluations, and case studies
  \end{quote}
\item
  \begin{quote}
  Open-source software or tools that demonstrate the graph-based
  approach in specific application domains
  \end{quote}
\item
  \begin{quote}
  Presentations at conferences and workshops to disseminate the findings
  and promote the adoption of the graph-based approach
  \end{quote}
\end{itemize}

\textbf{Goal:} Establish a Formal Foundation for Graph-Based
Representations of Sequent Calculi

\textbf{Objectives:}

5.1.1 Develop a rigorous mathematical framework for representing sequent
calculi using graph theory concepts.

5.1.2 Define formal semantics for the graph-based representation of
sequent calculi, ensuring consistency with the traditional semantics of
sequent calculi.

5.1.3 Explore the expressiveness and limitations of the graph-based
representation, identifying the types of logical systems that can be
effectively represented and the limitations of this approach.

\textbf{Activities:}

5.1.1.1 Analyze the structural and semantic properties of sequent
calculi to identify the key aspects that can be captured by graph-based
representations.

5.1.1.2 Define graph-theoretic constructs and notations to represent the
logical rules, formulas, and derivations of sequent calculi.

5.1.1.3 Formalize the relationships between graph-theoretic constructs
and the corresponding components of sequent calculi.

5.1.2.1 Develop a formal interpretation of graph-based representations
of sequent calculi, translating graph structures into logical formulas
and derivations.

5.1.2.2 Verify the consistency of the graph-based semantics with the
traditional semantics of sequent calculi, ensuring that the two
approaches produce equivalent results.

5.1.2.3 Explore the implications of the graph-based semantics for the
logical properties of sequent calculi, such as consistency,
completeness, and decidability.

5.1.3.1 Investigate the range of logical systems that can be effectively
represented using the graph-based approach, considering the
expressiveness and limitations of graph theory.

5.1.3.2 Analyze the computational complexity of reasoning tasks, such as
proof search and verification, within the graph-based representation.

5.1.3.3 Identify potential applications of the graph-based approach in
various areas of formal logic and computer science, such as automated
reasoning, metamathematics, and program analysis.

\textbf{Measurement:}

The success of this research will be evaluated based on the following
criteria:

\begin{itemize}
\item
  \begin{quote}
  Clarity and formality of the proposed mathematical framework for
  graph-based representations of sequent calculi
  \end{quote}
\item
  \begin{quote}
  Consistency and equivalence of the graph-based semantics with the
  traditional semantics of sequent calculi
  \end{quote}
\item
  \begin{quote}
  Expressiveness and limitations of the graph-based representation in
  capturing various types of logical systems
  \end{quote}
\item
  \begin{quote}
  Impact of the graph-based approach on computational complexity and
  reasoning tasks in formal logic
  \end{quote}
\item
  \begin{quote}
  Applicability of the graph-based approach in diverse areas of formal
  logic and computer science
  \end{quote}
\end{itemize}

\textbf{Outcomes:}

\begin{itemize}
\item
  \begin{quote}
  A formal mathematical framework for representing sequent calculi using
  graph theory concepts
  \end{quote}
\item
  \begin{quote}
  A formal semantics for the graph-based representation of sequent
  calculi, ensuring consistency with traditional semantics
  \end{quote}
\item
  \begin{quote}
  An analysis of the expressiveness and limitations of the graph-based
  representation in capturing various types of logical systems
  \end{quote}
\item
  \begin{quote}
  Identification of potential applications of the graph-based approach
  in formal logic and computer science
  \end{quote}
\end{itemize}

\textbf{Products:}

\begin{itemize}
\item
  \begin{quote}
  A research paper presenting the proposed framework, semantics,
  analysis, and potential applications
  \end{quote}
\item
  \begin{quote}
  Open-source software or tools for implementing and analyzing
  graph-based representations of sequent calculi
  \end{quote}
\item
  \begin{quote}
  Presentations at conferences and workshops to disseminate the findings
  and promote the adoption of the graph-based approach
  \end{quote}
\end{itemize}

Deliverables:

A comprehensive list of non-classical logics, including their axioms,
rules, and properties.

A new formal framework for representing and reasoning about
non-classical logics.

A new classification of non-classical logics.

A graph of sequent calculi that visually represents the relationships
between these logics.

A list of common patterns and themes in the properties of non-classical
logics.

A list of new areas of research suggested by the analysis of the graph
of sequent calculi.

\hypertarget{section}{%
\subsubsection{}\label{section}}

\hypertarget{section-1}{%
\subsubsection{}\label{section-1}}

\hypertarget{section-2}{%
\subsubsection{}\label{section-2}}

\hypertarget{goal-1-investigating-the-structural-properties-of-sequent-calculi-using-graphs}{%
\subsubsection{Goal 1: Investigating the Structural Properties of
Sequent Calculi Using
Graphs}\label{goal-1-investigating-the-structural-properties-of-sequent-calculi-using-graphs}}

\textbf{Subgoal 1.1: Develop a formal framework for representing sequent
calculi using graphs}

\begin{itemize}
\item
  \begin{quote}
  \textbf{Activities:\\
  }
  \end{quote}

  \begin{itemize}
  \item
    \begin{quote}
    \textbf{Conduct a thorough review of existing formal frameworks for
    representing logical systems, including graph-based approaches and
    algebraic representations.}
    \end{quote}
  \item
    \begin{quote}
    \textbf{Identify the key properties and requirements for a formal
    framework specifically tailored to sequent calculi, considering
    their unique structural and semantic features.}
    \end{quote}
  \item
    \begin{quote}
    \textbf{Design a novel graph-based formal framework that effectively
    captures the structure and semantics of sequent calculi, including
    their inferential rules, proof structures, and logical
    relationships.}
    \end{quote}
  \item
    \begin{quote}
    \textbf{Evaluate the expressiveness and flexibility of the proposed
    framework through the representation of a variety of sequent
    calculi, encompassing both well-established systems such as
    classical and intuitionistic sequent calculus, and newly developed
    sequent calculi with unique features.}
    \end{quote}
  \end{itemize}
\item
  \begin{quote}
  \textbf{How the activities will be carried out:\\
  }
  \end{quote}

  \begin{itemize}
  \item
    \begin{quote}
    \textbf{The review of existing frameworks will involve examining
    academic literature, attending conferences, and consulting with
    experts in the field of sequent calculus and formal logic.}
    \end{quote}
  \item
    \begin{quote}
    \textbf{The identification of key properties and requirements will
    involve analyzing the characteristics of sequent calculi,
    considering their inferential mechanisms, proof structures, and
    logical properties, and the needs of researchers working with these
    systems.}
    \end{quote}
  \item
    \begin{quote}
    \textbf{The design of the novel framework will involve utilizing
    graph theory concepts, formal logic principles, and software
    engineering techniques, ensuring the framework can effectively
    represent the intricate relationships and interactions within
    sequent calculi.}
    \end{quote}
  \item
    \begin{quote}
    \textbf{The evaluation of the framework will involve representing a
    range of sequent calculi, including both well-established systems
    and newly developed ones, to assess its ability to capture their
    structural and semantic features.}
    \end{quote}
  \end{itemize}
\end{itemize}

\textbf{Subgoal 1.2: Analyze the structural properties of sequent
calculi using graph-based techniques}

\begin{itemize}
\item
  \begin{quote}
  \textbf{Activities:\\
  }
  \end{quote}

  \begin{itemize}
  \item
    \begin{quote}
    \textbf{Apply the developed graph-based framework to analyze the
    structural properties of a range of sequent calculi, including their
    connectivity, symmetries, decomposition patterns, and relationships
    between different proof structures.}
    \end{quote}
  \item
    \begin{quote}
    \textbf{Identify and characterize common structural motifs and
    patterns that emerge across different sequent calculi, providing
    insights into their underlying computational properties and logical
    relationships.}
    \end{quote}
  \item
    \begin{quote}
    \textbf{Develop formal theorems that establish connections between
    the structural properties of sequent calculi and their logical
    properties, such as soundness, completeness, and decidability.}
    \end{quote}
  \end{itemize}
\item
  \begin{quote}
  \textbf{How the activities will be carried out:\\
  }
  \end{quote}

  \begin{itemize}
  \item
    \begin{quote}
    \textbf{The analysis of structural properties will involve employing
    graph algorithms and techniques, such as graph isomorphism, subgraph
    detection, and graph decomposition, to examine the graph-based
    representations of sequent calculi.}
    \end{quote}
  \item
    \begin{quote}
    \textbf{The identification of common patterns will involve utilizing
    statistical analysis and pattern recognition methods to identify
    recurring structural motifs and patterns across different sequent
    calculi.}
    \end{quote}
  \item
    \begin{quote}
    \textbf{The development of formal theorems will involve rigorous
    mathematical proofs based on the formal framework, graph theory
    concepts, and logical reasoning principles, establishing connections
    between structural features and logical properties.}
    \end{quote}
  \end{itemize}
\end{itemize}

\textbf{1.3: Develop graph-based algorithms for analyzing the structural
properties of sequent calculi}

\begin{itemize}
\item
  \begin{quote}
  \textbf{Activities:\\
  }
  \end{quote}

  \begin{itemize}
  \item
    \begin{quote}
    \textbf{Design and implement efficient graph algorithms specifically
    tailored to the analysis of sequent calculi, leveraging graph theory
    concepts and formal logic principles.}
    \end{quote}
  \item
    \begin{quote}
    \textbf{Optimize the developed algorithms for performance and
    scalability, considering the potentially large size and complexity
    of sequent calculus representations.}
    \end{quote}
  \item
    \begin{quote}
    \textbf{Integrate the algorithms into a software framework or
    toolset for analyzing sequent calculi, providing researchers with a
    user-friendly environment for exploring structural properties.}
    \end{quote}
  \end{itemize}
\item
  \begin{quote}
  \textbf{How the activities will be carried out:\\
  }
  \end{quote}

  \begin{itemize}
  \item
    \begin{quote}
    \textbf{The design of algorithms will involve utilizing graph theory
    techniques such as graph traversal, pattern matching, and graph
    decomposition, tailoring them to the specific structure and
    semantics of sequent calculi.}
    \end{quote}
  \item
    \begin{quote}
    \textbf{The optimization of algorithms will involve employing
    techniques such as caching, data structures, and parallel
    processing, to improve their execution time and memory usage.}
    \end{quote}
  \item
    \begin{quote}
    \textbf{The integration of algorithms into a software framework will
    involve utilizing software engineering principles, modular design,
    and user interface considerations, to create a user-friendly and
    extensible toolset.}
    \end{quote}
  \end{itemize}
\end{itemize}

\textbf{Subgoal 1.4: Apply graph-based algorithms to investigate the
structural properties of sequent calculi}

\begin{itemize}
\item
  \begin{quote}
  \textbf{Activities:\\
  }
  \end{quote}

  \begin{itemize}
  \item
    \begin{quote}
    \textbf{Utilize the developed graph-based algorithms to analyze the
    structural properties of a range of sequent calculi, including their
    connectivity, symmetries, decomposition patterns, and relationships
    between different proof structures.}
    \end{quote}
  \item
    \begin{quote}
    \textbf{Investigate the impact of structural properties on the
    computational behavior of sequent calculi, such as proof search
    efficiency and the complexity of proof structures.}
    \end{quote}
  \item
    \begin{quote}
    \textbf{Explore the application of graph-based techniques to the
    analysis of sequent calculus extensions and variations, such as
    modal sequent calculi and substructural sequent calculi.}
    \end{quote}
  \end{itemize}
\item
  \begin{quote}
  \textbf{How the activities will be carried out:\\
  }
  \end{quote}

  \begin{itemize}
  \item
    \begin{quote}
    \textbf{The application of algorithms will involve executing the
    developed algorithms on the graph-based representations of sequent
    calculi, collecting and analyzing the resulting data.}
    \end{quote}
  \item
    \begin{quote}
    \textbf{The investigation of the impact of structural properties
    will involve correlating the identified structural features with the
    computational behavior of sequent calculi, using metrics such as
    proof length and search space size.}
    \end{quote}
  \item
    \begin{quote}
    \textbf{The exploration of extensions and variations will involve
    applying the graph-based techniques to the representations of these
    systems, analyzing their structural properties and identifying
    patterns.}
    \end{quote}
  \end{itemize}
\end{itemize}

\begin{itemize}
\tightlist
\item
\end{itemize}

\textbf{Goal: Construct a comprehensive graph of sequent calculi and
leverage it to identify uncharted non-classical languages and derive
novel theorems within the realm of non-classical logic.}

\textbf{Objectives:}

\textbf{1.5.1 Construct a comprehensive graph of sequent calculi.}

\textbf{1.5.2 Identify uncharted non-classical languages based on the
graph of sequent calculi.}

\textbf{1.5.3 Derive novel theorems within the realm of non-classical
logic using the graph of sequent calculi.}

\textbf{Activities:}

\textbf{1.5.1.1 Gather and analyze existing sequent calculi.}

\textbf{1.5.1.2 Identify the relationships between different sequent
calculi.}

\textbf{1.5.1.3 Represent the relationships between sequent calculi as a
graph.}

\textbf{1.5.2.1 Analyze the graph of sequent calculi to identify
potential new non-classical languages.}

\textbf{1.5.2.2 Formulate and investigate new non-classical languages
based on the identified potential.}

\textbf{1.5.3.1 Extract novel theorems from the graph of sequent
calculi.}

\textbf{1.5.3.2 Prove the validity of the extracted theorems.}

\textbf{Evaluation:}

\textbf{The success of this research will be evaluated based on the
following criteria:}

\begin{itemize}
\item
  \begin{quote}
  \textbf{The comprehensiveness of the graph of sequent calculi.\\
  }
  \end{quote}
\item
  \begin{quote}
  \textbf{The number of new non-classical languages identified.\\
  }
  \end{quote}
\item
  \begin{quote}
  \textbf{The novelty and significance of the derived theorems.\\
  }
  \end{quote}
\end{itemize}

\textbf{Timeline:}

\textbf{1.5.1.1 Gather and analyze existing sequent calculi: 6 months}

\textbf{1.5.1.2 Identify the relationships between different sequent
calculi: 3 months}

\textbf{1.5.1.3 Represent the relationships between sequent calculi as a
graph: 3 months}

\textbf{1.5.2.1 Analyze the graph of sequent calculi to identify
potential new non-classical languages: 6 months}

\textbf{1.5.2.2 Formulate and investigate new non-classical languages
based on the identified potential: 12 months}

\textbf{1.5.3.1 Extract novel theorems from the graph of sequent
calculi: 6 months}

\textbf{1.5.3.2 Prove the validity of the extracted theorems: 12 months}

\textbf{Resources:}

\textbf{The research will require the following resources:}

\begin{itemize}
\item
  \begin{quote}
  \textbf{Access to academic literature and databases\\
  }
  \end{quote}
\item
  \begin{quote}
  \textbf{Travel funds to attend conferences\\
  }
  \end{quote}
\item
  \begin{quote}
  \textbf{Collaborations with experts in the field of sequent calculi
  and non-classical logic\\
  }
  \end{quote}
\end{itemize}

\textbf{Expected Outcomes:}

\begin{itemize}
\item
  \begin{quote}
  \textbf{A comprehensive graph of sequent calculi\\
  }
  \end{quote}
\item
  \begin{quote}
  \textbf{Identification of uncharted non-classical languages\\
  }
  \end{quote}
\item
  \begin{quote}
  \textbf{Derivation of novel theorems within the realm of non-classical
  logic\\
  }
  \end{quote}
\end{itemize}

\textbf{Dissemination:}

\textbf{The findings of this research will be disseminated through
publications in peer-reviewed journals, presentations at conferences,
and open-source software development.}

\hypertarget{goal-leverage-the-graph-of-sequent-calculi-to-identify-uncharted-non-classical-languages-and-derive-novel-theorems-within-the-domain-of-non-classical-logic.}{%
\paragraph{Goal: Leverage the graph of sequent calculi to identify
uncharted non-classical languages and derive novel theorems within the
domain of non-classical
logic.}\label{goal-leverage-the-graph-of-sequent-calculi-to-identify-uncharted-non-classical-languages-and-derive-novel-theorems-within-the-domain-of-non-classical-logic.}}

\textbf{Objectives:}

\textbf{1.6.1 Identify uncharted non-classical languages based on the
graph of sequent calculi.}

\textbf{1.6.2 Derive novel theorems within the realm of non-classical
logic using the graph of sequent calculi.}

\textbf{Activities:}

\textbf{1.6.1.1 Analyze the graph of sequent calculi to identify
structural patterns and relationships that suggest the existence of
uncharted non-classical languages.}

\textbf{1.6.1.2 Formulate hypotheses regarding the characteristics and
properties of potential uncharted non-classical languages based on the
identified patterns and relationships.}

\textbf{1.6.1.3 Investigate the formulated hypotheses by constructing
formal representations and analyzing the logical properties of the
proposed uncharted non-classical languages.}

\textbf{1.6.2.1 Identify promising areas within the graph of sequent
calculi that exhibit potential for deriving novel theorems in
non-classical logic.}

\textbf{1.6.2.2 Extract potential theorems from the identified promising
areas by systematically examining the relationships and properties of
sequent calculi within those regions.}

\textbf{1.6.2.3 Formalize and prove the extracted potential theorems
using rigorous mathematical techniques and logical reasoning.}

\textbf{Evaluation:}

\textbf{The success of this research will be evaluated based on the
following criteria:}

\begin{itemize}
\item
  \begin{quote}
  \textbf{The number of uncharted non-classical languages identified.\\
  }
  \end{quote}
\item
  \begin{quote}
  \textbf{The novelty and significance of the identified uncharted
  non-classical languages.\\
  }
  \end{quote}
\item
  \begin{quote}
  \textbf{The number of novel theorems derived in non-classical logic.\\
  }
  \end{quote}
\item
  \begin{quote}
  \textbf{The novelty and significance of the derived novel theorems.\\
  }
  \end{quote}
\end{itemize}

\textbf{Timeline:}

\textbf{1.6.1.1 Analyze the graph of sequent calculi to identify
structural patterns and relationships: 3 months}

\textbf{1.6.1.2 Formulate hypotheses regarding the characteristics and
properties of potential uncharted non-classical languages: 3 months}

\textbf{1.6.1.3 Investigate the formulated hypotheses: 12 months}

\textbf{1.6.2.1 Identify promising areas within the graph of sequent
calculi for deriving novel theorems: 3 months}

\textbf{1.6.2.2 Extract potential theorems from the identified promising
areas: 3 months}

\textbf{1.6.2.3 Formalize and prove the extracted potential theorems: 12
months}

\textbf{Resources:}

\textbf{The research will require the following resources:}

\begin{itemize}
\item
  \begin{quote}
  \textbf{Access to academic literature and databases\\
  }
  \end{quote}
\item
  \begin{quote}
  \textbf{Collaborations with experts in the field of sequent calculi
  and non-classical logic\\
  }
  \end{quote}
\item
  \begin{quote}
  \textbf{Access to computational resources for graph analysis and
  theorem proving\\
  }
  \end{quote}
\end{itemize}

\textbf{Expected Outcomes:}

\begin{itemize}
\item
  \begin{quote}
  \textbf{Identification of uncharted non-classical languages with
  unique properties and applications\\
  }
  \end{quote}
\item
  \begin{quote}
  \textbf{Derivation of novel theorems that expand our understanding of
  non-classical logic and its capabilities\\
  }
  \end{quote}
\end{itemize}

\textbf{Dissemination:}

\textbf{The findings of this research will be disseminated through
publications in peer-reviewed journals, presentations at conferences,
and open-source software development.}

\hypertarget{goal-2-investigating-the-relationship-between-sequent-calculi-and-substructural-logics}{%
\subsubsection{Goal 2: Investigating the Relationship between Sequent
Calculi and Substructural
Logics}\label{goal-2-investigating-the-relationship-between-sequent-calculi-and-substructural-logics}}

\textbf{Subgoal 2.1: Develop a formal framework for representing
conjugated logic pairs using graphs}

\begin{itemize}
\item
  \begin{quote}
  \textbf{Activities:\\
  }
  \end{quote}

  \begin{enumerate}
  \def\labelenumi{\arabic{enumi}.}
  \item
    \begin{quote}
    \textbf{Conduct a comprehensive review of existing formal frameworks
    for representing logical systems, including graph-based approaches
    and algebraic representations, with a particular focus on conjugated
    logic pairs.\\
    }
    \end{quote}
  \item
    \begin{quote}
    \textbf{Identify the key properties and requirements for a formal
    framework specifically tailored to conjugated logic pairs,
    considering their unique structural and semantic features.\\
    }
    \end{quote}
  \item
    \begin{quote}
    \textbf{Design a novel graph-based formal framework that effectively
    captures the intricate structure and semantics of conjugated logic
    pairs, including their ability to handle both classical and
    non-classical logics.\\
    }
    \end{quote}
  \item
    \begin{quote}
    \textbf{Evaluate the expressiveness and flexibility of the proposed
    framework through the representation of a variety of conjugated
    logic pairs, encompassing both well-established systems and newly
    developed ones.\\
    }
    \end{quote}
  \end{enumerate}
\item
  \begin{quote}
  \textbf{How the activities will be carried out:\\
  }
  \end{quote}

  \begin{enumerate}
  \def\labelenumi{\arabic{enumi}.}
  \item
    \begin{quote}
    \textbf{The review of existing frameworks will involve examining
    academic literature, attending conferences, and consulting with
    experts in the field of conjugated logic pairs and formal logic.\\
    }
    \end{quote}
  \item
    \begin{quote}
    \textbf{The identification of key properties and requirements will
    involve analyzing the characteristics of conjugated logic pairs,
    considering their ability to express both classical and
    non-classical logics, and the needs of researchers working with
    these systems.\\
    }
    \end{quote}
  \item
    \begin{quote}
    \textbf{The design of the novel framework will involve utilizing
    graph theory concepts, formal logic principles, and software
    engineering techniques, ensuring the framework can effectively
    represent the complex relationships and interactions within
    conjugated logic pairs.\\
    }
    \end{quote}
  \item
    \begin{quote}
    \textbf{The evaluation of the framework will involve representing a
    range of conjugated logic pairs, including both well-established
    systems such as classical logic and intuitionistic logic paired with
    their non-classical counterparts, and newly developed conjugated
    logic pairs with unique features.\\
    }
    \end{quote}
  \end{enumerate}
\end{itemize}

\textbf{Subgoal 2.2: Establish a connection between the graph-based
representations of sequent calculi and conjugated logic pairs}

\begin{itemize}
\item
  \begin{quote}
  \textbf{Activities:\\
  }
  \end{quote}

  \begin{enumerate}
  \def\labelenumi{\arabic{enumi}.}
  \item
    \begin{quote}
    \textbf{Analyze the graphical representations of sequent calculi and
    conjugated logic pairs to identify structural similarities and
    correspondences.\\
    }
    \end{quote}
  \item
    \begin{quote}
    \textbf{Develop formal mappings between the graph-based
    representations of sequent calculi and conjugated logic pairs,
    preserving their structural and semantic properties.\\
    }
    \end{quote}
  \item
    \begin{quote}
    \textbf{Utilize the established mappings to translate proofs and
    theorems between sequent calculi and conjugated logic pairs.\\
    }
    \end{quote}
  \item
    \begin{quote}
    \textbf{Investigate the implications of these mappings on the
    relationship between the underlying logical systems.\\
    }
    \end{quote}
  \end{enumerate}
\item
  \begin{quote}
  \textbf{How the activities will be carried out:\\
  }
  \end{quote}

  \begin{enumerate}
  \def\labelenumi{\arabic{enumi}.}
  \item
    \begin{quote}
    \textbf{The analysis of graphical representations will involve
    employing graph comparison algorithms and techniques to identify
    common patterns and structural features.\\
    }
    \end{quote}
  \item
    \begin{quote}
    \textbf{The development of formal mappings will involve constructing
    bijections or homomorphisms between the graph-based representations,
    ensuring that logical relationships are maintained.\\
    }
    \end{quote}
  \item
    \begin{quote}
    \textbf{The translation of proofs and theorems will involve applying
    the established mappings to transform logical expressions and proof
    structures.\\
    }
    \end{quote}
  \item
    \begin{quote}
    \textbf{The investigation of implications will involve analyzing the
    consequences of the mappings on the metatheoretical properties of
    the logical systems.\\
    }
    \end{quote}
  \end{enumerate}
\end{itemize}

\begin{itemize}
\item
  \begin{quote}
  \textbf{\hfill\break
  }
  \end{quote}
\end{itemize}

\hypertarget{secondary-objective-harnessing-the-power-of-non-classical-logics}{%
\subsubsection{Secondary Objective: Harnessing the Power of
Non-Classical
Logics}\label{secondary-objective-harnessing-the-power-of-non-classical-logics}}

\hypertarget{goal-3-investigating-the-relationship-between-sequent-calculi-and-supercalculi}{%
\subsubsection{Goal 3: Investigating the Relationship between Sequent
Calculi and
Supercalculi}\label{goal-3-investigating-the-relationship-between-sequent-calculi-and-supercalculi}}

\textbf{Subgoal 3.1: Develop a formal framework for representing
supercalculi using graphs}

\begin{itemize}
\item
  \begin{quote}
  \textbf{Activities:\\
  }
  \end{quote}

  \begin{itemize}
  \item
    \begin{quote}
    Conduct a thorough review of existing formal frameworks for
    representing logical systems, including graph-based approaches and
    algebraic representations, with a particular focus on supercalculi.
    \end{quote}
  \item
    \begin{quote}
    Identify the key properties and requirements for a formal framework
    specifically tailored to supercalculi, considering their unique
    structural and semantic features.
    \end{quote}
  \item
    \begin{quote}
    Design a novel graph-based formal framework that effectively
    captures the intricate structure and semantics of supercalculi,
    including their ability to handle multiple modalities and
    non-standard connectives.
    \end{quote}
  \item
    \begin{quote}
    Evaluate the expressiveness and flexibility of the proposed
    framework through the representation of a variety of supercalculi,
    encompassing both well-established systems and newly developed ones.
    \end{quote}
  \end{itemize}
\item
  \begin{quote}
  \textbf{How the activities will be carried out:\\
  }
  \end{quote}

  \begin{itemize}
  \item
    \begin{quote}
    The review of existing frameworks will involve examining academic
    literature, attending conferences, and consulting with experts in
    the field of supercalculi and formal logic.
    \end{quote}
  \item
    \begin{quote}
    The identification of key properties and requirements will involve
    analyzing the characteristics of supercalculi, considering their
    ability to express multimodal and non-classical logics, and the
    needs of researchers working with these systems.
    \end{quote}
  \item
    \begin{quote}
    The design of the novel framework will involve utilizing graph
    theory concepts, formal logic principles, and software engineering
    techniques, ensuring the framework can effectively represent the
    complex relationships and interactions within supercalculi.
    \end{quote}
  \item
    \begin{quote}
    The evaluation of the framework will involve representing a range of
    supercalculi, including both well-established systems such as modal
    logics and hybrid logics, and newly developed supercalculi with
    unique features.
    \end{quote}
  \end{itemize}
\end{itemize}

\begin{itemize}
\tightlist
\item
\end{itemize}

\hypertarget{goal-4-investigating-the-relationship-between-sequent-calculi-and-substructural-logics}{%
\subsubsection{Goal 4: Investigating the Relationship between Sequent
Calculi and Substructural
Logics}\label{goal-4-investigating-the-relationship-between-sequent-calculi-and-substructural-logics}}

\textbf{Subgoal 4.1: Develop a formal framework for representing
substructural logics using graphs}

\begin{itemize}
\item
  \begin{quote}
  \textbf{Activities:\\
  }
  \end{quote}

  \begin{itemize}
  \item
    \begin{quote}
    Conduct a comprehensive review of existing formal frameworks for
    representing substructural logics, including graph-based approaches
    and algebraic representations.
    \end{quote}
  \item
    \begin{quote}
    Identify the key properties and requirements for a formal framework
    specifically tailored to substructural logics.
    \end{quote}
  \item
    \begin{quote}
    Design a novel graph-based formal framework that effectively
    captures the structure and semantics of substructural logics.
    \end{quote}
  \item
    \begin{quote}
    Evaluate the expressiveness and flexibility of the proposed
    framework through the representation of a variety of substructural
    logics, including linear logic, relevance logic, and intuitionistic
    logic.
    \end{quote}
  \end{itemize}
\item
\item
  \begin{quote}
  \textbf{How the activities will be carried out:\\
  }
  \end{quote}

  \begin{itemize}
  \item
    \begin{quote}
    The review of existing frameworks will involve examining academic
    literature, attending conferences, and consulting with experts in
    the field.
    \end{quote}
  \item
    \begin{quote}
    The identification of key properties and requirements will involve
    analyzing the characteristics of substructural logics and the needs
    of researchers working with these systems.
    \end{quote}
  \item
    \begin{quote}
    The design of the novel framework will involve utilizing graph
    theory concepts, formal logic principles, and software engineering
    techniques.
    \end{quote}
  \item
    \begin{quote}
    The evaluation of the framework will involve representing a range of
    substructural logics, including both well-established systems and
    newly developed ones.
    \end{quote}
  \end{itemize}
\item
\end{itemize}

\textbf{Subgoal 4.2: Establish a connection between the graph-based
representations of sequent calculi and substructural logics}

\begin{itemize}
\item
  \begin{quote}
  \textbf{Activities:\\
  }
  \end{quote}

  \begin{itemize}
  \item
    \begin{quote}
    Analyze the graphical representations of sequent calculi and
    substructural logics to identify structural similarities and
    correspondences.
    \end{quote}
  \item
    \begin{quote}
    Develop formal mappings between the graph-based representations of
    sequent calculi and substructural logics, preserving their
    structural and semantic properties.
    \end{quote}
  \item
    \begin{quote}
    Utilize the established mappings to translate proofs and theorems
    between sequent calculi and substructural logics.
    \end{quote}
  \item
    \begin{quote}
    Investigate the implications of these mappings on the relationship
    between the underlying logical systems.
    \end{quote}
  \end{itemize}
\item
\item
  \begin{quote}
  \textbf{How the activities will be carried out:\\
  }
  \end{quote}

  \begin{itemize}
  \item
    \begin{quote}
    The analysis of graphical representations will involve employing
    graph comparison algorithms and techniques to identify common
    patterns and structural features.
    \end{quote}
  \item
    \begin{quote}
    The development of formal mappings will involve constructing
    bijections or homomorphisms between the graph-based representations,
    ensuring that logical relationships are maintained.
    \end{quote}
  \item
    \begin{quote}
    The translation of proofs and theorems will involve applying the
    established mappings to transform logical expressions and proof
    structures.
    \end{quote}
  \item
    \begin{quote}
    The investigation of implications will involve analyzing the
    consequences of the mappings on the metatheoretical properties of
    the logical systems.
    \end{quote}
  \end{itemize}
\item
\end{itemize}

\hypertarget{foundational-objective-laying-a-rigorous-theoretical-framework}{%
\subsubsection{Foundational Objective: Laying a Rigorous Theoretical
Framework}\label{foundational-objective-laying-a-rigorous-theoretical-framework}}

\hypertarget{goal-5-establishing-a-formal-foundation-for-sequent-calculi}{%
\subsubsection{Goal 5: Establishing a Formal Foundation for Sequent
Calculi}\label{goal-5-establishing-a-formal-foundation-for-sequent-calculi}}

\textbf{5.1: Develop a formal framework for representing sequent calculi
using graphs}

\begin{itemize}
\item
  \begin{quote}
  \textbf{Activities:}
  \end{quote}

  \begin{itemize}
  \item
    \begin{quote}
    Conduct a thorough review of existing formal frameworks for
    representing logical systems, including graph-based approaches and
    algebraic representations.
    \end{quote}
  \item
    \begin{quote}
    Identify the key properties and requirements for a formal framework
    specifically tailored to sequent calculi.
    \end{quote}
  \item
    \begin{quote}
    Design a novel graph-based formal framework that effectively
    captures the structure and semantics of sequent calculi.
    \end{quote}
  \item
    \begin{quote}
    Evaluate the expressiveness and flexibility of the proposed
    framework through the representation of a variety of non-classical
    logics.
    \end{quote}
  \end{itemize}
\item
  \begin{quote}
  \textbf{How the activities will be carried out:}
  \end{quote}

  \begin{itemize}
  \item
    \begin{quote}
    The review of existing frameworks will involve examining academic
    literature, attending conferences, and consulting with experts in
    the field.
    \end{quote}
  \item
    \begin{quote}
    The identification of key properties and requirements will involve
    analyzing the characteristics of sequent calculi and the needs of
    researchers working with these systems.
    \end{quote}
  \item
    \begin{quote}
    The design of the novel framework will involve utilizing graph
    theory concepts, formal logic principles, and software engineering
    techniques.
    \end{quote}
  \item
    \begin{quote}
    The evaluation of the framework will involve representing a range of
    non-classical logics, including both well-established systems and
    newly developed ones.
    \end{quote}
  \end{itemize}
\end{itemize}

\textbf{5.2: Use this framework to investigate the structural properties
of sequent calculi}

\begin{itemize}
\item
  \begin{quote}
  \textbf{Activities:}
  \end{quote}

  \begin{itemize}
  \item
    \begin{quote}
    Apply the developed formal framework to analyze the structural
    properties of a range of sequent calculi, including their
    connectivity, symmetries, and decomposition patterns.
    \end{quote}
  \item
    \begin{quote}
    Identify and characterize common structural motifs and patterns that
    emerge across different sequent calculi.
    \end{quote}
  \item
    \begin{quote}
    Develop formal theorems that establish relationships between the
    structural properties of sequent calculi and their logical
    properties, such as soundness and completeness.
    \end{quote}
  \end{itemize}
\item
  \begin{quote}
  \textbf{How the activities will be carried out:}
  \end{quote}

  \begin{itemize}
  \item
    \begin{quote}
    The analysis of structural properties will involve applying graph
    algorithms and techniques to the graphical representation of sequent
    calculi.
    \end{quote}
  \item
    \begin{quote}
    The identification of common patterns will involve employing pattern
    recognition methods and statistical analysis.
    \end{quote}
  \item
    \begin{quote}
    The development of formal theorems will involve rigorous
    mathematical proofs based on the formal framework and logical
    principles.
    \end{quote}
  \end{itemize}
\end{itemize}

\textbf{5.3: Apply this framework to the development of new
non-classical languages}

\begin{itemize}
\item
  \begin{quote}
  \textbf{Activities:}
  \end{quote}

  \begin{itemize}
  \item
    \begin{quote}
    Utilize the formal framework to design and construct novel
    non-classical logics with tailored properties and applications.
    \end{quote}
  \item
    \begin{quote}
    Employ the framework to systematically explore the space of possible
    non-classical logics, identifying new and interesting logical
    systems.
    \end{quote}
  \item
    \begin{quote}
    Formally analyze and evaluate the properties of the newly developed
    non-classical logics, including their expressiveness, consistency,
    and decidability.
    \end{quote}
  \end{itemize}
\item
  \begin{quote}
  \textbf{How the activities will be carried out:}
  \end{quote}

  \begin{itemize}
  \item
    \begin{quote}
    The design of new logics will involve utilizing the
    framework\textquotesingle s constructions to represent the desired
    logical properties and relationships.
    \end{quote}
  \item
    \begin{quote}
    The exploration of the space of non-classical logics will involve
    employing systematic search algorithms and techniques guided by the
    framework\textquotesingle s constraints.
    \end{quote}
  \item
    \begin{quote}
    The formal analysis of newly developed logics will involve applying
    logical reasoning methods and tools to the
    framework\textquotesingle s representation of these logics.
    \end{quote}
  \end{itemize}
\end{itemize}

\textbf{Deliverables:}

\begin{itemize}
\item
  \begin{quote}
  A comprehensive compendium of non-classical logics meticulously
  detailing their axioms, rules, and distinctive properties.
  \end{quote}
\item
  \begin{quote}
  A novel formal framework for representing and analyzing non-classical
  languages empowering a deeper understanding of their intricate
  constructs.
  \end{quote}
\item
  \begin{quote}
  A refined classification of non-classical logics organizing these
  diverse systems based on their inherent relationships and shared
  characteristics.
  \end{quote}
\item
  \begin{quote}
  A comprehensive graph of sequent calculi visually depicting the
  intricate web of connections between diverse non-classical logics.
  \end{quote}
\item
  \begin{quote}
  A catalog of common patterns and thematic elements that characterize
  the properties of non-classical logics providing insights into their
  unifying principles.
  \end{quote}
\item
  \begin{quote}
  An exhaustive list of new areas of research suggested by the analysis
  of the graph of sequent calculi opening up new frontiers for
  exploration in the realm of non-classical logic.
  \end{quote}
\end{itemize}

§ Goals statements identify the overall purpose of the project and a
general indication of intent.

\begin{enumerate}
\def\labelenumi{\arabic{enumi}.}
\item
  \begin{quote}
  \textbf{Identification of non-classical logics:} A comprehensive list
  of non-classical logics will be compiled, including their axioms,
  rules, and properties.
  \end{quote}
\item
  \begin{quote}
  \textbf{Classification of non-classical logics:} The non-classical
  logics will be classified based on their properties, such as the types
  of negation that they use, the consistency conditions that they
  satisfy, and the types of truth values that they employ.
  \end{quote}
\item
  \begin{quote}
  \textbf{Construction of the graph of sequent calculi:} The graph of
  sequent calculi will be constructed by connecting the different
  non-classical logics based on their relationships.
  \end{quote}
\item
  \begin{quote}
  \textbf{Analysis of the graph of sequent calculi:} The graph of
  sequent calculi will be analyzed to identify common patterns and
  themes, and to suggest new areas of research.
  \end{quote}
\item
  \begin{quote}
  Develop the foundations and fundamentals for a universal theory of
  semantic languages.
  \end{quote}
\item
  \begin{quote}
  Identify the precise metalinguistic conditions that define
  paraconsistent languages
  \end{quote}
\item
  \begin{quote}
  Identify the precise metalinguistic conditions that define
  paracomplete languages.
  \end{quote}
\item
  \begin{quote}
  Identify the precise metalinguistic conditions that define
  constructive languages.
  \end{quote}
\item
  \begin{quote}
  Produce an automated reasoner that utilizes the union of
  paraconsistent constructions and paracomplete constructions in a
  metasystem of constructive proofs and constructive refutation
  refutations.
  \end{quote}
\item
  \begin{quote}
  Schematically relate the subcalculi, supercalculi, hypocalculi, and
  hypercalculi graphs to universal quantum computing.
  \end{quote}
\end{enumerate}

§ Objectives are action statements with measurable outcomes to be
completed by a specified time and under specified conditions.

\hypertarget{approachmethodology}{%
\subsection{\texorpdfstring{\textbf{Approach/Methodology}}{Approach/Methodology}}\label{approachmethodology}}

§ How are you going to carry out your project?

The graph will represent calculi as nodes and their relationships as
edges, with edge types encoding different kinds of connections, such as
equivalence, embedding, and extension.

\textbf{Literature Review:}

\begin{itemize}
\item
  \begin{quote}
  Conduct an extensive review of relevant literature on sequent calculi
  including intuitionistic logic, counter-intuitionistic logic, linear
  logic, the logic of qubits, Ardeshir-Vaezian's sequent calculus U,
  Sambin's Basic Logic, and T-Norm hypersequent calculi.
  \end{quote}
\item
  \begin{quote}
  Criticize the fundamental concepts, principles, and applications of
  these calculi.
  \end{quote}
\item
  \begin{quote}
  Identify and analyze existing research related to the graph of
  supercalculi, conjugated calculi pairs, and subcalculi.
  \end{quote}
\end{itemize}

\textbf{2. Conceptual Framework Development:}

\begin{itemize}
\item
  \begin{quote}
  Formulate a clear and precise conceptual framework for the graph of
  supercalculi, conjugated calculi pairs, and subcalculi.
  \end{quote}
\item
  \begin{quote}
  Define the key components and relationships within the graph
  structure.
  \end{quote}
\item
  \begin{quote}
  Establish a formal representation of the graph using appropriate
  mathematical and graphical notation.
  \end{quote}
\item
  \begin{quote}
  Establish a libre open source programming language capable of
  representing all supercalculi, conjugated calculi pairs, and
  subcalculi and their semantics.
  \end{quote}
\item
  \begin{quote}
  Establish an automated reasoning suite utilizing all sequent calculi
  in a modular way to reason about finite subgraphs of the graph of
  supercalculi, conjugated calculi pairs, and subcalculi.
  \end{quote}
\end{itemize}

\textbf{3. Metamathematical Analysis:}

\begin{itemize}
\item
  \begin{quote}
  Employ metamathematical techniques to investigate the properties and
  structure of the graph of supercalculi, conjugated calculi pairs, and
  subcalculi.
  \end{quote}
\item
  \begin{quote}
  Analyze the graph\textquotesingle s connectivity, paraconsistency, and
  paracompleteness.
  \end{quote}
\item
  \begin{quote}
  Explore the relationships between the graph\textquotesingle s
  structure and the underlying logical systems.
  \end{quote}
\end{itemize}

\textbf{4. Metalinguistic Investigation:}

\begin{itemize}
\item
  \begin{quote}
  Utilize metalinguistic tools to examine the expressive power and
  limitations of the graph of supercalculi, conjugated calculi pairs,
  and subcalculi.
  \end{quote}
\item
  \begin{quote}
  Analyze the graph\textquotesingle s ability to represent and formalize
  various logical concepts and relationships.
  \end{quote}
\item
  \begin{quote}
  Represent the symmetries and dualities of not only the object
  languages but the metalanguages of the various calculi.
  \end{quote}
\item
  \begin{quote}
  Develop metalanguages for all semantic languages.
  \end{quote}
\item
  \begin{quote}
  Evaluate the graph\textquotesingle s effectiveness in capturing the
  nuances of semantic languages.
  \end{quote}
\end{itemize}

\textbf{5. Comparative Analysis:}

\begin{itemize}
\item
  \begin{quote}
  Compare and contrast the graph of supercalculi, conjugated calculi
  pairs, and subcalculi with alternative approaches to representing
  semantic languages.
  \end{quote}
\item
  \begin{quote}
  Identify the strengths and weaknesses of each approach in terms of
  expressiveness, conciseness, and computational efficiency.
  \end{quote}
\item
  \begin{quote}
  Discuss the implications of the graph-based approach for understanding
  and reasoning within and without semantic language.
  \end{quote}
\end{itemize}

§ What specific activities do you propose to meet the goals and
objectives you have outlined, and how will those activities be carried
out?

\textbf{Specific Activities:}

To achieve the outlined goals and objectives, the following specific
activities will be undertaken:

\begin{itemize}
\item
  \begin{quote}
  Gather and organize relevant literature on semantic language and graph
  of supercalculi, conjugated calculi pairs, and subcalculi..
  \end{quote}
\item
  \begin{quote}
  Construct a conceptual diagram or model to represent the graph of
  supercalculi, conjugated calculi pairs, and subcalculi.
  \end{quote}
\item
  \begin{quote}
  Develop formal definitions and theorems related to the
  graph\textquotesingle s structure and properties.
  \end{quote}
\item
  \begin{quote}
  Utilize metamathematical tools, such as proof theory and model theory,
  to analyze the graph\textquotesingle s behavior.
  \end{quote}
\item
  \begin{quote}
  Employ metalinguistic techniques to assess the expressive power and
  limitations of the graph.
  \end{quote}
\item
  \begin{quote}
  Conduct comparative analysis with alternative approaches to
  representing non-classical languages.
  \end{quote}
\item
  \begin{quote}
  Programmatically represent the works in a logical programming paradigm
  programming language in a published code repository.
  \end{quote}
\end{itemize}

\hypertarget{outcomes-benefits-results}{%
\subsection{\texorpdfstring{\textbf{Outcomes, Benefits,
Results}}{Outcomes, Benefits, Results}}\label{outcomes-benefits-results}}

The project will deliver the following outcomes.

\begin{itemize}
\item
  \begin{quote}
  A comprehensive list of non-classical logics, including their axioms,
  rules, and properties.
  \end{quote}
\item
  \begin{quote}
  A new formal framework for representing and reasoning about semantic
  languages.
  \end{quote}
\item
  \begin{quote}
  A new classification of semantic languages.
  \end{quote}
\item
  \begin{quote}
  A graph of sequent calculi that visually represents the relationships
  between these logics.
  \end{quote}
\item
  \begin{quote}
  A list of common patterns and themes in the properties of
  non-classical logics.
  \end{quote}
\item
  \begin{quote}
  A list of new areas of research suggested by the analysis of the graph
  of supercalculi, conjugated calculi pairs, and subcalculi.
  \end{quote}
\end{itemize}

The project will also have the following benefits.

\begin{itemize}
\item
  \begin{quote}
  It will provide a better understanding of the logical relationships
  between different sequent calculi.
  \end{quote}
\item
  \begin{quote}
  It will facilitate the development of new and more powerful sequent
  calculi.
  \end{quote}
\item
  \begin{quote}
  It will enable the application of sequent calculi to new areas.
  \end{quote}
\end{itemize}

The project will also produce the following results.

\begin{itemize}
\item
  \begin{quote}
  A new formal framework for representing and reasoning about semantic
  languages.
  \end{quote}
\item
  \begin{quote}
  A new classification of sublanguages, superlanguages, and
  hyperlanguages as well as subcalculi, supercalculi, and hypercalculi.
  \end{quote}
\item
  \begin{quote}
  A graph of supercalculi, conjugated calculi pairs, and subcalculi that
  visually represents the relationships between these logics.
  \end{quote}
\item
  \begin{quote}
  A list of common patterns and themes in the properties of
  non-classical logics and non-classical languages.
  \end{quote}
\item
  \begin{quote}
  A list of new areas of research suggested by the analysis of the graph
  of supercalculi, conjugated calculi pairs, and subcalculi.
  \end{quote}
\end{itemize}

§ Outcomes - What are the products of your work?

§ Impact - What are the benefits and results of your work?

§ Measurement - Can your outcomes, benefits and results be measured?

§ Products - What does the funding agency get in return for supporting
your proposal?

\hypertarget{project-directorprincipal-investigator-and-staff}{%
\subsection{\texorpdfstring{\textbf{Project Director/Principal
Investigator and
Staff}}{Project Director/Principal Investigator and Staff}}\label{project-directorprincipal-investigator-and-staff}}

§ List the qualifications and experience of the proposed project
director/principal investigator.

§ List the qualifications and experience of key project staff.

\textbf{Primary Objective: Establishing a Unified Understanding of
Non-Classical Logics}

\begin{itemize}
\tightlist
\item
\end{itemize}

\textbf{Gaps that this work can fill:}

\begin{itemize}
\item
  \begin{quote}
  \textbf{There is a lack of efficient and scalable graph-based
  algorithms specifically designed for analyzing the structural
  properties of sequent calculi.}
  \end{quote}
\item
  \begin{quote}
  \textbf{The impact of structural properties on the computational
  behavior of sequent calculi remains incompletely understood.}
  \end{quote}
\item
  \begin{quote}
  \textbf{The applicability of graph-based techniques to the analysis of
  sequent calculus extensions and variations has not been fully
  explored.}
  \end{quote}
\end{itemize}

\textbf{Gaps that this work can fill:}

\begin{itemize}
\item
  \begin{quote}
  \textbf{The current understanding of the structural properties of
  sequent calculi is limited and lacks a comprehensive and systematic
  approach to analysis.}
  \end{quote}
\item
  \begin{quote}
  \textbf{There is a need for rigorous and expressive graph-based
  techniques to effectively capture the intricate structure and
  relationships within sequent calculi.}
  \end{quote}
\item
  \begin{quote}
  \textbf{The identification of common structural patterns and their
  relationship to logical properties remains incompletely understood.}
  \end{quote}
\item
  \begin{quote}
  \textbf{The development of formal theorems that connect structural
  features to computational and logical properties can provide a deeper
  understanding of the nature of sequent calculi.}
  \end{quote}
\end{itemize}

\textbf{Gaps that this work can fill:}

\begin{itemize}
\item
  \begin{quote}
  \textbf{The current understanding of the relationship between sequent
  calculi and conjugated logic pairs is limited and lacks a
  comprehensive formal framework for analysis.\\
  }
  \end{quote}
\item
  \begin{quote}
  \textbf{There is a need for a rigorous and expressive formal framework
  that can effectively capture the connections between the graphical
  representations of these systems.\\
  }
  \end{quote}
\item
  \begin{quote}
  \textbf{The precise relationship between the structural properties of
  sequent calculi and the logical features of conjugated logic pairs
  remains unclear.\\
  }
  \end{quote}
\item
  \begin{quote}
  \textbf{The implications of the relationship between sequent calculi
  and conjugated logic pairs on proof theory and automated reasoning
  have not been fully explored.}
  \end{quote}
\end{itemize}

\textbf{Gaps that this work can fill:}

\begin{itemize}
\item
  \begin{quote}
  \textbf{The current understanding of formal frameworks for
  representing supercalculi is limited and lacks a unified and
  expressive approach.}
  \end{quote}
\item
  \begin{quote}
  \textbf{There is a need for a rigorous and flexible formal framework
  that can effectively capture the intricacies of supercalculi,
  including their multimodal and non-classical nature.}
  \end{quote}
\item
  \begin{quote}
  \textbf{The structural and semantic relationships between supercalculi
  and their underlying logical systems remain incompletely understood.}
  \end{quote}
\item
  \begin{quote}
  \textbf{The development of a graph-based formal framework can
  facilitate the analysis and comparison of different supercalculi and
  their applications.}
  \end{quote}
\end{itemize}

\textbf{Gaps that this work can fill:}

\begin{itemize}
\item
  \begin{quote}
  The current understanding of the relationship between sequent calculi
  and substructural logics is limited and lacks a comprehensive formal
  framework for analysis.
  \end{quote}
\item
  \begin{quote}
  There is a need for a rigorous and expressive formal framework that
  can effectively capture the connections between the graphical
  representations of these systems.
  \end{quote}
\item
  \begin{quote}
  The precise relationship between the structural properties of sequent
  calculi and the logical features of substructural logics remains
  unclear.
  \end{quote}
\item
  \begin{quote}
  The implications of the relationship between sequent calculi and
  substructural logics on proof theory and automated reasoning have not
  been fully explored.
  \end{quote}
\end{itemize}

\textbf{Gaps that this work can fill:}

\begin{itemize}
\item
  \begin{quote}
  The current understanding of formal frameworks for representing
  sequent calculi is fragmented and lacks a unified approach.
  \end{quote}
\item
  \begin{quote}
  There is a need for a rigorous and expressive formal framework that
  can effectively capture the intricacies of sequent calculi.
  \end{quote}
\item
  \begin{quote}
  The structural properties of sequent calculi remain largely
  unexplored, and their relationships to logical properties are not
  fully understood.
  \end{quote}
\item
  \begin{quote}
  The development of new non-classical logics is often ad hoc and lacks
  a systematic approach based on formal frameworks.
  \end{quote}
\end{itemize}

\textbf{Context of the project:}

\begin{itemize}
\item
  \begin{quote}
  \textbf{Sequent calculi play a crucial role in formal logic, providing
  a powerful tool for representing and reasoning about logical systems.}
  \end{quote}
\item
  \begin{quote}
  \textbf{A comprehensive understanding of the structural properties of
  sequent calculi can provide insights into their computational
  efficiency, expressiveness, and limitations.}
  \end{quote}
\item
  \begin{quote}
  \textbf{The development of efficient and scalable graph-based
  algorithms can facilitate the systematic analysis of sequent calculi,
  leading to the discovery of new properties and optimization
  strategies.}
  \end{quote}
\item
  \begin{quote}
  \textbf{The exploration of the impact of structural properties and the
  application of graph-based techniques to extensions and variations can
  contribute to advancements in proof theory, automated reasoning, and
  the design of new logical systems.}
  \end{quote}
\end{itemize}

\textbf{Context of the project:}

\begin{itemize}
\item
  \begin{quote}
  \textbf{Sequent calculi have emerged as powerful tools for formalizing
  and reasoning about a wide range of logical systems, with applications
  in proof theory, automated reasoning, and artificial intelligence.}
  \end{quote}
\item
  \begin{quote}
  \textbf{Establishing a comprehensive understanding of the structural
  properties of sequent calculi can provide insights into their
  computational behavior, expressiveness, and limitations.}
  \end{quote}
\item
  \begin{quote}
  \textbf{The development of graph-based techniques for analyzing
  sequent calculi can facilitate the identification of common patterns,
  the discovery of new properties, and the development of optimization
  strategies for automated reasoning systems.}
  \end{quote}
\item
  \begin{quote}
  \textbf{The establishment of formal connections between structural
  features and logical properties can contribute to advancements in
  proof theory, automated reasoning, and the design of new logical
  systems.}
  \end{quote}
\end{itemize}

\textbf{Context of the project:}

\begin{itemize}
\item
  \begin{quote}
  \textbf{Sequent calculi and conjugated logic pairs are powerful tools
  for formalizing and reasoning about logical systems, with applications
  in diverse areas such as artificial intelligence, linguistics, and
  computer science.\\
  }
  \end{quote}
\item
  \begin{quote}
  \textbf{Establishing a clear connection between sequent calculi and
  conjugated logic pairs can provide a unified framework for
  understanding and analyzing these systems.\\
  }
  \end{quote}
\item
  \begin{quote}
  \textbf{The development of formal mappings between graphical
  representations can facilitate the translation of proofs and theorems,
  enhancing interoperability and cross-fertilization between the two
  domains.\\
  }
  \end{quote}
\item
  \begin{quote}
  \textbf{The investigation of the implications of these connections can
  lead to new insights into the nature of conjugated logic pairs and
  their applications in formal logic, artificial intelligence, and other
  areas.}
  \end{quote}
\end{itemize}

\textbf{Context of the project:}

\begin{itemize}
\item
  \begin{quote}
  \textbf{Supercalculi have emerged as powerful tools for formalizing
  and reasoning about multimodal and non-classical logics, with
  applications in diverse areas such as artificial intelligence,
  linguistics, and computer science.}
  \end{quote}
\item
  \begin{quote}
  \textbf{Establishing a comprehensive formal framework for representing
  supercalculi using graphs can provide a unified foundation for
  understanding, analyzing, and comparing these systems.}
  \end{quote}
\item
  \begin{quote}
  \textbf{The development of formal mappings between graphical
  representations of supercalculi can facilitate the translation of
  proofs and theorems, enhancing interoperability and
  cross-fertilization between different supercalculi.}
  \end{quote}
\item
  \begin{quote}
  \textbf{The investigation of the implications of these connections can
  lead to new insights into the nature of supercalculi and their
  applications in formal logic, artificial intelligence, and other
  areas.}
  \end{quote}
\end{itemize}

\textbf{Context of the project:}

\begin{itemize}
\item
  \begin{quote}
  Sequent calculi and substructural logics are powerful tools for
  formalizing and reasoning about logical systems with non-classical
  features.
  \end{quote}
\item
  \begin{quote}
  Establishing a clear connection between sequent calculi and
  substructural logics can provide a unified framework for understanding
  and analyzing these systems.
  \end{quote}
\item
  \begin{quote}
  The development of formal mappings between graphical representations
  can facilitate the translation of proofs and theorems, enhancing
  interoperability and cross-fertilization between the two domains.
  \end{quote}
\item
  \begin{quote}
  The investigation of the implications of these connections can lead to
  new insights into the nature of substructural logics and their
  applications in proof theory, automated reasoning, and other areas.
  \end{quote}
\end{itemize}

\textbf{Context of the project:}

\begin{itemize}
\item
  \begin{quote}
  The research on sequent calculi has gained significant traction in
  recent years, driven by their applications in automated reasoning,
  proof theory, and artificial intelligence.
  \end{quote}
\item
  \begin{quote}
  The development of a formal framework for representing sequent calculi
  using graphs can provide a solid foundation for further advancements
  in this area.
  \end{quote}
\item
  \begin{quote}
  The investigation of the structural properties of sequent calculi can
  lead to a deeper understanding of their logical behavior and potential
  applications.
  \end{quote}
\item
  \begin{quote}
  The design of new non-classical logics using the formal framework can
  expand the repertoire of logical systems available for modeling and
  reasoning about complex phenomena.
  \end{quote}
\end{itemize}

\end{document}}*}


\chapter{Research Proposal Subclassical Graph of Calculi}
}

\hypertarget{manuscript-of-research-proposal-subclassical-graph-of-calculi}{%
\section{Manuscript of Research Proposal: Subclassical Graph of
Calculi}\label{manuscript-of-research-proposal-subclassical-graph-of-calculi}}

\hypertarget{title---the-title-should-be-as-descriptive-as-possible.}{%
\subsection{Title - The title should be as descriptive as
possible.}\label{title---the-title-should-be-as-descriptive-as-possible.}}

\hypertarget{introduction--}{%
\subsection{\texorpdfstring{ Introduction -
}{ Introduction - }}\label{introduction--}}

This section may include:

§ What is to be done and the context of the project.

§ What is being done both generally and specifically in the same or
related areas. (The reviewer should know that you know what is going on
in the area in which you are proposing.)

§ An explanation and justification for unique or innovative approaches.
(These are selling points about what makes your project special, unique
and compelling and why it should be funded.)

\hypertarget{need-statement}{%
\subsection{Need Statement}\label{need-statement}}

§ What needs to be done and why?

§ What significant needs are you trying to meet? Compared to other
projects in the same area, what sets yours apart in terms of need?

§ What services are to be delivered? Why? Use specifics from preliminary
studies, needs assessment, documentation, and data supporting your
proposal.

§ What gaps that your work can fill exist in the knowledge base of your
field?

§ Is the problem both significant and manageable? Do you have the
resources to handle the problem?

\hypertarget{goals-and-objectives}{%
\subsection{Goals and Objectives}\label{goals-and-objectives}}

§ Goals statements identify the overall purpose of the project and a
general indication of intent.

\begin{enumerate}
\def\labelenumi{\arabic{enumi}.}
\item
  \begin{quote}
  \textbf{Identification of non-classical logics:} A comprehensive list
  of non-classical logics will be compiled, including their axioms,
  rules, and properties.
  \end{quote}
\item
  \begin{quote}
  \textbf{Classification of non-classical logics:} The non-classical
  logics will be classified based on their properties, such as the types
  of negation that they use, the consistency conditions that they
  satisfy, and the types of truth values that they employ.
  \end{quote}
\item
  \begin{quote}
  \textbf{Construction of the subclassical graph of calculi:} The
  subclassical graph of calculi will be constructed by connecting the
  different non-classical logics based on their relationships.
  \end{quote}
\item
  \begin{quote}
  \textbf{Analysis of the subclassical graph of calculi:} The
  subclassical graph of calculi will be analyzed to identify common
  patterns and themes, and to suggest new areas of research.
  \end{quote}
\end{enumerate}

Develop towards a universal theory of semantic languages.

Identify the precise metalinguistic conditions that define
paraconsistent languages and the precise metalinguistic conditions that
define paracomplete languages.

Exactly define constructive languages.

Produce an automated reasoner that utilizes the union of paraconsistent
constructions and paracomplete constructions in a metasystem of proof
and refutation.

Schematically relate the subcalculi, supercalculi, and hypercalculi
graphs to universal quantum computing.

§ Objectives are action statements with measurable outcomes, to be
completed by a specified time and under specified conditions.

\hypertarget{approachmethodology}{%
\subsection{Approach/Methodology}\label{approachmethodology}}

§ How are you going to carry out your project?

The graph will represent calculi as nodes and their relationships as
edges, with edge types encoding different kinds of connections, such as
equivalence, embedding, and extension.

\textbf{Literature Review:}

\begin{itemize}
\item
  \begin{quote}
  Conduct an extensive review of relevant literature on subclassical
  calculi including intuitionistic logic, counter-intuitionistic logic,
  linear logic, the logic of qubits, Ardeshir-Vaezian's sequent calculus
  U, Sambin's Basic Logic, and T-Norm hypersequent calculi.
  \end{quote}
\item
  \begin{quote}
  Criticize the fundamental concepts, principles, and applications of
  these calculi.
  \end{quote}
\item
  \begin{quote}
  Identify and analyze existing research related to the Subclassical
  Graph of Calculi.
  \end{quote}
\end{itemize}

\textbf{2. Conceptual Framework Development:}

\begin{itemize}
\item
  \begin{quote}
  Formulate a clear and precise conceptual framework for the
  Subclassical Graph of Calculi.
  \end{quote}
\item
  \begin{quote}
  Define the key components and relationships within the graph
  structure.
  \end{quote}
\item
  \begin{quote}
  Establish a formal representation of the graph using appropriate
  mathematical and graphical notation.
  \end{quote}
\item
  \begin{quote}
  Establish a libre open source programming language capable of
  representing all subclassical calculi and their semantics.
  \end{quote}
\item
  \begin{quote}
  Establish an automated reasoning suite utilizing subclassical calculi
  in a modular way to reason about finite subgraphs of the subclassical
  graph of calculi.
  \end{quote}
\end{itemize}

\textbf{3. Metamathematical Analysis:}

\begin{itemize}
\item
  \begin{quote}
  Employ metamathematical techniques to investigate the properties and
  structure of the Subclassical Graph of Calculi.
  \end{quote}
\item
  \begin{quote}
  Analyze the graph\textquotesingle s connectivity, paraconsistency, and
  paracompleteness.
  \end{quote}
\item
  \begin{quote}
  Explore the relationships between the graph\textquotesingle s
  structure and the underlying logical systems.
  \end{quote}
\end{itemize}

\textbf{4. Metalinguistic Investigation:}

\begin{itemize}
\item
  \begin{quote}
  Utilize metalinguistic tools to examine the expressive power and
  limitations of the Subclassical Graph of Calculi.
  \end{quote}
\item
  \begin{quote}
  Analyze the graph\textquotesingle s ability to represent and formalize
  various logical concepts and relationships.
  \end{quote}
\item
  \begin{quote}
  Represent the symmetries and dualities of not only the object
  languages but the metalanguages of the various calculi.
  \end{quote}
\item
  \begin{quote}
  Develop subclassical metalanguages.
  \end{quote}
\item
  \begin{quote}
  Evaluate the graph\textquotesingle s effectiveness in capturing the
  nuances of subclassical logic.
  \end{quote}
\end{itemize}

\textbf{5. Comparative Analysis:}

\begin{itemize}
\item
  \begin{quote}
  Compare and contrast the Subclassical Graph of Calculi with
  alternative approaches to representing subclassical logic, such as
  sequent calculus and natural deduction.
  \end{quote}
\item
  \begin{quote}
  Identify the strengths and weaknesses of each approach in terms of
  expressiveness, conciseness, and computational efficiency.
  \end{quote}
\item
  \begin{quote}
  Discuss the implications of the graph-based approach for understanding
  and reasoning within and without subclassical logic.
  \end{quote}
\end{itemize}

§ What specific activities do you propose to meet the goals and
objectives you have outlined, and how will those activities be carried
out?

\textbf{Specific Activities:}

To achieve the outlined goals and objectives, the following specific
activities will be undertaken:

\begin{itemize}
\item
  \begin{quote}
  Gather and organize relevant literature on subclassical calculi and
  the Subclassical Graph of Calculi.
  \end{quote}
\item
  \begin{quote}
  Construct a conceptual diagram or model to represent the Subclassical
  Graph of Calculi.
  \end{quote}
\item
  \begin{quote}
  Develop formal definitions and theorems related to the
  graph\textquotesingle s structure and properties.
  \end{quote}
\item
  \begin{quote}
  Utilize metamathematical tools, such as proof theory and model theory,
  to analyze the graph\textquotesingle s behavior.
  \end{quote}
\item
  \begin{quote}
  Employ metalinguistic techniques to assess the expressive power and
  limitations of the graph.
  \end{quote}
\item
  \begin{quote}
  Conduct comparative analysis with alternative approaches to
  representing subclassical logic.
  \end{quote}
\item
  \begin{quote}
  Programmatically represent the works in a logical programming paradigm
  programming language in a published code repository.
  \end{quote}
\end{itemize}

\hypertarget{outcomes-benefits-results}{%
\subsection{Outcomes, Benefits,
Results}\label{outcomes-benefits-results}}

The project will deliver the following outcomes.

\begin{itemize}
\item
  \begin{quote}
  A comprehensive list of non-classical logics, including their axioms,
  rules, and properties.
  \end{quote}
\item
  \begin{quote}
  A new formal framework for representing and reasoning about
  subclassical logics.
  \end{quote}
\item
  \begin{quote}
  A new classification of subclassical logics.
  \end{quote}
\item
  \begin{quote}
  A subclassical graph of calculi that visually represents the
  relationships between these logics.
  \end{quote}
\item
  \begin{quote}
  A list of common patterns and themes in the properties of
  non-classical logics.
  \end{quote}
\item
  \begin{quote}
  A list of new areas of research suggested by the analysis of the
  subclassical graph of calculi.
  \end{quote}
\end{itemize}

The project will also have the following benefits.

\begin{itemize}
\item
  \begin{quote}
  It will provide a better understanding of the logical relationships
  between different subclassical calculi.
  \end{quote}
\item
  \begin{quote}
  It will facilitate the development of new and more powerful
  subclassical calculi.
  \end{quote}
\item
  \begin{quote}
  It will enable the application of subclassical calculi to new areas.
  \end{quote}
\end{itemize}

The project will also produce the following results.

\begin{itemize}
\item
  \begin{quote}
  A new formal framework for representing and reasoning about
  subclassical logics.
  \end{quote}
\item
  \begin{quote}
  A new classification of sublanguages, superlanguages, and
  hyperlanguages as well as subcalculi, supercalculi, and hypercalculi.
  \end{quote}
\item
  \begin{quote}
  A subclassical graph of calculi that visually represents the
  relationships between these logics.
  \end{quote}
\item
  \begin{quote}
  A list of common patterns and themes in the properties of
  non-classical logics.
  \end{quote}
\item
  \begin{quote}
  A list of new areas of research suggested by the analysis of the
  subclassical graph of calculi.
  \end{quote}
\end{itemize}

§ Outcomes - What are the products of your work?

§ Impact - What are the benefits and results of your work?

§ Measurement - Can your outcomes, benefits and results be measured?

§ Products - What does the funding agency get in return for supporting
your proposal?

\hypertarget{project-directorprincipal-investigator-and-staff}{%
\subsection{Project Director/Principal Investigator and
Staff}\label{project-directorprincipal-investigator-and-staff}}

§ List the qualifications and experience of the proposed project
director/principal investigator.

§ List the qualifications and experience of key project staff.

\hypertarget{manuscript-of-research-proposal-superclassical-graph-of-calculi}{%
\section{Manuscript of Research Proposal: Superclassical Graph of
Calculi}\label{manuscript-of-research-proposal-superclassical-graph-of-calculi}}

Also known as Post's lattice.

\hypertarget{manuscript-of-research-proposal-graph-of-supercalculi-and-subcalculi}{%
\section{Manuscript of Research Proposal: Graph of Supercalculi and
Subcalculi}\label{manuscript-of-research-proposal-graph-of-supercalculi-and-subcalculi}}

\hypertarget{manuscript-of-research-proposal-hyperclassical-graph-of-calculi}{%
\section{Manuscript of Research Proposal: Hyperclassical Graph of
Calculi}\label{manuscript-of-research-proposal-hyperclassical-graph-of-calculi}}

\hypertarget{writing-research-papers}{%
\section{Writing Research Papers}\label{writing-research-papers}}

\hypertarget{the-concept-paper}{%
\subsection{The Concept Paper}\label{the-concept-paper}}

``I have a great idea for a grant proposal! How do I get started?'' A
concept paper is the suggested starting point for the development of any
successful proposal. Generally three to five pages long, it outlines the
project with enough detail to clearly demonstrate what is being
proposed. Certain funding agencies require that a concept paper be
submitted prior to submission of a full proposal. Therefore, you need to
build a strong case for your project in a concise and persuasive manner.
Suggested guidelines for constructing a concept paper are as follows:

\hypertarget{title---the-title-should-be-as-descriptive-as-possible.-1}{%
\subsection{Title - The title should be as descriptive as
possible.}\label{title---the-title-should-be-as-descriptive-as-possible.-1}}

\hypertarget{introduction---1}{%
\subsection{\texorpdfstring{ Introduction -
}{ Introduction - }}\label{introduction---1}}

This section may include:

§ What is to be done and the context of the project.

§ What is being done both generally and specifically in the same or
related areas. (The reviewer should know that you know what is going on
in the area in which you are proposing.)

§ An explanation and justification for unique or innovative approaches.
(These are selling points about what makes your project special, unique
and compelling and why it should be funded.)

\hypertarget{need-statement-1}{%
\subsection{Need Statement}\label{need-statement-1}}

§ What needs to be done and why?

§ What significant needs are you trying to meet? Compared to other
projects in the same area, what sets yours apart in terms of need?

§ What services are to be delivered? Why? Use specifics from preliminary
studies, needs assessment, documentation, and data supporting your
proposal.

§ What gaps that your work can fill exist in the knowledge base of your
field?

§ Is the problem both significant and manageable? Do you have the
resources to handle the problem?

\hypertarget{goals-and-objectives-1}{%
\subsection{Goals and Objectives}\label{goals-and-objectives-1}}

§ Goals statements identify the overall purpose of the project and a
general indication of intent.

§ Objectives are action statements with measurable outcomes, to be
completed by a specified time and under specified conditions.

\hypertarget{approachmethodology-1}{%
\subsection{Approach/Methodology}\label{approachmethodology-1}}

§ How are you going to carry out your project?

§ What specific activities do you propose to meet the goals and
objectives you have outlined, and how will those activities be carried
out?

\hypertarget{outcomes-benefits-results-1}{%
\subsection{Outcomes, Benefits,
Results}\label{outcomes-benefits-results-1}}

§ Outcomes - What are the products of your work?

§ Impact - What are the benefits and results of your work?

§ Measurement - Can your outcomes, benefits and results be measured?

§ Products - What does the funding agency get in return for supporting
your proposal?

\hypertarget{project-directorprincipal-investigator-and-staff-1}{%
\subsection{Project Director/Principal Investigator and
Staff}\label{project-directorprincipal-investigator-and-staff-1}}

§ List the qualifications and experience of the proposed project
director/principal investigator.

§ List the qualifications and experience of key project staff.

\hypertarget{organizations}{%
\section{Organizations}\label{organizations}}

\hypertarget{philosophy-organizations}{%
\subsection{Philosophy Organizations}\label{philosophy-organizations}}

Philosophy of Science

Philosophy of Physics

Philosophy of Mind, Consciousness, and Free Will

Philosophy of Life

Philosophy of Simulation

Metaphysics

Metalinguistics

Metamathematics

\hypertarget{philosophy-of-science-association}{%
\subsubsection{\texorpdfstring{Philosophy of Science Association
}{Philosophy of Science Association }}\label{philosophy-of-science-association}}

\href{mailto:office@philsci.org}{\hl{\ul{office@philsci.org}}}

\hypertarget{iuhpst-international-union-of-history-and-philosophy-of-science-and-technology}{%
\subsubsection{IUHPST: INTERNATIONAL UNION OF HISTORY AND PHILOSOPHY OF
SCIENCE AND
TECHNOLOGY}\label{iuhpst-international-union-of-history-and-philosophy-of-science-and-technology}}

\hypertarget{dlmpst-division-of-logic-methodology-and-philosophy-of-science-and-technology}{%
\paragraph{DLMPST: DIVISION OF LOGIC, METHODOLOGY AND PHILOSOPHY OF
SCIENCE AND
TECHNOLOGY}\label{dlmpst-division-of-logic-methodology-and-philosophy-of-science-and-technology}}

Hosts the Congress for Logic, Philosophy and Methodology of Science and
Technology (CLMPST) every four years.

\hypertarget{international-federation-of-philosophical-societies}{%
\subsubsection{International Federation of Philosophical
Societies}\label{international-federation-of-philosophical-societies}}

World Congress of Philosophy every five years

\hypertarget{mathematical-organizations}{%
\subsection{Mathematical
Organizations}\label{mathematical-organizations}}

Non-Classical Mathematics

Metamathematics

Mathematical Logic

Constructive Mathematics

Mathematical Physics

\hypertarget{logical-organizations}{%
\subsection{Logical Organizations}\label{logical-organizations}}

Non-Classical Logical Languages

Metalogic

Constructive Logic

\hypertarget{logica-universalis-association-lua}{%
\subsubsection{Logica Universalis Association
(LUA)}\label{logica-universalis-association-lua}}

\href{https://sites.google.com/view/calcuttalogiccircleclc/home}{\ul{Calcutta
Logic Circle}} (LUA Member)

\hypertarget{computational-organizations}{%
\subsection{Computational
Organizations}\label{computational-organizations}}

Philosophy of Artificial Intelligence, consciousness, life, synthesis,
and simulation

Theory of Computing

Theory of Quantum Computing

Non-Classical Computing

\hypertarget{association-for-computing-machinery}{%
\subsubsection{Association for Computing
Machinery}\label{association-for-computing-machinery}}

\hypertarget{ieee}{%
\subsubsection{IEEE}\label{ieee}}

\hypertarget{computer-science}{%
\paragraph{Computer Science}\label{computer-science}}

\hypertarget{mathematical-foundations-of-computing-tcmf}{%
\subparagraph{\texorpdfstring{\href{https://www.computer.org/communities/technical-committees/tcmf}{\ul{Mathematical
Foundations of Computing
(TCMF)}}}{Mathematical Foundations of Computing (TCMF)}}\label{mathematical-foundations-of-computing-tcmf}}

\hypertarget{multiple-valued-logic-tcmvl}{%
\subparagraph{\texorpdfstring{\href{https://www.computer.org/communities/technical-committees/tcmvl}{\ul{Multiple-Valued
Logic
(TCMVL)}}}{Multiple-Valued Logic (TCMVL)}}\label{multiple-valued-logic-tcmvl}}

\hypertarget{physics-organizations}{%
\subsection{Physics Organizations}\label{physics-organizations}}

Methods of physics

Methods of theoretical physics

Methods of experimental physics

Mathematical physics

Metaphysics and metalinguistics of physical theory

Measurement

Constructive Physics, Constructive Realism

Quantum Physics

Quantum Computing

Blackhole Thermodynamics

Biophysics

Physics of Mind, Consciousness, and Life

\hypertarget{international-association-of-mathematical-physics}{%
\subsubsection{International Association of Mathematical
Physics}\label{international-association-of-mathematical-physics}}

Hosts the International Congress on Mathematical Physics (ICMP) every
three years.

\hypertarget{international-union-of-pure-and-applied-physics-iupap}{%
\subsubsection{International Union of Pure and Applied Physics
(IUPAP)}\label{international-union-of-pure-and-applied-physics-iupap}}

Hosts STATPHYS every three years on a different continent.

\hypertarget{linguistics-organizations}{%
\subsection{Linguistics Organizations}\label{linguistics-organizations}}

Formal Languages

Semantic Languages

Metalinguistics

Chomsky Hierarchy

Tarskian Languages

Kripkean Languages

Legal Languages

\hypertarget{permanent-international-committee-of-linguists-picl-comituxe9-international-permanent-des-linguistes-cipl}{%
\subsubsection{\texorpdfstring{Permanent International Committee of
Linguists (PICL) / Comité International Permanent des Linguistes
(\href{http://www.ciplnet.com}{\ul{CIPL}})}{Permanent International Committee of Linguists (PICL) / Comité International Permanent des Linguistes (CIPL)}}\label{permanent-international-committee-of-linguists-picl-comituxe9-international-permanent-des-linguistes-cipl}}

Hosts the International Congress of Linguists every 5 years.

\hypertarget{west-coast-conference-on-formal-linguistics}{%
\subsubsection{West Coast Conference on Formal
Linguistics}\label{west-coast-conference-on-formal-linguistics}}

Held by Center for the Study of Language and Information, home of the
Stanford Encyclopedia of Philosophy

\hypertarget{european-language-resources-association}{%
\subsubsection{European Language Resources
Association}\label{european-language-resources-association}}

Hosts the International Conference on Language Resources and Evaluation
every even year. Focuses on natural language processing.

\hypertarget{association-for-computational-linguistics}{%
\subsubsection{Association for Computational
Linguistics}\label{association-for-computational-linguistics}}

Hosts the International Conference on Computational Linguistics and
Intelligent Text Processing every year.

\hypertarget{california-state-university-fresno}{%
\subsubsection{California State University,
Fresno}\label{california-state-university-fresno}}

Hosts Western Conference on Linguistics every year. Focuses on
theoretical and descriptive linguistics.

\hypertarget{semantic-organizations}{%
\subsection{Semantic Organizations}\label{semantic-organizations}}

\hypertarget{individuals}{%
\section{Individuals}\label{individuals}}

Paola Zizzi

\textbf{Giovanni Sambin\\
} Professor of Mathematical Logic

Dipartimento di Matematica Pura e Applicata,\\
Università di Padova\\
Via Trieste, 63 - IV piano\\
35121 Padova (Italy)

ph. +39-049-8271487, fax. +39-049-8271499

sambin@math.unipd.it

\hypertarget{peers-from-the-5th-world-congress-on-paraconsistency}{%
\subsubsection{Peers from the 5th World Congress on
Paraconsistency}\label{peers-from-the-5th-world-congress-on-paraconsistency}}

Mihir Chakraborty

\ul{calcuttalogiccircle@gmail.com}

\begin{itemize}
\item
  \begin{quote}
  \ul{mihirc4@gmail.com}
  \end{quote}
\end{itemize}

Sankha Basu \href{mailto:sankha@iiitd.ac.in}{\ul{sankha@iiitd.ac.in}} -
Negation-Free Paraconsistency Coauthor: Sayantan Roy

Hidenori Kurokawa
\href{mailto:hkurokawa@gc.cuny.edu}{\ul{hkurokawa@gc.cuny.edu}} -
Hypersequent expert

James T Martin - co-author and collaborator

\hypertarget{draft}{%
\section{Draft}\label{draft}}

\hypertarget{introduction-v1}{%
\subsection{Introduction v1}\label{introduction-v1}}

This research proposal outlines a project to construct a subclassical
graph of calculi, which is a diagram that visually represents the
relationships between different non-classical logics. Non-classical
logics are a family of logics that generalize classical logic by
relaxing some of its axioms or rules. This can lead to a variety of
interesting and counterintuitive properties, such as the ability to
represent vagueness, uncertainty, and contradictions.

The subclassical graph of calculi will be constructed by identifying and
classifying the different types of non-classical logics that exist. This
will be done by considering the different axioms and rules that are used
in these logics, as well as the properties that they satisfy. The graph
will then be used to visualize the relationships between these logics,
and to identify common patterns and themes.

This project is motivated by the need for a better understanding of
non-classical logics and their applications. Non-classical logics have a
wide range of potential applications, including in artificial
intelligence, computer science, and mathematics. However, the current
state of knowledge about non-classical logics is fragmented and
difficult to navigate. The subclassical graph of calculi will provide a
much-needed overview of the field, and will help to identify new areas
of research.

The project will be carried out in the following stages:

\begin{enumerate}
\def\labelenumi{\arabic{enumi}.}
\item
  \begin{quote}
  \textbf{Identification of non-classical logics:} A comprehensive list
  of non-classical logics will be compiled, including their axioms,
  rules, and properties.
  \end{quote}
\item
  \begin{quote}
  \textbf{Classification of non-classical logics:} The non-classical
  logics will be classified based on their properties, such as the types
  of negation that they use, the consistency conditions that they
  satisfy, and the types of truth values that they employ.
  \end{quote}
\item
  \begin{quote}
  \textbf{Construction of the subclassical graph of calculi:} The
  subclassical graph of calculi will be constructed by connecting the
  different non-classical logics based on their relationships.
  \end{quote}
\item
  \begin{quote}
  \textbf{Analysis of the subclassical graph of calculi:} The
  subclassical graph of calculi will be analyzed to identify common
  patterns and themes, and to suggest new areas of research.
  \end{quote}
\end{enumerate}

\hypertarget{introduction-v2}{%
\subsection{Introduction v2}\label{introduction-v2}}

This research proposal outlines the development of a subclassical graph
of calculi, a novel approach to formalizing and analyzing the
relationships between different logical calculi. The proposed
subclassical graph will provide a comprehensive and systematic framework
for understanding the interplay of various logical systems, enabling
researchers to identify connections, uncover hidden relationships,
generate novel calculi, and explore new avenues of research.

\textbf{Context and Background}

Logical calculi are formal systems for reasoning and deriving
conclusions from a set of axioms or assumptions or rules such as
structural or inferential rules. They play a fundamental role in
mathematics, computer science, and philosophy, providing a rigorous
foundation for reasoning, proof construction, refutation construction,
modeling, and countermodeling. Over the centuries, a vast array of
logical calculi have been developed, each tailored to specific
applications and embodying distinct logical properties.

\textbf{Current State of Research}

The study of logical calculi has witnessed significant advancements in
recent decades, fueled by the development of powerful formal methods and
automated reasoning techniques. However, the vast and diverse landscape
of logical systems poses challenges in identifying connections and
overarching principles. Existing approaches to classifying and comparing
logical calculi often rely on ad hoc criteria, leading to a fragmented
and incomplete understanding of the relationships between different
systems.

\textbf{Unique and Innovative Approach}

The proposed subclassical graph of calculi addresses these limitations
by providing a unified and rigorous framework for analyzing the
relationships between logical calculi. The subclassical graph will
represent logical calculi as nodes and the relationships between them as
edges, where the edges capture the logical implications and syntactic
similarities between different systems.

\textbf{Significance and Impact}

The development of the subclassical graph of calculi will have
significant implications for research in logic and its applications. It
will provide a comprehensive and systematic framework for understanding
the interplay of various logical systems, enabling researchers to:

\begin{itemize}
\item
  \begin{quote}
  Identify connections between seemingly disparate calculi
  \end{quote}
\item
  \begin{quote}
  Uncover hidden relationships and patterns in the landscape of logical
  systems
  \end{quote}
\item
  \begin{quote}
  Explore new avenues of research by traversing the subclassical graph
  \end{quote}
\item
  \begin{quote}
  Develop novel logical calculi with tailored properties for specific
  applications
  \end{quote}
\end{itemize}

The subclassical graph of calculi has the potential to revolutionize our
understanding of logical systems and their applications, leading to
advancements in mathematics, computer science, and philosophy.

\hypertarget{introduction-v3}{%
\subsection{Introduction v3}\label{introduction-v3}}

\textbf{Introduction}

The present research proposal delves into the conceptualization and
development of a subclassical graph of calculi, an innovative framework
for analyzing and comparing various logical systems. This project aims
to establish a comprehensive and unified representation of subclassical
logics, encompassing both well-established and emerging systems. The
proposed subclassical graph of calculi will serve as a valuable tool for
researchers working in the fields of logic, mathematics, and computer
science.

The study of subclassical logics has gained significant traction in
recent years, driven by their potential applications in artificial
intelligence, knowledge representation, and reasoning under uncertainty.
However, the lack of a unified and systematic representation of
subclassical logics has hindered progress in this area. The proposed
subclassical graph of calculi addresses this challenge by providing a
structured and interconnected representation of various subclassical
systems.

The subclassical graph of calculi will be constructed by identifying the
key structural elements of subclassical logics, such as their axiomatic
systems, proof rules, and semantic interpretations. These elements will
then be represented as nodes in a graph, with connections between nodes
representing relationships between the corresponding logics. This
graphical representation will facilitate the analysis of logical
relationships, the identification of logical connections, and the
exploration of new logical systems.

The subclassical graph of calculi will offer several unique advantages
over existing approaches to subclassical logic analysis. Firstly, it
will provide a comprehensive and unified representation of subclassical
logics, encompassing both well-established and emerging systems.
Secondly, the graphical representation will facilitate the visualization
of logical relationships and the identification of logical connections.
Thirdly, the subclassical graph of calculi will serve as a generative
framework for exploring new logical systems and their properties.

In summary, the proposed subclassical graph of calculi represents an
innovative and transformative approach to the analysis and comparison of
subclassical logics. This research project has the potential to
significantly advance the field of subclassical logic and contribute to
the development of new artificial intelligence applications.

\hypertarget{introduction-v4}{%
\subsection{Introduction v4}\label{introduction-v4}}

\textbf{Introduction}

This research proposal outlines a novel approach to constructing a
subclassical graph of calculi, a theoretical framework for understanding
the relationships between different formal systems of logic and
computation. The proposed approach leverages recent advances in graph
theory and computational logic to provide a more comprehensive and
rigorous representation of the intricate connections between various
calculi.

The study of subclassical logics and their corresponding calculi has
gained significant traction in recent years, driven by the increasing
complexity and expressive power of formal systems in various domains,
including computer science, artificial intelligence, and linguistics.
Subclassical logics deviate from classical logic in their treatment of
phenomena such as truth values, negation, and implication, offering a
richer and more nuanced framework for modeling real-world phenomena.

The proposed research project builds upon existing work in this area,
which has primarily focused on developing individual subclassical
calculi and exploring their properties. While these contributions have
been valuable, they have not yet yielded a unified framework for
understanding the relationships between different calculi. This proposal
addresses this gap by introducing a subclassical graph of calculi, a
structured representation that captures the connections between various
calculi based on their shared properties and transformations.

The subclassical graph of calculi will be constructed using a
combination of graph-theoretic techniques and computational logic
methods. The graph will represent calculi as nodes and their
relationships as edges, with edge types encoding different kinds of
connections, such as equivalence, embedding, and extension.
Computational logic tools will be employed to automate the
identification and classification of these relationships, ensuring the
comprehensiveness and consistency of the graph.

The construction of the subclassical graph of calculi will provide
several significant contributions to the field of subclassical logic and
computation. Firstly, the graph will serve as a valuable tool for
visualizing and understanding the intricate relationships between
different calculi, facilitating the discovery of new connections and
patterns. Secondly, the graph will provide a foundation for developing
new techniques for interoperating and comparing calculi, enabling the
transfer of knowledge and tools across different formal systems.
Thirdly, the graph will serve as a benchmark for evaluating the
expressiveness and computational power of different calculi, aiding in
the selection of appropriate formal systems for specific applications.

In summary, this research proposal presents a novel approach to
constructing a subclassical graph of calculi, a theoretical framework
for understanding the relationships between different formal systems of
logic and computation. The proposed approach leverages recent advances
in graph theory and computational logic to provide a more comprehensive
and rigorous representation of the intricate connections between various
calculi. The construction of the subclassical graph of calculi will
provide several significant contributions to the field of subclassical
logic and computation, including improved understanding of calculi
relationships, new techniques for interoperating calculi, and a
benchmark for evaluating calculi expressiveness.

\hypertarget{introduction-v5}{%
\subsection{Introduction v5}\label{introduction-v5}}

\hypertarget{project-overview}{%
\subsubsection{Project Overview}\label{project-overview}}

The present research proposal delves into the exploration of
subclassical graphs of calculi, a novel and underexplored domain with
the potential to revolutionize our understanding of computation and
formal logic. The project aims to develop a comprehensive framework for
constructing and analyzing subclassical graphs, shedding light on their
inherent properties and applications in various fields.

\hypertarget{context-and-related-work}{%
\subsubsection{Context and Related
Work}\label{context-and-related-work}}

Existing research in subclassical logic has primarily focused on
axiomatic and algebraic approaches, while the graphical representation
of subclassical logics remains relatively uncharted territory.
Subclassical graphs offer a unique perspective on these logics,
providing a visual and intuitive means of understanding their structure
and dynamics.

\hypertarget{unique-and-innovative-approaches}{%
\subsubsection{Unique and Innovative
Approaches}\label{unique-and-innovative-approaches}}

This research project distinguishes itself from previous work in several
key aspects:

\begin{enumerate}
\def\labelenumi{\arabic{enumi}.}
\item
  \begin{quote}
  \textbf{Formalization of Subclassical Graphs:} The project proposes a
  rigorous formalization of subclassical graphs, establishing a precise
  mathematical foundation for their analysis and manipulation.
  \end{quote}
\item
  \begin{quote}
  \textbf{Algorithmic Techniques:} Novel algorithmic techniques will be
  developed to efficiently construct and manipulate subclassical graphs,
  enabling the exploration of large and complex systems.
  \end{quote}
\item
  \begin{quote}
  \textbf{Applications in Automated Reasoning:} The project will
  investigate the application of subclassical graphs in automated
  reasoning systems, potentially leading to more powerful and versatile
  inference mechanisms.
  \end{quote}
\end{enumerate}

\hypertarget{significance-and-contributions}{%
\subsubsection{Significance and
Contributions}\label{significance-and-contributions}}

The proposed research holds significant promise for advancing our
understanding of subclassical logic and its applications. The
development of a formal framework for subclassical graphs will provide a
valuable tool for researchers and practitioners alike, facilitating the
analysis, manipulation, and application of these logics.

Moreover, the project\textquotesingle s exploration of algorithmic
techniques and applications in automated reasoning is expected to yield
tangible benefits in various domains, including artificial intelligence,
computer science, and decision-making systems.

\hypertarget{introduction-v6}{%
\subsection{Introduction v6}\label{introduction-v6}}

The field of logic has witnessed significant advancements in recent
decades, fueled by the development of powerful formal methods and
automated reasoning techniques. However, the vast and diverse landscape
of logical systems poses challenges in identifying connections and
overarching principles. Existing approaches to classifying and comparing
logical calculi often rely on ad hoc criteria, leading to a fragmented
and incomplete understanding of the relationships between different
systems.

To address these limitations, this research proposal outlines the
development of a subclassical graph of calculi, a novel approach to
formalizing and analyzing the relationships between different logical
calculi. The proposed subclassical graph will provide a comprehensive
and systematic framework for understanding the interplay of various
logical systems, enabling researchers to identify connections, uncover
hidden relationships, generate novel calculi, and explore new avenues of
research.

Context and Background

Logical calculi are formal systems for reasoning and deriving
conclusions from a set of axioms or assumptions or rules such as
structural or inferential rules. They play a fundamental role in
mathematics, computer science, and philosophy, providing a rigorous
foundation for reasoning, proof construction, refutation construction,
modeling, and countermodeling. Over the centuries, a vast array of
logical calculi have been developed, each tailored to specific
applications and embodying distinct logical properties.

Current State of Research

The study of logical calculi has witnessed significant advancements in
recent decades, fueled by the development of powerful formal methods and
automated reasoning techniques. However, the vast and diverse landscape
of logical systems poses challenges in identifying connections and
overarching principles. Existing approaches to classifying and comparing
logical calculi often rely on ad hoc criteria, leading to a fragmented
and incomplete understanding of the relationships between different
systems.

Unique and Innovative Approach

The proposed subclassical graph of calculi addresses these limitations
by providing a unified and rigorous framework for analyzing the
relationships between logical calculi. The subclassical graph will
represent logical calculi as nodes and the relationships between them as
edges, where the edges capture the logical implications and syntactic
similarities between different systems.

Significance and Impact

The development of the subclassical graph of calculi will have
significant implications for research in logic and its applications. It
will provide a comprehensive and systematic framework for understanding
the interplay of various logical systems, enabling researchers to:

\begin{itemize}
\item
  \begin{quote}
  Identify connections between seemingly disparate calculi.
  \end{quote}
\item
  \begin{quote}
  Uncover hidden relationships and patterns in the landscape of logical
  systems.
  \end{quote}
\item
  \begin{quote}
  Explore new avenues of research by traversing the subclassical graph.
  \end{quote}
\item
  \begin{quote}
  Develop novel logical calculi with tailored properties for specific
  applications.
  \end{quote}
\end{itemize}

The subclassical graph of calculi has the potential to revolutionize our
understanding of logical systems and their applications, leading to
advancements in mathematics, computer science, and philosophy.

\hypertarget{introduction-v7}{%
\subsection{Introduction v7}\label{introduction-v7}}

The project aims to develop a comprehensive framework for constructing
and analyzing subclassical graphs, shedding light on their inherent
properties and applications in various fields.

Context and Related Work:

Existing research in subclassical logic has primarily focused on
axiomatic and algebraic approaches, while the graphical representation
of subclassical logics remains relatively uncharted territory.
Subclassical graphs offer a unique perspective on these logics,
providing a visual and intuitive means of understanding their structure
and dynamics.

What is being done both generally and specifically in the same or
related areas:

The study of subclassical logics and their corresponding calculi has
gained significant traction in recent years, driven by the increasing
complexity and expressive power of formal systems in various domains,
including computer science, artificial intelligence, and linguistics.
Subclassical logics deviate from classical logic in their treatment of
phenomena such as truth values, negation, and implication, offering a
richer and more nuanced framework for modeling real-world phenomena.

Unique and Innovative Approaches:

This research project distinguishes itself from previous work in several
key aspects:

\begin{enumerate}
\def\labelenumi{\arabic{enumi}.}
\item
  \begin{quote}
  \textbf{Formalization of Subclassical Graphs:} The project proposes a
  rigorous formalization of subclassical graphs, establishing a precise
  mathematical foundation for their analysis and manipulation.
  \end{quote}
\item
  \begin{quote}
  \textbf{Algorithmic Techniques:} Novel algorithmic techniques will be
  developed to efficiently construct and manipulate subclassical graphs,
  enabling the exploration of large and complex systems.
  \end{quote}
\item
  \begin{quote}
  \textbf{Applications in Automated Reasoning:} The project will
  investigate the application of subclassical graphs in automated
  reasoning systems, potentially leading to more powerful and versatile
  inference mechanisms.
  \end{quote}
\end{enumerate}

Significance and Contributions:

The proposed research holds significant promise for advancing our
understanding of subclassical logic and its applications. The
development of a formal framework for subclassical graphs will provide a
valuable tool for researchers and practitioners alike, facilitating the
analysis, manipulation, and application of these logics. Moreover, the
project\textquotesingle s exploration of algorithmic techniques and
applications in automated reasoning is expected to yield tangible
benefits in various domains, including artificial intelligence, computer
science, and decision-making systems.

\hypertarget{needs}{%
\subsection{Needs}\label{needs}}

\textbf{What needs to be done and why?}

Subclassical logics have emerged as powerful tools for formalizing
reasoning in a wide range of domains, including computer science,
mathematics, and linguistics. However, despite their growing importance,
there is a lack of a comprehensive and unified understanding of the
subclassical graph of calculi. This lack of understanding hinders the
development of new subclassical logics and makes it difficult to compare
and contrast existing logics.

\textbf{What significant needs are you trying to meet? Compared to other
projects in the same area, what sets yours apart in terms of need?}

This project aims to develop a comprehensive and unified understanding
of the subclassical graph of calculi. This will be achieved by
developing a new formal framework for representing and reasoning about
subclassical logics. This framework will be used to develop a new
classification of subclassical logics and to identify new relationships
between existing logics.

\textbf{What services are to be delivered? Why? Use specifics from
preliminary studies, needs assessment, documentation, and data
supporting your proposal.}

This project will deliver the following services:

\begin{itemize}
\item
  \begin{quote}
  A new formal framework for representing and reasoning about
  subclassical logics.
  \end{quote}
\item
  \begin{quote}
  A new classification of subclassical logics.
  \end{quote}
\item
  \begin{quote}
  A new set of relationships between existing subclassical logics.
  \end{quote}
\end{itemize}

These services will be used to develop new subclassical logics and to
improve the understanding of existing logics.

\textbf{What gaps that your work can fill exist in the knowledge base of
your field?}

There are several gaps in the knowledge base of subclassical logic that
this project can fill. These gaps include:

\begin{itemize}
\item
  \begin{quote}
  A lack of a comprehensive and unified understanding of the
  subclassical graph of calculi.
  \end{quote}
\item
  \begin{quote}
  A lack of a formal framework for representing and reasoning about
  subclassical logics.
  \end{quote}
\item
  \begin{quote}
  A lack of a clear and concise classification of subclassical logics.
  \end{quote}
\end{itemize}

\textbf{Is the problem both significant and manageable? Do you have the
resources to handle the problem?}

The problem of developing a comprehensive and unified understanding of
the subclassical graph of calculi is both significant and manageable.
The problem is significant because it is a fundamental problem in
subclassical logic that has not been solved. The problem is manageable
because there are a number of existing results that can be used to build
upon. The team working on this project has the resources necessary to
handle the problem. The team includes experts in subclassical logic and
formal methods.

In addition to the above, the project will also develop a new set of
tools for working with subclassical logics. These tools will include a
new theorem prover for subclassical logics and a new software library
for implementing subclassical logics.

\hypertarget{need-v2}{%
\subsection{Need v2}\label{need-v2}}

\textbf{What needs to be done and why?}

Subclassical calculi are a diverse and powerful family of formal systems
that have found applications in a wide range of areas, including
computer science, mathematics, and linguistics. However, there is no
single unifying framework for understanding the relationships between
these different calculi. The development of such a framework would be a
significant contribution to the field of logic.

\textbf{What significant needs are you trying to meet? Compared to other
projects in the same area, what sets yours apart in terms of need?}

A unifying framework for subclassical calculi would provide a number of
benefits, including:

\begin{itemize}
\item
  \begin{quote}
  A better understanding of the logical relationships between different
  subclassical calculi.
  \end{quote}
\item
  \begin{quote}
  The development of new and more powerful subclassical calculi.
  \end{quote}
\item
  \begin{quote}
  The application of subclassical calculi to new areas.
  \end{quote}
\end{itemize}

Our project is different from other projects in the same area in that it
takes a graph-based approach to understanding subclassical calculi. This
approach has the potential to provide a more comprehensive and
insightful understanding of these calculi than traditional approaches.

\textbf{What services are to be delivered? Why? Use specifics from
preliminary studies, needs assessment, documentation, and data
supporting your proposal.}

The proposed project will deliver the following services:

\begin{itemize}
\item
  \begin{quote}
  A graph-based representation of subclassical calculi.
  \end{quote}
\item
  \begin{quote}
  A methodology for comparing and contrasting different subclassical
  calculi.
  \end{quote}
\item
  \begin{quote}
  A set of tools for developing new subclassical calculi.
  \end{quote}
\end{itemize}

These services will be delivered by a team of researchers with expertise
in logic, graph theory, and computer science. The researchers will use a
variety of methods, including theoretical analysis, experimental
evaluation, and software development.

\textbf{What gaps that your work can fill exist in the knowledge base of
your field?}

There are a number of gaps in the knowledge base of our field that our
work can fill. These gaps include:

\begin{itemize}
\item
  \begin{quote}
  A lack of a unifying framework for understanding subclassical calculi.
  \end{quote}
\item
  \begin{quote}
  A lack of tools for comparing and contrasting different subclassical
  calculi.
  \end{quote}
\item
  \begin{quote}
  A lack of methods for developing new subclassical calculi.
  \end{quote}
\end{itemize}

\textbf{Is the problem both significant and manageable? Do you have the
resources to handle the problem?}

We believe that the problem of developing a unifying framework for
subclassical calculi is both significant and manageable. We have the
resources to handle the problem, including a team of experienced
researchers and access to a variety of computational resources.

We are confident that our project will make a significant contribution
to the field of logic.

\hypertarget{need-v3}{%
\subsection{Need v3}\label{need-v3}}

\textbf{Need Statement}

\textbf{What needs to be done and why?}

There is a need for a new approach to understanding and comparing
subclassical calculi. The existing approaches are often ad hoc and do
not provide a clear picture of the relationships between different
calculi. A graph-based approach has the potential to provide a more
systematic and insightful way to study subclassical calculi.

\textbf{What significant needs are you trying to meet? Compared to other
projects in the same area, what sets yours apart in terms of need?}

This project will address the following significant needs:

\begin{itemize}
\item
  \begin{quote}
  The need for a more systematic and insightful way to study
  subclassical calculi.
  \end{quote}
\item
  \begin{quote}
  The need for a better understanding of the relationships between
  different subclassical calculi.
  \end{quote}
\item
  \begin{quote}
  The need for a new tool that can be used to compare and contrast
  different subclassical calculi.
  \end{quote}
\end{itemize}

This project is different from other projects in the same area in that
it will take a graph-based approach to studying subclassical calculi.
This approach has the potential to provide a more systematic and
insightful way to study subclassical calculi than the existing
approaches.

\textbf{What services are to be delivered? Why? Use specifics from
preliminary studies, needs assessment, documentation, and data
supporting your proposal.}

This project will deliver the following services:

\begin{itemize}
\item
  \begin{quote}
  A new graph-based approach to studying subclassical calculi.
  \end{quote}
\item
  \begin{quote}
  A new tool that can be used to compare and contrast different
  subclassical calculi.
  \end{quote}
\item
  \begin{quote}
  A new understanding of the relationships between different
  subclassical calculi.
  \end{quote}
\end{itemize}

These services will be delivered using the following methods:

\begin{itemize}
\item
  \begin{quote}
  A literature review of existing approaches to studying subclassical
  calculi.
  \end{quote}
\item
  \begin{quote}
  The development of a new graph-based approach to studying subclassical
  calculi.
  \end{quote}
\item
  \begin{quote}
  The development of a new tool that can be used to compare and contrast
  different subclassical calculi.
  \end{quote}
\item
  \begin{quote}
  The application of the new approach and tool to a variety of
  subclassical calculi.
  \end{quote}
\end{itemize}

\textbf{What gaps that your work can fill exist in the knowledge base of
your field?}

This project will fill the following gaps in the knowledge base of our
field:

\begin{itemize}
\item
  \begin{quote}
  The gap in our understanding of the systematic and insightful way to
  study subclassical calculi.
  \end{quote}
\item
  \begin{quote}
  The gap in our understanding of the relationships between different
  subclassical calculi.
  \end{quote}
\item
  \begin{quote}
  The gap in the tool that can be used to compare and contrast different
  subclassical calculi.
  \end{quote}
\end{itemize}

\textbf{Is the problem both significant and manageable? Do you have the
resources to handle the problem?}

The problem is both significant and manageable. The problem is
significant because it is a fundamental problem in the field of logic.
The problem is manageable because there is a clear path to a solution.
The resources to handle the problem are available.

\textbf{Call to Action}

We urge you to support this project so that we can make a significant
contribution to our understanding of subclassical calculi.

\hypertarget{goals-and-objectives-v1}{%
\subsection{Goals and Objectives v1}\label{goals-and-objectives-v1}}

\textbf{Goals}

\begin{itemize}
\tightlist
\item
\end{itemize}

\textbf{Objectives}

\begin{itemize}
\item
  \begin{quote}
  Compile a comprehensive list of non-classical logics, including their
  axioms, rules, and properties.
  \end{quote}
\item
  \begin{quote}
  Classify the non-classical logics based on their properties, such as
  the types of negation that they use, the consistency conditions that
  they satisfy, and the types of truth values that they employ.
  \end{quote}
\item
  \begin{quote}
  Construct the subclassical graph of calculi by connecting the
  different non-classical logics based on their relationships.
  \end{quote}
\item
  \begin{quote}
  Analyze the subclassical graph of calculi to identify common patterns
  and themes, and to suggest new areas of research.
  \end{quote}
\item
  \begin{quote}
  Develop a new formal framework for representing and reasoning about
  subclassical logics.
  \end{quote}
\item
  \begin{quote}
  Develop a new classification of subclassical logics.
  \end{quote}
\item
  \begin{quote}
  Identify new relationships between existing subclassical logics.
  \end{quote}
\item
  \begin{quote}
  Develop a new set of tools for working with subclassical logics.
  \end{quote}
\end{itemize}

\textbf{Timeline}

\begin{itemize}
\item
  \begin{quote}
  Identify and classify the non-classical logics: 6 months.
  \end{quote}
\item
  \begin{quote}
  Construct the subclassical graph of calculi: 3 months.
  \end{quote}
\item
  \begin{quote}
  Analyze the subclassical graph of calculi: 2 months.
  \end{quote}
\item
  \begin{quote}
  Develop a new formal framework for representing and reasoning about
  subclassical logics: 4 months.
  \end{quote}
\item
  \begin{quote}
  Develop a new classification of subclassical logics: 3 months.
  \end{quote}
\item
  \begin{quote}
  Identify new relationships between existing subclassical logics: 2
  months.
  \end{quote}
\item
  \begin{quote}
  Develop a new set of tools for working with subclassical logics: 5
  months.
  \end{quote}
\end{itemize}

\textbf{Budget}

\begin{itemize}
\item
  \begin{quote}
  Personnel: \$50,000.
  \end{quote}
\item
  \begin{quote}
  Travel: \$10,000.
  \end{quote}
\item
  \begin{quote}
  Equipment: \$10,000.
  \end{quote}
\item
  \begin{quote}
  Subcontracts: \$10,000.
  \end{quote}
\item
  \begin{quote}
  Other expenses: \$10,000.
  \end{quote}
\end{itemize}

\textbf{Total:} \$100,000.

\textbf{Personnel}

\begin{itemize}
\item
  \begin{quote}
  Principal investigator: \$10,000.
  \end{quote}
\item
  \begin{quote}
  Research assistant: \$5,000.
  \end{quote}
\item
  \begin{quote}
  Software engineer: \$5,000.
  \end{quote}
\end{itemize}

\textbf{Travel}

\begin{itemize}
\item
  \begin{quote}
  Conference travel: \$5,000.
  \end{quote}
\item
  \begin{quote}
  Site visits: \$5,000.
  \end{quote}
\end{itemize}

\textbf{Equipment}

\begin{itemize}
\item
  \begin{quote}
  Computer hardware: \$5,000.
  \end{quote}
\item
  \begin{quote}
  Software licenses: \$5,000.
  \end{quote}
\end{itemize}

\textbf{Subcontracts}

\begin{itemize}
\item
  \begin{quote}
  Data analysis: \$5,000.
  \end{quote}
\item
  \begin{quote}
  Software development: \$5,000.
  \end{quote}
\end{itemize}

\textbf{Other expenses}

\begin{itemize}
\item
  \begin{quote}
  Publications: \$5,000.
  \end{quote}
\item
  \begin{quote}
  Legal fees: \$5,000.
  \end{quote}
\end{itemize}

\textbf{Total:} \$100,000.

\hypertarget{goals-and-objectives-v2}{%
\subsection{Goals and Objectives v2}\label{goals-and-objectives-v2}}

\textbf{Goals}

\begin{itemize}
\tightlist
\item
\end{itemize}

\textbf{Objectives}

\begin{itemize}
\item
  \begin{quote}
  To compile a comprehensive list of non-classical logics, including
  their axioms, rules, and properties.
  \end{quote}
\item
  \begin{quote}
  To develop a new formal framework for representing and reasoning about
  subclassical logics.
  \end{quote}
\item
  \begin{quote}
  To identify new relationships between existing non-classical logics.
  \end{quote}
\item
  \begin{quote}
  To develop a new set of tools for working with subclassical logics.
  \end{quote}
\item
  \begin{quote}
  To apply the subclassical graph of calculi to develop new and more
  powerful subclassical logics.
  \end{quote}
\item
  \begin{quote}
  To apply the subclassical graph of calculi to new areas, such as
  computer science, mathematics, and linguistics.
  \end{quote}
\end{itemize}

\hypertarget{goals-and-objectives-v3}{%
\subsection{Goals and Objectives v3}\label{goals-and-objectives-v3}}

\textbf{Goals:}

\begin{itemize}
\tightlist
\item
\end{itemize}

\textbf{Objectives:}

\begin{itemize}
\item
  \begin{quote}
  \textbf{Compile a comprehensive list of non-classical logics,
  including their axioms, rules, and properties.\\
  }
  \end{quote}

  \begin{itemize}
  \item
    \begin{quote}
    This will be completed by the end of the first quarter.
    \end{quote}
  \item
    \begin{quote}
    This will involve identifying and researching existing non-classical
    logics.
    \end{quote}
  \item
    \begin{quote}
    This will require a literature review and analysis of existing work.
    \end{quote}
  \end{itemize}
\item
\item
  \begin{quote}
  \textbf{Develop a new formal framework for representing and reasoning
  about subclassical logics.\\
  }
  \end{quote}

  \begin{itemize}
  \item
    \begin{quote}
    This will be completed by the end of the second quarter.
    \end{quote}
  \item
    \begin{quote}
    This will involve developing a new mathematical formalism for
    representing non-classical logics.
    \end{quote}
  \item
    \begin{quote}
    This will require the creation of new definitions, axioms, and
    rules.
    \end{quote}
  \end{itemize}
\item
\item
  \begin{quote}
  \textbf{Identify new relationships between existing non-classical
  logics.\\
  }
  \end{quote}

  \begin{itemize}
  \item
    \begin{quote}
    This will be completed by the end of the third quarter.
    \end{quote}
  \item
    \begin{quote}
    This will involve analyzing the subclassical graph of calculi to
    identify patterns and connections between different logics.
    \end{quote}
  \item
    \begin{quote}
    This will require the development of new methods for comparing and
    contrasting logics.
    \end{quote}
  \end{itemize}
\item
\item
  \begin{quote}
  \textbf{Develop a new set of tools for working with subclassical
  logics.\\
  }
  \end{quote}

  \begin{itemize}
  \item
    \begin{quote}
    This will be completed by the end of the fourth quarter.
    \end{quote}
  \item
    \begin{quote}
    This will involve developing new software tools for manipulating and
    analyzing non-classical logics.
    \end{quote}
  \item
    \begin{quote}
    This will require the creation of new algorithms and data
    structures.
    \end{quote}
  \end{itemize}
\item
\end{itemize}

\textbf{Timeline:}

\begin{itemize}
\item
  \begin{quote}
  Identify and classify the non-classical logics: 6 months.
  \end{quote}
\item
  \begin{quote}
  Construct the subclassical graph of calculi: 3 months.
  \end{quote}
\item
  \begin{quote}
  Analyze the subclassical graph of calculi: 2 months.
  \end{quote}
\item
  \begin{quote}
  Develop a new formal framework for representing and reasoning about
  subclassical logics: 4 months.
  \end{quote}
\item
  \begin{quote}
  Develop a new classification of subclassical logics: 3 months.
  \end{quote}
\item
  \begin{quote}
  Identify new relationships between existing subclassical logics: 2
  months.
  \end{quote}
\item
  \begin{quote}
  Develop a new set of tools for working with subclassical logics: 5
  months.
  \end{quote}
\end{itemize}

\textbf{Budget:}

\begin{itemize}
\item
  \begin{quote}
  Personnel: \$50,000.
  \end{quote}
\item
  \begin{quote}
  Travel: \$10,000.
  \end{quote}
\item
  \begin{quote}
  Equipment: \$10,000.
  \end{quote}
\item
  \begin{quote}
  Subcontracts: \$10,000.
  \end{quote}
\item
  \begin{quote}
  Other expenses: \$10,000.
  \end{quote}
\item
  \begin{quote}
  Total: \$100,000.
  \end{quote}
\end{itemize}

\textbf{Personnel:}

\begin{itemize}
\item
  \begin{quote}
  Principal investigator: \$10,000.
  \end{quote}
\item
  \begin{quote}
  Research assistant: \$5,000.
  \end{quote}
\item
  \begin{quote}
  Software engineer: \$5,000.
  \end{quote}
\item
  \begin{quote}
  Travel: Conference travel: \$5,000.
  \end{quote}
\item
  \begin{quote}
  Site visits: \$5,000.
  \end{quote}
\item
  \begin{quote}
  Equipment: Computer hardware: \$5,000.
  \end{quote}
\item
  \begin{quote}
  Software licenses: \$5,000.
  \end{quote}
\item
  \begin{quote}
  Subcontracts: Data analysis: \$5,000.
  \end{quote}
\item
  \begin{quote}
  Software development: \$5,000.
  \end{quote}
\item
  \begin{quote}
  Other expenses: Publications: \$5,000.
  \end{quote}
\item
  \begin{quote}
  Legal fees: \$5,000.
  \end{quote}
\item
  \begin{quote}
  Total: \$100,000.
  \end{quote}
\end{itemize}

\hypertarget{goals-and-objectives-v4}{%
\subsection{Goals and Objectives v4}\label{goals-and-objectives-v4}}

The following are the measurable outcomes of the project:

\begin{itemize}
\tightlist
\item
\end{itemize}

The following are the specific conditions under which the project will
be carried out:

\begin{itemize}
\item
  \begin{quote}
  The project will be carried out by a team of experts in logic, graph
  theory, and computer science.
  \end{quote}
\item
  \begin{quote}
  The project will use a variety of methods, including theoretical
  analysis, experimental evaluation, and software development.
  \end{quote}
\item
  \begin{quote}
  The project will be funded by a grant from a research institution.
  \end{quote}
\end{itemize}

The following is the timeline for the project:

\begin{itemize}
\item
  \begin{quote}
  Identify and classify the non-classical logics: 6 months.
  \end{quote}
\item
  \begin{quote}
  Construct the subclassical graph of calculi: 3 months.
  \end{quote}
\item
  \begin{quote}
  Analyze the subclassical graph of calculi: 2 months.
  \end{quote}
\item
  \begin{quote}
  Develop a new formal framework for representing and reasoning about
  non-classical logics: 4 months.
  \end{quote}
\item
  \begin{quote}
  Develop a new classification of non-classical logics: 3 months.
  \end{quote}
\item
  \begin{quote}
  Identify new relationships between existing non-classical logics: 2
  months.
  \end{quote}
\item
  \begin{quote}
  Develop a new set of tools for working with non-classical logics: 5
  months.
  \end{quote}
\end{itemize}

\hypertarget{goals-and-objectives-v5}{%
\subsection{Goals and Objectives v5}\label{goals-and-objectives-v5}}

Goals:

\begin{itemize}
\tightlist
\item
\end{itemize}

Objectives:

\begin{itemize}
\item
  \begin{quote}
  Compile a comprehensive list of non-classical logics, including their
  axioms, rules, and properties.
  \end{quote}
\item
  \begin{quote}
  Classify the non-classical logics based on their properties, such as
  the types of negation that they use, the consistency conditions that
  they satisfy, and the types of truth values that they employ.
  \end{quote}
\item
  \begin{quote}
  Construct the subclassical graph of calculi by connecting the
  different non-classical logics based on their relationships.
  \end{quote}
\item
  \begin{quote}
  Analyze the subclassical graph of calculi to identify common patterns
  and themes, and to suggest new areas of research.
  \end{quote}
\item
  \begin{quote}
  Develop a new formal framework for representing and reasoning about
  subclassical logics.
  \end{quote}
\item
  \begin{quote}
  Develop a new classification of subclassical logics.
  \end{quote}
\item
  \begin{quote}
  Identify new relationships between existing subclassical logics.
  \end{quote}
\item
  \begin{quote}
  Develop a new set of tools for working with subclassical logics.
  \end{quote}
\end{itemize}

Timeline:

\begin{itemize}
\item
  \begin{quote}
  Identify and classify the non-classical logics: 6 months.
  \end{quote}
\item
  \begin{quote}
  Construct the subclassical graph of calculi: 3 months.
  \end{quote}
\item
  \begin{quote}
  Analyze the subclassical graph of calculi: 2 months.
  \end{quote}
\item
  \begin{quote}
  Develop a new formal framework for representing and reasoning about
  subclassical logics: 4 months.
  \end{quote}
\item
  \begin{quote}
  Develop a new classification of subclassical logics: 3 months.
  \end{quote}
\item
  \begin{quote}
  Identify new relationships between existing subclassical logics: 2
  months.
  \end{quote}
\item
  \begin{quote}
  Develop a new set of tools for working with subclassical logics: 5
  months.
  \end{quote}
\end{itemize}

Budget:

\begin{itemize}
\item
  \begin{quote}
  Personnel: \$50,000.
  \end{quote}
\item
  \begin{quote}
  Travel: \$10,000.
  \end{quote}
\item
  \begin{quote}
  Equipment: \$10,000.
  \end{quote}
\item
  \begin{quote}
  Subcontracts: \$10,000.
  \end{quote}
\item
  \begin{quote}
  Other expenses: \$10,000.
  \end{quote}
\item
  \begin{quote}
  Total: \$100,000.
  \end{quote}
\end{itemize}

Personnel:

\begin{itemize}
\item
  \begin{quote}
  Principal investigator: \$10,000.
  \end{quote}
\item
  \begin{quote}
  Research assistant: \$5,000.
  \end{quote}
\item
  \begin{quote}
  Software engineer: \$5,000.
  \end{quote}
\end{itemize}

Travel:

\begin{itemize}
\item
  \begin{quote}
  Conference travel: \$5,000.
  \end{quote}
\item
  \begin{quote}
  Site visits: \$5,000.
  \end{quote}
\end{itemize}

Equipment:

\begin{itemize}
\item
  \begin{quote}
  Computer hardware: \$5,000.
  \end{quote}
\item
  \begin{quote}
  Software licenses: \$5,000.
  \end{quote}
\end{itemize}

Subcontracts:

\begin{itemize}
\item
  \begin{quote}
  Data analysis: \$5,000.
  \end{quote}
\item
  \begin{quote}
  Software development: \$5,000.
  \end{quote}
\end{itemize}

Other expenses:

\begin{itemize}
\item
  \begin{quote}
  Publications: \$5,000.
  \end{quote}
\item
  \begin{quote}
  Legal fees: \$5,000.
  \end{quote}
\end{itemize}

\hypertarget{goals-and-objectives-v6}{%
\subsection{Goals and Objectives v6}\label{goals-and-objectives-v6}}

\textbf{Goals}

\begin{itemize}
\tightlist
\item
\end{itemize}

\textbf{Objectives}

\begin{itemize}
\item
  \begin{quote}
  \textbf{Objective 1:} Define a graph representation of subclassical
  calculi that is both expressive and computationally efficient.
  \end{quote}
\item
  \begin{quote}
  \textbf{Measurable Outcome 1:} A formal definition of the graph
  representation of subclassical calculi, along with a proof of its
  expressiveness and computational efficiency.
  \end{quote}
\item
  \begin{quote}
  \textbf{Objective 2:} Develop algorithms for manipulating and
  analyzing graphs representing subclassical calculi.
  \end{quote}
\item
  \begin{quote}
  \textbf{Measurable Outcome 2:} Implementation of algorithms for
  manipulating and analyzing graphs representing subclassical calculi,
  along with a demonstration of their effectiveness and efficiency.
  \end{quote}
\item
  \begin{quote}
  \textbf{Objective 3:} Apply the graph representation of subclassical
  calculi to investigate the structural properties of a variety of
  subclassical logics.
  \end{quote}
\item
  \begin{quote}
  \textbf{Measurable Outcome 3:} A series of papers presenting new
  results on the structural properties of subclassical logics, obtained
  using the graph representation of subclassical calculi.
  \end{quote}
\item
  \begin{quote}
  \textbf{Objective 4:} Use the graph representation of subclassical
  calculi to develop new subclassical logics.
  \end{quote}
\item
  \begin{quote}
  \textbf{Measurable Outcome 4:} A series of papers presenting new
  subclassical logics, along with a demonstration of their properties
  and applications.
  \end{quote}
\end{itemize}

\hypertarget{goals-and-objectives-v7}{%
\subsection{Goals and Objectives v7}\label{goals-and-objectives-v7}}

\textbf{Goals}

\begin{itemize}
\tightlist
\item
\end{itemize}

\textbf{Objectives}

\begin{itemize}
\item
  \begin{quote}
  To create a database of subclassical logics, including their axioms,
  theorems, and proof systems.
  \end{quote}
\item
  \begin{quote}
  To develop algorithms for computing the entailment relations between
  subclassical logics.
  \end{quote}
\item
  \begin{quote}
  To use the subclassical graph of calculi to prove new theorems about
  subclassical logics.
  \end{quote}
\end{itemize}

\textbf{Relationships between objectives and measurements and outcomes}

The objective of creating a database of subclassical logics will be
measured by the number of logics in the database and the completeness of
the information about each logic. The objective of developing algorithms
for computing the entailment relations between subclassical logics will
be measured by the efficiency and accuracy of the algorithms. The
objective of using the subclassical graph of calculi to prove new
theorems about subclassical logics will be measured by the number of new
theorems that are proven.

\textbf{Goals and Objectives}

The overall goal of this project is to develop a better understanding of
subclassical logics. Subclassical logics are a generalization of
classical logic that allow for the possibility of truth values that are
neither true nor false. Subclassical logics have been used to model a
variety of phenomena, including quantum mechanics, fuzzy logic, and
paraconsistent reasoning.

The objectives of this project are to develop a subclassical graph of
calculi that captures the relationships between different types of
subclassical logics and to use the subclassical graph of calculi to
identify new subclassical logics and to prove new theorems about
subclassical logics.

The subclassical graph of calculi will be a directed graph in which the
nodes represent subclassical logics and the edges represent entailment
relations between subclassical logics. The entailment relation between
two subclassical logics L and L\textquotesingle{} is defined as follows:
L entails L\textquotesingle{} if every theorem of L is also a theorem of
L\textquotesingle.

The subclassical graph of calculi will be used to identify new
subclassical logics by identifying connected components of the graph. A
connected component of the graph is a subset of the nodes of the graph
such that there is a path between every pair of nodes in the subset.
Each connected component of the graph represents a different type of
subclassical logic.

The subclassical graph of calculi will also be used to prove new
theorems about subclassical logics by using the graph to identify lemmas
that can be used to prove the theorems. A lemma is a theorem that is
used to prove another theorem.

The development of the subclassical graph of calculi will make a
significant contribution to our understanding of subclassical logics.
The graph will be a valuable tool for identifying new subclassical
logics and for proving new theorems about subclassical logics.

\hypertarget{goals-and-objectives-v8}{%
\subsection{Goals and Objectives v8}\label{goals-and-objectives-v8}}

The goals and objectives of this project are to:

\begin{itemize}
\item
  \begin{quote}
  \textbf{Develop a comprehensive and unified representation of
  subclassical logics}. This will involve identifying and classifying
  the different types of subclassical logics that exist, as well as
  their axioms, rules, and properties.
  \end{quote}
\item
  \begin{quote}
  \textbf{Construct a subclassical graph of calculi}. This will be a
  diagram that visually represents the relationships between different
  subclassical logics. The graph will be constructed by connecting the
  different subclassical logics based on their relationships.
  \end{quote}
\item
  \begin{quote}
  \textbf{Analyze the subclassical graph of calculi}. This will involve
  identifying common patterns and themes in the graph, and suggesting
  new areas of research.
  \end{quote}
\item
  \begin{quote}
  \textbf{Develop novel logical calculi with tailored properties for
  specific applications}. This will involve using the subclassical graph
  of calculi to identify gaps in the existing landscape of logical
  systems, and then developing new calculi to fill those gaps.
  \end{quote}
\end{itemize}

The objectives of this project are to:

\begin{itemize}
\item
  \begin{quote}
  \textbf{Identify connections between seemingly disparate subclassical
  logics}. This will involve using the subclassical graph of calculi to
  identify relationships between logics that are not immediately
  apparent.
  \end{quote}
\item
  \begin{quote}
  \textbf{Uncover hidden relationships and patterns in the landscape of
  subclassical logics}. This will involve using the subclassical graph
  of calculi to identify patterns and relationships that are not easily
  visible when looking at individual logics in isolation.
  \end{quote}
\item
  \begin{quote}
  \textbf{Explore new avenues of research by traversing the subclassical
  graph}. This will involve using the subclassical graph of calculi to
  identify new directions for research in subclassical logic.
  \end{quote}
\item
  \begin{quote}
  \textbf{Develop novel logical calculi with tailored properties for
  specific applications}. This will involve using the subclassical graph
  of calculi to identify gaps in the existing landscape of logical
  systems, and then developing new calculi to fill those gaps.
  \end{quote}
\end{itemize}

The measurements and outcomes of this project are to:

\begin{itemize}
\item
  \begin{quote}
  \textbf{Develop a comprehensive and unified representation of
  subclassical logics}. This will be measured by the quality and
  completeness of the representation.
  \end{quote}
\item
  \begin{quote}
  \textbf{Construct a subclassical graph of calculi}. This will be
  measured by the accuracy and completeness of the graph.
  \end{quote}
\item
  \begin{quote}
  \textbf{Analyze the subclassical graph of calculi}. This will be
  measured by the identification of common patterns and themes in the
  graph.
  \end{quote}
\item
  \begin{quote}
  \textbf{Develop novel logical calculi with tailored properties for
  specific applications}. This will be measured by the development of
  new calculi that meet the specified requirements.
  \end{quote}
\end{itemize}

The relationships between the objectives and the measurements and
outcomes are as follows:

\begin{itemize}
\item
  \begin{quote}
  The objectives of identifying connections between subclassical logics,
  uncovering hidden relationships and patterns, exploring new avenues of
  research, and developing novel logical calculi are all related to the
  measurement of developing a comprehensive and unified representation
  of subclassical logics.
  \end{quote}
\item
  \begin{quote}
  The objectives of identifying connections between subclassical logics,
  uncovering hidden relationships and patterns, and exploring new
  avenues of research are all related to the measurement of constructing
  a subclassical graph of calculi.
  \end{quote}
\item
  \begin{quote}
  The objectives of analyzing the subclassical graph of calculi,
  developing novel logical calculi with tailored properties for specific
  applications, and developing a comprehensive and unified
  representation of subclassical logics are all related to the
  measurement of analyzing the subclassical graph of calculi.
  \end{quote}
\item
  \begin{quote}
  The objectives of developing novel logical calculi with tailored
  properties for specific applications, and developing a comprehensive
  and unified representation of subclassical logics are all related to
  the measurement of developing novel logical calculi with tailored
  properties for specific applications.
  \end{quote}
\end{itemize}

\hypertarget{goals-and-objectives-v9}{%
\subsection{Goals and Objectives v9}\label{goals-and-objectives-v9}}

\textbf{Goals}

The project has the following goals:

\begin{itemize}
\tightlist
\item
\end{itemize}

\textbf{Objectives}

The project has the following objectives:

\begin{enumerate}
\def\labelenumi{\arabic{enumi}.}
\item
  \begin{quote}
  \textbf{Construct a subclassical graph of calculi:} This objective
  will be achieved by identifying and classifying the different types of
  subclassical logics and then representing them as nodes in a graph.
  The edges of the graph will represent the relationships between the
  different logics, such as equivalence, embedding, and extension.
  \end{quote}
\item
  \begin{quote}
  \textbf{Develop a formal framework for representing and reasoning
  about subclassical graphs:} This objective will be achieved by
  defining a formal language for describing subclassical graphs and by
  developing algorithms for manipulating and reasoning about these
  graphs. The formal framework will be used to identify new
  relationships between existing subclassical logics and to develop new
  tools for working with subclassical logics.
  \end{quote}
\item
  \begin{quote}
  \textbf{Develop a new classification of subclassical logics:} This
  objective will be achieved by analyzing the relationships between the
  different types of subclassical logics and identifying patterns and
  trends. The new classification will be based on the properties of the
  logics, such as their axioms, rules, and semantics.
  \end{quote}
\item
  \begin{quote}
  \textbf{Identify new relationships between existing subclassical
  logics:} This objective will be achieved by exploring the subclassical
  graph and identifying new connections between the different logics.
  These new relationships may be based on the properties of the logics,
  their applications, or their historical development.
  \end{quote}
\item
  \begin{quote}
  \textbf{Develop new tools for working with subclassical logics:} This
  objective will be achieved by developing new software tools, such as a
  theorem prover and a software library, that can be used to work with
  subclassical logics. These tools will be based on the formal framework
  and the new classification of subclassical logics.
  \end{quote}
\end{enumerate}

\textbf{Measurements}

The project will measure its success by the following metrics:

\begin{itemize}
\item
  \begin{quote}
  The quality of the subclassical graph of calculi.
  \end{quote}
\item
  \begin{quote}
  The completeness and accuracy of the new classification of
  subclassical logics.
  \end{quote}
\item
  \begin{quote}
  The number of new relationships identified between existing
  subclassical logics.
  \end{quote}
\item
  \begin{quote}
  The usefulness of the new tools for working with subclassical logics.
  \end{quote}
\end{itemize}

\textbf{Outcomes}

The project aims to achieve the following outcomes:

\begin{itemize}
\item
  \begin{quote}
  A comprehensive and unified understanding of the subclassical graph of
  calculi.
  \end{quote}
\item
  \begin{quote}
  A new classification of subclassical logics that is based on their
  relationships in the subclassical graph.
  \end{quote}
\item
  \begin{quote}
  A new set of tools for working with subclassical logics.
  \end{quote}
\end{itemize}

These outcomes will have the following benefits:

\begin{itemize}
\item
  \begin{quote}
  Improved understanding of the relationships between different
  subclassical logics.
  \end{quote}
\item
  \begin{quote}
  Development of new and more powerful subclassical logics.
  \end{quote}
\item
  \begin{quote}
  Application of subclassical logics to new areas.
  \end{quote}
\end{itemize}

The project will also contribute to the field of logic by developing a
new formal framework for representing and reasoning about subclassical
graphs. This framework will be a valuable tool for researchers and
practitioners alike, facilitating the analysis, manipulation, and
application of these logics.

\hypertarget{goals-and-objectives-v10}{%
\subsection{Goals and Objectives v10}\label{goals-and-objectives-v10}}

The goals of the project are to:

\begin{itemize}
\item
  \begin{quote}
  \textbf{Develop a comprehensive and unified understanding of the
  subclassical graph of calculi}. This will be achieved by developing a
  new formal framework for representing and reasoning about subclassical
  logics. This framework will be used to develop a new classification of
  subclassical logics and to identify new relationships between existing
  logics.
  \end{quote}
\item
  \begin{quote}
  \textbf{Identify new connections between seemingly disparate
  subclassical calculi}. This will be done by analyzing the subclassical
  graph of calculi and identifying patterns and relationships that have
  not been previously recognized. This could lead to the development of
  new and more powerful subclassical logics.
  \end{quote}
\item
  \begin{quote}
  \textbf{Develop novel logical calculi with tailored properties for
  specific applications}. This will be done by leveraging the insights
  gained from the subclassical graph of calculi to design new logics
  that are better suited for specific tasks, such as reasoning under
  uncertainty or representing vagueness.
  \end{quote}
\item
  \begin{quote}
  \textbf{Explore the application of subclassical graphs in automated
  reasoning systems}. This could lead to the development of more
  powerful and versatile inference mechanisms for subclassical logics.
  \end{quote}
\item
  \begin{quote}
  \textbf{Establish a formal foundation for the development of
  superclassical and hyperclassical graphs of calculi}. This would pave
  the way for the study of more general classes of logical calculi, such
  as superclassical and hyperclassical logics.
  \end{quote}
\item
  \begin{quote}
  To develop a formal framework for representing and reasoning about
  subclassical graphs.
  \end{quote}
\item
  \begin{quote}
  To develop a new classification of subclassical logics based on their
  relationships in the subclassical graph.
  \end{quote}
\item
  \begin{quote}
  To identify new relationships between existing subclassical logics.
  \end{quote}
\item
  \begin{quote}
  To develop new tools for working with subclassical logics, such as a
  theorem prover and a software library.
  \end{quote}
\item
  \begin{quote}
  To develop a subclassical graph of calculi that captures the
  relationships between different types of subclassical logics.
  \end{quote}
\item
  \begin{quote}
  To use the subclassical graph of calculi to identify new subclassical
  logics and to prove new theorems about subclassical logics.
  \end{quote}
\item
  \begin{quote}
  To develop a formal framework for representing subclassical calculi
  using graphs.
  \end{quote}
\item
  \begin{quote}
  To use this framework to investigate the structural properties of
  subclassical calculi.
  \end{quote}
\item
  \begin{quote}
  To apply this framework to the development of new subclassical logics.
  \end{quote}
\item
  \begin{quote}
  To develop a comprehensive and unified understanding of the
  subclassical graph of calculi.
  \end{quote}
\item
  \begin{quote}
  To develop a new formal framework for representing and reasoning about
  subclassical logics.
  \end{quote}
\item
  \begin{quote}
  To develop a new classification of subclassical logics.
  \end{quote}
\item
  \begin{quote}
  To identify new relationships between existing subclassical logics.
  \end{quote}
\item
  \begin{quote}
  To develop a new set of tools for working with subclassical logics.
  \end{quote}
\item
  \begin{quote}
  A comprehensive list of non-classical logics, including their axioms,
  rules, and properties.
  \end{quote}
\item
  \begin{quote}
  A new formal framework for representing and reasoning about
  non-classical logics.
  \end{quote}
\item
  \begin{quote}
  A new classification of non-classical logics.
  \end{quote}
\item
  \begin{quote}
  A subclassical graph of calculi that visually represents the
  relationships between these logics.
  \end{quote}
\item
  \begin{quote}
  A list of common patterns and themes in the properties of
  non-classical logics.
  \end{quote}
\item
  \begin{quote}
  A list of new areas of research suggested by the analysis of the
  subclassical graph of calculi.
  \end{quote}
\item
  \begin{quote}
  To develop a comprehensive and unified understanding of the
  subclassical graph of calculi.
  \end{quote}
\item
  \begin{quote}
  To identify and classify the different types of non-classical logics
  that exist.
  \end{quote}
\item
  \begin{quote}
  To construct a subclassical graph of calculi that visually represents
  the relationships between these logics.
  \end{quote}
\item
  \begin{quote}
  To analyze the subclassical graph of calculi to identify common
  patterns and themes.
  \end{quote}
\item
  \begin{quote}
  To develop a comprehensive and unified understanding of the
  subclassical graph of calculi.
  \end{quote}
\item
  \begin{quote}
  To identify and classify the different types of non-classical logics
  that exist.
  \end{quote}
\item
  \begin{quote}
  To construct a subclassical graph of calculi that visually represents
  the relationships between these logics.
  \end{quote}
\item
  \begin{quote}
  To analyze the subclassical graph of calculi to identify common
  patterns and themes.
  \end{quote}
\item
  \begin{quote}
  To develop a comprehensive and unified understanding of the
  subclassical graph of calculi.
  \end{quote}
\item
  \begin{quote}
  To develop a new formal framework for representing and reasoning about
  subclassical logics.
  \end{quote}
\item
  \begin{quote}
  To develop a new classification of subclassical logics.
  \end{quote}
\item
  \begin{quote}
  To identify new relationships between existing subclassical logics.
  \end{quote}
\item
  \begin{quote}
  To develop a new set of tools for working with subclassical logics.
  \end{quote}
\end{itemize}

The objectives of the project are to:

\begin{itemize}
\item
  \begin{quote}
  \textbf{Identify and classify the different types of subclassical
  logics}. This will involve analyzing the axioms, rules, and semantics
  of different logics to determine their similarities and differences.
  \end{quote}
\item
  \begin{quote}
  \textbf{Construct the subclassical graph of calculi}. This will
  involve representing different subclassical logics as nodes in a graph
  and connecting them with edges that represent the relationships
  between them.
  \end{quote}
\item
  \begin{quote}
  \textbf{Analyze the subclassical graph of calculi to identify patterns
  and relationships}. This will involve using graph theory and other
  mathematical techniques to identify common structures and trends in
  the graph.
  \end{quote}
\item
  \begin{quote}
  \textbf{Develop new algorithms for reasoning about subclassical
  logics}. This will involve using the insights gained from the
  subclassical graph of calculi to design new algorithms that are more
  efficient and accurate for reasoning about subclassical logics.
  \end{quote}
\item
  \begin{quote}
  \textbf{Apply subclassical graphs to develop new automated reasoning
  systems}. This will involve using subclassical graphs to design new
  inference rules and strategies for automated reasoning.
  \end{quote}
\item
  \begin{quote}
  \textbf{Investigate the application of subclassical logics to new
  areas}. This will involve exploring the use of subclassical logics in
  areas such as artificial intelligence, computer science, and
  linguistics.
  \end{quote}
\end{itemize}

The measurements of the project are to:

\begin{itemize}
\item
  \begin{quote}
  \textbf{The number of subclassical logics that are identified and
  classified}.
  \end{quote}
\item
  \begin{quote}
  \textbf{The size and complexity of the subclassical graph of calculi}.
  \end{quote}
\item
  \begin{quote}
  \textbf{The number and significance of the patterns and relationships
  that are identified in the subclassical graph of calculi}.
  \end{quote}
\item
  \begin{quote}
  \textbf{The efficiency and accuracy of the new algorithms for
  reasoning about subclassical logics}.
  \end{quote}
\item
  \begin{quote}
  \textbf{The effectiveness of the new automated reasoning systems that
  are developed}.
  \end{quote}
\item
  \begin{quote}
  \textbf{The impact of the project on the development of new
  applications for subclassical logics}.
  \end{quote}
\end{itemize}

The outcomes of the project are to:

\begin{itemize}
\item
  \begin{quote}
  \textbf{A comprehensive and unified understanding of the subclassical
  graph of calculi}.
  \end{quote}
\item
  \begin{quote}
  \textbf{A new classification of subclassical logics}.
  \end{quote}
\item
  \begin{quote}
  \textbf{A new set of relationships between existing subclassical
  logics}.
  \end{quote}
\item
  \begin{quote}
  \textbf{New algorithms for reasoning about subclassical logics}.
  \end{quote}
\item
  \begin{quote}
  \textbf{New automated reasoning systems for subclassical logics}.
  \end{quote}
\item
  \begin{quote}
  \textbf{New applications for subclassical logics}.
  \end{quote}
\end{itemize}

\hypertarget{approachmethodology-v1}{%
\subsection{Approach/Methodology v1}\label{approachmethodology-v1}}

\textbf{1.}

\textbf{Timeline:}

The project is expected to be completed within a timeframe of
approximately six months, with the following breakdown:

\begin{itemize}
\item
  \begin{quote}
  Literature Review and Conceptual Framework Development: 1-2 months
  \end{quote}
\item
  \begin{quote}
  Metamathematical Analysis: 2-3 months
  \end{quote}
\item
  \begin{quote}
  Metalinguistic Investigation: 1-2 months
  \end{quote}
\item
  \begin{quote}
  Comparative Analysis and Conclusion: 1 month
  \end{quote}
\end{itemize}

Regular progress reports will be generated and shared to ensure
adherence to the timeline and identify any potential challenges or
roadblocks.

\hypertarget{approachmethodology-v2}{%
\subsection{Approach/Methodology v2}\label{approachmethodology-v2}}

This project will be carried out using a deductive approach, focusing
exclusively on metamathematical and metalinguistic methods. The
following activities are proposed to meet the goals and objectives of
the project:

\textbf{Activity 1: Literature Review}

A comprehensive literature review will be conducted to identify and
examine existing work on subclassical calculi, particularly in the areas
of metamathematics and metalinguistics. This will involve reviewing
relevant academic papers, books, and other resources.

\textbf{Activity 2: Graph-Theoretic Representation}

Subclassical calculi will be represented using graph theory. This will
involve defining the nodes and edges of the graphs, as well as the
relationships between them. The graph-theoretic representation will
allow for a visual and formal analysis of the calculi.

\textbf{Activity 3: Comparative Analysis}

The subclassical calculi will be compared and contrasted using the
graph-theoretic representation. This will involve identifying
similarities and differences between the calculi, as well as exploring
their relationships.

\textbf{Activity 4: Metamathematical and Metalinguistic Analysis}

The graph-theoretic representation will be used to carry out a
metamathematical and metalinguistic analysis of the subclassical
calculi. This will involve examining the properties of the calculi, as
well as their expressiveness and limitations.

\textbf{Activity 5: Synthesis and Conclusions}

The findings of the research will be synthesized and conclusions will be
drawn about the subclassical calculi. This will involve identifying key
insights and implications of the research, as well as proposing
directions for future research.

\textbf{Timeline}

The following timeline is proposed for the project:

\begin{itemize}
\item
  \begin{quote}
  \textbf{Month 1-3:} Literature review
  \end{quote}
\item
  \begin{quote}
  \textbf{Month 4-6:} Graph-theoretic representation
  \end{quote}
\item
  \begin{quote}
  \textbf{Month 7-9:} Comparative analysis
  \end{quote}
\item
  \begin{quote}
  \textbf{Month 10-12:} Metamathematical and metalinguistic analysis
  \end{quote}
\item
  \begin{quote}
  \textbf{Month 13-14:} Synthesis and conclusions
  \end{quote}
\end{itemize}

\textbf{Resources}

The following resources will be used to carry out the project:

\begin{itemize}
\item
  \begin{quote}
  Academic papers
  \end{quote}
\item
  \begin{quote}
  Books
  \end{quote}
\item
  \begin{quote}
  Online resources
  \end{quote}
\item
  \begin{quote}
  Graph theory software
  \end{quote}
\end{itemize}

\textbf{Evaluation}

The success of the project will be evaluated based on the following
criteria:

\begin{itemize}
\item
  \begin{quote}
  The comprehensiveness of the literature review
  \end{quote}
\item
  \begin{quote}
  The accuracy of the graph-theoretic representation
  \end{quote}
\item
  \begin{quote}
  The depth of the comparative analysis
  \end{quote}
\item
  \begin{quote}
  The rigor of the metamathematical and metalinguistic analysis
  \end{quote}
\item
  \begin{quote}
  The significance of the synthesis and conclusions
  \end{quote}
\end{itemize}

\textbf{Dissemination}

The findings of the research will be disseminated through a variety of
channels, including:

\begin{itemize}
\item
  \begin{quote}
  Academic papers
  \end{quote}
\item
  \begin{quote}
  Conference presentations
  \end{quote}
\item
  \begin{quote}
  Online publications
  \end{quote}
\end{itemize}

\hypertarget{approachmethodology-v3}{%
\subsection{Approach/Methodology v3}\label{approachmethodology-v3}}

\textbf{How are you going to carry out your project?}

The research project will be carried out using a deductive approach,
focusing exclusively on metamathematical and metalinguistic methods.
This means that the research will focus on the formal properties of
subclassical calculi, rather than their computational or semantic
properties.

The specific activities that will be carried out to meet the goals and
objectives of the project are as follows:

\begin{enumerate}
\def\labelenumi{\arabic{enumi}.}
\item
  \begin{quote}
  \textbf{Conduct a literature review} of existing work on subclassical
  calculi. This will involve identifying and reviewing relevant papers,
  books, and other resources.
  \end{quote}
\item
  \begin{quote}
  \textbf{Develop a formal framework} for representing subclassical
  calculi. This framework will be used to represent the syntax and
  semantics of subclassical calculi in a precise and unambiguous way.
  \end{quote}
\item
  \begin{quote}
  \textbf{Use the formal framework} to investigate the metamathematical
  properties of subclassical calculi. This will involve proving theorems
  about the expressiveness, decidability, and complexity of subclassical
  calculi.
  \end{quote}
\item
  \begin{quote}
  \textbf{Develop a metalinguistic approach} to reasoning about
  subclassical calculi. This will involve developing techniques for
  using metamathematical concepts to reason about the properties of
  subclassical calculi.
  \end{quote}
\end{enumerate}

\textbf{What specific activities do you propose to meet the goals and
objectives you have outlined, and how will those activities be carried
out?}

The specific activities that will be carried out to meet the goals and
objectives of the project are as follows:

\begin{itemize}
\item
  \begin{quote}
  \textbf{Goal 1:} To identify the key features of subclassical calculi.
  \end{quote}

  \begin{itemize}
  \item
    \begin{quote}
    \textbf{Activity 1.1:} Conduct a literature review of existing work
    on subclassical calculi.
    \end{quote}
  \item
    \begin{quote}
    \textbf{Activity 1.2:} Develop a formal framework for representing
    subclassical calculi.
    \end{quote}
  \item
    \begin{quote}
    \textbf{Activity 1.3:} Use the formal framework to identify the key
    features of subclassical calculi.
    \end{quote}
  \end{itemize}
\item
  \begin{quote}
  \textbf{Goal 2:} To investigate the metamathematical properties of
  subclassical calculi.
  \end{quote}

  \begin{itemize}
  \item
    \begin{quote}
    \textbf{Activity 2.1:} Use the formal framework to prove theorems
    about the expressiveness of subclassical calculi.
    \end{quote}
  \item
    \begin{quote}
    \textbf{Activity 2.2:} Use the formal framework to prove theorems
    about the decidability of subclassical calculi.
    \end{quote}
  \item
    \begin{quote}
    \textbf{Activity 2.3:} Use the formal framework to prove theorems
    about the complexity of subclassical calculi.
    \end{quote}
  \end{itemize}
\item
  \begin{quote}
  \textbf{Goal 3:} To develop a metalinguistic approach to reasoning
  about subclassical calculi.
  \end{quote}

  \begin{itemize}
  \item
    \begin{quote}
    \textbf{Activity 3.1:} Develop techniques for using metamathematical
    concepts to reason about the properties of subclassical calculi.
    \end{quote}
  \item
    \begin{quote}
    \textbf{Activity 3.2:} Apply the metalinguistic approach to
    reasoning about specific examples of subclassical calculi.
    \end{quote}
  \end{itemize}
\end{itemize}

\textbf{How will those activities be carried out?}

The activities will be carried out using a variety of methods,
including:

\begin{itemize}
\item
  \begin{quote}
  \textbf{Literature review:} The literature review will be conducted
  using a variety of sources, including academic journals, books, and
  online resources.
  \end{quote}
\item
  \begin{quote}
  \textbf{Formalization:} The formal framework for representing
  subclassical calculi will be developed using a variety of formal
  languages, including logic programming languages and proof assistants.
  \end{quote}
\item
  \begin{quote}
  \textbf{Theorem proving:} The theorems about the expressiveness,
  decidability, and complexity of subclassical calculi will be proved
  using a variety of techniques, including induction, model theory, and
  proof theory.
  \end{quote}
\item
  \begin{quote}
  \textbf{Metalinguistic analysis:} The metalinguistic approach to
  reasoning about subclassical calculi will be developed using a variety
  of metamathematical concepts, including reflection, abstraction, and
  self-reference.
  \end{quote}
\end{itemize}

\textbf{Timeline}

The research project is expected to take approximately 2 years to
complete. The following is a tentative timeline for the project:

\begin{itemize}
\item
  \begin{quote}
  \textbf{Year 1:\\
  }
  \end{quote}

  \begin{itemize}
  \item
    \begin{quote}
    Conduct literature review
    \end{quote}
  \item
    \begin{quote}
    Develop formal framework
    \end{quote}
  \end{itemize}
\item
  \begin{quote}
  \textbf{Year 2:\\
  }
  \end{quote}

  \begin{itemize}
  \item
    \begin{quote}
    Investigate metamathematical properties
    \end{quote}
  \item
    \begin{quote}
    Develop metalinguistic approach
    \end{quote}
  \item
    \begin{quote}
    Write dissertation
    \end{quote}
  \end{itemize}
\end{itemize}

The timeline is subject to change, and the actual time required to
complete the project may vary.

\textbf{Resources}

The research project will require a variety of resources, including:

\begin{itemize}
\item
  \begin{quote}
  \textbf{Access to academic journals and books:} The researcher will
  need access to a variety of academic journals and books in order to
  conduct the literature review.
  \end{quote}
\item
  \begin{quote}
  \textbf{Computing resources:} The researcher will need access to
  computing resources in order to develop the formal framework and prove
  theorems.
  \end{quote}
\item
  \begin{quote}
  \textbf{Travel funds:} The researcher may need to travel to
  conferences and workshops in order to present their work and
  collaborate with other researchers.
  \end{quote}
\end{itemize}

The researcher will apply for funding to support the project. The
funding will be used to cover the cost of resources and travel.

\hypertarget{timeline}{%
\subsection{Timeline}\label{timeline}}

\begin{itemize}
\item
  \begin{quote}
  Identify and classify the non-classical logics: 6 months.
  \end{quote}
\item
  \begin{quote}
  Construct the subclassical graph of calculi: 3 months.
  \end{quote}
\item
  \begin{quote}
  Analyze the subclassical graph of calculi: 2 months.
  \end{quote}
\item
  \begin{quote}
  Develop a new formal framework for representing and reasoning about
  non-classical logics: 4 months.
  \end{quote}
\item
  \begin{quote}
  Develop a new classification of non-classical logics: 3 months.
  \end{quote}
\item
  \begin{quote}
  Identify new relationships between existing non-classical logics: 2
  months.
  \end{quote}
\item
  \begin{quote}
  Develop a new set of tools for working with non-classical logics: 5
  months.
  \end{quote}
\end{itemize}

\hypertarget{outcomes-benefits-results-v1}{%
\subsection{Outcomes, Benefits, Results
v1}\label{outcomes-benefits-results-v1}}

\hypertarget{outcomes}{%
\subsubsection{Outcomes}\label{outcomes}}

The proposed research project aims to develop a subclassical graph of
calculi, a novel approach to formalizing and analyzing the relationships
between different logical calculi. The subclassical graph will represent
logical calculi as nodes and the relationships between them as edges,
where the edges capture the logical implications and syntactic
similarities between different systems.

The project is expected to have the following outcomes:

\begin{itemize}
\item
  \begin{quote}
  \textbf{Comprehensive and systematic framework for understanding the
  interplay of various logical systems}. The subclassical graph will
  provide a structured and interconnected representation of various
  subclassical systems. This graphical representation will facilitate
  the analysis of logical relationships, the identification of logical
  connections, and the exploration of new logical systems.
  \end{quote}
\item
  \begin{quote}
  \textbf{Improved understanding of calculi relationships}. The
  subclassical graph will serve as a valuable tool for visualizing and
  understanding the intricate relationships between different calculi,
  facilitating the discovery of new connections and patterns.
  \end{quote}
\item
  \begin{quote}
  \textbf{New techniques for interoperating and comparing calculi}. The
  subclassical graph will provide a foundation for developing new
  techniques for interoperating and comparing calculi, enabling the
  transfer of knowledge and tools across different formal systems.
  \end{quote}
\item
  \begin{quote}
  \textbf{Benchmark for evaluating the expressiveness and computational
  power of different calculi}. The subclassical graph will serve as a
  benchmark for evaluating the expressiveness and computational power of
  different calculi, aiding in the selection of appropriate formal
  systems for specific applications.
  \end{quote}
\end{itemize}

\textbf{Benefits}

The proposed research project is expected to have the following
benefits:

\begin{itemize}
\item
  \begin{quote}
  \textbf{Advance our understanding of subclassical logic and its
  applications}. The development of a formal framework for subclassical
  graphs will provide a valuable tool for researchers and practitioners
  alike, facilitating the analysis, manipulation, and application of
  these logics.
  \end{quote}
\item
  \begin{quote}
  \textbf{Develop novel logical calculi with tailored properties for
  specific applications}. The subclassical graph will serve as a
  generative framework for exploring new logical systems and their
  properties.
  \end{quote}
\item
  \begin{quote}
  \textbf{Contribute to the development of new artificial intelligence
  applications}. The subclassical graph can be used to develop new
  artificial intelligence applications that can reason under uncertainty
  and handle vagueness.
  \end{quote}
\item
  \begin{quote}
  \textbf{Enhance our understanding of computation and formal logic}.
  The subclassical graph can be used to develop new insights into the
  relationships between different formal systems and their computational
  properties.
  \end{quote}
\end{itemize}

\textbf{Results}

The proposed research project is expected to have the following results:

\begin{itemize}
\item
  \begin{quote}
  \textbf{A formalization of subclassical graphs}. The project will
  propose a rigorous formalization of subclassical graphs, establishing
  a precise mathematical foundation for their analysis and manipulation.
  \end{quote}
\item
  \begin{quote}
  \textbf{Algorithmic techniques for constructing and manipulating
  subclassical graphs}. Novel algorithmic techniques will be developed
  to efficiently construct and manipulate subclassical graphs, enabling
  the exploration of large and complex systems.
  \end{quote}
\item
  \begin{quote}
  \textbf{Applications of subclassical graphs in automated reasoning}.
  The project will investigate the application of subclassical graphs in
  automated reasoning systems, potentially leading to more powerful and
  versatile inference mechanisms.
  \end{quote}
\end{itemize}

\hypertarget{impact}{%
\subsubsection{Impact}\label{impact}}

The proposed research project has the potential to have a significant
impact on the field of logic and its applications. The development of a
formal framework for subclassical graphs will provide a valuable tool
for researchers and practitioners alike, facilitating the analysis,
manipulation, and application of these logics. Moreover, the
project\textquotesingle s exploration of algorithmic techniques and
applications in automated reasoning is expected to yield tangible
benefits in various domains, including artificial intelligence, computer
science, and decision-making systems.

\hypertarget{measurement}{%
\subsubsection{Measurement}\label{measurement}}

The success of the proposed research project will be measured by the
following criteria:

\begin{itemize}
\item
  \begin{quote}
  \textbf{The quality of the formalization of subclassical graphs}. The
  formalization should be rigorous, precise, and well-founded.
  \end{quote}
\item
  \begin{quote}
  \textbf{The efficiency and effectiveness of the algorithmic
  techniques}. The algorithms should be efficient in terms of
  computational time and memory usage, and they should be able to handle
  large and complex subclassical graphs.
  \end{quote}
\item
  \begin{quote}
  \textbf{The impact of the applications of subclassical graphs}. The
  applications should demonstrate the usefulness of subclassical graphs
  in solving real-world problems.
  \end{quote}
\end{itemize}

\hypertarget{products}{%
\subsubsection{Products}\label{products}}

The proposed research project will produce the following products:

\begin{itemize}
\item
  \begin{quote}
  \textbf{A formalization of subclassical graphs}. The formalization
  will be published in a peer-reviewed journal or conference
  proceedings.
  \end{quote}
\item
  \begin{quote}
  \textbf{Algorithmic techniques for constructing and manipulating
  subclassical graphs}. The algorithms will be implemented in a software
  library and made available for public use.
  \end{quote}
\item
  \begin{quote}
  \textbf{Applications of subclassical graphs in automated reasoning}.
  The applications will be demonstrated in a series of research papers.
  \end{quote}
\end{itemize}

\hypertarget{outcomes-benefits-results-v2}{%
\subsection{Outcomes, Benefits, Results
v2}\label{outcomes-benefits-results-v2}}

\hypertarget{the-project-will-deliver-the-following-outcomes.}{%
\subsubsection{The project will deliver the following
outcomes.}\label{the-project-will-deliver-the-following-outcomes.}}

\begin{itemize}
\item
  \begin{quote}
  A comprehensive list of non-classical logics, including their axioms,
  rules, and properties.
  \end{quote}
\item
  \begin{quote}
  A new formal framework for representing and reasoning about
  subclassical logics.
  \end{quote}
\item
  \begin{quote}
  A new classification of subclassical logics.
  \end{quote}
\item
  \begin{quote}
  A subclassical graph of calculi that visually represents the
  relationships between these logics.
  \end{quote}
\item
  \begin{quote}
  A list of common patterns and themes in the properties of
  non-classical logics.
  \end{quote}
\item
  \begin{quote}
  A list of new areas of research suggested by the analysis of the
  subclassical graph of calculi.
  \end{quote}
\end{itemize}

The project will also have the following benefits.

\begin{itemize}
\item
  \begin{quote}
  It will provide a better understanding of the logical relationships
  between different subclassical calculi.
  \end{quote}
\item
  \begin{quote}
  It will facilitate the development of new and more powerful
  subclassical calculi.
  \end{quote}
\item
  \begin{quote}
  It will enable the application of subclassical calculi to new areas.
  \end{quote}
\end{itemize}

The project will also produce the following results.

\begin{itemize}
\item
  \begin{quote}
  A new formal framework for representing and reasoning about
  subclassical logics.
  \end{quote}
\item
  \begin{quote}
  A new classification of subclassical logics.
  \end{quote}
\item
  \begin{quote}
  A subclassical graph of calculi that visually represents the
  relationships between these logics.
  \end{quote}
\item
  \begin{quote}
  A list of common patterns and themes in the properties of
  non-classical logics.
  \end{quote}
\item
  \begin{quote}
  A list of new areas of research suggested by the analysis of the
  subclassical graph of calculi.
  \end{quote}
\end{itemize}

\hypertarget{outcomes-benefits-results-v3}{%
\subsection{Outcomes, Benefits, Results
v3}\label{outcomes-benefits-results-v3}}

\hypertarget{outcomes-1}{%
\subsubsection{Outcomes}\label{outcomes-1}}

The development of the subclassical graph of calculi will have
significant implications for research in logic and its applications. It
will provide a comprehensive and systematic framework for understanding
the interplay of various logical systems, enabling researchers to:

\begin{itemize}
\item
  \begin{quote}
  Identify connections between seemingly disparate calculi.
  \end{quote}
\item
  \begin{quote}
  Uncover hidden relationships and patterns in the landscape of logical
  systems.
  \end{quote}
\item
  \begin{quote}
  Explore new avenues of research by traversing the subclassical graph.
  \end{quote}
\item
  \begin{quote}
  Develop novel logical calculi with tailored properties for specific
  applications.
  \end{quote}
\end{itemize}

\textbf{Benefits}

The development of the subclassical graph of calculi will provide
several benefits to the field of logic and computation. It will:

\begin{itemize}
\item
  \begin{quote}
  Facilitate the understanding and comparison of different logical
  systems.
  \end{quote}
\item
  \begin{quote}
  Aid in the development of new logical calculi.
  \end{quote}
\item
  \begin{quote}
  Enhance the application of logical calculi in various fields, such as
  artificial intelligence, computer science, and mathematics.
  \end{quote}
\end{itemize}

\textbf{Results}

The successful development of the subclassical graph of calculi will
result in several key outcomes:

\begin{itemize}
\item
  \begin{quote}
  A comprehensive and systematic framework for understanding the
  relationships between logical calculi.
  \end{quote}
\item
  \begin{quote}
  A set of tools for analyzing and visualizing the relationships between
  logical calculi.
  \end{quote}
\item
  \begin{quote}
  A collection of novel logical calculi with tailored properties for
  specific applications.
  \end{quote}
\end{itemize}

\hypertarget{impact-1}{%
\subsubsection{Impact}\label{impact-1}}

The development of the subclassical graph of calculi will have a
significant impact on the field of logic and its applications. It will:

\begin{itemize}
\item
  \begin{quote}
  Revolutionize our understanding of logical systems and their
  applications.
  \end{quote}
\item
  \begin{quote}
  Lead to advancements in mathematics, computer science, and philosophy.
  \end{quote}
\item
  \begin{quote}
  Enable the development of new and more powerful tools for reasoning
  and computation.
  \end{quote}
\end{itemize}

\hypertarget{measurement-1}{%
\subsubsection{Measurement}\label{measurement-1}}

The impact of the subclassical graph of calculi will be measured by:

\begin{itemize}
\item
  \begin{quote}
  The number of citations of the research papers that describe the
  development and application of the subclassical graph of calculi.
  \end{quote}
\item
  \begin{quote}
  The number of new logical calculi that are developed using the
  subclassical graph of calculi.
  \end{quote}
\item
  \begin{quote}
  The number of new applications of logical calculi that are developed
  using the subclassical graph of calculi.
  \end{quote}
\end{itemize}

\hypertarget{products-1}{%
\subsubsection{Products}\label{products-1}}

The products of this research will include:

\begin{itemize}
\item
  \begin{quote}
  A rigorous formalization of subclassical graphs.
  \end{quote}
\item
  \begin{quote}
  A set of algorithms for constructing and manipulating subclassical
  graphs.
  \end{quote}
\item
  \begin{quote}
  A collection of novel logical calculi that are developed using the
  subclassical graph of calculi.
  \end{quote}
\item
  \begin{quote}
  A series of research papers that describe the development and
  application of the subclassical graph of calculi.
  \end{quote}
\end{itemize}

\hypertarget{outcomes-benefits-results-v4}{%
\subsection{Outcomes, Benefits, Results
v4}\label{outcomes-benefits-results-v4}}

\hypertarget{outcomes-2}{%
\subsubsection{Outcomes}\label{outcomes-2}}

The proposed research project aims to develop a subclassical graph of
calculi, a novel approach to formalizing and analyzing the relationships
between different logical calculi. The subclassical graph will provide a
comprehensive and systematic framework for understanding the interplay
of various logical systems, enabling researchers to:

\begin{itemize}
\item
  \begin{quote}
  Identify connections between seemingly disparate calculi.
  \end{quote}
\item
  \begin{quote}
  Uncover hidden relationships and patterns in the landscape of logical
  systems.
  \end{quote}
\item
  \begin{quote}
  Explore new avenues of research by traversing the subclassical graph.
  \end{quote}
\item
  \begin{quote}
  Develop novel logical calculi with tailored properties for specific
  applications.
  \end{quote}
\end{itemize}

\textbf{Benefits}

The development of the subclassical graph of calculi will have
significant implications for research in logic and its applications. It
will provide a valuable tool for researchers and practitioners alike,
facilitating the analysis, manipulation, and application of these
logics. Moreover, the project\textquotesingle s exploration of
algorithmic techniques and applications in automated reasoning is
expected to yield tangible benefits in various domains, including
artificial intelligence, computer science, and decision-making systems.

\textbf{Results}

The project will produce a number of significant results, including:

\begin{itemize}
\item
  \begin{quote}
  A rigorous formalization of subclassical graphs.
  \end{quote}
\item
  \begin{quote}
  Novel algorithmic techniques for constructing and manipulating
  subclassical graphs.
  \end{quote}
\item
  \begin{quote}
  A comprehensive survey of existing subclassical logics and their
  relationships.
  \end{quote}
\item
  \begin{quote}
  The identification of new connections and patterns in the landscape of
  logical systems.
  \end{quote}
\item
  \begin{quote}
  The development of novel logical calculi with tailored properties for
  specific applications.
  \end{quote}
\end{itemize}

\hypertarget{impact-2}{%
\subsubsection{Impact}\label{impact-2}}

The proposed research has the potential to revolutionize our
understanding of logical systems and their applications. It will provide
a valuable tool for researchers and practitioners alike, facilitating
the analysis, manipulation, and application of these logics. Moreover,
the project\textquotesingle s exploration of algorithmic techniques and
applications in automated reasoning is expected to yield tangible
benefits in various domains, including artificial intelligence, computer
science, and decision-making systems.

\hypertarget{measurements}{%
\subsubsection{Measurements}\label{measurements}}

The success of the project will be measured by the following criteria:

\begin{itemize}
\item
  \begin{quote}
  The quality of the subclassical graph of calculi, as determined by
  experts in the field of logic.
  \end{quote}
\item
  \begin{quote}
  The effectiveness of the algorithmic techniques for constructing and
  manipulating subclassical graphs.
  \end{quote}
\item
  \begin{quote}
  The completeness and accuracy of the survey of existing subclassical
  logics and their relationships.
  \end{quote}
\item
  \begin{quote}
  The number of new connections and patterns identified in the landscape
  of logical systems.
  \end{quote}
\item
  \begin{quote}
  The development of novel logical calculi with tailored properties for
  specific applications.
  \end{quote}
\end{itemize}

\hypertarget{products-2}{%
\subsubsection{Products}\label{products-2}}

The project will produce the following products:

\begin{itemize}
\item
  \begin{quote}
  A rigorous formalization of subclassical graphs.
  \end{quote}
\item
  \begin{quote}
  Novel algorithmic techniques for constructing and manipulating
  subclassical graphs.
  \end{quote}
\item
  \begin{quote}
  A comprehensive survey of existing subclassical logics and their
  relationships.
  \end{quote}
\item
  \begin{quote}
  A number of research papers published in peer-reviewed journals.
  \end{quote}
\item
  \begin{quote}
  A presentation at a major international conference in logic.
  \end{quote}
\item
  \begin{quote}
  A software tool for visualizing and analyzing subclassical graphs.
  \end{quote}
\end{itemize}

\hypertarget{project-directorprincipal-investigator-and-staff-v1}{%
\subsection{Project Director/Principal Investigator and Staff
v1}\label{project-directorprincipal-investigator-and-staff-v1}}

\textbf{Qualifications and Experience of the Proposed Project
Director/Principal Investigator}

The Project Director/Principal Investigator (PD/PI) should have a strong
background in subclassical logic and experience in leading research
projects. The ideal candidate will have a Ph.D. in a relevant field,
such as mathematics, computer science, or logic, and a proven track
record of publishing research papers in top-tier journals. The PD/PI
should also have experience in collaborating with other researchers and
in mentoring students.

\textbf{Qualifications and Experience of Key Project Staff}

The key project staff should have expertise in subclassical logic and
related areas, such as proof theory, model theory, and computer science.
The ideal candidates will have Ph.D.s in relevant fields and experience
in conducting research on subclassical logic. The key project staff
should also be able to work independently and as part of a team.

\textbf{Minimum Ideal Qualifications for the Principal Investigator}

The minimum ideal qualifications for the PI are a
master\textquotesingle s degree in a relevant field, such as
mathematics, computer science, or logic, and a strong track record of
research in subclassical logic. The PI should also have experience in
leading research projects and in mentoring students.

\textbf{Ideal Project Director or Project Directors}

The ideal project director or project directors would be determined by
the relevant International learned societies and unions. These
organizations would be able to identify individuals with the expertise
and experience necessary to lead this research project.

\textbf{Additional Considerations}

In addition to the qualifications and experience listed above, the PD/PI
and key project staff should also have the following qualities:

\begin{itemize}
\item
  \begin{quote}
  Strong communication and interpersonal skills
  \end{quote}
\item
  \begin{quote}
  Ability to think creatively and solve problems
  \end{quote}
\item
  \begin{quote}
  Ability to work independently and as part of a team
  \end{quote}
\item
  \begin{quote}
  Commitment to excellence in research
  \end{quote}
\end{itemize}

We are confident that with a strong team of researchers in place, this
project will make significant contributions to our understanding of
subclassical logic.

\hypertarget{project-directorprincipal-investigator-and-staff-v2}{%
\subsection{Project Director/Principal Investigator and Staff
v2}\label{project-directorprincipal-investigator-and-staff-v2}}

\hypertarget{principal-investigator}{%
\subsubsection{Principal Investigator}\label{principal-investigator}}

\textbf{Qualifications and Experience of the Proposed Project
Director/Principal Investigator}

\begin{itemize}
\item
  \begin{quote}
  Master of Science in Mathematics from {[}University Name{]},
  {[}Year{]}
  \end{quote}
\item
  \begin{quote}
  Doctor of Philosophy in Mathematics from {[}University Name{]},
  {[}Year{]}
  \end{quote}
\item
  \begin{quote}
  Postdoctoral Fellowship in Mathematics at {[}University Name{]},
  {[}Year{]} - {[}Year{]}
  \end{quote}
\item
  \begin{quote}
  Research Scientist at {[}Research Institution{]}, {[}Year{]} - Present
  \end{quote}
\item
  \begin{quote}
  Author of over {[}Number{]} peer-reviewed publications in top
  mathematics journals
  \end{quote}
\item
  \begin{quote}
  Extensive experience in the field of subclassical logic
  \end{quote}
\end{itemize}

\textbf{Ideal Qualifications for the Principle Investigator}

\begin{itemize}
\item
  \begin{quote}
  Master\textquotesingle s degree in mathematics, computer science, or
  philosophy
  \end{quote}
\item
  \begin{quote}
  Proven experience in research on subclassical logic
  \end{quote}
\item
  \begin{quote}
  Strong publication record in top academic journals
  \end{quote}
\item
  \begin{quote}
  Excellent communication and collaboration skills
  \end{quote}
\end{itemize}

\textbf{Ideal Selection Process for the Project Director or Project
Directors}

\begin{itemize}
\item
  \begin{quote}
  The ideal project director or project directors would be determined by
  a panel of experts from the relevant international learned societies
  and unions.
  \end{quote}
\item
  \begin{quote}
  The panel would consider the following factors when making its
  decision:
  \end{quote}

  \begin{itemize}
  \item
    \begin{quote}
    The qualifications and experience of the candidates
    \end{quote}
  \item
    \begin{quote}
    The relevance of the candidates\textquotesingle{} research to the
    project
    \end{quote}
  \item
    \begin{quote}
    The candidates\textquotesingle{} ability to lead and manage a
    large-scale research project
    \end{quote}
  \end{itemize}
\end{itemize}

\hypertarget{key-staff}{%
\subsubsection{Key Staff}\label{key-staff}}

\textbf{Qualifications and Experience of Key Project Staff}

\begin{itemize}
\item
  \begin{quote}
  {[}Name{]}, PhD in Mathematics, {[}Year{]}
  \end{quote}

  \begin{itemize}
  \item
    \begin{quote}
    Expertise in intuitionistic logic
    \end{quote}
  \item
    \begin{quote}
    Author of over {[}Number{]} peer-reviewed publications in top
    mathematics journals
    \end{quote}
  \end{itemize}
\item
  \begin{quote}
  {[}Name{]}, PhD in Computer Science, {[}Year{]}
  \end{quote}

  \begin{itemize}
  \item
    \begin{quote}
    Expertise in linear logic
    \end{quote}
  \item
    \begin{quote}
    Author of over {[}Number{]} peer-reviewed publications in top
    computer science journals
    \end{quote}
  \end{itemize}
\item
  \begin{quote}
  {[}Name{]}, PhD in Philosophy, {[}Year{]}
  \end{quote}

  \begin{itemize}
  \item
    \begin{quote}
    Expertise in the philosophy of mathematics
    \end{quote}
  \item
    \begin{quote}
    Author of over {[}Number{]} peer-reviewed publications in top
    philosophy journals
    \end{quote}
  \end{itemize}
\end{itemize}

\textbf{Additional Considerations}

\begin{itemize}
\item
  \begin{quote}
  The project director or project directors should be based at a major
  research university with strong research facilities in mathematics,
  computer science, or philosophy.
  \end{quote}
\item
  \begin{quote}
  The project director or project directors should have access to a
  network of collaborators who are experts in subclassical logic.
  \end{quote}
\end{itemize}

\hypertarget{project-directorprincipal-investigator-and-staff-v3}{%
\subsection{Project Director/Principal Investigator and Staff
v3}\label{project-directorprincipal-investigator-and-staff-v3}}

\hypertarget{project-director}{%
\subsubsection{Project Director}\label{project-director}}

\hypertarget{principal-investigator-1}{%
\subsubsection{Principal Investigator}\label{principal-investigator-1}}

\begin{itemize}
\item
  \begin{quote}
  \textbf{Qualifications:\\
  }
  \end{quote}

  \begin{itemize}
  \item
    \begin{quote}
    Master\textquotesingle s degree in mathematics, computer science, or
    a related field
    \end{quote}
  \item
    \begin{quote}
    5+ years of experience in research on subclassical logics
    \end{quote}
  \item
    \begin{quote}
    Proven track record of publishing in top-tier conferences and
    journals
    \end{quote}
  \item
    \begin{quote}
    Experience in leading and managing research projects
    \end{quote}
  \end{itemize}
\item
  \begin{quote}
  \textbf{Experience:\\
  }
  \end{quote}

  \begin{itemize}
  \item
    \begin{quote}
    Has published extensively on subclassical logics, including
    intuitionistic logic, counter-intuitionistic logic, and linear
    logic.
    \end{quote}
  \item
    \begin{quote}
    Has led several successful research projects on subclassical logics.
    \end{quote}
  \item
    \begin{quote}
    Has served on the program committee for several top-tier conferences
    on subclassical logics.
    \end{quote}
  \end{itemize}
\end{itemize}

\hypertarget{key-project-staff}{%
\subsubsection{Key Project Staff}\label{key-project-staff}}

Ian D.L.N. Mclean, dedicated personal library of relevant research and
several decades of self-motivated and self-driven study.

James Martin, \ldots,

\begin{itemize}
\item
  \begin{quote}
  \textbf{Qualifications:\\
  }
  \end{quote}

  \begin{itemize}
  \item
    \begin{quote}
    Bachelor\textquotesingle s degree in mathematics, computer science,
    or a related field
    \end{quote}
  \item
    \begin{quote}
    2+ years of experience in research on subclassical logics
    \end{quote}
  \item
    \begin{quote}
    Experience in programming and software development
    \end{quote}
  \end{itemize}
\item
  \begin{quote}
  \textbf{Experience:\\
  }
  \end{quote}

  \begin{itemize}
  \item
    \begin{quote}
    Has experience in programming and software development.
    \end{quote}
  \item
    \begin{quote}
    Has experience in conducting research on subclassical logics.
    \end{quote}
  \item
    \begin{quote}
    Has experience in collaborating with other researchers.
    \end{quote}
  \end{itemize}
\end{itemize}

\textbf{Ideal Project Director or Project Directors}

The ideal project director or project directors would be determined by
the relevant International learned societies and unions. These societies
and unions would be able to identify individuals with the necessary
expertise and experience to lead this project to success.

In addition to the qualifications and experience listed above, the ideal
project director or project directors would also have the following
qualities:

\begin{itemize}
\item
  \begin{quote}
  Strong leadership skills
  \end{quote}
\item
  \begin{quote}
  Excellent communication and interpersonal skills
  \end{quote}
\item
  \begin{quote}
  Ability to work independently and as part of a team
  \end{quote}
\item
  \begin{quote}
  Commitment to excellence
  \end{quote}
\end{itemize}

\hypertarget{project-directorprincipal-investigator-and-staff-v4}{%
\subsection{Project Director/Principal Investigator and Staff
v4}\label{project-directorprincipal-investigator-and-staff-v4}}

\textbf{Qualifications}

\begin{itemize}
\item
  \begin{quote}
  Master\textquotesingle s degree in a relevant field, such as
  mathematics, computer science, logic, or philosophy
  \end{quote}
\item
  \begin{quote}
  Proven experience in conducting research in subclassical logic
  \end{quote}
\item
  \begin{quote}
  Strong understanding of graph theory, lattice theory, sequent calculi,
  metalinguistics, metamathematics, formal language theory, constructive
  logic, paraconsistent and paracomplete methods of reasoning,
  foundations of non-classical logic, Sambin\textquotesingle s Basic
  Logic, hypersequent calculi, deviant logics, philosophy of
  mathematics, philosophy of logic, philosophy of language, philosophy
  of paraconsistency, philosophy of paracompleteness, philosophy of
  metamathematics, philosophy of metalinguistics, philosophy of
  semantics, and formal deductive theories
  \end{quote}
\item
  \begin{quote}
  Excellent communication and writing skills
  \end{quote}
\end{itemize}

\textbf{Experience}

\begin{itemize}
\item
  \begin{quote}
  Published at least five peer-reviewed articles in top academic
  journals
  \end{quote}
\item
  \begin{quote}
  Presented research at major conferences
  \end{quote}
\item
  \begin{quote}
  Supervised graduate students
  \end{quote}
\end{itemize}

\textbf{Key Project Staff}

\textbf{Qualifications}

\begin{itemize}
\item
  \begin{quote}
  Bachelor\textquotesingle s degree in a relevant field, such as
  mathematics, computer science, logic, or philosophy
  \end{quote}
\item
  \begin{quote}
  Experience with subclassical logic
  \end{quote}
\item
  \begin{quote}
  Strong understanding of graph theory, lattice theory, sequent calculi,
  metalinguistics, metamathematics, formal language theory, constructive
  logic, paraconsistent and paracomplete methods of reasoning,
  foundations of non-classical logic, Sambin\textquotesingle s Basic
  Logic, hypersequent calculi, deviant logics, philosophy of
  mathematics, philosophy of logic, philosophy of language, philosophy
  of paraconsistency, philosophy of paracompleteness, philosophy of
  metamathematics, philosophy of metalinguistics, philosophy of
  semantics, and formal deductive theories
  \end{quote}
\item
  \begin{quote}
  Excellent communication and writing skills
  \end{quote}
\end{itemize}

\textbf{Experience}

\begin{itemize}
\item
  \begin{quote}
  Conducted research in subclassical logic
  \end{quote}
\item
  \begin{quote}
  Presented research at conferences
  \end{quote}
\item
  \begin{quote}
  Assisted with the preparation of manuscripts for publication
  \end{quote}
\end{itemize}

\textbf{Ideal Project Director or Project Directors}

The ideal project director or project directors would be determined by
the relevant International learned societies and unions. These societies
and unions would have a deep understanding of the field of subclassical
logic and would be able to identify individuals with the necessary
expertise and qualifications to lead the project.

\textbf{Relevant Fields}

\begin{itemize}
\item
  \begin{quote}
  Graph theory
  \end{quote}
\item
  \begin{quote}
  Lattice theory
  \end{quote}
\item
  \begin{quote}
  Sequent calculi
  \end{quote}
\item
  \begin{quote}
  Metalinguistics
  \end{quote}
\item
  \begin{quote}
  Metamathematics
  \end{quote}
\item
  \begin{quote}
  Formal language theory
  \end{quote}
\item
  \begin{quote}
  Constructive logic
  \end{quote}
\item
  \begin{quote}
  Paraconsistent and paracomplete methods of reasoning
  \end{quote}
\item
  \begin{quote}
  Foundations of non-classical logic
  \end{quote}
\item
  \begin{quote}
  Sambin\textquotesingle s Basic Logic
  \end{quote}
\item
  \begin{quote}
  Hypersequent calculi
  \end{quote}
\item
  \begin{quote}
  Deviant logics
  \end{quote}
\item
  \begin{quote}
  Philosophy of mathematics
  \end{quote}
\item
  \begin{quote}
  Philosophy of logic
  \end{quote}
\item
  \begin{quote}
  Philosophy of language
  \end{quote}
\item
  \begin{quote}
  Philosophy of paraconsistency
  \end{quote}
\item
  \begin{quote}
  Philosophy of paracompleteness
  \end{quote}
\item
  \begin{quote}
  Philosophy of metamathematics
  \end{quote}
\item
  \begin{quote}
  Philosophy of metalinguistics
  \end{quote}
\item
  \begin{quote}
  Philosophy of semantics
  \end{quote}
\item
  \begin{quote}
  Formal deductive theories
  \end{quote}
\end{itemize}

\hypertarget{project-directorprincipal-investigator-and-staff-v5}{%
\subsection{Project Director/Principal Investigator and Staff
v5}\label{project-directorprincipal-investigator-and-staff-v5}}

\textbf{Project Director/Principal Investigator}

\begin{itemize}
\item
  \begin{quote}
  \textbf{Qualifications\\
  }
  \end{quote}

  \begin{itemize}
  \item
    \begin{quote}
    PhD in Mathematics, with a focus in graph theory or lattice theory
    \end{quote}
  \item
    \begin{quote}
    Experience in research on subclassical logics
    \end{quote}
  \item
    \begin{quote}
    Strong publication record in relevant journals
    \end{quote}
  \item
    \begin{quote}
    Experience in leading and managing research projects
    \end{quote}
  \end{itemize}
\item
  \begin{quote}
  \textbf{Experience\\
  }
  \end{quote}

  \begin{itemize}
  \item
    \begin{quote}
    10 years of experience as a research scientist
    \end{quote}
  \item
    \begin{quote}
    5 years of experience as a project leader
    \end{quote}
  \item
    \begin{quote}
    Authored or co-authored over 20 peer-reviewed articles on
    subclassical logics
    \end{quote}
  \item
    \begin{quote}
    Presented research at numerous international conferences
    \end{quote}
  \end{itemize}
\end{itemize}

\textbf{Key Project Staff}

\begin{itemize}
\item
  \begin{quote}
  \textbf{Research Scientist\\
  }
  \end{quote}

  \begin{itemize}
  \item
    \begin{quote}
    PhD in Mathematics, with a focus in graph theory or lattice theory
    \end{quote}
  \item
    \begin{quote}
    Experience in research on subclassical logics
    \end{quote}
  \item
    \begin{quote}
    Strong publication record in relevant journals
    \end{quote}
  \end{itemize}
\item
  \begin{quote}
  \textbf{Research Scientist\\
  }
  \end{quote}

  \begin{itemize}
  \item
    \begin{quote}
    PhD in Computer Science, with a focus in formal language theory
    \end{quote}
  \item
    \begin{quote}
    Experience in research on subclassical logics
    \end{quote}
  \item
    \begin{quote}
    Strong programming skills
    \end{quote}
  \end{itemize}
\item
  \begin{quote}
  \textbf{Research Assistant\\
  }
  \end{quote}

  \begin{itemize}
  \item
    \begin{quote}
    Master\textquotesingle s degree in Mathematics or Computer Science
    \end{quote}
  \item
    \begin{quote}
    Experience with graph theory or lattice theory
    \end{quote}
  \item
    \begin{quote}
    Strong programming skills
    \end{quote}
  \end{itemize}
\end{itemize}

\textbf{Ideal Qualifications for the Principal Investigator}

\begin{itemize}
\item
  \begin{quote}
  Master\textquotesingle s degree in the relevant fields and domains
  \end{quote}
\item
  \begin{quote}
  Expertise in graph theory, lattice theory, sequent calculi,
  metalinguistics, metamathematics, formal language theory, constructive
  logic, paraconsistent and paracomplete methods of reasoning,
  foundations of non-classical logic, Sambin\textquotesingle s Basic
  Logic, hypersequent calculi, deviant logics, philosophy of
  mathematics, philosophy of logic, philosophy of language, philosophy
  of paraconsistency, philosophy of paracompleteness, philosophy of
  metamathematics, philosophy of metalinguistics, philosophy of
  semantics, and formal deductive theories
  \end{quote}
\item
  \begin{quote}
  Experience in leading and managing research projects
  \end{quote}
\item
  \begin{quote}
  Strong publication record in relevant journals
  \end{quote}
\end{itemize}

\textbf{Ideal Project Director or Project Directors}

The ideal project director or project directors would be determined by
the relevant International learned societies and unions. These societies
and unions would be able to identify researchers with the necessary
expertise and experience to lead this project.

Please note that these are just ideal qualifications, and it is possible
that a project director or project directors with slightly different
qualifications could be successful in leading this project. The most
important factor is that the project director or project directors have
the necessary expertise and experience to lead the project to a
successful conclusion.

\hypertarget{project-directorprincipal-investigator-and-staff-v6}{%
\subsection{Project Director/Principal Investigator and Staff
v6}\label{project-directorprincipal-investigator-and-staff-v6}}

\textbf{Project Director/Principal Investigator}

\begin{itemize}
\item
  \begin{quote}
  \textbf{Qualifications:\\
  }
  \end{quote}

  \begin{itemize}
  \item
    \begin{quote}
    Master\textquotesingle s degree in mathematics, computer science, or
    a related field
    \end{quote}
  \item
    \begin{quote}
    Extensive experience in graph theory, lattice theory, sequent
    calculi, metalinguistics, metamathematics, formal language theory,
    constructive logic, paraconsistent and paracomplete methods of
    reasoning, foundations of non-classical logic,
    Sambin\textquotesingle s Basic Logic, hypersequent calculi, deviant
    logics, philosophy of mathematics, philosophy of logic, philosophy
    of language, philosophy of paraconsistency, philosophy of
    paracompleteness, philosophy of metamathematics, philosophy of
    metalinguistics, philosophy of semantics, and formal deductive
    theories
    \end{quote}
  \item
    \begin{quote}
    Proven ability to lead and manage research projects
    \end{quote}
  \item
    \begin{quote}
    Strong publication record in top-tier journals and conferences
    \end{quote}
  \end{itemize}
\item
  \begin{quote}
  \textbf{Experience:\\
  }
  \end{quote}

  \begin{itemize}
  \item
    \begin{quote}
    10+ years of experience in research and teaching
    \end{quote}
  \item
    \begin{quote}
    Experience in developing and leading research teams
    \end{quote}
  \item
    \begin{quote}
    Experience in securing and managing research grants
    \end{quote}
  \item
    \begin{quote}
    Experience in mentoring junior researchers
    \end{quote}
  \end{itemize}
\end{itemize}

\textbf{Key Project Staff}

\begin{itemize}
\item
  \begin{quote}
  \textbf{Qualifications:\\
  }
  \end{quote}

  \begin{itemize}
  \item
    \begin{quote}
    Bachelor\textquotesingle s degree in mathematics, computer science,
    or a related field
    \end{quote}
  \item
    \begin{quote}
    Strong background in graph theory, lattice theory, sequent calculi,
    metalinguistics, metamathematics, formal language theory,
    constructive logic, paraconsistent and paracomplete methods of
    reasoning, foundations of non-classical logic,
    Sambin\textquotesingle s Basic Logic, hypersequent calculi, deviant
    logics, philosophy of mathematics, philosophy of logic, philosophy
    of language, philosophy of paraconsistency, philosophy of
    paracompleteness, philosophy of metamathematics, philosophy of
    metalinguistics, philosophy of semantics, and formal deductive
    theories
    \end{quote}
  \item
    \begin{quote}
    Experience in research and teaching
    \end{quote}
  \end{itemize}
\item
  \begin{quote}
  \textbf{Experience:\\
  }
  \end{quote}

  \begin{itemize}
  \item
    \begin{quote}
    5+ years of experience in research
    \end{quote}
  \item
    \begin{quote}
    Experience in working on research teams
    \end{quote}
  \item
    \begin{quote}
    Experience in presenting research at conferences
    \end{quote}
  \end{itemize}
\end{itemize}

\textbf{Ideal Project Director or Project Directors}

The ideal project director or project directors would be determined by
the relevant International learned societies and unions. These societies
and unions would be able to identify researchers with the necessary
qualifications and experience to lead the project.

\textbf{Relevant Fields}

The relevant fields for this project include:

\begin{itemize}
\item
  \begin{quote}
  Graph theory
  \end{quote}
\item
  \begin{quote}
  Lattice theory
  \end{quote}
\item
  \begin{quote}
  Sequent calculi
  \end{quote}
\item
  \begin{quote}
  Metalinguistics
  \end{quote}
\item
  \begin{quote}
  Metamathematics
  \end{quote}
\item
  \begin{quote}
  Formal language theory
  \end{quote}
\item
  \begin{quote}
  Constructive logic
  \end{quote}
\item
  \begin{quote}
  Paraconsistent and paracomplete methods of reasoning
  \end{quote}
\item
  \begin{quote}
  Foundations of non-classical logic
  \end{quote}
\item
  \begin{quote}
  Sambin\textquotesingle s Basic Logic
  \end{quote}
\item
  \begin{quote}
  Hypersequent calculi
  \end{quote}
\item
  \begin{quote}
  Deviant logics
  \end{quote}
\item
  \begin{quote}
  Philosophy of mathematics
  \end{quote}
\item
  \begin{quote}
  Philosophy of logic
  \end{quote}
\item
  \begin{quote}
  Philosophy of language
  \end{quote}
\item
  \begin{quote}
  Philosophy of paraconsistency
  \end{quote}
\item
  \begin{quote}
  Philosophy of paracompleteness
  \end{quote}
\item
  \begin{quote}
  Philosophy of metamathematics
  \end{quote}
\item
  \begin{quote}
  Philosophy of metalinguistics
  \end{quote}
\item
  \begin{quote}
  Philosophy of semantics
  \end{quote}
\item
  \begin{quote}
  Formal deductive theories
  \end{quote}
\end{itemize}

\hypertarget{research-proposal-subclassical-graph-of-calculi}{%
\subsection{Research Proposal: Subclassical Graph of
Calculi}\label{research-proposal-subclassical-graph-of-calculi}}

\textbf{Introduction}

The field of logic has witnessed significant advancements in recent
decades, fueled by the development of powerful formal methods and
automated reasoning techniques. However, the vast and diverse landscape
of logical systems poses challenges in identifying connections and
overarching principles. Existing approaches to classifying and comparing
logical calculi often rely on ad hoc criteria, leading to a fragmented
and incomplete understanding of the relationships between different
systems.

To address these limitations, this research proposal outlines the
development of a subclassical graph of calculi, a novel approach to
formalizing and analyzing the relationships between different logical
calculi. The proposed subclassical graph will provide a comprehensive
and systematic framework for understanding the interplay of various
logical systems, enabling researchers to identify connections, uncover
hidden relationships, generate novel calculi, and explore new avenues of
research.

\textbf{Context and Background}

Logical calculi are formal systems for reasoning and deriving
conclusions from a set of axioms or assumptions or rules such as
structural or inferential rules. They play a fundamental role in
mathematics, computer science, and philosophy, providing a rigorous
foundation for reasoning, proof construction, refutation construction,
modeling, and countermodeling. Over the centuries, a vast array of
logical calculi have been developed, each tailored to specific
applications and embodying distinct logical properties.

\textbf{Current State of Research}

The study of logical calculi has witnessed significant advancements in
recent decades, fueled by the development of powerful formal methods and
automated reasoning techniques. However, the vast and diverse landscape
of logical systems poses challenges in identifying connections and
overarching principles. Existing approaches to classifying and comparing
logical calculi often rely on ad hoc criteria, leading to a fragmented
and incomplete understanding of the relationships between different
systems.

\textbf{Unique and Innovative Approach}

The proposed subclassical graph of calculi addresses these limitations
by providing a unified, systematic, and rigorous framework for analyzing
the relationships between logical calculi. The subclassical graph will
represent logical calculi as nodes and the relationships between them as
edges, where the edges capture the logical implications and syntactic
similarities between different systems.

The primary method of constructing and deducing the subclassical calculi
is applications of symmetry and non-unique dualities between calculi;
this allows for a general method of producing at least four supercalculi
and subcalculi from any one of them and some number of known calculi
such as classical logic (LK), intuitionistic logic (LJ), Dual
Intuitionistic Logic (LDJ), Counter Intuitionistic Logic (CoLJ),
Sambin's Basic, U, and Ordered Multiplicative-Additive Linear Logic
(OMALL).

\textbf{Significance and Contributions}

The development of the subclassical graph of calculi will have
significant implications for research in logic and its applications. It
will provide a comprehensive and systematic framework for understanding
the interplay of various logical systems, enabling researchers to:

\begin{itemize}
\item
  \begin{quote}
  Identify connections between seemingly disparate calculi.
  \end{quote}
\item
  \begin{quote}
  Uncover hidden relationships and patterns in the landscape of logical
  systems.
  \end{quote}
\item
  \begin{quote}
  Explore new avenues of research by traversing the subclassical graph.
  \end{quote}
\item
  \begin{quote}
  Develop novel logical calculi with tailored properties for specific
  applications.
  \end{quote}
\end{itemize}

The subclassical graph of calculi has the potential to revolutionize our
understanding of logical systems and their applications, leading to
advancements in mathematics, computer science, and philosophy.

\textbf{Conclusion}

The proposed research project aims to develop a comprehensive framework
for constructing and analyzing subclassical graphs of calculi, shedding
light on their inherent properties and applications in various fields.
The project will make significant contributions to the field of
subclassical logic and computation by providing a valuable tool for
researchers and practitioners alike, facilitating the analysis,
manipulation, and application of these logics. Moreover, the
project\textquotesingle s exploration of algorithmic techniques and
applications in automated reasoning is expected to yield tangible
benefits in various domains, including artificial intelligence, computer
science, and decision-making systems.

\hypertarget{section}{%
\section{}\label{section}}



\chapter{Sequence A}
}

Output:

--- End Refutation Experiment: Bell Scenario --- --- End Refutation
Experiment: Liar Paradox --- --- Main Execution - Sequence A (No Liar
Paradox First) ---nn--- Bell Scenario Experiment (Confirmation) ---
Relation: Bell Scenario Relation (Non-Local) - ENCODED

--- Hypothesis Test: H\_Bell - Non-Reflexivity under Classical
Interpretation --- Hypothesis (H\_Bell): Classical interpretation
-\textgreater{} bell-scenario-relation is NOT reflexive (threshold 0.5).
Null Hypothesis (H\_Bell'): Classical interpretation -\textgreater{}
bell-scenario-relation IS reflexive (threshold 0.5). Reflexive under
CLASSICAL-INTERPRETATION interpretation? (Threshold: 0.5) EVENT-A is
related to itself: (0.0 QUANTUM-INTERPRETATION 1.0) EVENT-B is related
to itself: (0.0 QUANTUM-INTERPRETATION 1.0) Outcome: Experiment REFUTES
H\_Bell, FAILS to refute H\_Bell'. (Unexpected Classical Reflexivity)

--- Hypothesis Test: H\_Bell\_Quantum - Reflexivity under Quantum
Interpretation --- Hypothesis (H\_Bell\_Quantum): Quantum interpretation
-\textgreater{} bell-scenario-relation IS reflexive (threshold 0.5).
Null Hypothesis (H\_Bell\_Quantum'): Quantum interpretation
-\textgreater{} bell-scenario-relation is NOT reflexive (threshold 0.5).
Reflexive under QUANTUM-INTERPRETATION interpretation? (Threshold: 0.5)
EVENT-A is related to itself: NIL Outcome: Experiment REFUTES
H\_Bell\_Quantum, FAILS to refute H\_Bell\_Quantum'. (Unexpected
Non-Reflexivity under Quantum Interpretation) --- End Bell Scenario
Experiment (Confirmation) --- --- Tarskian Consequence Relation (Local
Example) --- Relation: Tarskian Consequence Relation (Local) - ENCODED
Reflexive under Classical Interpretation? (Threshold 0.5) P is
consequence of itself: (1.0) Q is consequence of itself: (1.0) --- End
Tarskian Consequence Relation (Local Example) --- --- Generalized
Decision Procedure --- Statement: LIAR-STATEMENT Using Encoded Relation:
LIAR-PARADOX-RELATION Evaluating under interpretations:
(CLASSICAL-INTERPRETATION NON-CLASSICAL-INTERPRETATION
PARACONSISTENT-INTERPRETATION DIALETHEIST-INTERPRETATION
CONTEXTUAL-INTERPRETATION-1 CONTEXTUAL-INTERPRETATION-2) Decision under
CLASSICAL-INTERPRETATION Interpretation: (0.0
NON-CLASSICAL-INTERPRETATION 0.8 PARACONSISTENT-INTERPRETATION 1.0
DIALETHEIST-INTERPRETATION 1.0 CONTEXTUAL-INTERPRETATION-1 (LAMBDA
(STATEMENT) (IF (EQ STATEMENT STATEMENT-P) (VECTOR 0.7) (VECTOR 0.0)))
CONTEXTUAL-INTERPRETATION-2 (LAMBDA (STATEMENT) (IF (EQ STATEMENT
STATEMENT-Q) (VECTOR 0.7) (VECTOR 0.0)))) Decision under
NON-CLASSICAL-INTERPRETATION Interpretation: NIL Decision under
PARACONSISTENT-INTERPRETATION Interpretation: NIL Decision under
DIALETHEIST-INTERPRETATION Interpretation: NIL Decision under
CONTEXTUAL-INTERPRETATION-1 Interpretation: NIL Decision under
CONTEXTUAL-INTERPRETATION-2 Interpretation: NIL

--- Non-Singular Outcome Analysis --- Interpretation-Dependent
Decisions: Decisions vary across interpretations. Generalized logical
relations offer interpretation-dependent outcomes. --- Generalized
Decision Procedure --- Statement: EVENT-A Using Encoded Relation:
BELL-SCENARIO-RELATION Evaluating under interpretations:
(CLASSICAL-INTERPRETATION QUANTUM-INTERPRETATION) Decision under
CLASSICAL-INTERPRETATION Interpretation: (0.0 QUANTUM-INTERPRETATION
1.0) Decision under QUANTUM-INTERPRETATION Interpretation: NIL

--- Non-Singular Outcome Analysis --- Interpretation-Dependent
Decisions: Decisions vary across interpretations. Generalized logical
relations offer interpretation-dependent outcomes. --- Generalized
Decision Procedure --- Statement: P Using Encoded Relation:
TARSKIAN-RELATION Evaluating under interpretations:
(CLASSICAL-INTERPRETATION) Decision under CLASSICAL-INTERPRETATION
Interpretation: (1.0)

--- Non-Singular Outcome Analysis --- Interpretation-Dependent
Decisions: Decisions vary across interpretations. Generalized logical
relations offer interpretation-dependent outcomes. --- Refutation
Experiment: Bell Scenario --- Attempting to refute: H\_Bell -
Non-Reflexivity of bell-scenario-relation under
:classical-interpretation Trying to find a :classical-interpretation
where bell-scenario-relation \emph{is} reflexive. Relation: Bell
Scenario Relation (Non-Local) - ENCODED Checking for Reflexivity under a
\emph{modified} :classical-interpretation\ldots{} Reflexive under
REFUTATION-CLASSICAL-INTERPRETATION interpretation? Reflexive under
REFUTATION-CLASSICAL-INTERPRETATION interpretation? (Threshold: 0.5)
EVENT-A is related to itself: NIL Outcome: Refutation Experiment
\emph{FAILS} to refute H\_Bell (under modified
:classical-interpretation). (H\_Bell remains robust) --- Refutation
Experiment: Liar Paradox --- Attempting to refute: H\_Liar\_NonClassical
- Reflexivity of liar-paradox-relation under
:non-classical-interpretation Trying to find a
:non-classical-interpretation where liar-paradox-relation is \emph{not}
reflexive. Relation: Liar Paradox Relation (Non-Local, Self-Referential)
- ENCODED Checking for Reflexivity under a \emph{modified}
:non-classical-interpretation\ldots{} Reflexive under
REFUTATION-NON-CLASSICAL-INTERPRETATION interpretation? Reflexive under
REFUTATION-NON-CLASSICAL-INTERPRETATION interpretation? (Threshold: 0.5)
STATEMENT-P is related to itself: NIL Outcome: Refutation Experiment
\emph{FAILS} to refute H\_Liar\_NonClassical (under modified
:non-classical-interpretation). (H\_Liar\_NonClassical remains robust)
--- Non-Classical Logical Relations Methodology --- Methodology based on
Generalized Logical Relations and Non-Locality: - Hypothesize: Contrast
classical vs.~non-classical relations, graded relatedness, matrix/vector
spaces. - Define: Define local/non-local relations, reflexivity,
symmetry, transitivity, context-dependence, graded relatedness, n-to-m
relations, matrix/vector spaces. - Theoremize: Derive theorems on
non-locality impact on Tarskian properties, non-singular outcomes,
algebraic properties. - Experiment: Test hypotheses/theorems via
paradoxes, decision procedures, refutation experiments, vector spaces,
non-algebraic properties. - Analyze Outcomes: Interpretation-dependent
decisions, non-singular outcomes, refutation robustness, vector spaces,
graded relatedness, algebraic properties. - Refine Theory: Refine theory
based on experiments, considering graded relatedness, refutation
outcomes, mathematical representations, non-algebraic behavior. --- End
Non-Classical Logical Relations Methodology --- n--- n-to-m Relation
Design --- Evaluating n-m-relation with inputs (STATEMENT-X STATEMENT-Y
STATEMENT-Z) and interpretations (CONTEXT-A CONTEXT-B CONTEXT-DEFAULT):
n-m-relation output: \#(0.3 0.1) --- End n-to-m Relation Design ---nn---
Experiment: Testing Non-Transitivity of n-m-relation (Consistency) ---
Relation Output AB (Inputs: (STATEMENT-X STATEMENT-Y), Interpretations:
(CONTEXT-AB CONTEXT-AB)): \#(0.9 0.7), Consistency: 0.9 Relation Output
BC (Inputs: (STATEMENT-Y STATEMENT-Z), Interpretations: (CONTEXT-BC
CONTEXT-BC)): \#(0.9 0.7), Consistency: 0.9 Relation Output AC (Inputs:
(STATEMENT-X STATEMENT-Z), Interpretations: (CONTEXT-AC CONTEXT-AC)):
\#(0.3 0.7), Consistency: 0.3

--- Hypothesis Test: H\_Transitivity - n-m-relation Consistency IS
Transitive --- Hypothesis (H\_Transitivity): High consistency(AB) AND
high consistency(BC) -\textgreater{} high consistency(AC) (threshold
0.7). Null Hypothesis (H\_Transitivity'): n-m-relation Consistency is
NOT Transitive. Outcome: Experiment FAILS to refute H\_Transitivity',
REFUTES H\_Transitivity. (Demonstrates Non-Transitivity) --- End
Experiment: Testing Non-Transitivity of n-m-relation (Consistency) ---
n--- Generalized n-m-relation Example --- Evaluating
generalized-n-m-relation with inputs (STATEMENT-X STATEMENT-Y),
interpretations (CONTEXT-A CONTEXT-B), and logical-relation \#:
generalized-n-m-relation output: \#(0.9 0.7) --- End Generalized
n-m-relation Example ---nn--- End Main Execution - Sequence A ---n\%

\end{document}}*}


\chapter{Sequence B}
}

Output:

--- End Refutation Experiment: Bell Scenario --- --- End Refutation
Experiment: Liar Paradox --- --- Main Execution - Sequence B (Liar
Paradox First) ---nn--- Liar Paradox Experiment (Confirmation) ---
Relation: Liar Paradox Relation (Non-Local, Self-Referential) - ENCODED

--- Hypothesis Test: H\_Liar\_NonClassical - Reflexivity under
Non-Classical Interpretation --- Hypothesis (H\_Liar\_NonClassical):
Non-classical interpretation -\textgreater{} liar-paradox-relation IS
reflexive (threshold 0.5). Null Hypothesis (H\_Liar\_NonClassical'):
Non-classical interpretation -\textgreater{} liar-paradox-relation is
NOT reflexive (threshold 0.5). Reflexive under
NON-CLASSICAL-INTERPRETATION interpretation? (Threshold: 0.5)
STATEMENT-P is related to itself: NIL Outcome: Experiment REFUTES
H\_Liar\_NonClassical, FAILS to refute H\_Liar\_NonClassical'.
(Unexpected Non-Reflexivity under Non-Classical Interpretation)

--- Hypothesis Test: H\_Liar\_Classical - Non-Reflexivity under
Classical Interpretation --- Hypothesis (H\_Liar\_Classical): Classical
interpretation -\textgreater{} liar-paradox-relation is NOT reflexive
(threshold 0.5). Null Hypothesis (H\_Liar\_Classical'): Classical
interpretation -\textgreater{} liar-paradox-relation IS reflexive
(threshold 0.5). Reflexive under CLASSICAL-INTERPRETATION
interpretation? (Threshold: 0.5) STATEMENT-P is related to itself: (0.0
NON-CLASSICAL-INTERPRETATION 0.8 PARACONSISTENT-INTERPRETATION 1.0
DIALETHEIST-INTERPRETATION 1.0 CONTEXTUAL-INTERPRETATION-1 (LAMBDA
(STATEMENT) (IF (EQ STATEMENT STATEMENT-P) (VECTOR 0.7) (VECTOR 0.0)))
CONTEXTUAL-INTERPRETATION-2 (LAMBDA (STATEMENT) (IF (EQ STATEMENT
STATEMENT-Q) (VECTOR 0.7) (VECTOR 0.0)))) STATEMENT-Q is related to
itself: (0.0 NON-CLASSICAL-INTERPRETATION 0.8
PARACONSISTENT-INTERPRETATION 1.0 DIALETHEIST-INTERPRETATION 1.0
CONTEXTUAL-INTERPRETATION-1 (LAMBDA (STATEMENT) (IF (EQ STATEMENT
STATEMENT-P) (VECTOR 0.7) (VECTOR 0.0))) CONTEXTUAL-INTERPRETATION-2
(LAMBDA (STATEMENT) (IF (EQ STATEMENT STATEMENT-Q) (VECTOR 0.7) (VECTOR
0.0)))) Outcome: Experiment REFUTES H\_Liar\_Classical, FAILS to refute
H\_Liar\_Classical'. (Unexpected Classical Reflexivity) --- End Liar
Paradox Experiment (Confirmation) --- --- Bell Scenario Experiment
(Confirmation) --- Relation: Bell Scenario Relation (Non-Local) -
ENCODED

--- Hypothesis Test: H\_Bell - Non-Reflexivity under Classical
Interpretation --- Hypothesis (H\_Bell): Classical interpretation
-\textgreater{} bell-scenario-relation is NOT reflexive (threshold 0.5).
Null Hypothesis (H\_Bell'): Classical interpretation -\textgreater{}
bell-scenario-relation IS reflexive (threshold 0.5). Reflexive under
CLASSICAL-INTERPRETATION interpretation? (Threshold: 0.5) EVENT-A is
related to itself: (0.0 QUANTUM-INTERPRETATION 1.0) EVENT-B is related
to itself: (0.0 QUANTUM-INTERPRETATION 1.0) Outcome: Experiment REFUTES
H\_Bell, FAILS to refute H\_Bell'. (Unexpected Classical Reflexivity)

--- Hypothesis Test: H\_Bell\_Quantum - Reflexivity under Quantum
Interpretation --- Hypothesis (H\_Bell\_Quantum): Quantum interpretation
-\textgreater{} bell-scenario-relation IS reflexive (threshold 0.5).
Null Hypothesis (H\_Bell\_Quantum'): Quantum interpretation
-\textgreater{} bell-scenario-relation is NOT reflexive (threshold 0.5).
Reflexive under QUANTUM-INTERPRETATION interpretation? (Threshold: 0.5)
EVENT-A is related to itself: NIL Outcome: Experiment REFUTES
H\_Bell\_Quantum, FAILS to refute H\_Bell\_Quantum'. (Unexpected
Non-Reflexivity under Quantum Interpretation) --- End Bell Scenario
Experiment (Confirmation) --- --- Tarskian Consequence Relation (Local
Example) --- Relation: Tarskian Consequence Relation (Local) - ENCODED
Reflexive under Classical Interpretation? (Threshold 0.5) P is
consequence of itself: (1.0) Q is consequence of itself: (1.0) --- End
Tarskian Consequence Relation (Local Example) --- --- Generalized
Decision Procedure --- Statement: LIAR-STATEMENT Using Encoded Relation:
LIAR-PARADOX-RELATION Evaluating under interpretations:
(CLASSICAL-INTERPRETATION NON-CLASSICAL-INTERPRETATION
PARACONSISTENT-INTERPRETATION DIALETHEIST-INTERPRETATION
CONTEXTUAL-INTERPRETATION-1 CONTEXTUAL-INTERPRETATION-2) Decision under
CLASSICAL-INTERPRETATION Interpretation: (0.0
NON-CLASSICAL-INTERPRETATION 0.8 PARACONSISTENT-INTERPRETATION 1.0
DIALETHEIST-INTERPRETATION 1.0 CONTEXTUAL-INTERPRETATION-1 (LAMBDA
(STATEMENT) (IF (EQ STATEMENT STATEMENT-P) (VECTOR 0.7) (VECTOR 0.0)))
CONTEXTUAL-INTERPRETATION-2 (LAMBDA (STATEMENT) (IF (EQ STATEMENT
STATEMENT-Q) (VECTOR 0.7) (VECTOR 0.0)))) Decision under
NON-CLASSICAL-INTERPRETATION Interpretation: NIL Decision under
PARACONSISTENT-INTERPRETATION Interpretation: NIL Decision under
DIALETHEIST-INTERPRETATION Interpretation: NIL Decision under
CONTEXTUAL-INTERPRETATION-1 Interpretation: NIL Decision under
CONTEXTUAL-INTERPRETATION-2 Interpretation: NIL

--- Non-Singular Outcome Analysis --- Interpretation-Dependent
Decisions: Decisions vary across interpretations. Generalized logical
relations offer interpretation-dependent outcomes. --- Generalized
Decision Procedure --- Statement: EVENT-A Using Encoded Relation:
BELL-SCENARIO-RELATION Evaluating under interpretations:
(CLASSICAL-INTERPRETATION QUANTUM-INTERPRETATION) Decision under
CLASSICAL-INTERPRETATION Interpretation: (0.0 QUANTUM-INTERPRETATION
1.0) Decision under QUANTUM-INTERPRETATION Interpretation: NIL

--- Non-Singular Outcome Analysis --- Interpretation-Dependent
Decisions: Decisions vary across interpretations. Generalized logical
relations offer interpretation-dependent outcomes. --- Generalized
Decision Procedure --- Statement: P Using Encoded Relation:
TARSKIAN-RELATION Evaluating under interpretations:
(CLASSICAL-INTERPRETATION) Decision under CLASSICAL-INTERPRETATION
Interpretation: (1.0)

--- Non-Singular Outcome Analysis --- Interpretation-Dependent
Decisions: Decisions vary across interpretations. Generalized logical
relations offer interpretation-dependent outcomes. --- Refutation
Experiment: Bell Scenario --- Attempting to refute: H\_Bell -
Non-Reflexivity of bell-scenario-relation under
:classical-interpretation Trying to find a :classical-interpretation
where bell-scenario-relation \emph{is} reflexive. Relation: Bell
Scenario Relation (Non-Local) - ENCODED Checking for Reflexivity under a
\emph{modified} :classical-interpretation\ldots{} Reflexive under
REFUTATION-CLASSICAL-INTERPRETATION interpretation? Reflexive under
REFUTATION-CLASSICAL-INTERPRETATION interpretation? (Threshold: 0.5)
EVENT-A is related to itself: NIL Outcome: Refutation Experiment
\emph{FAILS} to refute H\_Bell (under modified
:classical-interpretation). (H\_Bell remains robust) --- Refutation
Experiment: Liar Paradox --- Attempting to refute: H\_Liar\_NonClassical
- Reflexivity of liar-paradox-relation under
:non-classical-interpretation Trying to find a
:non-classical-interpretation where liar-paradox-relation is \emph{not}
reflexive. Relation: Liar Paradox Relation (Non-Local, Self-Referential)
- ENCODED Checking for Reflexivity under a \emph{modified}
:non-classical-interpretation\ldots{} Reflexive under
REFUTATION-NON-CLASSICAL-INTERPRETATION interpretation? Reflexive under
REFUTATION-NON-CLASSICAL-INTERPRETATION interpretation? (Threshold: 0.5)
STATEMENT-P is related to itself: NIL Outcome: Refutation Experiment
\emph{FAILS} to refute H\_Liar\_NonClassical (under modified
:non-classical-interpretation). (H\_Liar\_NonClassical remains robust)
--- Non-Classical Logical Relations Methodology --- Methodology based on
Generalized Logical Relations and Non-Locality: - Hypothesize: Contrast
classical vs.~non-classical relations, graded relatedness, matrix/vector
spaces. - Define: Define local/non-local relations, reflexivity,
symmetry, transitivity, context-dependence, graded relatedness, n-to-m
relations, matrix/vector spaces. - Theoremize: Derive theorems on
non-locality impact on Tarskian properties, non-singular outcomes,
algebraic properties. - Experiment: Test hypotheses/theorems via
paradoxes, decision procedures, refutation experiments, vector spaces,
non-algebraic properties. - Analyze Outcomes: Interpretation-dependent
decisions, non-singular outcomes, refutation robustness, vector spaces,
graded relatedness, algebraic properties. - Refine Theory: Refine theory
based on experiments, considering graded relatedness, refutation
outcomes, mathematical representations, non-algebraic behavior. --- End
Non-Classical Logical Relations Methodology --- n--- n-to-m Relation
Design --- Evaluating n-m-relation with inputs (STATEMENT-X STATEMENT-Y
STATEMENT-Z) and interpretations (CONTEXT-A CONTEXT-B CONTEXT-DEFAULT):
n-m-relation output: \#(0.3 0.1) --- End n-to-m Relation Design ---nn---
Experiment: Testing Non-Transitivity of n-m-relation (Consistency) ---
Relation Output AB (Inputs: (STATEMENT-X STATEMENT-Y), Interpretations:
(CONTEXT-AB CONTEXT-AB)): \#(0.9 0.7), Consistency: 0.9 Relation Output
BC (Inputs: (STATEMENT-Y STATEMENT-Z), Interpretations: (CONTEXT-BC
CONTEXT-BC)): \#(0.9 0.7), Consistency: 0.9 Relation Output AC (Inputs:
(STATEMENT-X STATEMENT-Z), Interpretations: (CONTEXT-AC CONTEXT-AC)):
\#(0.3 0.7), Consistency: 0.3

--- Hypothesis Test: H\_Transitivity - n-m-relation Consistency IS
Transitive --- Hypothesis (H\_Transitivity): High consistency(AB) AND
high consistency(BC) -\textgreater{} high consistency(AC) (threshold
0.7). Null Hypothesis (H\_Transitivity'): n-m-relation Consistency is
NOT Transitive. Outcome: Experiment FAILS to refute H\_Transitivity',
REFUTES H\_Transitivity. (Demonstrates Non-Transitivity) --- End
Experiment: Testing Non-Transitivity of n-m-relation (Consistency) ---
n--- Generalized n-m-relation Example --- Evaluating
generalized-n-m-relation with inputs (STATEMENT-X STATEMENT-Y),
interpretations (CONTEXT-A CONTEXT-B), and logical-relation \#:
generalized-n-m-relation output: \#(0.9 0.7) --- End Generalized
n-m-relation Example ---nn--- End Main Execution - Sequence B ---n\%



\chapter{Signatures of Logos Theories}
}

†A: Complex Conjugate/Transpose Negation (Orthogonal/Perpendicular
Shift).

A ∥ B: Additive Implication (Parallel Processes/Concurrent Choice). *

A ∦ B: Additive Non-Implication (Non-Parallel/Mutually Exclusive
Processes).

$\neg$ : Linear Negation (Classical).

⊸: Multiplicative Implication (Linear Implication).

⊸̸: Multiplicative Non-Implication (Type 1A).

▹A: Type 1A Negation (Paracomplete).

◃A: Type 1B Negation (Paraconsistent).

\&, $\oplus$ , $\otimes$ , $\parr$ : Additive and Multiplicative Conjunction/Disjunction.

$\exists$ $\forall$  \{$\bot$ ,$\top$ ,$\neg$ ,$\lor$ ,$\land$ ,$\to$ ,$\nrightarrow$ ,$\leftrightarrow$ ,$\oplus$ ,$\downarrow$ ,$\uparrow$ \} \{p, q\} $\vdash$ $\nmodels$ $\models$ $\Gamma$ $\Delta$

Signature of Theories with Logos Formalization

The theories discussed in this paper will be referred to as
\emph{theories with standard formalization}. They can be briefly
characterized as theories which are formalized within the first-order
predicate logic (with identity, without variable predicates).

\hypertarget{syntactic-definitions}{%
\subsection{Syntactic Definitions}\label{syntactic-definitions}}

\hypertarget{definition-variables-and-constants-of-a-theory}{%
\subsubsection{Definition: Variables and Constants of a
Theory}\label{definition-variables-and-constants-of-a-theory}}

The symbols which occur in expressions of a given theory T are divided
into \emph{variables} and \emph{constants}.

The set of variables is assumed to be denumerable and hence infinite;
the set of constants is either finite or denumerable. All the variables
are treated as ranging over the same set of elements.

\hypertarget{section}{%
\subsubsection{}\label{section}}

\hypertarget{definition-non-logical-constants-of-a-theory-aka-signature-of-a-theory}{%
\subsubsection{Definition: non-Logical Constants of a Theory AKA
Signature of a
Theory}\label{definition-non-logical-constants-of-a-theory-aka-signature-of-a-theory}}

The non-logical constants are the \emph{predicates} (or \emph{relation
symbols}), the \emph{operation symbols}, and the \emph{individual
constants}.

With every predicate and every operation symbol a positive integer is
correlated which is called the \emph{rank} of the symbol. Thus, we may
have in T \emph{unary} predicates and operation symbols (I.E. symbols of
rank 1), \emph{binary} predicates and operation symbols (symbols of rank
2), etc. The identity symbol, though regarded as a logical constant, is
included in the set of binary predicates.

\hypertarget{definition-terms}{%
\subsubsection{Definition: Terms}\label{definition-terms}}

Among expressions (I.E. finite concatenations of symbols) we distinguish
\emph{terms} and \emph{formulas}.

The simplest, so-called atomic, terms are the variables and the
individual constants; a compound term is obtained by combining \emph{n}
simpler terms by means of an operation symbol of rank \emph{n}.

\hypertarget{definition-formulas}{%
\subsubsection{Definition: Formulas}\label{definition-formulas}}

Similarly, an atomic formula is obtained by combining \emph{n} arbitrary
terms by means of a predicate of rank \emph{n}; compound formulas are
built from simpler ones by means of sentential connectives and
quantifier expressions (I.E. quantifiers followed by variables like $\forall$ x
or $\exists$ y).

\hypertarget{definition-sentences}{%
\subsubsection{Definition: Sentences}\label{definition-sentences}}

An occurrence of a variable in a formula may be either \emph{free} or
\emph{bound}; a formula in which no variable occurs free is called a
\emph{sentence}.

\hypertarget{semantic-definitions}{%
\subsection{Semantic Definitions}\label{semantic-definitions}}

\hypertarget{model-theoretical-semantics}{%
\paragraph{Model-Theoretical
Semantics}\label{model-theoretical-semantics}}

Another method of defining logically derivable and logically valid is
available which essentially involves the use of some semantical notions
and the notion of satisfaction.

\hypertarget{definition-possible-realizations-of-a-theory-and-the-universe-of-ux1d57d}{%
\subsubsection{Definition: Possible Realizations of a Theory and the
Universe of
$\mathbb{R}$ }\label{definition-possible-realizations-of-a-theory-and-the-universe-of-ux1d57d}}

We assume that all the non-logical constants of T have been arranged in
a (finite or infinite) sequence \textless{}\textbf{C\_0}, \ldots,
\textbf{C\_n}, \ldots\textgreater, without repeating terms.

We consider systems $\mathbb{R}$  formed by a non-empty set U and by a sequence
\textless C\_0, \ldots, C\_n, \ldots\textgreater{} of certain
mathematical entities, with the same number of terms as the sequence of
non-logical constants.

The mathematical nature of each C\_n depends on the logical character of
the corresponding constant \textbf{C\_n}. Thus,

if \textbf{C\_n} is a unary predicate, then C\_n is a subset of U; more
generally, if \textbf{C\_n} is an \emph{m}-ary predicate, then C\_n is
an \emph{m}-ary relation the field of which is a subset of U.

If \textbf{C\_n} is an \emph{m}-ary operation symbol, C\_n is an
\emph{m}-ary operation (function of \emph{m} arguments) defined over
arbitrary ordered \emph{m}-tuples \textless{}\emph{x\_1}, \ldots,
\emph{x\_m}\textgreater{} of elements of U and assuming elements of U as
values.

If \textbf{C\_n} is an individual constant, C\_n is simply an element of
U.

Such a system (sequence) $\mathbb{R}$  = \textless U, C\_0, \ldots, C\_n,
\ldots\textgreater{} is called a \emph{possible realization} or simply a
\emph{realization} of T; the set U is called the \emph{universe} of $\mathbb{R}$ .

\hypertarget{definition-satisfaction}{%
\subsubsection{Definition: Satisfaction}\label{definition-satisfaction}}

We assume it to be clear under what conditions a sentence $\Phi$  of T is said
to \emph{be satisfied} or to \emph{hold} in a given realization $\mathbb{R}$ .

Roughly speaking, this means that $\Phi$  turns out to be true if

(i) all the variables occurring in T are assumed to range over the set
U;

(ii) the logical constants are interpreted in the usual way;

(iii) each of the non-logical constants \textbf{C\_n} is understood to
denote the corresponding term C\_n in $\mathbb{R}$ .

Assume, e.g., that the term \textbf{C\_n} in the sequence of constants
is a unary predicate and that consequently C\_n is a subset of U. Then
the sentence $\forall$ x \textbf{C\_n} x holds in $\mathbb{R}$  if and only if every element
of U is an element of C\_n and hence C\_n coincides with U.

\hypertarget{definition-valid-sentences-by-non-logical-axioms}{%
\subsubsection{Definition: Valid Sentences by Non-Logical
Axioms}\label{definition-valid-sentences-by-non-logical-axioms}}

Often we single out a (finite or infinite) set of sentences called
\emph{non-logical axioms}, and define a sentence to be valid if and only
if it is derivable from this set-\/-or, what amounts to the same, from
the set of all axioms, both logical and non-logical.

In all the theories with standard formalization the same symbols are
assumed to be used as variables and logical constants; apart from
differences in non-logical constants, the same expressions are regarded
as formulas, sentences, logical axioms, and logically valid sentences.

However, the notions of validity in these theories may of course exhibit
essential differences.

\hypertarget{theoretical-definitions}{%
\subsection{Theoretical Definitions}\label{theoretical-definitions}}

\hypertarget{definition-uniqueness-of-an-axiomatic-theory-in-standard-formalization}{%
\subsubsection{Definition: Uniqueness of an Axiomatic Theory in Standard
Formalization}\label{definition-uniqueness-of-an-axiomatic-theory-in-standard-formalization}}

An axiomatic theory is uniquely determined by its non-logical constants
and non-logical axioms.

\hypertarget{definition-inessential-extensions-of-theories-in-standard-formalization}{%
\subsubsection{Definition: Inessential Extensions of Theories in
Standard
Formalization}\label{definition-inessential-extensions-of-theories-in-standard-formalization}}

An extension T\_2 of T\_1 is called \emph{inessential} if every constant
of T\_2 which does not occur in T\_1 is an individual constant and if
every valid sentence of T\_2 is derivable in T\_2 from a set of valid
sentences of T\_1.

If T\_1 is axiomatic, then an inessential extension of T\_1 is obtained
by adding some new individual constants, but without adding any new
non-logical axioms.

By saying that a sentence $\Phi$  is derivable \emph{in a theory} T from a set
A we stress the fact that, in deriving $\Phi$ , we may use both sentences of A
and logical axioms of T. It is easily seen that, whenever $\Phi$  is derivable
from A in some theory T, it is also derivable from A in every theory
T\textquotesingle{} which contains all the non-logical constants
occurring in $\Phi$  and in sentences of A.

\hypertarget{definition-union-of-theories-in-standard-formalization}{%
\subsubsection{Definition: Union of Theories in Standard
Formalization}\label{definition-union-of-theories-in-standard-formalization}}

Among the extensions common to two given theories T\_1 and T\_2 there is
always a smallest one, which is a subtheory of any other common
extension; this smallest common extension is referred to as the
\emph{union} of the given theories.

The union T of T\_1 and T\_2 is fully characterized by the following two
conditions:

(i) the set of all non-logical constants of T is the (set theoretical)
union of the set of all non-logical constants of T\_1 and T\_2;

(ii) a sentence is valid in T if and only if it is derivable in T from a
set of sentences which are valid in T\_1 or T\_2.

If the theories T\_1 and T\_2 are axiomatic, we can construct T by
postulating, in addition to (i), the analogous condition for the set of
non-logical axioms.

}*}


\chapter{Subclassical Calculi Philosophy}

\documentclass{article}

\usepackage{amsmath}
\usepackage{ebproof}
\usepackage{fullpage}
\usepackage[utf8]{inputenc}
\usepackage{newunicodechar}
\usepackage{stix}

\newunicodechar{Γ}{\Gamma}
\newunicodechar{Δ}{\Delta}

\newunicodechar{Θ}{\Theta}
\newunicodechar{Λ}{\Lambda}

\newunicodechar{Ξ}{\Xi}
\newunicodechar{Π}{\Pi}

\newunicodechar{Φ}{\Phi}
\newunicodechar{Ψ}{\Psi}

\newunicodechar{Ω}{\Omega}

\newunicodechar{⊢}{\vdash}

\newunicodechar{⊕}{\oplus}
\newunicodechar{¬}{\neg}

\newunicodechar{⊗}{\otimes}
\newunicodechar{→}{\rightarrow}
\newunicodechar{←}{\leftarrow}
\newunicodechar{↔}{\leftrightarrow}
\newunicodechar{↮}{\nleftrightarrow}


\newunicodechar{⅋}{\upand}
\newunicodechar{↛}{\nrightarrow}
\newunicodechar{↚}{\nleftarrow}

\newunicodechar{⊥}{\bot}
\newunicodechar{⊤}{\top}

\setlength{\parindent}{0em}

\author{James Martin, Ian D.L.N. Mclean}
\title{The Lattice of Conservative Non-Classical and Classical Sequent Calculi}

\begin{document}

\maketitle

\begin{abstract}
Subclassical sequent calculi are defined by functional incompleteness and a strict subset of de Morgan dualities for the quantified predicate calculi.
Superclassical sequent calculi are defined by at least functional completeness and classical de Morgan duality.
\end{abstract}

\part{Preliminaries}
\begin{center}
	\begin{flushleft}
		The key concept to the construction of logical calculi that are distinctly different from classical logic is functional incompleteness with respect to the Boolean domain and Boolean functions.
	\end{flushleft}
	\begin{flushleft}
		If a calculus has any classical logical connectives such that we can express every theorem of the classical calculus then our logical calculi would degenerate to the classical calculus and we'd lose in general the specificity that is gained by constructive reasoning.
	\end{flushleft}
	\begin{flushleft}
		Roughly speaking, if any set of operators or functions is not a subset of at least one of the five functionally incomplete sets then that set of operators or functions is functionally complete.
	\end{flushleft}
	\begin{flushleft}
		We generalize this in a manner analogous to Tarski's original formal interpretation as applied to the problem of decidability of theories in standard formalization. If we have collections of non-classical or sub-classical operators or functions that can be interpreted in at least one of the five functionally incomplete sets then those collections of operators or functions are functionally incomplete with respect to the Boolean domain and Boolean functions.
	\end{flushleft}
	\begin{flushleft}
		Finally, if we restrict ourselves to only those logical connectives which are classical then the systems can not extend to each other and it seems can not interpret each other, but if we extend these systems by some non-classical logical connective that is compatible with a given set of functionally incomplete logical connectives then we can extend between these systems and interpret between them.
	\end{flushleft}
	\section{Interpretations}
		\subsection{Operational Interpretations}
		\subsection{Structural Interpretations}
		\subsection{Systematic Interpretations}
	\section{Formal Definition of Subclassical Theories}
		\subsection{Monotonic Theories}
		$\left\{ Γ ⊢ A \right\} \bigcup \left\{ A ⊢ Δ \right\}$
		\subsection{Truth-Preserving Theories}
		$\left\{ A ⊢ Δ \right\}$
		\subsection{False-Preserving Theories}
		$\left\{ Γ ⊢ A \right\}$
		\subsection{Affine Theories}
		$\left\{ Γ ⊢ A \right\} \bigcap \left\{ A ⊢ Δ \right\}=\emptyset$
		\subsection{Self-Dual Theories}
		$\left\{ A ⊢ A \right\}$
	
\end{center}

\part{Conservative Duality}
Key Concepts:

Sequent calculus: A formal system for representing and reasoning about logical formulas.

Theorem: A logical statement that can be proven using a sequent calculus.

Antitheorem: A logical statement that can be disproven using a sequent calculus.

Classical logic: A system of logic that includes the laws of commutativity, association, distributivity, identity, double negation, excluded middle, and non-contradiction.

Superclassical logic: A system of logic that extends classical logic by adding additional theorems or antitheorems.

Subclassical logic: A system of logic that is weaker than classical logic, meaning it does removes some of the theorems or antitheorems of classical logic while preserving at least one theorem or antitheorem.

Definition 1: A sequent calculus is isomorphic if it proves the same theorems as another calculus but not more and disproves the same theorems as another calculus but not more.

Definition 2: A sequent calculus is a subtension of another calculus if it proves the fewer theorems or disproves fewer antitheorems than that calculus and either proves some of the same theorems as that calculus or disproves some of the same antitheorems as that calculus.

Definition 3: A sequent calculus is a conservative extension of another calculus if it proves all of the same theorems as that calculus and disproves all the same antitheorems as that calculus and proves strictly more theorems than that calculus or disproves strictly more antitheorems than that calculus.

Definition 4: A sequent calculus is superclassical if it preserves all of the theorems of classical logic and it proves more theorems than classical logic or it disproves more antitheorems than classical logic.

Definition 5: A sequent calculus is subclassical if it does not preserve all of the theorems of classical logic or if it does not preserve all the antitheorems of classical logic.

Definition 6: Two sequent calculi are conservative duals of each other if they are mutually exclusive and share some theorems or antitheorems.

Definition 1: Isomorphic sequent calculus:

Two sequent calculi, C1 and C2, are isomorphic if they prove the same theorems as each other and disprove the same antitheorems as each other. In other words, for any sequent S, if C1 proves S, then C2 also proves S, and if C1 disproves S, then C2 also disproves S.

Definition 2: Subtension sequent calculus:

A sequent calculus, C1, is a subtension of another calculus, C2, if C1 proves fewer theorems or disproves fewer antitheorems than C2, and either C1 proves some of the same theorems as C2 or disproves some of the same antitheorems as C2. In other words, C1 is a subset of C2 in terms of the theorems it proves and the antitheorems it disproves.

Definition 3: Conservative extension sequent calculus:

A sequent calculus, C1, is a conservative extension of another calculus, C2, if C1 proves all of the same theorems as C2 and disproves all the same antitheorems as C2, and C1 proves strictly more theorems than C2 or disproves strictly more antitheorems than C2. In other words, C1 extends C2 by adding more theorems or disproving more antitheorems.

Definition 4: Superclassical sequent calculus:

A sequent calculus is superclassical if it preserves all of the theorems of classical logic and it proves more theorems than classical logic or it disproves more antitheorems than classical logic. In other words, a superclassical sequent calculus goes beyond the scope of classical logic in terms of the theorems it proves or the antitheorems it disproves.

Definition 5: Subclassical sequent calculus:

A sequent calculus is subclassical if it does not preserve all of the theorems of classical logic or if it does not preserve all the antitheorems of classical logic. In other words, a subclassical sequent calculus is weaker than classical logic in terms of the theorems it proves or the antitheorems it disproves.

Definition 6: Conservative duals of sequent calculi:

Two sequent calculi, C1 and C2, are conservative duals of each other if they are mutually exclusive and share some theorems or antitheorems. In other words, they share some number of theorems in common but have some number of theorems or antitheorems that are independent of each other calculi.

Definition 1:
Two sequent calculi, C1 and C2, are isomorphic sequent calculi if they prove the same theorems and disprove the same antitheorems for all sequents S.

Definition 2: Conservative subtension sequent calculus:
A sequent calculus, C1, is a conservative sequent subtension of another calculus, C2, if C1 proves fewer theorems or disproves fewer antitheorems than C2, and either C1 proves some of the same theorems as C2 or disproves some of the same antitheorems as C2, and it does not prove theorems that its conservative extension does not prove and it does not disprove antitheorems that its conservative extension does not disprove. In other words, C1 is a subset of C2 in terms of the theorems it proves and the antitheorems it disproves.

Definition 3: Conservative extension sequent calculus:
A sequent calculus, C1, is a conservative sequent calculus extension of another calculus, C2, if C1 proves all of the same theorems as C2 and disproves all the same antitheorems as C2, and C1 proves strictly more theorems than C2 or disproves strictly more antitheorems than C2.

Definition 4: Superclassical sequent calculus:
A sequent calculus is superclassical if it preserves all of the theorems of classical logic and it proves more theorems than classical logic or it disproves more antitheorems than classical logic. In other words, a superclassical sequent calculus goes beyond the scope of classical logic in terms of the theorems it proves or the antitheorems it disproves.

Definition 5: Subclassical sequent calculus:
A sequent calculus is subclassical if it does not preserve all of the theorems of classical logic or if it does not preserve all the antitheorems of classical logic. In other words, a subclassical sequent calculus is weaker than classical logic in terms of the theorems it proves or the antitheorems it disproves.

Definition 6: Conservative duals of sequent calculi:
Two sequent calculi, C1 and C2, are conservative duals of each other if they are mutually exclusive and share some theorems or antitheorems, and they do not prove all the same theorems and disprove all the same antitheorems. In other words, they share some number of theorems in common but have some number of theorems or antitheorems that are independent of each other calculi.

Definition 7: A sequent calculus is a conservative non-classical sequent calculus if it does not contradict any theorems, antitheorems, or refutations of classical logic; equivalently, a sequent calculus is a conservative non-classical sequent calculus if it is compatible with classical logic; equivalently, a sequent calculus is a conservative non-classical sequent calculus if all theorems and antitheorems of classical logic are admissible or valid in the sequent calculus.

   ∀x. ( x ⇔ (∃y. (Subtend(x, y) ∨ Extend(x, y) ∨ Dual(x, y) ∨ Isomorph(x, y))))
   
   ∀x,y.(Dual(x,y)↔∃z,w.(Extend(x,z)∧Extend(y,z)∧Subtension(w,x)∧Subtension(w,y)))
   
∀x,y.(Subtension(x,y) ⇔ (Theorem(x)→Theorem(y)∧AntiTheorem(x)→AntiTheorem(y))) (Definition of subtension)

∀x,y.(Extend(x,y) ⇔ (Theorem(y)→Theorem(x)∧AntiTheorem(y)→AntiTheorem(x))) (Definition of extension)

∀x,y.(Dual(x,y) ⇔ Negate(Isomorph(x, y), SequentCalculus(x)) ∧ ∃z,w.(SuperCalc(x,z) ∧ SuperCalc(y,z) ∧ SubCalc(w,x) ∧ SubCalc(w,y))) (Definition of dual)

∃x∃y. (Doubtful(Deniable(x, SequentCalculus(y)), SequentCalculus(y))) ⇔ Clanemalo(x, SequentCalculus(y))
∃x∃y. (Deniable(Doubtful(x, SequentCalculus(y)), SequentCalculus(y))) ⇔ negIndependent(x, SequentCalculus(y)) ⇔ Negate(AntiTheorem(x, SequentCalculus(y)), SequentCalculus(y))

∃x∃y. Negate(x, SequentCalculus(y)) ⇔ Clanemalo(x, SequentCalculus(y)) ∨ negIndependent(x, SequentCalculus(y)) ∨ Deniable(x, SequentCalculus(y)) ∨ Doubtful(x, SequentCalculus(y))

∀x, y. Independent(x, y) ⇔ (¬Proves(y, x) ∧ ¬Disproves(y, x)) ⇔ (posIndependent(x, y) ∧ negIndependent(x, y)) ⇔ ∀z. (¬Theorem(z, x) ∧ ¬AntiTheorem(z, x))
∀x∃y. (posIndepenedent(x, y) ⇔ Negate(Theorem(x, SequentCalculus(y)), SequentCalculus(y)))

   Definition of subtension   
A Sequent Calculus, SequentCalc0, is called a subtension of a Sequent Calculus, SequentCalc1, if every theorem in SequentCalc0 is also in SequentCalc1 and every antitheorem in SequentCalc0 is also in SequentCalc1; under the same conditions, SequentCalc1 is an extension of SequentCalc0.

Theorem 1: For any two sequent calculi C1 and C2, the following are mutually exclusive and jointly exhaustive:
C1 is a conservative extension of C2.
C2 is a conservative extension of C1.
C1 and C2 are conservative duals of each other.
C1 is a conservative subtension of C2.
C2 is a conservative subtension of C1.
Corollary1: There are no other ways for two sequent calculi to be related to each other.
Theorem 2: The lattice of sequent calculi has a supremum sequent calculus and an infimum sequent calculus.
Corollary 2: The supremum sequent calculus is superclassical.
Corollary 3: The infimum sequent calculus is subclassical.

Theorems:

Theorem 1: For any two sequent calculi C1 and C2, the following are mutually exclusive and jointly exhaustive:
C1 is a conservative extension of C2.
C2 is a conservative extension of C1.
C1 and C2 are conservative duals of each other.
C1 is a conservative subtension of C2.
C2 is a conservative subtension of C1.

Corollary: There are no other ways for two sequent calculi to be related to each other.

Theorem 2: The lattice of sequent calculi has a supremum sequent calculus and an infimum sequent calculus.

Corollary 1: The supremum sequent calculus is superclassical.

Corollary 2: The infimum sequent calculus is subclassical.
\begin{center}
	
	\section{Structural Monotone Calculus}
		\subsection{Structural Rules}
		\begin{center}
			\[
			\begin{prooftree}
			\infer0[Id]{A ⊢ A}
			\end{prooftree}
			\]
			
			\[
			\begin{prooftree}
			\hypo{Γ ⊢ A}
			\hypo{A ⊢ Δ}
			\infer2[Cut]{Γ ⊢ Δ}
			\end{prooftree}
			\]
			
			\[
			\begin{prooftree}
			\hypo{Γ ⊢ Δ}
			\infer1[wL]{Γ, A ⊢ Δ}
			\end{prooftree}
			\qquad
			\begin{prooftree}
			\hypo{Γ ⊢ Δ}
			\infer1[Rw]{Γ ⊢ A, Δ}
			\end{prooftree}
			\]
			
			\[
			\begin{prooftree}
			\hypo{Γ, A, A ⊢ Δ}
			\infer1[cL]{Γ, A ⊢ Δ}
			\end{prooftree}
			\qquad
			\begin{prooftree}
			\hypo{Γ ⊢ A, A Δ}
			\infer1[Rc]{Γ ⊢ A, Δ}
			\end{prooftree}
			\]
			
			\[
			\begin{prooftree}
			\hypo{Γ_0, A, B, Γ_1 ⊢ Δ}
			\infer1[pL]{Γ_0, B, A, Γ_1 ⊢ Δ}
			\end{prooftree}
			\qquad
			\begin{prooftree}
			\hypo{Γ ⊢ Δ_1, A, B, Δ_0}
			\infer1[Rp]{Γ ⊢ Δ_1, B, A, Δ_0}
			\end{prooftree}
			\]
		\end{center}
		
		\subsection{Unit Rules}
		\begin{center}
			\[
			\begin{prooftree}
			\infer0{Γ, ⊥ ⊢ Δ}
			\end{prooftree}
			\quad
			\begin{prooftree}
			\infer0{ Γ ⊢ ⊤, Δ}
			\end{prooftree}
			\]
		\end{center}
		
		\subsection{Operational Rules}
		\begin{center}
		
			\subsubsection{Multiplicatives}
			\begin{center}
				\[
				\begin{prooftree}
				\hypo{Γ, A, B ⊢ Δ}
				\infer1{Γ, A ⊗ B ⊢ Δ}
				\end{prooftree}
				\quad
				\begin{prooftree}
				\hypo{Γ ⊢ A, Δ}
				\hypo{Γ ⊢ B, Δ}
				\infer2{Γ ⊢ A ⊗ B, Δ}
				\end{prooftree}
				\]
				
				\[
				\begin{prooftree}
				\hypo{Γ, A ⊢ Δ}
				\hypo{Γ, B ⊢ Δ}
				\infer2{Γ, A ⅋ B ⊢ Δ}
				\end{prooftree}
				\quad
				\begin{prooftree}
				\hypo{Γ ⊢ A, B, Δ}
				\infer1{Γ ⊢ A ⅋ B, Δ}
				\end{prooftree}
				\]
			\end{center}
		\end{center}
		
		\subsection{Theorems}
		\begin{center}
		\end{center}

	\section{Structural Truth-Preserving Calculus}
		\subsection{Structural Rules}
		\begin{center}
			\[
			\begin{prooftree}
			\infer0[Id]{A ⊢ A}
			\end{prooftree}
			\]
			
			\[
			\begin{prooftree}
			\hypo{Γ ⊢ A}
			\hypo{A ⊢ Δ}
			\infer2[Cut]{Γ ⊢ Δ}
			\end{prooftree}
			\]
			
			\[
			\begin{prooftree}
			\hypo{Γ ⊢ Δ}
			\infer1[wL]{Γ, A ⊢ Δ}
			\end{prooftree}
			\qquad
			\begin{prooftree}
			\hypo{Γ ⊢ Δ}
			\infer1[Rw]{Γ ⊢ A, Δ}
			\end{prooftree}
			\]
			
			\[
			\begin{prooftree}
			\hypo{Γ, A, A ⊢ Δ}
			\infer1[cL]{Γ, A ⊢ Δ}
			\end{prooftree}
			\qquad
			\begin{prooftree}
			\hypo{Γ ⊢ A, A Δ}
			\infer1[Rc]{Γ ⊢ A, Δ}
			\end{prooftree}
			\]
			
			\[
			\begin{prooftree}
			\hypo{Γ_0, A, B, Γ_1 ⊢ Δ}
			\infer1[pL]{Γ_0, B, A, Γ_1 ⊢ Δ}
			\end{prooftree}
			\qquad
			\begin{prooftree}
			\hypo{Γ ⊢ Δ_1, A, B, Δ_0}
			\infer1[Rp]{Γ ⊢ Δ_1, B, A, Δ_0}
			\end{prooftree}
			\]
		\end{center}
		
		\subsection{Unit Rules}
		\begin{center}
			\[
			\begin{prooftree}
			\infer0{ Γ ⊢ ⊤, Δ}
			\end{prooftree}
			\]
		\end{center}
		
		\subsection{Operational Rules}
		\begin{center}
			
			\subsubsection{Multiplicatives}
			\begin{center}
				\[
				\begin{prooftree}
				\hypo{Γ, A, B ⊢ Δ}
				\infer1{Γ, A ⊗ B ⊢ Δ}
				\end{prooftree}
				\quad
				\begin{prooftree}
				\hypo{Γ ⊢ A, Δ}
				\hypo{Γ ⊢ B, Δ}
				\infer2{Γ ⊢ A ⊗ B, Δ}
				\end{prooftree}
				\]
				
				\[
				\begin{prooftree}
				\hypo{Γ, A ⊢ Δ}
				\hypo{Γ, B ⊢ Δ}
				\infer2{Γ, A ⅋ B ⊢ Δ}
				\end{prooftree}
				\quad
				\begin{prooftree}
				\hypo{Γ ⊢ A, B, Δ}
				\infer1{Γ ⊢ A ⅋ B, Δ}
				\end{prooftree}
				\]
				
				\[
				\begin{prooftree}
				\hypo{Γ ⊢ A, Δ}
				\hypo{Γ, B ⊢ Δ}
				\infer2{Γ, A → B ⊢ Δ}
				\end{prooftree}
				\quad
				\begin{prooftree}
				\hypo{Γ, A ⊢ B, Δ}
				\infer1{Γ ⊢ A → B, Δ}
				\end{prooftree}
				\]
				
				\[
				\begin{prooftree}
				\hypo{Γ, A ⊢ Δ}
				\hypo{Γ ⊢ B, Δ}
				\infer2{Γ, A ← B ⊢ Δ}
				\end{prooftree}
				\quad
				\begin{prooftree}
				\hypo{Γ, B ⊢ A, Δ}
				\infer1{Γ ⊢ A ← B, Δ}
				\end{prooftree}
				\]
				
				\[
				\begin{prooftree}
				\hypo{Γ ⊢ A, B, Δ}
				\infer0{Γ, A ⊢ A, Δ}
				\infer0{Γ, B ⊢ B, Δ}
				\hypo{Γ, A, B ⊢ Δ}
				\infer4{Γ, A ↔ B ⊢ Δ}
				\end{prooftree}
				\quad
				\begin{prooftree}
				\hypo{Γ, A ⊢ B, Δ}
				\hypo{Γ, B ⊢ A, Δ}
				\infer2{Γ ⊢ A ↔ B, Δ}
				\end{prooftree}
				\]
			\end{center}
		\end{center}
		
		\subsection{Theorems}
			\begin{center}
			\end{center}

	\section{Structural False-Preserving Calculus}
		\subsection{Structural Rules}
		\begin{center}
			\[
			\begin{prooftree}
			\infer0[Id]{A ⊢ A}
			\end{prooftree}
			\]
			
			\[
			\begin{prooftree}
			\hypo{Γ ⊢ A}
			\hypo{A ⊢ Δ}
			\infer2[Cut]{Γ ⊢ Δ}
			\end{prooftree}
			\]
			
			\[
			\begin{prooftree}
			\hypo{Γ ⊢ Δ}
			\infer1[wL]{Γ, A ⊢ Δ}
			\end{prooftree}
			\qquad
			\begin{prooftree}
			\hypo{Γ ⊢ Δ}
			\infer1[Rw]{Γ ⊢ A, Δ}
			\end{prooftree}
			\]
			
			\[
			\begin{prooftree}
			\hypo{Γ, A, A ⊢ Δ}
			\infer1[cL]{Γ, A ⊢ Δ}
			\end{prooftree}
			\qquad
			\begin{prooftree}
			\hypo{Γ ⊢ A, A Δ}
			\infer1[Rc]{Γ ⊢ A, Δ}
			\end{prooftree}
			\]
			
			\[
			\begin{prooftree}
			\hypo{Γ_0, A, B, Γ_1 ⊢ Δ}
			\infer1[pL]{Γ_0, B, A, Γ_1 ⊢ Δ}
			\end{prooftree}
			\qquad
			\begin{prooftree}
			\hypo{Γ ⊢ Δ_1, A, B, Δ_0}
			\infer1[Rp]{Γ ⊢ Δ_1, B, A, Δ_0}
			\end{prooftree}
			\]
		\end{center}
		
		\subsection{Unit Rules}
		\begin{center}
			\[
			\begin{prooftree}
			\infer0{Γ, ⊥ ⊢ Δ}
			\end{prooftree}
			\]
		\end{center}
		
		\subsection{Operational Rules}
		\begin{center}
			
			\subsubsection{Multiplicatives}
			\begin{center}
				\[
				\begin{prooftree}
				\hypo{Γ, A, B ⊢ Δ}
				\infer1{Γ, A ⊗ B ⊢ Δ}
				\end{prooftree}
				\quad
				\begin{prooftree}
				\hypo{Γ ⊢ A, Δ}
				\hypo{Γ ⊢ B, Δ}
				\infer2{Γ ⊢ A ⊗ B, Δ}
				\end{prooftree}
				\]
				
				\[
				\begin{prooftree}
				\hypo{Γ, A ⊢ Δ}
				\hypo{Γ, B ⊢ Δ}
				\infer2{Γ, A ⅋ B ⊢ Δ}
				\end{prooftree}
				\quad
				\begin{prooftree}
				\hypo{Γ ⊢ A, B, Δ}
				\infer1{Γ ⊢ A ⅋ B, Δ}
				\end{prooftree}
				\]
				
				\[
				\begin{prooftree}
				\hypo{Γ, A ⊢ B, Δ}
				\infer1{Γ, A ↛ B ⊢ Δ}
				\end{prooftree}
				\quad
				\begin{prooftree}
				\hypo{Γ ⊢ A, Δ}
				\hypo{Γ, B ⊢ Δ}
				\infer2{Γ ⊢ A ↛ B, Δ}
				\end{prooftree}
				\]
				
				\[
				\begin{prooftree}
				\hypo{Γ, B ⊢ A, Δ}
				\infer1{Γ, A ↚ B ⊢ Δ}
				\end{prooftree}
				\quad
				\begin{prooftree}
				\hypo{Γ, A ⊢ Δ}
				\hypo{Γ ⊢ B, Δ}
				\infer2{Γ ⊢ A ↚ B, Δ}
				\end{prooftree}
				\]
				
				\[
				\begin{prooftree}
				\hypo{Γ, A ⊢ B, Δ}
				\hypo{Γ, B ⊢ A, Δ}
				\infer2{Γ, A ↮ B ⊢ Δ}
				\end{prooftree}
				\quad
				\begin{prooftree}
				\hypo{Γ ⊢ A, B, Δ}
				\infer0{Γ, A ⊢ A, Δ}
				\infer0{Γ, B ⊢ B, Δ}
				\hypo{Γ, A, B ⊢ Δ}
				\infer4{Γ ⊢ A ↮ B, Δ}
				\end{prooftree}
				\]
			\end{center}
		\end{center}
		
		\subsection{Theorems}
		\begin{center}
		\end{center}

	\section{Structural Affine Calculus}
		\subsection{Structural Rules}
		\begin{center}
			\[
			\begin{prooftree}
			\infer0[Id]{A ⊢ A}
			\end{prooftree}
			\]
			
			\[
			\begin{prooftree}
			\hypo{Γ ⊢ A}
			\hypo{A ⊢ Δ}
			\infer2[Cut]{Γ ⊢ Δ}
			\end{prooftree}
			\]
			
			\[
			\begin{prooftree}
			\hypo{Γ ⊢ Δ}
			\infer1[wL]{Γ, A ⊢ Δ}
			\end{prooftree}
			\qquad
			\begin{prooftree}
			\hypo{Γ ⊢ Δ}
			\infer1[Rw]{Γ ⊢ A, Δ}
			\end{prooftree}
			\]
			
			\[
			\begin{prooftree}
			\hypo{Γ, A, A ⊢ Δ}
			\infer1[cL]{Γ, A ⊢ Δ}
			\end{prooftree}
			\qquad
			\begin{prooftree}
			\hypo{Γ ⊢ A, A Δ}
			\infer1[Rc]{Γ ⊢ A, Δ}
			\end{prooftree}
			\]
			
			\[
			\begin{prooftree}
			\hypo{Γ_0, A, B, Γ_1 ⊢ Δ}
			\infer1[pL]{Γ_0, B, A, Γ_1 ⊢ Δ}
			\end{prooftree}
			\qquad
			\begin{prooftree}
			\hypo{Γ ⊢ Δ_1, A, B, Δ_0}
			\infer1[Rp]{Γ ⊢ Δ_1, B, A, Δ_0}
			\end{prooftree}
			\]
		\end{center}
		
		\subsection{Unit Rules}
		\begin{center}
			\[
			\begin{prooftree}
			\infer0{Γ, ⊥ ⊢ Δ}
			\end{prooftree}
			\quad
			\begin{prooftree}
			\infer0{ Γ ⊢ ⊤, Δ}
			\end{prooftree}
			\]
		\end{center}
		
		\subsection{Operational Rules}
		\begin{center}
			
			\subsubsection{Multiplicatives}
			\begin{center}
								\[
				\begin{prooftree}
				\hypo{Γ ⊢ A, Δ}
				\infer1{Γ, ¬ A ⊢ Δ}
				\end{prooftree}
				\quad
				\begin{prooftree}
				\hypo{Γ, A ⊢ Δ}
				\infer1{Γ ⊢ ¬A, Δ}
				\end{prooftree}
				\]

				\[
				\begin{prooftree}
				\hypo{Γ ⊢ A, B, Δ}
				\infer0{Γ, A ⊢ A, Δ}
				\infer0{Γ, B ⊢ B, Δ}
				\hypo{Γ, A, B ⊢ Δ}
				\infer4{Γ, A ↔ B ⊢ Δ}
				\end{prooftree}
				\quad
				\begin{prooftree}
				\hypo{Γ, A ⊢ B, Δ}
				\hypo{Γ, B ⊢ A, Δ}
				\infer2{Γ ⊢ A ↔ B, Δ}
				\end{prooftree}
				\]
				
				\[
				\begin{prooftree}
				\hypo{Γ, A ⊢ B, Δ}
				\hypo{Γ, B ⊢ A, Δ}
				\infer2{Γ, A ↮ B ⊢ Δ}
				\end{prooftree}
				\quad
				\begin{prooftree}
				\hypo{Γ ⊢ A, B, Δ}
				\infer0{Γ, A ⊢ A, Δ}
				\infer0{Γ, B ⊢ B, Δ}
				\hypo{Γ, A, B ⊢ Δ}
				\infer4{Γ ⊢ A ↮ B, Δ}
				\end{prooftree}
				\]
			\end{center}
		\end{center}
		
		\subsection{Theorems}
		\begin{center}
		\end{center}

	\section{Structural Self-Dual Calculus}
		\subsection{Structural Rules}
		\begin{center}
			\[
			\begin{prooftree}
			\infer0[Id]{A ⊢ A}
			\end{prooftree}
			\]
			
			\[
			\begin{prooftree}
			\hypo{Γ ⊢ A}
			\hypo{A ⊢ Δ}
			\infer2[Cut]{Γ ⊢ Δ}
			\end{prooftree}
			\]
			
			\[
			\begin{prooftree}
			\hypo{Γ ⊢ Δ}
			\infer1[wL]{Γ, A ⊢ Δ}
			\end{prooftree}
			\qquad
			\begin{prooftree}
			\hypo{Γ ⊢ Δ}
			\infer1[Rw]{Γ ⊢ A, Δ}
			\end{prooftree}
			\]
			
			\[
			\begin{prooftree}
			\hypo{Γ, A, A ⊢ Δ}
			\infer1[cL]{Γ, A ⊢ Δ}
			\end{prooftree}
			\qquad
			\begin{prooftree}
			\hypo{Γ ⊢ A, A Δ}
			\infer1[Rc]{Γ ⊢ A, Δ}
			\end{prooftree}
			\]
			
			\[
			\begin{prooftree}
			\hypo{Γ_0, A, B, Γ_1 ⊢ Δ}
			\infer1[pL]{Γ_0, B, A, Γ_1 ⊢ Δ}
			\end{prooftree}
			\qquad
			\begin{prooftree}
			\hypo{Γ ⊢ Δ_1, A, B, Δ_0}
			\infer1[Rp]{Γ ⊢ Δ_1, B, A, Δ_0}
			\end{prooftree}
			\]
		\end{center}
		
		\subsection{Operational Rules}
		\begin{center}
			
			\subsubsection{Multiplicatives}
			\begin{center}
				\[
				\begin{prooftree}
				\hypo{Γ ⊢ A, Δ}
				\infer1{Γ, ¬ A ⊢ Δ}
				\end{prooftree}
				\quad
				\begin{prooftree}
				\hypo{Γ, A ⊢ Δ}
				\infer1{Γ ⊢ ¬A, Δ}
				\end{prooftree}
				\]
			\end{center}
		\end{center}
		
		\subsection{Theorems}
		\begin{center}
		\end{center}

\end{center}

\part{The Commutative Layer}
\begin{center}
	The commutative monotone, truth-preserving, false-preserving, and affine sequent systems.
\end{center}

\part{The Non-Commutative Erasure Layer}
\begin{center}
	The non-commutative erasable monotone, truth-preserving, false-preserving, and affine sequent systems.
\end{center}

\part{The Non-Commutative Cloning Layer}
\begin{center}
	The non-commutative clonable monotone, truth-preserving, false-preserving, and affine sequent systems.
\end{center}

\part{The No-Cloning and No-Erasure Non-Commutative Layer}
\begin{center}
	The non-commutative monotone, truth-preserving, false-preserving, and affine sequent systems.
\end{center}



\end{document}


\chapter{Subclassical Systems}

\documentclass{article}

\usepackage{amsmath}
\usepackage{ebproof}
\usepackage{fullpage}
\usepackage[utf8]{inputenc}
\usepackage{newunicodechar}
\usepackage{stix}

\newunicodechar{Γ}{\Gamma}
\newunicodechar{Δ}{\Delta}

\newunicodechar{Θ}{\Theta}
\newunicodechar{Λ}{\Lambda}

\newunicodechar{Ξ}{\Xi}
\newunicodechar{Π}{\Pi}

\newunicodechar{Φ}{\Phi}
\newunicodechar{Ψ}{\Psi}

\newunicodechar{Ω}{\Omega}

\newunicodechar{⊢}{\vdash}

\newunicodechar{⊕}{\oplus}
\newunicodechar{¬}{\neg}

\newunicodechar{⊗}{\otimes}
\newunicodechar{→}{\rightarrow}
\newunicodechar{←}{\leftarrow}
\newunicodechar{↔}{\leftrightarrow}
\newunicodechar{↮}{\nleftrightarrow}


\newunicodechar{⅋}{\upand}
\newunicodechar{↛}{\nrightarrow}
\newunicodechar{↚}{\nleftarrow}

\newunicodechar{⊥}{\bot}
\newunicodechar{⊤}{\top}

\setlength{\parindent}{0em}

\author{James Martin, Ian D.L.N. Mclean}
\title{The Lattice of Subclassical Sequent Calculi}

\begin{document}

\maketitle

\begin{abstract}
The lattice of sequent calculi formed from the combination of functionally incomplete sets of operations and structural rules in a classical metalanguage; purely classical logical connectives and Boolean functions that are functionally incomplete are isolated from each other, so in order to extend systems of functional incompleteness we need to extend them by at least one non-classical operator.
\end{abstract}

\part{Preliminaries}
\begin{center}
	\begin{flushleft}
		The key concept to the construction of logical calculi that are distinctly different from classical logic is functional incompleteness with respect to the Boolean domain and Boolean functions.
	\end{flushleft}
	\begin{flushleft}
		If a calculus has any classical logical connectives such that we can express every theorem of the classical calculus then our logical calculi would degenerate to the classical calculus and we'd lose in general the specificity that is gained by constructive reasoning.
	\end{flushleft}
	\begin{flushleft}
		Roughly speaking, if any set of operators or functions is not a subset of at least one of the five functionally incomplete sets then that set of operators or functions is functionally complete.
	\end{flushleft}
	\begin{flushleft}
		We generalize this in a manner analogous to Tarski's original formal interpretation as applied to the problem of decidability of theories in standard formalization. If we have collections of non-classical or sub-classical operators or functions that can be interpreted in at least one of the five functionally incomplete sets then those collections of operators or functions are functionally incomplete with respect to the Boolean domain and Boolean functions.
	\end{flushleft}
	\begin{flushleft}
		Finally, if we restrict ourselves to only those logical connectives which are classical then the systems can not extend to each other and it seems can not interpret each other, but if we extend these systems by some non-classical logical connective that is compatible with a given set of functionally incomplete logical connectives then we can extend between these systems and interpret between them.
	\end{flushleft}
	\section{Interpretations}
		\subsection{Operational Interpretations}
		\subsection{Structural Interpretations}
		\subsection{Systematic Interpretations}
	\section{Formal Definition of Subclassical Theories}
		\subsection{Monotonic Theories}
		$\left\{ Γ ⊢ A \right\} \bigcup \left\{ A ⊢ Δ \right\}$
		\subsection{Truth-Preserving Theories}
		$\left\{ A ⊢ Δ \right\}$
		\subsection{False-Preserving Theories}
		$\left\{ Γ ⊢ A \right\}$
		\subsection{Affine Theories}
		$\left\{ Γ ⊢ A \right\} \bigcap \left\{ A ⊢ Δ \right\}=\emptyset$
		\subsection{Self-Dual Theories}
		$\left\{ A ⊢ A \right\}$
	
\end{center}

\part{The Structural Layer}
\begin{center}
	The structural monotone, truth-preserving, false-preserving, and affine sequent systems.
	
	\section{Structural Monotone Calculus}
		\subsection{Structural Rules}
		\begin{center}
			\[
			\begin{prooftree}
			\infer0[Id]{A ⊢ A}
			\end{prooftree}
			\]

			\[
			\begin{prooftree}
			\hypo{Γ ⊢ A}
			\hypo{A ⊢ Δ}
			\infer2[Cut]{Γ ⊢ Δ}
			\end{prooftree}
			\]

			\[
			\begin{prooftree}
			\hypo{Γ ⊢ Δ}
			\infer1[wL]{Γ, A ⊢ Δ}
			\end{prooftree}
			\qquad
			\begin{prooftree}
			\hypo{Γ ⊢ Δ}
			\infer1[Rw]{Γ ⊢ A, Δ}
			\end{prooftree}
			\]

			\[
			\begin{prooftree}
			\hypo{Γ, A, A ⊢ Δ}
			\infer1[cL]{Γ, A ⊢ Δ}
			\end{prooftree}
			\qquad
			\begin{prooftree}
			\hypo{Γ ⊢ A, A Δ}
			\infer1[Rc]{Γ ⊢ A, Δ}
			\end{prooftree}
			\]

			\[
			\begin{prooftree}
			\hypo{Γ_0, A, B, Γ_1 ⊢ Δ}
			\infer1[pL]{Γ_0, B, A, Γ_1 ⊢ Δ}
			\end{prooftree}
			\qquad
			\begin{prooftree}
			\hypo{Γ ⊢ Δ_1, A, B, Δ_0}
			\infer1[Rp]{Γ ⊢ Δ_1, B, A, Δ_0}
			\end{prooftree}
			\]
		\end{center}

		\subsection{Unit Rules}
		\begin{center}
			\[
			\begin{prooftree}
			\infer0{Γ, ⊥ ⊢ Δ}
			\end{prooftree}
			\quad
			\begin{prooftree}
			\infer0{ Γ ⊢ ⊤, Δ}
			\end{prooftree}
			\]
		\end{center}
		
		\subsection{Operational Rules}
		\begin{center}
		
			\subsubsection{Multiplicatives}
			\begin{center}
				\[
				\begin{prooftree}
				\hypo{Γ, A, B ⊢ Δ}
				\infer1{Γ, A ⊗ B ⊢ Δ}
				\end{prooftree}
				\quad
				\begin{prooftree}
				\hypo{Γ ⊢ A, Δ}
				\hypo{Γ ⊢ B, Δ}
				\infer2{Γ ⊢ A ⊗ B, Δ}
				\end{prooftree}
				\]
				
				\[
				\begin{prooftree}
				\hypo{Γ, A ⊢ Δ}
				\hypo{Γ, B ⊢ Δ}
				\infer2{Γ, A ⅋ B ⊢ Δ}
				\end{prooftree}
				\quad
				\begin{prooftree}
				\hypo{Γ ⊢ A, B, Δ}
				\infer1{Γ ⊢ A ⅋ B, Δ}
				\end{prooftree}
				\]
			\end{center}
		\end{center}
		
		\subsection{Theorems}
		\begin{center}
		\end{center}

	\section{Structural Truth-Preserving Calculus}
		\subsection{Structural Rules}
		\begin{center}
			\[
			\begin{prooftree}
			\infer0[Id]{A ⊢ A}
			\end{prooftree}
			\]

			\[
			\begin{prooftree}
			\hypo{Γ ⊢ A}
			\hypo{A ⊢ Δ}
			\infer2[Cut]{Γ ⊢ Δ}
			\end{prooftree}
			\]

			\[
			\begin{prooftree}
			\hypo{Γ ⊢ Δ}
			\infer1[wL]{Γ, A ⊢ Δ}
			\end{prooftree}
			\qquad
			\begin{prooftree}
			\hypo{Γ ⊢ Δ}
			\infer1[Rw]{Γ ⊢ A, Δ}
			\end{prooftree}
			\]

			\[
			\begin{prooftree}
			\hypo{Γ, A, A ⊢ Δ}
			\infer1[cL]{Γ, A ⊢ Δ}
			\end{prooftree}
			\qquad
			\begin{prooftree}
			\hypo{Γ ⊢ A, A Δ}
			\infer1[Rc]{Γ ⊢ A, Δ}
			\end{prooftree}
			\]

			\[
			\begin{prooftree}
			\hypo{Γ_0, A, B, Γ_1 ⊢ Δ}
			\infer1[pL]{Γ_0, B, A, Γ_1 ⊢ Δ}
			\end{prooftree}
			\qquad
			\begin{prooftree}
			\hypo{Γ ⊢ Δ_1, A, B, Δ_0}
			\infer1[Rp]{Γ ⊢ Δ_1, B, A, Δ_0}
			\end{prooftree}
			\]
		\end{center}

		\subsection{Unit Rules}
		\begin{center}
			\[
			\begin{prooftree}
			\infer0{ Γ ⊢ ⊤, Δ}
			\end{prooftree}
			\]
		\end{center}
		
		\subsection{Operational Rules}
		\begin{center}

			\subsubsection{Multiplicatives}
			\begin{center}
				\[
				\begin{prooftree}
				\hypo{Γ, A, B ⊢ Δ}
				\infer1{Γ, A ⊗ B ⊢ Δ}
				\end{prooftree}
				\quad
				\begin{prooftree}
				\hypo{Γ ⊢ A, Δ}
				\hypo{Γ ⊢ B, Δ}
				\infer2{Γ ⊢ A ⊗ B, Δ}
				\end{prooftree}
				\]

				\[
				\begin{prooftree}
				\hypo{Γ, A ⊢ Δ}
				\hypo{Γ, B ⊢ Δ}
				\infer2{Γ, A ⅋ B ⊢ Δ}
				\end{prooftree}
				\quad
				\begin{prooftree}
				\hypo{Γ ⊢ A, B, Δ}
				\infer1{Γ ⊢ A ⅋ B, Δ}
				\end{prooftree}
				\]

				\[
				\begin{prooftree}
				\hypo{Γ ⊢ A, Δ}
				\hypo{Γ, B ⊢ Δ}
				\infer2{Γ, A → B ⊢ Δ}
				\end{prooftree}
				\quad
				\begin{prooftree}
				\hypo{Γ, A ⊢ B, Δ}
				\infer1{Γ ⊢ A → B, Δ}
				\end{prooftree}
				\]

				\[
				\begin{prooftree}
				\hypo{Γ, A ⊢ Δ}
				\hypo{Γ ⊢ B, Δ}
				\infer2{Γ, A ← B ⊢ Δ}
				\end{prooftree}
				\quad
				\begin{prooftree}
				\hypo{Γ, B ⊢ A, Δ}
				\infer1{Γ ⊢ A ← B, Δ}
				\end{prooftree}
				\]

				\[
				\begin{prooftree}
				\hypo{Γ ⊢ A, B, Δ}
				\infer0{Γ, A ⊢ A, Δ}
				\infer0{Γ, B ⊢ B, Δ}
				\hypo{Γ, A, B ⊢ Δ}
				\infer4{Γ, A ↔ B ⊢ Δ}
				\end{prooftree}
				\quad
				\begin{prooftree}
				\hypo{Γ, A ⊢ B, Δ}
				\hypo{Γ, B ⊢ A, Δ}
				\infer2{Γ ⊢ A ↔ B, Δ}
				\end{prooftree}
				\]
			\end{center}
		\end{center}

		\subsection{Theorems}
			\begin{center}
			\end{center}

	\section{Structural False-Preserving Calculus}
		\subsection{Structural Rules}
		\begin{center}
			\[
			\begin{prooftree}
			\infer0[Id]{A ⊢ A}
			\end{prooftree}
			\]

			\[
			\begin{prooftree}
			\hypo{Γ ⊢ A}
			\hypo{A ⊢ Δ}
			\infer2[Cut]{Γ ⊢ Δ}
			\end{prooftree}
			\]
			
			\[
			\begin{prooftree}
			\hypo{Γ ⊢ Δ}
			\infer1[wL]{Γ, A ⊢ Δ}
			\end{prooftree}
			\qquad
			\begin{prooftree}
			\hypo{Γ ⊢ Δ}
			\infer1[Rw]{Γ ⊢ A, Δ}
			\end{prooftree}
			\]

			\[
			\begin{prooftree}
			\hypo{Γ, A, A ⊢ Δ}
			\infer1[cL]{Γ, A ⊢ Δ}
			\end{prooftree}
			\qquad
			\begin{prooftree}
			\hypo{Γ ⊢ A, A Δ}
			\infer1[Rc]{Γ ⊢ A, Δ}
			\end{prooftree}
			\]

			\[
			\begin{prooftree}
			\hypo{Γ_0, A, B, Γ_1 ⊢ Δ}
			\infer1[pL]{Γ_0, B, A, Γ_1 ⊢ Δ}
			\end{prooftree}
			\qquad
			\begin{prooftree}
			\hypo{Γ ⊢ Δ_1, A, B, Δ_0}
			\infer1[Rp]{Γ ⊢ Δ_1, B, A, Δ_0}
			\end{prooftree}
			\]
		\end{center}
		
		\subsection{Unit Rules}
		\begin{center}
			\[
			\begin{prooftree}
			\infer0{Γ, ⊥ ⊢ Δ}
			\end{prooftree}
			\]
		\end{center}
		
		\subsection{Operational Rules}
		\begin{center}

			\subsubsection{Multiplicatives}
			\begin{center}
				\[
				\begin{prooftree}
				\hypo{Γ, A, B ⊢ Δ}
				\infer1{Γ, A ⊗ B ⊢ Δ}
				\end{prooftree}
				\quad
				\begin{prooftree}
				\hypo{Γ ⊢ A, Δ}
				\hypo{Γ ⊢ B, Δ}
				\infer2{Γ ⊢ A ⊗ B, Δ}
				\end{prooftree}
				\]

				\[
				\begin{prooftree}
				\hypo{Γ, A ⊢ Δ}
				\hypo{Γ, B ⊢ Δ}
				\infer2{Γ, A ⅋ B ⊢ Δ}
				\end{prooftree}
				\quad
				\begin{prooftree}
				\hypo{Γ ⊢ A, B, Δ}
				\infer1{Γ ⊢ A ⅋ B, Δ}
				\end{prooftree}
				\]

				\[
				\begin{prooftree}
				\hypo{Γ, A ⊢ B, Δ}
				\infer1{Γ, A ↛ B ⊢ Δ}
				\end{prooftree}
				\quad
				\begin{prooftree}
				\hypo{Γ ⊢ A, Δ}
				\hypo{Γ, B ⊢ Δ}
				\infer2{Γ ⊢ A ↛ B, Δ}
				\end{prooftree}
				\]

				\[
				\begin{prooftree}
				\hypo{Γ, B ⊢ A, Δ}
				\infer1{Γ, A ↚ B ⊢ Δ}
				\end{prooftree}
				\quad
				\begin{prooftree}
				\hypo{Γ, A ⊢ Δ}
				\hypo{Γ ⊢ B, Δ}
				\infer2{Γ ⊢ A ↚ B, Δ}
				\end{prooftree}
				\]

				\[
				\begin{prooftree}
				\hypo{Γ, A ⊢ B, Δ}
				\hypo{Γ, B ⊢ A, Δ}
				\infer2{Γ, A ↮ B ⊢ Δ}
				\end{prooftree}
				\quad
				\begin{prooftree}
				\hypo{Γ ⊢ A, B, Δ}
				\infer0{Γ, A ⊢ A, Δ}
				\infer0{Γ, B ⊢ B, Δ}
				\hypo{Γ, A, B ⊢ Δ}
				\infer4{Γ ⊢ A ↮ B, Δ}
				\end{prooftree}
				\]
			\end{center}
		\end{center}
		
		\subsection{Theorems}
		\begin{center}
		\end{center}

	\section{Structural Affine Calculus}
		\subsection{Structural Rules}
		\begin{center}
			\[
			\begin{prooftree}
			\infer0[Id]{A ⊢ A}
			\end{prooftree}
			\]

			\[
			\begin{prooftree}
			\hypo{Γ ⊢ A}
			\hypo{A ⊢ Δ}
			\infer2[Cut]{Γ ⊢ Δ}
			\end{prooftree}
			\]

			\[
			\begin{prooftree}
			\hypo{Γ ⊢ Δ}
			\infer1[wL]{Γ, A ⊢ Δ}
			\end{prooftree}
			\qquad
			\begin{prooftree}
			\hypo{Γ ⊢ Δ}
			\infer1[Rw]{Γ ⊢ A, Δ}
			\end{prooftree}
			\]

			\[
			\begin{prooftree}
			\hypo{Γ, A, A ⊢ Δ}
			\infer1[cL]{Γ, A ⊢ Δ}
			\end{prooftree}
			\qquad
			\begin{prooftree}
			\hypo{Γ ⊢ A, A Δ}
			\infer1[Rc]{Γ ⊢ A, Δ}
			\end{prooftree}
			\]

			\[
			\begin{prooftree}
			\hypo{Γ_0, A, B, Γ_1 ⊢ Δ}
			\infer1[pL]{Γ_0, B, A, Γ_1 ⊢ Δ}
			\end{prooftree}
			\qquad
			\begin{prooftree}
			\hypo{Γ ⊢ Δ_1, A, B, Δ_0}
			\infer1[Rp]{Γ ⊢ Δ_1, B, A, Δ_0}
			\end{prooftree}
			\]
		\end{center}
		
		\subsection{Unit Rules}
		\begin{center}
			\[
			\begin{prooftree}
			\infer0{Γ, ⊥ ⊢ Δ}
			\end{prooftree}
			\quad
			\begin{prooftree}
			\infer0{ Γ ⊢ ⊤, Δ}
			\end{prooftree}
			\]
		\end{center}

		\subsection{Operational Rules}
		\begin{center}

			\subsubsection{Multiplicatives}
			\begin{center}
								\[
				\begin{prooftree}
				\hypo{Γ ⊢ A, Δ}
				\infer1{Γ, ¬ A ⊢ Δ}
				\end{prooftree}
				\quad
				\begin{prooftree}
				\hypo{Γ, A ⊢ Δ}
				\infer1{Γ ⊢ ¬A, Δ}
				\end{prooftree}
				\]

				\[
				\begin{prooftree}
				\hypo{Γ ⊢ A, B, Δ}
				\infer0{Γ, A ⊢ A, Δ}
				\infer0{Γ, B ⊢ B, Δ}
				\hypo{Γ, A, B ⊢ Δ}
				\infer4{Γ, A ↔ B ⊢ Δ}
				\end{prooftree}
				\quad
				\begin{prooftree}
				\hypo{Γ, A ⊢ B, Δ}
				\hypo{Γ, B ⊢ A, Δ}
				\infer2{Γ ⊢ A ↔ B, Δ}
				\end{prooftree}
				\]

				\[
				\begin{prooftree}
				\hypo{Γ, A ⊢ B, Δ}
				\hypo{Γ, B ⊢ A, Δ}
				\infer2{Γ, A ↮ B ⊢ Δ}
				\end{prooftree}
				\quad
				\begin{prooftree}
				\hypo{Γ ⊢ A, B, Δ}
				\infer0{Γ, A ⊢ A, Δ}
				\infer0{Γ, B ⊢ B, Δ}
				\hypo{Γ, A, B ⊢ Δ}
				\infer4{Γ ⊢ A ↮ B, Δ}
				\end{prooftree}
				\]
			\end{center}
		\end{center}

		\subsection{Theorems}
		\begin{center}
		\end{center}

	\section{Structural Self-Dual Calculus}
		\subsection{Structural Rules}
		\begin{center}
			\[
			\begin{prooftree}
			\infer0[Id]{A ⊢ A}
			\end{prooftree}
			\]
			
			\[
			\begin{prooftree}
			\hypo{Γ ⊢ A}
			\hypo{A ⊢ Δ}
			\infer2[Cut]{Γ ⊢ Δ}
			\end{prooftree}
			\]

			\[
			\begin{prooftree}
			\hypo{Γ ⊢ Δ}
			\infer1[wL]{Γ, A ⊢ Δ}
			\end{prooftree}
			\qquad
			\begin{prooftree}
			\hypo{Γ ⊢ Δ}
			\infer1[Rw]{Γ ⊢ A, Δ}
			\end{prooftree}
			\]
			
			\[
			\begin{prooftree}
			\hypo{Γ, A, A ⊢ Δ}
			\infer1[cL]{Γ, A ⊢ Δ}
			\end{prooftree}
			\qquad
			\begin{prooftree}
			\hypo{Γ ⊢ A, A Δ}
			\infer1[Rc]{Γ ⊢ A, Δ}
			\end{prooftree}
			\]
			
			\[
			\begin{prooftree}
			\hypo{Γ_0, A, B, Γ_1 ⊢ Δ}
			\infer1[pL]{Γ_0, B, A, Γ_1 ⊢ Δ}
			\end{prooftree}
			\qquad
			\begin{prooftree}
			\hypo{Γ ⊢ Δ_1, A, B, Δ_0}
			\infer1[Rp]{Γ ⊢ Δ_1, B, A, Δ_0}
			\end{prooftree}
			\]
		\end{center}
		
		\subsection{Operational Rules}
		\begin{center}

			\subsubsection{Multiplicatives}
			\begin{center}
				\[
				\begin{prooftree}
				\hypo{Γ ⊢ A, Δ}
				\infer1{Γ, ¬ A ⊢ Δ}
				\end{prooftree}
				\quad
				\begin{prooftree}
				\hypo{Γ, A ⊢ Δ}
				\infer1{Γ ⊢ ¬A, Δ}
				\end{prooftree}
				\]
			\end{center}
		\end{center}

		\subsection{Theorems}
		\begin{center}
		\end{center}

\end{center}

\part{The Commutative Layer}
\begin{center}
	The commutative monotone, truth-preserving, false-preserving, and affine sequent systems.
\end{center}

\part{The Non-Commutative Erasure Layer}
\begin{center}
	The non-commutative erasable monotone, truth-preserving, false-preserving, and affine sequent systems.
\end{center}

\part{The Non-Commutative Cloning Layer}
\begin{center}
	The non-commutative clonable monotone, truth-preserving, false-preserving, and affine sequent systems.
\end{center}

\part{The No-Cloning and No-Erasure Non-Commutative Layer}
\begin{center}
	The non-commutative monotone, truth-preserving, false-preserving, and affine sequent systems.
\end{center}



\end{document}


\chapter{Theories in Standard Formalization}
}

A theory T in standard formalization is a pair
(\textbf{Lang},\textbf{Axioms}), where:

\textbf{Lang} is a first-order language with the following components:

\begin{itemize}
\item
  \begin{quote}
  A set of variables V.
  \end{quote}
\item
  \begin{quote}
  A set of non-logical constants C.
  \end{quote}
\item
  \begin{quote}
  A set of logical constants, which includes the following symbols:
  \end{quote}

  \begin{itemize}
  \item
    \begin{quote}
    The connectives: $\neg$ , $\lor$ , $\land$ , $\to$ , $\leftrightarrow$ .
    \end{quote}
  \item
    \begin{quote}
    The quantifiers: $\forall$ , $\exists$ .
    \end{quote}
  \item
    \begin{quote}
    The identity symbol =.
    \end{quote}
  \end{itemize}
\item
  \begin{quote}
  A set of function symbols F, where each function symbol f has a
  natural number n as its rank, and for each n-tuple of terms t1, t2,
  ..., tn, the function symbol f applied to t1, t2, ..., tn is a term.
  \end{quote}
\item
  \begin{quote}
  A set of predicate symbols P, where each predicate symbol p has a
  natural number n as its rank, and for each n-tuple of terms t1, t2,
  ..., tn, the predicate symbol p applied to t1, t2, ..., tn is a
  formula.
  \end{quote}
\end{itemize}

A is a set of formulas in \textbf{L}. The formulas in \textbf{Axioms}
are called the axioms of T.

A theory T is said to be satisfiable if there exists a model M of T. A
model M of T is a structure M = (D, ⊦) such that:

\begin{itemize}
\item
  \begin{quote}
  D is a non-empty set called the domain of M.
  \end{quote}
\item
  \begin{quote}
  ⊦ is an interpretation function that assigns to each constant c in C
  an element d of D, to each function symbol f in F an n-ary function
  f\^{}M from D\^{}n to D, and to each predicate symbol p in P an n-ary
  relation p\^{}M over D.
  \end{quote}
\item
  \begin{quote}
  For each axiom $\alpha$  in \textbf{Axioms}, $\alpha$  is true in M, where "true in M"
  is defined inductively as follows:
  \end{quote}

  \begin{itemize}
  \item
    \begin{quote}
    Variables are interpreted as themselves.
    \end{quote}
  \item
    \begin{quote}
    $\neg$ $\alpha$  is true in M if $\alpha$  is not true in M.
    \end{quote}
  \item
    \begin{quote}
    $\alpha$  $\lor$  $\beta$  is true in M if $\alpha$  is true in M or $\beta$  is true in M (or both).
    \end{quote}
  \item
    \begin{quote}
    $\alpha$  $\land$  $\beta$  is true in M if $\alpha$  is true in M and $\beta$  is true in M.
    \end{quote}
  \item
    \begin{quote}
    $\alpha$  $\to$  $\beta$  is true in M if $\alpha$  is not true in M or $\beta$  is true in M.
    \end{quote}
  \item
    \begin{quote}
    $\forall$ x$\alpha$ (x) is true in M if for all elements d of D, $\alpha$ (d) is true in M.
    \end{quote}
  \item
    \begin{quote}
    $\exists$ x$\alpha$ (x) is true in M if there exists an element d of D such that $\alpha$ (d)
    is true in M.
    \end{quote}
  \end{itemize}
\end{itemize}

If there exists no model of T, then T is said to be unsatisfiable.

}*}


\chapter{UD Theorem Summary}
}

$\exists$ $\forall$ $\neg$ $\land$ $\lor$ $\to$ $\nrightarrow$ $\leftrightarrow$ $\oplus$

\hypertarget{theorem-1-of-undecidable-theories-part-i}{%
\subsection{Theorem 1 of Undecidable Theories Part
I}\label{theorem-1-of-undecidable-theories-part-i}}

For all theories T in standard formalization.

If T is complete then (T is decidable ⨁ T is essentially undecidable and
T is decidable ⇔ T is axiomatizable).

\hypertarget{theorem-1-complete}{%
\subsubsection{\texorpdfstring{Theorem 1 COMPLETE
}{Theorem 1 COMPLETE }}\label{theorem-1-complete}}

For any complete theory T the following conditions are equivalent: T is
undecidable. T is essentially undecidable. T is non-axiomatizable.

\hypertarget{theorem-1-axiom}{%
\subsubsection{Theorem 1 AXIOM ?}\label{theorem-1-axiom}}

For any non-axiomatizable theory T the following three conditions are
equivalent: T is undecidable. T is essentially undecidable. T is
complete?

\hypertarget{theorem-1-consist}{%
\subsubsection{Theorem 1 CONSIST ?}\label{theorem-1-consist}}

For every non-consistent theory the following are equivalent:

T is finitely axiomatizable, T is axiomatizable, T is complete.

fAX(T), AX(T), and Complete(T).

\hypertarget{theorem-1-decide-x}{%
\subsubsection{Theorem 1 DECIDE X}\label{theorem-1-decide-x}}

\st{For any decidable theory T the following three conditions are
equivalent: T is axiomatizable. T is finitely axiomatizable. T is
consistent.}\\
\strut \\
Probably deals with SU(T) and not complete(T).

\hypertarget{theorem-2-of-undecidable-theories-part-i}{%
\subsection{Theorem 2 of Undecidable Theories Part
I}\label{theorem-2-of-undecidable-theories-part-i}}

For all theories T in standard formalization.

T is essentially undecidable if and only if T is consistent and:\\
for all T\_e at least one of the following properties is false

T\_e is consistent.

T\_e is complete.

T\_e is an extension of T

T\_e has the same constants as T.

T\_e is axiomatizable.

For all theories T in standard formalization.

T is essentially undecidable if and only if T is consistent and:

for all T\_e. (T\_e is paraconsistent, T\_e is complete, T\_e is an
extension of T, T\_e has the same constants as T, T\_e is recursively
enumerably axiomatizable).

For all theories T in standard formalization.

T is essentially undecidable if and only if T is consistent and: for all
T\_e. (T\_e is consistent, T\_e is paracomplete, T\_e is an extension of
T, T\_e has the same constants as T, T\_e is recursively enumerably
axiomatizable).

For all theories T in standard formalization.

T is essentially undecidable if and only if T is consistent and: for all
T\_e. (T\_e is consistent, T\_e is complete, T\_e is a non-conservative
extension of T, T\_e has the same constants as T, T\_e is recursively
enumerably axiomatizable).

For all theories T in standard formalization.

T is essentially undecidable if and only if T is consistent and: for all
T\_e. (T\_e is consistent, T\_e is complete, T\_e is an extension of T,
T\_e has the different constants than T, T\_e is recursively enumerably
axiomatizable).

For all theories T in standard formalization.

T is essentially undecidable if and only if T is consistent and: for all
T\_e. (T\_e is consistent, T\_e is complete, T\_e is a extension of T,
T\_e has the same constants as T, T\_e is non-axiomatizable).

\hypertarget{lemma}{%
\subsubsection{\texorpdfstring{Lemma }{Lemma }}\label{lemma}}

$\forall$ T' in standard formalization.

Consistent(T') and Decide(T') if and only if
$\exists$ T\textquotesingle'{[}Consistent(T'\textquotesingle)$\land$
Complete(T'\textquotesingle)$\land$ Decide(T'\textquotesingle)$\land$ Axiomatize(T\textquotesingle')$\land$ Extend(T',
T\textquotesingle\textquotesingle)$\land$ sameConstants(T',
T\textquotesingle\textquotesingle){]}.

Equivalently,

$\forall$ T' in standard formalization.

NAND(Consistent(T'), Decide(T')) if and only if
$\forall$ T\textquotesingle'{[}$\neg$ Consistent(T'\textquotesingle)$\lor$ $\neg$ Complete(T'\textquotesingle)$\lor$ $\neg$ Decide(T'\textquotesingle)$\lor$ $\neg$ Axiomatize(T\textquotesingle')$\lor$ $\neg$ Extend(T',
T\textquotesingle\textquotesingle){]}.

\hypertarget{theorem-2-for-hereditary-undecidability}{%
\subsection{Theorem 2 for Hereditary
Undecidability}\label{theorem-2-for-hereditary-undecidability}}

For all theories T in standard formalization.

T is hereditarily undecidable if and only if T is consistent and:\\
for all T\_s at least one of the following properties is false

T\_s is consistent.

T\_s is complete.

T\_s is a subtheory of T

T\_s has the same constants as T.

T\_s is axiomatizable.

\hypertarget{theorem-2-exclusive-essential-undecidability-of-t}{%
\subsection{Theorem 2: Exclusive Essential Undecidability of
T}\label{theorem-2-exclusive-essential-undecidability-of-t}}

EU(T)$\nrightarrow$ HU(T) iff Consistent(T) $\land$  $\forall$ T\_e $\exists$ T\_s{[}(Extend(T,
T\_e)$\to$ C(T\_e))$\land$ (Subtend(T, T\_s)$\nrightarrow$ C(T\_s)){]}

\hypertarget{theorem-2-exclusive-hereditary-undecidability-of-t}{%
\subsection{Theorem 2: Exclusive Hereditary Undecidability of
T}\label{theorem-2-exclusive-hereditary-undecidability-of-t}}

HU(T)$\nrightarrow$ EU(T) iff Consistent(T) $\land$  $\forall$ T\_s $\exists$ T\_e{[}(Subtend(T,
T\_s)$\to$ C(T\_s))$\land$ (Extend(T, T\_e)$\nrightarrow$ C(T\_e)){]}

\hypertarget{theorem-2-total-undecidability-of-t}{%
\subsection{Theorem 2: Total Undecidability of
T}\label{theorem-2-total-undecidability-of-t}}

C(x) = $\neg$ Consistent(x)$\lor$ $\neg$ Complete(x)$\lor$ $\neg$ Axiomatize(x)

EU(T) and HU(T) iff Consistent(T) $\land$  $\forall$ Y. ( (Extend(T, Y) or Subtend(T, Y)
) $\to$  C(Y) )

\hypertarget{theorem-2-denial-of-the-essential-or-hereditary-undecidability-of-t}{%
\subsection{Theorem 2: Denial of the Essential or Hereditary
Undecidability of
T}\label{theorem-2-denial-of-the-essential-or-hereditary-undecidability-of-t}}

Neither EU(T) nor HU(T) iff Consistent(T)$\to$ {[}$\exists$ T\_e. (Extend(T,
T\_e)$\nrightarrow$ C(T\_e))$\land$ $\exists$ T\_s.(Subtend(T, T\_s)$\nrightarrow$ C(T\_s)){]}

\hypertarget{generalization-of-theorem-2-of-ut}{%
\subsection{Generalization of Theorem 2 of
UT}\label{generalization-of-theorem-2-of-ut}}

C(x) = $\neg$ Consistent(x)$\lor$ $\neg$ Complete(x)$\lor$ $\neg$ Axiomatize(x)$\lor$ $\neg$ sameConstants(x)

EU(T) and HU(T) iff Consistent(T) $\land$  $\forall$ Y. ( (Extend(T, Y) or Subtend(T, Y)
) $\to$  C(Y) )

EU(T) and not HU(T) iff Consistent(T) $\land$  $\forall$ T\_e $\exists$ T\_s{[}Extend(T,
T\_e)$\to$ C(T\_e){]}$\land$ {[}Subtend(T, T\_s)$\land$ $\neg$ C(T\_s){]}

not EU(T) and HU(T) iff Consistent(T) $\land$  $\forall$ T\_s $\exists$ T\_e{[}Subtend(T,
T\_s)$\to$ C(T\_s){]}$\land$ {[}Extend(T, T\_e)$\land$ $\neg$ C(T\_e){]}

Neither EU(T) nor HU(T) iff Consistent(T)$\to$ {[}$\exists$ T\_e. (Extend(T,
T\_e)$\nrightarrow$ C(T\_e))$\land$ $\exists$ T\_s.(Subtend(T, T\_s)$\nrightarrow$ C(T\_s)){]}

\hypertarget{theorem-3-of-ut-for-hu}{%
\subsection{Theorem 3 of UT for HU}\label{theorem-3-of-ut-for-hu}}

Let T\_1 and T\_2 be two theories such that T\_1 is a consistent
subtheory of T\_2. If HU(T\_2) then HU(T\_1).

\hypertarget{theorem-3-of-ut-for-tu}{%
\subsection{Theorem 3 of UT for TU}\label{theorem-3-of-ut-for-tu}}

If any theory T is isomorphic to or an extension of or a subtheory of a
totally undecidable theory then that theory is totally undecidable.

\hypertarget{definition}{%
\subsection{Definition}\label{definition}}

A subtheory T\_1 of T\_2 is called inessential if every constant of T\_2
which does not occur in T\_1 is an individual constant and if every
valid sentence of T\_2 is derivable in T\_2 from a set of valid
sentences of T\_1.

\hypertarget{theorem-4-of-ut-for-hu}{%
\subsection{Theorem 4 of UT for HU}\label{theorem-4-of-ut-for-hu}}

Let T\_1 and T\_2 be two theories such that T\_1 is an inessential
subtheory of T\_2.

T\_1 is undecidable or hereditarily undecidable if and only if T\_2 is
respectively undecidable or hereditarily undecidable.

\hypertarget{generalization-of-theorem-4-of-ut}{%
\subsection{Generalization of Theorem 4 of
UT}\label{generalization-of-theorem-4-of-ut}}

Let T\_1 and T\_2 be two theories such that T\_2 is an inessential
extension of T\_1 or T\_1 is an inessential subtheory of T\_2.

T\_1 is undecidable or totally undecidable if and only if T\_2 is
respectively undecidable or totally undecidable.

\hypertarget{theorem-5-of-ut-for-finite-subtheories}{%
\subsection{Theorem 5 of UT for Finite
Subtheories}\label{theorem-5-of-ut-for-finite-subtheories}}

Let T\_1 and T\_2 be two theories with the same constants such that T\_1
is a finite subtheory of T\_2.

If T\_1 is undecidable then T\_2 is also undecidable.

\hypertarget{generalization-of-theorem-6-of-ut}{%
\subsection{Generalization of Theorem 6 of
UT}\label{generalization-of-theorem-6-of-ut}}

Let T\_1 and T\_2 be two compatible theories such that every constant of
T\_2 is also a constant of T\_1. If {[}EU(T\_2) or TU(T\_2){]} and
fAxiomatize(T\_2), then {[}HU(T\_1) or TU(T\_1){]}.

\hypertarget{theorem}{%
\subsection{Theorem}\label{theorem}}

If a theory is essentially undecidable and it has a hereditarily
undecidable extension then both the theory and its supertheory is
totally undecidable.

For all T in STDF.

If HU(T) and there exists T\textquotesingle{} {[}EU(Subtend(T,
T\textquotesingle){]} then TU(T) and TU(T\textquotesingle).

\hypertarget{theorem-1}{%
\subsection{Theorem}\label{theorem-1}}

If a theory is essentially undecidable and a hereditarily undecidable
theory can be interpreted in an extension of the essentially undecidable
theory then the theory, the hereditarily undecidable theory, and the
extension of the essentially undecidable theory are all totally
undecidable.

\hypertarget{theorem-2}{%
\subsection{Theorem}\label{theorem-2}}

If a theory is hereditarily undecidable and it has an essentially
undecidable subtheory then both the theory and its subtheory are totally
undecidable.

For all T in STDF.

If EU(T) and there exists T\textquotesingle{}
{[}HU(Extend(T,T\textquotesingle){]} then TU(T) and
TU(T\textquotesingle).

\hypertarget{theorem-3}{%
\subsection{Theorem}\label{theorem-3}}

If a theory is hereditarily undecidable and an essentially undecidable
theory can be interpreted in a subtheory of the essentially undecidable
theory then the theory, the essentially undecidable theory, and the
subtheory of the hereditarily undecidable theory are all totally
undecidable.

\hypertarget{theorem-4}{%
\subsection{Theorem}\label{theorem-4}}

For all theories T in standard formalization.

T is totally undecidable if and only if there does not exist a
consistent decidable extension of HU(T) and there does not exist a
consistent decidable subtheory of EU(T).

\hypertarget{definition-1}{%
\subsection{Definition}\label{definition-1}}

A theory is said to be decidedly pinched if it only has HU subtensions
and EU extensions.

\hypertarget{definition-2}{%
\subsection{Definition}\label{definition-2}}

A theory is said to be essentially decidable if and only if the theory
is decidable and every extension of the theory is decidable.

\hypertarget{definition-3}{%
\subsection{Definition}\label{definition-3}}

A theory is said to be hereditarily decidable if and only if the theory
is decidable and every subtension of the theory is decidable.

\hypertarget{definition-4}{%
\subsection{Definition}\label{definition-4}}

A theory is said to be totally decidable if and only if the theory is
decidable and every extension and subtension of the theory is decidable.

\hypertarget{conjecture}{%
\subsection{Conjecture}\label{conjecture}}

There exists some morphism that preserves the non-triviality of a theory
in transformations from theory to theory.

\hypertarget{conjecture-1}{%
\subsection{Conjecture}\label{conjecture-1}}

There exists some morphism that preserves non-triviality and expressive
strength of a theory from theory to theory.

\hypertarget{lemma-of-theorem-2-of-ut}{%
\subsection{Lemma of Theorem 2 of UT}\label{lemma-of-theorem-2-of-ut}}

For all T' in standard formalization.

Consistent(T') and Decide(T') if and only if T' has a Consistent(T'),
Complete(T'), Decide(T'), Axiomatize(T'), Extend(T').

\hypertarget{theorem-5}{%
\subsection{Theorem}\label{theorem-5}}

A theory is essentially decidable if and only if Axiomatize(T) and for
all T\_e, Extend(T, T\_e) and Axiomatize(T\_e).

\hypertarget{theorem-6}{%
\subsection{Theorem}\label{theorem-6}}

A theory is totally decidable if and only if Axiomatize(T) and for all
Y, {[}Extend(T, Y) or Subtend(T, Y){]} and Axiomatize(Y).

\hypertarget{theorem-7}{%
\subsection{Theorem}\label{theorem-7}}

A theory is totally axiomatizable if and only if Axiomatize(T) and for
all Y, {[}Extend(T, Y) or Subtend(T, Y){]} and Axiomatize(Y).

\hypertarget{summary-of-combinations}{%
\subsection{Summary of combinations}\label{summary-of-combinations}}

{[}Complete(x)$\leftrightarrow$ Axiomatize(x){]}$\to$ Decide(x).

$\neg$ Decide(x)$\to$ {[}Complete(x)$\oplus$ Axiomatize(x){]}.

SU(T)$\oplus$ Complete(T)

Any axiomatizable theory in SU will not be complete.

Theories which are axiomatizable can be in Decide, SU, or EU.

Theories which are non-axiomatizable can be in SU or EU or TU.

Theories which are complete can be in Decide or EU.

Theories which are not complete can be in Decide, SU, or EU.\\
\strut \\
Theories which are finitely axiomatizable can be in Decide, SU, or EU.

Theories which are not finitely axiomatizable can be in Decide, SU, or
EU.\\
\strut \\
Theories which are consistent can be in Decide, SU, or EU.\\
Theories which are not consistent can be in Decide or SU(?).\\
\strut \\
Theories which are inconsistent strictly in Decide.

Theories which are not axiomatizable and complete are strictly in EU.

Theories which are axiomatizable and not complete can be in Decide, SU,
or EU.

Theories which are not axiomatizable and not complete can be in SU or
EU.

Theories which are axiomatizable and complete strictly in Decide.

Theories which are finitely axiomatizable and complete are strictly in
Decide.

Theories which are finitely axiomatizable and not complete can be in
Decide, SU, or EU.

Theories which are not finitely axiomatizable and complete can be in
Decide, or EU.

Theories which are not finitely axiomatizable and not complete can be in
Decide, SU, or EU.

\hypertarget{jointly-exhaustive-and-mutually-exclusive-theories-theorem}{%
\subsection{Jointly Exhaustive and Mutually Exclusive Theories
Theorem}\label{jointly-exhaustive-and-mutually-exclusive-theories-theorem}}

{[}Axiomatize(T)$\nrightarrow$ fAxiomatize(T){]}$\oplus$ fAxiomatize(T)$\oplus$ $\neg$ Axiomatize(T) if and
only if Inconsistent(T)$\oplus$ Paraconsistent(T)$\oplus$ Consistent(T) if and only if
Decide(T)$\oplus$ SU(T)$\oplus$ EU(T) if and only if Incomplete$\oplus$ Complete$\oplus$ Overcomplete.

\hypertarget{inductive-su-theorem}{%
\subsection{Inductive SU Theorem}\label{inductive-su-theorem}}

Every SU(T) has a SU(T) extension that has a Consistent, Complete,
Decidable, and Axiomatizable extension.

\hypertarget{recursive-su-theorem}{%
\subsection{Recursive SU Theorem}\label{recursive-su-theorem}}

Every SU(T) has a SU(T) and a Decide(T) extension.

\hypertarget{theorem-8}{%
\subsection{Theorem}\label{theorem-8}}

The inductive and recursive definitions of SU(T) are equivalent.

\hypertarget{axiom}{%
\subsection{AXIOM}\label{axiom}}

\hypertarget{theorem-9}{%
\subsubsection{Theorem}\label{theorem-9}}

\st{A theory in standard formalization is axiomatizable if the set of
non-logical axioms of the theory can be put in one to one correspondence
with a countably infinite or countably finite set}; if the set is
countably infinite then the non-logical axioms form a set of all finite
strings of binary digits.

\hypertarget{theorem-10}{%
\subsubsection{Theorem}\label{theorem-10}}

A theory in standard formalization is finitely axiomatizable if the set
of non-logical axioms of the theory can be put into one to one
correspondence with a finite string of binary digits.

\hypertarget{theorem-11}{%
\subsubsection{Theorem}\label{theorem-11}}

\st{A theory in standard formalization is non-axiomatizable if the set
of non-logical axioms can neither be put into one to one correspondence
with a finite string of binary digits nor with a set of all finite
strings of binary digits.}

Equivalently, a theory in standard formalization is non-axiomatizable if
axiomatizing the theory would entail an arbitrary contradiction.

\hypertarget{conjecture-2}{%
\subsubsection{Conjecture}\label{conjecture-2}}

There exists theories in standard formalization that are relatively
non-axiomatizable in paraconsistent semantics.

\hypertarget{definition-5}{%
\paragraph{Definition}\label{definition-5}}

A paraconsistent theory is relatively (in classical logic)
non-axiomatizable if the contradiction that would be entailed by
axiomatization of the theory is tolerable or removable in the
paraconsistent theory or the paraconsistent metatheory.

\hypertarget{conjecture-3}{%
\subsubsection{Conjecture}\label{conjecture-3}}

There exists theories in standard formalization that are absolutely
non-axiomatizable in paraconsistent semantics.

\hypertarget{theorem-12}{%
\subsubsection{\texorpdfstring{\st{Theorem}}{Theorem}}\label{theorem-12}}

\st{A theory is finitely axiomatizable if and only if the theory is
axiomatizable and the complement of the theory is axiomatizable.}

\st{fAX(T) iff AX(T) $\leftrightarrow$  AX($\neg$ T)}

\hypertarget{theorem-13}{%
\subsubsection{\texorpdfstring{\st{Theorem}}{Theorem}}\label{theorem-13}}

\st{A theory is non-finitely axiomatizable if and only if the theory is
axiomatizable and the complement of the theory is non-axiomatizable.}

\st{iAX(T) iff AX(T) and $\neg$ AX($\neg$ T)}

\hypertarget{theorem-14}{%
\subsubsection{\texorpdfstring{\st{Theorem}}{Theorem}}\label{theorem-14}}

\st{A theory is non-axiomatizable if and only if the axiomatizability of
the theory implies that the complement of the theory is axiomatizable
and non-axiomatizable.}

\st{nAX(T) iff AX(T)$\to$ {[}AX($\neg$ T) and $\neg$ AX($\neg$ T){]}}

\hypertarget{theorem-15}{%
\subsection{\texorpdfstring{\st{Theorem}}{Theorem}}\label{theorem-15}}

A paraconsistent theory is complete if and only if the complement of
that theory is paraconsistent and complete;

This is equivalent to saying that the theory and its complement are
axiomatizable which is equivalent to saying that the theory and its
complement are finitely axiomatizable.

A paraconsistent theory and its paraconsistent complement are complete
if and only if they are finitely axiomatizable.

\st{Paraconsistent(T) and Complete(T) iff Paraconsistent($\neg$ T) and
Complete($\neg$ T) iff AX(T) and AX($\neg$ T) iff fAX(T).}

}*}


\end{document}